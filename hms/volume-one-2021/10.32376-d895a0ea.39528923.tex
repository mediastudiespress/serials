% see the original template for more detail about bibliography, tables, etc: https://www.overleaf.com/latex/templates/handout-design-inspired-by-edward-tufte/dtsbhhkvghzz

\documentclass{tufte-handout}

%\geometry{showframe}% for debugging purposes -- displays the margins

\usepackage{amsmath}

\usepackage{hyperref}

\usepackage{fancyhdr}

\usepackage{hanging}

\hypersetup{colorlinks=true,allcolors=[RGB]{97,15,11}}

\fancyfoot[L]{\emph{History of Media Studies}, vol. 1, 2021}


% Set up the images/graphics package
\usepackage{graphicx}
\setkeys{Gin}{width=\linewidth,totalheight=\textheight,keepaspectratio}
\graphicspath{{graphics/}}

\title[Recuperating Areas]{Recuperating Areas: Research on Media and Communication History and South Asian Studies} % longtitle shouldn't be necessary

% The following package makes prettier tables.  We're all about the bling!
\usepackage{booktabs}

% The units package provides nice, non-stacked fractions and better spacing
% for units.
\usepackage{units}

% The fancyvrb package lets us customize the formatting of verbatim
% environments.  We use a slightly smaller font.
\usepackage{fancyvrb}
\fvset{fontsize=\normalsize}

% Small sections of multiple columns
\usepackage{multicol}

% Provides paragraphs of dummy text
\usepackage{lipsum}

% These commands are used to pretty-print LaTeX commands
\newcommand{\doccmd}[1]{\texttt{\textbackslash#1}}% command name -- adds backslash automatically
\newcommand{\docopt}[1]{\ensuremath{\langle}\textrm{\textit{#1}}\ensuremath{\rangle}}% optional command argument
\newcommand{\docarg}[1]{\textrm{\textit{#1}}}% (required) command argument
\newenvironment{docspec}{\begin{quote}\noindent}{\end{quote}}% command specification environment
\newcommand{\docenv}[1]{\textsf{#1}}% environment name
\newcommand{\docpkg}[1]{\texttt{#1}}% package name
\newcommand{\doccls}[1]{\texttt{#1}}% document class name
\newcommand{\docclsopt}[1]{\texttt{#1}}% document class option name


\begin{document}

\begin{titlepage}

\begin{fullwidth}
\noindent\LARGE\emph{Launch essay
} \hspace{85mm}\includegraphics[height=1cm]{logo3.png}\\
\noindent\hrulefill\\
\vspace*{1em}
\noindent{\Huge{Recuperating Areas: Research on Media and Communication History and South Asian Studies\par}}

\vspace*{1.5em}

\noindent\LARGE{Shiv Ganesh\par}\marginnote{\emph{Shiv Ganesh, ``Recuperating Areas: Research on Media and Communication History and South Asian Studies,'' \emph{History of Media Studies} 1 (2021), \href{https://doi.org/10.32376/d895a0ea.39528923}{https://doi.org/ 10.32376/d895a0ea.39528923}.} \vspace*{0.75em}}
\vspace*{0.5em}
\noindent{{\large\emph{University of Texas at Austin}, \href{mailto:shiv.ganesh@austin.utexas.edu}{shiv.ganesh@austin.utexas.edu}\par}} \marginnote{\href{https://creativecommons.org/licenses/by-nc/4.0/}{\includegraphics[height=0.5cm]{by-nc.png}}}

% \vspace*{0.75em} % second author

% \noindent{\LARGE{<<author 2 name>>}\par}
% \vspace*{0.5em}
% \noindent{{\large\emph{<<author 2 affiliation>>}, \href{mailto:<<author 2 email>>}{<<author 2 email>>}\par}}

% \vspace*{0.75em} % third author

% \noindent{\LARGE{<<author 3 name>>}\par}
% \vspace*{0.5em}
% \noindent{{\large\emph{<<author 3 affiliation>>}, \href{mailto:<<author 3 email>>}{<<author 3 email>>}\par}}

\end{fullwidth}

\vspace*{1em}


\newthought{In this short} piece I argue, as provocation, that a recuperated
area-focused approach to the history of research on media and
communication can remedy the twin problems of theoretical
universalization and methodological parochialism that continue to
persist in the field despite decades of critique. The former, of course,
implicates the modernist project of developing theories of communication
that stretch to explain much of the complexity of human behavior, and
the latter involves the problematic assumption that what works well by
way of technique in one place ought to be reproducible in others.

The need to do this is particularly urgent for scholars who are invested
both in historicizing communication as well as the history of
communication research per se. This is precisely because social
scientific disciplines that seek to produce general laws about the world
tend to be ahistorical and produce a kind of amnesia about the specific
historical and cultural origins of concepts themselves, whether those
are attribution theories, hypotheses about behavioral change, or even
the concept of behavior itself. In contrast, contemporary area studies
are by definition and character interdisciplinary; have objects of study
that, at least at first blush, are spatially and temporally bounded; and
in the current moment, mitigate historical disciplinary impulses towards
theoretical colonization as well as methodological blinkers.

It is crucial to note, however, that the history of area studies is
itself steeped in problematic discourses of containment. Said's famous
critique of area studies argued that it was a manifestly orientalist


\enlargethispage{2\baselineskip}

\vspace*{2em}

\noindent{\emph{History of Media Studies}, vol. 1, 2021}




 \end{titlepage}



\noindent practice, where western experts produce knowledge about others, based on
geographical and epistemic categories that suited the administrative
calculus of Cold War politics: hence the revolving door in the
midcentury United States between directors of area studies programs and
the Federal Administration, especially the Department of
State.\footnote{For cogent critiques of the history of area studies, see
  Hossein Khosrowjah, ``A Brief History of Area Studies and
  International Studies,'' \emph{Arab Studies Quarterly} 31, nos. 3--4
  (2011); and Vicente Rafael, ``The Cultures of Area Studies in the
  United States'' \emph{Social Text} 41 (Winter 1994).} Yet, the very
critique of these political and ideological configurations opened up
fresh directions for inquiry and resulted in the increased uptake of
work on transnationalism, coloniality, feminism, race, ethnicity, and
sexuality in those same areas, often by scholars whose demographic
profiles, political commitments and epistemic sensibilities were very
different from those that preceded them.\footnote{See, for example,
  Laurie Sears's assessment of contemporary Southeast Asian studies in
  the essay ``Postcolonial Identities, Feminist Criticism and Southeast
  Asian Studies,'' in \emph{Knowing Southeast Asian Subjects}, ed.
  Laurie Sears (Seattle: University of Washington Press, 2007).} Because
of this, these former Cold War monikers themselves now sometimes
function as resistant categories. The term ``South Asia'' for example,
unlike other areas such as ``Indic Studies'' or ``Indology,'' ensures
that accounts of subcontinental politics and social life pay more
scrupulous attention to political hegemonies, historical divides, and
religious strife.

So rather than dispense with the notion of an area itself, I adopt as my
first investment a recuperated area approach that is suited to advance
how we view the histories of research on media and communication
studies. This view does not implicate any unified categories for
knowledge production; rather, it embraces transnationalism and the
provisionality of borders, presumes epistemic multiplicity, and
privileges knowledge production that is attuned sharply to transnational
epistemic inequities. Further, it tilts towards privileging local and
Indigenous knowledge production; and critically, draws inspiration from
Chandra Mohanty's cogent decolonizing gesture, and treats the US itself
as an area.\footnote{See Chandra Talpade Mohanty, ``Under Western Eyes
  Revisited: Feminist Solidarity through Anticapitalist Struggles,''
  \emph{Signs: Journal of Women in Culture and Society} 28, no. 2
  (2002).}

My second investment in this piece is to speculate about how, in telling
the history of media and communication research, one might also usefully
problematize this history itself. I realize this impulse is shared by
virtually every scholar who historicizes the history of communication
inquiry and practice, but it assumes particular significance if one is
to step outside the largely American and European ambits in which such
histories are narrated, internalized, and ultimately, reified. This is
precisely what contemporary area studies approaches militate against.
South Asia has a rich history of media and communication studies which,
while available in English, is considerably under-cited by media
historians in the west, except when references are made to communication
or media research ``outside'' the west. This, in turn, results in
anything outside Euro-American history being defined largely in terms of
its difference. For instance, Robert Oliver's \emph{Communication and
Culture in Ancient India and China}, while in many ways a scrupulous
account of the place of rhetoric in the old and complex history of these
two irreducibly different parts of the world, compromises itself by
explicitly defining ``Eastern'' rhetoric in terms of its difference with
the western tradition.\footnote{Robert Oliver, \emph{Communication and
  Culture in Ancient India and China} (Syracuse: Syracuse University
  Press, 1971).} The approach I describe pushes back on this sort of
deep orientalism because it insists that western frameworks
provincialize their own work.

South Asian Studies are an interesting staging ground upon which to rest
an area-focused view of research on communication and media history.
There are many ways to write the history of research on media and
communication studies in South Asia, as there are multiple senses in
which one can apprehend what counts as ``communication
history''---ranging from accounting for the trajectory of a predefined
body of work, to looking at what is constituted as communication across
a range of scholarship, to reviewing how media and communication
scholarship itself recounts the history of media.\footnote{Peter
  Simonson et al., ``The History of Communication History,'' in
  \emph{The Handbook of Communication History}, eds. Peter Simonson et
  al. (New York: Routledge, 2013).} Four approaches appear to be
particularly salient in South Asia, each of which inevitably involves
different kinds of inclusions and exclusions: a history of process, of
specific media, of institutions, and of events (as opposed to, say, a
focus on theories, methods, levels of analysis, or schools of thought).
Of particular note is the fact that scholarly engagement with a
particular kind of communication or media history has inevitably been
accompanied by a scholarly review of the history of that engagement.

A focus on \emph{process} entails examinations of the role that
modernist logics of development, rationalization, and governmentality
have played in the highly uneven and split growth of media and
communication systems. A case in point is Bella Mody's analysis of the
development of telecommunications in India, particularly its
(neo)liberalization and privatization in the 1980s and 1990s, which she
argues paradoxically reinscribed the power of the state.\footnote{Bella
  Mody, ``State Consolidation through Liberalization of
  Telecommunications Services in India,'' \emph{Journal of
  Communication} 45, no 4 (1995): 115.} Pradip Thomas's programmatic
work on participatory development communication also documents the
impact of modernist processes of rationalization and commodification
upon community-based organizing efforts, focusing particularly upon
research that illuminates how radical change can be domesticated by
institutionalized NGO structures that rationalize, bureaucratize, and
ultimately depoliticize potentially transformative movements.\footnote{Pradip
  Thomas, \emph{Participatory Development Communication: Philosophical
  Premises}. (New Delhi: Sage Publications, 1994). Also, Pradip Thomas,
  ``Development Communication and Social Change in Historical Context,''
  in \emph{The Handbook of Development Communication and Social Change},
  eds. Karin Wilkins, Thomas Tufte, and Rafael Obregon (Oxford: John
  Wiley \& Sons, 2014).}

Another approach to writing a South Asian history of media and
communication research is \emph{medium-focused.} Some scholars have
focused on the emergence of print media in the region, the first ever
newspaper being \emph{Hicky's Bengal Gazette}, published in 1780 by a
renegade Irishman, which vigorously criticized Warren Hastings for two
years before the East India Company seized control of it two years
later.\footnote{Andrew Otis\emph{, Hicky's Bengal Gazette: The Untold
  Story of India's First Newspaper} (New Delhi: Westland Publications,
  2018).} Akhtar and his colleagues have analyzed how nineteenth century
vernacular press in the subcontinent both supported and critiqued the
British Raj\footnote{M. Javaid Akhtar, Azra Asghar Ali, and Shahnaz
  Akhtar. ``The Role of Vernacular Press in the Subcontinent During the
  British Rule: A Study of Perceptions,'' \emph{Pakistan Journal of
  Social Sciences} 30, no. 1 (2010).} and, like others, have discussed
its pivotal role in developing and creating community and public
consciousness that often crossed caste and religious lines.\footnote{Nazakat
  Hussain, ``Role of Vernacular Press During British Rule in India,''
  \emph{International Education and Research Journal} 3, no. 5 (2017);
  and John Vilanilam, \emph{Communication in India: A Sociological
  Perspective} (New Delhi: Sage Publications, 2005).} A myriad of other
scholars have discussed the emergence of television in India and its
central role in modern politics.\footnote{Nalin Mehta, ed.,
  \emph{Television in India: Satellites, Politics and Cultural Change}
  (London: Routledge, 2008).} And there has been considerable attention
paid in the last two decades to digital media; quite possibly the first
book on the subject, \emph{India's Information Revolution}, was authored
by Arvind Singhal and Everett Rogers in 1988.\footnote{Arvind Singhal
  and Everett Rogers, \emph{India's Information Revolution} (New Delhi:
  Sage Publications, 1989).} Histories of communication studies in South
Asia are implicitly, and sometimes explicitly, built around the history
of the study of any (or all) of these media.

A third approach to scholarship on media and communication history is
\emph{institutional.} Commentary in the journal \emph{Economic and
Political Weekly}, for example, has vigorously analyzed the role of the
Press Council of India since its establishment in 1966, alternately
bemoaning its power in the 1970s and critiquing its impotence in the
current era. Other scholars have written about the establishment of the
Indian Institute of Mass Communication in 1965 by the Ministry of
Broadcasting and Information.\footnote{Keval Kumar, \emph{Mass
  Communication in India,} 5th ed. (Mumbai: Jaico Publishing House,
  2020).} While these accounts often focus upon the history of specific
media and research institutions, they implicitly and sometimes
explicitly attend to the intellectual history that lies behind those
institutions. In a 1989 special issue of \emph{Media, Culture \&
Society} on Indian media and mass communication research, for instance,
Vasudeva and Chakravarty analyze the epistemology produced by the
Institute, examining the lasting influence of Nehruvian ``scientific
temper'' upon the production of research.\footnote{Sunita Vasudeva and
  Pradip Chakravarty, ``The Epistemology of Indian Mass Communication
  Research,'' \emph{Media, Culture \& Society} 11, no. 3 (1989).}

A final approach to the history of media and communication research is
event-based. Creating event-histories provides a glimpse into the
unfolding temporal impact of specific kinds of media. Rajagopal's
analysis of the rise of right-wing Hindu politics, for example, was
based on a deft analysis of the impact that highly watched, multi-year
serializations of the Hindu epics \emph{Ramayana} and \emph{Mahabharata}
had on publics across the country during a period when the state-funded
broadcaster, Doordarshan, was moving away from its avowed emphasis on
information and education, and towards entertainment.\footnote{Arvind
  Rajagopal, \emph{Politics after Television: Hindu Nationalism and the
  Reshaping of the Public in India}, (Cambridge: Cambridge University
  Press, 2001).} Contemporary studies of the impact of \emph{Ramayana}
(including a recent re-telecast during the 2020 Covid-19 lockdown)
almost always recount the multiple strands of commentary and scholarship
that have accompanied the broadcast of the series over the last four
decades.\footnote{See, for example, Ahutosh Kumar Pandey and Amitabh
  Srivastava, ``Effects of Retelecast of TV Serial Ramayana on Indian
  Audience During Lockdown: A Survey-Based Study,'' \emph{Journal of
  Critical Reviews} 7, no. 19 (2020).}

One pivotal and highly studied event in the history of media and
communication research in the subcontinent occurred during the 1970s,
and recounting it provides a window not only into how media and
communication research unfolded in the subcontinent, but also
illustrates the insight that an area-oriented approach can bring to
research on communication and media history. In 1975, after nearly a
decade of negotiation and preparation under the leadership of Dr. Vikram
Sarabhai, popularly considered the founder of India's Space Program, and
the signing of a 1969 agreement between the United States and India,
NASA used its ATS-6 technology satellite to beam television programs
about agriculture, family planning, and national integration from ground
stations in Ahmedabad and Delhi to over 2,400 villages in six different
parts of India.\footnote{Noshir Contractor, Arvind Singhal, and Everett
  M. Rogers. ``Metatheoretical Perspectives on Satellite Television and
  Development in India,'' \emph{Journal of Broadcasting \& Electronic
  Media} 31, no. 2 (1988).} These areas were specially selected because
they were largely rural, impoverished, and home to Dalit and Adivasi
communities. M.S. Gore, the Director of the Tata Institute of Social
Sciences was commissioned to lead field studies of the impact that the
broadcasts were having on communities. The experiment was called
Satellite Instructional Television Experiment---more popularly, SITE.

The experiment continued for a full year and was discontinued by NASA in
1976 despite protests from a number of groups across South Asia,
including Arthur C. Clarke in Sri Lanka.\footnote{Raman Srinivasan, ``No
  Free Launch: Designing the Indian National Satellite,'' in
  \emph{Beyond the Ionosphere: Fifty Years of Satellite Communication},
  ed. Andrew J Butrica (NASA: NASA History Series, 1997).} As many as
nine hundred people per television set watched SITE programming
(although that number dwindled to about ninety people per television set
in 1976). SITE is fascinating not only because it launched an entire
generation of studies of television impacts in India, but also because
it began the development of television and research on it across every
country in South Asia. SITE sparked an intense amount of academic study
of the relationship between television and social change. Scholars have
heralded it as the largest television experiment in the world, and
others have referred to it as the most heavily studied communication
event in history.\footnote{John Vilanilam, \emph{Communication in India:
  A Sociological Perspective} (New Delhi: Sage Publications, 2005): 147.}

SITE has, of course, come in for considerable criticism. Many scholars
have critiqued the impoverished view of television and its impact
embedded in the study, as well as the unabashed elitism, itself a
product of Cold War politics, that led to the creation of the experiment
in the first place. As Siddiqi has recently argued, the experiment, a
collaboration between elite groups in two different parts of the world,
inevitably positioned rural communities in different parts of the
country, often Dalit and Adivasi, as inactive recipients of knowledge,
and in this sense, served to reinforce two starkly different images of
India---one tech savvy, and the other deeply impoverished---that persist
to this day.\footnote{Asif A. Siddiqi, ``Whose India? SITE and the
  Origins of Satellite Television in India,'' \emph{History and
  Technology} 36, nos. 3-4 (2020).} More deeply, perhaps, SITE also
makes visible the fact that much communication and media research in the
country continues to be generated by Savarna upper-caste scholars, with
Dalits either reduced to passive objects of research, and caste, more
often than not, elided altogether as a subject of inquiry.

SITE can also be understood as helping reinscribe Indian hegemony in
South Asia. This is not only because its scholarly research attracts
greater international visibility in media and communication studies than
does research from Sri Lanka, Bangladesh, Pakistan, Nepal, or Bhutan,
but also because SITE is an integral part of a larger technology
transfer project embedded in Cold War relationships and a rivalry with
Pakistan that helped India accrue scientific and technological expertise
at the expense of its neighbors.

And finally, a lesson from the SITE experiment that underscores an
argument for developing and promoting an area-based approach to
communication and media studies: SITE demonstrates that, from the
get-go, research and scholarship about media and communication, even
when it focuses upon one area, is almost inevitably transnational in
some form. SITE was not an experiment that was internal to India, nor
even between the US and India; it was caught up in a set of geopolitical
relationships between state (and subsequently corporate) actors, and it
carries a lesson for media historians: namely, that areas are always
constituted by externals. In this sense, all aspects of media and
communication history are always already international. And while
bifurcating ``US'' from ``international'' (i.e. non-US) media and
communication history may have been a useful staging ground upon which
to expand how one tells such history,\footnote{See, for example, Peter
  Simonson and David W. Park, eds., \emph{The International History of
  Communication Study} (New York: Routledge, 2015).} it is time to move
beyond that. For instance, rather than think about what impact ``US''
research has had on the world, it might behoove us to think more deeply
about how US research itself has been constituted by externals, not only
in terms of the tired relationship with ancient Greece in the speech
communication wing of the discipline, but also in terms of the bodies of
immigrants who have performed that research---the number of departments
across the US which have relied upon graduate students who then became
immigrants and then highly successful, US-based communication
researchers in their own right. The history of communication studies is
not static; it is composed of ``the flows of people, the intersecting
trajectories of individuals and social networks, and the peopling of
institutions.''\footnote{David Park and Meghan Grosse, ``International
  Vectors in U.S. Graduate Education in Communication,'' in \emph{The
  International History of Communication Study}, eds. Peter Simonson and
  David W. Park (New York: Routledge, 2015), 303.} Acknowledging this
means that we have to invert how we study areas---and that is exactly
what the study of areas themselves helps us accomplish.







\section{Bibliography}\label{bibliography}

\begin{hangparas}{.25in}{1} 



Akhtar, M. Javaid, Azra Asghar Ali, and Shahnaz Akhtar. ``The Role of
Vernacular Press in the Subcontinent During the British Rule: A Study of
Perceptions.'' \emph{Pakistan Journal of Social Sciences} 30, no. 1
(2010): 71--84.

Contractor, Noshir, Arvind Singhal, and Everett M. Rogers.
``Metatheoretical Perspectives on Satellite Television and Development
in India.'' \emph{Journal of Broadcasting \& Electronic Media} 31, no. 2
(1988): 129--48. \url{https://doi.org/10.1080/08838158809386690}.

Hussain, Nazakat. ``Role of Vernacular Press During British Rule in
India.'' \emph{International Education and Research Journal} 3, no. 5
(2017): 231--33.
\url{http://ierj.in/journal/index.php/ierj/article/view/911}.

Khosrowjah, Hossein. ``A Brief History of Area Studies and International
Studies.'' \emph{Arab Studies Quarterly} 31, no. 3-4 (2011): 131--42.
\href{h\%20https://www.jstor.org/stable/41858661}{https://www.jstor.org/stable/41858661}.

Kumar, Keval. \emph{Mass Communication in India}. 5th ed. Mumbai: Jaico
Publishing House, 2020.

Nalin Mehta, ed. \emph{Television in India: Satellites, Politics and
Cultural Change}. London: Routledge, 2008.

Mody, Bella. ``State Consolidation through Liberalization of
Telecommunications Services in India.'' \emph{Journal of Communication}
45, no. 4 (1995): 107--24.
\url{https://doi.org/10.1111/j.1460-2466.1995.tb00757.x}.

Mohanty, Chandra Talpade. ``Under Western Eyes Revisited: Feminist
Solidarity through Anticapitalist Struggles.'' \emph{Signs: Journal of
Women in Culture and Society} 28, no. 2 (2002): 499--535.
\url{https://doi.org/10.1086/342914}.

Oliver, Robert. \emph{Communication and Culture in Ancient India and
China}. Syracuse: Syracuse University Press, 1971.

Otis, Andrew. \emph{Hicky's Bengal Gazette: The Untold Story of India's
First Newspaper.} New Delhi: Westland Publications, 2018.

Pandey, Ahutosh Kumar, and Amitabh Srivastava. ``Effects of Retelecast
of TV Serial Ramayana on Indian Audience During Lockdown: A Survey-Based
Study.'' \emph{Journal of Critical Reviews} 7, no. 19 (2020): 3470--77.

Park, David, and Meghan Grosse. ``International Vectors in U.S. Graduate
Education in Communication.'' In \emph{The International History of
Communication Study}, edited by Peter Simonson and David W. Park,
303--23. New York: Routledge, 2015.

Rafael, Vicente. ``The Cultures of Area Studies in the United States.''
\emph{Social Text} 41 (Winter 1994): 91--111.

Rajagopal, Arvind. \emph{Politics after Television: Hindu Nationalism
and the Reshaping of the Public in India}. Cambridge: Cambridge
University Press, 2001.

Sears, Laurie. ``Postcolonial Identities, Feminist Criticism and
Southeast Asian Studies.'' In \emph{Knowing Southeast Asian Subjects},
edited by Laurie Sears 35--75. Seattle: University of Washington Press,
2007.

Siddiqi, Asif A. ``Whose India? SITE and the Origins of Satellite
Television in India.'' \emph{History and Technology} 36, no. 3--4
(2020): 452--74. \url{https://doi.org/10.1080/07341512.2020.1864118}.

Simonson, Peter, and David W. Park, eds. \emph{The International History
of Communication Study}. New York: Routledge, 2015.

Simonson, Peter, Janice Peck, Robert T. Craig, and John P. Jackson, Jr.
``The History of Communication History.'' In \emph{The Handbook of
Communication History}, edited by Peter Simonson, Janice Peck, Robert T.
Craig, and John P. Jackson, Jr., 13--57. New York: Routledge, 2013.

Simonson, Peter, Janice Peck, Robert T. Craig, and John P. Jackson, Jr.,
eds. \emph{The Handbook of Communication History}. New York: Routledge,
2013.

Singhal, Arvind, and Everett Rogers. \emph{India's Information
Revolution}. New Delhi: Sage Publications, 1989.

Srinivasan, Raman. ``No Free Launch: Designing the Indian National
Satellite.'' In \emph{Beyond the Ionosphere: Fifty Years of Satellite
Communication}, edited by Andrew J Butrica. NASA: NASA History Series,
1997.

Thomas, Pradip. \emph{Participatory Development Communication:
Philosophical Premises}. New Delhi: Sage Publications, 1994.

---------. ``Development Communication and Social Change in Historical
Context.'' In \emph{The Handbook of Development Communication and Social
Change}, edited by Karin Wilkins, Thomas Tufte, and Rafael Obregon,
7--19. Oxford: John Wiley \& Sons, 2014.

Vasudeva, Sunita, and Pradip Chakravarty. ``The Epistemology of Indian
Mass Communication Research.'' \emph{Media, Culture \& Society} 11, no.
3 (1989): 415--33. \url{https://doi.org/10.1177/016344389011004005}.

Vilanilam, John. \emph{Communication in India: A Sociological
Perspective}. New Delhi: Sage Publications, 2005.



\end{hangparas}


\end{document}