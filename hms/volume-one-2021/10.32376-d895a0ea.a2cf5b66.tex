% see the original template for more detail about bibliography, tables, etc: https://www.overleaf.com/latex/templates/handout-design-inspired-by-edward-tufte/dtsbhhkvghzz

\documentclass{tufte-handout}

%\geometry{showframe}% for debugging purposes -- displays the margins

\usepackage{amsmath}

\usepackage{hyperref}

\usepackage{fancyhdr}

\usepackage{hanging}

\hypersetup{colorlinks=true,allcolors=[RGB]{97,15,11}}

\fancyfoot[L]{\emph{History of Media Studies}, vol. 1, 2021}


% Set up the images/graphics package
\usepackage{graphicx}
\setkeys{Gin}{width=\linewidth,totalheight=\textheight,keepaspectratio}
\graphicspath{{graphics/}}

\title[History of Media Studies, in the Plural]{History of Media Studies, in the Plural} % longtitle shouldn't be necessary

% The following package makes prettier tables.  We're all about the bling!
\usepackage{booktabs}

% The units package provides nice, non-stacked fractions and better spacing
% for units.
\usepackage{units}

% The fancyvrb package lets us customize the formatting of verbatim
% environments.  We use a slightly smaller font.
\usepackage{fancyvrb}
\fvset{fontsize=\normalsize}

% Small sections of multiple columns
\usepackage{multicol}

% Provides paragraphs of dummy text
\usepackage{lipsum}

% These commands are used to pretty-print LaTeX commands
\newcommand{\doccmd}[1]{\texttt{\textbackslash#1}}% command name -- adds backslash automatically
\newcommand{\docopt}[1]{\ensuremath{\langle}\textrm{\textit{#1}}\ensuremath{\rangle}}% optional command argument
\newcommand{\docarg}[1]{\textrm{\textit{#1}}}% (required) command argument
\newenvironment{docspec}{\begin{quote}\noindent}{\end{quote}}% command specification environment
\newcommand{\docenv}[1]{\textsf{#1}}% environment name
\newcommand{\docpkg}[1]{\texttt{#1}}% package name
\newcommand{\doccls}[1]{\texttt{#1}}% document class name
\newcommand{\docclsopt}[1]{\texttt{#1}}% document class option name


\begin{document}

\begin{titlepage}

\begin{fullwidth}
\noindent\LARGE\emph{Launch essay
} \hspace{85mm}\includegraphics[height=1cm]{logo3.png}\\
\noindent\hrulefill\\
\vspace*{1em}
\noindent{\Huge{History of Media Studies, in the Plural\par}}

\vspace*{1.5em}

\noindent\LARGE{David W. Park}\par}\marginnote{\emph{David W. Park, ``History of Media\\\noindent Studies, in the Plural,'' \emph{History of \\\noindent Media Studies} 1 (2021). \href{https://doi.org/10.32376/d895a0ea.a2cf5b66}{https://doi.org/ 10.32376/d895a0ea.a2cf5b66}} \vspace*{0.75em}}
\vspace*{0.5em}
\noindent{{\large\emph{park@lfc.edu}, \href{mailto:Lake Forest College}{Lake Forest College}\par}} \marginnote{\href{https://creativecommons.org/licenses/by-nc/4.0/}{\includegraphics[height=0.5cm]{by-nc.png}}}

\vspace*{0.75em}

\noindent{\LARGE{Jefferson Pooley} \href{https://orcid.org/0000-0002-3674-1930}{\includegraphics[height=0.5cm]{orcid.png}}\par}
\vspace*{0.5em}
\noindent{{\large\emph{pooley@muhlenber.edu}, \href{mailto:Muhlenberg College}{Muhlenberg College}\par}}

\vspace*{0.75em}

\noindent{\LARGE{Peter Simonson} \href{https://orcid.org/0000-0001-7156-467X}{\includegraphics[height=0.5cm]{orcid.png}}\par}
\vspace*{0.5em}
\noindent{{\large\emph{University of Colorado Boulder}, \href{mailto:peter.simonson@colorado.edu}{peter.simonson@colorado.edu}\par}}

\end{fullwidth}

\vspace*{1em}


\newthought{We launch} \emph{History of Media Studies} with a concession that is also
a confession: The journal's object, the history of media studies, is
porous and tensile, verging on shapeless. The main reason is that
``media studies'' itself has no agreed-upon referent. The phrase only
gained traction---as a singular, field-designating noun---in the late
1960s and 1970s, long after the study of media and communication had
taken variable root in the university. There were ``media studies'' in
the plural, but no such thing as a field by that name. Even today, many
academics who work on topics related to media and communication would,
if asked to name their field, give some other answer. Self-described
``media studies'' scholars, meanwhile, might squint quizzically at the
field-definitions of their putative peers.

We take the term's slippages to be an editorial advantage. The journal's
premise is that what counts as the history of media studies is itself up
for grabs. The task of our authors, put another way, is to define the
field's scope, through the accretion of their articles'
arguments-by-example. There are disciplinary labels with more stable
referents---``communication research'' and ``cinema studies'' are
two---but they are each partial enclosures. So one virtue of ``media
studies'' is that it is porous and tensile.

Another virtue comes by way of the media concept itself. With roots
dating back to at least the Enlightenment, \emph{medium} and \emph{mediation}

\enlargethispage{2\baselineskip}

\vspace*{2em}

\noindent{\emph{History of Media Studies}, vol. 1, 2021}


 \end{titlepage}


\noindent have become increasingly fundamental to understanding
the social textures of the modern
world.\footnote{John Guillory, ``Enlightening Mediation,'' in \emph{This is
  Enlightenment}, ed. \emph{} Clifford Siskin and William Warner
  (Chicago: University of Chicago Press, 2010).
} They are capacious
concepts, at the center of all communicative processes---tied in one way
to their materiality, in another to processes of representation,
articulation, translation, and standing-between. The dominant, plural
form of ``media''---as commercial organs of mass communication---only
took hold in the 1950s, at the height of the broadcast
era.\footnote{Anna Shechtman, ``Command of Media's Metaphors,''~\emph{Critical
  Inquiry}~47, no. 4 (2021).
} We invoke ``media'' in its
rich, catholic sense, as an invitation to ventilate what the history of
its study might be.

As a baseline, we expect to publish work that spans the history of the
humanities and the social sciences. Scholarship on the history of
communication research, cultural studies, film studies, information
science, journalism studies, speech, and rhetoric will, we anticipate,
feature in the editorial mix. Many of the subjects, discourses, and
institutions that our authors investigate won't slot so easily. Scholars
of media have identified with dozens of other disciplines, across local
and national contexts barely represented in the published literature.
Many students of media, moreover, were never affiliated with
universities---working instead for commercial firms, government
agencies, or nonprofit groups, or embedded within social movements. Our
aim is to publish their stories too.

\hypertarget{exclusions-in-the-historiography-of-media-studies}{%
\section{Exclusions in the Historiography of Media
Studies}\label{exclusions-in-the-historiography-of-media-studies}}

The mission of the journal is shaped by the historical moment. Our
fields are belatedly reckoning with the legacies and ongoing
consequences of an intersecting array of structural inequities: their
whiteness and patriarchy; the hegemonies of the English language and of
US forms of thinking and research; the longstanding effects of
colonialism and anti-Blackness; the many forms of hierarchy and
exclusion emanating from cores and peripheries around the globe; and the
neoliberal colonization of universities and academic publishing, to name
some of the most prominent. To this point, the bulk of critical writing
on these subjects has rightly focused on their contemporary
manifestations. \emph{History of Media Studies} aims to provide a forum,
too, for investigating the historical dynamics through which the media
studies fields have reached this point---both in terms of dominant
social patterns that have defined them and the mix of critical
consciousness and alternative practices that might yield other ways of
advancing our intellectual work. Beyond publishing such genealogies of
the present, the journal is also committed to its own alternative
practices and to supporting collective efforts that have energized the
present moment.
\newpage
\emph{History of Media Studies} aims to contribute to what might be
called the de-centering of the centers within these interrelated
patterns of marginalization. This will not be easy or straightforward
work. The inherited habits of the field are lodged deep in its
practices---and in the bodies and minds of those of us who work in it.
Those habits cut across geopolitics, language, ideology, political
economy, and the intersectionalities of social identity. All of these
continue to unfold at a particularly perilous time in history, when deep
inequalities, political crises, and climate change threaten the
possibility of a shared, livable future on the planet. We have few
illusions about the significance of a new scholarly journal in these
contexts, and we do not want to overstate the importance of efforts it
makes. That said, we believe that journals refract the wider practices
and distortions of academic fields in microcosm, and they should do
their parts in addressing broader calls of the present. A new journal
can open at least modest space for thinking and doing things
differently, serving as a small laboratory that might inform efforts
elsewhere.

One might wonder how a journal whose founding editors are three white
cis-men from the United States can in good faith hope to contribute to
those efforts. We take those doubts seriously and in fact share them.
The short response is that we could not hope to do much on our own. We
resonate with Mohan Dutta's argument that we must design and enact
different kinds of communication if we are to transform the inherited
ways of producing scholarship and constituting our
fields.\footnote{Mohan J. Dutta, ``Whiteness, Internationalization, and Erasure:
  Decolonizing Futures from the Global South,'' \emph{Communication and
  Critical/Cultural Studies} 17, no. 2 (2020).
} In the \emph{History of
Media Studies} project, that begins with our
\href{https://hms.mediastudies.press/editorial}{Editorial Board}, which
will be a more participatory body than many such entities, actively
shaping the work we do and how we do it. We are very fortunate to have a
geographically diverse and remarkably talented board, whose membership
will continue to grow. The
\href{https://hms.mediastudies.press/launch-essays}{launch essays}
they've written here are the most visible signs of their guiding
participation in this endeavor, but by no means the only ones. Board
members are helping us to imagine and implement a journal guided by the
ideals of care, craft, and collegial friendship across borders, directed
toward investigating the history of media studies in all its
complexities around the world. In that context, the editors of
\emph{History of Media Studies} see our roles as facilitators of a
collective process of bringing something new into the world. We are
committed to leveraging structural privileges we might have to aid that
endeavor and moving forward through dialogical openness to difference
and critique.

The Colombian-born anthropologist Alberto Escobar, inspired by the
Mexican Zapatistas, has advocated for designing a world for ``the
pluriverse, \emph{a world where many worlds
fit}.''\footnote{Alberto Escobar, \emph{Designs for the Pluriverse: Radical
  Interdependence, Autonomy, and the Making of Worlds} (Durham, NC: Duke
  University Press, 2018), xvi.
} What would histories of
media studies written for a pluriverse entail, and how might the journal
be designed to facilitate them? We don't pretend to know, but we are
eager to create spaces for experimentation, and we have a few
commitments at this early juncture. One pertains to language, which, as
Susana Martínez Guillem writes, is ``a fundamental axis of power
relations'' that should be featured in efforts to re-shape the
contemporary field.\footnote{Susan Martínez Guillem, ``Sacando la Lengua in Rhetorical Theory and
  Criticism,'' \emph{Rhetoric, Politics \& Culture} 1, no. 1 (2021), 45.
  See also Silvio Waisbord, ``Communication Studies without Frontiers?
  Translation and Cosmopolitanism across Academic Cultures,''
  \emph{International Journal of Communication} 10 (2016); and Ana
  Cristina Suzina, ``English as \emph{Lingua Franca}: On the
  Sterilisation of Scientific Work,'' \emph{Media, Culture \& Society}
  43, no. 1 (2021).
} The
marginalization of scholars and scholarship operating in languages
outside of English is obvious to anyone outside the English-only center
of US-dominated academic fields. As one modest step in a different
direction, \emph{History of Media Studies} will review and publish
manuscripts in Spanish as well as English, and we hope to expand beyond
those two languages in the future.

A second commitment is to support the writing of histories of media
studies from around the world, particularly from the Global South and
other regions that have been peripheral, at best, in the
English-language historiography to date. This means more than simply
generating a fuller record of the history of media studies around the
globe, though we believe that's an important goal. It also means
supporting historiographical analogues of ``theory from the South'' that
might reorient received ways of writing histories of media studies, or
at least help create the sorts of ``common spaces with room for
differentiation'' advocated by proponents of academic
cosmopolitanism.\footnote{Jean Comaroff and John Comaroff, \emph{Theory from the South: Or How
  Euro-America is Evolving toward Africa} (New York: Routledge, 2012);
  and Hanan Badr and Sarah Anne Ganter, ``Towards Cosmopolitan Media and
  Communication Studies: Bringing Diverse Epistemic Perspectives into
  the Field,'' \emph{Global Media Journal (German Edition)} 11, no. 1
  (2021).
} This must be done
in ways that don't perpetuate the practice of scholars from the Global
North talking about Southern contexts without including voices from the
regions.\footnote{Sarah Anne Ganter and Félix Ortega, ``The Invisibility of Latin
  American Scholarship in European Media and Communication Studies:
  Challenges and Opportunities of De-Westernization and Academic
  Cosmopolitanism,'' \emph{International Journal of Communication} 13
  \emph{} (2019).
} The commitment also
entails ``provincializing'' the traditionally unmarked particularities
that have masqueraded as universals, or at least as the
taken-for-granted baseline---histories that, intentionally or not, have
presented themselves as histories of \emph{the} field, when they are
really the histories of a handful of successful, institutionally
well-placed, and overwhelmingly male scholars who either published in
English or were prominent enough to have their work translated.

This in turn opens to a third commitment of the journal: to focus not
just on media studies as it has developed within academic institutions,
be they core or peripheral, but also on what we call
``alter-traditions'' of reflective inquiry about media, again broadly
conceived. What conceptualizations, educational practices, normative
frameworks, and forms of reflective social practice about media have
developed outside the academy? Some of these traditions are of course
embedded in the commercial sector, and others trace their origins to the
work of prominent religious organizations like the Catholic Church.
\emph{History of Media Studies} encourages research that brings these
histories into closer conversation with histories of academic media
studies; but we particularly encourage work that excavates the
alter-traditions of Indigenous and other subaltern groups and the
reflective cultural practices that represent less-recognized varieties
of media study---a body of work that Latin Americans and other scholars
from the Global South have been at the forefront of producing, often
connected to other kinds of decolonial
efforts.\footnote{See, for instance, Claudia Magallanes Blanco and José Manuel Ramos
  Rodríguez, eds., \emph{Miradas Propias: Pueblos Indígenas,
  Comunicación y Medios en la Sociedad Global} (Quito: Ediciones
  CIESPAL, 2016); and Erick R. Torrico Villanueva, ``La Comunicación
  Decolonial, Perspectiva In/Surgente,'' \emph{Revista Latinoamericana
  de Ciencias de la Comunicación} 15, no. 28 (2018).
}

\hypertarget{beyond-open-access}{%
\section{Beyond Open Access}\label{beyond-open-access}}

The journal's mission commitments extend to its mode and manner of
publication. \emph{History of Media Studies} is open access (OA), which
means that readers don't pay for articles or subscriptions. In that
respect, \emph{History of Media Studies} resembles many newer journals,
even some launched by the big five commercial publishers. The
difference, a crucial one for us, is author fees: We don't charge any,
on principle. Many open access journals, even those published by
scholarly societies, require authors to pay an ``article processing
charge'' (APC) that typically runs \$3,000 to \$5,000. Open access for
readers, we believe, should not be traded for new barriers to
authorship. Instead of author fees, we support our operations through
what is commonly called \emph{collective funding}: direct support from
libraries and foundations.\footnote{\emph{History of Media Studies} is participating in a new approach to
  bring nonprofit publishers and library funders together on a web-based
  matching platform. The idea is that librarians and other funders can
  support publishers on the basis of shared values. The journal is among
  the participants in the Open Access Community Investment Program
  (OACIP), organized by the North American library consortium LYRASIS.
  On the idea of mission-aligned funding exchanges, see Jefferson
  Pooley, ``Collective Funding to Reclaim Scholarly Publishing,''
  \emph{The Commonplace}, August 16, 2021,
  \url{https://commonplace.knowledgefutures.org/pub/erpw9udj}.
} What
collective funding means is that our submissions are evaluated on their
editorial merits, without regard to personal or institutional wealth.
Among other things, this fee-free status supports our mission to publish
authors and topics from around the world---since most scholars outside a
handful of rich North American institutions and wealthy European
countries can't afford APCs.

In line with our no-fee policy, we interpret ``open access'' in a more
demanding, value-laden way than the typical scholarly outlet. We believe
that ownership and governance matter---that sustainable OA publishing
should be nonprofit, community-led, and transparent. Together with our
scholar-led publisher,
\href{https://mediastudies.press}{mediastudies.press}, we subscribe to
Jean-Sebastian Caux's ``Genuine Open Access
Principles.''\footnote{See ``Open Access Principles,''
  \href{http://mediastudies.press}{mediastudies.press},
  \url{https://www.mediastudies.press/oa-principles}; and Jean-Sébastien
  Caux, ``Genuine Open Access Principles,'' Jean-Sébastien Caux,
  \url{https://jscaux.org/blog/post/2018/05/05/genuine-open-access/}.
} We are committed to
using open infrastructure wherever possible, which extends to the
publishing software itself: PubPub, the MIT-linked open-source platform
premised on reclaiming scholarly communication for the academic
community.\footnote{PubPub is a project of the nonprofit Knowledge Futures Group, which
  emerged from a partnership between MIT Press and the MIT Media Lab.
  See ``Our Mission,'' PubPub, \url{https://www.pubpub.org/about}. On
  the Knowledge Futures Group's vision for scholarly publishing, see
  Gabriel Stein et al., ``Clarivate, ProQuest, and our Resistance to
  Commercializing Knowledge,'' \emph{The Commonplace},
} Our governance and
finances are transparent and open for scrutiny, with a special accent on
our scholar-led operations.\textsuperscript{12}

Our plans to limit the volume of articles we publish---typically no more
than ten a year---is a deliberate means to a value-oriented end. As
editors, we can afford to take a slow, care-based approach to authors
and their submissions. We think of this as a craft ideal, one that
consciously resists the harried facelessness that drives most commercial
journals.\textsuperscript{13} \emph{History of Media
Studies} substitutes artisanal editing and humane peer review for
ScholarOne and the metric tide. Our\marginnote{May 18, 2021,
  \url{https://commonplace.knowledgefutures.org/pub/kp81ylos/}.} author \marginnote{\textsuperscript{12} See ``Transparency,''
  \href{http://mediastudies.press}{mediastudies.press},
  \url{https://www.mediastudies.press/transparency}.
}
 reports,\marginnote{\textsuperscript{13} For an astute reflection on the interplay between commercial capture
  and editors' own post-defeat malaise and capitulation, see Mark
  Gibson, ``Editing After Exit--Alienation and Counter--Alienation in
  the Cultures of Cultural Studies Journals,'' \emph{Continuum} 35, no.
  3 (2021).
} as one example,
don't merely include download and citation counts, but also quoted
passages and citation contexts.

The issue of metrics raises what is, for us, an important point. We
believe that our care-based ethic is compatible with both editorial
quality and best practices in scholarly publishing. Every article
receives a Creative Commons license and a DOI, with downloads available
in PDF and seven other formats, including machine-readable JATS XML with
swift and accurate Google Scholar
indexing.\setcounter{footnote}{13}\footnote{Indeed, \emph{History of Media Studies}, on launch, meets all the
  compliance criteria for the European funders' ``Plan S,'' including
  the technical requirements. See cOAlition S, ``Plan S Principles,''
  Plan S, \url{https://www.coalition-s.org/plan_s_principles/}. The
  journal's publisher,
  \href{http://mediastudies.press}{mediastudies.press}, is a member of
  Crossref, the Open Access Scholarly Publishing Association (OASPA) and
  the Radical Open Access collective, with vetted affiliations including
  the Directory of Open Access Books (DOAB), Project MUSE, and OAPEN.
  History of Media Studies will apply for listing in the Directory of
  Open Access Journals (DOAJ) and membership in the Committee on
  Publication Ethics (COPE), when eligible after one year of operation.
  See ``About this Journal,'' \emph{History of Media Studies}, 2021,
  \url{https://hms.mediastudies.press/about}.
} The journal's copy
editors are skilled line-editors too, with masthead and article-level
credit in recognition of their vital work.

Among our aims is to help broaden what a scholarly article looks like.
We chose the PubPub platform, in part, for its extensive support for
multimedia formats, on the assumption that historians of media studies
might illuminate these fields' pasts in dialogue with new forms of
scholarly communication. Consider the typical archives-based historical
paper: We envision publishing archival documents and other supporting
media within the articles that cite them. We plan to publish archival
collections and half-forgotten public domain works too, anchored by new
introductions. We will even re-publish refereed, openly licensed work
appearing elsewhere, with linked ``Replies'' solicited for these
``overlay'' works as well as for original articles. The journal is open
to entire submissions whose arguments are rendered in audio, video, and
other non-textual forms.

\emph{History of Media Studies}' slow-scholarship values also guide our
approach to peer review. We are committed to a humane, developmental
review process, with the goal to improve manuscripts through collegial
exchange. Inspired by the example of \emph{Public Philosophy}, we see
our role as more than reviewer-herders in service of a one-off
decision.\footnote{See ``Formative Peer Review (FPR),'' \emph{Public Philosophy Journal},
  \url{https://publicphilosophyjournal.org/overview/}.
} We aim to cultivate in
reviewers, too, an ethos of ongoing, supportive involvement with an
author and her manuscript. \emph{History of Media Studies} employs
double-anonymous review by default, but encourages more open modes at
authors' discretion. For example, we support \emph{signed review}, in
which reviewers sign their comments and may continue to consult with
authors throughout the revision process. We will also experiment with
\emph{community review}, in which an article draft gets published early
in the process, with public, signed comments encouraged---and with
iterative revisions supported by PubPub's robust versioning
support.\footnote{See ``Peer Review,'' \emph{History of Media Studies},
  \url{https://hms.mediastudies.press/peer-review.}
}

Among our goals is to help foster a far-flung community of scholars
working on the history of media studies and its sister fields. We
maintain, in cooperation with the journal, a working group on the
History of Media Studies. At each monthly session, a scholar presents on
a work-in progress in remote sessions attended by academics from around
the world. The same welcoming, developmental culture informs the working
group, as an extension of the journal's editorial
ethos.\footnote{ The working group is hosted by the Consortium for History of Science,
  Technology and Medicine (CHSTM). See ``Working Group on the History of
  Media Studies,'' \emph{History of Media Studies},
  \url{https://hms.mediastudies.press/working-group}.}

\hypertarget{a-new-journal}{%
\section{A New Journal}\label{a-new-journal}}

The fifteen short essays in this launch series, authored by Editorial
Board members, reflect the values we've outlined here. The journal's
commitment to attend to structural inequities and exclusions is
reflected across many of the launch essays. Wendy Willems, in her
contribution, marks a painful irony in recent attempts to de-Westernize
and decolonize media studies, which have often silenced the actual
histories of the field outside the Global North and longstanding
decolonial struggles in the African diaspora. She warns of the dangers
of decolonization ``becoming an empty metaphor'' and challenges
historians of the field to move beyond mere inclusion to consider ``how
the act of including different vantage points challenges, subverts and
problematizes'' dominant understandings of the field. Armond Towns
stakes out complementary ground, reviving a vigorous tradition of Black
studies that has been occluded by scholarship that ``situated Black life
solely in \emph{reaction} to white racism,'' and that can be renewed
through an ``alternative epistemological project that\ldots would
require the reorganization of the world as it currently sits.'' He asks
us to read the history of media and communication studies alongside the
history of Black studies and see how some currents of the former
``developed in fear of Black and decolonial revolution.'' Decentering
the white, Euro-American North from different geo-intellectual ground,
Liu Hailong and Qin Yidan ask, ``What would have been different if
communication study had been born in China?'' This opens into their
discussion of the particularities of the Chinese field and the
possibilities that its recent turn toward media embedded in Chinese
experience might signal fresh beginnings.

Other essays in the series take up the geopolitical de-centering of the
Global North from different perspectives. Mohammad Ayish's contribution
situates the history of media studies in the Arab world's technological,
political, and socio-cultural contexts, tracking a shift from a
development to an empowerment frame. He draws out transnational and
cross-regional entanglements that displace the nation-state as
historical locus. In his careful consideration of media studies in
Argentina, Mariano Zarowsky reminds us that ``to speak of a field of
knowledge'' such as media studies ``implies studying a \emph{process of
formation} rather than starting from pre-existing entities.'' He brings
this process of formation to ground-level, highlighting the importance
of ``temporal formations and specific biographical contexts,'' as well
as the interaction of regional articulations with global processes. In
his essay, Shiv Ganesh outlines a program for an ``area-focused
approach'' to the history of media studies, with the aim to challenge
the twinned ills of ``theoretical universalization and methodological
parochialism'' in these fields. He illustrates the value of the approach
with the South Asian case, staking out ground to find alternatives to
casting ``anything outside Euro-American history being defined largely
in terms of its difference.'' The biography, thought, and rich legacies
of Jesús Martín-Barbero are the focus of Raúl Fuentes Navarro's essay.
Martín-Barbero's \emph{De los Medios a las Mediaciones,} Fuentes Navarro
argues, would define Latin America ``as distinct among Western cultural
and linguistic regions.'' A foundational work in the Ibero-American
world that has shaped thinking since its publication in 1987, the book
also provides a map for needed historiography of the field in the
region.

Many of the launch essays concern prescriptions and adjustments in
historiographical focus or technique. Stefanie Averbeck-Lietz, for
instance, probes a project on journalism about the League of Nations to
draw out the marginal place of historical methods in German
communication and media studies. Along the way, she challenges us to
think about the methods by which we write histories of the field and the
histories of those methods themselves. Thomas Wiedemann and Michael
Meyen, in their essay, reflect on a major German-language
project---\emph{Biografisches Lexikon der Kommunikationswissenschaft} or
BLexKom---that they help steward. They trace the historiographical
promise and peril that the initiative has exposed, including questions
of whose story gets told and by whom. Sarah Cordonnier touches on
similar themes in her survey of the historiography of French media
research, identifying elisions and ``black holes'' in the literature.
She raises big questions about how to write histories of a field with so
many permutations, and whether we can find a way to become
``\,`contemporaries in discipline' in spite of all the differences.'' In
her contribution, Wendy Leeds-Hurwitz makes the case for ``theory
groups'' as a prism for doing history, one that reveals the social
infrastructures that support visibility and influence. She implicitly
responds to the question of method that Averbeck-Lietz raises, making
space for using group communication theory to understand the social life
of ideas.

Other essays point to productive slippages of the ``media studies''
label that the journal hopes to amplify. Sue Collins, in her
contribution, shows how the study of mediated authority exposes the
limits of what ``communication'' or ``film studies'' alone could do for
us. She advocates that ``the history of communication and media studies
better integrate film and cultural histories into its corpus.'' In a
similar spirit, Filipa Subtil forcefully demonstrates that the
philosophy of technology tacitly subtends the history of media studies
in ways that should challenge our preconceptions regarding what media
are. She urges us to remedy a situation where, with a few exceptions,
``historians of media studies have not granted enough attention to the
question of technology.'' Maria Löblich, meanwhile, provocatively blurs
the lines between the history of communication studies and the
phenomenon of collective identity, allowing cross fertilization of those
two scholarly fields. Löblich's project, both analytic and
reconstructive, allows us to think about ``how communication studies
were historically tied to symbolic systems in society and what degree of
autonomy they had.'' Finally, Ira Wagman challenges us to ``trouble the
history of `media studies' in as many contexts as possible,'' drawing
out the \emph{Miranda Prorsus} (the 1957 Papal encyclical on motion
pictures, television, and radio) as an illustration of religion as one
such context. In so doing, he offers an excellent example of histories
that examine the conceptualization and study of media outside academic
contexts.

Both individually and as a group, the Editorial Board's launch essays at
once exemplify dimensions of the journal's mission and develop it in
ways that exceed the imaginations of the editors who formulated it.
Plurality has unique potential. Here it is manifest through writing that
arises from scholarly lives animated by diverse problematics, thought
styles, languages, political cultures, and institutional contexts. We
are grateful for their creative responses to our invitation, which we
hope you will read and circulate freely, in the spirit of open access.





\section{Bibliography}\label{bibliography}

\begin{hangparas}{.25in}{1} 



``About this Journal.''~\emph{History of Media
Studies}.~\url{https://hms.mediastudies.press/about}.

Badr, Hanan, and Sarah Anne Ganter. ``Towards Cosmopolitan Media and
Communication Studies: Bringing Diverse Epistemic Perspectives into the
Field.''~\emph{Global Media Journal (German Edition)}~11, no. 1 (2021):
1--3. \url{https://doi.org/10.22032/dbt.49164}.

Caux, Jean-Sébastien. ``Genuine Open Access Principles.'' Jean-Sébastien
Caux.~\href{https://jscaux.org/blog/post/2018/05/05/genuine-open-access/}{https://jscaux.org/blog/post/2018/05/05/genuine-open-access}.

cOAlition S. ``Plan S Principles.'' Plan
S.~\url{https://www.coalition-s.org/plan_s_principles/}

Comaroff, Jean, and John Comaroff.~\emph{Theory from the South: Or How
Euro-America is Evolving toward Africa}. New York: Routledge, 2012.


Dutta, Mohan J. ``Whiteness, Internationalization, and Erasure:
Decolonizing Futures from the Global South.''~\emph{Communication and
Critical/Cultural Studies} 17, no. 2 (2020): 228--35.
\url{https://doi.org/10.1080/14791420.2020.1770825}.

Escobar,~Alberto. \emph{Designs for the Pluriverse: Radical
Interdependence, Autonomy, and the Making of Worlds}. Durham, NC: Duke
University Press, 2018.

``Formative Peer Review (FPR).''~\emph{Public Philosophy Journal}.
\url{https://publicphilosophyjournal.org/overview/}.

Ganter, Sarah Anne, and Félix Ortega. ``The Invisibility of Latin
American Scholarship in European Media and Communication Studies:
Challenges and Opportunities of De-Westernization and Academic
Cosmopolitanism,''~\emph{International Journal of Communication} 13
(2019): 68--91. \url{https://ijoc.org/index.php/ijoc/article/view/8449}.

Gibson, Mark. ``Editing After Exit--Alienation and Counter--Alienation
in the Cultures of Cultural Studies Journals.''~\emph{Continuum}~35, no.
3 (2021): 356--68. \url{https://doi.org/10.1080/10304312.2021.1902159}.

Guillory, John. ``Enlightening Mediation.'' In~\emph{This is
Enlightenment}, edited by Clifford Siskin and William Warner, 37--66.
Chicago: University of Chicago Press, 2010.

Magallanes Blanco, Claudia, and José Manuel Ramos Rodríguez,
eds.~\emph{Miradas Propias: Pueblos Indígenas, Comunicación y Medios en
la Sociedad Global}. Quito: Ediciones CIESPAL, 2016.

Martínez Guillem, Susan. ``Sacando la Lengua in Rhetorical Theory and
Criticism.''~\emph{Rhetoric, Politics \& Culture} 1, no. 1 (2021):
45--54.

``Open Access
Principles.''~mediastudies.press.~\url{https://www.mediastudies.press/oa-principles}.

``Our Mission.'' PubPub.~\url{https://www.pubpub.org/about}.

``Peer Review.''~\emph{History of Media
Studies}.~\url{https://hms.mediastudies.press/peer-review}.

Pooley, Jefferson. ``Collective Funding to Reclaim Scholarly
Publishing,''~\emph{The Commonplace}. August 16,
2021.~\url{https://commonplace.knowledgefutures.org/pub/erpw9udj}.

Shechtman, Anna. ``Command of Media's Metaphors.''~\emph{Critical
Inquiry}~47, no. 4 (2021): 644--74.
\url{https://doi.org/10.1086/714512}.

Stein, Gabriel, Travis Rich, Zach Verdin, and Catherine Ahearn.
``Clarivate, ProQuest, and our Resistance to Commercializing
Knowledge.''~\emph{The Commonplace}. May 18,
2021.~\url{https://commonplace.knowledgefutures.org/pub/kp81ylos/}.

Suzina, Ana Cristina. ``English as~\emph{Lingua Franca}: On the
Sterilisation of Scientific Work,''~\emph{Media, Culture \& Society} 43,
no. 1 (2021): 171--79. \url{https://doi.org/10.1177/0163443720957906}.

Torrico Villanueva, Erick R. ``La Comunicación Decolonial, Perspectiva
In/Surgente.''~\emph{Revista Latinoamericana de Ciencias de la
Comunicación}~15, no. 28 (2018): 74--81.
\url{http://revista.pubalaic.org/index.php/alaic/article/view/1150}.

``Transparency.''~mediastudies.press.
\url{https://www.mediastudies.press/transparency}

Waisbord, Silvio. ``Communication Studies without Frontiers? Translation
and Cosmopolitanism across Academic Cultures.''~\emph{International
Journal of Communication} 10 (2016): 868--86.
\url{https://ijoc.org/index.php/ijoc/article/view/3483}.

``Working Group on the History of Media Studies.''~\emph{History of
Media Studies}. \url{https://hms.mediastudies.press/working-group}.



\end{hangparas}


\end{document}