% see the original template for more detail about bibliography, tables, etc: https://www.overleaf.com/latex/templates/handout-design-inspired-by-edward-tufte/dtsbhhkvghzz

\documentclass{tufte-handout}

%\geometry{showframe}% for debugging purposes -- displays the margins

\usepackage{amsmath}

\usepackage{hyperref}

\usepackage{fancyhdr}

\usepackage{hanging}

\hypersetup{colorlinks=true,allcolors=[RGB]{97,15,11}}

\fancyfoot[L]{\emph{History of Media Studies}, vol. 1, 2021}


% Set up the images/graphics package
\usepackage{graphicx}
\setkeys{Gin}{width=\linewidth,totalheight=\textheight,keepaspectratio}
\graphicspath{{graphics/}}

\title[Wonderful Invention!]{Wonderful Invention! } % longtitle shouldn't be necessary

% The following package makes prettier tables.  We're all about the bling!
\usepackage{booktabs}

% The units package provides nice, non-stacked fractions and better spacing
% for units.
\usepackage{units}

% The fancyvrb package lets us customize the formatting of verbatim
% environments.  We use a slightly smaller font.
\usepackage{fancyvrb}
\fvset{fontsize=\normalsize}

% Small sections of multiple columns
\usepackage{multicol}

% Provides paragraphs of dummy text
\usepackage{lipsum}

% These commands are used to pretty-print LaTeX commands
\newcommand{\doccmd}[1]{\texttt{\textbackslash#1}}% command name -- adds backslash automatically
\newcommand{\docopt}[1]{\ensuremath{\langle}\textrm{\textit{#1}}\ensuremath{\rangle}}% optional command argument
\newcommand{\docarg}[1]{\textrm{\textit{#1}}}% (required) command argument
\newenvironment{docspec}{\begin{quote}\noindent}{\end{quote}}% command specification environment
\newcommand{\docenv}[1]{\textsf{#1}}% environment name
\newcommand{\docpkg}[1]{\texttt{#1}}% package name
\newcommand{\doccls}[1]{\texttt{#1}}% document class name
\newcommand{\docclsopt}[1]{\texttt{#1}}% document class option name


\begin{document}

\begin{titlepage}

\begin{fullwidth}
\noindent\LARGE\emph{Launch essay
} \hspace{85mm}\includegraphics[height=1cm]{logo3.png}\\
\noindent\hrulefill\\
\vspace*{1em}
\noindent{\Huge{Wonderful Invention!\par}}

\vspace*{1.5em}

\noindent\LARGE{Ira Wagman}\par}\marginnote{\emph{Ira Wagman, ``Wonderful Invention!'' \emph{History of Media Studies} 1 (2021), \href{https://doi.org/10.32376/d895a0ea.ef8f548f}{https://doi.org/ 10.32376/d895a0ea.ef8f548f}.} \vspace*{0.75em}}
\vspace*{0.5em}
\noindent{{\large\emph{Carleton University}, \href{mailto:ira.wagman@carleton.ca}{ira.wagman@carleton.ca}\par}} \marginnote{\href{https://creativecommons.org/licenses/by-nc/4.0/}{\includegraphics[height=0.5cm]{by-nc.png}}}

% \vspace*{0.75em} % second author

% \noindent{\LARGE{<<author 2 name>>}\par}
% \vspace*{0.5em}
% \noindent{{\large\emph{<<author 2 affiliation>>}, \href{mailto:<<author 2 email>>}{<<author 2 email>>}\par}}

% \vspace*{0.75em} % third author

% \noindent{\LARGE{<<author 3 name>>}\par}
% \vspace*{0.5em}
% \noindent{{\large\emph{<<author 3 affiliation>>}, \href{mailto:<<author 3 email>>}{<<author 3 email>>}\par}}

\end{fullwidth}

\vspace*{1em}


\newthought{On September 8,} 1957, the Roman Catholic Church published the papal
encyclical on motion pictures, radio, and television. The title,
\emph{Miranda Prorsus}, is a Latin term meaning both ``absolutely
remarkable'' and ``remarkable advance'' in English. These different
connotations coalesced around powerful media technologies in the
encyclical's opening phrase ``Miranda prorsus technicae artis
inventa.''\footnote{Pius VII, \emph{Miranda Prorsus}, Encyclical Letter,
  September 8, 1957,
  \url{https://www.vatican.va/content/pius-xii/en/encyclicals/documents/hf_p-xii_enc_08091957_miranda-prorsus.html}
  (last accessed September 19, 2021). The opening phrase translates to
  ``remarkable technological inventions.''} The encyclical outlined Pope
Pius VII's position on cinema, radio, television, and the press, and it
provided advice for clergy, administrators, and laity on how best to
approach these powerful media technologies. Since encyclicals provide
papal guidance on matters of public concern, this one also represented
an attempt by the Catholic Church to consider media as sites through
which to comprehend broader social and spiritual problems. With that
twin purpose in mind, \emph{Miranda Prorsus} asks\emph{:} Are media
technologies threats to moral virtue, or can they be deployed as tools
for eternal salvation?

In what follows I suggest that \emph{Miranda Prorsus} can draw our
attention to a few different lines of argument we might explore in
accounting for the history of media studies. First, it offers us the
occasion to consider how knowledge about media travels within
communities that exist outside the academy. At the risk of sounding
flippant, I also want to say that it draws our attention to the fact
that ``media studies'' is something different organizations do all the
time. The more we recognize this fact, the better equipped we could be
to appreciate the pathways of exchange, circulation, imitation, and
incorporation that exist between academic and non-expert social
institutions. Second, in conceptualizing media both as a contemporary
concern and one grounded within the ambit of theological doctrine,
\emph{Miranda Prorsus} also asks us to consider the histories of
different styles of reasoning about media. To put it a different way,
then, I believe that

\enlargethispage{2\baselineskip}

\vspace*{2em}

\noindent{\emph{History of Media Studies}, vol. 1, 2021}




 \end{titlepage}



\noindent \emph{Miranda Prorsus} can be read as an
\emph{argument about media} as much as it constitutes a set of
instructions for other people about \emph{what to do} with the means of
mass communication. As such it calls on us to account for the history of
thinking about media in ways that make them objects for study and
intervention, whether by academics or, in this case, by the Vatican.
Both of these themes let us appreciate the benefits of a capacious sense
of what we understand as ``media studies'' force us to think more
carefully about what we have historically understood as media.

As primary documents encyclicals provide rich sources for thinking about
the communication of religious doctrine. Although intended for clergy
and laity, journalists, Catholic and non-Catholic alike, routinely cover
their dissemination. This gives encyclicals the quality, as Clare
Donagle explains, of ``an extended press release within which a given
pope can declare his agenda to the world.''\footnote{Clare Donagle,
  ``The Politics of Extra/Ordinary Time: Encyclical Thinking,''
  \emph{Cogent Arts and Humanities} 4, no. 1 (2017): n.p.} In their
rhetorical style, encyclicals also express a way of thinking about
matters of institutional concern that mixes biblical quotations with
previous church doctrine. This has the effect of situating the Catholic
Church's intervention within a particular temporal frame, one that
``lies outside of history, the possibility of eternal life through
Christian faith.''\footnote{Donagle, ``The Politics of Extra/Ordinary
  Time.''} For Donagle, the tension between an encyclical's policy
objectives and its reminder about the ephemerality of human existence
shows how encyclicals reflect a tension ``between Christianity's
evangelical and existential aspects.''\footnote{Donagle, ``The Politics
  of Extra/Ordinary Time.''}

With \emph{Miranda Prorsus} we can see the forward-looking view of the
Catholic Church refracted through its more traditional ontological and
theological visions. In contrast to the harder-edged encyclical on
cinema to U.S. clergy issued by Pius's predecessor, \emph{Vigilanti
Cura} (\emph{Vigilant Care}), which called for closer controls and even
the censorship of films, \emph{Miranda Prorsus} adopted a tone I would
characterize as anxiously optimistic. Blending prior church statements
on media with quotations from scripture and references to Thomas
Aquinas, the encyclical characterized media in ways familiar to
introductory media studies courses: showing considerable respect for its
technical possibilities; highlighting its capacity to appeal to sensory
perceptions and to provide knowledge of the world; noting its ability to
blur boundaries of public and domestic space; and recognizing
how~``these new forms of art exercise very great influence on the manner
of thinking and acting of individuals and every group of
men.''\footnote{Pius VII, \emph{Miranda Prorsus.}} The document shows
the church wrestling with the twin forces of media power: On the one
hand, it recognized that these technologies~``may be spread among men
like good seed which bring forth fruits of truth and goodness''; on the
other hand, it later acknowledged that ``not all obey the
gospel.''\footnote{Pius VII, \emph{Miranda Prorsus.}}

In \emph{Miranda Prorsus} we see Pius attempting to tip the balance in
his preferred direction. The encyclical argues that Catholics ranging
from film critics to bishops must be agents in promoting the responsible
arts of mediated communication, to play an active role in publicizing
Christian doctrine and encourage media producers to create works that
might fulfill the higher potential of art. The document details how some
church officials have made use of the capacity of long-distance
communication to ensure that ``Our voice, passing in sure and safe
flight over the expanse of the sea and land and even over the troubled
emotions of souls may reach men's minds in a healing influence, in
accordance with the demands of the task of the supreme apostolate,
confided to Us and today extended without limit.''\footnote{Pius VII,
  \emph{Miranda Prorsus.}} Assessing the various media forms in turn,
the encyclical provides instructions for national authorities to lobby
for more programs with Catholic themes, to monitor and measure media
content, and to train both the producers of media and its audiences to
take seriously their respective roles in making and consuming the
products of these powerful means of mass communication.

Furthermore, \emph{Miranda Prorsus} urged the clergy, its
administrators, and the Catholic Action groups and affiliated
organizations operating locally and nationally to play active roles in
teaching the laity and in informing the wider public about how to best
make forms of mass communication achieve their spiritual objectives. To
do this meant priest should have ``a sound knowledge of all questions
which confront the souls of Christians with regards to Motion Pictures,
Radio, and Television'' and ``must know what modern, science, art and
technique assert whenever they touch on the end of man and his moral and
religious life.''\footnote{Pius VII, \emph{Miranda Prorsus.}} This
involved encouraging clergy to provide guidance for film, television,
and radio producers, actors, broadcasters, and distributors, and to make
use of media to produce religious content reflecting Christian doctrine.
With some of the groundwork established, \emph{Miranda Prorsus} can be
viewed as one of a series of major statements the church made about
media during this time. Such efforts continued during the Second Vatican
Council decree, \emph{Inter Mirifica} (\emph{Among the Wonders}),
outlining the church's statement on social communication, which was
published just six years later. The document, along with the papal
instruction \emph{Communio et Progressio} (\emph{Community and
Progress}, 1971), played an important role in the disciplinary
development of communication studies within universities in several
places around the world.\footnote{For example, see Raúl Fuentes-Navarro,
  ``Institutionalization and Internationalization of the Field of
  Communication Studies in Mexico and Latin America,'' and Maria
  Immacolata Vassallo de Lopes and Richard Romancini, ``History of
  Communication}

In this cursory treatment of \emph{Miranda Prorsus} we can envision some
pathways that could enrich the historiography of media studies. In one
sense, it reminds us of the powerful place of religious institutions in
the historical development of media studies. Menahem Blondheim and
Hanael\marginnote{in Brazil: The Institutionalization of an
  Interdisciplinary Field,'' both in \emph{The International History of
  Communication Study}, ed. Peter Simonson and David W. Park (New York:
  Routledge, 2016), for invaluable analyses on the institutionalization
  of communication studies programs in Mexico and Latin America and
  Brazil, respectively.} Rosenberg use the term \emph{media theology} to think about ``the
historical and theoretical project of religious thinking on the
relevance of communications to the relationship between God and humans
and to the ideological implications of those ideas.''\footnote{Menahem
  Blondheim and Hanael Rosenberg, ``Media Theology: New Communication
  Technologies as Religious Constructs, Metaphors, and Experiences,''
  \emph{New Media and Society} 19, no 2 (2016): 44.} If we were to
follow Jeremy Stolow's argument that religion and technology ``operate
along a series of analogous binaries, including faith and reason,
fantasy and reality, enchantment and disenchantment, magic and science,
and fabrication and fact,'' then it appears reasonable to ask about the
influence of theological thinking on the development of the scholarly
field of media studies.\footnote{Jeremy Stolow, ``Introduction:
  Religion, Technology, and the Things in Between,'' in \emph{Deus in
  Machina: Religion, Technology, and the Things in Between}, ed. Jeremy
  Stolow (New York: Fordham University Press, 2013), 2.} Moreover,
thinking in terms of media theology asks us to recognize how studies of
media have been used to represent religious ideas and to recognize how
religious institutions themselves \emph{study media} so as to
incorporate them into their own theological activities.

In this respect, we can also consider the encyclical as a form of public
deliberation, of thinking out loud about the characteristics of media
technologies---of working it out on paper, if you will. ``Reasoning is
done in public as well as in private,'' Ian Hacking writes, ``but also
by talking and arguing and showing.''\footnote{Ian Hacking,
  \emph{Historical Ontology} (Cambridge: Harvard University Press,
  2002), 180.~} From this we can say that what we may want to call
``media studies'' can be understood as a mentality or set of
dispositions that people occupy at different times and in different
contexts, one in which everyone from academic researchers to non-profit
organizations sees the expansion of communication technologies as
signaling an understanding that broader social and political problems
need to be comprehended in media terms. In Anna Shechtman's account of
debates over how to understand media at a 1959 gathering in
Pennsylvania's Pocono Mountains, we see a range of elite thinkers, from
Hannah Arendt to Paul Lazarsfeld to Daniel Bell to James Baldwin, who
consider the concept of media as ``as a rhetorical vehicle---an
overdetermined metaphor for the technical, ideological, and
environmental conditions of modern life.''\footnote{Anna Shechtman,
  ``Command of Media Metaphors,'' \emph{Critical Inquiry} 47, no. 2
  (2021): 649.} Indeed, situated within \emph{Miranda Prorsus} we find
the broader concern about media being used as a tool ``in those nations
where atheistic communism is rampant'' to ``root out all religious
ideals from the mind,'' a reminder of how \emph{media} is a term used to
think \emph{through} and not just \emph{about}.\footnote{Pius VII,
  \emph{Miranda Prorsus.} For further discussion, see Frank J. Coppa,
  ``Pope Pius XII and the Cold War: The Confrontation between
  Catholicism and Communism,'' in \emph{Religion and the Cold War}, ed.
  Dianne Kirby (London: Palgrave Macmillan, 2003).}

In that process of thinking things through, we can also consider the
decisions made to discard or ignore other forms of argument, evidence,
or rhetorical styles, as part of different exercises of power and
privilege. From this perspective, \emph{Miranda Prorsus} must be
integrated into the \emph{history of arguments about media} to better
appreciate what kinds of reasons have been put forward, accepted, and
ignored in the process of making an abstract concept such as media its
own object of study. If we consider Amin Alhassan's assertion that
processes of knowledge production often erase their connection with
colonial and postcolonial subjects and locations, then any discussion of
different styles of reasoning about media must be carried out with an
aim to re-establishing those connections.\footnote{Amin Alhassan, ``The
  Canonic Economy of Communication and Culture: The Centrality of the
  Postcolonial Margins,'' \emph{Canadian Journal of Communication} 32,
  no. 1 (2007).} This would encourage us to think more closely about the
lived effects of the application of academic knowledge in different
settings, and to further prompt greater questioning of the canonicity of
the concepts that form part of our field of inquiry.

In its appeal for the development of local and regional institutional
infrastructure for media education, \emph{Miranda Prorsus} also
encourages us to account for the networks of bishops, Catholic Action
groups, public information efforts, and mass education campaigns as
sites for the production, dissemination, and debate about lay media
knowledge.\footnote{An excellent example of this is François Yelle's
  discussion of the impact of media-related encyclicals---including
  \emph{Miranda Prorsus}---in Quebec. See François Yelle, ``Fifty Years
  Ago Today When the Pope Knew Everything There Was to Know About
  Media'' (paper presented at Two Days of Canada conference, Brock
  University, St. Catherine's, ON, November 6, 2008).} It asks us to
consider the extent to which the Catholic Church was engaged in research
about media; the role that intellectuals played within those efforts,
and the distribution of that research through its different
publications. Accounting for the institutional and infrastructural
systems of media-knowledge transmission will better equip us to
re-situate religious institutions within historiographical accounts of
disciplinary development. It will also show how those institutions have
acted as sites where media theories are transformed into theological
praxis and as clearinghouses for the dissemination of academic and lay
knowledge about media.

Finally, the encyclical also pushes us to think about the places where
media studies knowledge has been used, misused, and abused in different
institutional settings, as part of its pedagogical and educational
objectives. What role have media technologies---and the specific
knowledge about media that would have inspired their use---played in the
church's efforts in the fields of education and media literacy? In the
Canadian context in which I am writing this article, it is impossible to
not consider this question in relation to the legacy of the residential
and day school systems operated by successive governments in partnership
with the Catholic Church (as well as churches from other denominations)
as part of a system of murder, trauma, separation, disempowerment, and
cultural genocide among the country's First Nations, Métis, and Inuit
communities. According to the 2015 Summary of the Final Report of
Canada's Truth and Reconciliation Commission, it was during the 1950s
and 1960s that ``the prime mission of the residential schools was the
cultural transformation of Aboriginal children.''\footnote{Truth and
  Reconciliation Commission of Canada, \emph{Honouring the Truth,
  Reconciling the Future: Summary of the Final Report of the Truth and
  Reconciliation Commission of Canada}. (Ottawa: Truth and
  Reconciliation Commission of Canada, 2015),~5.~} That history forces
us to consider whether an encyclical like \emph{Miranda Prorsus}
foreshadowed the deployment of media to support these objectives and to
inquire about the kinds of scholarly knowledge that might have used to
support structures of forced assimilation and violence both in Canada
and elsewhere. Addressing that question would prove valuable for
thinking more seriously about the overlapping institutional and
disciplinary histories, epistemological frameworks, and ontological
categories that constitute the way we have researched, taught, and
studied media in different contexts and over time.

Let us, then, embark on this journey to trouble the history of ``media
studies'' in as many contexts as possible. May it begin by going
directly to some of the institutions that have been doing media studies
alongside so many of us for a long time. Let us ask about the extent to
which that knowledge has served to support systems of oppression,
control, racism, and colonialism; to appreciate the dynamic between
scientific and lay knowledge; to encourage counter-histories of our
field; to multiply the contexts in which we appreciate media as an
object of study; and to draw attention to ignored and marginalized
voices.\footnote{For example, see Ellen Seiter, \emph{Television and New
  Media Audiences} (Oxford: Oxford University Press, 1996), 58­--60, and
  Patrick McCurdy ``Theorizing `Lay Theories of Media': A Case Study of
  the Dissent! Network at the 2005 Gleneagles G8 Summit,''
  \emph{International Journal of Communication} 5 (2011).~} And let us
more reflexively consider the ways that ``media studies'' can offer a
productive approach for undertaking more complex and subtle histories of
other areas of academic study.







\section{Bibliography}\label{bibliography}

\begin{hangparas}{.25in}{1} 



Alhassan, Amin. ``The Canonic Economy of Communication and Culture: The
Centrality of the Postcolonial Margins.'' \emph{Canadian Journal of
Communication} 32, no. 1 (2007): 103­--18.
\url{https://doi.org/10.22230/cjc.2007v32n1a1803}.

Blondheim, Menahem, and Hanael Rosenberg. ``Media Theology: New
Communication Technologies as Religious Constructs, Metaphors, and
Experiences.'' \emph{New Media and Society} 19, no. 1 (2016): 43--51.
\url{https://doi.org/10.1177\%2F1461444816649915}.

Coppa, Frank J. ``Pope Pius XII and the Cold War: The Post-War
Confrontation between Catholicism and Communism.'' In \emph{Religion and
the Cold War}, edited by Dianne Kirby, 50--66. London: Palgrave
Macmillan, 2003.

Donagle, Clare. ``The Politics of Extra/Ordinary Time: Encyclical
Thinking.'' \emph{Cogent Arts and Humanities} 4, no. 1 (2017): n.p.
\url{https://doi.org/10.1080/23311983.2017.1390918}.

Fuentes-Navarro, Raul. ``Institutionalization and Internationalization
of the Field of Communication Studies in Mexico and Latin America.'' In
\emph{The International History of Communication Study}, edited by Peter
Simonson and David W. Park, 325--46. New York: Routledge, 2016.

Hacking, Ian. \emph{Historical Ontology.} Cambridge: Harvard University
Press, 2002.

Lopes, Maria Immacolata Vassallo de, and Richard Romancini. ``History of
Communication in Brazil: The Institutionalization of an
Interdisciplinary Field.'' In \emph{The International History of
Communication Study}, edited by Peter Simonson and David W. Park,
346--66. New York: Routledge, 2016.

McCurdy, Patrick. ``Theorizing `Lay Theories of Media': A Case Study of
the Dissent! Network at the 2005 Gleneagles G8 Summit.''
\emph{International Journal of Communication}, no. 5 (2011): 619--30.
\url{https://ijoc.org/index.php/ijoc/article/view/842}.

Pius VII, \emph{Miranda Prorsus.} Encyclical Letter. September 8, 1957.
\url{https://www.vatican.va/content/pius-xii/en/encyclicals/documents/hf_p-xii_enc_08091957_miranda-prorsus.html}.

Seiter, Ellen. \emph{Television and New Media Audiences.} Oxford: Oxford
University Press, 1999.

Shechtman, Anna. ``Command of Media Metaphors.'' \emph{Critical Inquiry}
47, no. 4 (2021): 644--74. \url{https://doi.org/10.1086/714512}.

Stolow, Jeremy. ``Introduction: Religion, Technology, and the Things in
Between.'' In \emph{Deus in Machina: Religion, Technology and the Things
In Between}, edited by Jeremy Stolow, 1--24. New York: Fordham
University Press, 2013.

Truth and Reconciliation Commission of Canada. \emph{Honouring the
Truth, Reconciling for the Future: Summary of the Final Report of the
Truth and Reconciliation Commission of Canada}. Ottawa: Truth and
Reconciliation Commission of Canada, 2015.

Yelle, François. ``Fifty Years Ago Today When the Pope Knew Everything
There Was to Know About Media.'' Paper presented at Two Days of Canada
conference, Brock University, St. Catherine's, ON, November 6, 2008.



\end{hangparas}


\end{document}