% see the original template for more detail about bibliography, tables, etc: https://www.overleaf.com/latex/templates/handout-design-inspired-by-edward-tufte/dtsbhhkvghzz

\documentclass{tufte-handout}

%\geometry{showframe}% for debugging purposes -- displays the margins

\usepackage{amsmath}

\usepackage{hyperref}

\usepackage{fancyhdr}

\usepackage{hanging}

\hypersetup{colorlinks=true,allcolors=[RGB]{97,15,11}}

\fancyfoot[L]{\emph{History of Media Studies}, vol. 1, 2021}


% Set up the images/graphics package
\usepackage{graphicx}
\setkeys{Gin}{width=\linewidth,totalheight=\textheight,keepaspectratio}
\graphicspath{{graphics/}}

\title[Can the History of Communication and Media Research?]{Can the History of Communication and Media Research Proceed without the Philosophy of Technology?} % longtitle shouldn't be necessary

% The following package makes prettier tables.  We're all about the bling!
\usepackage{booktabs}

% The units package provides nice, non-stacked fractions and better spacing
% for units.
\usepackage{units}

% The fancyvrb package lets us customize the formatting of verbatim
% environments.  We use a slightly smaller font.
\usepackage{fancyvrb}
\fvset{fontsize=\normalsize}

% Small sections of multiple columns
\usepackage{multicol}

% Provides paragraphs of dummy text
\usepackage{lipsum}

% These commands are used to pretty-print LaTeX commands
\newcommand{\doccmd}[1]{\texttt{\textbackslash#1}}% command name -- adds backslash automatically
\newcommand{\docopt}[1]{\ensuremath{\langle}\textrm{\textit{#1}}\ensuremath{\rangle}}% optional command argument
\newcommand{\docarg}[1]{\textrm{\textit{#1}}}% (required) command argument
\newenvironment{docspec}{\begin{quote}\noindent}{\end{quote}}% command specification environment
\newcommand{\docenv}[1]{\textsf{#1}}% environment name
\newcommand{\docpkg}[1]{\texttt{#1}}% package name
\newcommand{\doccls}[1]{\texttt{#1}}% document class name
\newcommand{\docclsopt}[1]{\texttt{#1}}% document class option name


\begin{document}

\begin{titlepage}

\begin{fullwidth}
\noindent\LARGE\emph{Launch essay
} \hspace{85mm}\includegraphics[height=1cm]{logo3.png}\\
\noindent\hrulefill\\
\vspace*{1em}
\noindent{\Huge{Can the History of Communication and\\\noindent Media Research Proceed without the\\\noindent Philosophy of Technology?\par}}

\vspace*{1.5em}

\noindent\LARGE{Filipa Subtil} \href{https://orcid.org/0000-0003-2556-2192}{\includegraphics[height=0.5cm]{orcid.png}}\par}\marginnote{\emph{Filipa Subtil, ``Can the History of Communication and Media Research Proceed without the Philosophy of Technology?,'' \emph{History of Media Studies} 1 (2021), \href{https://doi.org/10.32376/d895a0ea.dab6ca65}{https://doi.org/ 10.32376/d895a0ea.dab6ca65}.} \vspace*{0.75em}}
\vspace*{0.5em}
\noindent{{\large\emph{Instituto Politécnico de Lisboa}, \href{mailto:fsubtil@escs.ipl.pt}{fsubtil@escs.ipl.pt}\par}} \marginnote{\href{https://creativecommons.org/licenses/by-nc/4.0/}{\includegraphics[height=0.5cm]{by-nc.png}}}

% \vspace*{0.75em} % second author

% \noindent{\LARGE{<<author 2 name>>}\par}
% \vspace*{0.5em}
% \noindent{{\large\emph{<<author 2 affiliation>>}, \href{mailto:<<author 2 email>>}{<<author 2 email>>}\par}}

% \vspace*{0.75em} % third author

% \noindent{\LARGE{<<author 3 name>>}\par}
% \vspace*{0.5em}
% \noindent{{\large\emph{<<author 3 affiliation>>}, \href{mailto:<<author 3 email>>}{<<author 3 email>>}\par}}

\end{fullwidth}

\vspace*{1em}


\newthought{Historians of communication} and media studies have never been very
interested in technology, but surely there is thinking about technology
in media studies, even if it is not often explicit. Consider the case of
uses and gratifications research as developed by Herta Herzog and later
elaborated by Elihu Katz, which tended to regard psychological and
sociological variables as real and primary, and the media as a
second-hand factor and manifestation of those variables. Does this
approach not contain the assumption that media technologies are merely
technical things used to accomplish certain ends? And consequently, that
these things are value-neutral---that technological objects do not play
a primary role in culture? Consider the case of Harold A. Innis: Does he
deserve the pejorative ``technological determinist'' for emphasizing
that the specific technological characteristics of a prevalent medium in
a given society condition the social practices of communication,
institutions, and systems of social organization and
power?\footnote{Harold A. Innis, \emph{Empire and Communications} (Toronto: University
  of Toronto Press, 1970); Harold A. Innis, \emph{The Bias of
  Communication} (Toronto: University of Toronto Press, 1999).
} Is it plausible to think
that certain technologies might themselves have political properties?

And how to evaluate the approach of the \emph{technische Medien} of
Friedrich A. Kittler, as he writes in the preface to \emph{Gramophone,
Film, Typewriter} that the ``media determine our situation'' and ``what
remains of people is what media can store and
communicate''?\footnote{Friedrich A. Kittler, \emph{Gramophone, Film, Typewriter} (Stanford,
  CA: Stanford University Press, 1999), xxxix.
} Should we take his
thesis that our knowledge is critically dependent on the cultural
techniques we invent? These questions have been hanging over the history
of communication and media research for a long time. They have become
even more pressing with the emergence 

\enlargethispage{2\baselineskip}

\vspace*{2em}

\noindent{\emph{History of Media Studies}, vol. 1, 2021}




 \end{titlepage}


\noindent of computer networks,
digitalization, and digital algorithmic calculation devices.

This essay is based on the observation that, while many philosophers and
theorists of technology are studying the media and importing ideas from
media studies into the philosophy of
technology,\footnote{Among others, see Günther Anders, ``The World as Phantom and as
  Matrix,'' Dissent 3, no.1 (1956); Langdon Winner, ``Mythinformation,''
  in \emph{The Whale and Reactor: A Search for Limits in an Age of High
  Technology} (Chicago: University of Chicago Press, 1989); Albert
  Borgmann, \emph{Holding On to Reality: The Nature of Information at
  the Turn of the Millennium} (Chicago: University of Chicago Press,
  1999); Pierre Musso, \emph{Télécommunications et Philosophie des
  Réseaux} (Paris: PUF, 1997); Pierre Musso, \emph{Critique des Réseaux}
  (Paris: PUF, 2003); and Don Ihde, \emph{Listening and Voice} (Albany:
  State University of New York Press, 2007).
} historians of media
studies have not granted enough attention to the question of technology.
A rare exception was a recent forum organized by Lana Rakow in
\emph{Journalism \& Mass Communication
Quarterly.}\footnote{Lana Rakow, ed., ``Philosophy of Technology: Who is in the Saddle?''
  (Invited Forum), \emph{Journalism \& Mass Communication Quarterly} 96,
  no. 2 (2019). The forum is composed of short texts authored by Lana
  Rakow, Jeremy Swartz, Carolyn Marvin, Robert K. Logan, and Beth
  Coleman.
} The fact is that
historians of media and communication research have not grappled with
the ideological and philosophical assumptions around technology that are
present in the work of the field's major historical figures. I contend
that we must start a dialogue with the philosophy of technology, and
with the social studies of science and technology, in order to analyze
the discipline's unconscious and embedded presuppositions about the
relationship between technology and media. Looking for other ways to
question technology could pave the way for research that is historical
and theoretically founded on the structural transformations of
communication in the modern era.

With this prospect in mind, I present notes that lead to a reflection
about the history of communication and media studies, and future paths
for research, which draws on the contributions that philosophers of
technology have made to thinking about media technologies. In these
notes, I focus on two crucial problems: (1) definitions of technology
and (2) the modern relationship between myth/utopia and communication
technologies.

\hypertarget{defining-technology-derivative-agent-central-actor-trickster}{%
\section{Defining Technology: Derivative Agent? Central Actor?
Trickster?}\label{defining-technology-derivative-agent-central-actor-trickster}}


The definition of technology most deeply rooted in modern life is that
technologies are means to free human beings from their limitations and
to positively transform human life. They are instruments, tools,
artifacts, things to accomplish desired ends. Admittedly, technical
objects can be used for good or for evil, but it is their use that may
be improper, not the instrument, tool, or artifact itself. According to
this perspective, technology is a value-neutral human product. There is
no reason to question technology, but only its use, or at most, the
context in which it is submersed. This notion has accompanied the modern
world's bet on the expansion of technological capacity. This
expansion---that is, an increase in human power---has been viewed as a
necessary condition for humans to solve the most diverse problems and
even to establish a materially abundant and harmonious society. It is
part of the modern belief in Progress, a \newpage
\noindent collective mentality that
understands History as a record of the improvement of the conditions of
human life.\footnote{Leo Marx, ``The Domination of Nature and the Redefinition of
  Progress,'' in \emph{Progress: Fact or Illusion}, ed. Leo Marx and
  Bruce Mazlish (Ann Arbor: University of Michigan Press, 1996).
}

The conception of technology as a means is implicit in many studies
based on the analysis of the uses and appropriations of the media and,
to some extent, their effects. In truth, it must be recognized that
technology develops by interacting with social and economic processes
and forces. It is also undeniable that different uses or appropriations
of media and technologies exist. The same point can be made about the
effects of technologies, which may be driven by users' utilization or
social policies. Without a doubt, it is the sociological milieu of the
audience which conditions the use of media, and human perception is an
active process of organization and structuring. But analysis of use, as
a method, is fundamentally blind to the cultural (as opposed to merely
sociological) nature of human life. That is to say, humans have a basic
cultural disposition to filter experiences in symbolic forms, as
Clifford Geertz\footnote{Clifford Geertz,~\emph{The Interpretation of Cultures}~(New
  York:~Basic Books, 1973).
} and the tradition
of cultural studies has clarified so well. To this shortcoming we add
another: If this approach to analysis based on uses and effects has
tended to be reductionist because it does not recognize the central
place that culture occupies in the life of symbolic beings, it has also
tended to be reductionist because it has not understood technology
(material culture) as an actor itself, a means and also an end.

In philosophical thinking about technology, Langdon Winner has been a
prominent voice in rejecting the exaggerated application of the social
determinist perspective.\footnote{Langdon Winner, \emph{Autonomous Technology: Technics-out-of Control
  as a Theme in Political Thought} (Cambridge, MA: MIT Press 1978);
  Langdon Winner, \emph{The Whale and the Reactor: A Search for Limits
  in an Age of High Technology} (Chicago: University of Chicago Press,
  1989).
} Instead
of reducing technical artifacts to the interaction of social forces, he
has urged that attention be paid to the characteristics of technical
objects and to the significance of those characteristics. He sees his
thinking in connection with the philosophical precept of Edmund Husserl,
of returning to things, and combines that principle with influences from
Ludwig Wittgenstein, Karl Marx, Lewis Mumford, and Jacques Ellul. In
1986, Winner argued that artifacts can contain political properties and
that technologies enhance ``forms of life.'' He provided various
examples to show that certain technologies were prepared beforehand to
favor certain social interests and patterns of
power.\footnote{See Langdon Winner, ''Do Artefacts Have Politics?'' in \emph{The Whale
  and Reactor}: \emph{A Search for Limits in an Age of High Technology}
  (Chicago: University of Chicago Press, 1989).
} More than three decades
later, this idea is very evident when we realize that, for example,
fossil fuels and green energies are articulated with very different
values and interests---or even when we realize the biases of the
algorithms that guide us in many decisions through an infrastructure of
calculations. It is also more evident today that societies can choose
technological structures that can, in turn, variously influence the ways
of working, traveling, consuming, communicating, and deciding.

Another thinker about technology, Carl Mitcham, distinguishes two ways,
two ideal types, of thinking about technology and its relation to human
life in general. There is, first, the engineering approach, which sees
technology as the core of what it is to be human, and therefore sees no
problem with the expansion of technology into all areas of life. The
second approach is the humanities-rooted philosophy of technology, which
asserts there are other legitimate forms of knowing, acting, and being
in the world, other than the technological
sort.\footnote{Carl Mitcham,~\emph{Thinking through Technology: The Path between
  Engineering and Philosophy} (Chicago: University of Chicago Press,
  1994).
} Mitcham, who is versed in
both engineering thought and thinkers such as Martin Heidegger, José
Ortega y Gasset, and Ivan Illich, defends a complex notion of technology
and maintains that it has four dimensions: object, knowledge, activity,
and volition. Thus, technology is a mere thing, but it is also a thing
to think with, a thing to act with, and a thing that influences the
shape of culture. For both Winner and Mitcham, a value-neutral vision of
technology prevents its inscription in culture, its critical analysis,
and leads to the neglect of the intentions---the social, economic and
political interests---of those who design, develop, finance, and control
it.

We find these definitions of technology in the work of media theorist
James W. Carey, who writes that technology is more than a ``group of
purposeful instruments''; they are ``things that shape the self and the
mind,'' that ``serve as instruments of
action.''\footnote{James W. Carey, ``Afterword: The Culture in Question,'' in \emph{James
  W. Carey. A Critical Reader}, ed. Eve Stryker Munson and Catherine A.
  Warren (Minneapolis: University of Minnesota Press, 1997), 316.
} He emphasizes that in
all industrial societies, and particularly in the US, technology is
``also the central character and actor in our {[}North Americans'{]}
social drama, and an end as well as a
means.''\footnote{Carey, ``Afterword,'' 316.
} Moreover, Carey goes as
far as to say that technology ``plays the role of the trickster in
American culture,''\footnote{Carey, ``Afterword,'' 316.
} given that
machines are not only believed to ``make history,'' but they also ``play
\ldots{} the role of a superlegislator with a dominating voice in the
conversation of the culture.''\footnote{Carey, ``Afterword,'' 317.
}
Could Carey's communication-as-culture approach be inserted into the
tradition that Mitcham calls the philosophy of technology in the
humanities? It should be noted that Carey, whose relationship with the
work of Innis was not a marriage of convenience, defended in his famed
1983 essay on the telegraph the claim that a thorough treatment of the
consequences of that instrument would demonstrate that it altered the
spatial and temporal boundaries of human interaction and brought about
new forms of language, ordinary knowledge, structures of social
relations, and economic and political
power.\footnote{James W. Carey, ``Technology and Ideology: The Case of the
  Telegraph,'' in \emph{Communication as Culture. Essays on Media and
  Society}, Rev. Ed. (New York: Routledge, 2009). See also, Filipa
  Subtil, ``Du Télégraphe à Internet: Enjeux Politiques Liés au
  Technologies de l'Information,'' in \emph{La Contribution en Ligne.
  Pratiques Participatives à l'Ére du Capitalisme Informationnel}, eds.
  Serge Proulx, José L. Garcia, and Lorna Heaton (Montreal: Les Presses
  d'Université du Québec, 2014).
} In turn, we can ask
whether and to what extent the engineering perspective of technology, as
defined by Mitcham, may be unconsciously incorporated into the thinking
of communication and media research that, contrary to Carey, does not
consider technology as a cultural and political force and thus
undermines the neutral vision of technology. In order to understand the
media, is the philosophy of technology necessary?

\hypertarget{the-question-of-the-demythologization-of-the-technological-communicational-utopia}{%
\section{The Question of the Demythologization of the
Technological-Communicational
Utopia}\label{the-question-of-the-demythologization-of-the-technological-communicational-utopia}}

The relationship between communication technologies, myth, and utopia is
another major topic in the question of technology. Social reality is not
only composed of what is in force, but also by hopes, ideas, myths, and
utopias---and technology in the modern world has appeared as the
necessary resource for the achievement of the continued improvement of
human life and the social world. In \emph{The Prophets of Paris,} the
historian Frank E. Manuel shows us how Turgot, Condorcet, Saint-Simon,
Fourier, Comte, and their followers were heralds of Progress as an
idealization of perfectible societies and had solutions for the ills and
problems of a wretched mankind.\footnote{Frank E. Manuel, \emph{The Prophets of Paris} (New York: Harper
  Torchbooks, 1962). See also Frank E. Manuel and Fritzie P. Manuel,
  \emph{Utopian Thought in the Western World} (Cambridge, MA: Harvard
  University Press, 1979).
}
Saint-Simon, one of Marx's utopian socialists and one of his
inspirations, was inclined toward a Promethean and technocratic version
of Progress. In France, the philosopher Pierre Musso has distinguished
himself for arguing that the imaginary of the Internet and cyberculture
go back to the great Saint-Simonian utopian project of universal
interconnection made possible by
technologies.\footnote{Pierre Musso, \emph{Télécommunications et Philosophie des Réseaux}
  (Paris: PUF, 1997); Pierre Musso, \emph{Critique des Réseaux} (Paris:
  PUF, 2003); and Pierre Musso, dir. \emph{Réseaux et Société} (Paris:
  PUF, 2003). About Musso, see José L. Garcia, ed., \emph{Pierre Musso
  and the Network Society: From Saint-Simonianism to the Internet}
  (Cham: Springer, 2016).
} In the US, Leo Marx
pointed to the interactions between culture, utopia, and technology
after the waning of the country's pastoral self-image through the
concept of ``technological sublime.'' Carey followed this lead and
appropriated this concept in the two texts he wrote with John J.
Quirk\footnote{James W. Carey and John J. Quirk, ``The Mythos of Electronic
  Revolution,'' in \emph{Communication as Culture. Essays on Media and
  Society}, Rev. Ed. (New York: Routledge, 2009); James W. Carey and
  John J. Quirk, ``The History of the Future,'' in \emph{Communication
  as Culture. Essays on Media and Society}. Rev. Ed. (New York:
  Routledge, 2009).
} for ``a project of
de-mystification,'' as Jefferson
Pooley\footnote{Jefferson D. Pooley, \emph{James W. Carey and Communication Research:
  Reputation at the University's Margins} (New York: Peter Lang, 2016),
  65.
} clarified, of the sublime
electronic rhetoric as an ideology of the future. To demythologize the
computer revolution was also Winner's purpose in
``Mythinformation.''\footnote{Langdon Winner, ``Mythinformation,'' in \emph{The Whale and Reactor}:
  \emph{A Search for Limits in an Age of High Technology} (Chicago:
  University of Chicago Press, 1989).
} More
recently, Vincent Mosco proposed the expression ``digital sublime'' in
order to dismantle the myths and power of
cyberspace.\footnote{Vincent Mosco, \emph{The} \emph{Digital Sublime}: \emph{Myth, Power,
  and Cyberspace} (Cambridge, MA: MIT Press, 2004).
}

However, we actually find the most robust tradition of demystifying the
technological-communicational utopia in a vast gallery of French media
theorists. Lucien Sfez,\footnote{Lucien Sfez, \emph{Technique et Idéologie. Un Enjeu de Pouvoir}
  (Paris: \href{https://fr.wikipedia.org/wiki/Le_Seuil}{Le
  Seuil},~\href{https://fr.wikipedia.org/wiki/2002}{2002}).
} who was
Pierre Musso's doctoral dissertation adviser, took decisive steps on a
path that includes Armand
Mattelart,\footnote{Armand Mattelart, \emph{Histoire de l'Utopie Planétaire: De la Cité
  Prophétique à la Société Globale} (Paris: PUF, 1999).
} Philipe
Breton,\textsuperscript{23} Patrick
Flichy,\textsuperscript{24} and Dominique
Cardon.\textsuperscript{25} These theorists steer
away from the cybernetic tradition of Norbert Wiener regarding the
supposed similarity between nervous systems and electronic machines;
from the ``network society'' of Manuel Castells, where the logic of a
network is the new social morphology of
societies;\textsuperscript{26} and from the image of
the Internet as a ``collective intelligence,'' ``thinking network,'' or
``planetary brain'' of Pierre
Lévy.\textsuperscript{27} Among the cited French
theorists, we find one of the most systematic attempts to delegitimize
the technological utopias of communication. They accentuate and at the
same time call into question the fascination that these utopias have
been able to mobilize. They\marginnote{\textsuperscript{23} Philippe Breton,~\emph{L'Utopie de la Communication. L'Émergence de
  l'Homme sans Intérieur} (Paris: La Découverte, 1992).
} have\marginnote{\textsuperscript{24} Patrick Flichy, \emph{Une Histoire de la Communication Moderne. Espace
  Public et Vie Privée} (Paris: La Découverte, 1991).
} shown\marginnote{\textsuperscript{25} Dominique Cardon, \emph{À Quoi Rêvent les Algorithmes. Nos Vies à
  l'Heure des Big Data} (Paris: Seuil, 2015).
} us\marginnote{\textsuperscript{26} Manuel Castells, \emph{The Information Age: Economy, Society, and
  Culture} (Oxford: Blackwell, 1996, 1997, 1998).
} how\marginnote{\textsuperscript{27} Pierre Levy, \emph{L'Intelligence Collective. Pour une Anthropologie
  du Cyberespace} (Paris: La Découverte,~1994).} many new technologies of
the past have fallen very short of the unreasonable expectations that
were placed on them, how they failed in their promises of achieving a
promising future for all of humanity, bringing instead new plagues to
our societies: mass manipulation, symbolic violence, incentive to
consumerism, addictive behaviors, simulacra, and loss of meaning.

In conclusion: Refusing the naive paths of denying technology or
enthusiastically embracing it, is it not imperative to study how media
technologies and modes of communication and our social aspirations have
been articulated? In order to answer this question, this text proposes
that the field (1) should be more attendant to the explicit (and often
implicit) views of technology contained in past media scholarship; and
(2) should broaden our understanding of what counts as ``history of
media studies'' to include the corpus of work by philosophers of
technology.~







\section{Bibliography}\label{bibliography}

\begin{hangparas}{.25in}{1} 



Anders, Günther. ``The World as Phantom and as Matrix.'' \emph{Dissent}
3, no.1 (1956), 14--24.

Borgmann, Albert. \emph{Holding On to Reality: The Nature of Information
at the Turn of the Millennium}. Chicago: University of Chicago Press,
1999.

Breton, Philippe.~\emph{L'Utopie de la Communication. L'Émergence de
l'Homme sans Intérieur}. Paris: La Découverte, 1992.

Cardon, Dominique. \emph{À Quoi Rêvent les Algorithmes. Nos Vies à
l'Heure des Big Data.} Paris: Seuil, 2015.

Carey, James W. ``Afterword: The Culture in Question.'' In \emph{James
W. Carey. A Critical Reader}. Edited by Eve Stryker Munson and Catherine
A. Warren, 308--339. Minneapolis: University of Minnesota Press, 1997.

---------. ``Technology and Ideology: the Case of the Telegraph.'' In
\emph{Communication as Culture. Essays on Media and Society}. Revised
Edition, 155--77. New York: Routledge, 2009.

Carey, James W., and John J. Quirk. ``The Mythos of Electronic
Revolution.'' In \emph{Communication as Culture. Essays on Media and
Society}. Revised Edition, 87--108. New York: Routledge, 2009.

---------. ``The History of the Future.'' In \emph{Communication as
Culture. Essays on Media and Society}. Revised Edition, 133--54. New
York: Routledge, 2009.

Castells, Manuel\emph{. The Information Age: Economy, Society, and
Culture}. Oxford: Blackwell, 1996, 1997, 1998.

Flichy, Patrick. \emph{Une Histoire de la Communication Moderne. Espace
Public et Vie Privée}. Paris: La Découverte, 1991.

Garcia, José L., ed. \emph{Pierre Musso and the Network Society: From
Saint-Simonianism to the Internet}. Cham: Springer, 2016.

Geertz,~Clifford. \emph{The Interpretation of Cultures}. New York:~Basic
Books, 1973.

Innis, Harold A. \emph{Empire and Communications}. Toronto: University
of Toronto Press, 1970.

---------. \emph{The Bias of Communication}. Toronto: University of
Toronto Press, 1999.

Kittler, Friedrich A. \emph{Gramophone, Film, Typewriter.} Stanford, CA:
Stanford University Press, 1999.

Levy, Pierre. \emph{L'Intelligence Collective. Pour une Anthropologie du
Cyberespace.} Paris: La Découverte,~1994.

Pooley, Jefferson D. \emph{James W. Carey and Communication Research:
Reputation at the University's Margins}. New York: Peter Lang, 2016.

Manuel, Frank E. \emph{The Prophets of Paris.} New York: Harper
Torchbooks, 1962.

Manuel, Frank E., and Fritzie P. Manuel. \emph{Utopian Thought in the
Western World.} Cambridge, MA: Harvard University Press, 1979.

Marx, Leo. ``The Domination of Nature and the Redefinition of
Progress.'' In Progress: Fact or Illusion, edited by Leo Marx and Bruce
Mazlish, 201--18. Ann Arbor: University of Michigan Press, 1996.

---------. \emph{The Machine in the Garden. Technology and the Pastoral
Ideal in America}. Oxford: Oxford University Press, 2000.

Mattelart, Armand. \emph{Histoire de l'Utopie Planétaire: De la Cité
Prophétique à la Société Globale}. Paris: PUF, 1999.

Mitcham, Carl.~\emph{Thinking through Technology: The Path between
Engineering and Philosophy}. Chicago: University of Chicago Press, 1994.

Mosco, Vincent. \emph{The} \emph{Digital Sublime}: \emph{Myth, Power,
and Cyberspace}. Cambridge, MA: MIT Press, 2004.

Musso, Pierre. \emph{Télécommunications et Philosophie des Réseaux}.
Paris: PUF, 1997.

---------. \emph{Critique des Réseaux}. Paris: PUF, 2003.

---------, dir. \emph{Réseaux et Société}. Paris: PUF, 2003.

Rakow, Lana, ed. ``Philosophy of Technology: Who is in the Saddle?''
(Invited Forum). \emph{Journalism \& Mass Communication Quarterly} 96,
no. 2 (2019), 351--66.
\url{https://doi.org/10.1177\%2F1077699019841380}.

Sfez, Lucien. \emph{Technique et Idéologie. Un Enjeu de Pouvoir}. Paris:
Le Seuil,~2002.

Subtil, Filipa. ``Du Télégraphe à Internet: Enjeux Politiques liés au
Technologies de l'Information.'' In \emph{La Contribution en Ligne.
Pratiques Participatives à l'Ére du Capitalisme Informationnel}, edited
by Serge Proulx, José L. Garcia, and Lorna Heaton, 115-125. Montreal:
Les Presses d'Université du Québec, 2014.

Winner, Langdon. \emph{The Whale and the Reactor: A Search for Limits in
an Age of High Technology}. Chicago: University of Chicago Press, 1989.

---------. \emph{Autonomous Technology: Technics-out-of Control as a
Theme in Political Thought}. Cambridge: MIT Press 1978.







\end{hangparas}

\vspace*{1.5em}

\textbf{Acknowledgements:} The author would like to thank the editors
for their attentive reading, comments, and suggestions.

\vspace*{0.75em}

\noindent\textbf{Funding:} This work is financed by Portuguese national funds
through FCT---Foundation for Science and Technology, through the scope
of the project Refª UIDB/05021/2020.

\end{document}