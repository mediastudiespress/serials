% see the original template for more detail about bibliography, tables, etc: https://www.overleaf.com/latex/templates/handout-design-inspired-by-edward-tufte/dtsbhhkvghzz

\documentclass{tufte-handout}

%\geometry{showframe}% for debugging purposes -- displays the margins

\usepackage{amsmath}

\usepackage{hyperref}

\usepackage{fancyhdr}

\usepackage{hanging}

\hypersetup{colorlinks=true,allcolors=[RGB]{97,15,11}}

\fancyfoot[L]{\emph{History of Media Studies}, vol. 1, 2021}


% Set up the images/graphics package
\usepackage{graphicx}
\setkeys{Gin}{width=\linewidth,totalheight=\textheight,keepaspectratio}
\graphicspath{{graphics/}}

\title[Looking Back Together]{Looking Back Together to Become `Contemporaries in Discipline'} % longtitle shouldn't be necessary

% The following package makes prettier tables.  We're all about the bling!
\usepackage{booktabs}

% The units package provides nice, non-stacked fractions and better spacing
% for units.
\usepackage{units}

% The fancyvrb package lets us customize the formatting of verbatim
% environments.  We use a slightly smaller font.
\usepackage{fancyvrb}
\fvset{fontsize=\normalsize}

% Small sections of multiple columns
\usepackage{multicol}

% Provides paragraphs of dummy text
\usepackage{lipsum}

% These commands are used to pretty-print LaTeX commands
\newcommand{\doccmd}[1]{\texttt{\textbackslash#1}}% command name -- adds backslash automatically
\newcommand{\docopt}[1]{\ensuremath{\langle}\textrm{\textit{#1}}\ensuremath{\rangle}}% optional command argument
\newcommand{\docarg}[1]{\textrm{\textit{#1}}}% (required) command argument
\newenvironment{docspec}{\begin{quote}\noindent}{\end{quote}}% command specification environment
\newcommand{\docenv}[1]{\textsf{#1}}% environment name
\newcommand{\docpkg}[1]{\texttt{#1}}% package name
\newcommand{\doccls}[1]{\texttt{#1}}% document class name
\newcommand{\docclsopt}[1]{\texttt{#1}}% document class option name


\begin{document}

\begin{titlepage}

\begin{fullwidth}
\noindent\LARGE\emph{Launch essay
} \hspace{85mm}\includegraphics[height=1cm]{logo3.png}\\
\noindent\hrulefill\\
\vspace*{1em}
\noindent{\Huge{Looking Back Together to Become\\\noindent `Contemporaries in Discipline'\par}}

\vspace*{1.5em}

\noindent\LARGE{Sarah Cordonnier}\par}\marginnote{\emph{Sarah Cordonnier, ``Looking Back Together to Become `Contemporaries in Discipline,''' \emph{History of Media Studies} 1 (2021), \href{https://doi.org/10.32376/d895a0ea.b8153251}{https://doi.org/ 10.32376/d895a0ea.b8153251}.} \vspace*{0.75em}}
\vspace*{0.5em}
\noindent{{\large\emph{Université Lyon II}, \href{mailto:sarah.cordonnier@gmail.com}{sarah.cordonnier@gmail.com}\par}} \marginnote{\href{https://creativecommons.org/licenses/by-nc/4.0/}{\includegraphics[height=0.5cm]{by-nc.png}}}

% \vspace*{0.75em} % second author

% \noindent{\LARGE{<<author 2 name>>}\par}
% \vspace*{0.5em}
% \noindent{{\large\emph{<<author 2 affiliation>>}, \href{mailto:<<author 2 email>>}{<<author 2 email>>}\par}}

% \vspace*{0.75em} % third author

% \noindent{\LARGE{<<author 3 name>>}\par}
% \vspace*{0.5em}
% \noindent{{\large\emph{<<author 3 affiliation>>}, \href{mailto:<<author 3 email>>}{<<author 3 email>>}\par}}

\end{fullwidth}

\vspace*{1em}


\newthought{Why should our} history matter? How can we produce it without being
historians ourselves? And for whom shall this history be written? I
propose to revisit these basic questions by relying on the French case,
and by putting this case into dialogue with others in order to find
collective ways of answering them.

\hypertarget{why-should-our-history-matter}{%
\section{Why Should Our History
Matter?}\label{why-should-our-history-matter}}

During the second half of the 1990s, twenty years after the
institutional recognition of the discipline in
1975,\footnote{About the history of SIC, see Stefanie Averbeck-Lietz et al.,
  ``Communication Studies in France: Looking for a~`Terre du milieu'?''
  \emph{Publizistik} 64 (2019).
} the French \emph{sciences de
l'information et de la communication} (SIC) were put under historical
scrutiny. Among other initiatives, the most consistent work came from a
structured study group that was itself created in 1997 after ``agitated,
even heated debates,'' according to its leader Robert
Boure.\footnote{Robert Boure, ed., \emph{Les origines des Sciences de l'information et
  de la communication}, \emph{Regards croisés} (Villeneuve d'Ascq:
  Presses du Septentrion, 2002): 9. All translations from the French are
  mine.
} Hosted by the French
Society of Information and Communication Sciences, this group was called
``scientific theories and practices'' and was dedicated to the history
of the French field. In 2002 it gave rise to a ``first production,
original in our field, {[}which{]} calls for others, more ambitious in
terms of both questions and
methodologies.''\footnote{Boure, \emph{Les origines}, 14.
} This book was
developed through seminars, discussions around texts produced by the
participants, and the collective definition of three axes articulating
the topics they dealt with: addressing the public of the SIC
``practitioners,'' focusing on the institutional dimensions first and on
their articulation with the production of ideas, and clarifying and
contextualizing the origins of the
discipline.\footnote{Boure, \emph{Les origines}, 10--11.
} And yet, after this
``first production,'' and with the exception of a series of
\enlargethispage{2\baselineskip}

\vspace*{2em}

\noindent{\emph{History of Media Studies}, vol. 1, 2021}




 \end{titlepage}


\noindent follow-up
articles published mainly by the same
authors,\footnote{In \emph{Questions de communication} 10 (2006), 11, and 12 (2007),
  \url{https://journals.openedition.org/questionsdecommunication/}.
} the history of the French
discipline has been almost totally abandoned after this ten-year period
(1997--2007).



Many local reasons could explain this waning of interest: the resolution
of internal conflicts, the acceptance of the heterogeneity of the
discipline, a lack of time due to new orientations in national academic
policies, etc. ``Normal'' disciplinary life paradoxically runs counter
to a concern for disciplinary development as such. ``Discipline''
becomes for most communication scholars a convenient and therefore
invisible device. This story is not isolated. Outside of France one also
finds numerous cases where interest in the history of sciences dedicated
to communication has vanished after a period of greater historical study
that was based on personal acquaintance and shared concerns for the
(fragile) institutional premises of the
discipline.\footnote{For the US context, see Jefferson Pooley, ``The Declining Significance
  of Disciplinary Memory,'' in \emph{Handbuch
  kommunikationswissenschaftliche Erinnerungsforschung}, ed. Netzwerk
  Kommunikationswissenschaftliche Erinnerungsforschung (Berlin: de
  Gruyter, forthcoming 2021).
} In these enterprises,
which both require and constitute a collective, history depends on
extra-cognitive incentives and interests more than on mere academic
research. History was then necessary because it was both a political
issue and a political tool to make room, socially and academically, for
something that did not yet exist.

Our contemporary international contexts are characterized by a massive
increase in the number of researchers; by the professionalization,
standardization, and anonymity of academic practices; and by the
development of areas of specialization. Skillful and engaged historical
work seems to be out of reach almost everywhere now. Wolfgang Donsbach
prefaced his own remarks about the state of the discipline by saying:

\begin{quote}
Let me warn you first: Any account of a state of a discipline is limited
in scope and is biased. It is limited because the field grows faster
than the capacity of the average scholar to process and digest new
information and thus keep an overview. And it is -- by default -- biased
because people differ in what they think is good and what they think is
relevant research, thus disagreeing on what is the ``right way to
scientific knowledge.'' \footnote{Wolfgang Donsbach, ``The Identity of Communication Research,''
  \emph{Journal of Communication} 56, no. 3 (2006): 437.
}
\end{quote}

The fragmentation and progressive specialization of any growing
discipline, and especially ours, has important consequences in an
international environment. Previous political stakes have vanished along
with the sense that the participants know ``all and everyone'' in the
field. At the same time, scientific requirements are more stringent,
decontextualized, and depersonalized. In this situation, meaningful
history implies not only theoretical reading and methodological
compliance, but also clarification of irritating differences that turn
out to be more ``cultural'' than ``scientific,'' as pertains to the
authors under consideration, the method of developing an argument, the
writing style, the respective weight of the theory and the empirical
material, the interest in remote domains or areas, and the acceptance
that there is more than one, in the words of Donsbach, ``right way to
scientific knowledge.'' In sum, meaningful history relies upon the
resolute creation of an international ``milieu,'' where the delicate
balance between methodological demand and political significance can be
shaped afresh.

\hypertarget{looking-back-how-to-produce-history-without-being-a-historian}{%
\section{Looking Back: How to Produce History without Being a
Historian?}\label{looking-back-how-to-produce-history-without-being-a-historian}}

Pioneering and engaged research about the history of the field of
communication was not always so attentive to the requirements of
science, or of good history. Indeed, the mediocrity of much of this work
has attracted significant critical
attention.\footnote{A detailed list of authors who ``sharply criticized the traditional
  historiography of the field'' is given in Maria Löblich and Andreas M.
  Scheu, ``Writing the History of Communication Studies,''
  \emph{Communication Theory} 21 (2011): 1. See also, among others:
  Hanno Hardt, ``Foreword,'' in \emph{The History of Media and
  Communication Research: Contested Memories}, ed. David Park and
  Jefferson Pooley (New York, Peter Lang, 2008): xi.
} Still, we must confront
these methodological issues, as historians almost completely neglect the
history of our field, which is thus only produced by its members.

If the prospect of extra-disciplinary study raises questions about the
(non)involvement of researchers and their (lack of) specialized or
insider knowledge, these same questions apply to researchers taking
their own discipline as an object. Ethically speaking, can we adopt a
historical, sociological, or epistemological approach without being a
historian, sociologist, or epistemologist? Don't we risk producing a
history that is biased, hagiographic, inaccurate---or all three at the
same time?

Robert Boure and the other ``historians'' of the SIC carefully develop
and make explicit their reflexive position as amateur historians
involved in their object of study, constantly paying attention to the
pitfalls they can encounter as such. Their stated intention is

\begin{quote}
to produce a non-hagiographic and non-retrospective history; and
consequently, a non-official history which is not the magnified account
of the emergence and then the success of current scientific and/or
institutional ``truths'' which one readily imagines to be stable---but
which, on the contrary, endeavors to take into account the elementary
requirements of the historical method\ldots{} while remaining meaningful
(because it is open to contemporary
questioning).\footnote{Boure, \emph{Les origines}, 10.
}
\end{quote}

The solution they found consists in mingling scientific concern and
relevance, while addressing ``time'' with the tools of our own
discipline. On top of a ``historical'' approach, Boure suggests more
specifically a genealogical method, a method that ``abandons the logic
of linearity, of temporal evolutionary continuity (and thus of the
search for origins or precursors) to highlight instead the various
temporalities.''\footnote{Boure, \emph{Les origines}, 39. See also Hardt, ``Foreword,'' esp. xv
  and xvi.
} Boure also
observes that the non-historian researchers cannot

\begin{quote}
have the luxury of rejecting, if not presentism, at least any allusion
to the present. If only for two reasons: on the one hand, they will
probably only be read by researchers in their own field, and on the
other hand, they will have to answer to the latter. Therefore, it is
difficult to see how they can leave aside the questions which, \emph{hic
et nunc}, interest their ``contemporaries in
discipline.''\footnote{Boure, \emph{Les origines}, 38.
}
\end{quote}

\noindent Highlighting various temporalities and, in an international perspective,
highlighting various \emph{spatialities} as well, could help us chart a
path to resolving the contradictions of a history without historians,
with the help of other disciplines and the strengths of our own.

\hypertarget{whom-are-the-histories-of-others-addressed-to}{%
\section{Whom Are the Histories of ``Others'' Addressed
to?}\label{whom-are-the-histories-of-others-addressed-to}}

Building upon the two preceding questions, we see that the production of
a meaningful history in and of our field requires both a purpose and the
elucidation of a point of view, both within and outside the discipline,
the academic environment, and the immediate national context. Neglecting
these, as it is too often the case in international editorial projects,
will result in ``bad'' histories. And yet, these two criteria are still
not enough.

I mentioned earlier how French researchers, like those of other
countries, largely turned away from the history of their own national
discipline. In a context where academic internationalization is mostly
promoted as a seductive notion, but without further indications about
what it implies regarding the ``traveling
theories''\footnote{Edward Said, ``Traveling Theory,'' in \emph{The World, the Text, and
  the Critic} (Cambridge, MA: Harvard University Press, 1983).
} and the ways to deal
with them, a crucial question must be posed: Why should we be interested
in reading about far-away situations and histories? In a way, this
question squares the first two ranges of problems and, if not addressed,
international publications can do more harm than good: By definition,
the information they give about national context is truncated,
simplified, decontextualized, and often hard for a foreigner to
understand. Researchers from one country are better able to situate
their disciplinary history within a national history, and yet these same
researchers are also, by dint of their immersion in that national
context, more likely to function within that context in a manner that
reflects their own various and not always avowable interests. The
inverse of this situation applies to researchers from other countries.
Moreover, through their ``neutral'' editorial postures, scholarship like
this can give a false impression of comparability, or even homogeneity,
of some of the issues in the national and international fields:

\begin{quote}
``World tours'' as illustrated by English-language volumes that bring
together dozens of country cases and conference panels featuring
multi-country experts should not be seen as synonymous with
``de-westernization.'' \ldots{} Making international research available
does not change prevalent parochialism; in fact, it can exist separately
in a crowded and fragmented field of
research.\footnote{Silvio Waisbord and Claudia Mellado, ``De-Westernizing Communication
  Studies: A Reassessment,'' \emph{Communication Theory} 24, no. 4
  (2014): 365.
}
\end{quote}

That is how the ``international'' literature can conceal certain crucial
differences between and within countries, to the detriment of the most
dominated: nationally, ``the forgotten, the unknown, the unrecognized or
the `exotic,'\,''\footnote{Robert Boure, ``Les sciences de l'information et de la communication
  au risque de l'expertise? Sur et sous des pratiques scientifiques,''
  \emph{Réseaux} 82--83 (1997): 246.
} ``the `losers'
and `outsiders' of the discipline, \ldots{} those who have not succeeded
in making a career in it, those who have left it, those within it who
have refused to conform to its
standards,''\footnote{Stéphane Olivesi, ``À propos de l'institutionnalisation des SIC. Pour
  une histoire localisée,'' \emph{Questions de communication} 12 (2007):
  223.
} and internationally,
by creating ``black holes'':

\begin{quote}
Eastern Europe was for a long time in modern history a veritable ``black
hole'' about which little was known in Western academia, except to a
minority of specialists. While this has started to change with the
advent of the third wave of democratization in 1989/90, when much
academic interest became focused on it, in the field of communication
and media studies Eastern Europe is still very much perceived as the
undifferentiated ``other,'' somewhat like the ``global
south.''\footnote{Zrinjka Peruško and Dina Vozab, ``The Field of Communication in
  Croatia: toward a Comparative History of Communication Studies in
  Central and Eastern Europe,'' in the \emph{International History of
  Communication Study}, ed. Peter Simonson and David Park (New York \&
  London: Routledge, 2016), 213.
}
\end{quote}

The vigilance about these processes must remain constant, as they cannot
be solved once and for all.\footnote{ Stefanie Averbeck-Lietz gives us precious insights about the path(s)
  to follow and its pitfalls: Stefanie Averbeck-Lietz, ed.,
  \emph{Kommunikationswissenschaft im internationalen Vergleich.
  Transnationale Perspektiven} (Wiesbaden: Springer VS, 2017).} How
can we be interested in ``others'' without making them exotic? How can
we (re)discover forgotten theories, to share our disparate ways of
producing knowledge? How can we become ``contemporaries in discipline''
in spite of all the differences? Finding ways, space, and time to
confront these questions together could be a beautiful challenge for an
authentically and ethically international journal.







\section{Bibliography}\label{bibliography}

\begin{hangparas}{.25in}{1} 



Averbeck-Lietz, Stefanie, ed. \emph{Kommunikationswissenschaft im
internationalen Vergleich. Transnationale Perspektiven}. Wiesbaden:
Springer VS, 2017.

Averbeck-Lietz, Stefanie, Fabien Bonnet, Sarah Cordonnier, and Carsten
Wilhelm. ``Communication Studies in France: Looking for a~`Terre du
milieu'?'' \emph{Publizistik} 64 (2019): 363--80.
\url{https://doi.org/10.1007/s11616-019-00504-3}.

Boure, Robert. ``Les sciences de l'information et de la communication au
risque de l'expertise? Sur et sous des pratiques scientifiques
{[}Information and communication sciences at the risk of expertise?
Above and below scientific practices{]}.'' \emph{Réseaux} 82--83 (1997):
233--53. \url{https://doi.org/10.3406/reso.1997.3068}.

Boure, Robert, ed. \emph{Les origines des Sciences de l'information et
de la communication. Regards croisés {[}The origins of Information and
Communication Sciences. Crossed views{]}}. Villeneuve d'Ascq: Presses du
Septentrion, 2002.

Donsbach, Wolfgang. ``The Identity of Communication Research.''
\emph{Journal of Communication} 56, no. 3 (2006): 437--48.
\url{https://doi.org/10.1111/j.1460-2466.2006.00294.x}.

Hardt, Hanno. ``Foreword.'' In \emph{The History of Media and
Communication Research: Contested Memories}, edited by David Park and
Jefferson Pooley, xi---xvii. New York: Peter Lang, 2008.

Löblich, Maria, and Andreas M. Scheu. ``Writing the History of
Communication Studies: A Sociology of Science Approach.''
\emph{Communication Theory} 21, no. 1 (2011): 1--22.
\url{https://doi.org/10.1111/j.1468-2885.2010.01373.x}.

Olivesi, Stéphane. ``À propos de l'institutionnalisation des SIC. Pour
une histoire `localisée' {[}About the institutionalization of the
History of Information and Communication Sciences. For a `localized'
History{]}.'' \emph{Questions de communication} 12 (2007): 203--26.
\url{https://doi.org/10.4000/questionsdecommunication.2392}.

Peruško, Zrinjka and Dina Vozab. ``The Field of Communication in
Croatia: toward a Comparative History of Communication Studies in
Central and Eastern Europe.'' In \emph{The International History of
Communication Study}, edited by Peter Simonson and David Park. New York:
Routledge, 2016: 213--34.

Pooley, Jefferson. ``The Declining Significance of Disciplinary
Memory.'' In \emph{Handbuch kommunikationswissenschaftliche
Erinnerungsforschung}, edited by the Netzwerk
Kommunikationswissenschaftliche Erinnerungsforschung. Berlin: de
Gruyter, forthcoming 2021.

Said, Edward. ``Traveling Theory.'' In \emph{The World, the Text, and
the Critic,} 226--47. Cambridge, MA: Harvard University Press, 1983.

Waisbord, Silvio, and Claudia Mellado. ``De-Westernizing Communication
Studies: A Reassessment.'' \emph{Communication Theory} 24, no. 4 (2014):
361--72. \url{https://doi.org/10.1111/comt.12044}.



\end{hangparas}


\end{document}