% see the original template for more detail about bibliography, tables, etc: https://www.overleaf.com/latex/templates/handout-design-inspired-by-edward-tufte/dtsbhhkvghzz

\documentclass{tufte-handout}

%\geometry{showframe}% for debugging purposes -- displays the margins

\usepackage{amsmath}

\usepackage{hyperref}

\usepackage{fancyhdr}

\usepackage{hanging}

\hypersetup{colorlinks=true,allcolors=[RGB]{97,15,11}}

\fancyfoot[L]{\emph{History of Media Studies}, vol. 1, 2021}


% Set up the images/graphics package
\usepackage{graphicx}
\setkeys{Gin}{width=\linewidth,totalheight=\textheight,keepaspectratio}
\graphicspath{{graphics/}}

\title[Against the `Vocation of Autopsy']{Against the `Vocation of Autopsy': Blackness and/in US Communication Histories} % longtitle shouldn't be necessary

% The following package makes prettier tables.  We're all about the bling!
\usepackage{booktabs}

% The units package provides nice, non-stacked fractions and better spacing
% for units.
\usepackage{units}

% The fancyvrb package lets us customize the formatting of verbatim
% environments.  We use a slightly smaller font.
\usepackage{fancyvrb}
\fvset{fontsize=\normalsize}

% Small sections of multiple columns
\usepackage{multicol}

% Provides paragraphs of dummy text
\usepackage{lipsum}

% These commands are used to pretty-print LaTeX commands
\newcommand{\doccmd}[1]{\texttt{\textbackslash#1}}% command name -- adds backslash automatically
\newcommand{\docopt}[1]{\ensuremath{\langle}\textrm{\textit{#1}}\ensuremath{\rangle}}% optional command argument
\newcommand{\docarg}[1]{\textrm{\textit{#1}}}% (required) command argument
\newenvironment{docspec}{\begin{quote}\noindent}{\end{quote}}% command specification environment
\newcommand{\docenv}[1]{\textsf{#1}}% environment name
\newcommand{\docpkg}[1]{\texttt{#1}}% package name
\newcommand{\doccls}[1]{\texttt{#1}}% document class name
\newcommand{\docclsopt}[1]{\texttt{#1}}% document class option name


\begin{document}

\begin{titlepage}

\begin{fullwidth}
\noindent\LARGE\emph{Launch essay
} \hspace{85mm}\includegraphics[height=1cm]{logo3.png}\\
\noindent\hrulefill\\
\vspace*{1em}
\noindent{\Huge{Against the `Vocation of Autopsy': Blackness and/in US Communication Histories\par}}

\vspace*{1.5em}

\noindent\LARGE{Armond R. Towns}\par}\marginnote{\emph{Armond R. Towns, ``Against the `Vocation of Autopsy': Blackness and/in US Communication Histories,'' \emph{History of Media Studies} 1 (2021), \href{https://doi.org/10.32376/d895a0ea.89f81da7}{https://doi.org/ 10.32376/d895a0ea.89f81da7}.} \vspace*{0.75em}}
\vspace*{0.5em}
\noindent{{\large\emph{Carleton University}, \href{mailto:armondtowns@cunet.carleton.ca}{armondtowns@cunet.carleton.ca}\par}} \marginnote{\href{https://creativecommons.org/licenses/by-nc/4.0/}{\includegraphics[height=0.5cm]{by-nc.png}}}

% \vspace*{0.75em} % second author

% \noindent{\LARGE{<<author 2 name>>}\par}
% \vspace*{0.5em}
% \noindent{{\large\emph{<<author 2 affiliation>>}, \href{mailto:<<author 2 email>>}{<<author 2 email>>}\par}}

% \vspace*{0.75em} % third author

% \noindent{\LARGE{<<author 3 name>>}\par}
% \vspace*{0.5em}
% \noindent{{\large\emph{<<author 3 affiliation>>}, \href{mailto:<<author 3 email>>}{<<author 3 email>>}\par}}

\end{fullwidth}

\vspace*{1em}


\newthought{The January 1970} National Development Project (NDP) on Rhetoric (also
known as Wingspread) in Wisconsin has been celebrated as important to
the contemporary development of media, rhetoric, speech, and
communication in the United States (just called ``communication
studies'' going forward). Organized by prominent scholars like Lloyd
Bitzer and Edwin Black,\footnote{``Critical Moments in NCA's History,'' \emph{Spectra: The Magazine of
  the National}
  \emph{Communication Association} 50, nos. 1-- 2 (2014): 30--36.
} the
conference, as Peter Simonson argues, was deemed central to a
``collective effort to bend rhetorical theory toward the social problems
of the day,'' ``presumably'' guided by ``designated thought leaders
gathered at the conference.''\footnote{Peter Simonson, ``The Short History of Rhetorical Theory,''
  \emph{Philosophy \& Rhetoric}, 53, no. 1 (2020): 83.
} A
year later, the NDP conference proceedings were published as the book,
\emph{The Prospect of Rhetoric: Report of the National Development
Project},\footnote{Lloyd Bitzer and Edwin Black, \emph{The Prospect of Rhetoric: Report
  of the National Development Project} (Englewood Cliffs, NJ:
  Prentice-Hall, 1971).
} a text still used today.

Though to less fanfare, in May 1970, the NDP's National Conference on
Rhetoric (NCR) in Illinois would meet to discuss Wingspread's
presentations.\footnote{I am pulling this largely from Bitzer and Black's \emph{The Prospect
  of Rhetoric}, as well as the final report of the NDP, both of which note the importance of two national development project conferences,
  the Wingspread and the National Conference on Rhetoric.
} Occurring just days
after the Kent State massacre, the Illinois conference invited two
important conference attendees, Arthur Smith (now Molefi Kete Asante)
and Lyndrey Niles.\footnote{Simonson, ``Short History of Rhetorical Theory,'' 84.
} Though they
went unmentioned in Lloyd Bitzer's final NDP report (on both Wingspread
and the NCR), Asante and Niles undoubtedly influenced the call that
emerged from this report for study to ``be undertaken on the nature of
invention in non-western cultures'' and for Black Panther Party speeches
to be deemed rhetorical artifacts.\footnote{Lloyd Bitzer, \emph{Final Report of the National Development Project
  on Rhetoric} (New York: Speech Association of America, 1970), 13.
}
Here, we can see how Black scholars played critical roles in expanding
communication studies' approach to race, racism, colonialism, and
Pan-Africanism. Asante and Niles used their platform in Illinois to
argue
\enlargethispage{2\baselineskip}

\vspace*{2em}

\noindent{\emph{History of Media Studies}, vol. 1, 2021}


 \end{titlepage}
 
\noindent that communication studies was situated in a particular
conjuncture that the field could no longer ignore: worldwide decolonial,
Black radical struggle.\footnote{Reynaldo Anderson, Marnel Niles Goins, and Sheena Howard, ``Liberalism
  and Its Discontents: Black Rhetoric and the Cultural Transformation of
  Rhetorical Studies in the Twentieth Century,'' in \emph{A Century of
  Communication Studies: The Unfinished Conversation}, ed. Pat Gehrke
  and Keith William (New York: Routledge, 2015), 174.
} A few
examples of the Black and decolonial context, in relation to the United
States, will suffice: Black street rebellions raged in U.S. inner cities
of the 1960s and 1970s, often in reaction to racist police brutality;
just off the U.S. coast, in the late 1950s, Fidel Castro led the
revolutionary transformation of Cuba from a U.S. protectorate/colony to
an independent, socialist project; the Vietnam War was in full swing, as
the North Vietnamese fought to break U.S. neocolonial aspirations; in
1957, Kwame Nkrumah led Ghana to independence from British colonial
rule, a rule that the United States backed against Nkrumah; and, the
early 1960s saw the establishment of the Sandinista National Liberation
Front, organizing socialist resistance to U.S. occupation in Nicaragua.

Many Black communication studies scholars viewed their actions as in
line with these larger radical politics, but they were far from dominant
in the field. In the mid- to late twentieth century, as NDP rhetorical
scholars connected with post--World War II communications scholars to
form what became U.S. communication
studies,\footnote{In 1970, the Speech Association of America, which ran the NDP, became
  the Speech Communication Association, signifying the increased
  importance given to more communication studies scholarship. Of course,
  today speech has dropped completely out of the association's name,
  which is now the National Communication Association.
} the new discipline gained
a mixture of conservative and progressive thought. In 1969, for example,
while scholars like Asante, Lucia Hawthorne, and Jack Daniel created the
Speech Association of America's (SAA) Black
Caucus,\footnote{For more on this, see Anderson, Goins, and Howard, ``Liberalism and
  Its Discontents,'' 174; and Jack Daniel, \emph{Changing the Players
  and the Game: A Personal Account of the Speech Communication
  Association Black Caucus Origins} (Annandale, CA: Speech Communication
  Association, 1995), 11.
} other communication
scholars weaponized their World War II training in propaganda against
the new enemy: Cold War
socialism.\footnote{Jefferson Pooley, ``The New History of Mass Communication Research,''
  in \emph{The History of Media and Communication Research: Contested
  Memories}, ed. David W. Park and Jefferson Pooley (New York: Peter
  Lang, 2008), 43--69.
} Some of this
conservative thought would be challenged in the Black Caucus, which
worked to produce a supportive hub for Black communication scholars,
increase opportunities for Black students to enter graduate schools, and
to assist in understanding Black modes of knowing and communicating.
Niles and Asante used the NCR to extend what the Black Caucus was
already arguing: a field organized in highly Western terms could no
longer afford to ignore how Black thought existed beyond Westernism.

Thus, to reduce these scholars to communication studies may prove too
simplistic. At the Illinois conference, we see an intersection between
communication studies and Black studies---the mid-twentieth-century
field that called for the valorization of Black people's epistemologies
and the material transformation of Black lives worldwide. For the
important Black studies scholar Walter Rodney, the call to theorize
Black epistemologies exceeded the Western concept of ``the Negro,''
speaking to \emph{Black people}, who were potentially all the
``non-whites---the hundreds of millions of people whose homelands are in
Asia and Africa, with another few millions in the
Americas.''\footnote{This is, of course, also the time when the word \emph{Negro} would be
  replaced by \emph{Black} in the United States as a new mode of
  self-identification. For more on this, see Walter Rodney, \emph{The
  Groundings with My Brothers} (London: Bogle-L'Ouverture Publications,
  1996), 16.
} We should take
seriously Rodney's words: blackness was not solely a racial
identity---reliant on white, racist classifications of
African-descendent peoples (the Negro)---but a politico-economic
relation for revolutionary purposes (a new proletariat). Here, white
people's racial classifications could tell us little about blackness.
Likewise, Asante and Niles argued that racism was but \emph{one} element
of study that communication scholars could focus on, not the only one.
As Asante suggested at the 1972 Black Communication Conference, in
addition to racism, Black communication could focus on topics as diverse
as the study of nonverbal communication, because ``it is possible that
the kinesic norms of blacks will differ from those of
whites.''\footnote{Molefi Asante, ``Theoretical Research Issues in Black Communication,''
  1972, \url{https://files.eric.ed.gov/fulltext/ED082250.pdf}, 8.
}

To study solely racism may lead to what another important Black studies
scholar, Vincent Harding, called ``the vocation of autopsy,'' which
situated Black life solely in \emph{reaction} to white racism. In his
1974 essay, ``The Vocation of the Black Scholar,'' Harding argued that
autopsy was the domain of white constructs of Black people (the Negro),
amounting to the ``analysis of human history without
celebration.''\footnote{Vincent Harding, ``The Vocation of the Black Scholar and the Struggles
  of the Black Community,'' in \emph{Education and Black Struggle: Notes
  from the Colonized World}, ed. Vincent Harding and Julius Nyerere
  (Cambridge, MA: Harvard Educational Review, 1974), 9.
} Further, part of
the Black scholar's vocation was to break away from reacting to white
people. In contrast to the vocation of the Black scholar, autopsy was a
vocation that, whether progressive or conservative, centered white
definitions of Black people, deeming us as wholly different from all
other people (incommunicable), inhibiting revolutionary, collective
action. Rather than use white definitions, Black studies sought to
create a world in which Black people served as legitimate creators of
knowledge of and about ourselves---which may end up saving
\emph{everyone}.

Asante and Niles were not mere bystanders in this larger Black studies
movement, but full participants. Niles, a graduate of Temple University,
would see the Black student movement firsthand, via the establishment of
Temple's Black studies department ``by the administration in 1969 as a
response to intense demands of the black students on
campus.''\footnote{Mario Small, ``Departmental Conditions and the Emergence of New
  Disciplines: Two Cases in the Legitimation of African-American
  Studies,'' \emph{Theory and Society} 28, no. 5 (1999): 669.
} That same year, Niles
would garner praise for his leadership of the SAA's Summer Conference,
where he led a workshop on Black
rhetoric.\footnote{Lyndrey Niles, ``Report of Workshop A: Black Rhetoric,'' in
  \emph{Proceedings: Speech Association of America Summer Conference V
  Theme: Research and Action}, ed. James Roever (New York: Speech
  Communication Association, 1969), 3--7,
  \url{https://files.eric.ed.gov/fulltext/ED042785.pdf}
} In 1973, Niles brought
together communication studies and Black studies in his completed
dissertation, ``The Status of Speech Communication Programs at
Predominantly Black Four Year Colleges:
1971--1972.''\footnote{Lyndrey Niles, ``The Status of Speech Communication Programs at
  Predominately Black Four Year Colleges: 1971--1972'' (PhD diss.,
  Temple University, 1973).
} And of course,
Asante would go on to become one of the most prolific and important
Black studies scholars of the twentieth and twenty-first
centuries.\textsuperscript{17}

All of this is to say that the presence of Black studies scholars at the
NCR cannot be deemed as a call for the study of solely racism in
communication studies, an area of study that remains important yet
overemphasized today. Instead, their presence was also a call to study
Black modes of knowing and communicating in the world. We cannot read
the 1960s\marginnote{\textsuperscript{17} Some of Asante's many accomplishments include starting the first Black
  studies PhD program at Temple University, as well as his leadership in
  Black studies at the University of California, Los Angeles. For more
  on this, see Ronald Jackson and Sonja Givens, \emph{Black Pioneers in
  Communication Research} (Thousand Oaks, CA: Sage, 2016); and Small,
  ``Departmental Conditions and the Emergence of New Disciplines,'' 670.
}\setcounter{footnote}{17} and 1970s as truly about social death, as per contemporary
theorizations.\footnote{I do not mean social death, in the Orlando Patterson sense, but social
  death as it is being used in some circles of Black studies,
  particularly building off the work of Frank Wilderson, which, of
  course, pulls from Patterson. For more of this Black studies reading
  of social death, particularly in communication studies, see Lisa
  Corrigan, \emph{Black Feelings: Race and Affect in the Long Sixties}
  (Oxford: University of Mississippi Press, 2020); Casey Ryan Kelly,
  ``White Pain,'' \emph{Quarterly Journal of Speech}, 107, no. 2 (2021):
  209--33; Ashely Noel Mack and Bryan McCann, ``\,`Harvey Weinstein,
  Monster': Antiblackness and the Myth of the Monstrous Rapist,''
  \emph{Communication and Critical/Cultural Studies}, 18, no. 2 (2021):
  103--20; and Frank Wilderson, \emph{Red, White \& Black: Cinema and
  the Structure of U.S. Antagonisms} (Durham, NC: Duke University Press,
  2010).
} In other words, we
cannot argue that Black people in the mid-twentieth century were
illustrating that Black death was constitutive of white life, that
blackness always lies in \emph{reaction} to whiteness. Instead, as Adom
Getachew and Karuna Mantena
argue,\footnote{Adom Getachew and Karuna Mantena, ``Anticolonialism and the
  Decolonization of Political Theory,'' \emph{Critical Times} (2021),
  \href{http://doi.org/10.1215/26410478-9355193}{doi.org/10.1215/26410478-9355193}.
} we should take heed of the
1960s and 1970s conjuncture, one in which blackness was deemed
\emph{not} Negroness, in which Black people thought \emph{beyond} the
West. This Black radical, decolonial, and socialist struggle (i.e., new
global solidarities, \emph{not} incommunicabilities) appeared, to some
at least, to be on the verge of toppling the West. It is in this context
that Black studies entered the university (though it existed well before
and far beyond the university). At the NCR, Black scholars expressed
this Black studies tradition: they argued that Black people have
alternative modes of thinking, knowing, \emph{and} communicating, toward
a better world.

The question we must ask today is simple: Do we currently have a
widespread proliferation of Black (and other) modes of knowing in
communication studies? Founded in
1988,\footnote{``Editor's Introduction,'' \emph{Howard Journal of Communications} 1,
  no. 1 (1988): 1--2.
} the \emph{Howard Journal of
Communications} may constitute our closest, most consistent example of a
United States--based communication publication that centers people of
color in ways that exceeds racism as the defining aspect of our lives.
Yet beyond \emph{Howard}, much of communication studies that centers
people of color, written by scholars of \emph{all} races, remain largely
in Harding's vocation of autopsy---the promotion of Black death as
synonymous with white life. In addition, much of these studies
self-classify as ``Black studies.'' But I contend that to reduce Black
studies to \emph{solely} the study of racism is antithetical to Black
studies. Doing so ensures that we cannot leave white people or white
theorizing behind; instead, it recenters whiteness as a motivator of
\emph{all} aspects of Black life. Of course, few Black people live life
obsessed with what white people think about us; those who do may find
themselves in largely white, middle-class spaces---not representative of
the experience of Black people worldwide. One can imagine that if Black
studies were reduced to individual racisms faced by bourgeois Black
academics, not the material conditions of capitalism that organize Black
life, then a host of white scholars could situate themselves as experts
on Black life---as blackness here would be reduced to other white
people's racism. The Black Caucus predicted this, stating that in the
1960s, their members sought to create the Black Rhetoric Institute to
ease their own paranoia ``regarding the possibility of White scholars
taking what we {[}Black scholars{]} had taught them, and using their
access to the `means of production' to dominate the field of Black
Rhetoric.''\footnote{Daniel, \emph{Changing the Players and the Games}, 12.
} But, as the Black
Caucus argued, to study Black ways of knowing did not mean they sought
to solely react to racism, but to accept that racism constituted only
one component of what it meant to be Black. Aside from the \emph{Howard}
journal, do we, as a field, have such ways of thinking about blackness,
beyond current understandings of whiteness and racism? I have found
little.

To be clear, I am not arguing that white communication studies scholars
cannot, or should not, study race, racism, whiteness, or blackness (they
must). I am also not arguing that only white scholars spread the
vocation of autopsy in the field (they are not alone). Instead, my
concern is with why limited reads of race, racism, and whiteness are
deemed \emph{Black studies} in communication? And do such reductive
reads of Black studies signify a communication studies that cherry-picks
recent Black studies as representative of \emph{all} Black
studies?\footnote{To be clear, this is not to say that \emph{all} communication studies
  centers white racism. Scholars like Ronald Jackson, Herman Gray,
  Reynaldo Anderson, Marnel Niles Goins, Sheena Howard, and others have
  done a good job of digging into this history of Black studies from
  multiple angles. And of course, some of the founders of the Black
  Caucus, such as Asante, Dorthy Pennington, Jack Daniel, Lucia
  Hawthorne, Melbourne Cummings, and others, provide us with a
  foundation for Black studies. Yet there is a more recent trend of
  reducing blackness to race that I am speaking to here.
} Rather than just race
studies, we need a Black studies read of race, one not obsessed with
death, or apocalyptic ends, but with alternative epistemological
projects that, if taken seriously, would require the reorganization of
the world as it currently sits.

The historicization of Black studies that I am trying to do, with
admittedly not enough space, may reform what communication history
means, particularly as we remember that Black studies scholars attended
the NCR. What if, instead of taking limited reads of race, racism, and
whiteness as a starting point for U.S. communication studies on people
of color (the autopsy), we took Black studies as the starting point to
rewrite histories of communication studies? To do so requires new
histories, ones that consider communication studies in context. For
example, the post--World War II, mid- to late-twentieth-century creation
of a new academic discipline called ``communication studies'' (via the
merger of rhetoric, communication, media studies, and more) is
inseparable from the Black and decolonial struggle at the time,
particularly given that some of the Black Caucus's peers weaponized
their capitalistic positions and actions against
socialism.\footnote{Importantly, this discussion of socialism-as-enemy in communication
  studies lies largely with the social scientific, World War II
  communications scholars, not as much with the NDP rhetoric scholars,
  who, at the time, might be considered largely ``liberal.'' However,
  what I am trying to speak to is how communications scholars would
  slowly join with the NDP scholars to create a new field called
  communication studies, toward the mid- to late-twentieth century. It
  is this merger that, I argue, ensures that today's U.S. field was
  built off Cold War concerns. For more on the earlier, social
  scientific communications prior to the merger, see Christopher
  Simpson, ``Universities, Empire, and the Production of Knowledge: An
  Introduction,'' in \emph{Universities and Empire: Money and Politics
  in the Social Sciences during the Cold War}, ed. Christopher Simpson
  (New York: The New Press, 1999), xii.
} To understand
mid-twentieth-century Black studies, which often championed socialism in
a fight against capitalism and racism, means to consider Black studies
as a threat \emph{both} to the United States' capitalistic order
\emph{and} to communication studies. To read the history of Black
studies alongside communication history may point us to uncomfortable
understandings: that at least some mid-twentieth-century communication
studies developed in fear of Black and decolonial revolution. Thus, to
accept Black studies as the new starting point may undo communication
studies (and communication-as-autopsy) and make it anew. It would be
truly, as Jefferson Pooley argues, a time for new histories of the
field.\footnote{ Pooley, ``New History of Mass Communication Research.''}







\section{Bibliography}\label{bibliography}

\begin{hangparas}{.25in}{1} 



Anderson, Reynaldo, Marnel Niles Goins, and Sheena Howard. ``Liberalism
and Its Discontents: Black Rhetoric and the Cultural Transformation of
Rhetorical Studies in the Twentieth Century.'' In \emph{A Century of
Communication Studies: The Unfinished Conversation}, edited by Pat
Gehrke and Keith William, 166--86. New York: Routledge, 2015.

Asante, Molefi. ``Theoretical Research Issues in Black Communication.''
1972. 1--13. \url{https://files.eric.ed.gov/fulltext/ED082250.pdf}.

Bitzer, Lloyd. \emph{Final Report of the National Development Project on
Rhetoric}. New York: Speech Association of America, 1970.

Bitzer, Lloyd, and Edwin Black. \emph{The Prospect of Rhetoric: Report
of the National Development Project}. Englewood Cliffs, NJ:
Prentice-Hall, 1971.

Corrigan, Lisa. \emph{Black Feelings: Race and Affect in the Long
Sixties}. Oxford: University of Mississippi Press, 2020.

``Critical Moments in NCA's History.'' \emph{Spectra: The Magazine of
the National Communication Association} 50, nos. 1--2 (2014): 30--36,
\url{https://www.natcom.org/sites/default/files/publications/NCA_Spectra_2014_March-May.pdf}.

Daniel, Jack. \emph{Changing the Players and the Game: A Personal
Account of the Speech Communication Association Black Caucus Origins.}
Annandale, CA: Speech Communication Association, 1995.

``Editor's Introduction.'' \emph{Howard Journal of Communications} 1,
no. 1 (1988): 1--2.

Getachew, Adom, and Karuna Mantena. ``Anticolonialism and the
Decolonization of Political Theory.'' \emph{Critical Times} (2021): n.p.
\url{https://doi.org/10.1215/26410478-9355193}.

Harding, Vincent. ``The Vocation of the Black Scholar and the Struggles
of the Black Community.'' In \emph{Education and Black Struggle: Notes
from the Colonized World}, edited by Vincent Harding and Julius Nyerere,
3--29. Cambridge, MA: Harvard Educational Review, 1974.

Jackson, Ronald, and Sonja Givens. \emph{Black Pioneers in Communication
Research}. Thousand Oaks, CA: Sage, 2016.

Kelly, Casey Ryan. ``White Pain.'' \emph{Quarterly Journal of Speech}
107, no. 2 (2021): 209--33.
\url{https://doi.org/10.1080/00335630.2021.1903537}.

Mack, Ashely Noel, and Bryan McCann. ``\,`Harvey Weinstein, Monster':
Antiblackness and the Myth of the Monstrous Rapist.''
\emph{Communication and Critical/Cultural Studies}, 18, no. 2 (2021):
103--20. \url{https://doi.org/10.1080/14791420.2020.1854802}.

Niles, Lyndrey. ``The Status of Speech Communication Programs at
Predominately Black Four Year Colleges: 1971--1972.'' PhD diss., Temple
University, 1973.

---------. ``Report of Workshop A: Black Rhetoric.'' In
\emph{Proceedings: Speech Association of America Summer Conference V
Theme: Research and Action}, edited by James Roever, 3--7. New York:
Speech Communication Association, 1969.

Pooley, Jefferson. ``The New History of Mass Communication Research.''
In \emph{The History of Media and Communication Research: Contested
Memories}, edited by David W. Park and Jefferson Pooley, 43--69. New
York: Peter Lang, 2008.

Rodney, Walter. \emph{The Groundings with My Brothers}. London:
Bogle-L'Ouverture Publications, 1996.

Simonson, Peter. ``The Short History of Rhetorical Theory.''
\emph{Philosophy \& Rhetoric} 53, no. 1 (2020): 75--88.
\url{https://doi.org/10.5325/philrhet.53.1.0075}.

Simpson, Christopher. ``Universities, Empire, and the Production of
Knowledge: An Introduction.'' In \emph{Universities and Empire: Money
and Politics in the Social Sciences during the Cold War}, edited by
Christopher Simpson, xi--xxx. New York: The New Press, 1999.

Small, Mario. ``Departmental Conditions and the Emergence of New
Disciplines: Two Cases in the Legitimation of African-American
Studies.'' \emph{Theory and Society} 28, no. 5 (1999): 659--707.
\url{https://www.jstor.org/stable/3108589}.

Wilderson, Frank. \emph{Red, White \& Black: Cinema and the Structure of
U.S. Antagonisms}. Durham, NC: Duke University Press, 2010.



\end{hangparas}


\end{document}