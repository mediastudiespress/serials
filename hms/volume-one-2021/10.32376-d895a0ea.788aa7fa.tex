% see the original template for more detail about bibliography, tables, etc: https://www.overleaf.com/latex/templates/handout-design-inspired-by-edward-tufte/dtsbhhkvghzz

\documentclass{tufte-handout}

%\geometry{showframe}% for debugging purposes -- displays the margins

\usepackage{amsmath}

\usepackage{hyperref}

\usepackage{fancyhdr}

\usepackage{hanging}

\hypersetup{colorlinks=true,allcolors=[RGB]{97,15,11}}

\fancyfoot[L]{\emph{History of Media Studies}, vol. 1, 2021}


% Set up the images/graphics package
\usepackage{graphicx}
\setkeys{Gin}{width=\linewidth,totalheight=\textheight,keepaspectratio}
\graphicspath{{graphics/}}

\title[Challenges of Doing Historical Research in Communication Studies]{Challenges of Doing Historical Research in Communication Studies: On the Necessity to Write a Methodologically Informed History of the Methods of Communication Studies} % longtitle shouldn't be necessary

% The following package makes prettier tables.  We're all about the bling!
\usepackage{booktabs}

% The units package provides nice, non-stacked fractions and better spacing
% for units.
\usepackage{units}

% The fancyvrb package lets us customize the formatting of verbatim
% environments.  We use a slightly smaller font.
\usepackage{fancyvrb}
\fvset{fontsize=\normalsize}

% Small sections of multiple columns
\usepackage{multicol}

% Provides paragraphs of dummy text
\usepackage{lipsum}

% These commands are used to pretty-print LaTeX commands
\newcommand{\doccmd}[1]{\texttt{\textbackslash#1}}% command name -- adds backslash automatically
\newcommand{\docopt}[1]{\ensuremath{\langle}\textrm{\textit{#1}}\ensuremath{\rangle}}% optional command argument
\newcommand{\docarg}[1]{\textrm{\textit{#1}}}% (required) command argument
\newenvironment{docspec}{\begin{quote}\noindent}{\end{quote}}% command specification environment
\newcommand{\docenv}[1]{\textsf{#1}}% environment name
\newcommand{\docpkg}[1]{\texttt{#1}}% package name
\newcommand{\doccls}[1]{\texttt{#1}}% document class name
\newcommand{\docclsopt}[1]{\texttt{#1}}% document class option name


\begin{document}

\begin{titlepage}

\begin{fullwidth}
\noindent\LARGE\emph{Launch essay
} \hspace{85mm}\includegraphics[height=1cm]{logo3.png}\\
\noindent\hrulefill\\
\vspace*{1em}
\noindent{\Huge{Challenges of Doing Historical Research in Communication Studies: On the Necessity to Write a Methodologically Informed\\\noindent History of the Methods of Communication Studies\par}}

\vspace*{1.5em}

\noindent\LARGE{Stefanie Averbeck-Lietz \par}\marginnote{\emph{Stefanie Averbeck-Lietz, ``Challenges of Doing Historical Research in Communication Studies: On the Necessity to Write a Methodologically Informed History of the Methods of Communication Studies,'' \emph{History of Media Studies} 1 (2021), \href{https://doi.org/10.32376/d895a0ea.788aa7fa}{https://doi.org/ 10.32376/d895a0ea.788aa7fa}.} \vspace*{0.75em}}
\vspace*{0.5em}
\noindent{{\large\emph{Universität Bremen}, \href{mailto:averbeck.lietz@uni-bremen.de}{averbeck.lietz@uni-bremen.de}\par}} \marginnote{\href{https://creativecommons.org/licenses/by-nc/4.0/}{\includegraphics[height=0.5cm]{by-nc.png}}}

% \vspace*{0.75em} % second author

% \noindent{\LARGE{<<author 2 name>>}\par}
% \vspace*{0.5em}
% \noindent{{\large\emph{<<author 2 affiliation>>}, \href{mailto:<<author 2 email>>}{<<author 2 email>>}\par}}

% \vspace*{0.75em} % third author

% \noindent{\LARGE{<<author 3 name>>}\par}
% \vspace*{0.5em}
% \noindent{{\large\emph{<<author 3 affiliation>>}, \href{mailto:<<author 3 email>>}{<<author 3 email>>}\par}}

\end{fullwidth}

\vspace*{1em}


\newthought{In the following} essay, I want to outline the challenges of \emph{doing}
historical research in communication studies. A pair of challenges, in
particular, affect media and communication studies and their overall
history:

\begin{enumerate}
\item
  The history of methods is not yet written in the field of
  communication research, beyond studies of ``great figures'' and their
  outstanding contributions with a focus on paradigm history, as in the
  case of Paul F. Lazarsfeld.\footnote{Wolfgang R. Langenbucher, ed.,
    \emph{Paul F. Lazarsfeld} (München: Ölschläger, 1990); and Jefferson
    Pooley, ``Lazarsfeld, Paul F.,'' in \emph{The International
    Encyclopedia in Communication Theory and Philosophy}, ed. Klaus
    Bruhn Jensen and Robert T. Craig (New York: Wiley-Blackwell, 2016).}
  In addition to Lazarsfeld, in the German-speaking context at least, a
  great deal of attention is given to Max Weber and his early
  twentieth-century plan for a \emph{presse-enquête}, a proposed
  mixed-method scheme of content analysis, document analysis,
  interviews, and participatory observation. Weber never managed to
  realize this plan, but interestingly enough, anticipated Lazarsfeld's
  model of teamwork conducted by research groups.\footnote{Michael Meyen
    and Maria Löblich, \emph{Klassiker der Kommunikationswissenschaft}
    (Konstanz: UVK 2006), 145--61; Siegfried Weischenberg, \emph{Max
    Weber und die Entzauberung der Medienwelt: Theorien und Querelen --
    eine andere Fachgeschichte} (Wiesbaden: Springer} 

\end{enumerate}


\enlargethispage{2\baselineskip}

\vspace*{2em}

\noindent{\emph{History of Media Studies}, vol. 1, 2021}




 \end{titlepage}

\begin{enumerate}

\setcounter{enumi}{1}

\item
  The\marginnote{VS 2012), 78--164; Michael
    Meyen and Stefanie Averbeck-Lietz, ``Nicht standardisierte Methoden
    in der Kommunikationswissenschaft: Eine Entwicklungsgeschichte zur
    Einführung,`` in \emph{Handbuch nicht standardisierte Methoden in
    der Kommunikationswissenschaft}, ed. Stefanie Averbeck-Lietz and
    Michael Meyen (Wiesbaden: Springer VS, 2016).} Weber's idea had
  already inspired empirical research by students and doctoral
  candidates in sociology and newspaper studies by the early Weimar
  period.\footnote{Stefanie Averbeck, \emph{Kommunikation als Prozess:
    Soziologische Perspektiven in der Zeitungswissenschaft 1927--1933}
    (Münster: LIT, 1999).} More recently, Herta Herzog's methodological
  innovations in early uses-and-gratifications research has attracted
  interest.\footnote{Elisabeth Klaus and Josef Seethaler, eds.,
    \emph{What Do We Really Know about Herta Herzog} (Frankfurt: Peter
    Lang, 2016).} lack of a methodologically reflective history
  is---paradoxically---still more glaring for the subfield of
  communication history, with its double inheritance from communication
  studies and from disciplinary history.
\end{enumerate}

Fully aware that the leading (trans-)national associations of
communication studies include divisions devoted to communication history
(like the ICA, IAMCR, ECREA, and the DGPuK in Germany), the focus of
this article is exclusively on Germany, with particular attention to the
subfield of communication history.

After a promising start in 1916, German newspaper studies
\\\noindent {[}\emph{Zeitungswissenschaft}{]} cooperated with the Nazis after
Hitler's 1933 seizure of power. For the regime, newspaper studies were
helping to legitimate the dictatorship and educate
journalists.\footnote{Arnulf Kutsch, ``Die Entstehung des Deutschen
  Zeitungswissenschaftlichen Verbandes,'' \emph{Jahrbuch für
  Kommunikationsgeschichte} 12 (2010); and Jochen Jedraszczyk,
  ``Politische Überformung: Hans Amandus Münster und die
  Instrumentalisierung der Leipziger Zeitungswissenschaft im
  Nationalsozialismus,'' in \emph{Die Entdeckung der
  Kommunikationswissenschaft: 100 Jahre Kommunikationswissenschaft in
  Leipzig}, ed. Erik Koenen (Köln: Herbert von Halem, 2016).} After
1945, German newspaper studies scholars found themselves discredited for
their collaboration. As a break with the past, the field adopted a new
label, \emph{Publizistikwissenschaft}, and sought to emulate the US's
quantitatively oriented communication research.

Maria Löblich, in her analysis of the German discipline's reconstruction
during the 1950s and 1960s, traced the ascendant social science
paradigm, with its strong focus on positivism and critical rationalism
and a special interest in standardized content analysis.\footnote{Maria
  Löblich, \emph{Die empirisch-sozialwissenschaftliche Wende in der
  Publizstik- und Kommunikationswissenschaft} (Köln: von Halem, 2010).}
Petra Klein's foundational work on Henk Prakke and
\emph{Publizistikwissenschaft} at the University of Münster points in
the same direction: A new social science orientation, in conscious
contrast to the historical-philosophical orientation of the 1920s,
prevailed.\footnote{Petra Klein, \emph{Henk Prakke und die funktionale
  Publizistik: Über die Entgrenzung der Publizistik- zur
  Kommunikationswissenschaft} (Münster: Lit, 2006).}

It should be highlighted that, from an institutional viewpoint,
communication studies {[}\emph{Kommunikationswissenschaft}{]} and media
studies {[}\emph{Medienwissenschaft}{]} in Germany are two different
disciplines. Media studies developed during the 1970s out of literature,
language, and film studies, with its own professional association,
\emph{Gesellschaft für Medienwissenschaft}. Furthermore, degree programs
in communication and in media studies are typically separated from each
other (though the Universities of Leipzig and Bremen are exceptions).

Löblich's groundbreaking work on the history of German communication
studies was broadly adopted internationally to understand the postwar
shift toward positivism in German communication studies, but in fact the
historiography of the field of communication studies as well as of its
methods remains a \emph{marginal} topic within the discipline. Academic
positions in the German field of communication studies oriented to the
discipline's history or to communication history have declined in
numbers over the last decade.\footnote{Michael Meyen, ``Die historische
  Perspektive in der Kommunikationswissenschaft: Spuren einer
  Verlustgeschichte,'' in \emph{Geschichte, Öffentlichkeit,
  Kommunikation: Festschrift für Bernd Sösemann}, ed. Patrick Merziger
  et al. (Stuttgart: Franz Steiner, 2010).} Currently, university
departments do not undertake many efforts to establish historical
analysis at the heart of the discipline, and most neglect the history of
communication studies. Indeed, the legitimacy of \emph{doing} historical
communication research and historicizing the field of communication
studies is under extreme pressure in German communication studies. There
are only a few professors still working in this field and they do it
often in addition to other subjects. There are few resources and
restricted career opportunities compared to other subfields of
communication studies. In response to such conditions, the German
\emph{Yearbook for Communication Research} {[}\emph{Jahrbuch für
Kommunikationsgeschichte}{]} is slated to publish a debate on the
legitimization of the historical field within communication studies.
Contributors include well-known authors in the field of the history of
communication and of communication studies, such as Maria Löblich, Erik
Koenen, Michael Meyen, Simon Sax, Josef Seethaler, Jürgen Wilke, and
others. They discuss the lack of attention to history in German
communication studies in a broad manner---the historiography of media
and communication as research objects---but also the history of the
discipline. This debate is very relevant, but it also illustrates the
problem once more: The small milieu of historically-oriented
communication researchers in German, Swiss, and Austrian communication
studies is---in that debate yet again---more or less isolated from the
broader field of communication studies in German-speaking
countries.\footnote{I do not refer to the \emph{discipline} of history
  and its subfield of media history here.}

This isolation of the historical field is mirrored when it comes to
methods. The remainder of this essay is not on methods in the subfield
of communication history as such, but instead on the theoretical and
methodologically relevant problem of how to produce a research project
in applied historical communication research that has limited resources
available and is restricted in terms of time and people-power, but also
restricted regarding knowledge on how to transfer methods of
communication research to historical research topics and objects. At
least in the Germanophone context, these problems are so fundamental
that the more ambitious project of thinking about methodological
meta-reflection seems far off indeed. Knowing more about the history of
methods and their underlying methodological reflections---this is my
argument---might help us to advance methodologies in general and to
close the gap between historical and systematic lines of communication
research. I will not fulfill that task in this short article; my aim
here, instead, is to highlight the problem.

In an ongoing project on the communication history of the League of
Nations (financed by the German Research Foundation or DFG), including a
team of three researchers (project leader, postdoc, and doctoral
student) and supported by two student assistants (from the BA program in
communication studies at the University of Bremen), we sketch a
collective biography, archive analysis, and a historical content
analysis.\footnote{Arne Gellrich, Erik Koenen, and Stefanie
  Averbeck-Lietz, ``The Epistemic Project of Open Diplomacy and the
  League of Nations: Co-evolution between Diplomacy, PR and
  Journalism,'' \emph{Corporate Communications} 24, no. 4 (2020). The
  student assistants are Gina Franke and Miriam Sachs---many thanks for
  the important work they do.} Only the content analysis is discussed
here.

Content analysis is one of the classic methods in communication studies,
not only since the days of Harold Lasswell and Bernard Berelson in the
quantitative realm, but also with Siegfried Kracauer as a precursor of
qualitative content analysis. A review of standard German literature in
the field shows that there are only a few overviews on the historical
paths and development of the methods available, and that they mostly
come with a narrow German- and also English-language focus.\footnote{Heinz
  Bonfadelli, \emph{Medieninhaltsforschung} (Konstanz: UVK 2002),
  82--86; Werner Früh, \emph{Inhaltsanalyse}, 9th ed. (Konstanz: UVK
  2017), 11--13; Stefanie Averbeck-Lietz, ``Qualitative Inhaltsanalyse
  und Diskursanalyse: Überlegungen zu Gemeinsamkeiten, Unterschieden und
  Grenzen,'' in \emph{Diskursanalyse für die Kommunikationswissenschaft:
  Theorie, Vorgehen, Erweiterungen}, eds. Thomas Wiedemann and Christine
  Lohmeier (Wiesbaden: Springer VS, 2019), 90--91.} There are, moreover,
only a few methodological hints as to how one can apply this method to a
historical research object, typically the printing press---especially
related to the question of how to combine non-standardized and
standardized methodologies in \emph{historical} content
analysis.\footnote{Jürgen Wilke, ``Quantitative Methoden in der
  Kommunikationsgeschichte,'' in \emph{Wege zur
  Kommunikationsgeschichte}, eds. Manfred Bobrowski and Wolfang R.
  Langenbucher (München: Olschläger 1987); Wilke, ``Quantitative
  Verfahren in der Kommunikationsgeschichte,'' in
  \emph{Kommunikationsgeschichte: Positionen und Werkzeuge}: \emph{Ein
  diskursives Hand- und Lehrbuch}, eds. Klaus Arnold, Markus Behmer, and
  Bernd Semrad (Münster: Lit, 2008); Ute Nawratil and Pilomen
  Schönhagen, ``Die Qualitative Inhaltsanalyse. Rekonstruktion der
  Kommunikationswirklichkeit,'' in \emph{Qualitative Methoden in der
  Kommunikationswissenschaft: Ein Studienbuch}, ed. Hans Wagner
  (Baden-Baden: Nomos, 2009); Stefanie Averbeck-Lietz, \emph{Qualitative
  Inhaltsanalyse}, 2019; Rudolf Stöber, ``Historische Methoden in der
  Kommunikationswissenschaft: Die Standards einer Triangulation,'' in
  \emph{Handbuch nicht standardisierte Methoden in der
  Kommunikationswissenschaft}, eds. Stefanie Averbeck-Lietz and Michael
  Meyen (Wiesbaden: Springer VS, 2016), 314--15.}

In our project, we learned that not only content, but also
\emph{context} (as Kracauer told us) is relevant.\footnote{Siegfried
  Kracauer, ``The Challenge of Qualitative Content Analysis,''
  \emph{Public Opinion Quarterly} 16, no. 4 (1952).} Furthermore, the
practice of coding by coders without sufficient background knowledge is
challenging: In historical research, student assistants and even other
coders often lack knowledge of the historical context, not least with
regard to media, communication, and journalism history, which was only
rarely taught in German BA and MA programs in communication studies over
the last several decades.

How was the League of Nations covered by the press? We wanted to learn
about this by historical newspaper analysis, operationalized by the
standard method of content analysis, conducted by a team of people with
diverse knowledge in German and global history, in media and
communication history, as well as in methods and their history. So, we
faced the daily challenges of research projects when it comes to
historical research in communication studies of any kind: a lack of
contextual knowledge and a lack of clear-cut methodology.

What we developed is a semi-standardized procedure to analyze the
coverage of the League of Nations in the historical press of the Weimar
Republic in Germany. Why semi-standardized? We are still at the stage of
learning inductively from our material, the former German daily
\emph{Vorwärts}, the official newspaper of the Social Democratic Party,
founded in 1878. Fine-tuning our categories and sensitizing our coding
practices has taken longer than we thought. Doing this kind of research
is not free from difficulties and, again, context-related
decision-making. At first, our aim was to conduct a comparative content
analysis on the coverage of the League of Nations in different national
contexts: Switzerland (for the news factor of locality, as the League
was established in Geneva), France and Great Britain (the superpowers in
the League), and Germany (as the newcomer in 1926 and an early departee
in 1933 with the rise of the Nazi state). With only three people and
three student assistants (all also involved in the other parts of the
project, the collective biography of journalists mostly done by Erik
Koenen and Arne Gellrich, and the document analysis from the Geneva
archives of the League's Information section mostly done by me), it was
not possible. We failed to conduct an extensive content analysis of
several European newspapers. But the time restrictions were only one of
the challenges.

In official German digital newspaper databases, most of the historical
newspapers are not sufficiently prepared for standardized content
analysis. The database of scanned newspapers is small and often---due to
the variant-rich traditional typefaces like \emph{Fraktur}---not
well-suited to Optical Character Recognition (OCR) software.\footnote{Erik
  Koenen, ``Digitale Perspektiven in der Kommunikations- und
  Mediengeschichte: Erkenntnispotentiale und Forschungsszenarien für die
  historische Presseforschung,'' \emph{Publizistik} 63, no. 4 (2018);
  and Lisa Bolz, ``Nachrichtenpräsentation im 19. Jahrhundert: Der
  Wandel von Nachrichtenproduktion und Berichterstattung durch
  technische Innovationen,'' \emph{Medien \& Zeit} 35, no. 1 (2020).} In
fact, the newspaper under analysis, the \emph{Vorwärts,} is available in
an OCR format, but only page-wise and not by article (what we would have
needed). So we first identified---this step was OCR-based---the relevant
articles by searching for the keyword \emph{Völkerbund} (League of
Nations), but then had to extract on our own the PDFs \emph{per article}
without being able to rely on automated forms of content analysis. This
process of selection and curation of our material under analysis
represented a huge workload, even before any codebook could be
developed. Fortunately, Koenen, an expert in historical text- and
data-mining, was part of the team---so we knew what was \emph{not}
possible to do. Our process in this project was and is analog content
analysis.\footnote{Erik Koenen, Falko Krause, and Simon Sax, ``Die
  Berliner Volkszeitung digital erforschen: Digitales Kuratieren,
  Metadaten, Text Mining: Praktiken und Potentiale historischer
  Presseforschung in digitalen Kontexten,'' in \emph{Digitale
  Kommunikation und Kommunikationsgeschichte: Perspektiven, Potentiale,
  Problemfelder,} ed. Christian Schwarzenegger et al. (in press, 2022).}

Our codebook started with methodological reflections and the adoption of
standard categories from prior work by Meyen and Schweiger as well as by
Rössler, with particular attention to the deductive categories
\emph{authors, sources, topics, actors named per article,} and
\emph{places and organizations named per article}.\footnote{Michael
  Meyen and Wolfgang Schweiger, ``Sattsam bekannte Uniformität? Eine
  Inhaltsanalyse der DDR-Tageszeitungen Neues Deutschland und Junge Welt
  (1960--1989),'' \emph{Medien \& Kommunikationswissenschaft} 56, no. 1
  (2017); and Patrick Rössler, \emph{Inhaltsanalyse}, 3rd ed. (Konstanz:
  UVK (UTB), 2017).} Nevertheless, our codebook of now seventeen pages
grew inductively while we did the coding. During a process that lasted
more than a year, we fine-tuned the codes and also extended them via
close reading. To give an example: The coverage of 1920 (in which World
War I, peace, and the general aims of the League were reported) was
significantly different from the coverage 1925 (in which the preparation
of the German entrance to the League is the dominant topic). Compared to
1920, the 1925 coverage in \emph{Vorwärts} confronted us with the task
of coding new individual actors, new corporate actors, and other states
as actors. Jürgen Wilke highlighted the same problem, the volatility of
content items, and opted for ``pragmatic solutions''\footnote{Wilke,
  \emph{Quantitative Verfahren}, 335.}---to standardize coding without
losing the hermeneutic relation to the journalistic article under
analysis. At first we were frustrated by this process of constantly
adapting the codebook, but then we learned its deeper meaning: We saw
how a now-historical journal was dynamically reporting, driven by events
but also by its own partisan views on German politics, not least in
relation to France, and touched by normative views on multilateralism,
peace-building, and humanitarian problems in interwar times.
Interestingly enough, the \emph{Vorwärts} was pro-League, but
nevertheless often denounced the League, ironically, for having been
more of a club to talk than to act---maybe underestimating the ``soft
power'' of communication which, nowadays, is counted as one of the
League's great achievements.\footnote{Joseph S. Nye, \emph{Soft Power:
  The Means to Success in World Politics} (New York: Public Affairs,
  2004).}

We are sure that, when we change our focus, object, and material---when
we switch to Swiss or French papers, for example---we can use our
codebook developed for analyzing a German newspaper. But the codebook
will probably have to be adapted again, to new individual and corporate
actors, additional reference nations, national goals, and normative
views that come to the fore in the reporting.

What can we learn from this example and from our research experience for
the history of communication and media studies? There is, first, not
much reflection on the methodology of historical communication research.
A second, related issue is the dearth of published work on the history
of the field's prevailing methodological frameworks for historical
research questions and historical media texts.

Now we are especially interested in learning how other research teams
handle such problems. We are planning a workshop on the methodology of
communication history in spring 2022, together with the DGPuK History
Section and the Institute for Newspaper Research at Dortmund, headed by
the press historian Astrid Blome. The history of methodology---and
therefore the history of communication studies---is included as a main
topic. The workshop is intended as a step toward institutionalizing
reflection on historical methodology---to restore such reflection
(again) to the milieu of scientific organizations.\footnote{For further
  information, see ``Erkenntnisperspektiven und Methoden der
  Kommunikations- und Mediengeschichte: Anwendungsfelder,
  Herausforderungen, Innovationen, Praxis,'' Dortmund Überrascht
  (website), accessed September 13, 2021.} It is sobering to realize
that, more than 35 years ago, at a DGPuK conference on communication
history in Vienna, this was a crucial demand.\footnote{Manfred Bobrowsky
  and Wolfgang R. Langenbucher, eds., \emph{Wege zur
  Kommunikationsgeschichte} (München: Ölschläger, 1987).}

It is not accidental that I name a number of colleagues and scholars,
and often name the same ones whether I am referring to communication
studies history or to methods in historical analysis. This shows again
the narrowness of the German-speaking milieu of historical communication
research, at least in comparison to the field in the US and in broader
European contexts. The German niche needs very much to integrate into
transnational networks like the History of Media Studies working group
of the Consortium for the History of Science, Technology, and Medicine
(CHSTM). This will, hopefully, strengthen the German milieu, not least
with the aim to advance international visibility and career paths for
young German-speaking researchers. As a result, writing and talking in
English is a challenge, but a must to support the whole subfield---in
Germany and beyond.







\section{Bibliography}\label{bibliography}

\begin{hangparas}{.25in}{1} 



Averbeck, Stefanie. \emph{Kommunikation als Prozess: Soziologische
Perspektiven in der Zeitungswissenschaft 1927--1933}. Münster: LIT,
1999.

Averbeck-Lietz, Stefanie. ``Qualitative Inhaltsanalyse und
Diskursanalyse: Überlegungen zu Gemeinsamkeiten, Unterschieden und
Grenzen.'' In \emph{Diskursanalyse für die Kommunikationswissenschaft:
Theorie, Vorgehen, Erweiterungen,} edited by Thomas Wiedemann and
Christine Lohmeier, 83--115. Wiesbaden: Springer VS, 2019.

Bobrowski, Manfred, and Wolfgang R. Langebucher, eds. \emph{Wege zur
Kommunikationsgeschichte}. München: Ölschläger, 1987.

Bonfadelli, Heinz. \emph{Medieninhaltsforschung}. Konstanz: UVK (UTB),
2002.

Bolz, Lisa. ``Nachrichtenpräsentation im 19. Jahrhundert: Der Wandel von
Nachrichtenproduktion und Berichterstattung durch technische
Innovationen.'' \emph{Medien \& Zeit} 35, no. 1 (2020): 4--13.
\href{https://medienundzeit.at/lisa-bolz-nachrichtenpräsentation-im-19-jahrhundert/}{https://medienundzeit.at/lisa-bolz-nachrichtenpräsentation-im-19-jahrhundert}.

``Erkenntnisperspektiven und Methoden der Kommunikations- und
Mediengeschichte: Anwendungsfelder, Herausforderungen, Innovationen,
Praxis.'' Dortmund Überrascht. Accessed September 13, 2021.
\url{https://www.dortmund.de/de/leben_in_dortmund/nachrichtenportal/alle_nachrichten/nachricht.jsp?nid=674159}.

Früh, Werner. \emph{Inhaltsanalyse}. 9th ed. Konstanz: UVK (UTB), 2017.

Gellrich, Arne, Erik Koenen, and Stefanie Averbeck-Lietz. ``The
Epistemic Project of Open Diplomacy and the League of Nations:
Co-evolution between Diplomacy, PR and Journalism.'' \emph{Corporate
Communications} 24, no. 4 (2020): 607--21.
\url{https://doi.org/10.1108/CCIJ-11-2019-0129}.

Jedraszczyk, Jochen. ``Politische Überformung: Hans Amandus Münster und
die Instrumentalisierung der Leipziger Zeitungswissenschaft im
Nationalsozialismus.'' In \emph{Die Entdeckung der
Kommunikationswissenschaft: 100 Jahre Kommunikationswissenschaft in
Leipzig}, edited by Erik Koenen, 185--214. Köln: Herbert von Halem,
2016.

Bellingradt, Daniel, Astrid Blome, Patrick Merziger, and Rudolf Stöber.
\emph{Jahrbuch für Kommunikationsgeschichte}. München: Franz Steiner,
2018--2021.

Klein, Petra. \emph{Henk Prakke und die funktionale Publizistik: Über
die Entgrenzung der Publizistik- zur Kommunikationswissenschaft}.
Münster: Lit, 2006.

Klaus, Elisabeth, and Josef Seethaler, eds. \emph{What Do We Really Know
About Herta Herzog?} Frankfurt: Peter Lang, 2016.

Kracauer, Siegfried. ``The Challenge of Qualitative Content Analysis.''
\emph{Public Opinion Quarterly} 16, no. 4 (1952): 631--42.
\url{https://psycnet.apa.org/doi/10.1086/266427}.

Koenen, Erik. ``Digitale Perspektiven in der Kommunikations- und
Mediengeschichte. Erkenntnispotentiale und Forschungsszenarien für die
historische Presseforschung.'' \emph{Publizistik} 63, no. 4 (2018):
535--56. \url{https://doi.org/10.1007/s11616-018-0459-4}.

Koenen, Erik, Falko Krause, and Simon Sax. ``Die Berliner Volkszeitung
digital erforschen: Digitales Kuratieren, Metadaten, Text Mining:
Praktiken und Potentiale historischer Presseforschung in digitalen
Kontexten.'' In \emph{Digitale Kommunikation und
Kommunikationsgeschichte: Perspektiven, Potentiale, Problemfelde,}
edited by Christian Schwarzenegger, Erik Koenen, Christian Pentzold,
Thomas Birkner, and Christian Katzenbach (in press, 2022).

Kutsch, Arnulf. ``Die Entstehung des Deutschen
Zeitungswissenschaftlichen Verbandes.'' \emph{Jahrbuch für
Kommunikationsgeschichte} 12 (2010): 120--44.
\url{http://www.jstor.org/stable/20852787}.

Langenbucher, Wolfgang R., ed. \emph{Paul F. Lazarsfeld}. München:
Ölschläger, 1987.

Löblich, Maria. \emph{Die empirisch-sozialwissenschaftliche Wende in der
Publizistik- und Kommunikationswissenschaft}. Köln: von Halem, 2010.

Meyen, Michael. ``Die historische Perspektive in der
Kommunikationswissenschaft: Spuren einer Verlustgeschichte.'' In
\emph{Geschichte, Öffentlichkeit, Kommunikation: Festschrift für Bernd
Sösemann zum 60. Geburtstag}, edited by Patrick Merziger, Rudolf Stöber,
Esther-Beate Körber et al., 271--80. Stuttgart: Franz Steiner, 2010.

Meyen, Meyen, and Wolfgang Schweiger. ``Sattsam bekannte Uniformität?
Eine Inhaltsanalyse der DDR-Tageszeitungen Neues Deutschland und Junge
Welt (1960--1989).'' \emph{Medien \& Kommunikationswissenschaft} 56, no.
1 (2008): 82--100. \url{https://doi.org/10.5771/1615-634x-2008-1-82}.

Meyen, Michael, and Stefanie Averbeck-Lietz. ``Nicht standardisierte
Methoden in der Kommunikationswissenschaft: Eine Entwicklungsgeschichte
zur Einführung.'' In \emph{Handbuch nicht standardisierte Methoden in
der Kommunikationswissenschaft,} edited by Stefanie Averbeck-Lietz and
Michael Meyen, 1--15. Wiesbaden: Springer VS, 2016.

Nawratil, Ute, and Pilomen Schönhagen. ``Die Qualitative Inhaltsanalyse:
Rekonstruktion der Kommunikationswirklichkeit.'' In \emph{Qualitative
Methoden in der Kommunikationswissenschaft: Ein Studienbuch}, edited by
Hans Wagner, 333--46. Baden-Baden: Nomos, 2009.

Pooley, Jefferson. ``Lazarsfeld, Paul F.'' In \emph{The International
Encyclopedia in Communication Theory and Philosophy}, edited by Klaus
Bruhn Jensen and Robert T. Craig, 1--7. New York: Wiley-Blackwell, 2016.
\url{https://doi.org/10.1002/9781118766804.wbiect155}.

Rössler, Patrick. \emph{Inhaltsanalyse}. 3rd ed. Konstanz: UVK (UTB),
2017.

Stöber, Rudolf. ``Historische Methoden in der
Kommunikationswissenschaft.'' In \emph{Handbuch nicht standardisierte
Methoden in der Kommunikationswissenschaft}, edited by Michael Meyen and
Stefanie Averbeck-Lietz, 303--19. Wiesbaden: Springer VS, 2016.

Weischenberg, Siegfried. \emph{Max Weber und die Entzauberung der
Medienwelt: Theorien und Querelen -- eine andere Fachgeschichte}.
Wiesbaden: Springer VS, 2012.

Wilke, Jürgen. ``Quantitative Verfahren in der
Kommunikationsgeschichte.'' In \emph{Kommunikationsgeschichte:
Positionen und Werkzeug: Ein diskursives Hand- und Lehrbuch,} edited by
Klaus Arnold, Markus Behmer, and Bernd Semrad, 323--43. Münster: Lit,
2008.



\end{hangparas}


\end{document}