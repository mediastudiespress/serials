% see the original template for more detail about bibliography, tables, etc: https://www.overleaf.com/latex/templates/handout-design-inspired-by-edward-tufte/dtsbhhkvghzz

\documentclass{tufte-handout}

%\geometry{showframe}% for debugging purposes -- displays the margins

\usepackage{amsmath}

\usepackage{hyperref}

\usepackage{fancyhdr}

\usepackage{hanging}

\hypersetup{colorlinks=true,allcolors=[RGB]{97,15,11}}

\fancyfoot[L]{\emph{History of Media Studies}, vol. 1, 2021}


% Set up the images/graphics package
\usepackage{graphicx}
\setkeys{Gin}{width=\linewidth,totalheight=\textheight,keepaspectratio}
\graphicspath{{graphics/}}

\title[Unearthing Bundles of Baffling Silences]{Unearthing Bundles of Baffling Silences: The Entangled and Racialized Global Histories of Media and Media Studies} % longtitle shouldn't be necessary

% The following package makes prettier tables.  We're all about the bling!
\usepackage{booktabs}

% The units package provides nice, non-stacked fractions and better spacing
% for units.
\usepackage{units}

% The fancyvrb package lets us customize the formatting of verbatim
% environments.  We use a slightly smaller font.
\usepackage{fancyvrb}
\fvset{fontsize=\normalsize}

% Small sections of multiple columns
\usepackage{multicol}

% Provides paragraphs of dummy text
\usepackage{lipsum}

% These commands are used to pretty-print LaTeX commands
\newcommand{\doccmd}[1]{\texttt{\textbackslash#1}}% command name -- adds backslash automatically
\newcommand{\docopt}[1]{\ensuremath{\langle}\textrm{\textit{#1}}\ensuremath{\rangle}}% optional command argument
\newcommand{\docarg}[1]{\textrm{\textit{#1}}}% (required) command argument
\newenvironment{docspec}{\begin{quote}\noindent}{\end{quote}}% command specification environment
\newcommand{\docenv}[1]{\textsf{#1}}% environment name
\newcommand{\docpkg}[1]{\texttt{#1}}% package name
\newcommand{\doccls}[1]{\texttt{#1}}% document class name
\newcommand{\docclsopt}[1]{\texttt{#1}}% document class option name


\begin{document}

\begin{titlepage}

\begin{fullwidth}
\noindent\LARGE\emph{Launch essay
} \hspace{85mm}\includegraphics[height=1cm]{logo3.png}\\
\noindent\hrulefill\\
\vspace*{1em}
\noindent{\Huge{Unearthing Bundles of Baffling Silences: The Entangled and Racialized Global Histories of Media and Media Studies\par}}

\vspace*{1.5em}

\noindent\LARGE{Wendy Willems}\par}\marginnote{\emph{Wendy Willems, ``Unearthing Bundles of Baffling Silences: The Entangled and Racialized Global Histories of Media and Media Studies,'' \emph{History of Media Studies} 1 (2021), \href{https://doi.org/10.32376/d895a0ea.52801916}{https://doi.org/ 10.32376/d895a0ea.52801916}.} \vspace*{0.75em}}
\vspace*{0.5em}
\noindent{{\large\emph{London School of Economics}, \href{mailto:w.willems@lse.ac.uk}{w.willems@lse.ac.uk}\par}} \marginnote{\href{https://creativecommons.org/licenses/by-nc/4.0/}{\includegraphics[height=0.5cm]{by-nc.png}}\vspace*{0.5em}}

% \vspace*{0.75em} % second author

% \noindent{\LARGE{<<author 2 name>>}\par}
% \vspace*{0.5em}
% \noindent{{\large\emph{<<author 2 affiliation>>}, \href{mailto:<<author 2 email>>}{<<author 2 email>>}\par}}

% \vspace*{0.75em} % third author

% \noindent{\LARGE{<<author 3 name>>}\par}
% \vspace*{0.5em}
% \noindent{{\large\emph{<<author 3 affiliation>>}, \href{mailto:<<author 3 email>>}{<<author 3 email>>}\par}}

\end{fullwidth}

\vspace*{1em}


\newthought{As Michel-Rolph Trouillot} argued, ``any historical narrative is a
particular bundle of silences.''\footnote{Michel-Rolph Trouillot,
  \emph{Silencing the Past: Power and the Production of History}
  (Boston, MA: Beacon Press, 1995), 27.} This also applies to the way we
narrate the history of media institutions and technologies, as well as
to how we document the historical emergence of the field of media and
communication studies. A key task of media historians should be to
unearth the field's multiple silences and to reveal how these are linked
to the exercise of power as, in Trouillot's words, ``the ultimate mark
of power may be its invisibility; the ultimate challenge, the exposition
of its roots.''\footnote{Trouillot, \emph{Silencing the Past}, xix.}

In a 2014 \emph{Communication Theory} article entitled ``Provincializing
Hegemonic Histories of Media and Communication Studies: Toward a
Genealogy of Epistemic Resistance in Africa,'' I critiqued how calls for
the ``internationalizing'' or ``de-Westernizing'' of media and
communication studies implicitly silence a much longer history of media
and communication studies outside the so-called West.\footnote{Wendy
  Willems, ``Provincializing Hegemonic Histories of Media and
  Communication Studies: Toward a Genealogy of Epistemic Resistance in
  Africa,'' \emph{Communication Theory} 24, no. 4 (2014): 415--34.} I
argued that these calls suggested that scholars in the non-West had
somehow not previously engaged in critical knowledge production on media
and communication. My article reinscribed the epistemological and
historical foundations of media and communication studies in Africa,
which hegemonic histories of the field had marginalized. It called for
an acknowledgment of the multiple genealogies of media and communication
studies in different parts of the world.

\enlargethispage{2\baselineskip}

\vspace*{2em}

\noindent{\emph{History of Media Studies}, vol. 1, 2021}




 \end{titlepage}





Since my article's publication, demands for internationalization and
de-Westernization have increasingly been replaced with calls for
``decolonization'' in the wake of the 2015 \#RhodesMustFall protests at
the University of Cape Town and other universities across the globe, as
well as following the 2015 and 2020 \#BlackLivesMatter protests.
Demonstrations have pointed to the need for universities to transform in
a number of ways, including teaching more diverse curricula, making
higher education more accessible to students from marginalized economic
backgrounds, revising teaching methodologies and research ethics to
render them more democratic and less hierarchical, hiring a diverse pool
of faculty, and addressing how universities have benefited from slavery
or fed into colonialism through eugenics and scientific racism, as well
as the need for reparations. Again, it is important to acknowledge here
the longer genealogy of demands for decolonization and liberation in the
African context, ranging from W. E. B. Du Bois to Frantz Fanon to Ngũgĩ
wa Thiong'o.

These calls for decolonization have provoked a response in our field,
primarily from U.S.-based, African, and Latin American scholars who---if
somewhat separately―have drawn attention to a multitude of problem
areas:\footnote{See, for example, the discussions within the
  Communication and Decoloniality Working Group as part of the Latin
  American Association of Communication Researchers (ALAIC) conference,
  University of Costa Rica, July 30--August 1, 2018,
  https://www.alaic.org/site/wp-content/uploads/2018/06/GI\_4.pdf (last
  accessed September 1, 2021).} the continued marginalization of
scholars of color in publication rates, citation rates, and editorial
journal positions;\footnote{Paula Chakravartty, Rachel Kuo, Victoria
  Grubbs, and Charlton McIlwain, ``\#CommunicationSoWhite,''
  \emph{Journal of Communication} 68, no. 2 (2018): 254--66.} the need
for systemic redress;\footnote{Eve Ng, Khadijah Costley White, and
  Anamik Saha, ``\#CommunicationSoWhite: Race and Power in the Academy
  and Beyond,'' \emph{Communication, Culture and Critique} 13, no. 2
  (2020): 143--51.} the silence on the history of European and American
imperialism in graduate communication studies syllabi and canonical
texts in media and communications;\footnote{For the silence on
  imperialism in graduate syllabi, see Paula Chakravartty and Sarah J.
  Jackson, ``The Disavowal of Race in Communication Theory,''
  \emph{Communication and Critical/Cultural Studies} 17, no. 2 (2020):
  210--19. For the silence on imperialism in canonical texts, see
  Roopali Mukherjee, ``Of Experts and Tokens: Mapping a Critical Race
  Archaeology of Communication,'' \emph{Communication, Culture and
  Critique} 13, no. 2 (2020): 152--67.} the characterization of research
on media, communication, and race as addressing peripheral rather than
core issues;\footnote{Mukherjee, ``Of Experts and Tokens.''} the need to
center Africa in media and communication studies and to problematize
claims to universality in much of the work focused on the United States
and Europe;\textsuperscript{9}
the marginalization of African media studies in the U.S.
academy;\textsuperscript{10} and the relevance of decolonial approaches in making
sense of media and communications in Africa and the Global
South.\textsuperscript{11}

This body of work has once more highlighted that our field has always
been raced,\textsuperscript{12} as
evidenced by the white vantage points (presumed to be universal) adopted
in canonical texts centered in the field, as well as by the longer
history of institutionally racist practices in universities, journals,
and professional associations. While both the older and newer calls for
decolonization may have different meanings in distinct geographical
contexts, they are ultimately connected in their response to the
afterlives of shared racialized histories of slavery and colonialism and
their contestation of anti-Blackness and anti-Indigeneity in various
parts of the world.

These studies offer much food for thought to historians of media and
media studies. They point to the need for more inclusive and complicated\marginnote{\textsuperscript{9} Bruce Mutsvairo, \emph{Palgrave Handbook of Media
  and Communication Research in Africa} (London: Palgrave Macmillan,
  2018). Winston Mano and viola c. milton, \emph{Routledge Handbook of
  African Media and Communication Studies} (London: Routledge, 2021).}
histories\marginnote{\textsuperscript{10} Moradewun Adejunmobi, ``African Media Studies and
  Marginality at the Center,''~\emph{Black Camera}~7, no. 2 (2016):
  125--39. Wunpini F. Mohammed, ``Decolonizing African Media
  Studies,''~\emph{Howard Journal of Communications},~32, no. 2
  (2021):~123--38.} of our field, ones that acknowledge both its multiple global
origins and the racialized history of media and media studies. Yet
increasingly, the notion of decolonization runs the risk of turning into
an empty metaphor,\footnote{Eve Tuck and K. Wayne Yang, ``Decolonization
  Is Not a Metaphor,'' \emph{Decolonization: Indigeneity, Education, and
  Society} 1, no. 1 (2012): 1--40.} used\marginnote{\textsuperscript{11} Sarah Chiumbu and Mehita Iqani, \emph{Media Studies:
  Critical African and Decolonial Approaches} (Cape Town: Oxford
  University Press, 2019). Last Moyo, \emph{The Decolonial Turn in Media
  Studies in Africa and the Global South} (London: Palgrave, 2020).
  Kebhuma Langmia and Agnes L. Lando, \emph{Digital Communications at
  Crossroads in Africa: A Decolonial Approach} (New York: Palgrave
  MacMillan, 2020).} to tick\marginnote{\textsuperscript{12} See also Judith N. Martin and Thomas K. Nakayama,
  ``Communication as Raced,'' in \emph{Communication As . . .}, ed.
  Gregory J. Shepherd et al. (New York: Sage, 2006), 75--83.}\setcounter{footnote}{12} boxes or to attract new
pools of student customers who can populate a diverse classroom that
will enhance the competitiveness of the neoliberal university. Too
frequently, the call for decolonization translates into a call for
diversity and inclusion. While the latter are important, it is crucial
to go a step further and ask how the act of including different vantage
points challenges, subverts, and problematizes dominant theoretical
approaches and concepts, received histories, and canonical texts in our
field.

Armond Towns offers a good start here in critiquing both the way in
which media history has been written and how received histories have
become canonized in the field.\footnote{Armond Towns, ``The (Black)
  Elephant in the Room: McLuhan and the Racial," \emph{Canadian Journal
  of Communication} 44, no. 4 (2019): 545--54.} His work on Marshall
McLuhan shows McLuhan's failure to acknowledge the crucial role of Black
bodies in the emergence of Western media technologies. Relatedly, while
media historians might have examined how the BBC promoted the idea of
Empire through its overseas service, they have less frequently asked how
histories of slavery and colonialism enabled the institution's
emergence, and what implications this question might have for debates on
reparations.

Similarly, Gurminder K. Bhambra highlights the erasure of slavery and
colonialism in the Frankfurt School's theorization of
modernity.\footnote{Gurminder K. Bhambra, ``Decolonizing Critical
  Theory?~Epistemological Justice, Progress,
  Reparations,''~\emph{Critical Times}~4, no. 1 (2021): 73--89.} As she
argues, modernity did not ``emerge from separation or rupture, but
through the connected and entangled histories of European
colonization.'' \footnote{Bhambra, ``Decolonizing Critical Theory?'' 81.}
What, for example, would Jürgen Habermas's eighteenth-century European
public sphere look like if its emergence had been understood in the
context of slavery and the slave trade? While the role of media and
technology in perpetuating racism is relatively well documented, media
and communication studies has yet to acknowledge the \emph{constitutive}
nature of race,\footnote{See also Lyndsey Beutin, ``Sylvia Wynter and
  the History of Communication: A New World View,'' paper presented at
  the virtual ICA pre-conference ``Exclusions in the History and
  Historiography of Communication Studies,'' May 26--27, 2021.}
recognizing how histories of slavery and colonialism made possible
particular media institutions and technologies.

The intimate histories extant among Africa, Europe, and the United
States do not only relate to the history of media institutions and
technologies but also to that of media and communication studies as a
field. In his work on McLuhan, Towns highlights how McLuhan appropriated
the racist ideas of John Carothers on ``the African
mind.\textsuperscript{''}\footnote{Towns, ``The (Black) Elephant in the
  Room.''} Carothers was a British psychiatrist~who worked for the
Kenyan colonial government. Other influential scholars in our field
built their careers drawing on fieldwork in Africa. For example, Leonard
W. Doob, a psychologist at Yale University associated with the field of
cognitive psychology and propaganda studies, researched the link between
media and modernization. In his book \emph{Communication in Africa: A
Search for Boundaries}, one of the first academic monographs on
communication in Africa, Doob discusses the sociocultural, linguistic,
and psychological variables impinging on communication patterns in
Africa.\footnote{Leonard W. Doob, \emph{Communication in Africa: A
  Search for Boundaries} (New Haven, CT: Yale University Press, 1961).}
In other work, Doob sought to measure the levels of ``psychological
modernisation'' in Africa and to assess the role of media in the process
of modernization based on empirical research in Kenya, Tanzania, Uganda,
and Somalia.\footnote{Leonard, W. Doob, ``Scales for Assaying
  Psychological Modernization in Africa,'' \emph{Public Opinion
  Quarterly} 31, no. 3 (1967): 414--21.}

A number of studies have situated the work of modernization scholars
such as Wilbur Schramm and Daniel Lerner within the political context of
the Cold War, but less often have commentators viewed those scholars'
research through the prism of race or sufficiently examined how their
fieldwork in Africa shaped both their individual careers and early
formations of media and communications studies on the African continent.
Doing so would offer us a better understanding of the racialized and
entangled histories of media and communication studies across different
continents.







\section{Bibliography}\label{bibliography}

\begin{hangparas}{.25in}{1} 


Adejunmobi, Moradewun. ``African Media Studies and Marginality at the
Center.''~\emph{Black Camera}~7, no. 2 (2016):
125--39.~\url{https://muse.jhu.edu/article/619765}.

Beutin, Lyndsey. ``Sylvia Wynter and the History of Communication: A New
World View,'' paper presented at the virtual ICA pre-conference
``Exclusions in the History and Historiography of Communication
Studies,'' May 26--27, 2021.

Bhambra, Gurminder K. ``Decolonizing Critical Theory?~Epistemological
Justice, Progress, Reparations.''~\emph{Critical Times}~4, no. 1 (2021):
73--89. \url{https://doi.org/10.1215/26410478-8855227}.

Chakravartty, Paula, and Sarah J. Jackson. ``The Disavowal of Race in
Communication Theory.'' \emph{Communication and Critical/Cultural
Studies} 17, no. 2 (2020): 210--19.
\url{https://doi.org/10.1080/14791420.2020.1771743}.

Chakravartty, Paula, Rachel Kuo, Victoria Grubbs, and Charlton McIlwain.
``\#CommunicationSoWhite.'' \emph{Journal of Communication} 68, no. 2
(2018): 254--66. \url{https://doi.org/10.1093/joc/jqy003}.

Chiumbu, Sarah, and Mehita Iqani. \emph{Media Studies: Critical African
and Decolonial Approaches}. Cape Town: Oxford University Press, 2019.

Doob, Leonard W. \emph{Communication in Africa: A Search for
Boundaries}. New Haven, CT: Yale University Press, 1961.

---------. ``Scales for Assaying Psychological Modernization in
Africa.'' \emph{Public Opinion Quarterly} 31, no. 3 (1967): 414--21.
\url{https://doi.org/10.1086/267540}.

Langmia, Kebhuma, and Agnes L. Lando. \emph{Digital Communications at
Crossroads in Africa: A Decolonial Approach}. New York: Palgrave
MacMillan, 2020.

Mano, Winston, and viola c. milton. \emph{Routledge Handbook of African
Media and Communication Studies}. London: Routledge, 2021.

Martin, Judith N., and Thomas K. Nakayama. ``Communication as Raced.''
In \emph{Communication As . . .}, edited by Gregory J. Shepherd, Jeffrey
St. John, and Ted Striphas, 75--83. New York: Sage, 2006.

Mohammed, Wunpini F.~``Decolonizing African Media
Studies.''~\emph{Howard Journal of Communications},~32, no. 2
(2021):~123--38.~\url{https://doi.org/10.1080/10646175.2021.1871868}.

Moyo, Last. \emph{The Decolonial Turn in Media Studies in Africa and the
Global South}. London: Palgrave, 2020.

Mukherjee, Roopali. ``Of Experts and Tokens: Mapping a Critical Race
Archaeology of Communication.'' \emph{Communication, Culture and
Critique} 13, no. 2 (2020): 152--67.
\url{https://doi.org/10.1093/ccc/tcaa009}.

Mutsvairo, Bruce. \emph{Palgrave Handbook of Media and Communication
Research in Africa}. London: Palgrave Macmillan, 2018.

Ng, Eve, Khadijah Costley White, and Anamik Saha.
``\#CommunicationSoWhite: Race and Power in the Academy and Beyond.''
\emph{Communication, Culture and Critique} 13, no. 2 (2020): 143--51.
\url{https://doi.org/10.1093/ccc/tcaa011}.

Towns, Armond. ``The (Black) Elephant in the Room: McLuhan and the
Racial." \emph{Canadian Journal of Communication} 44, no. 4 (2019):
545--54. \url{https://doi.org/10.22230/cjc.2019v44n4a3721}.

Trouillot, Michel-Rolph. \emph{Silencing the Past: Power and the
Production of History}. Boston, MA: Beacon Press, 1995.

Tuck, Eve, and K. Wayne Yang. ``Decolonization Is Not a Metaphor.''
\emph{Decolonization: Indigeneity, Education, and Society} 1, no. 1
(2012): 1--40.
\url{https://jps.library.utoronto.ca/index.php/des/article/view/18630/15554}.

Willems, Wendy. ``Provincializing Hegemonic Histories of Media and
Communication Studies: Toward a Genealogy of Epistemic Resistance in
Africa.'' \emph{Communication Theory} 24, no. 4 (2014): 415--34.
\url{https://doi.org/10.1111/comt.12043}.



\end{hangparas}


\end{document}