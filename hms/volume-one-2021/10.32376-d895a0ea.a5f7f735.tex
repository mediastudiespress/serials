% see the original template for more detail about bibliography, tables, etc: https://www.overleaf.com/latex/templates/handout-design-inspired-by-edward-tufte/dtsbhhkvghzz

\documentclass{tufte-handout}

%\geometry{showframe}% for debugging purposes -- displays the margins

\usepackage{amsmath}

\usepackage{hyperref}

\usepackage{fancyhdr}

\usepackage{hanging}

\hypersetup{colorlinks=true,allcolors=[RGB]{97,15,11}}

\fancyfoot[L]{\emph{History of Media Studies}, vol. 1, 2021}


% Set up the images/graphics package
\usepackage{graphicx}
\setkeys{Gin}{width=\linewidth,totalheight=\textheight,keepaspectratio}
\graphicspath{{graphics/}}

\title[Communication Research in Latin America]{Communication Research in Latin America: Will the "Nocturnal Map" Survive or Fade Away?} % longtitle shouldn't be necessary

% The following package makes prettier tables.  We're all about the bling!
\usepackage{booktabs}

% The units package provides nice, non-stacked fractions and better spacing
% for units.
\usepackage{units}

% The fancyvrb package lets us customize the formatting of verbatim
% environments.  We use a slightly smaller font.
\usepackage{fancyvrb}
\fvset{fontsize=\normalsize}

% Small sections of multiple columns
\usepackage{multicol}

% Provides paragraphs of dummy text
\usepackage{lipsum}

% These commands are used to pretty-print LaTeX commands
\newcommand{\doccmd}[1]{\texttt{\textbackslash#1}}% command name -- adds backslash automatically
\newcommand{\docopt}[1]{\ensuremath{\langle}\textrm{\textit{#1}}\ensuremath{\rangle}}% optional command argument
\newcommand{\docarg}[1]{\textrm{\textit{#1}}}% (required) command argument
\newenvironment{docspec}{\begin{quote}\noindent}{\end{quote}}% command specification environment
\newcommand{\docenv}[1]{\textsf{#1}}% environment name
\newcommand{\docpkg}[1]{\texttt{#1}}% package name
\newcommand{\doccls}[1]{\texttt{#1}}% document class name
\newcommand{\docclsopt}[1]{\texttt{#1}}% document class option name


\begin{document}

\begin{titlepage}

\begin{fullwidth}
\noindent\LARGE\emph{Launch essay
} \hspace{85mm}\includegraphics[height=1cm]{logo3.png}\\
\noindent\hrulefill\\
\vspace*{1em}
\noindent{\Huge{Communication Research in Latin America: Will the "Nocturnal Map" Survive or Fade Away?\par}}

\vspace*{1.5em}

\noindent\LARGE{Raúl Fuentes Navarro}\par}\marginnote{\emph{Raúl Fuentes Navarro, ``Communication Research in Latin America: Will the "Nocturnal Map" Survive or Fade Away?'' \emph{History of Media Studies} 1 (2021), \href{https://doi.org/10.32376/d895a0ea.a5f7f735}{https://doi.org/ 10.32376/d895a0ea.a5f7f735}.} \vspace*{0.75em}}
\vspace*{0.5em}
\noindent{{\large\emph{Universidad de Guadalajara}, \href{mailto:raul@iteso.mx}{raul@iteso.mx}\par}} \marginnote{\href{https://creativecommons.org/licenses/by-nc/4.0/}{\includegraphics[height=0.5cm]{by-nc.png}}}

% \vspace*{0.75em} % second author

% \noindent{\LARGE{<<author 2 name>>}\par}
% \vspace*{0.5em}
% \noindent{{\large\emph{<<author 2 affiliation>>}, \href{mailto:<<author 2 email>>}{<<author 2 email>>}\par}}

% \vspace*{0.75em} % third author

% \noindent{\LARGE{<<author 3 name>>}\par}
% \vspace*{0.5em}
% \noindent{{\large\emph{<<author 3 affiliation>>}, \href{mailto:<<author 3 email>>}{<<author 3 email>>}\par}}

\end{fullwidth}

\vspace*{1em}


\newthought{On June 12,} 2021, at the age of 83, Jesús Martín-Barbero passed away,
defeated by COVID-19 and a complex variety of diseases and pains. For
decades, his health had not been at its best: among other infirmities,
he had endured chronic cardiac and muscular problems. But the recent
death of his wife, Elvira, undoubtedly caused the worst of his
suffering, for with her he lost his lifetime emotional and practical
support. May they both rest in peace.

Although I do not intend this text as an obituary note, some questions
concerning Martín-Barbero's legacy to the field of communication studies
cannot be ignored or postponed. In this context, this article
principally aims to reiterate some of the issues and enlightenments that
have emerged from long-term readings and conversations with him, and to
project them beyond the Latin American academic
field.\footnote{Raúl Fuentes-Navarro, \emph{Un campo cargado de futuro: El estudio de
  la comunicación en América Latina} (Mexico City: FELAFACS, 1992); Raúl
  Fuentes-Navarro, ``Institutionalization and Internationalization of
  the Field of Communication Studies in Mexico and Latin America,'' in
  \emph{The International History of Communication Study}, ed. Peter
  Simonson and David W. Park (New York: Routledge, 2016).
}

Among Martín-Barbero's theoretical and epistemological
contributions,\footnote{Carlos A. Scolari, ``From (New) Media to (Hyper)Mediations: Recovering
  Jesús Martín-Barbero's Mediation Theory in the Age of Digital
  Communication and Cultural Convergence,'' \emph{Information,
  Communication \& Society} 18, no. 9 (2015); Maria Immacolata Vassallo}
his claims for the
recognition of communication as a strategic factor in every
sociocultural dimension were explicitly aimed at political and ethical
debates, formulated in terms of a historical understanding of
interdeterminant processes.\textsuperscript{3}
Following Paulo Freire's example (but Paul Ricœur's as well),
Martín-Barbero developed some of his work's core concepts in a
metaphorical mode. In this sense, the ``nocturnal map to explore the new
field,''\textsuperscript{4} drawn in the final pages
of his canonic book \emph{De los medios a las mediaciones} (\emph{From
the Media to Mediations}), condenses the sense of the work done and the
effort required, of the ``history re-known, and the future
sought,''\textsuperscript{5} as a process of culture
lived in common, mediated by the reading.

\enlargethispage{3\baselineskip}

\vspace*{2em}

\noindent{\emph{History of Media Studies}, vol. 1, 2021}




 \end{titlepage}





The\marginnote{de Lopes, ``The Barberian Theory of Communication,'' \emph{MATRIZes}
  12, no. 1 (2018).
}  book\marginnote{\textsuperscript{3} Jesús Martín-Barbero, ``A Latin American Perspective on
  Communication/Cultural Mediation,'' \emph{Global Media and
  Communication} 2, no. 3 (2006).
} was\marginnote{\textsuperscript{4} Martín-Barbero took the expression ``nocturnal map'' from Antoine de
  Saint-Exupéry, \emph{Piloto de guerra} (Madrid: Ediciones CS, 1969).
} written\marginnote{\textsuperscript{5} Jesús Martín-Barbero, \emph{De los medios a las mediaciones:
  Comunicación, cultura y hegemonía} (Mexico City: Gustavo Gili, 1987).
  All translations from the Spanish of this edition are the author's.
}\setcounter{footnote}{5} ``accepting that since times are not for
synthesis,'' there are ``many areas of everyday reality that are still
to be explored, zones into whose exploration we cannot advance except by
groping or only with a nocturnal map.'' Such a map would serve to
investigate ``no other things but domination, production and work, but
seen from the other side: that of the gaps, consumption and pleasure. A
map not for escape, but for the recognition of the situation from
mediations and subjects.''\footnote{Martín-Barbero, \emph{De los medios a las mediaciones}, 229.
} The
metaphor meant so much to Martín-Barbero that he later adopted the
nickname ``mestizo cartographer.''\footnote{Jesús Martín-Barbero, ``Aventuras de un cartógrafo mestizo en el campo
  de la comunicación,'' \emph{Panorama económico} 7, no. 7 (1999), 12.
}

Jesús Martín-Barbero was a Colombian---or better, a Latin
American---philosopher, born in Spain and educated in Belgium and
France, who ``discovered'' communication practices in Colombia and
decided to convert himself into a scholar in this field. He took
semiology as his first approach, but soon he began to think of
communication trans-disciplinarily. When \emph{De los medios a las
mediaciones} first appeared in 1987, Martín-Barbero was already a
well-known and respected educator and researcher of communication and
culture throughout the Spanish-language world. The book heightened his
recognition, making him the leading scholar of the field in
Ibero-America (i.e., Latin America plus Spain and Portugal). The work's
translation into English (1993), Portuguese (1997), and French (2002)
broadened his international
renown.\footnote{Jesús Martín-Barbero, \emph{Communication, Culture and Hegemony: From
  the Media to Mediations}, trans. Elizabeth Fox and Robert A. White
  (London: Sage, 1993); \emph{Dos meios às mediações: Comunicação,
  cultura e hegemonia} (Rio de Janeiro: UFRJ, 1997); \emph{Des médias
  aux médiations: Communication, culture et hégémonie} (Paris: CNRS,
  2002).
}

In an early review of the book,\footnote{Raúl Fuentes-Navarro, ``Pensar la comunicación desde la cultura,''
  \emph{Signo y pensamiento} 8, no. 14 (1989).
} I
brought out two reading keys explicitly stated by the author in its
first pages. One was his call to ``lose the object in order to gain the
process''---departing from mediations and subjects, not from mass
culture or the media, and taking historical dynamics as the axis for the
study of cultural processes that ``articulate communicative practices
with social movements.''\footnote{Martín-Barbero, \emph{Communication, Culture and Hegemony}, 187.
} The
other key lay in the articulating model he proposed, and its application
to the reading process itself: the book is presented as an instrumental
element of mediation between subjects, practices, and projects of
transformation, which simultaneously turns out to be a historical
recount and, as such, a meta research-oriented
device.\footnote{Raúl Fuentes-Navarro, ``Investigación y meta-investigación sobre
  comunicación en América Latina,'' \emph{MATRIZes} 13, no. 1 (2019).
} \emph{De los medios a las
mediaciones} thus conveys, under its complex discursive structure,
original contributions to the historiography of media and communication
practices and cultures, as well as to the history of the academic field.
Yet even more, for Latin America, it has played the role of an
influential, reflexive work, like others from the decade, one that would
define the area as distinct among Western cultural and linguistic
regions.\footnote{Among the other works were, Armand Mattelart, \emph{L'invention de la
  communication} (Paris: La Découverte, 1994); and John Durham Peters,
  \emph{Speaking into the Air: A History of the Idea of Communication}
  (Chicago: University of Chicago Press, 1999).
}

\emph{From the Media to Mediations} (or, \emph{Communication, Culture
and Hegemony}, the title the editors of the English translation chose,
unfortunately inverting the original title and subtitle) consists of
three parts. The first, ``The People and the Masses in Culture: The
Highlights of the Debate,'' unfolds an erudite recovery of Antonio
Gramsci, Raymond Williams, Pierre Bourdieu, and Michel de Certeau, among
other authors, to give an account of the current state of the debate on
culture, while also laying the foundations to advance Martín-Barbero's
own proposals for \emph{hegemony} as the key concept for thinking the
sociocultural mediations of communication.

The second part of the book, titled ``The Historical Matrices of Mass
Mediation,'' incorporates work Martín-Barbero had previously
disseminated in widely cited articles, papers, and lectures---though his
proposals make more sense as a whole in the book, where they clarify and
complement each other. Finally, the third part focuses on the problems
and research proposals that, without the previous one hundred and fifty
pages, would solidly sustain their relevance, but that acquire much
greater weight with them. Titled ``Modernization and Mass Mediation in
Latin America,'' this section aims to integrate reflection on ``Latin
America as a space of debate and
combat.''\footnote{Martín-Barbero, \emph{De los medios a las mediaciones}, 163.
} It divides into two
chapters, first, ``The Processes: From Nationalisms to Transnationals,''
and second, ``The Methods: From Media to Mediations.'' Here unfolds the
core of Martín-Barbero's legacy to the field: the methodological shift
``from media to mediations'' and the continuing process of ``mapping
out'' those mediations as heuristic tools and empirical references in a
dynamic transdisciplinary model of ``communication within culture and
culture within politics.''\footnote{Martín-Barbero, \emph{De los medios a las mediaciones,} 228.
}

\begin{quote}
The significance of the theoretical and methodological shift indicated
in this chapter's title is already broadly illustrated in the
description of the historical transformations outlined above. . . . Over
the last few years, a Latin American movement, dissolving
pseudo-theoretical issues and cutting through ideological inertias, has
opened up a new way of thinking about the constitution of mass society,
namely, from the perspective of transformations in subaltern cultures.
Communication in Latin America has been profoundly affected by external
transnationalization but also by the emergence of new social actors and
new cultural identities. Thus, communication has become a strategic
arena for the analysis of the obstacles and contradictions that move
these societies, now at the crossroads between accelerated
underdevelopment and compulsive modernization. Because communication is
the meeting point of so many new conflicting and integrating forces, the
centre of the debate has shifted from media to
mediations.\footnote{Martín-Barbero, \emph{Communication, Culture and Hegemony}, 187.
}
\end{quote}

Ten, and then twenty and thirty years after the book's publication,
several scholarly communities celebrated the continuing interest in and
influence of \emph{De los medios a las mediaciones} with special issues
of their books and journals.\footnote{María Cristina Laverde and Rossana Reguillo, eds., \emph{Mapas
  nocturnos: Diálogos con la obra de Jesús Martín-Barbero} (Santafé de
  Bogotá: Universidad Cen-} From
diverse angles and countries, a good number of scholars, including
myself,\textsuperscript{17} provided empirical
information and critical interpretations about the multiple and
contradictory\marginnote{tral/Siglo del Hombre editores, 1998); Miquel
  de Moragas, José Luis Terrón, and Omar Rincón, eds., \emph{``De los
  medios a las mediaciones'' de Jesús Martín-Barbero: Treinta años
  después} (Bellaterra: Institut de la Comunicació, Universitat Autònoma
  de Barcelona, 2017); ``Jesús Martín-Barbero: Treinta años de \emph{De
  los medios a las mediaciones},'' special issue, \emph{MATRIZes} 12,
  no. 1 (2018); ``Más allá de las mediaciones y la hibridación,''
  special issue, \emph{Versión: Estudios de comunicación y política,}
  no. 3 (2019).
} processes\marginnote{\texsuperscript{17} Raúl Fuentes-Navarro, ``Un texto cargado de futuro: Apropiaciones y
  proyecciones de \emph{De los medios a las mediaciones} en América
  Latina,'' in Laverde and Reguillo, \emph{Mapas nocturnos}; Raúl
  Fuentes-Navarro, ``Apropiaciones y proyecciones de \emph{De los medios
  a las mediaciones} en el campo académico de la comunicación: Una
  revisión de su impacto, veinte años después,'' \emph{Anuario CONEICC
  de Investigación de la Comunicación} 14 (2007); Raúl Fuentes-Navarro,
  ``\emph{De los medios a las mediaciones}: Reflexiones en sus treinta
  años, desde una perspectiva sociocultural,'' in Moragas, Terrón, and
  Rincón, \emph{``De los medios a las mediaciones'' de Jesús
  Martín-Barbero}.
}\setcounter{footnote}{17} of reading and assimilating Martín-Barbero's
proposals. Without ever going through revisions beyond new prologues
across its six Spanish editions,\footnote{Jesús Martín-Barbero, ``\emph{From the Media to Mediations}---Three
  Introductions,'' \emph{MATRIZes} 12, no. 1 (2018).
}
the book remains an indispensable reference in many articles, essays,
course bibliographies, and graduate theses---as it has been, almost
without variation, for more than three decades.

In the many interventions of his career, Martín-Barbero expressed his
vision that two opposing foundational conceptions characterized Latin
American communication studies, and the necessity to surpass them: ``On
the one hand, there was the Functional paradigm,'' which related the
study of communication to the diffusion of innovations; and ``on the
other hand, there was the Theory of Dependence,'' which asserted that
mass communication formed ``part of the process that included the
domination that Latin American countries had to put up with.'' The
arguments behind Martín-Barbero's defense of his position about the
field's strategies to cope with the changing configurations of the
communication-culture-politics dimensions of Latin American societies
became progressively explicit and coherent as he debated, sometimes
fiercely, with leaders and followers of other schools of thought,
especially some critical political economists and idealistic
postmodernists. The history of the field he traced, as well as the
theoretical and methodological approach he sustained, became one of the
most influential sources of the ``Latin American school of thought.'' It
shaped both research practices and the academic training of
professionals trying ``to understand the role played by communication
processes and the mass media in the changes that were taking place in
Latin America'' (and elsewhere) before and after the turn of the
century.\footnote{Jesús Martín-Barbero, ``Communication as an Academic Field: Latin
  America,'' in \emph{The International Encyclopedia of Communication},
  ed. Wolfgang Donsbach (New York: Wiley, 2008). For the ``elsewhere,''
  see Nick Couldry and Andreas Hepp, ``Conceptualizing Mediatization:
  Contexts, Traditions, Arguments,'' \emph{Communication Theory} 23, no.
  3 (2013).
}

But the growing distance between research on the rapidly evolving
communication and sociocultural mediations, on the one
hand,\footnote{``When a certain communicational product (television news, telenovela,
  advertising) is the object of study, from its \emph{industrial
  format}, the researcher can trigger elements of its \emph{narrative}
  in articulation with the \emph{logics of production} exploring the
  \emph{technicity}. He can also articulate the communicational
  phenomenon with the \emph{competences of reception} through the
  mediations of \emph{rituality} or \emph{sensoriality}. It is therefore
  a matter to draw up a specific strategic use of the maps of mediations
  for specific empirical research'' (Vassallo de Lopes, ``The Barberian
  Theory of Communication,'' 60--61).
} and the implications of
this approach for professional training within universities ever more
dependent on the market economy, on the other, have steadily heightened
the constitutive ``tensions'' subtending the field. Martín-Barbero often
expressed his commitment to the study of mass communication and the
shift that ``prevents it from being conceived as a simple matter of
markets and consumption, thus demanding the analysis of communication as
a decisive space in which the public sphere is being redefined and
democracy is being
reconstituted.''\textsuperscript{21}

Taking such an emphasis on \emph{praxis}, along with ``nocturnal maps''
and similar heuristic approaches as constitutive axes for Latin American
academic production on communication, we could describe the recent
history of the field in the region as marked by a ``disintegrated
internationalization,''\textsuperscript{22} one
related to ``the stark contrast between the progressive\marginnote{\textsuperscript{21} Martín-Barbero, ``A Latin American Perspective on
  Communication/Cultural Mediation,'' 284.
}
institutionalization\marginnote{\textsuperscript{22} Raúl Fuentes-Navarro, ``La investigación de la comunicación en América
  Latina: Una internacionalización desintegrada,'' \emph{Oficios
  terrestres}, no. 31 (2014).
}\setcounter{footnote}{22} of the field,'' on the one hand, and ``its scant
influence beyond the Ibero-American and Hispanic academic spaces'' and
``waning'' dialogue with other academic communities, on the
other.\footnote{Florencia Enghel and Martín Becerra, ``(Re)Situating Latin America in
  International Communication Theory,'' \emph{Communication Theory} 28,
  no. 2 (2018); Florencia Enghel and Martín Becerra, ``How to
  Incorporate Latin American Communication Studies into Northern/Western
  Circles? Reflections on Academic Pluralism as Co-production,'' in
  \emph{Media and Governance in Latin America: Toward a Plurality of
  Voices}, ed. Ximena Orchard, Sara Garcia Santamaria, Julieta Brambila,
  and Jairo Lugo-Ocando (New York: Peter Lang, 2020).
} These are lines of ongoing
and emerging debates, in which history provides a key dimension.

Martín-Barbero's foundational work and the critical assimilation and
development of his legacy by peers and successors seem crucial for the
most needed (re)interpretations and (re)orientations of the historical
processes of consolidation of the field of communication studies in
Latin America. Hopefully, it can also guide the consolidation of an
enriched historiography of those questions, advancing toward some
``daytime mediations'' and more productive dialogues with other
international and regional communities and academic cultures. In a brief
but deep and luminous essay, signed in 2019, Martín-Barbero reflected on
the transformations of his work, ``traversed by history and culture,''
and arrived at the following certitude: ``What it leads us to and what
configuration we give to the world today depends entirely on how
creatively and critically we confront those spaces that, otherwise,
remain entirely conquered by the same history of progress and the same
capitalist temporality of which we are
heirs.''\footnote{ Jesús Martín-Barbero, ``Mapas nocturnos y mediaciones diurnas,''
  \emph{Philosophical Readings} 11, no. 3 (2019).}





\section{Bibliography}\label{bibliography}

\begin{hangparas}{.25in}{1} 



Couldry, Nick, and Andreas Hepp. ``Conceptualizing Mediatization:
Contexts, Traditions, Arguments.'' \emph{Communication Theory} 23, no. 3
(2013): 191--202. \url{https://doi.org/10.1111/comt.12019}.

Enghel, Florencia, and Martín Becerra. ``(Re)Situating Latin America in
International Communication Theory.'' \emph{Communication Theory} 28,
no. 2 (2018): 111--30.
\url{https://academic.oup.com/ct/article-abstract/28/2/111/4994890}.

---------. ``How to Incorporate Latin American Communication Studies
into Northern/Western Circles? Reflections on Academic Pluralism as
Co-production.'' In \emph{Media and Governance in Latin America: Toward
a Plurality of Voices}, edited by Ximena Orchard, Sara Garcia
Santamaria, Julieta Brambila, and Jairo Lugo-Ocando, 59--73. New York:
Peter Lang, 2020.

Fuentes-Navarro, Raúl. ``Pensar la comunicación desde la cultura.''
\emph{Signo y pensamiento} 8, no. 14 (1989): 119--27.
\url{https://revistas.javeriana.edu.co/index.php/signoypensamiento/article/view/3502}.

---------. \emph{Un campo cargado de futuro: El estudio de la
comunicación en América Latina}. Mexico City: FELAFACS, 1992.

---------. ``Un texto cargado de futuro: Apropiaciones y proyecciones de
\emph{De los medios a las mediaciones} en América Latina.'' In
\emph{Mapas nocturnos: Diálogos con la obra de Jesús Martín-Barbero},
edited by María Cristina Laverde and Rossana Reguillo, 181--97. Santafé
de Bogotá: Universidad Central/Siglo del Hombre editores, 1998.

---------. ``Apropiaciones y proyecciones de \emph{De los medios a las
mediaciones} en el campo académico de la comunicación: Una revisión de
su impacto, veinte años después.'' \emph{Anuario CONEICC de
Investigación de la Comunicación} 14 (2007): 149--66.

---------. ``La investigación de la comunicación en América Latina: Una
internacionalización desintegrada.'' \emph{Oficios terrestres}, no. 31
(2014): 11--22.
\url{https://perio.unlp.edu.ar/ojs/index.php/oficiosterrestres/article/view/2424/2154}.

---------. ``Institutionalization and Internationalization of the Field
of Communication Studies in Mexico and Latin America.'' In \emph{The
International History of Communication Study}, edited by Peter Simonson
and David W. Park, 325--45. New York: Routledge, 2016.

---------. ``\emph{De los medios a las mediaciones}: Reflexiones en sus
treinta años, desde una perspectiva sociocultural.'' In \emph{``De los
medios a las mediaciones'' de Jesús Martín-Barbero}, edited by Miquel de
Moragas, José Luis Terrón, and Omar Rincón, 118--20. Bellaterra:
Institut de la Comunicació, Universitat Autònoma de Barcelona, 2017.

---------. ``Investigación y meta-investigación sobre comunicación en
América Latina.'' \emph{MATRIZes} 13, no. 1 (2019): 27--48.
\url{http://dx.doi.org/10.11606/issn.1982-8160.v13i1p27-48}.

``Jesús Martín-Barbero: Treinta años de \emph{De los medios a las
mediaciones}.'' Special issue, \emph{MATRIZes} 12, no. 1 (2018).
\url{https://www.revistas.usp.br/matrizes/issue/view/10638}.

Laverde, María Cristina, and Rossana Reguillo, eds. \emph{Mapas
nocturnos: Diálogos con la obra de Jesús Martín-Barbero}. Santafé de
Bogotá: Universidad Central/Siglo del Hombre editores, 1998.

Martín-Barbero, Jesús. \emph{De los medios a las mediaciones:
Comunicación, cultura y hegemonía}. Mexico City: Gustavo Gili, 1987.

---------. \emph{Communication, Culture, and Hegemony: From the Media to
Mediations}. Trans. Elizabeth Fox and Robert A. White. London: Sage,
1993.

---------. \emph{Dos meios às mediações: Comunicação, cultura e
hegemonia}. Rio de Janeiro: UFRJ, 1997.

---------. ``Aventuras de un cartógrafo mestizo en el campo de la
comunicación.'' \emph{Panorama económico} 7, no. 7 (1999): 12--23.
\url{https://doi.org/10.32997/2463-0470-vol.7-num.7-1999-528}.

---------. \emph{Des médias aux médiations: Communication, culture et
hégémonie}. Paris: CNRS, 2002.

---------. ``A Latin American Perspective on Communication/Cultural
Mediation.'' \emph{Global Media and Communication} 2, no. 3 (2006):
279--97.
\url{https://journals.sagepub.com/doi/10.1177/1742766506069579}.

---------. ``Communication as an Academic Field: Latin America.'' In
\emph{The International Encyclopedia of Communication}, edited by
Wolfgang Donsbach, 614--20. New York: Wiley, 2008.

---------. ``\emph{From the Media to Mediations}---Three
Introductions.'' \emph{MATRIZes} 12, no. 1 (2018): 9--31.
\url{http://dx.doi.org/10.11606/issn.1982-8160.v12i1p9-31}.

---------. ``Mapas nocturnos y mediaciones diurnas.''
\emph{Philosophical Readings} 11, no. 3 (2019): 193--98.
\url{https://doi.org/10.5281/zenodo.3560373}.

``Más allá de las mediaciones y la hibridación.'' Special issue,
\emph{Versión: Estudios de comunicación y política,} no. 3 (2019).
\url{https://versionojs.xoc.uam.mx/index.php/version/issue/view/151\#pkp_content_nav}.

Mattelart, Armand. \emph{L'invention de la communication}. Paris: La
Découverte, 1994.

Moragas, Miquel de, José Luis Terrón, and Omar Rincón, eds. \emph{``De
los medios a las mediaciones'' de Jesús Martín-Barbero: Treinta años
después}. Bellaterra: Institut de la Comunicació, Universitat Autònoma
de Barcelona, 2017.

Peters, John Durham. \emph{Speaking into the Air: A History of the Idea
of Communication}. Chicago: University of Chicago Press, 1999.

Saint-Exupéry, Antoine de. \emph{Piloto de guerra}. Madrid: Ediciones
CS, 1969.

Scolari, Carlos A. ``From (New) Media to (Hyper)Mediations: Recovering
Jesús Martín-Barbero's Mediation Theory in the Age of Digital
Communication and Cultural Convergence.'' \emph{Information,
Communication \& Society} 18, no. 9 (2015): 1092--107.
\url{https://repositori.upf.edu/bitstream/handle/10230/27654/scolari_infcomsoc_from.pdf?sequence=1}.

Vassallo de Lopes, Maria Immacolata. ``The Barberian Theory of
Communication.'' \emph{MATRIZes} 12, no. 1 (2018): 39--63.
\url{http://dx.doi.org/10.11606/issn.1982-8160.v12i1p39-63}.



\end{hangparas}


\end{document}
