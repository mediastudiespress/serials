% see the original template for more detail about bibliography, tables, etc: https://www.overleaf.com/latex/templates/handout-design-inspired-by-edward-tufte/dtsbhhkvghzz

\documentclass{tufte-handout}

%\geometry{showframe}% for debugging purposes -- displays the margins

\usepackage{amsmath}

\usepackage{hyperref}

\usepackage{fancyhdr}

\usepackage{hanging}

\hypersetup{colorlinks=true,allcolors=[RGB]{97,15,11}}

\fancyfoot[L]{\emph{History of Media Studies}, vol. 1, 2021}


% Set up the images/graphics package
\usepackage{graphicx}
\setkeys{Gin}{width=\linewidth,totalheight=\textheight,keepaspectratio}
\graphicspath{{graphics/}}

\title[Biographical Encyclopedia of Communication Study]{Biographical Encyclopedia of Communication Study: Fostering Historiography and Memory in the Field} % longtitle shouldn't be necessary

% The following package makes prettier tables.  We're all about the bling!
\usepackage{booktabs}

% The units package provides nice, non-stacked fractions and better spacing
% for units.
\usepackage{units}

% The fancyvrb package lets us customize the formatting of verbatim
% environments.  We use a slightly smaller font.
\usepackage{fancyvrb}
\fvset{fontsize=\normalsize}

% Small sections of multiple columns
\usepackage{multicol}

% Provides paragraphs of dummy text
\usepackage{lipsum}

% These commands are used to pretty-print LaTeX commands
\newcommand{\doccmd}[1]{\texttt{\textbackslash#1}}% command name -- adds backslash automatically
\newcommand{\docopt}[1]{\ensuremath{\langle}\textrm{\textit{#1}}\ensuremath{\rangle}}% optional command argument
\newcommand{\docarg}[1]{\textrm{\textit{#1}}}% (required) command argument
\newenvironment{docspec}{\begin{quote}\noindent}{\end{quote}}% command specification environment
\newcommand{\docenv}[1]{\textsf{#1}}% environment name
\newcommand{\docpkg}[1]{\texttt{#1}}% package name
\newcommand{\doccls}[1]{\texttt{#1}}% document class name
\newcommand{\docclsopt}[1]{\texttt{#1}}% document class option name


\begin{document}

\begin{titlepage}

\begin{fullwidth}
\noindent\LARGE\emph{Launch essay
} \hspace{85mm}\includegraphics[height=1cm]{logo3.png}\\
\noindent\hrulefill\\
\vspace*{1em}
\noindent{\Huge{Biographical Encyclopedia of\\\noindent Communication Study: Fostering\\\noindent Historiography and Memory in the Field\par}}

\vspace*{1.5em}

\noindent\LARGE{Thomas Wiedemann}\par}\marginnote{\emph{Thomas Wiedemann and Michael Meyen, ``Biographical Encyclopedia of Communication Study: Fostering Historiography and Memory in the Field,'' \emph{History of Media Studies} 1 (2021), \href{https://doi.org/10.32376/d895a0ea.74443024}{https://doi.org/ 10.32376/d895a0ea.74443024}.} \vspace*{0.75em}}
\vspace*{0.5em}
\noindent{{\large\emph{Ludwig Maximillian Universität München},\\\noindent \href{mailto:thomas.wiedemann@ifkw.lmu.de}{thomas.wiedemann@ifkw.lmu.de}\par}} \marginnote{\href{https://creativecommons.org/licenses/by-nc/4.0/}{\includegraphics[height=0.5cm]{by-nc.png}}}

\vspace*{0.75em} 

\noindent{\LARGE{Michael Meyen}\par}
\vspace*{0.5em}
\noindent{{\large\emph{Ludwig Maximillian Universität München},\\\noindent \href{mailto:michael.meyen@ifkw.lmu.de}{michael.meyen@ifkw.lmu.de}\par}}

% \vspace*{0.75em} % third author

% \noindent{\LARGE{<<author 3 name>>}\par}
\vspace*{0.5em}
% \noindent{{\large\emph{<<author 3 affiliation>>}, \href{mailto:<<author 3 email>>}{<<author 3 email>>}\par}}

\end{fullwidth}

\vspace*{1em}


\newthought{This essay reflects} our work and experience as editors of the
\emph{Biographical Encyclopedia of Communication Study}
(\emph{Biografisches Lexikon der Kommunikationswissenschaft}, or
BLexKom),\footnote{BLexKom is supported by the publisher Herbert von Halem, Cologne, and
  designed as an open-access platform. It is available online at
  \url{http://blexkom.halemverlag.de}.
} which presents the main
protagonists in the German-speaking communication field from its
beginnings until today and simultaneously showcases the literature on
the field's history. Considering our project as a work in progress that
encourages scholars to contribute, we launched BLexKom in 2013 with a
small number of entries (about 20) and some additional interview
material. Since then, BLexKom has grown tremendously in size and
dimension. We now count more than 140 biographical entries and several
dozen in-depth interviews. We also integrated supplementary formats and
sections into our platform, including research papers, canonic texts,
autobiographical reflections, features (e.g., on journalism education in
the German Democratic Republic), and obituaries.

This essay aims to provide a deeper insight into the BLexKom project,
whose relevance extends well beyond the German case. To be clear, we do
not consider BLexKom as a research output per se, nor do we think of it
as the only legitimate history of communication
\enlargethispage{2\baselineskip}

\vspace*{2em}

\noindent{\emph{History of Media Studies}, vol. 1, 2021}




 \end{titlepage}



\noindent as an academic
discipline in Germany. Instead, by offering as many sources as possible,
we designed our platform in an effort to foster research on the
discipline's development, appearance, and orientation, and to stimulate
memory processes in the field. To support this claim and make the
argument for BLexKom plausible, it is first necessary to describe the
theoretical alignment of our project and make its conceptualization
comprehensible. Second, it seems equally important to discuss our role
as editors, in particular regarding the challenges associated with our
decision-making as source providers. Beyond that, we believe that our
experience with BLexKom might hold interest for research projects that
deal with the history of communication studies in different contexts,
especially more contemporary ones. We therefore, third, report in this
essay on the reactions and, in particular, on the critical feedback from
the community, which has in turn triggered self-reflection processes for
us. After all, BLexKom does not leave people indifferent, and perhaps
this is something we should consider when reasoning on the difficult
situation of specialized historiography in communication
studies,\footnote{Jefferson Pooley, ``The Declining Significance of Disciplinary Memory:
  The Case of Communication Research,'' in \emph{Handbuch
  kommunikationswissenschaftliche Erinnerungsforschung}, ed. Christian
  Pentzold and Christine Lohmeier (Berlin: de Gruyter, in press).
} which, to a certain
extent, may also apply to the social sciences in general.

\hypertarget{theoretical-alignment-of-blexkom}{%
\section{Theoretical Alignment of
BLexKom}\label{theoretical-alignment-of-blexkom}}

Although BLexKom includes different formats that allow foci on
intellectual and institutional approaches as well, the encyclopedia
platform is mainly conceptualized as a biographical window onto the
history of communication studies. This does not mean reproducing the
myths about the so-called founding fathers of the discipline, which have
been criticized as ``thin
hagiography''\footnote{Jefferson Pooley and David W. Park, Introduction to \emph{The History
  of Media and Communication Research: Contested Memories}, ed. David W.
  Park and Jefferson Pooley (New York: Peter Lang, 2008), 4.
} and
``great-men-make-history-tales''\footnote{Maria Löblich and Andreas M. Scheu, ``Writing the History of
  Communication Studies: A Sociology of Science Approach,''
  \emph{Communication Theory} 21, no. 1 (2011): 4.
}---the
case of Elisabeth Noelle-Neumann, to give just one example, makes clear
that the literature on the history of the German field is far from an
exception to the rule here.\footnote{Hans Mathias Kepplinger, ``Political Correctness and Academic
  Principles: A Reply to Simpson,'' \emph{Journal of Communication} 47,
  no. 4 (1997): 102--7.
} More
precisely, BLexKom draws on the sociology of science's assumption that
the development of an academic discipline results from cognitive and
social parameters.\footnote{Pierre Bourdieu, \emph{Science of Science and Reflexivity} (Chicago:
  Chicago University Press, 2004).
} We additionally
assume that this development always reflects the background of its most
important figures and the structures they are or were confronted with.
What is more, considering the impact of dominant agents, so to speak,
proves particularly relevant for a small and rather new academic field
such as communication studies, which remained underdeveloped and without
major prestige during most of its existence and, in many parts of the
world, for a long time was mainly a product of external influences.

That said, many historians of communication studies would agree that
beyond anniversaries and the need for legitimization by tradition, the
founders and early figures of the discipline, especially in the German
context, have fallen into oblivion. Karl Bücher, for example, who
launched the country's first communication department at Leipzig
University in 1916,\footnote{Hanno Hardt, \emph{Social Theories of the Press: Constituents of
  Communication Research, 1840s to 1920s} (Lanham, MD: Rowman \&
  Littlefield, 2001), 99--131; Thomas Wiedemann, Michael Meyen, and Iván
  Lacasa-Mas, ``One Hundred Years Communication Study in Europe: Karl
  Bücher's Impact on the Discipline's Reflexive Project,'' \emph{Studies
  in Communication and Media} 7, no. 1 (2018): 13--16.
} or Walter
Hagemann, at the top of the West German field after World War
II,\footnote{Thomas Wiedemann, ``Practical Orientation as a Survival Strategy: The
  Development of \emph{Publizistikwissenschaft} by Walter Hagemann,'' in
  \emph{The International History of Communication Study}, ed. Peter
  Simonson and David W. Park (New York: Routledge, 2016), 116--21.
} are no longer cited, and
today's students rarely know the circumstances these distant ancestors
dealt with. Of course, beginning in the mid-1950s, the shift to an
empirical social scientific discipline changed terminology, theories,
research objects, and methods.\footnote{Maria Löblich, ``German Publizistikwissenschaft and Its Shift from a
  Humanistic to an Empirical Social Scientific Discipline: Elisabeth
  Noelle-Neumann, Emil Dovifat and the Publizistik Debate,''
  \emph{European Journal of Communication} 22, no. 1 (2007): 81--84.
} For
empirically oriented communication scholars in Germany, the United
States became the most important point of reference. Furthermore, the
distance to once leading protagonists of the field was amplified by the
tendency of minimizing subjectivity and personal traces in research
output and theoretical approaches. Consequently, students also learn
little about the authors of even current texts.

If one argues with Jan Assmann's concept of memory, this seems
unsurprising, because only a small part of the \emph{communicative
memory} (the recent past shared by contemporaries) passes over into the
\emph{cultural memory} (the absolute past fixed in storage media such as
texts, icons, rituals, and performances). Once there, most of it remains
in the ``archive'' (the canon that is available but not really present),
whereas just a few references to the past are strategically used and
serve as frames of orientation and
identification.\footnote{Jan Assmann, ``Communicative and Cultural Memory,'' in \emph{Cultural
  Memory Studies: An International and Interdisciplinary Handbook}, ed.
  Astrid Erll and Ansgar Nünning (Berlin: de Gruyter, 2008), 109--18.
} Such processes
clearly also take place in academic fields, especially if they exhibit a
high degree of heterogeneity and interdisciplinarity, which is, in a
global perspective, still the case for communication studies. Within
this framework, we see BLexKom as a research tool against oblivion that
moves beyond enriching the archive. Since the biographical entries and
the interview material, in particular, facilitate links between field
agents, intellectual orientations, institutional traditions, and
research communities that sometimes extend beyond specific content or
citation, they might have an effect on the discipline's cultural memory
as a whole---and even stimulate the communicative memory once again.

\emph{Decision-Making as Source Providers}

The biographical entries with information about German communication
scholars' origin, socialization, career steps, and key publications---at
an average length of about 800 words---form the heart of BLexKom. Our
long-term aspiration is to eventually include all scholars who teach or
taught as full professors at university departments listed by the German
Communication Association (Deutsche Gesellschaft für Publizistik- und
Kommunikationswissenschaft) and the corresponding associations in
Austria and Switzerland. Using these classifications to sketch the
discipline's core is, of course, anything but unproblematic. We
therefore added two more selection criteria: communication scholars who,
first, hold or held a habilitation in the field, or, second, made an
important contribution to the discipline such as an outstanding
literature reference, which brings us to around 300 targeted
encyclopedia entries. Targeting the main protagonists of the discipline
with such a sample undeniably runs the risk of marginalizing those
figures who focused on teaching, did not complete their habilitations,
or failed to secure professorships. Yet, without any judgement, we
assume that managing to give the discipline a face requires resources,
and that means a permanent employment contract with a university and a
minimum of purely academic reputation (i.e., for the German case,
qualification degrees and publications).

In comparison, the selection criteria for our interview section are less
rigid: We regularly ask renowned representatives of the field for an
interview---very often on the occasion of their retirement, which allows
a less inhibited look back. According to BLexKom's theoretical
background (in particular, Pierre Bourdieu's sociology), these
interviews of about 6,000 words systematically explore habitus patterns
and capital resources and, furthermore, provide insights into the logic
of the field. Nevertheless, we must take them for what they are:
subjective reports shaped by personal interests that form part of each
interviewee's ``story about the
self,''\footnote{Anthony Giddens, \emph{Modernity and Self-Identity: Self and Society
  in the Late Modern Age} (Stanford, CA: Stanford University Press,
  1991), 54.
} which underlines the
source character of BLexKom. Consequently, working with such research
material requires a high degree of critical interpretation.

Even though we have established an editorial board and rely on the
publisher's help in case of technical difficulties, we perform our
duties as editors of BLexKom on a large scale. More specifically, we see
ourselves as responsible for producing or soliciting new content,
acquiring portrait photos and additional image material, doing the
editorial work, and ensuring the timely publication of articles and
updates, as well as quality management and website maintenance.
Moreover, the advantages of an online platform give us the opportunity
to think about innovations from time to time. When we launched our
encyclopedia project eight years ago, for example, we hadn't even
thought of sections like features or obituaries, which now attract a
considerable number of clicks.

Our editorial activities of course also confront challenges that
sometimes make a pragmatic approach necessary. In particular, given the
unfinished nature of BLexKom, we have felt obliged to continuously
increase the number of biographical entries and fill at least the most
significant lacunae of the encyclopedia, especially at the beginning. As
a consequence, despite the spirit of teamwork in our project and the
fact that BlexKom counts almost seventy authors to date, we have written
quite a few contributions ourselves. Writing an encyclopedia entry
requires more than superficial knowledge of the history and structures
of the field, as well as the right mixture of proximity and distance
toward the communication scholar in question. Put differently, just
another promotion tool with biographical information and mission
statements is not compatible with BLexKom's vision, which is why we have
decided not to have the encyclopedia entries authorized (unlike the
interviews).

\hypertarget{reactions-from-the-field-and-lessons-learned}{%
\section{Reactions from the Field and Lessons
Learned}\label{reactions-from-the-field-and-lessons-learned}}

Our encyclopedia project has received both encouraging feedback and
criticism. The latter, which is probably more interesting here,
referred, first, to BLexKom's agent-oriented approach, which (according
to the critique) could crowd out the knowledge gained through
intellectual and institutional histories. In addition, some commentators
regarded the encyclopedia entries' personal information as too private,
even irrelevant. Linked to this, some readers raised the concern that
our interviews positioned interviewees in a particular way and that
their statements were highly subjective, sometimes even false.

A second set of criticisms referred to the distorting effect that
resulted from our project's work-in-progress character, especially in
the early years, with many blind spots and, unfortunately, a
considerable gender bias in the number of entries. Furthermore, some
critics viewed the platform as too closely related to our own academic
environment.

In a third pattern, critics lingered on the supposed power of BLexKom
when writing communication study's history, and thus on the monopoly
position of our project. In short, they asked: Who defines what may
belong to the field's memory, and does BLexKom not first need to be
officially legitimized by the German Communication Association, with all
possible implications?

Similar concerns will likely confront many research projects that strive
for a look behind the scenes of a discipline's development, especially
when they do not always follow established paths and frameworks, or even
scratch the \emph{illusio} of the
field.\footnote{ Bourdieu, \emph{Science of Science and Reflexivity}.} And indeed, there is no
denying that BLexKom, with a sociological and historical sense of the
field, offers some ammunition. Regarding our project, we met the
criticisms received by committing to even greater transparency,
openness, and self-reflection, which included initiating a two-day
workshop in 2016 that allowed for an detailed discussion of the points
of critique.

Beyond that, as a response to these criticisms (at the workshop and
otherwise), we endeavored to provide further information about the
sociology of science approach central to our biographical encyclopedia
project, and to make clear that the collected interviews are also in
line with our theoretical background and thus have a strictly scientific
character. We further illustrated, by integrating additional sections
into our platform (open as well to historiographical texts dealing with
ideas and theories or with organizations and institutional settings),
that BLexKom's focus on agents by no means intended to ignore the
importance of other perspectives on the discipline's history and
structures. Perhaps these complementary research pieces written by
several historians of the field might also serve to re-emphasize the
source character of BLexKom's encyclopedia entries and interviews.

Another response to the criticisms was, as mentioned, that we worked
even harder to quickly eliminate as many obvious encyclopedia gaps as
possible and to ensure a well-balanced and diverse spectrum of
portraits, which certainly became easier as the number of biographical
entries grew over the years. Related to this, we strove to increase the
number of authors on our platform, especially with scholars further away
from us and outside our close network (including, of course, colleagues
from Austria and Switzerland).

To counter the power/monopoly position of BLexKom, we further decided to
share responsibilities and to establish an editorial board consisting of
four experts on the history of the German-speaking communications field,
which gives our project a broader footing. In addition, we repeatedly
presented BLexKom to the scientific community, explaining our approach,
disclosing our decisions as editors, and providing deeper insights into
the creation of content. We particularly stressed the team spirit behind
our project and called for collaboration---if not with original
contributions, then at least with comments and feedback, corrections and
additional hints, but also with suggestions for further entries and
other relevant research material.

Today, BLexKom remains an independent platform based exclusively on
academic expertise that sometimes even involves students' research
projects. According to its objective, we hope that BLexKom will foster
research and become recognized for what it intends to be: an inspiring,
at times critical, but in any case essential service to the community of
German communication studies.







\section{Bibliography}\label{bibliography}

\begin{hangparas}{.25in}{1} 



Assmann, Jan. ``Communicative and Cultural Memory.'' In \emph{Cultural
Memory Studies: An International and Interdisciplinary Handbook}, edited
by Astrid Erll and Ansgar Nünning, 109--18. Berlin: de Gruyter, 2008.

Bourdieu, Pierre. \emph{Science of Science and Reflexivity}. Chicago:
Chicago University Press, 2004.

Giddens, Anthony. \emph{Modernity and Self-Identity: Self and Society in
the Late Modern Age}. Stanford, CA: Stanford University Press, 1991.

Hardt, Hanno. \emph{Social Theories of the Press: Constituents of
Communication Research, 1840s to 1920s}. Lanham, MD: Rowman \&
Littlefield, 2001.

Kepplinger, Hans Mathias. ``Political Correctness and Academic
Principles: A Reply to Simpson.'' \emph{Journal of Communication} 47,
no. 4 (1997): 102--7.
\url{https://doi.org/10.1111/j.1460-2466.1997.tb02728.x}.

Löblich, Maria. ``German Publizistikwissenschaft and Its Shift from a
Humanistic to an Empirical Social Scientific Discipline: Elisabeth
Noelle-Neumann, Emil Dovifat and the Publizistik Debate.''
\emph{European Journal of Communication} 22, no. 1 (2007): 69--88.
\url{https://doi.org/10.1177\%2F0267323107073748}.

Löblich, Maria, and Andreas M. Scheu. ``Writing the History of
Communication Studies: A Sociology of Science Approach.''
\emph{Communication Theory} 21, no. 1 (2011): 1--22.
\url{https://doi.org/10.1111/j.1468-2885.2010.01373.x}.

Pooley, Jefferson. ``The Declining Significance of Disciplinary Memory:
The Case of Communication Research.'' In \emph{Handbuch
kommunikationswissenschaftliche Erinnerungsforschung}, edited by
Christian Pentzold and Christine Lohmeier. Berlin: de Gruyter, in press.

Pooley, Jefferson, and David W. Park. Introduction to \emph{The History
of Media and Communication Research: Contested Memories}, edited by
David W. Park and Jefferson Pooley, 1--15. New York: Peter Lang, 2008.

Wiedemann, Thomas. ``Practical Orientation as a Survival Strategy: The
Development of \emph{Publizistikwissenschaft} by Walter Hagemann.'' In
\emph{The International History of Communication Study}, edited by Peter
Simonson and David W. Park, 109--29. New York: Routledge, 2016.

Wiedemann, Thomas, Michael Meyen, and Iván Lacasa-Mas. ``One Hundred
Years Communication Study in Europe: Karl Bücher's Impact on the
Discipline's Reflexive Project.'' \emph{Studies in Communication and
Media} 7, no. 1 (2018): 7--30.
\href{http://dx.doi.org/10.5771/2192-4007-2018-1-6}{https://doi.org/10.5771/2192-4007-2018-1-6}.



\end{hangparas}


\end{document}