% see the original template for more detail about bibliography, tables, etc: https://www.overleaf.com/latex/templates/handout-design-inspired-by-edward-tufte/dtsbhhkvghzz

\documentclass{tufte-handout}

%\geometry{showframe}% for debugging purposes -- displays the margins

\usepackage{amsmath}

\usepackage{hyperref}

\usepackage{fancyhdr}

\usepackage{hanging}

\hypersetup{colorlinks=true,allcolors=[RGB]{97,15,11}}

\fancyfoot[L]{\emph{History of Media Studies}, vol. 1, 2021}


% Set up the images/graphics package
\usepackage{graphicx}
\setkeys{Gin}{width=\linewidth,totalheight=\textheight,keepaspectratio}
\graphicspath{{graphics/}}

\title[What Film and Cultural Histories Can Teach Us About YouTubers]{What Film and Cultural Histories Can Teach Us About YouTubers} % longtitle shouldn't be necessary

% The following package makes prettier tables.  We're all about the bling!
\usepackage{booktabs}

% The units package provides nice, non-stacked fractions and better spacing
% for units.
\usepackage{units}

% The fancyvrb package lets us customize the formatting of verbatim
% environments.  We use a slightly smaller font.
\usepackage{fancyvrb}
\fvset{fontsize=\normalsize}

% Small sections of multiple columns
\usepackage{multicol}

% Provides paragraphs of dummy text
\usepackage{lipsum}

% These commands are used to pretty-print LaTeX commands
\newcommand{\doccmd}[1]{\texttt{\textbackslash#1}}% command name -- adds backslash automatically
\newcommand{\docopt}[1]{\ensuremath{\langle}\textrm{\textit{#1}}\ensuremath{\rangle}}% optional command argument
\newcommand{\docarg}[1]{\textrm{\textit{#1}}}% (required) command argument
\newenvironment{docspec}{\begin{quote}\noindent}{\end{quote}}% command specification environment
\newcommand{\docenv}[1]{\textsf{#1}}% environment name
\newcommand{\docpkg}[1]{\texttt{#1}}% package name
\newcommand{\doccls}[1]{\texttt{#1}}% document class name
\newcommand{\docclsopt}[1]{\texttt{#1}}% document class option name


\begin{document}

\begin{titlepage}

\begin{fullwidth}
\noindent\LARGE\emph{Launch essay
} \hspace{85mm}\includegraphics[height=1cm]{logo3.png}\\
\noindent\hrulefill\\
\vspace*{1em}
\noindent{\Huge{What Film and Cultural Histories Can Teach Us About YouTubers\par}}

\vspace*{1.5em}

\noindent\LARGE{Sue Collins\par}\marginnote{\emph{Sue Collins, ``What Film and Cultural Histories Can Teach Us About YouTubers,'' \emph{History of Media Studies} 1 (2021), \href{https://doi.org/10.32376/d895a0ea.de74debe}{https://doi.org/ 10.32376/d895a0ea.de74debe}.} \vspace*{0.75em}}
\vspace*{0.5em}
\noindent{{\large\emph{Michigan Technological University}, \href{mailto:scollins@mtu.edu}{scollins@mtu.edu}\par}} \marginnote{\href{https://creativecommons.org/licenses/by-nc/4.0/}{\includegraphics[height=0.5cm]{by-nc.png}}}

% \vspace*{0.75em} % second author

% \noindent{\LARGE{<<author 2 name>>}\par}
% \vspace*{0.5em}
% \noindent{{\large\emph{<<author 2 affiliation>>}, \href{mailto:<<author 2 email>>}{<<author 2 email>>}\par}}

% \vspace*{0.75em} % third author

% \noindent{\LARGE{<<author 3 name>>}\par}
% \vspace*{0.5em}
% \noindent{{\large\emph{<<author 3 affiliation>>}, \href{mailto:<<author 3 email>>}{<<author 3 email>>}\par}}

\end{fullwidth}

\vspace*{1em}


\newthought{In this brief} essay, I limit my commentary to the example of political
authority situated in the history of the mediated entertainment persona
engaged in political communication. The ``part played by people'' in
communication processes\footnote{} is a well-examined topic reflected in
the established but ongoing history of media effects, and more recently,
in audience reception and media industry studies. Whether we locate any
of these research areas in either ``communication'' or ``media studies''
seems to be a banal question, given the growing and generally welcomed
interdisciplinarity of the field. One thing about our enterprise seems
certain: engaging media, John Nerone tells us, can hardly be avoided for
scholars in the humanities and social sciences; thus, many turn to
``historical work as a mode of explanation.''\footnote{John Nerone,
  ``Introduction: Mapping the Field of Media History,'' in \emph{The
  International Encyclopedia of Media Studies}: \emph{Media History and
  the Foundations of Media Studies}, eds. Angharad N. Valdivia and John
  Nerone (Oxford: Wiley-Blackwell, 2013), 19.}

What is less straightforward, though, at least in the North American
context, is the place of film studies' contributions to cultural history
that bears on communication and media studies, owing in part to cinema
studies' partitioning of itself from traditional academic boundaries at
its inception and its early emphasis on auteur theory and textual
analysis. With his recent survey of four prominent journals over several
decades, Phil Novak has shown that although film scholarship is assumed
to be dominated by textual interpretation, in fact, the ``most important
category of work in the field is, and has always been, history (although
what has been meant by history has changed at points over the
years).''\footnote{Phillip Novak, \emph{Interpretation and Film Studies:
  Movie Made Meanings} (Cham, Switzerland: Palgrave Macmillian, 2020),
  114.} When the Society for Cinema Studies added \emph{Media} to its
name in 2002, it did so to better reflect the diversity of its members'
interdisciplinary research on the moving image affected by convergences
across related industries and in

\enlargethispage{2\baselineskip}

\vspace*{2em}

\noindent{\emph{History of Media Studies}, vol. 1, 2021}




 \end{titlepage}

\noindent reception practices, accelerated by
digitization.\footnote{\emph{Cinema Journal}, accordingly, changed its
  name to the \emph{Journal of Cinema and Media Studies} in 2018.} Students of film history vis-à-vis cinema studies, however, are less
likely to attend the National or International Communication
Associations, notwithstanding film historian Janet Staiger's call for
cinema scholars to practice historiography more in terms of \emph{media}
history than film history.\footnote{Janet Staiger, ``The Future of the
  Past,'' \emph{Cinema Journal} 44, no. 1 (2004).} One reason for this
may be a historical positioning of film primarily within the division of
``mass communication'' and thus within a circumscribed social scientific
association to a ``milestone,'' or research event, such as in the
propaganda and effects research of the 1930s Payne Fund studies, or the
\emph{Why We Fight} films of World War II.\footnote{Lana F. Rakow notes
  that Shearon Lowery and Melvin DeFleurs's twenty-five year
  canonization of certain research projects in their textbook,
  \emph{Milestones in Mass Communication Research}, has produced a
  tenacious telos of communication history. Rakow, ``Feminist
  Historiography and the Field: Writing New Histories,'' in \emph{The
  History of Media and Communication Research: Contested Memories}, eds.
  David W. Park and Jefferson Pooley (New York: Peter Lang, 2008),
  116--17.}

Just as film scholarship benefits from new approaches to studying
different media enabled by new technologies, so too are communication
and media studies enriched by cinema studies' historical work related to
semiotics and structuralism, critical theory and psychoanalytic theories
of spectatorship, feminism, political economic analysis, and audience
reception. Good film history is also good media (and cultural) history
when film scholars trading in cultural theory, for example, situate
institutional examinations of Hollywood or nontheatrical film practices
in broader historical contexts to account for technological, economic,
regulatory, and social change in ways that reveal the connections or
gaps between co-existing media. But more to the point in the
media-saturated twenty-first century, if ``media use and media effects
may now materialize everywhere, anytime, and with respect to any sort of
context,'' but without clear determinations of any single source of
information in ``our always-on environment,''\footnote{Peter Vorderer,
  David W. Park, and Sarah Lutz, ``A History of Media Effects Research
  Traditions,'' in \emph{Media Effects: Advances in Theory and
  Research}, eds. Mary Beth Oliver, Arthur A. Raney, and Jennings Bryant
  (New York: Routledge, 2020), 11--12.} media histories informed by
broad cultural histories refusing medium or discipline-specific
isolation or the reification of research events can better account for
structural and phenomenological conditions of media ubiquity, power, and
the role that people play.\footnote{Film historian Richard Maltby makes
  this argument with respect to cinema studies' preoccupation with
  textual analysis and genre in ``How Can Cinema History Matter More?''
  \emph{Screening the Past} 22 (2007).}

To this end, I suggest that the history of communication and media
studies better integrate film and cultural histories into its corpus in
order to facilitate productive connections among cognate areas even if
they may vary in their methodological approaches. To take one example,
if we wish to understand the historical conditions of existence that
help explain, at least theoretically, the arrival, prevalence, and
disruption of the entertaining ``YouTuber'' celebrity pundit in today's
highly polarized, post-truth digital environment, then we must grapple
with, among other things, what Susan Herbst calls ``media-derived
authority''; that is, the mechanisms of legitimation enabled by
mediation onto the communicator. Charismatic authority aside,
media-derived authority is ``a culturally and historically situated
formation'' because authority itself ``evolves over time\ldots within
particular sets of institutions and configurations of social
forces.''\footnote{Susan Herbst, ``Political Authority in a Mediated
  Age,'' \emph{Theory and Society} 32, no. 4 (2003): 491.} Certainly,
the YouTuber celebrity profits (and suffers) from media-specific
affordances of the 2.0 platform, but if we want to trace historical
paths from the present construction of popular authority, arguably we
must get at not merely the perception of expertise or credibility in
political knowledge (politics and punditry), but at affective relations
involving authenticity and intimacy in performance. Unsurprisingly, in
some ways the YouTuber celebrity pundit is not unlike their talk radio
or cable counterpart, and the success of these popular figures in
political entertainment has roots, of course, in broadcasting, but I
would argue also in cinema and the celebrity culture that precedes the
moving image.

Mass communication research directs us to some ``canonical'' historical
texts, and for good reason. Among these is Robert Merton's study of
popular singer Kate Smith's eighteen-hour radio bond drive in 1943,
during which Smith sold thirty-nine million dollars in war
bonds.\footnote{Robert K. Merton, with Marjorie Fiske and Alberta
  Curtis, \emph{Mass Persuasion: The Social Psychology of a War Bond
  Drive} (New York: Howard Fertig, 1946).} Similar to the 1938 broadcast
``War of the Worlds,'' the radio marathon provided communications
researchers with the opportunity to conduct in-depth interviews on radio
listeners,\footnote{See Hadley Cantril, Hazel Gaudet, and Herta Herzog,
  \emph{The Invasion From Mars: A Study in the Psychology of Panic}
  (Princeton: Princeton University Press, 1940).} and in the case of
Smith, to document the ways in which audiences reported their
perceptions of her enormous stardom refigured into an ordinary American
patriot. Not only was \emph{Mass Persuasion} remarkable for the rarity
of its subject and methodology at the time---a reception study on a
proto-media event to measure the socio-political influence of an
entertainer in the public sphere---but as Simonson points out, the
concept of ``public image'' as an amalgam of Smith's character (or
construction of her professional persona) ``indexed an emerging politics
of celebrity, made possible in part by media technologies which brought
the distant famous seemingly close up to the masses.''\footnote{Peter
  Simonson, ``Celebrity, Public Image, and American Political Life:
  Rereading Robert K. Merton's \emph{Mass Persuasion},'' \emph{Political
  Communication} 23, no. 3 (2006): 278. It should be noted that
  historian Alan Brinkley observed radio's function in creating
  perceptions of ``immediate intimacy and friendship between the speaker
  and his audience'' in his study of Father Charles Coughlin and
  Louisiana politician Huey Long in the 1930s, \emph{Voices of Protest:
  Huey Long, Father Coughlin, and the Great Depression} (New York:
  Alfred A. Knopf, 1982), 192--3.}

To be sure, Merton's observation on the congruity between Smith's
stardom and her ``mother image'' and sincere ``plain folk'' on-air
persona anticipated Richard Dyer's work on the film ``star image'' and
its authenticating processes by over thirty years, as well as Richard
deCordova's genealogy of the film star system by more than another
ten.\footnote{Richard Dyer, \emph{Stars} (London: British Film
  Institute, 1979); Richard deCordova, \emph{Picture Personalities: The
  Emergence of the Star System in America} (Urbana: University of
  Illinois Press, 1990).} But as contributions to film history, Dyer and
deCordova show how what becomes Hollywood stardom develops out of
complex industrial and discursive relations that constituted cinema and
produced fan culture, while other histories trace film stardom to
precedence in stage performance or chart its overlapping history with
theatre, vaudeville, and film, and still other cultural histories offer
a broader context of celebrity culture emerging in the nineteenth
century.\footnote{See, for example, Benjamin McArthur, \emph{Actors and
  American culture, 1880--1920} (Philadelphia: Temple University} These histories trace early
celebrity or stardom's authentication to a measure of whether the image
of the extraordinary person is consonant with who the public or fan
thinks the ``real'' (or ordinary) person really is, and it is from this
process that audiences experience pleasure\marginnote{Press,
  1984); Henry Jenkins, \emph{What Made Pistachio Nuts? Early Sound
  Comedy and the Vaudeville Aesthetic} (New York: Columbia University,
  1992); Richard Sennett, \emph{The Fall of Public Man: On the Social
  Psychology of Capitalism} (New York: Knopf, 1977); Warren I. Susman,
  ``\,`Personality' and the Making of Twentieth-Century Culture,''
  \emph{Culture as History: The Transformation of American Society in
  the Twentieth Century} (New York: Pantheon, 1984); Charles L. Ponce de
  Leon, \emph{Self-exposure: Human Interest Journalism and the Emergence
  of Celebrity in America, 1890--1940} (Chapel Hill: University of North
  Carolina Press, 2002); Antoine Lilti locates the birth of celebrity in
  the eighteenth century in \emph{The Invention of Celebrity,
  1750--1850} (Malden, MA: Polity, 2017).} in celebrity consumption,
knowing that the person deserves their celebrity status. Celebrity
authentication is also a mechanism of media-derived authority within the
field of entertainment because it grants a perception of legitimacy,
although the process functions differently depending on the industry
sector such that celebrities can produce varying sensations of affect
and identification in the subjectivities of audiences who apprehend
them, whether in film, television, popular music, or
infotainment.\footnote{With respect to the first three sectors of
  entertainment, see P. David Marshall, \emph{Celebrity and Power: Fame
  in Contemporary Culture} (Minneapolis: University of Minnesota Press,
  1997).}

With respect to the construction of \emph{political} authority---by
which I mean, borrowing from Bourdieu, the symbolic power to act as a
spokesperson through the conversion of celebrity capital from the field
of entertainment into the field of politics---Merton argued that the war
bond drive furthered Smith's public image as ``\emph{primarily} a
patriot rather than an entertainer,'' and that her persona
``monopolize{[}d{]} public imagination,'' and at no time was it
``subject to a counterpropaganda.''\footnote{Merton, \emph{Mass
  Persuasion}, 102, 172 (emphasis in original).} Perhaps the latter
point is less remarkable if we take into account the larger context of
``total'' war mobilization, during which nearly every sector of business
accommodated the federal government to an unprecedented extent,
particularly communication-related industries whose trades were needed
to manage and channel public support. For their part, Hollywood stars
sprung forth to personally sell war bonds, not for the first time, but
for the second, though on a much larger scale than during World War
I.\footnote{See, for example, Clayton R. Koppes and Gregory D. Black,
  \emph{Hollywood Goes to War: How Politics, Profits, and Propaganda
  Shaped World War II Movies} (Berkeley: University of California Press,
  1990); Roy Hoopes, \emph{When the Stars Went to War: Hollywood and
  World War II} (New York: Random House, 1994); Thomas Schatz,
  \emph{Boom and Bust: The American Cinema in the 1940s} (Berkeley:
  University of California Press, 1999); Kathryn Cramer Brownell,
  \emph{Showbiz Politics: Hollywood in American Political Life} (Chapel
  Hill: University of North Carolina, 2014); Giorgio Bertellini,
  \emph{The Divo and the Duce} (Berkeley: University of California
  Press, 2019).} Notably, by World War II, film studios were promoting
the idea of ``Hollywood'' as synecdoche for the entertainment industry
more broadly, as they also began to poach and consolidate talent from
other sectors of cultural production (Broadway, radio, and emerging
television). During the war, thousands of live and broadcast appearances
of ``Hollywood personalities'' were coordinated and controlled by
film-industry producers, starting in 1942, a year before Smith's radio
marathon.\footnote{Sue Collins, ``Star Testimonies: World War and the
  Cultural Politics of Authority,'' in \emph{Cinema's Military
  Industrial Complex}, eds. Haidee Wasson and Lee Grieveson (Berkeley:
  University of California Press, 2018).} Although Merton et al.'s study
does consider the fact of other celebrities selling bonds, the
observation that such an ``anomaly'' indicated that an entertainer ``can
take on the attributes ordinarily reserved for the moral leader''---what
the research identified as sincerity, philanthropy, and patriotism---are
indeed some of the moral attributes constituting Mary Pickford's wartime
star image, among other silent film stars who sold war bonds during the
Great War.\footnote{Merton, \emph{Mass Persuasion,} 82.}

Of course, there are important distinctions between these
socio-political and cultural contexts, and the specificity of Merton's
details, articulating both the authentication of celebrity and the
phenomenology of fandom, informed subsequent research on mass
communication, such as Horton and Wohl's theory of para-social
interaction, which highlighted the television talk show host's
construction of pseudo-intimacy between himself and his
audience.\footnote{Donald Horton and Richard R. Wohl, ``Mass
  Communication and Para-Social Interaction,'' \emph{Psychiatry} 19, no.
  3 (1956): 216. The same year, Kurt and Gladys Lang theorized
  television's impact on social distance between public figures and
  audiences in their article, ``The Television Personality in Politics:
  Some Considerations,'' \emph{Public Opinion Quarterly} 20, no. 1
  (1956). At the dawn of ``media studies,'' Joshua Meyrowitz employed
  Erving Goffman's dramaturgical model to extend the construction of
  intimacy to his ``media friends'' construct, \emph{No Sense of Place:
  The Impact of Electronic Media on Social Behavior} (New York: Oxford
  University Press, 1985). Also, the same year, Richard Schickel in his
  critique of celebrity traced the same notion to communication
  technology of the late nineteenth and early twentieth centuries, and
  in particular to the camera close-up shot, sound production, gossip
  columns, tabloids, paparazzi, and the institution of Hollywood in
  \emph{Intimate Strangers}: \emph{The Culture of Celebrity} (New York:
  Doubleday, 1986).} In this case, the televisual apparatus produces the
conditions of a ``simulacrum of conversation'' that occurs when
audiences engage the mass medium as if it were an interpersonal mode of
communication in a continuous relationship. The spectator comes to
identify herself as a fan, taking pleasure in the idea that she
``knows'' the persona more intimately than other people; ``that she
`understands' his character and appreciates his values and
motives.''\footnote{Horton and Wohl, ``Mass Communication and
  Para-Social Interaction,'' 216.} Here too, it is important to note the
precedent in film star and celebrity history on the discursive
construction of intimacy orchestrated by the rise of fan magazines and
celebrity journalism in the 1910s. The focus on stars' private lives by
fan magazines and trade press, in particular, was an industrial strategy
to authenticate stars through a sense of intimacy, even as other presses
may have exposed stars to scandal.\footnote{Lee Grieveson, ``Stars and
  Audiences in Early American Cinema,'' \emph{Screening the Past} 14
  (2002); Samantha Barbas, \emph{Movie Crazy: Fans, Stars, and the Cult
  of Celebrity} (New York: Palgrave, 2001); Richard Schickel,
  \emph{Intimate Strangers}.} If, as Thomas Elsaesser suggests, ``No
medium replaces another, or simply supersedes the previous one,'' it
would appear that stardom as a cultural construct or commodity can claim
relevance as remediation from live theater to YouTube and its prominent
social influencers.\footnote{Thomas Elsaesser, ``The New Film History as
  Media Archaeology,'' \emph{Cinémas} 14, no. 2-3 (2004): 93.}

Finally, by way of revealing my motivation to pursue this line of
argument for a more inclusive history of media and communication
studies, I would like to end by suggesting another entailment that
occurs to me: To flesh out the historical roots and cultural
significance of our YouTuber celebrity pundit, it would be useful to
(re)consider the theoretical import of opinion leadership as a form of
authority or testimonial in \emph{mediated contexts that are treated
like or confused as interpersonal ones}. Leaving aside the much
discussed ``dominant paradigm'' debate and critique of the limited
effects model, Katz and Lazarsfeld identified opinion leaders as
ordinary but key people who, when exposed to media in turn, may confirm
or shape the ideas, attitudes, and behavior of others through
interpersonal networks.\footnote{Notably, for my purpose, according to
  Everett M. Rogers, the general conception of opinion leadership can be
  attributed to Walter Lippmann, who while alluding to the idea without
  naming it in \emph{Public Opinion} (1922), influenced Edward Bernays,
  who in turn inspired Lazarsfeld's coining of the concept in 1944,
  \emph{A History of Communication Study: A Biographical Approach} (New
  York: Free Press, 1994), 287. Bernays, it turns out, did not limit the
  idea to interpersonal contexts; rather, there are various forms and
  degrees of authority held by people who mold public opinion, such as
  persons listed in \emph{Who's Who}, including ``leading theatrical or
  cinema} Forwarded at a particular moment in communication
research, opinion leadership provided cover, in a sense, from media
manipulation by situating interpersonal ``gregariousness'' into a model
of horizontal two-step flow grounded in community networks and social
conformity.\textsuperscript{25} But if we take the notion of the ordinary as a measure
or perception of authenticity and legitimacy, and if we consider the
remediation of para-social intimacy, perhaps the history and
phenomenology of fandom has more in common with two-step flow's
interpersonal networks of social conformity than researchers have
considered. If so, this might tell us something we need to know about
how confirmation bias and the construction of outrage feed the
algorithm, and thus grow the persona.







\section{Bibliography}\label{bibliography}\marginnote{producers'' and ``recognized leaders of fashion.'' In fact,
  Bernays understood in the late 1920s that politicians could capitalize
  off of their associations with famous actors, since audiences ``like
  people who amuse them,'' \emph{Propaganda} (New York: H. Liveright,
  1928), 32--33.}

\begin{hangparas}{.25in}{1} 



Barbas\marginnote{\textsuperscript{25} David W. Park, ``The Two-Step Flow vs \emph{The
  Lonely Crowd},'' in \emph{The History of Media and Communication
  Research}, eds. David W. Park and Jefferson Pooley (New York: Peter
  Lang, 2008).}, Samantha. \emph{Movie Crazy: Fans, Stars, and the Cult of
Celebrity}. New York: Palgrave, 2001.

Bernays, Edward. \emph{Propaganda}. New York: H. Liveright, 1928.

Bertellini, Giorgio. \emph{The Divo and the Duce}. Berkeley: University
of California Press, 2019.

Brinkley, Alan. \emph{Voices of Protest: Huey Long, Father Coughlin, and
the Great Depression}. New York: Alfred A. Knopf, 1982.

Cantril, Hadley, Hazel Gaudet, and Herta Herzog. \emph{The Invasion From
Mars: A Study in the Psychology of Panic}. Princeton: Princeton
University Press, 1940.

Collins, Sue. ``Star Testimonies: World War and the Cultural Politics of
Authority.'' In \emph{Cinema's Military Industrial Complex}, edited by
Haidee Wasson and Lee Grieveson, 281--304. Berkeley: University of
California Press, 2018.

Brownell, Kathryn Cramer. \emph{Showbiz Politics: Hollywood in American
Political Life}. Chapel Hill: University of North Carolina, 2014.

deCordova, Richard. \emph{Picture Personalities: The Emergence of the
Star System in America}. Urbana: University of Illinois Press, 1990.

Dyer, Richard. \emph{Stars}. London: British Film Institute, 1979.

Elsaesser, Thomas. ``The New Film History as Media Archaeology.''
\emph{Cinémas} 14, no. 2--3 (2004): 75--117.
\url{https://doi.org/10.7202/026005ar}.

Grieveson, Lee. ``Stars and Audiences in Early American Cinema.''
\emph{Screening the Past} 14 (2002): n.p. \href{http://www.screeningthepast.com/issue-14-classics-re-runs/stars-and-audiences-in-early-american-cinema}{http://www.screeningthepast.} \href{http://www.screeningthepast.com/issue-14-classics-re-runs/stars-and-audiences-in-early-american-cinema}{com/issue-14-classics-re-runs/stars-and-audiences-in-early-american-cinema}.

Herbst, Susan. ``Political Authority in a Mediated Age.'' \emph{Theory
and Society} 32, no. 4 (2003): 481--503.
\url{https://doi.org/10.1023/A:1025571226279}.

Hoopes, Roy. \emph{When the Stars Went to War: Hollywood and World War
II}. New York: Random House, 1994.

Horton, Donald, and Richard R. Wohl. ``Mass Communication and
Para-Social Interaction.'' \emph{Psychiatry} 19, no. 3 (1956): 215--29.
\url{https://doi.org/10.1080/00332747.1956.11023049}.

Jenkins, Henry. \emph{What Made Pistachio Nuts? Early Sound Comedy and
the Vaudeville Aesthetic}. New York: Columbia University Press, 1992.

Katz, Elihu, and Paul F. Lazarsfeld. \emph{Personal Influence: The Part
Played by People in the Flow of Mass Communications}. Glencoe, IL: Free
Press, 1955.

Koppes, Clayton R., and Gregory D. Black. \emph{Hollywood Goes to War:
How Politics, Profits, and Propaganda Shaped World War II Movies}.
Berkeley: University of California Press, 1990.

Lang, Kurt, and Gladys Lang. ``The Television Personality in Politics:
Some Considerations.'' \emph{Public Opinion Quarterly} 20, no. 1 (1956):
103--12. \url{http://www.jstor.org/stable/2746556}.

Lazarsfeld, Paul F., Bernard Berelson, and Hazel Gaudet. \emph{The
People's Choice: How the Voter Makes Up His Mind in a Presidential
Campaign}. New York: Duell, Sloan and Pearce, 1944.

Lilti, Antoine. \emph{The Invention of Celebrity, 1750--1850}. Malden,
MA: Polity, 2017.

Maltby, Richard. ``How Can Cinema History Matter More?'' \emph{Screening
the Past} 22 (2007): n.p.
\url{http://www.screeningthepast.com/issue-22-tenth-anniversary/how-can-cinema-history-matter-more/}.

Marshall, P. David. \emph{Celebrity and Power: Fame in Contemporary
Culture}. Minneapolis: University of Minnesota Press, 1997.

McArthur, Benjamin. \emph{Actors and American Culture, 1880--1920}.
Philadelphia: Temple University Press, 1984.

Merton, Robert K., with Marjorie Fiske and Alberta Curtis. \emph{Mass
Persuasion: The Social Psychology of a War Bond Drive}. New York: Howard
Fertig, 1946.

Meyrowitz, Joshua. \emph{No Sense of Place: The Impact of Electronic
Media on Social Behavior}. New York: Oxford University Press, 1985.

Nerone, John. ``Introduction: Mapping the Field of Media History.'' In
\emph{The International Encyclopedia of Media Studies: Media History and
the Foundations of Media Studies}, edited by Angharad N. Valdivia and
John Nerone, 1--17. Oxford: Wiley-Blackwell, 2013.

Novak, Phillip. \emph{Interpretation and Film Studies: Movie Made
Meanings}. Cham, Switzerland: Palgrave Macmillian, 2020.

Park, David W. ``The Two-Step Flow vs The Lonely Crowd.'' In \emph{The
History of Media and Communication Research: Contested Memories}, edited
by David W. Park and Jefferson Pooley, 251--65. New York: Peter Lang,
2008.

Ponce de Leon, Charles L. \emph{Self-exposure: Human Interest Journalism
and the Emergence of Celebrity in America, 1890--1940}. Chapel Hill:
University of North Carolina Press, 2002.

Rakow, Lana F. ``Feminist Historiography and the Field: Writing New
Histories.'' In \emph{The History of Media and Communication Research:
Contested Memories}, edited by David W. Park and Jefferson Pooley,
113--140. New York: Peter Lang, 2008.

Rogers, Everett M. \emph{A History of Communication Study: A
Biographical Approach}. New York: Free Press, 1994.

Schatz, Thomas. \emph{Boom and Bust: The American Cinema in the 1940s}.
Berkeley: University of California Press, 1999.

Sennett, Richard. \emph{The Fall of Public Man: On the Social Psychology
of Capitalism}. New York: Knopf, 1977.

Schickel, Richard. \emph{Intimate Strangers: The Culture of Celebrity}.
New York: Doubleday, 1986.

Simonson, Peter. ``Celebrity, Public Image, and American Political Life:
Rereading Robert K. Merton's \emph{Mass Persuasion}.'' \emph{Political
Communication} 23, no. 3 (2006): 271--84.
\url{https://doi.org/10.1080/10584600600808794}.

Staiger, Janet. ``The Future of the Past.'' \emph{Cinema Journal} 44,
no. 1 (2004): 126--29. \url{https://doi.org/10.1353/cj.2004.0054}.

Susman, Warren I. ``\,`Personality' and the Making of Twentieth-Century
Culture.'' \emph{Culture as History: The Transformation of American
Society in the Twentieth Century}, 271--85. New York: Pantheon, 1984.

Vorderer, Peter, David W. Park, and Sarah Lutz. ``A History of Media
Effects Research Traditions." In \emph{Media Effects: Advances in Theory
and Research}. edited by Mary Beth Oliver, Arthur A. Raney, and Jennings
Bryant, 1--15. 4th ed. New York: Routledge, 2020.



\end{hangparas}


\end{document}