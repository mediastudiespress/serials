% see the original template for more detail about bibliography, tables, etc: https://www.overleaf.com/latex/templates/handout-design-inspired-by-edward-tufte/dtsbhhkvghzz

\documentclass{tufte-handout}

%\geometry{showframe}% for debugging purposes -- displays the margins

\usepackage{amsmath}

\usepackage{hyperref}

\usepackage{fancyhdr}

\usepackage{hanging}

\hypersetup{colorlinks=true,allcolors=[RGB]{97,15,11}}

\fancyfoot[L]{\emph{History of Media Studies}, vol. 1, 2021}


% Set up the images/graphics package
\usepackage{graphicx}
\setkeys{Gin}{width=\linewidth,totalheight=\textheight,keepaspectratio}
\graphicspath{{graphics/}}

\title[Communication Studies in Argentina in the 1960s and ’70s]{Communication Studies in Argentina in the 1960s and ’70s: Specialized Knowledge and Intellectual Intervention Between the Local and the Global} % longtitle shouldn't be necessary

% The following package makes prettier tables.  We're all about the bling!
\usepackage{booktabs}

% The units package provides nice, non-stacked fractions and better spacing
% for units.
\usepackage{units}

% The fancyvrb package lets us customize the formatting of verbatim
% environments.  We use a slightly smaller font.
\usepackage{fancyvrb}
\fvset{fontsize=\normalsize}

% Small sections of multiple columns
\usepackage{multicol}

% Provides paragraphs of dummy text
\usepackage{lipsum}

% These commands are used to pretty-print LaTeX commands
\newcommand{\doccmd}[1]{\texttt{\textbackslash#1}}% command name -- adds backslash automatically
\newcommand{\docopt}[1]{\ensuremath{\langle}\textrm{\textit{#1}}\ensuremath{\rangle}}% optional command argument
\newcommand{\docarg}[1]{\textrm{\textit{#1}}}% (required) command argument
\newenvironment{docspec}{\begin{quote}\noindent}{\end{quote}}% command specification environment
\newcommand{\docenv}[1]{\textsf{#1}}% environment name
\newcommand{\docpkg}[1]{\texttt{#1}}% package name
\newcommand{\doccls}[1]{\texttt{#1}}% document class name
\newcommand{\docclsopt}[1]{\texttt{#1}}% document class option name


\begin{document}

\begin{titlepage}

\begin{fullwidth}
\noindent\LARGE\emph{Launch essay
} \hspace{85mm}\includegraphics[height=1cm]{logo3.png}\\
\noindent\hrulefill\\
\vspace*{1em}
\noindent{\Huge{Communication Studies in Argentina in the 1960s and ’70s: Specialized Knowledge and Intellectual Intervention Between the Local and the Global\par}}

\vspace*{1.5em}

\noindent\LARGE{Mariano Zarowsky\par}\marginnote{\emph{Mariano Zarowsky, ``Communication Studies in Argentina in the 1960s and ’70s: Specialized Knowledge and Intellectual Intervention Between the Local and the Global,'' \emph{History of Media Studies} 1 (2021), \href{https://doi.org/10.32376/d895a0ea.42a0a7aa}{https://doi.org/ 10.32376/d895a0ea.42a0a7aa}.} \vspace*{0.75em}}
\vspace*{0.5em}
\noindent{{\large\emph{Universidad de Buenos Aires}, \href{mailto:zarowskymariano@gmail.com}{zarowskymariano@gmail.com}\par}} \marginnote{\href{https://creativecommons.org/licenses/by-nc/4.0/}{\includegraphics[height=0.5cm]{by-nc.png}}}

% \vspace*{0.75em} % second author

% \noindent{\LARGE{<<author 2 name>>}\par}
% \vspace*{0.5em}
% \noindent{{\large\emph{<<author 2 affiliation>>}, \href{mailto:<<author 2 email>>}{<<author 2 email>>}\par}}

% \vspace*{0.75em} % third author

% \noindent{\LARGE{<<author 3 name>>}\par}
% \vspace*{0.5em}
% \noindent{{\large\emph{<<author 3 affiliation>>}, \href{mailto:<<author 3 email>>}{<<author 3 email>>}\par}}

\end{fullwidth}

\vspace*{1em}


\newthought{The emergence of} communication studies in Latin America in the 1960s and
'70s offers a productive viewpoint to think historically about social
processes of emergence and consolidation of specialized discourses and
knowledge. This essay aims to present some of the concepts and questions
that have guided my research on the intellectual history of
communication studies in Argentina. The concept of \emph{communication
intellectuals}, along with those of \emph{intellectual traditions} and
\emph{trajectories,} offers valuable routes to address relationships
between the production of specialized knowledge and broader social
processes. In peripheral countries such as Argentina, it is also
inescapable to ask about the global and transnational flow of ideas: How
do the local and the global interact in the production of knowledge? How
do we approach these relationships? Although this essay systematizes
concepts and conclusions that refer to precise spatiotemporal
coordinates, it also offers a more general reflection on ways of
thinking and writing the intellectual history of communication studies
as a contribution to its development in other regions and comparative
analyses thereof.

At the beginning of the 1960s, a set of discourses began to crystallize
in Argentina that took communication, mass media, and culture as a field
of knowledge problems to be defined and legitimized. Those who advanced
these discourses established credentials to inter-

\enlargethispage{2\baselineskip}

\vspace*{2em}

\noindent{\emph{History of Media Studies}, vol. 1, 2021}




 \end{titlepage}



\noindent vene in public debates
and created distinct spaces of intellectual production and
dissemination. Part of a broader reorganization of social and cultural
hierarchies, they drew attention to new issues and developed new
theoretical tools and ways of understanding the relations between
intellectuals and society---all of which would shape the emergence of
communication studies in the country. Here I refer to the reflections of
Jaime Rest (1927--1979) on the relations between mass culture, popular
culture, and media technologies in the late 1950s; to the work of Eliseo
Verón (1935­--2014) and Oscar Masotta (1930--1979) at the crossroads of
sociology, structuralist linguistics, and psychoanalysis; to the bridges
that Héctor Schmucler (1931--2018) was able to build between leftist
militancy, editorial praxis, and knowledge production on the pages of
the journals \emph{Pasado y Presente} (1963--1965), \emph{Los Libros}
(1969­1976), and \emph{Comunicación y Cultura} (1973--1985); to the
blend of literary criticism, reflection on popular culture, and
editorial activity that Aníbal Ford (1934­--2009) engaged in while
working at the publisher Centro Editor de América Latina and the journal
\emph{Crisis} (1973­--1976); to the formulation of a political economy
of communication by Heriberto Muraro (b. 1937) in the early 1970s; and
to the political and epistemic debate that, during his exile in Mexico,
Schmucler himself, together with Nicolás Casullo (1944­--2008) and
Sergio Caletti (1947­--2015), led in the pages of the journals
\emph{Controversia} (1979­­--1981) and \emph{Comunicación y
Cultura}.\footnote{Mariano Zarowsky, \emph{Los estudios en comunicación
  en la Argentina: Ideas, intelectuales, tradiciones político-culturales
  (1956-1985)} (Buenos Aires: Eudeba, 2017).}

The notion of \emph{communication intellectuals} is a productive
framework for thinking about these processes and historical
figures\emph{.} It does not refer to a group defined by its thematic or
disciplinary specialization, but to intellectual contemporaries who
located their own conditions and novel field of action in the space they
carved out between a new type of theoretical problematic and political
intervention. Investigating links between communication, culture, and
technology; mass mediated messages and ideologies; cultural industries
and popular cultures; collective action and social meanings; and between
media and the reproduction or transformation of the social order; the
communication intellectuals were projected as public figures because of
their ability to give their research social, cultural, and eventually,
political significance. The notion of communication intellectuals is
productive to investigate how, in the 1960s and 1970s, thinking about
communication in Argentina intervened in national debates and dilemmas.
It sheds light on the ways in which these new modes of thought and
specific social practices (university teaching, research, publishing
magazines and books) represented an intervention into cultural
issues---especially left-wing culture---and into more general political
disputes within the context of an intense and dizzying process of
hegemonic reconfiguration. From this perspective, the intellectual
history of communication in Argentina places us at the intersection of
two fields of problems. In an \emph{epistemic dimension}, it confronts
us with the question of the social conditions of the production of
knowledge about the social. In a \emph{socio-historical dimension}, the
trajectories of communication intellectuals leads us to broader
movements of culture and politics: We are dealing with actors and
discourses that participated in a process of modernization and
theoretical renewal that unfolded simultaneously with a sensitivity to
radical change and a strong impulse towards public intervention.

To speak of a field of knowledge and discourses on communication and
culture implies studying a \emph{process of formation} rather than
starting from pre-existing entities. Tracing a genealogy of this
\emph{formation of discourses}---which in Argentina took the form,
\emph{a posteriori}, of an area of specialized knowledge on
communication---means exploring the spectrum of transformations,
variants, and borrowings that shaped it and impacted an open array of
relationships with other disciplines and emerging issues in the cultural
and political fields. The intellectual and cultural history of
communication studies in Argentina places us in a zone of crossings and
porous borders: between knowledge from different disciplines,
intellectual traditions, and theoretical currents (psychoanalysis,
sociology, literary criticism, anthropology, linguistics, Marxism,
structuralism); between the question of the scientific status of
knowledge about the social and openings into a broader culture oriented
toward change; and between the local scene and the transnational realm
in which ideas are produced, circulated, and taken up.

Thinking about a culture in a specific national configuration implies,
following Raymond Williams, accounting for the coexistence of
heterogeneous, interacting elements, dominant and emergent, but also
\emph{traditions} arising from the past that are updated and recovered
as active social forces.\footnote{Raymond Williams, \emph{Marxismo y
  literatura} (Buenos Aires: Las Cuarenta, 2009).} More than a
succession of stages or currents, the notion of \emph{intellectual
traditions} allows us to capture the heterogeneity within an era and,
above all\emph{,} the intermixing and borrowings among the various
formations that organized the communication intellectuals in Argentina
during the 1960s and 1970s. Always in a state of dissolving, sometimes
buried, other times on their way to being reformulated and updated (as
they are always selective), the notion of intellectual traditions allows
us to recognize both the common elements and the lines of
differentiation among various formations, thus illuminating the textures
and heterogeneous edges that modulated a cultural dynamic and its
movements.\footnote{For an analysis that uses Raymond Williams's
  cultural materialism to define traditions as the material contexts for
  the production of ideas and intellectual interventions, see Carlos
  Altamirano, \emph{Intelectuales: Notas de investigación} (Buenos
  Aires: Norma, 2006), 127--29.} Intellectual traditions operate as
disciplinary frameworks and theoretical matrices that function as
epistemic sources; as values and political cultures in which discourses
are inscribed; and, finally, as ways of conceiving and reflexively
defining intellectual activity itself. The communication intellectuals
in Argentina during the 1960s and 1970s were thus variably humanists,
avant-gardists, and/or populists; organic and/or committed; modern
and/or scientific; Marxists, liberals, socialists and/or Peronists;
Gramscians and/or structuralists.

These intellectual traditions do not exist outside of temporal
incarnations and specific biographical contexts. It is thus fitting to
approach them as manifest through ideas, but also through the practices
and vital vicissitudes of their privileged bearers: the
subjects.\footnote{Horacio Tarcus, \emph{Marx en Argentina: Sus primeros
  lectores obreros, intelectuales y científicos} (Buenos Aires: Siglo
  XXI, 2007), 53.} The notion of \emph{intellectual itinerary} proves to
be extremely productive in reconstructing these processes, where the
work of thought is linked to and unfolds within the specificity of
historical experiences. Thus, as opposed to biographical histories that
are constructed through the prism of what Pierre Bourdieu called forms
of biographical ``illusion,''\footnote{Pierre Bourdieu, ``La ilusión
  biográfica,'' in \emph{Razones prácticas: Sobre la teoría de la
  acción}, trans. Thomas Kauf (Barcelona: Anagrama, 1997).} the notion
of intellectual itinerary, as François Dosse writes, ``makes it possible
to encounter what was the present of the subject of a biography in its
indeterminacy and ignorance.''\footnote{François Dosse, \emph{La marcha
  de las Ideas: Historia de los intelectuales. Historia intelectual}
  (Valencia: Publicaciones de la Universidad de Valencia, 2006), 46.}
That is, in contrast to the traditional (linear and cumulative) history
of the disciplines, it makes it possible to approach the cuts and
discontinuities of which they are made. Finally, the notion of
intellectual itineraries allows us to highlight the links that connect
the formation of disciplines with broader socio-cultural networks.

Is the study of intellectual trajectories and traditions as connected
with broader formations (beyond the strictly disciplinary) only
productive in analyzing the early formation of a discipline, with its
more fragile and porous boundaries with other knowledge and social
spaces---or is it also useful for the study of more professionalized
fields? In the same sense, were the connections between knowledge
production and political practice a Latin American particularity of the
second half of the twentieth century, or did they also mark other
moments and geographies? There is no doubt that, in the 1960s and 1970s,
Argentina---and Latin America in general⸺ experienced a moment of
profound social transformation, intense political and cultural upheaval,
and renewal of cultural and university institutions. This macro-social
laboratory shaped artistic, academic, and intellectual practices and
helped produce creative and original forms of knowledge in a number of
specialized fields, communication studies among them.\footnote{On Chile
  as a "laboratory" for the field of communication in the 1970s, see
  Mariano Zarowsky, \emph{Del Laboratorio chileno a la
  comunicación-mundo: Un itinerario intelectual de Armand Mattelart}
  (Buenos Aires: Biblos, 2013).}

Correspondingly, it would be useful to think comparatively about the
history of communication studies in Argentina and Latin America and the
history of other schools or currents that also emerged in dense or
agitated moments of history and shaped the field on a global scale. I am
thinking, among other examples, of the Frankfurt School in its moment of
emergence and then in its exile in the 1930s and 1940s, or the
Birmingham Cultural Studies circle in the 1960s and 1970s. A red thread
binds these experiences---the one that unites social movement and the
question of the dominated or subaltern, of their condition and destiny.
The figure of the intellectual, exiled for political reasons, condenses
in the 20th century as a sort of ``epistemological privilege'' of the
critical tradition: a character in transit, the exile sees through the
prism of the world that abandons the society that shelters him, and with
the prism of the new world, the society he leaves behind, gaining new
perspectives. He does so from the standpoint of history's defeated,
carrying on his shoulders a vital question: How did it come to this?
Exile thus favors a kind of estrangement effect, with productive
epistemic consequences.\footnote{On the ``epistemological privilege'' of
  exile and the point of view of the ``vanquished of history'' in the
  critical tradition, see Enzo Traverso, \emph{La historia como campo de
  batalla: Interpretar las violencias del siglo XX} (Buenos Aires: Fondo
  de Cultura Económica, 2012); and Traverso, \emph{Melancolía de
  izquierda. Marxismo, historia y memoria} (Buenos Aires: Fondo de
  Cultura Económica, 2018). For an analysis of exile in the specific
  case of communication studies in Argentina, see Mariano Zarowsky,
  ``Del exilio a los nuevos paradigmas: los intelectuales argentinos de
  la comunicación en México (de \emph{Controversia} a \emph{Comunicación
  y cultura}),'' \emph{Comunicación y sociedad} 12\emph{,} no. 24
  (2015); and Zarowsky, \emph{Del laboratorio chileno a la
  comunicación-mundo}.}

How can we study the interactions between the local and the global,
which are so pronounced in peripheral countries such as Argentina?
Knowledge about communication in Argentina and Latin America was largely
configured out of regional articulations interacting with global
processes. The connections between researchers in Argentina and Chile in
the 1960s and '70s and the contemporaneous but \emph{differentiated}
circulation of structuralism and French semiology in both countries is a
good example, as Eliseo Verón has previously observed.\footnote{Eliseo
  Verón, ``Acerca de la producción social del conocimiento: el
  'estructuralismo' y la semiología en Argentina y Chile,''
  \emph{Lenguajes,} 1 (April 1974).} The examination of this interaction
then requires work on different scales, where the study of \emph{local
scenarios} proves vital: It is a matter of highlighting the specific
debates, national intellectual traditions, and local conjunctures within
which specialized discourses on communication emerged and acquired their
prominence and particularity. This does not imply ignoring other
dimensions. On the contrary, this ``national cut,'' as José Aricó calls
it,\footnote{José Aricó proposes this ``national cut'' in his study of
  the travels of Antonio Gramsci's thought in Latin America. He argues
  that to approach this ``geography'' an ``inversion of the terms'' is
  required in ``conditions of showing the existing connections between
  processes of reality and processes of the elaboration of theory.'' He
  adds: ``a deeper inquiry into the demands of reality that movements in
  society and culture carry with them when they appropriate his
  {[}Gramsci's{]} reflections implies dilating our search to such
  extremes that only a `national' cut makes it possible.'' José Aricó,
  \emph{La cola del diablo: Itinerario de Gramsci en América Latina}
  (Buenos Aires: Siglo XXI, 2005), 43.} helps us better gauge how local
practices of knowledge production interact with the international
circulation of ideas. It was through specific activities in local
society and culture that the communication intellectuals in Argentina
linked themselves to international flows of ideas, making original
appropriations and contributions. It is in the \emph{interaction}
between the global and the local, in short, that knowledge about the
social is produced; and it is in active \emph{appropriations} and local
\emph{uses} that new knowledge is generated with materials drawn from
other contexts of production.\footnote{Federico Neiburg and Mariano
  Plotkin, ``Intelectuales y expertos: Hacia una sociología histórica de
  la producción del conocimiento sobre la} Situating ourselves in this space of
interaction also allows us to highlight the ways in which the capacity
to manage transnational flows of information and knowledge helps to
create hierarchies and organize positions within the peripheral academic
and intellectual field.\textsuperscript{12}

The history of communication studies in Argentina represents, in short,
a productive entry point to analyze social processes of constructing
knowledge about the social and, more broadly, to address the
relationships among intellectuals, culture, and politics in the country\marginnote{sociedad en la Argentina,'' in
  \emph{Intelectuales y expertos: La constitución del conocimiento
  social en la argentina,} eds. Federico Neiburg and Mariano Plotkin
  (Buenos Aires: Paidós, 2004).}
and\marginnote{\textsuperscript{12} Mariano Plotkin and Eduardo Zimmermann,
  ``Introducción, saberes de Estado en la Argentina, siglos XIX y XX,''
  in \emph{Los saberes del Estado,} eds. Mariano Plotkin and Eduardo
  Zimmermann (Buenos Aires: Edhasa, 2012), 20--21.} region in the 1960s and '70s. The itinerary of communication
intellectuals in Argentina can be read as a chapter of the country's
recent history. At the same time, in this essay, from the perspective
that guided my research and the conclusions I reached there, I've made
suggestions and raised questions about the intellectual history of
communication studies as a field which I hope might open up
possibilities for inquiry in and about other locations.







\section{Bibliography}\label{bibliography}

\begin{hangparas}{.25in}{1} 



Altamirano, Carlos. \emph{Intelectuales: Notas de investigación.} Buenos
Aires: Norma, 2006.

Aricó, José. \emph{La cola del diablo: Itinerario de Gramsci en América
Latina}. Buenos Aires: Siglo XXI, 2005.

Bourdieu, Pierre. ``La ilusión biográfica.'' In \emph{Razones prácticas:
Sobre la teoría de la acción,} translated by Thomas Kauf, 74--83.
Barcelona: Anagrama, 1997.

Dosse, François. \emph{La marcha de las ideas: Historia de los
intelectuales. Historia intelectual.} Valencia: Publicaciones de la
Universidad de Valencia, 2006.

Neiburg, Federico, and Mariano Plotkin, ``Intelectuales y expertos:
Hacia una sociología histórica de la producción del conocimiento sobre
la sociedad en la argentina.'' In \emph{Intelectuales y expertos: La
constitución del conocimiento social en la argentina,} edited by
Federico Neiburg and Mariano Plotkin, 15--30. Buenos Aires: Paidós,
2004.

Plotkin, Mariano, and Eduardo Zimmermann, ``Introducción, saberes de
Estado en la Argentina, siglos XIX y XX.'' In \emph{Los saberes del
Estado,} edited by Mariano Plotkin and Eduardo Zimmermann, 9--28\emph{.}
Buenos Aires: Edhasa, 2012.

Tarcus, Horacio. \emph{Marx en Argentina: Sus primeros lectores obreros,
intelectuales y científicos.} Buenos Aires: Siglo XXI, 2007.

Traverso, Enzo. \emph{La historia como campo de batalla: Interpretar las
violencias del siglo} XX. Buenos Aires: Fondo de Cultura Económica,
2012.

---------. \emph{Melancolía de izquierda: Marxismo, historia y memoria.}
Buenos Aires: Fondo de Cultura Económica, 2018.

Verón, Eliseo. ``Acerca de la producción social del conocimiento: el
`estructuralismo' y la semiología en Argentina y Chile.''
\emph{Lenguajes} 1 (1974): 96--126.

Williams, Raymond. \emph{Marxismo y literatura.} Buenos Aires: Las
Cuarenta, 2009.

Zarowsky, Mariano. \emph{Los estudios en comunicación en la Argentina:
Ideas, intelectuales, tradiciones político-culturales (1956--1985).}
Buenos Aires: Biblos, 2017.

\_\_\_\_. ``Del exilio a los nuevos paradigmas: los intelectuales
argentinos de la comunicación en México (de \emph{Controversia} a
\emph{Comunicación y cultura}).'' \emph{Comunicación y sociedad}
12\emph{,} no. 24 (2015): 127--60.

\_\_\_\_. \emph{Del laboratorio chileno a la comunicación mundo: Un
itinerario intelectual de Armand Mattelart.} Buenos Aires: Biblos, 2013.



\end{hangparas}


\end{document}