% see the original template for more detail about bibliography, tables, etc: https://www.overleaf.com/latex/templates/handout-design-inspired-by-edward-tufte/dtsbhhkvghzz

\documentclass{tufte-handout}

%\geometry{showframe}% for debugging purposes -- displays the margins

\usepackage{amsmath}

\usepackage{hyperref}

\usepackage{fancyhdr}

\usepackage{hanging}

\hypersetup{colorlinks=true,allcolors=[RGB]{97,15,11}}

\fancyfoot[L]{\emph{History of Media Studies}, vol. 1, 2021}


% Set up the images/graphics package
\usepackage{graphicx}
\setkeys{Gin}{width=\linewidth,totalheight=\textheight,keepaspectratio}
\graphicspath{{graphics/}}

\title[The Role of Theory Groups in the Lives of Ideas]{The Role of Theory Groups in the Lives of Ideas} % longtitle shouldn't be necessary

% The following package makes prettier tables.  We're all about the bling!
\usepackage{booktabs}

% The units package provides nice, non-stacked fractions and better spacing
% for units.
\usepackage{units}

% The fancyvrb package lets us customize the formatting of verbatim
% environments.  We use a slightly smaller font.
\usepackage{fancyvrb}
\fvset{fontsize=\normalsize}

% Small sections of multiple columns
\usepackage{multicol}

% Provides paragraphs of dummy text
\usepackage{lipsum}

% These commands are used to pretty-print LaTeX commands
\newcommand{\doccmd}[1]{\texttt{\textbackslash#1}}% command name -- adds backslash automatically
\newcommand{\docopt}[1]{\ensuremath{\langle}\textrm{\textit{#1}}\ensuremath{\rangle}}% optional command argument
\newcommand{\docarg}[1]{\textrm{\textit{#1}}}% (required) command argument
\newenvironment{docspec}{\begin{quote}\noindent}{\end{quote}}% command specification environment
\newcommand{\docenv}[1]{\textsf{#1}}% environment name
\newcommand{\docpkg}[1]{\texttt{#1}}% package name
\newcommand{\doccls}[1]{\texttt{#1}}% document class name
\newcommand{\docclsopt}[1]{\texttt{#1}}% document class option name


\begin{document}

\begin{titlepage}

\begin{fullwidth}
\noindent\LARGE\emph{Launch essay
} \hspace{85mm}\includegraphics[height=1cm]{logo3.png}\\
\noindent\hrulefill\\
\vspace*{1em}
\noindent{\Huge{The Role of Theory Groups in the Lives of Ideas\par}}

\vspace*{1.5em}

\noindent\LARGE{Wendy Leeds-Hurwitz}\par}\marginnote{\emph{Wendy Leeds-Hurwitz, ``The Role of Theory Groups in the Lives of Ideas,'' \emph{History of Media Studies} 1 (2021). \href{https://doi.org/10.32376/d895a0ea.0b35e36e}{https://doi.org/10.32376/d895a0ea.0b35e36e}} \vspace*{0.75em}}
\vspace*{0.5em}
\noindent{{\large\emph{University of Wisconsin-Parkside}, \href{mailto:wendy.leeds.hurwitz@gmail.com}{wendy.leeds.hurwitz@gmail.com}\par}} \marginnote{\href{https://creativecommons.org/licenses/by-nc/4.0/}{\includegraphics[height=0.5cm]{by-nc.png}}}

% \vspace*{0.75em} % second author

% \noindent{\LARGE{<<author 2 name>>}\par}
% \vspace*{0.5em}
% \noindent{{\large\emph{<<author 2 affiliation>>}, \href{mailto:<<author 2 email>>}{<<author 2 email>>}\par}}

% \vspace*{0.75em} % third author

% \noindent{\LARGE{<<author 3 name>>}\par}
% \vspace*{0.5em}
% \noindent{{\large\emph{<<author 3 affiliation>>}, \href{mailto:<<author 3 email>>}{<<author 3 email>>}\par}}

\end{fullwidth}

\vspace*{1em}


\newthought{We tend to} take for granted that good ideas are, and should be, accepted
on their merits. However, it turns out that having a good idea, while
certainly an essential beginning point, is no more than that. Ideas
exist only insofar as they are put forward by people (who work in
socially structured systems of recognition, reward, and power, of
course), and so the people who invent, adopt, promote, or expand ideas
turn out to be critically important to the study of the history of
ideas. As I explained a few years ago, ``Good ideas are not sufficient;
neither self-generating nor self-sustaining, ideas do not arise \emph{de
nova} (from nowhere, out of nothing), and they rarely are accepted
simply because someone recognized them as intrinsically
valuable.''\footnote{Wendy Leeds-Hurwitz, ``The Emergence of Language and Social
  Interaction Research as a Specialty,'' in \emph{The Social History of
  Language and Social Interaction Research}, ed. Wendy Leeds-Hurwitz
  (Cresskill: Hampton Press, 2010), 4.
} But what else is
required if good ideas are to be accepted? The short answer is that each
idea needs at least two specific types of people, as well as some sort
of ``theory group'' within which these people can share and promote the
idea. In what follows, I will briefly summarize theory groups, and then
mention  a few implications.

Theory groups were initially outlined nearly fifty years ago by Belver
C. Griffith and Nicholas C. Mullins, so they are not particularly
new.\footnote{Griffith and Mullins, ``Coherent Social Groups in Scientific Change:
  `Invisible Colleges' May Be Consistent Throughout
  Science,''~\emph{Science}~177, no. 4053 (1972); Mullins, ``The
  Development of a Scientific Specialty: The Phage Group and the Origins
  of Molecular Biology,'' \emph{Minerva} 10 (1972); and Mullins,
  \emph{Theories and Theory Groups in Contemporary American Sociology}
  (New York: Harper
} Yet a quick search shows no
one talking about theory groups in the major communication journals, and
no relevant citations in the \emph{History of Communication Research
Bibliography,} so they require explanation if they are to attract the
attention they deserve from communication
researchers.\textsuperscript{3} Griffith and Mullins
examined theory groups within sociology, emphasizing a series of
developmental stages:


\enlargethispage{2\baselineskip}

\vspace*{2em}

\noindent{\emph{History of Media Studies}, vol. 1, 2021}




 \end{titlepage}




\begin{itemize}
\item
  Normal (intellectual leader appears)\marginnote{\& Row, 1973). ``Invisible colleges'' is an earlier
  and related term referring to a group of researchers who are not based
  at the same institution, yet who share research informally within the
  group, rather than only through more formal conference presentations
  and publications available to everyone. Diana Crane, \emph{Invisible
  Colleges: Diffusion of Knowledge in Scientific Communities} (Chicago:
  University of Chicago Press, 1972). An example would be scholars who
  were graduate students together, and who still share informal
  descriptions of their research and findings with one another, despite
  having moved to different institutions, and having no obvious, public,
  visible connection. They are invisible because their connections will
  not be known to everyone, whereas colleagues working in the same
  department are more obviously visible as part of a single group.
  Invisible colleges are often most easily tracked through
  acknowledgments or overlapping lists of citations because members
  often review one another's drafts of research reports prior to
  publication. The concept of invisible colleges is useful, but need not
  concern us further here, given that it has already made it into the
  major communication journals, albeit not terribly frequently, but
  enough so it is clearly not a new concept. See also Randall Collins
  for another approach to what he terms ``the social production of
  ideas,'' this time with a grand overview of world history. Like Crane,
  this is an approach which has already been noticed, at least to some
  extent, within communication, and so it does not require so much
  introduction to readers. Collins, \emph{The Sociology of Philosophies}
  (Cambridge: Harvard University Press, 1998), chap. 15; and Collins,
  ``\emph{The Sociology of Philosophies}: A Precis,''~\emph{Philosophy
  of the Social Sciences}~30, no. 2 (2000): 195.}
\item
  Network (organizational leader appears)\marginnote{\textsuperscript{3} See ``History of Communication Research Bibliography,'' Annenberg
  School for Communication Library Archives, accessed September 14,
  2021,~\url{https://ascla.asc.upenn.edu/communications-scholars-history-project/bibliography/}.
}
\item
  Cluster (founders now have students, with 7--25 members in the central
  group)
\item
  Specialty (textbook appears, work is recognized, with 20--100+
  members)
\end{itemize}

Over the next few decades, Stephen O. Murray adapted and expanded that
research, stressing what was needed for groups of people to share ideas,
and for others to take them up, this time using anthropology as the
discipline providing examples.\setcounter{footnote}{3}\footnote{Stephen O. Murray, \emph{Group Formation in Social Science} (Edmonton:
  Linguistic Research, 1983); Murray, \emph{Theory Groups and the Study
  of Language in North America: A Social History} (Amsterdam: John
  Benjamins, 1994); Murray, \emph{American Sociolinguistics: Theorists
  and Theory}
} My
concern here will be with the value of theory groups for understanding
the disciplinary history of communication, because I think we have too
often ignored the role of theory groups in the lives of ideas.

A theory group is typically explained as a loose amalgam of individuals,
held together by a shared theoretical orientation (rather than a group
based at the same institution), requiring the leadership (both
intellectual and organizational) of one or several individuals, with
colleagues, students, or followers to carry their ideas forward. Ideas
can only exist through continued attention, so the study of theory
groups becomes the study of who invents and attends to particular ideas,
sharing them with others, encouraging their development and expansion.
Ideas can be promoted, shared, and become taken for granted; they can
equally easily be ignored, forgotten, discredited, and thus disappear
from communal memory. As Murray puts it, ``The fate of ideas can be
considered as depending on social processes rather than the intrinsic
merit of the ideas.''\textsuperscript{5} Typically,
scholars of communication, as in many other disciplines, emphasize
ideas, often to the exclusion of the people who develop and discuss
them. (Of course, this is not as true of historians of the field, who
have often examined both people and contexts.) My argument here is that
more of us need more often attend to the context of ideas---that is, the
people surrounding and supporting them.

Assuming that Griffith and Mullins are correct, theory groups are
essential to our study of ideas, and therefore should form part of any
research into disciplinary history. Assuming Murray is correct, we must
ask: What exactly needs to happen for an idea to be taken up by a group?
To arrive at the answer, we need to understand how theory groups work
generally, and also to study their function within specific disciplinary
contexts. People develop ideas, and share them (or not), so it is
impossible to study ideas without also considering the people who invent
them and elaborate upon them (typically by teaching their students,
discussing with colleagues or at conferences, or publishing to reach a
still larger group). Ideas have no independent existence---they cannot
survive\marginnote{\emph{Groups} (Amsterdam: John Benjamins, 1998); Murray,
  \emph{American Anthropology and Company: Historical Explorations}
  (Lincoln: University of Nebraska Press, 2013); and Murray et al., ``An
  Interview with Stephen O. Murray on Stephen O. Murray as Historian of
  Anthropology (and More),'' in \emph{Centering the Margins of
  Anthropology's History}, ed. Regna Darnell and Frederic W. Gleach
  (Lincoln: University of Nebraska Press, 2021).} if\marginnote{\textsuperscript{5} Murray, \emph{Theory Groups and the Study of Language}, 49.
}\setcounter{footnote}{5} people do not talk (and/or write) about them. But once
they're shared with others, either orally or in written form, then they
have a chance of spreading and being further developed. Of course, they
can still be forgotten at any point if no one continues to write about
them, read about them, and/or discuss them.

Murray examines in detail three factors, each necessary but insufficient
for the formation of a theory group: (1) good ideas, (2) intellectual
leadership, and (3) organizational leadership, across dozens of such
groups (see Figure 1).

\begin{figure}
\centering
\includegraphics{figureone.jpeg}
\textbf{\textsubscript{Figure 1: Relationship of Elements}}
\end{figure}

Good ideas, even when paired with intellectual leadership, aren't
enough; organizational leadership also turns out to be required. (The
two types of leadership can be embodied by the same person, or not.)
Murray's research demonstrates this by showing that a single recognized
scholar writing up a good idea does not always result in that idea being
adopted by others.\footnote{The issue is not necessarily a matter of holding a marginal position,
  or not getting recognition despite deserving it, but more the fact
  that there are two quite distinct criteria: being a successful
  intellectual leader, and being an effective strategic organizer for
  how, when, and where to ensure an idea is heard, accepted, and then
  applied and expanded upon.
} Either that
scholar or someone else must also be good at organization and thinking
about the multiple steps needed beyond simply having a good idea:
sharing that idea with students, organizing colloquia or conference
presentations, writing up textbooks, and developing a new journal as an
appropriate outlet, if one does not yet exist. Murray argues for the
study of how ideas are communicated by and within groups, given that
``changes in science are made by groups, not by the automatic breeding
of ideas by other ideas, nor by single individuals, however brilliant
their thoughts and research.''\footnote{Murray, \emph{Group Formation in Social Science}, 389.
}

Murray studies more elements than just the three named so far, including
the difference between a rhetoric of continuity and a rhetoric of
revolution (drawing on Kuhn),\footnote{Thomas Kuhn, \emph{The Structure of Scientific Revolutions} (Chicago:
  University of Chicago Press, 1962).
} the
difference between a claim to novelty and actual novelty, the impact of
``eliteness'' of the institutional context (Ivy League versus community
college), and the relevance of generation (faculty vs. students). He
shows that theory groups can have a single center (members being based
at the same university) or be distributed geographically (as when
students of the intellectual and/or organizational leader(s) move away
to begin their own careers at various other universities). Similarly,
monodisciplinary or interdisciplinary groups can both be successful, so
long as the organizational leadership in an interdisciplinary group is
particularly strong. This is required since advancement and prestige are
typically determined within disciplines rather than across or between
them.

What are the implications for accepting the essential role that theory
groups occupy in shaping how ideas are developed and spread in
disciplinary-history research? The central, and most obvious,
implication is that those writing disciplinary history should attend
explicitly to the theory groups putting forward new ideas, and not just
focus on the development / expansion / revision of an original idea and
its relationship to prior or later
ideas.\footnote{Some good examples specifically relevant to media studies can be found
  in David W. Park and Jefferson Pooley, eds.,~\emph{The History of
  Media and Communication Research: Contested Memories} (New York: Peter
  Lang, 2008); Peter Simonson and David W. Park, eds.,~\emph{The
  International History of Communication Study} (New York: Routledge,
  2016); and Simonson et al., eds.,~\emph{The Handbook of Communication
  History} (New York: Routledge, 2013).
} Once we recognize that
ideas only exist through the people who discuss them, then we must
expand the range of what we cover when we write the history of those
ideas. A significant part of the task must be to sort out who originated
the idea, who carried it forward, who presented it in what context, to
what reception, etc. In short, we must pay attention to the part played
by the people who gave life to the idea.

A second implication concerns the role of multidisciplinary or
interdisciplinary research on academic careers. There is often a penalty
for such projects, given that rewards typically occur within
disciplines, so those who hold joint appointments or publish across
disciplinary lines may, for example, not be given adequate recognition
for publications in journals new to the committee considering promotion
and tenure---even when this non-recognition is inadvertent. Perhaps it
is time to break down the borders between disciplines altogether. After
all, they are merely fictions we have constructed, and so we should be
deliberate about ignoring them when they prove a barrier to connections
with others sharing overlapping interests, others who might become
valuable members of our own present or future theory groups, spreading
our ideas further than we can
alone.\footnote{Wendy Leeds-Hurwitz, ``These Fictions We Call Disciplines,''
  \emph{Electronic Journal of Communication/La Revue Electronique de
  Communication} 22, no. 3--4 (2012).
}

A third implication relates to the potential fit between group
communication research, especially intra-group communication, and theory
groups. As a result, many communication scholars have directly relevant
knowledge which might be applied to the task of understanding the
communicative dynamics of theory groups beyond what scholars from other
disciplines bring. This could help assure communication a seat at the
table with other disciplinary historians (many of whom have been doing
this for far longer than we have), given that we would not just be
studying our own history but sharing the ways in which our disciplinary
lens might have value for everyone else.

Murray concludes, ``These {[}theory{]} groups are vanguard parties, not
representative of the whole population of working scientists. Over the
course of their careers, most scientists are never involved in groups
advancing new theoretical
perspectives.''\footnote{Murray, \emph{Theory Groups and the Study of Language}, 486.
} This should lead
us to reconsider not only the ways in which we conduct research into
disciplinary history, but also the ways in which we organize academia
and even socialize future generations of academics. Would it not be
beneficial for all if everyone had the opportunity to develop and work
with ground-breaking, original
ideas?\footnote{ There is no space here to examine any theory group in detail. For one
  such example, see Wendy Leeds-Hurwitz and Adam Kendon, ``The
  \emph{Natural History of an Interview} and the Microanalysis of
  Behavior in Social Interaction: A Critical Moment in Research
  Practice,'' in \emph{Holisms of Communication: The Early History of
  Audio-Visual Sequence Analysis}, ed. James McElvenny and Andrea Ploder
  (Berlin: Language Science Press, 2021), 145-200.}





\section{Bibliography}\label{bibliography}

\begin{hangparas}{.25in}{1} 



Collins, Randall. \emph{The Sociology of Philosophies: A Global Theory
of Intellectual Change}. Cambridge: Harvard University Press, 1998.

Collins, Randall. ``\emph{The Sociology of Philosophies}: A
Precis.''~\emph{Philosophy of the Social Sciences}~30, no. 2 (2000):
157--201.

Crane, Diana. \emph{Invisible Colleges: Diffusion of Knowledge in
Scientific Communities}. Chicago: University of Chicago Press, 1972.

Griffith, Belver C., and Nicholas C. Mullins. "Coherent Social Groups in
Scientific Change: `Invisible Colleges' May Be Consistent Throughout
Science."~\emph{Science}~177, no. 4053 (1972): 959--64.

``History of Communication Research Bibliography.'' Annenberg School for
Communication Library Archives. Accessed September 14,
2021.~\href{https://ascla.asc.upenn.edu/communications-scholars-history-project/bibliography/}{https://ascla.asc.upenn.edu/communications-scholars-history-project/bibliography/}.

Kuhn, Thomas. \emph{The Structure of Scientific Revolutions}. Chicago:
University of Chicago Press, 1962.

Leeds-Hurwitz, Wendy. ``The Emergence of Language and Social Interaction
Research as a Specialty.'' In \emph{The Social History of Language and
Social Interaction Research: People, Places, Ideas}, edited by Wendy
Leeds-Hurwitz, 3--60. Cresskill: Hampton Press, 2010.

Leeds-Hurwitz, Wendy. ``These Fictions We Call Disciplines.''
\emph{Electronic Journal of Communication/La Revue Electronique de
Communication} 22, no. 3--4 (2012).
\url{http://www.cios.org/EJCPUBLIC/022/3/022341.html}.

Leeds-Hurwitz, Wendy, and Adam Kendon. ``The \emph{Natural History of an
Interview} and the Microanalysis of Behavior in Social Interaction: A
Critical Moment in Research Practice.'' In \emph{Holisms of
Communication: The Early History of Audio-Visual Sequence Analysis},
edited by James McElvenny and Andrea Ploder, 145--200. Berlin: Language
Science Press, 2021.

Mullins, Nicholas C. ``The Development of a Scientific Specialty: The
Phage Group and the Origins of Molecular Biology.'' \emph{Minerva} 10
(1972), 51--82.

Mullins, Nicholas C. \emph{Theories and Theory Groups in Contemporary
American Sociology}. New York: Harper \& Row, 1973.

Murray, Stephen O. \emph{Group Formation in Social Science}. Edmonton:
Linguistic Research, 1983.

Murray, Stephen O. \emph{Theory Groups and the Study of Language in
North America: A Social History}. Amsterdam: John Benjamins, 1994.

Murray, Stephen O. \emph{American Sociolinguistics: Theorists and Theory
Groups}. Amsterdam: John Benjamins, 1998.

Murray, Stephen O. \emph{American Anthropology and Company: Historical
Explorations}. Lincoln: University of Nebraska Press, 2013.

Murray, Stephen O., Wendy Leeds-Hurwitz, Regna Darnell, Nathan
Dawthorne, and Robert Oppenheim. ``An Interview with Stephen O. Murray
on Stephen O. Murray as Historian of Anthropology (and More).'' In
\emph{Centering the Margins of Anthropology's History}, edited by Regna
Darnell and Frederic W. Gleach, 243--68. Lincoln: University of Nebraska
Press, 2021.

Park, David W., and Jefferson Pooley, eds.~\emph{The History of Media
and Communication Research: Contested Memories}. New York: Peter Lang,
2008.

Simonson, Peter, and David W. Park, eds.~\emph{The International History
of Communication Study}. New York: Routledge, 2016.

Simonson, Peter, Janice Peck, Robert T. Craig, and John P. Jackson, Jr.,
eds.~\emph{The Handbook of Communication History}. New York: Routledge,
2013.



\end{hangparas}


\end{document}