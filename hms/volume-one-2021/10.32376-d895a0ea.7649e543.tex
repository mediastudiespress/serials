% see the original template for more detail about bibliography, tables, etc: https://www.overleaf.com/latex/templates/handout-design-inspired-by-edward-tufte/dtsbhhkvghzz

\documentclass{tufte-handout}

%\geometry{showframe}% for debugging purposes -- displays the margins

\usepackage{amsmath}

\usepackage{hyperref}

\usepackage{fancyhdr}

\usepackage{hanging}

\hypersetup{colorlinks=true,allcolors=[RGB]{97,15,11}}

\fancyfoot[L]{\emph{History of Media Studies}, vol. 1, 2021}


% Set up the images/graphics package
\usepackage{graphicx}
\setkeys{Gin}{width=\linewidth,totalheight=\textheight,keepaspectratio}
\graphicspath{{graphics/}}

\title[Historia de los Estudios sobre Medios, en plural]{Historia de los Estudios sobre Medios, en plural} % longtitle shouldn't be necessary

% The following package makes prettier tables.  We're all about the bling!
\usepackage{booktabs}

% The units package provides nice, non-stacked fractions and better spacing
% for units.
\usepackage{units}

% The fancyvrb package lets us customize the formatting of verbatim
% environments.  We use a slightly smaller font.
\usepackage{fancyvrb}
\fvset{fontsize=\normalsize}

% Small sections of multiple columns
\usepackage{multicol}

% Provides paragraphs of dummy text
\usepackage{lipsum}

% These commands are used to pretty-print LaTeX commands
\newcommand{\doccmd}[1]{\texttt{\textbackslash#1}}% command name -- adds backslash automatically
\newcommand{\docopt}[1]{\ensuremath{\langle}\textrm{\textit{#1}}\ensuremath{\rangle}}% optional command argument
\newcommand{\docarg}[1]{\textrm{\textit{#1}}}% (required) command argument
\newenvironment{docspec}{\begin{quote}\noindent}{\end{quote}}% command specification environment
\newcommand{\docenv}[1]{\textsf{#1}}% environment name
\newcommand{\docpkg}[1]{\texttt{#1}}% package name
\newcommand{\doccls}[1]{\texttt{#1}}% document class name
\newcommand{\docclsopt}[1]{\texttt{#1}}% document class option name


\begin{document}

\begin{titlepage}

\begin{fullwidth}
\noindent\LARGE\emph{Launch essay
} \hspace{85mm}\includegraphics[height=1cm]{logo3.png}\\
\noindent\hrulefill\\
\vspace*{1em}
\noindent{\Huge{Historia de los Estudios sobre Medios,\\\noindent en plural\par}}

\vspace*{1.5em}

\noindent\LARGE{David W. Park}\par\marginnote{\emph{David W. Park, Jefferson Pooley, and Pete Simonson, ``Historia de los Estudios sobre Medios, en plural,'' \emph{History of Media Studies} 1 (2021), \href{https://doi.org/10.32376/d895a0ea.7649e543}{https://doi.org/ 10.32376/d895a0ea.7649e543}.} \vspace*{0.75em}}
\vspace*{0.5em}
\noindent{{\large\emph{Lake Forest College}, \href{mailto:park@lfc.edu}{park@lfc.edu}\par}} \marginnote{\href{https://creativecommons.org/licenses/by-nc/4.0/}{\includegraphics[height=0.5cm]{by-nc.png}}}

\vspace*{0.75em} 

\noindent{\LARGE{Jefferson Pooley}\par}
\vspace*{0.5em}
\noindent{{\large\emph{Muhlenberg College}, \href{mailto:pooley@muhlenberg.edu}{pooley@muhlenberg.edu}\par}}

\vspace*{0.75em} % third author

\noindent{\LARGE{Peter Simonson}\par}
\vspace*{0.5em}
\noindent{{\large\emph{University of Colorado Boulder}, \href{mailto:peter.simonson@colorado.edu}{peter.simonson@colorado.edu}\par}}

\end{fullwidth}

\vspace*{1em}

\section{Traducción de Raúl Fuentes-Navarro}

\newthought{Lanzamos la publicación} de \emph{Historia de los Estudios sobre Medios}
con una concesión que es al mismo tiempo una confesión: el objeto de la
revista, la historia de los estudios sobre medios, es poroso y
extensible, casi amorfo. La razón principal es que los ``estudios sobre
medios'' no tienen un referente compartido. La expresión se impuso
solamente ---como un sustantivo singular que designa a un campo--- a
finales de los años sesenta y en los setenta, mucho después de que el
estudio de los medios y la comunicación se enraizara de diferentes
maneras en la universidad. Había ``estudios sobre medios'', en plural,
pero no existía algo como un campo con ese nombre. Incluso hoy en día
muchos académicos que trabajan en temas relacionados con los medios y la
comunicación, si se les pidiera que nombraran su campo, darían alguna
otra respuesta. Por su parte, los académicos que se autodefinen como
practicantes de ``estudios sobre medios'', podrían entrecerrar los ojos
ante la definición de su campo por parte de sus presuntos colegas. 


\enlargethispage{2\baselineskip}

\vspace*{2em}

\noindent{\emph{History of Media Studies}, vol. 1, 2021}




 \end{titlepage}


Consideramos que los deslices del término son una ventaja editorial. La
premisa de la revista es que lo que cuenta como historia de los estudios
sobre medios está de por sí en juego. La tarea de nuestros autores,
dicho de otro modo, es definir el alcance del campo mediante la
acumulación de argumentos ejemplares en sus artículos. Hay etiquetas
disciplinares con referentes más estables ---como ``investigación en
comunicación'' o ``estudios de cine''---, pero cada una de ellas es una
delimitación parcial. Por eso una de las virtudes de los ``estudios
sobre medios'' es su porosidad y ambivalencia.

Otra virtud proviene del propio concepto de medios (de comunicación).
Con raíces que se remontan al menos hasta la Ilustración, ``medios'' y
``mediación'' se han vuelto cada vez más fundamentales para entender las
texturas sociales del mundo moderno.\footnote{John Guillory,
  ``Enlightening Mediation'', en \emph{This is Enlightenment}, ed. por
  Clifford Siskin y William Warner (Chicago: University of Chicago
  Press, 2010).} Son conceptos amplios, centrales a todos los procesos
comunicativos, vinculados de una manera a su materialidad, y de otra, a
los procesos de representación, articulación, traducción e
intermediación. La forma dominante, ``medios'' en plural ---como órganos
comerciales de comunicación de masas---, se afianzó apenas en la década
de 1950, en el apogeo de la era de la radiodifusión.\footnote{Anna
  Shechtman, ``Command of Media's Metaphors'',~\emph{Critical
  Inquiry}~47, no. 4 (2021).} Invocamos a ``los medios'', en su rico
sentido católico, como una invitación a ventilar lo que podría ser la
historia de su estudio.

Como punto de partida, esperamos publicar trabajos que abarquen la
historia de las humanidades y de las ciencias sociales. Esperamos que
los estudios académicos sobre la historia de la investigación en
comunicación, los estudios culturales, los estudios cinematográficos, la
ciencia de la información, los estudios de periodismo, el discurso
(\emph{speech}) y la retórica, figuren en la mezcla editorial. Muchos de
los temas, discursos e instituciones que nuestros autores investigan no
encajarán tan fácilmente. Los estudiosos de los medios se han
identificado con docenas de otras disciplinas, a lo largo de contextos
locales y nacionales apenas representados en la literatura publicada.
Además, muchos estudiantes de medios nunca han estado afiliados a
universidades sino que trabajan para empresas comerciales, agencias
gubernamentales o grupos sin fines de lucro, o están integrados a
movimientos sociales. Nuestro propósito es publicar también sus
historias.

\hypertarget{exclusiones-en-la-historiografuxeda-de-los-estudios-sobre-medios}{%
\section{Exclusiones en la historiografía de los estudios sobre
medios}\label{exclusiones-en-la-historiografuxeda-de-los-estudios-sobre-medios}}

La misión de la revista está determinada por el momento histórico.
Nuestros campos se están enfrentando tardíamente a las herencias y
consecuencias actuales de una serie de inequidades estructurales que se
entrecruzan: su blanquitud y el patriarcado; las hegemonías de la lengua
inglesa y de las formas estadounidenses de pensamiento e investigación;
los efectos persistentes del colonialismo y la antinegritud; las muchas
formas de jerarquización y exclusión que emanan de los centros y las
periferias de todo el mundo, y la colonización neoliberal de
universidades y publicaciones académicas, por nombrar algunas de las más
destacadas. Hasta ahora, la mayor parte de los textos críticos sobre
estos temas se ha centrado, con razón, en sus manifestaciones
contemporáneas. \emph{Historia de los Estudios sobre Medios} busca
constituirse en un foro para investigar, además, la dinámica histórica
por la que los campos de estudio de los medios han llegado a este punto,
tanto en términos de los patrones sociales dominantes que los han
definido como de la mezcla de conciencia crítica y prácticas
alternativas que podrían dar lugar a otras maneras de avanzar en nuestro
trabajo intelectual. Más allá de publicar estas genealogías del
presente, la revista también se compromete con sus propias prácticas
alternativas y con el apoyo a los esfuerzos colectivos que han
dinamizado el momento actual.

\emph{Historia de los Estudios sobre Medios} tiene como objetivo
contribuir a lo que podría llamarse el descentramiento de los centros en
estos patrones interrelacionados de marginación. Esto no será un trabajo
fácil ni sencillo. Los hábitos heredados del campo están profundamente
arraigados en sus prácticas ---y en los cuerpos y mentes de quienes
trabajamos en él. Estos hábitos atraviesan la geopolítica, el lenguaje,
la ideología, la economía política y las interseccionalidades de la
identidad social. Todo esto continúa desarrollándose en un momento
particularmente peligroso de la historia, cuando las desigualdades
profundas, las crisis políticas y el cambio climático amenazan la
posibilidad de un futuro compartido y habitable en el planeta. No nos
hacemos muchas ilusiones sobre la importancia de una nueva revista
académica en estos contextos, y no queremos exagerar la importancia de
los esfuerzos que realiza. Dicho esto, creemos que las revistas
refractan las prácticas y las distorsiones más amplias de los campos
académicos en un microcosmos, y que deberían hacer su parte para
demandar más del presente. Una revista nueva puede al menos abrir un
espacio modesto para pensar y hacer las cosas de manera diferente,
sirviendo como un pequeño laboratorio que podría dar forma a esfuerzos
en otros lugares.

Uno podría preguntarse cómo una revista cuyos editores fundadores son
tres cishombres blancos de Estados Unidos puede esperar de buena fe
contribuir a esos esfuerzos. Nos tomamos esas dudas en serio y, de
hecho, las compartimos. La respuesta breve es que no podemos esperar
hacer mucho por nuestra cuenta. Estamos de acuerdo con el argumento de
Mohan Dutta de que debemos diseñar y poner en práctica diferentes tipos
de comunicación si queremos transformar las formas heredadas de producir
conocimientos académicos y constituir nuestros campos.\footnote{Mohan J.
  Dutta, ``Whiteness, Internationalization, and Erasure: Decolonizing
  Futures from the Global South'', \emph{Communication and
  Critical/Cultural Studies} 17, no. 2 (2020).} En el proyecto de
\emph{Historia de los Estudios sobre Medios}, esto comienza con nuestro
Consejo Editorial, que será un órgano más participativo que muchas
entidades de este tipo, dando forma activa al trabajo que hacemos y a la
manera en que lo hacemos. Somos muy afortunados de contar con un consejo
geográficamente diverso y de gran talento, cuyo número de miembros
seguirá creciendo. Los ensayos de lanzamiento que han escrito aquí son
los signos más visibles de su participación en esta iniciativa, pero de
ninguna manera los únicos. Los miembros del consejo nos ayudan a
imaginar y a poner en marcha una revista guiada por los ideales de
cuidado, oficio y amistad colegiada más allá de las fronteras, dirigida
hacia la investigación de la historia de los estudios sobre medios en
toda su complejidad alrededor del mundo. En este contexto, los editores
de \emph{Historia de los Estudios sobre Medios} vemos nuestro papel como
facilitadores de un proceso colectivo de traer algo nuevo al mundo.
Estamos comprometidos a aprovechar los privilegios estructurales que
podamos tener para contribuir a esa labor y avanzar a través de la
apertura dialógica a la diferencia y la crítica.

El antropólogo de origen colombiano Arturo Escobar, inspirado por los
zapatistas mexicanos, ha propuesto diseñar un mundo para ``el
pluriverso, un mundo en el que quepan muchos mundos''.\footnote{Arturo
  Escobar, \emph{Designs for the Pluriverse: Radical Interdependence,
  Autonomy, and the Making of Worlds} (Durham, NC: Duke University
  Press, 2018), xvi.} ¿Qué supondrían las historias de los estudios
sobre medios escritas para un pluriverso y cómo podría diseñarse la
revista para facilitarlas? No pretendemos saberlo, pero queremos crear
espacios para la experimentación y tenemos algunos compromisos de
entrada. Uno de ellos tiene que ver con el lenguaje, que como señala
Susana Martínez Guillem es ``un eje fundamental de las relaciones de
poder'' que debería figurar en los esfuerzos por remodelar el campo
contemporáneo.\footnote{Susana Martínez Guillem, ``Sacando la Lengua in
  Rhetorical Theory and Criticism'', \emph{Rhetoric, Politics \&
  Culture} 1, no. 1 (2021), 45. Véase también Silvio Waisbord,
  ``Communication Studies without Frontiers? Translation and
  Cosmopolitanism across Academic Cultures'', \emph{International
  Journal of Communication} 10 (2016), y Ana Cristina Suzina, ``English
  as \emph{Lingua Franca}: On the Sterilisation of Scientific Work'',
  \emph{Media, Culture \& Society} 43, no. 1 (2021).} La marginación de
los académicos y de los estudios que trabajan en lenguas distintas del
inglés es obvia para cualquiera que se encuentre fuera del centro de
exclusividad de este idioma en los campos académicos dominados por
Estados Unidos. Como un paso modesto en otra dirección, \emph{Historia
de los Estudios sobre Medios} revisará y publicará manuscritos tanto en
español como en inglés, y esperamos ampliarnos en el futuro más allá de
estos dos idiomas.

Un segundo compromiso es apoyar la escritura de historias de los
estudios sobre medios de todo el mundo, en especial del Sur Global y de
otras regiones que han sido periféricas, cuando mucho, en la
historiografía en lengua inglesa hasta hoy. Esto significa más que
simplemente generar un registro más completo de la historia de los
estudios sobre medios en todo el mundo, aunque creemos que es un
objetivo importante. Significa también apoyar las historiografías
análogas a la ``teoría desde el Sur'', que podrían reorientar las formas
heredadas de escribir las historias de los estudios sobre medios, o al
menos ayudar a crear algunos ``espacios comunes con margen para la
diferenciación'', propuestos por los defensores del cosmopolitismo
académico.\footnote{Jean Comaroff y John Comaroff, \emph{Theory from the
  South: Or How Euro-America is Evolving toward Africa} (Nueva York:
  Routledge, 2012), y Hanan Badr y Sarah Anne Ganter, ``Towards
  Cosmopolitan Media and Communication Studies: Bringing Diverse
  Epistemic Perspectives into the Field'', \emph{Global Media Journal}
  (edición alemana) 11, no. 1 (2021).} Esto debe hacerse de manera que
no se perpetúe la práctica de que los académicos del Norte Global hablen
de los contextos del Sur sin incluir las voces de esas
regiones.\footnote{Sarah Anne Ganter y Félix Ortega, ``The Invisibility
  of Latin American Scholarship in European Media and Communication
  Studies: Challenges and Opportunities of De-Westernization and
  Academic Cosmopolitanism'', \emph{International Journal of
  Communication} 13 (2019).} Asimismo, el compromiso implica
``provincializar'' las particularidades tradicionalmente no marcadas que
se han enmascarado como universales, o al menos como las historias de
base dadas por sentadas que, de manera intencional o no, se han
presentado como historias \emph{del} campo, cuando en realidad son las
historias de un puñado de académicos exitosos, institucionalmente bien
ubicados y abrumadoramente de género masculino que publicaron en inglés
o fueron lo suficientemente prominentes como para que su trabajo fuera
traducido.

Esto, a su vez, da pie a un tercer compromiso de la revista: centrarse
no solo en los estudios sobre medios tal y como se han desarrollado al
interior de las instituciones académicas, ya sean centrales o
periféricas, sino también en lo que llamamos ``tradiciones alternas'' de
la investigación reflexiva sobre los medios, de nuevo concebida en
sentido amplio. ¿Qué conceptualizaciones, prácticas educativas, marcos
normativos y formas de práctica social reflexiva sobre los medios se han
desarrollado fuera de la academia? Algunas de estas tradiciones están,
por supuesto, integradas en el sector comercial y otras tienen su origen
en la labor de destacadas organizaciones religiosas, como la Iglesia
católica. \emph{Historia de los Estudios sobre Medios} fomenta la
investigación que acerca estas historias a una conversación más cercana
con las historias de los estudios académicos, pero alentamos sobre todo
el trabajo de excavación de las tradiciones alternas de los grupos
indígenas y de otros grupos subalternos y las prácticas culturales
reflexivas que representan variedades menos reconocidas de estudios
sobre medios ---un cuerpo de trabajo en que han estado a la vanguardia
investigadores latinoamericanos y otros estudiosos del Sur Global, a
menudo en conexión con otros tipos de esfuerzos decoloniales.\footnote{Véase,
  por ejemplo, Claudia Magallanes Blanco y José Manuel Ramos Rodríguez,
  eds., \emph{Miradas propias: pueblos indígenas, comunicación y medios
  en la sociedad global} (Quito: Ediciones CIESPAL, 2016), y Erick R.
  Torrico Villanueva, \emph{Hacia la comunicación decolonial} (Sucre:
  Universidad Andina Simón Bolívar, 2016).}

\hypertarget{muxe1s-alluxe1-del-acceso-abierto}{%
\section{Más allá del acceso
abierto}\label{muxe1s-alluxe1-del-acceso-abierto}}

Los compromisos en la misión de la revista se extienden a su modo y
forma de publicación. \emph{Historia de los Estudios sobre Medios} es de
acceso abierto (OA), lo que significa que los lectores no pagan por los
artículos ni por las suscripciones. En este sentido, \emph{Historia de
los Estudios sobre Medios} se parece a muchas revistas nuevas,
incluyendo a algunas lanzadas por las cinco grandes editoriales
comerciales. La diferencia, crucial para nosotros, son las cuotas de
autores: no cobramos ninguna, por principio. Muchas revistas de acceso
abierto, incluso las publicadas por sociedades académicas, exigen a los
autores el pago de una ``cuota de tramitación del artículo'' (APC) que
suele estar entre 3 mil y 5 mil dólares. Creemos que el acceso abierto
para los lectores no debe cambiarse por nuevas barreras a la autoría. En
lugar de las cuotas de autores, financiamos nuestras operaciones a
través de lo que comúnmente se denomina ``financiación colectiva''
(\emph{collective funding}): el apoyo directo de bibliotecas y
fundaciones.\footnote{\emph{Historia de los Estudios sobre Medios} está
  participando en un nuevo modelo que integra a editores sin fines de
  lucro y bibliotecarios patrocinadores en una plataforma web
  compatible. La idea es que los bibliotecarios y otros financiadores
  puedan apoyar a los editores sobre la base de valores compartidos. La
  revista se encuentra entre los participantes del Programa de Inversión
  Comunitaria de Acceso Abierto (Open Access Community Investment
  Program, OACIP), organizado por el consorcio bibliotecario
  norteamericano LYRASIS. Acerca de la idea sobre los intercambios de
  financiamiento alineados a la misión, véase Jefferson Pooley,
  ``Collective Funding to Reclaim Scholarly Publishing'', \emph{The
  Commonplace}, 16 de Agosto de 2021,
  \url{https://commonplace.knowledgefutures.org/pub/erpw9udj}.} Lo que
significa la financiación colectiva es que los envíos que recibimos se
evalúan por sus méritos editoriales, sin tener en cuenta la riqueza
personal o institucional. Entre otras cosas, este estatus de gratuidad
apoya nuestra misión de publicar autores y temas de todo el mundo, ya
que la mayoría de los académicos, más allá de un puñado de instituciones
norteamericanas y de países europeos ricos, no puede pagar las APC.

En consonancia con nuestra política de gratuidad, interpretamos el
``acceso abierto'' de una manera más exigente y cargada de valores que
el típico medio académico. Creemos que la propiedad y la gobernanza son
importantes, que la publicación sostenible del acceso abierto debe
hacerse sin ánimo de lucro, dirigida por la comunidad y de manera
transparente. Junto con nuestra editorial dirigida por académicos,
mediastudies.press, suscribimos los ``Principios del auténtico acceso
abierto'' de Jean-Sebastian Caux.\footnote{Véase ``Open Access
  Principles'', mediastudies.press,
  \url{https://www.mediastudies.press/oa-principles}, y Jean-Sébastien
  Caux, ``Genuine Open Access Principles'', Jean-Sébastien Caux,
  \url{https://jscaux.org/blog/post/2018/05/05/genuine-open-access/}.}
Nos comprometemos a utilizar una infraestructura abierta siempre que sea
posible, incluyendo los programas mismos de publicación: PubPub, la
plataforma de código abierto vinculada al MIT, con la premisa de
recuperar la comunicación académica para la comunidad
académica.\footnote{PubPub es un proyecto del Knowledge Futures Group,
  sin fines de lucro, que surgió de la asociación entre la MIT Press y
  el MIT Media Lab. Véase ``Our Mission'', PubPub,
  \url{https://www.pubpub.org/about}. Sobre la visión de las
  publicaciones académicas del Knowledge Futures Group, véase Gabriel
  Stein et al., ``Clarivate, ProQuest, and our Resistance to
  Commercializing Knowledge'', \emph{The Commonplace}, 18 de mayo de
  2021, \url{https://commonplace.knowledgefutures.org/pub/kp81ylos/}.}
Nuestra gobernanza y finanzas son transparentes y están abiertas al
escrutinio, con acento especial en nuestras operaciones dirigidas por
académicos.\footnote{Véase ``Transparency'', mediastudies.press,
  \url{https://www.mediastudies.press/transparency}.}

Nuestro plan de limitar el volumen de artículos que publicamos
---normalmente no más de diez al año--- es un medio deliberado para
alcanzar un fin orientado al valor. Como editores, podemos permitirnos
adoptar un enfoque lento y cuidadoso con los autores y sus envíos.
Consideramos que se trata de un ideal artesanal, opuesto conscientemente
a la apurada ausencia de rostro que caracteriza a la mayoría de las
revistas comerciales.\footnote{Para una aguda reflexión sobre la
  interacción entre la captura comercial y el propio malestar y
  capitulación de los editores tras la derrota, véase Mark Gibson,
  ``Editing After Exit--Alienation and Counter--Alienation in the
  Cultures of Cultural Studies Journals'', \emph{Continuum} 35, no. 3
  (2021).} \emph{Historia de los Estudios sobre Medios} sustituye la
edición artesanal y la revisión humana por pares por ScholarOne y la
marea métrica. Los informes a nuestros autores, por ejemplo, no se
limitan a recuentos de descargas y de citas sino también incluyen
pasajes citados y contextos de citación.

La cuestión de la métrica plantea lo que es, para nosotros, un punto
importante. Creemos que nuestra ética basada en el cuidado es compatible
tanto con la calidad editorial como con las mejores prácticas en la
publicación académica. Todos los artículos reciben una licencia Creative
Commons y un DOI, y se pueden descargar en PDF y en otros siete
formatos, incluido el XML JATS de lectura mecánica, con una indexación
rápida y precisa en Google Scholar.\footnote{En efecto, \emph{Historia
  de los Estudios sobre Medios}, durante su lanzamiento, atiende todos
  los criterios de cumplimiento del ``Plan S'' de los financiadores
  europeos, incluidos los requerimientos técnicos. Véase cOAlition S,
  ``Plan S Principles'', Plan S,
  \url{https://www.coalition-s.org/plan_s_principles/}. El editor de la
  revista, mediastudies.press, es miembro de Crossref, la Asociación de
  Publicaciones Académicas de Acceso Abierto (Open Access Scholarly
  Publishing Association, OASPA) y el colectivo Radical Open Access, con
  asociaciones verificadas que incluyen al Directory of Open Access
  Books (DOAB), el Proyecto MUSE y la OAPEN. \emph{Historia de los
  Estudios sobre Medios} solicitará su incorporación en el listado del
  Directory of Open Access Journals (DOAJ) y su afiliación al Committee
  on Publication Ethics (COPE), cuando sea eligible después de un año de
  operación. Véase ``About this Journal'', \emph{History of Media
  Studies}, 2021, \url{https://hms.mediastudies.press/about}.} Los
correctores de la revista son también hábiles editores en línea, con
créditos de cabecera y de artículo en reconocimiento a su vital labor.

Uno de nuestros objetivos es ayudar a ampliar el aspecto de un artículo
académico. Elegimos la plataforma PubPub, en parte, por su amplio
soporte de los formatos multimedia, con la idea de que los historiadores
de los medios podrían iluminar el pasado de estos campos en diálogo con
las nuevas formas de comunicación académica. Consideremos el típico
artículo histórico basado en archivos: prevemos la publicación de
documentos de archivo y otros soportes dentro de los artículos que los
citan. También planeamos publicar colecciones de archivos y obras de
dominio público medio olvidadas, con introducciones nuevas. Incluso
volveremos a publicar trabajos arbitrados con licencia abierta
aparecidos en otros lugares, con ``Respuestas'' enlazadas solicitadas
para estos trabajos ``superpuestos'', así como para los artículos
originales. La revista está abierta a la presentación de trabajos
completos cuyos argumentos se presenten en audio, video y otras formas
no textuales.

Los valores de cuidado lento de \emph{Historia de los Estudios sobre
Medios} también guían nuestro enfoque de la revisión por pares. Estamos
comprometidos con un proceso de revisión humano y de desarrollo, con el
objetivo de mejorar los manuscritos a través del intercambio colegiado.
Inspirados por el ejemplo de \emph{Public Philosophy}, vemos nuestro
papel como algo más que revisores al servicio de una decisión
puntual.\footnote{Véase ``Formative Peer Review (FPR)'', \emph{Public
  Philosophy Journal},
  \url{https://publicphilosophyjournal.org/overview/}.} Nuestro objetivo
es cultivar en los revisores, también, un ethos de participación
continua, de apoyo al autor y su manuscrito. \emph{Historia de los
Estudios sobre Medios} emplea por \emph{default} la revisión doblemente
anónima, pero fomenta modos más abiertos a discreción de los autores.
Por ejemplo, apoyamos la \emph{revisión firmada}, en la que los
revisores firman sus comentarios y pueden seguir consultando con los
autores a lo largo del proceso de revisión. También experimentaremos con
la \emph{revisión comunitaria}, en la que un borrador de un artículo se
publica en una fase temprana del proceso, invitando a aportar
comentarios firmados ---y con revisiones iterativas respaldadas por el
robusto soporte de versiones de PubPub.\footnote{Véase ``Peer Review'',
  \emph{History of Media Studies},
  \url{https://hms.mediastudies.press/peer-review.}}

Entre nuestros objetivos está ayudar a fomentar una comunidad a
distancia de estudiosos que trabajan en la historia de los estudios
sobre medios y sus campos afines. Mantenemos, en cooperación con la
revista, un grupo de trabajo sobre la historia de los estudios sobre
medios. En cada sesión mensual, un investigador presenta un trabajo en
proceso en una sesión remota en la que participan académicos de todo el
mundo. La misma cultura de bienvenida y desarrollo informa al grupo de
trabajo, como una extensión del espíritu editorial de la
revista.\footnote{El grupo de trabajo está alojado en el Consortium for
  History of Science, Technology and Medicine (CHSTM). Véase ``Working
  Group on the History of Media Studies'', \emph{History of Media
  Studies}, \url{https://hms.mediastudies.press/working-group}.}

\hypertarget{una-revista-nueva}{%
\section{Una revista nueva}\label{una-revista-nueva}}

Los quince ensayos breves de esta serie de lanzamiento, escritos por
miembros del Consejo Editorial, reflejan los valores que hemos esbozado
aquí. El compromiso de la revista de prestar atención a las inequidades
y exclusiones estructurales se refleja en muchos de los ensayos de
lanzamiento. Wendy Willems, en su contribución, señala una dolorosa
ironía en los recientes intentos de des-occidentalizar y decolonizar los
estudios sobre los medios, que a menudo han silenciado las historias
reales del campo fuera del Norte Global y las luchas decoloniales de
larga data de la diáspora africana. Ella advierte los riesgos de que la
decolonización ``se convierta en una metáfora vacía'' y desafía a los
historiadores de este campo a ir más allá de la mera inclusión para
considerar ``cómo el acto de incluir diferentes puntos de vista desafía,
subvierte y problematiza'' las concepciones dominantes del campo. Armond
Towns cubre un terreno complementario, reviviendo una vigorosa tradición
de los estudios negros que ha sido ocluida por los estudios que
``situaban la vida de los negros únicamente como reacción al racismo
blanco'', y que puede renovarse mediante un ``proyecto epistemológico
alternativo que ... requeriría la reorganización del mundo tal y como se
encuentra actualmente''. Nos pide que leamos la historia de los estudios
sobre medios y comunicación en paralelo con la historia de los estudios
sobre negros y que veamos cómo algunas corrientes de los primeros ``se
desarrollaron con temor a la revolución negra y decolonial''.
Descentrando el Norte blanco y euroamericano desde perspectivas
geo-intelectuales diferentes, Liu Hailong y Qin Yidan se preguntan:
``¿Qué hubiera sido diferente si el estudio de la comunicación hubiera
nacido en China?''. Esto da paso a su debate sobre las particularidades
del campo chino y las posibilidades de que su reciente giro hacia los
medios incrustados en la experiencia china pueda marcar un nuevo
comienzo.

Otros ensayos de la serie abordan el descentramiento geopolítico del
Norte Global desde diferentes perspectivas. La contribución de Mohammad
Ayish sitúa la historia de los estudios sobre medios en los contextos
tecnológico, político y sociocultural del mundo árabe, siguiendo el
cambio de un marco de desarrollo a uno de empoderamiento. El autor
destaca los vínculos transnacionales y transregionales que desplazan al
Estado-nación como \emph{locus} histórico. En su cuidadoso examen de los
estudios sobre medios en Argentina, Mariano Zarowsky nos recuerda que
``hablar de un campo de conocimiento'' como el de los estudios sobre
medios ``implica estudiar un \emph{proceso de formación} más que partir
de entidades preexistentes''. Lleva este proceso de formación al nivel
del terreno, destacando la importancia de ``las formaciones temporales y
los contextos biográficos específicos'', así como la interacción de las
articulaciones regionales con los procesos globales. En su ensayo, Shiv
Ganesh esboza un programa para un ``enfoque centrado en el área'' para
la historia de los estudios sobre medios, con el propósito de desafiar
los males gemelos de la ``universalización teórica y el parroquialismo
metodológico'' en estos campos. Ilustra el valor del acercamiento con el
caso del Sudeste Asiático, replanteando el terreno para encontrar
alternativas a proyectar ``cualquier cosa fuera de la historia
euroamericana que se define en gran medida en términos de su
diferencia''. La biografía, el pensamiento y el rico legado de Jesús
Martín-Barbero son el objeto del ensayo de Raúl Fuentes-Navarro.
\emph{De los medios a las mediaciones} de Martín-Barbero, argumenta
Fuentes-Navarro, definiría a América Latina ``como distinta entre las
regiones culturales y lingüísticas occidentales''. Una obra fundacional
en el mundo iberoamericano que ha marcado el pensamiento desde su
publicación en 1987, el libro también proporciona un mapa para la
necesaria historiografía del campo en la región.

Muchos de los ensayos de lanzamiento se refieren a prescripciones y
ajustes en el enfoque o la técnica historiográfica. Stefanie
Averbeck-Lietz, por ejemplo, analiza un proyecto de periodismo sobre la
Liga de las Naciones, para señalar el lugar marginal que ocupan los
métodos históricos en los estudios de comunicación y medios en Alemania.
En el camino, nos invita a reflexionar sobre los métodos con los que se
escriben las historias del campo y las propias historias de esos
métodos. Thomas Wiedemann y Michael Meyen, en su ensayo, reflexionan
sobre un importante proyecto en idioma alemán ---\emph{Biografisches
Lexikon der Kommunikationswissenschaft} o \emph{BLexKom}--- que ayudan a
dirigir. En este sentido, se refieren a la promesa y al peligro
historiográficos que la iniciativa ha puesto de manifiesto, incluyendo
las cuestiones de quién cuenta la historia de quién. Sarah Cordonnier
aborda temas similares en su estudio de la historiografía de la
investigación francesa sobre los medios, identificando elisiones y
``agujeros negros'' en la literatura. Plantea grandes interrogantes
sobre cómo escribir historias de un campo con tantas permutaciones, y si
podemos encontrar la manera de ser ``contemporáneos en la disciplina a
pesar de todas las diferencias''. En su contribución, Wendy
Leeds-Hurwitz defiende los ``grupos de teoría'' como un prisma para
hacer historia, que revela las infraestructuras sociales que apoyan la
visibilidad y la influencia. Responde implícitamente a la cuestión del
método que plantea Averbeck-Lietz, dejando espacio para el uso de la
teoría de la comunicación de grupos para comprender la vida social de
las ideas.

Otros ensayos señalan los deslices productivos de la etiqueta ``estudios
sobre medios'' que la revista espera ampliar. Sue Collins, en su
contribución, muestra cómo el estudio de la autoridad mediada expone los
límites de lo que los ``estudios de comunicación'' o ``estudios
cinematográficos'' por sí solos podrían hacer por nosotros. Aboga por
que ``la historia de la comunicación y los estudios sobre medios
integren mejor a las historias del cine y de la cultura en su corpus''.
Con un espíritu similar, Filipa Subtil demuestra enérgicamente que la
filosofía de la tecnología subyace de forma tácita a la historia de los
estudios sobre medios, de manera que debería desafiar nuestras ideas
preconcebidas sobre lo que son los medios. Nos insta a remediar una
situación en la que, con algunas excepciones, ``los historiadores de los
estudios sobre medios no han prestado suficiente atención a la cuestión
de la tecnología''. Por su parte, Maria Löblich, de forma provocativa
desdibuja las fronteras entre la historia de los estudios de
comunicación y el fenómeno de la identidad colectiva, permitiendo la
fertilización cruzada de estos dos campos académicos. El proyecto de
Löblich, a la vez analítico y reconstructivo, nos permite pensar en
``cómo los estudios de comunicación fueron atados históricamente a los
sistemas simbólicos de la sociedad y qué grado de autonomía tenían''.
Por último, Ira Wagman nos desafía a ``problematizar la historia de los
`estudios sobre medios' en tantos contextos como sea posible'', tomando
la \emph{Miranda Prorsus} (encíclica papal de 1957 sobre el cine, la
televisión y la radio), como una ilustración de la religión de uno de
esos contextos. Al hacerlo, ofrece un excelente ejemplo de historias que
examinan la conceptualización y el estudio de los medios fuera de los
contextos académicos.

Tanto en lo individual como en conjunto, los ensayos de lanzamiento del
Consejo Editorial ejemplifican dimensiones de la misión de la revista y
la desarrollan de maneras que superan la imaginación de los editores que
la formularon. La pluralidad tiene un potencial único. Aquí se
manifiesta mediante escrituras que emergen de vidas académicas animadas
por diversas problemáticas, estilos de pensamiento, lenguas, culturas
políticas y contextos institucionales. Agradecemos sus creativas
respuestas a nuestra invitación, que esperamos ustedes lean y hagan
circular libremente, en el espíritu del acceso abierto.







\section{Bibliography}\label{bibliography}

\begin{hangparas}{.25in}{1} 



``About this Journal''. \emph{History of Media Studies}.
\url{https://hms.mediastudies.press/about}.

Badr, Hanan y Sarah Anne Ganter. ``Towards Cosmopolitan Media and
Communication Studies: Bringing Diverse Epistemic Perspectives into the
Field''. \emph{Global Media Journal} (edición alemana) 11, no. 1 (2021):
1--3. \url{https://doi.org/10.22032/dbt.49164}.

Caux, Jean-Sébastien. ``Genuine Open Access Principles''. Jean-Sébastien
Caux.
\href{https://jscaux.org/blog/post/2018/05/05/genuine-open-access/}{https://jscaux.org/blog/post/2018/05/05/genuine-open-access}.

cOAlition S. ``Plan S Principles''. Plan S.
\url{https://www.coalition-s.org/plan_s_principles/}

Comaroff, Jean y John Comaroff. \emph{Theory from the South: Or How
Euro-America is Evolving toward Africa}. Nueva York: Routledge, 2012.

Dutta, Mohan J. ``Whiteness, Internationalization, and Erasure:
Decolonizing Futures from the Global South''. \emph{Communication and
Critical/Cultural Studies} 17, no. 2 (2020): 228--35.
\url{https://doi.org/10.1080/14791420.2020.1770825}.

Escobar, Arturo. \emph{Designs for the Pluriverse: Radical
Interdependence, Autonomy, and the Making of Worlds}. Durham, NC: Duke
University Press, 2018.

``Formative Peer Review (FPR)''. \emph{Public Philosophy Journal}.
\url{https://publicphilosophyjournal.org/overview/}.

Ganter, Sarah Anne y Félix Ortega. ``The Invisibility of Latin American
Scholarship in European Media and Communication Studies: Challenges and
Opportunities of De-Westernization and Academic Cosmopolitanism'',
\emph{International Journal of Communication} 13 (2019): 68--91.
\url{https://ijoc.org/index.php/ijoc/article/view/8449}.

Gibson, Mark. ``Editing After Exit--Alienation and Counter--Alienation
in the Cultures of Cultural Studies Journals''. \emph{Continuum} 35, no.
3 (2021): 356--68. \url{https://doi.org/10.1080/10304312.2021.1902159}.

Guillory, John. ``Enlightening Mediation''. En \emph{This is
Enlightenment}, editado por Clifford Siskin y William Warner, 37--66.
Chicago: University of Chicago Press, 2010.

Magallanes Blanco, Claudia y José Manuel Ramos Rodríguez, eds.
\emph{Miradas propias: pueblos indígenas, comunicación y medios en la
sociedad global}. Quito: Ediciones CIESPAL, 2016.

Martínez Guillem, Susana. ``Sacando la Lengua in Rhetorical Theory and
Criticism''. \emph{Rhetoric, Politics \& Culture} 1, no. 1 (2021):
45--54. \url{https://muse.jhu.edu/article/801950}.

``Open Access Principles''. mediastudies.press.
\url{https://www.mediastudies.press/oa-principles}.

``Our Mission''. PubPub. \url{https://www.pubpub.org/about}.

``Peer Review''. \emph{History of Media Studies}.
\url{https://hms.mediastudies.press/peer-review}.

Pooley, Jefferson. ``Collective Funding to Reclaim Scholarly
Publishing'', \emph{The Commonplace}. 16 de agosto de 2021.
\url{https://commonplace.knowledgefutures.org/pub/erpw9udj}.

Shechtman, Anna. ``Command of Media's Metaphors''. \emph{Critical
Inquiry} 47, no. 4 (2021): 644--74.
\url{https://doi.org/10.1086/714512}.

Stein, Gabriel, Travis Rich, Zach Verdin y Catherine Ahearn.
``Clarivate, ProQuest, and our Resistance to Commercializing
Knowledge'', \emph{The Commonplace}. 18 de mayo de 2021.
\url{https://commonplace.knowledgefutures.org/pub/kp81ylos/}.

Suzina, Ana Cristina. ``English as \emph{Lingua Franca}: On the
Sterilisation of Scientific Work'', \emph{Media, Culture \& Society} 43,
no. 1 (2021): 171--79. \url{https://doi.org/10.1177/0163443720957906}.

Torrico Villanueva, Erick R. \emph{Hacia la comunicación decolonial}.
Sucre: Universidad Andina Simón Bolívar, 2016.

``Transparency''. mediastudies.press.
\url{https://www.mediastudies.press/transparency}

Waisbord, Silvio. ``Communication Studies without Frontiers? Translation
and Cosmopolitanism across Academic Cultures''. \emph{International
Journal of Communication} 10 (2016): 868--86.
\url{https://ijoc.org/index.php/ijoc/article/view/3483}.

``Working Group on the History of Media Studies''. \emph{History of
Media Studies}. \url{https://hms.mediastudies.press/working-group}.



\end{hangparas}


\end{document}