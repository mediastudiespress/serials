% see the original template for more detail about bibliography, tables, etc: https://www.overleaf.com/latex/templates/handout-design-inspired-by-edward-tufte/dtsbhhkvghzz

\documentclass{tufte-handout}

%\geometry{showframe}% for debugging purposes -- displays the margins

\usepackage{amsmath}

\usepackage{hyperref}

\usepackage{fancyhdr}

\usepackage{hanging}

\hypersetup{colorlinks=true,allcolors=[RGB]{97,15,11}}

\fancyfoot[L]{\emph{History of Media Studies}, vol. 1, 2021}


% Set up the images/graphics package
\usepackage{graphicx}
\setkeys{Gin}{width=\linewidth,totalheight=\textheight,keepaspectratio}
\graphicspath{{graphics/}}

\title[Emerging Digital Transitions in the Arab World]{Emerging Digital Transitions in the Arab World: Implications for the Region’s Communication Studies} % longtitle shouldn't be necessary

% The following package makes prettier tables.  We're all about the bling!
\usepackage{booktabs}

% The units package provides nice, non-stacked fractions and better spacing
% for units.
\usepackage{units}

% The fancyvrb package lets us customize the formatting of verbatim
% environments.  We use a slightly smaller font.
\usepackage{fancyvrb}
\fvset{fontsize=\normalsize}

% Small sections of multiple columns
\usepackage{multicol}

% Provides paragraphs of dummy text
\usepackage{lipsum}

% These commands are used to pretty-print LaTeX commands
\newcommand{\doccmd}[1]{\texttt{\textbackslash#1}}% command name -- adds backslash automatically
\newcommand{\docopt}[1]{\ensuremath{\langle}\textrm{\textit{#1}}\ensuremath{\rangle}}% optional command argument
\newcommand{\docarg}[1]{\textrm{\textit{#1}}}% (required) command argument
\newenvironment{docspec}{\begin{quote}\noindent}{\end{quote}}% command specification environment
\newcommand{\docenv}[1]{\textsf{#1}}% environment name
\newcommand{\docpkg}[1]{\texttt{#1}}% package name
\newcommand{\doccls}[1]{\texttt{#1}}% document class name
\newcommand{\docclsopt}[1]{\texttt{#1}}% document class option name


\begin{document}

\begin{titlepage}

\begin{fullwidth}
\noindent\LARGE\emph{Launch essay
} \hspace{85mm}\includegraphics[height=1cm]{logo3.png}\\
\noindent\hrulefill\\
\vspace*{1em}
\noindent{\Huge{Emerging Digital Transitions in the Arab\\\noindent World: Implications for the Region’s\\\noindent Communication Studies\par}}

\vspace*{1.5em}

\noindent\LARGE{Mohammad Ayish} \href{https://orcid.org/0000-0002-1262-078X}{\includegraphics[height=0.5cm]{orcid.png}}\par}\marginnote{\emph{Mohammad Ayish, ``Emerging Digital Transitions in the Arab World: Implications for the Region’s Communication Studies,'' \emph{History of Media Studies} 1 (2021), \href{https://doi.org/10.32376/d895a0ea.69b4fc2d}{https://doi.org/ 10.32376/d895a0ea.69b4fc2d}.} \vspace*{0.75em}}
\vspace*{0.5em}
\noindent{{\large\emph{American University of Sharjah}, \href{mailto:mayish@aus.edu}{mayish@aus.edu}\par}} \marginnote{\href{https://creativecommons.org/licenses/by-nc/4.0/}{\includegraphics[height=0.5cm]{by-nc.png}}}

% \vspace*{0.75em} % second author

% \noindent{\LARGE{<<author 2 name>>}\par}
% \vspace*{0.5em}
% \noindent{{\large\emph{<<author 2 affiliation>>}, \href{mailto:<<author 2 email>>}{<<author 2 email>>}\par}}

% \vspace*{0.75em} % third author

% \noindent{\LARGE{<<author 3 name>>}\par}
% \vspace*{0.5em}
% \noindent{{\large\emph{<<author 3 affiliation>>}, \href{mailto:<<author 3 email>>}{<<author 3 email>>}\par}}

\end{fullwidth}

\vspace*{1em}


\newthought{Throughout its eighty-year} history, communication studies in the Arab
World has generally been shaped by a convergence of national political
agendas, cultural concerns, technological transitions, and
Western-centered perspectives on modernization, dependency,
globalization and empowerment.\footnote{Mohammad Ayish, ``Arab Television Goes Commercial: A Case Study of the
  Middle East Broadcasting Centre,'' \emph{Gazette} 59, no. 6 (1997):
  473.
}
During the first three decades of post-colonial independence, the
region's media education programs, scholarship, and professional
orientations were hugely informed by both modernization and dependency
perspectives, while the globalization perspective was widely popular in
the 1990s, coinciding with the end of the Cold War and the advent of
transnational satellite television. The empowerment perspective, on the
other hand, was gaining momentum only at the dawn of the
21\textsuperscript{st} Century, as the Arab World came to gradually
embrace digital and online communications at institutional and personal
levels. While the region's governments and business sectors saw a huge
potential in digital/online communications as key forces of political
stability and economic prosperity, the region's communities at large
seemed to view them as enablers of greater democratization and
grassroots political engagement. Across the region in general and in the
Arabian Gulf countries in particular, huge investments have been
channeled into building world-class digital communications
infrastructures, allowing greater citizen access to an emerging virtual
public sphere. At universities and research communities, this ``digital
phase'' has been marked by new pedagogical, conceptual, and
methodological shifts in media education programs; new research
activities; and new professional orientations towards

\enlargethispage{2\baselineskip}

\vspace*{2em}

\noindent{\emph{History of Media Studies}, vol. 1, 2021}




 \end{titlepage}



\noindent
cyber-communications and virtual engagements. I argue here that while
digital transitions will continue to define the Arab World's political,
cultural, and economic living experiences for many years to come, they
are bound to have significant implications for media studies in the
region.

\hypertarget{communication-studies-and-its-engagement-with-modernization-dependency-and-globalization-a-historical-overview}{%
\section{Communication Studies and its Engagement with Modernization,
Dependency and Globalization: A Historical
Overview}\label{communication-studies-and-its-engagement-with-modernization-dependency-and-globalization-a-historical-overview}}

Although the American University in Cairo was the first institution of
higher education in the Arab World to host a journalism program in 1935,
the distinctive features of media education and scholarship in the
region did not take shape until the onset of the post-colonial era in
which newly-independent Arab states saw mass media as key engines of
socio-economic transformation and political integration. For these new
governments, mass communication channels, especially radio, were
instrumental in communicating ``developmental'' messages to
mostly-illiterate national populations in line with Lerner's
modernization framework of national
development.\footnote{Daniel Lerner, \emph{The Passing of Traditional Societies: Modernizing
  the Middle East} (Glencoe, IL: Free Press, 1958).
} Around that concept,
the region's media education and scholarship were harnessed to prepare
new generations of professional communicators and researchers who
strongly believed in the power of media to bring about positive
socio-economic and cultural changes within their under-developed
communities. In the 1960s and 1970s, universities in countries such as
Egypt, Tunisia, Jordan, and Morocco became home to journalism and mass
communication programs and to what came to be known as ``developmental
communication'' research traditions.

In the meantime, communication studies in some Arab countries aligned
with the former Soviet bloc---such as Egypt, Iraq, Syria, Libya, and
Algeria---promoted a more critical media outlook typical of the
dependency perspective. Media education programs and scholarship in
those countries focused on topics such as propaganda, media imperialism,
and Western information hegemony. This communication studies tradition
was hugely sustained by UNESCO discussions of a New World Information
and Communication Order (NWICO) that addressed one-way information flows
from the North (Western nations) to the South (Third World nations). It
promoted a view of media as guardians of national identity and tools of
political mobilization and national integration. Egyptian scholar Awatef
Abdul Rahman of Cairo University has argued that international media
channels were used to undermine pan-Arabism as a political ideology in
the region.\footnote{Awatef Abdul Rahman, \emph{Communication and Cultural Dependency
  Issues in the Third World} {[}in Arabic{]} (Kuwait: National Council
  for Culture and Arts, 1984).
}

In the 1990s, communication studies in the Arab World were bound to be
impacted by emerging globalization, itself enabled by the rise of
transnational communication technologies such as satellite television.
Many Arab media scholars argued that globalization was posing a serious
threat to Arab-Islamic values and traditions and called on national
media to maintain indigenous cultures and languages. Amin's critical
view of globalization as a brutal and hegemonic process enabled by
modern communications technologies\footnote{Jalal Amin, \emph{Globalization} {[}in Arabic{]} (Cairo: Dar Al
  Ma'aref, 1996).
}
generated a good deal of interest among academics and policy makers
across the region. Media education curricula came to include new courses
on globalization and satellite television, while research works sought
to throw light on how transnational media were impacting the region's
communications\footnote{Mohammad Ayish, ``Arab Television Goes Commercial: A Case Study of the
  Middle East Broadcasting Centre,'' \emph{Gazette} 59, no. 6 (1997):
  473-494.
} and socio-political
and cultural norms.\footnote{Douglas Boyd, \emph{Broadcasting in the Middle East} (Des Moines,
  Iowa: Iowa State University Press, 1999).
}

\hypertarget{the-current-state-of-communication-studies-the-empowerment-perspective}{%
\section{The Current State of Communication Studies: the Empowerment
Perspective}\label{the-current-state-of-communication-studies-the-empowerment-perspective}}

As the new century dawned on the Arab region, it was clear that digital
and online communications were presenting a serious challenge to
traditional, state-operated mass media. The ability of social media
platforms to give a voice to the average person in a state-controlled
communication environment was hailed across the region as highly
empowering. And the implications for communications studies have been
quite immense. Universities across the region came to introduce new
curricula aiming at educating new generations of young communicators in
digital media techniques and concepts. Topics such as \emph{social
media, citizen journalism, media convergence, digital photography,
digital video, online journalism, visual storytelling, virtual public
relations, digital advertising, new media theories, social analytics,
big data, website design, webcasting, blogging,} and \emph{cybermedia
ethics and laws} have made headlines in the region's media education
curricular maps.\footnote{Mohammad Ayish, ``Arab Media Studies'' in \emph{The International
  History of Communication Study}, ed. Peter Simonson and David Park
  (New York: Routledge, 2016), 476.
}

During my thirty-five year career as a media educator and researcher, I
have seen communication studies in the Arab World adapt dynamically at
the regional, national, and global scale to significant political,
economic, and technological developments. But the rush to engage with
digital and online communications on the part of media educators,
scholars, and professionals seems remarkably unprecedented. If long-time
state domination of the region's media landscape was perceived as the
black tunnel in which the region's communications have historically
evolved, digital empowerment came to be seen as ``the light at the end
of that tunnel.'' Three political/military developments that fostered
the region's perceptions of the power of social media and digital
communications in the past two decades seem to be cases in point: the
Anglo-American invasion of Iraq (2003), the Arab Spring (2011-2013), and
the Palestinian-Israeli military conflict of May 2021. The US occupation
of Iraq in April 2003 came to receive wide coverage on emerging online
national and international news platforms that were widely perceived
across the region as more diverse and freer than state-operated media.
Ralph Berenger, who had worked for 13 years at both the American
University of Cairo and the American University of Sharjah initiated his
``Media Go to Cyber War'' scholarly project that culminated in a
valuable collection of research works dealing with cyber media's
handling of the Iraq conflict.\footnote{Ralph Berenger, ed., \emph{Cybermedia Go to War: Role of Converging
  Media During and After the 2003 Iraq War} (Spokane, Washington:
  Marquette Books, 2006).
} The
real spike in scholarly interest in online media in the Arab World,
however, came during the second decade of the century and was triggered
mainly by upheavals across the region referred to as the ``Arab
Spring.'' Social media platforms accessible to the region's populations
were widely viewed as catalysts for the popular uprisings. The region's
researchers drew on empowerment theory to address social media
contributions to the Arab Spring, but the majority of those studies were
descriptive and drew on historical and conventional quantitative
methods.\footnote{ Adam Smidi and Saif Shahin, ``Social Media and Social Mobilization in
  the Middle East: A Survey of Research on the Arab Spring,''
  \emph{India Quarterly} 73, no. 2 (June 2017): 196--209.} Finally, the third
development most likely to attract scholarly attention in the region has
been the May 8-14, 2021 outbreak of violence in Palestine. During this
conflict, social media was viewed as a key tool for mobilizing public
opinion, and the resulting conversation about the role of social media
in politics has raised high expectations about what cyberspace could
deliver in times of crisis, particularly when state-run communications
are failing to provide the full picture of the conflict.

\hypertarget{implications-for-communication-studies}{%
\section{Implications for Communication
Studies}\label{implications-for-communication-studies}}

It is clear that the history of communication studies in the Arab World
will continue to be defined by regional and global political,
technological, and socio-cultural developments. The region's growing
investments in digital communications have converged with regional
political conflicts, social upheavals, and cultural concerns to position
digital and online features at the center of the region's communication
education and scholarship. These developments offer both opportunities
and challenges for media educators, scholars, and professional
practitioners. Media education programs could benefit from the
empowering aspects of digital communications in enhancing knowledge
about political participation, democratization, multiculturalism, and
diversity in their curricula. In research, media studies could also make
use of the emerging field of digital humanities and especially big data
analytics for long-term historical investigations of media behavior
across time as well as for enhancing qualitative aspects of scholarship.

On the other hand, communication studies in the region will continue to
face the challenge of freedom deficiency in the region. We have to
remember that as much as digital communications have provided
individuals with access to a global public sphere, they have also
enabled states to tighten their grip on the media landscape. Cybermedia
laws enacted over the past decades across the Arab World could be hugely
detrimental to the development of sustainable communication studies in
areas of education, research, and professional practice. In addition,
media educators should be sensitized to the fact that teaching online
and digital media is not a technical concern but a pursuit that also
involves significant social and cultural knowledge that students should
engage in. In this context, the liberal arts education model that
enables students' exposure to knowledge from the humanities and social
sciences should be maintained in digitally-centered media education
programs. In digital communication-focused scholarship, the region's
researchers should be mindful of that fact that algorithm-driven
analysis would be no alternative to their interpretations of data to
understand the human living experience.







\section{Bibliography}\label{bibliography}

\begin{hangparas}{.25in}{1} 



Abdul Rahman, Awatef. \emph{Communication and Cultural Dependency Issues
in the Third World}. {[}In Arabic.{]} Kuwait: National Council for
Culture and Arts, 1984.

Amin, Jalal. \emph{Globalization}. {[}In Arabic.{]} Cairo: Dar Al
Ma'aref, 1996.

Ayish, Mohammad. ``Arab Television Goes Commercial: A Case Study of the
Middle East Broadcasting Centre.'' \emph{Gazette} 59, no. 6 (1997):
473-494. \url{https://doi.org/10.1177/0016549297059006004}.

Ayish, Mohammad. ``Arab Media Studies.'' In \emph{The International
History of Communication Study}, edited by Peter Simonson and David
Park, 474-493. New York: Rutledge, 2016.

Berenger, Ralph, ed. \emph{Cybermedia Go to War: Role of Converging
Media During and After the 2003 Iraq War}. Spokane, Washington:
Marquette Books, 2006.

Boyd, Douglas. \emph{Broadcasting in the Middle East}. Des Moines: Iowa
State University Press, 1999.

Lerner, Daniel. \emph{The Passing of Traditional Societies: Modernizing
the Middle East}. New York: Free Press, 1958.

Smidi, Adam, and Saif Shahin. ``Social Media and Social Mobilization in
the Middle East: A Survey of Research on the Arab Spring.'' \emph{India
Quarterly} 73, no. 2 (June 2017): 196--209.
\url{https://doi.org/10.1177/0974928417700798}.



\end{hangparas}


\end{document}