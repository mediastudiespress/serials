% see the original template for more detail about bibliography, tables, etc: https://www.overleaf.com/latex/templates/handout-design-inspired-by-edward-tufte/dtsbhhkvghzz

\documentclass{tufte-handout}

%\geometry{showframe}% for debugging purposes -- displays the margins

\usepackage{amsmath}

\usepackage{hyperref}

\usepackage{fancyhdr}

\usepackage{hanging}

\hypersetup{colorlinks=true,allcolors=[RGB]{97,15,11}}

\fancyfoot[L]{\emph{History of Media Studies}, vol. 1, 2021}


% Set up the images/graphics package
\usepackage{graphicx}
\setkeys{Gin}{width=\linewidth,totalheight=\textheight,keepaspectratio}
\graphicspath{{graphics/}}

\title[Toward a New Media Study in China]{Toward a New Media Study in China: History and Approach } % longtitle shouldn't be necessary

% The following package makes prettier tables.  We're all about the bling!
\usepackage{booktabs}

% The units package provides nice, non-stacked fractions and better spacing
% for units.
\usepackage{units}

% The fancyvrb package lets us customize the formatting of verbatim
% environments.  We use a slightly smaller font.
\usepackage{fancyvrb}
\fvset{fontsize=\normalsize}

% Small sections of multiple columns
\usepackage{multicol}

% Provides paragraphs of dummy text
\usepackage{lipsum}

% These commands are used to pretty-print LaTeX commands
\newcommand{\doccmd}[1]{\texttt{\textbackslash#1}}% command name -- adds backslash automatically
\newcommand{\docopt}[1]{\ensuremath{\langle}\textrm{\textit{#1}}\ensuremath{\rangle}}% optional command argument
\newcommand{\docarg}[1]{\textrm{\textit{#1}}}% (required) command argument
\newenvironment{docspec}{\begin{quote}\noindent}{\end{quote}}% command specification environment
\newcommand{\docenv}[1]{\textsf{#1}}% environment name
\newcommand{\docpkg}[1]{\texttt{#1}}% package name
\newcommand{\doccls}[1]{\texttt{#1}}% document class name
\newcommand{\docclsopt}[1]{\texttt{#1}}% document class option name


\begin{document}

\begin{titlepage}

\begin{fullwidth}
\noindent\LARGE\emph{Launch essay
} \hspace{85mm}\includegraphics[height=1cm]{logo3.png}\\
\noindent\hrulefill\\
\vspace*{1em}
\noindent{\Huge{Toward a New Media Study in China:\\\noindent History and Approach\par}}

\vspace*{1.5em}

\noindent\LARGE{Hailong Liu}\par}\marginnote{\emph{Hailong Liu and Yidan Qin, ``Toward a New Media Study in China: History and Approach,'' \emph{History of Media Studies} 1 (2021), \href{https://doi.org/10.32376/d895a0ea.e7da2342}{https://doi.org/ 10.32376/d895a0ea.e7da2342}.} \vspace*{0.75em}}
\vspace*{0.5em}
\noindent{{\large\emph{Remnin University}, \href{mailto:liuhailong@ruc.edu.cn}{liuhailong@ruc.edu.cn}\par}} \marginnote{\href{https://creativecommons.org/licenses/by-nc/4.0/}{\includegraphics[height=0.5cm]{by-nc.png}}}

\vspace*{0.75em} 

\noindent{\LARGE{Yidan Qin} \href{https://orcid.org/0000-0002-1771-2018}{\includegraphics[height=0.5cm]{orcid.png}}\par}
\vspace*{0.5em}
\noindent{{\large\emph{Remnin University}, \href{mailto:qydlly@ruc.edu.cn}{qydlly@ruc.edu.cn}\par}}

% \vspace*{0.75em} % third author

% \noindent{\LARGE{<<author 3 name>>}\par}
\vspace*{0.5em}
% \noindent{{\large\emph{<<author 3 affiliation>>}, \href{mailto:<<author 3 email>>}{<<author 3 email>>}\par}}

\end{fullwidth}

\vspace*{1em}


\newthought{In his ethnography} of film production and cinema-going practices in
Nigeria, Brian Larkin posed an interesting question: What would media
theory have looked like if it had originated in Nigeria, rather than in
Europe and the United States?\footnote{Brian Larkin, \emph{Signal and
  Noise: Media, Infrastructure, and Urban Culture in Nigeria} (Durham,
  NC: Duke University Press, 2008), 253.} This question may elicit
insightful answers worldwide, as it invites us to adopt a
``de-Westernized'' perspective of communication and media theories. The
present essay considers similar problematics to Larkin: What difference
would it have made if communication studies had been born in China?
Answering this question helps us rethink the history and particularity
of the field in China.

It is not necessary to emphasize the particularity of Chinese
communication studies because the established discipline in China
borrowed a great deal from U.S. communication research, or more
precisely, from U.S. mass communication research. Their similarities are
evident. As a result of historical outcomes dating back to the 1980s,
they share remarkably similar textbooks, educational vocabularies, and
academic discourses. Moreover, the state-driven goal of building
world-class universities and the always-present requirement of
international publicity and visibility in recent years have further
strengthened the processes of ``Americanization'' in the discipline of
communication in China. The number of manuscripts from China in
English-language communication journals is soaring, and the younger
generation of Chinese communication researchers with backgrounds in U.S.
education keeps increasing.

\enlargethispage{3\baselineskip}

\vspace*{2em}

\noindent{\emph{History of Media Studies}, vol. 1, 2021}




 \end{titlepage}


Chinese communication studies, however, differs from its analogues in
the United States by dint of its own social cultural heritage and its
fusion of Western academic resources. This essay focuses on three
distinct characteristics of Chinese communication studies: (1) the role
played by the state in academia; (2) the tradition of Chinese practical
rationality; and (3) a new trend led by the reconceptualization of media
in recent years. Although the first two characteristics can also be
found in U.S. communication studies, they take on a new significance in
the Chinese case. The final characteristic---the reconceptualization of
media---provides a more striking contrast and may give us a significant
demonstration of meaningful divergence from the U.S. model.

\enlargethispage{1\baselineskip}

\hypertarget{the-presence-of-the-state}{%
\section{The Presence of the State}\label{the-presence-of-the-state}}

Much as Jefferson Pooley's ``new history of mass communication''
narration cast a light on the role of the state in U.S. communication
studies,\footnote{Jefferson Pooley, ``The New History of Mass
  Communication Research,'' in \emph{The History of Media and
  Communication Research: Contested Memories}, ed. David W. Park and
  Jefferson Pooley (New York: Peter Lang, 2008).} the state also played
a determining role in the founding and diffusion of the communication
discipline in China. One of the definitive moments in the solidification
of Chinese communication studies was Wilbur Schramm's visit to China in
1982, sponsored by the Department of Education of Guangdong Province in
China for distance-education training.\footnote{Yelu Yu, ``The
  Ice-Breaking Travel of the Science of Communication and `Chinese
  Communication' in China (1982--2002),'' in \emph{The Thirty Years of
  Chinese Communication Study}, ed. Yihong Wang and Yiqing Hu (Beijing:
  Encyclopedia of China Publishing House, 2010), 611--12.} At the time,
Schramm was not just a scholar of communication but also a modernization
theorist deeply involved in the United States' Cold War
strategy.\footnote{Hailong Liu, ``Wilbur Schramm and Communication Study
  in China: Cultural Cold War and Modernization Consensus,''
  \emph{Journalism \& Communication} 27, no. 6 (2020).} For
modernization theorists like him, the ``mission of the empire'' meant
using media to help developing countries achieve
modernization.\footnote{Liu, ``Wilbur Schramm and Communication Study,''
  102.} The primary purpose of this modernization, based on U.S. and
Western European models, was to curb the spread of communism.\footnote{M.
  E. Latham, \emph{Modernization as Ideology: American Social Science
  and ``Nation Building'' in the Kennedy Era} (Chapel Hill: University
  of North Carolina Press, 2000), 53--56.}

Chinese officials, however, found in the American tradition of mass
communication study a tool they could employ in pursuit of the goal of
the so-called Four Modernizations: the modernization of agriculture,
industry, national defense, and science and technology.\footnote{Liu,
  ``Wilbur Schramm and Communication Study in China,'' 103--4.}
Therefore, although the aim of the two countries' modernization efforts
differed, their similar usage of modernization discourses based on their
own political purposes led them to reach a consensus.\footnote{Liu,
  ``Wilbur Schramm and Communication Study in China,'' 103--4.}

This kind of modernization consensus proved highly
superficial.\footnote{Liu, ``Wilbur Schramm and Communication Study in
  China,'' 105.} Focusing mainly on the technical aspect of economic
growth, the consensus set aside the striking ideological differences
between the two countries. The arrival of any kind of political crisis
would soon render it fragile and influence the discipline's development.
After the initial stirrings of modern academic communication study in
China in the 1980s, communication studies has indeed experienced several
ups and downs due to external factors. These ups and downs demonstrated
the significant role played by the state.

First, after the Tiananmen Square incident in 1989, Chinese officials
criticized communication studies as bourgeois liberalization in the
field of journalism, suspending all related academic activities. Second,
at the beginning of the twenty-first century, China made significant
advancements in the areas of reform and opening up. Under these
circumstances, Chinese communication study became effectively
depoliticized, turning into a management tool instead. Research fields
such as public relations and online public opinion studies quickly
became popular at this time, gaining full official support. Third, since
the political and trade disputes between China and the United States in
2018, some Chinese officials have suggested that American ideology
hidden in social sciences such as communication studies might endanger
unity among the Chinese people.

\hypertarget{chinese-practical-rationality}{%
\section{Chinese Practical
Rationality}\label{chinese-practical-rationality}}

The second defining characteristic of Chinese communication studies we
address here is its practical rationality. This tradition of practical
rationality has a long history in Chinese philosophy. It shapes
communication studies in two ways.

First, this tradition of thinking emphasizes ethical thoughts and
practical values, and it exerts great influence on Chinese academia.
Without exception, Chinese communication study is practice oriented.
When communication studies entered China, early researchers attached
more importance to its practical parts (such as audience research) than
to theoretical construction. This orientation continues today. Many
Chinese researchers have busied themselves with practical research
topics such as online public opinion monitoring and analysis, the
convergence of old and new media, the construction and communication of
the national image of China, and international communication. These
topics are indisputably important and can of course be found in many
other countries, including the United States. But in China, this type of
research has proved dominant. Often, governmental policies guide the
questions asked and the methods of problematization.

Second, the tradition of Chinese practical rationality downplays
abstract and logical reasoning. As some researchers have observed,
compared to Western science, Chinese traditional science does not rely
on experiments and the language of mathematical logic to construct
theory.\footnote{Georgette Wang, Vincent Shen, and Ven-hwei Lo,
  ``Chinese Communication Theory Construction: Mission Impossible?''
  \emph{Mass Communication Research}, no. 70 (2002): 7.} Intuition and
imagination have guided its understanding of the world,\footnote{Wang,
  Shen, and Lo, ``Chinese Communication Theory Construction.''} and it
therefore takes the form of insightful but always unfalsifiable
observations and explanations rather than verifiable theoretical
propositions. Communication research, especially during the early stage
of communication study in China, was full of personal observations and
short on rigorous methods. Embedded in this background, some
communication scholars emphasized that Chinese researchers first needed
to transform and operationalize their traditional ideas and ordinary
concepts before entering into a dialogue with Western
academics.\footnote{Wang, Shen, and Lo, ``Chinese Communication Theory
  Construction,'' 8--9.}
  
\enlargethispage{1\baselineskip}

\hypertarget{the-current-trend-toward-a-new-media-study}{%
\section{The Current Trend toward a New Media
Study}\label{the-current-trend-toward-a-new-media-study}}

The third characteristic of Chinese communication study is its so-called
media turn. Generally, the dominant paradigm in communication research
established after World War II focuses principally on mass media and has
an obvious agenda bias of effects research based on media content. In
this way, media are often considered objective tools for the
transmission of information. For example, Schramm defined media as
``machines interposed in the communication process to multiply and
extend the delivery of information.''\footnote{Wilbur Schramm and
  William Earl Porter, \emph{Men, Women, Message, and Media:
  Understanding Human Communication} (New York: Harper \& Row, 1973),
  132.} Marshall McLuhan gave a new direction, totally different from
Schramm's. He showcased an insightful definition of the medium: ``the
medium is the message'' (emphasizing the form rather than the content of
the medium) and the ``extension of man'' (the embodiment interpretation
of the medium).\footnote{Marshall McLuhan, \emph{Understanding Media:
  The Extensions of Man}, 2nd ed. (New York: New American Library,
  1964), viii.} After McLuhan, and partly through his influence, we have
seen the development of numerous media theories. Instead of regarding
media as carriers of messages, these media theories highlight media's
constitutive roles in human existence, an idea that largely drives
traditional communication scholarship. As an outcome of this influence,
communication study has begun to reconceptualize its key concepts (such
as \emph{media} and \emph{communication}), informing the so-called media
turn.

Chinese scholars have demonstrated their insights and creativity during
this new turn. To some extent, this results from the specific
circumstances in China. In recent years, digital media and the related
infrastructure construction projects are developing rapidly in the
country. The speed of progress surpasses that in many developed
countries. In part, this is the result of full governmental
support---the digital media industry is one of the most important fields
for Chinese economic development, so it benefits from a great deal of
governmental support. This embrace of digital media can also be traced
to Chinese people's optimistic attitude toward new technology, an
attitude deeply rooted in ideas of social evolutionism and social
revolution introduced to China from the West in the late nineteenth
century. Since then, the Chinese philosophy of history has changed
almost completely from one of degeneration or a cyclical theory of
history to a progressive and revolutionary one. Influenced by these
ideas, Chinese people tend to consider new media more progressive than
old media.

In the context of the flourishing market for digital media, Chinese
scholars developed their special attention to forms of media and their
relation to human beings. They are involved in this media turn in two
ways: general theoretical reflection and further empirical research
embedded in local experiences. These efforts largely exemplify Chinese
scholar Huang Dan's appeal. He once proposed that, to correct the
overwhelming research bias of mass media content and effects,
\emph{medium} itself should serve as the fundamental perspective of
communication study.\footnote{Dan Huang, ``Preface to Meijie Daoshuo
  Translation Series,'' in \emph{Virilio and the Media}, ed. John
  Armitage (Beijing: Communication University of China Press, 2019), 13.}
By employing a broader conceptualization of media, he suggested, we
would be better prepared to grasp the essence of communication study and
therefore better organize the whole field.\footnote{Huang, ``Preface to
  Meijie Daoshuo Translation Series,'' 13.}

Such appeal is noticeably increasing. Many Chinese researchers like
Huang Dan have begun to extend the traditional understanding of media.
Alongside ``representative media'' such as radio, film, and television,
they pay attention to ``unrepresentative media'' such as roads, cities,
bodies, technologies, infrastructure, and viruses. These scholars have
done much to investigate previously undervalued topics. For example,
some researchers have begun to focus on the ``new type of tie'' between
city and communication, forming a community of urban communication
studies, based at Fudan University, to respond to the varied new media
practices and urbanization of Chinese society from the perspective of
media.\footnote{For research on this topic, see Dan Huang, Preface to
  \emph{Urban Communication: On the History and Reality of the Chinse
  City}, ed. Dan Huang (Shanghai: Shanghai Jiaotong University Press,
  2015), 1--7; Wei Sun, ``Research Approaches to and Theoretical
  Innovation for Urban Communication,'' \emph{Modern Communication} 40,
  no. 12 (2018): 29--40.} Some researchers focused their attention on
the body and its relation to communication. They have used the concept
of \emph{embodiment} to reconsider the role of the body in the process
of communication, reflecting on varied embodied practices under the
conditions of new media.\footnote{For research on this topic, see
  Hailong Liu, ``Body Agenda and the Future of Communication Studies,''
  \emph{Journal of Journalism \& Communication} 40, no. 2 (2018):
  37--46; Hailong Liu and Kairong Shu, ``Embodiment and Body Idea in
  Communication Studies: A Perspective of Phenomenology of Perception
  and Cognitive Science,'' \emph{Journal of Lanzhou University (Social
  Sciences)} 47, no. 2 (2019): 80--89; Hailong Liu, Zhuoxiao Xie, and
  Kairong Shu, ``Networked Bodies: Human as Viruses and Patches in
  Technological Systems,'' \emph{Journalism Research}, no.5 (2021):
  40--55, 122--23; Bifeng Rui and Zhen Ang, ``Embodied Communication:
  From the Perspective of Cognitive Linguistics,'' \emph{Modern
  Communication} 43, no. 4 (2021): 33--39; Zhuoxiao Xie, ``Bodies as
  Mobile Media: The Embodied Communication Practices and Communicator's
  Bodies in the Cross-Border Shopping,'' \emph{Chinese Journal of
  Journalism \& Communication} 43, no. 3 (2021): 40--57.} Additionally,
we have witnessed the blossoming of subfields, including media
materiality studies,\textsuperscript{19} media archaeology,\textsuperscript{20} media
geography,\textsuperscript{21} and
media anthropology.\textsuperscript{22} Evidently, the new media turn is neither a
replacement of nor a supplement to existing communication studies. It is
its own mode of thought that invites us to re-consider communication as
a whole.

\hypertarget{conclusion}{%
\section{Conclusion}\label{conclusion}}

Let us go back to the questions we raised at the beginning of this
essay. Our tentative answer would be that, if communication studies had
originated in China, the striking political and cultural disparities
would make it different from American communication studies. First, the
discipline would be more practically oriented and would need to meet the
need of political and commercial propaganda. Second, Chinese scholars
may develop philosophical ways of considering media and communication in
the style of scholars from France, Germany, or Japan\marginnote{\textsuperscript{19} For research on this topic, see Gehao
  Zhang and Lei Zhang, ``A Dialectic Thinking between Material Being and
  Anthropogenic Seeing: The Materiality Turn in Media and Cultural
  Analysis,'' \emph{Global Journal of Media Studies} 6, no. 2 (2019):
  103--15; Guohua Zeng, ``Media and Communication Materiality Studies:
  Theoretical Origins, Research Approaches and Subareas,'' \emph{Chinese
  Journal of Journalism \& Communication} 42, no. 11 (2020): 6--24.}, rather\marginnote{\textsuperscript{20} For research on this topic, see Hongzhe
  Wang, ``Eye of the Sky and the Deep: Media Archeology of Crittercam,''
  \emph{Film Art}, no. 3 (2018): 23--26; Chang Shi, ``Yesterday Once
  More: The Rise of Media Archaeology and Its Awareness of Problems,''
  \emph{Journalism \& Communication} 26, no. 7 (2019): 33--53, 126--27;
  Ji Pan and Lingyan Li. ``Media Research, Technological Innovation and
  Knowledge Production: Insights from Media Archeology-Conversation with
  Professor Siegfried Zielinski,'' \emph{Chinese Journal of Journalism
  \& Communication} 42, no. 7 (2020): 96--113.} than\marginnote{\textsuperscript{21} For research on this topic, see Jianbin Guo, ``Media
  Anthropology: A Research Based on Literatures,'' in \emph{Media World
  and Media Anthropology}, ed. Qiyao Deng (Guangzhou: Sun Yat-Sen
  University Press, 2015), 17--45; Xinru Sun, ``Media Anthropology as a
  `Cultural Method,'\,'' \emph{Nanjing Journal of Social Sciences}, no.
  5 (2019): 113--20, 156; Xinru Sun and Hong Duan, ``Rethinking
  `Embeddedness': Relational Dimension of Media Anthropology,''
  \emph{Nanjing Journal of Social Sciences}, no. 9 (2020): 103--11.}
conducting\marginnote{\textsuperscript{22} For research on this topic, see Yan Yuan,
  ``When Geographers Talk about Media and Communication, What Do They
  Talk About? Comments on Paul Adams' \emph{Geographies of Media and
  Communication},'' \emph{Chinese Journal of Journalism \&
  Communication} 41, no. 7 (2019): 157--76; Yan Yuan, ``The Materiality
  of Television and the Politics of Mobility: A Media Geographic Study
  from Two Urban Villages,'' \emph{Journalism \& Communication} 23, no.
  6 (2016): 92--104, 128.}\setcounter{footnote}{22} communication research with quantitative methods.

As mentioned above, globalization, an increasing amount of academic
exchange, and especially state-encouraged international publication have
greatly reduced the differences between Chinese and American
communication studies. One characteristic of Chinese communication study
lining up with its American analogues is that it has also held a
relatively narrow view of \emph{media} and \emph{communication} in the
past few decades. Audience research and effects research long dominated
the research agenda. However, this traditional perspective gradually
lost its theoretical competitiveness for today's digital society.

The media turn discussed above has brought opportunities for Chinese
scholars to transcend the original American communication establishment
and develop a new kind of media study. During this process, European
media studies and philosophical currents such as German media theory,
mediatization theory, and mediology were reappraised for their possible
value to understand the current situation here in China. Some years ago,
John Peters said, ``Please do not think me extravagant if I say that we
North American media scholars risk the fate of Casaubon without
grappling with recent German work. We need not swallow it whole, but
grapple we must.''\footnote{John Durham Peters, ``Strange Sympathies:
  Horizons of Media Theory in America and Germany,'' \emph{Electronic
  Book Review}, June 4, 2009.} Today, Chinese scholars, like scholars
from elsewhere around the globe, are doing just this kind of grappling.
Compared with the situation in the 1980s, this time we are witnessing
more theoretical reflection in Chinese academia, instead of simple
learning and imitation. A new kind of media study embedded in Chinese
experiences is worth the wait.







\section{Bibliography}\label{bibliography}

\begin{hangparas}{.25in}{1} 



Guo, Jianbin. ``Media Anthropology: A Research Based on Literatures.''
In \emph{Media World and Media Anthropology}, edited by Qiyao Deng,
17--45. Guangzhou: Sun Yat-Sen University Press, 2015.

Huang, Dan. ``Preface to Meijie Daoshuo Translation Series.'' In
\emph{Virilio and the Media}, edited by John Armitage, 1--14. Beijing:
Communication University of China Press, 2019.

---------. Preface to \emph{Urban Communication: On the History and
Reality of the Chinse City}, edited by Dan Huang, 1--7. Shanghai:
Shanghai Jiaotong University Press, 2015.

Larkin, Brian. \emph{Signal and Noise: Media, Infrastructure, and Urban
Culture in Nigeria}. Durham, NC: Duke University Press, 2008.

Latham, M. E. \emph{Modernization as Ideology: American Social Science
and ``Nation Building'' in the Kennedy Era}. Chapel Hill: University of
North Carolina Press, 2000.

Liu, Hailong. ``Body Agenda and the Future of Communication Studies.''
\emph{Chinese Journal of Journalism \& Communication} 40, no. 2 (2018):
37--46. \url{https://doi.org/10.13495/j.cnki.cjjc.2018.02.006}.

Liu, Hailong. ``Communication of Virus.'' \emph{The Thinker}, no. 29
(2020):1--5.

Liu, Hailong. ``Wilbur Schramm and Communication Study in China:
Cultural Cold War and Modernization Consensus.'' \emph{Journalism \&
Communication} 27, no. 6 (2020): 92--109, 128.

Liu, Hailong, and Kairong Shu. ``Embodiment and Body Idea in
Communication Studies: A Perspective of Phenomenology of Perception and
Cognitive Science.'' \emph{Journal of Lanzhou University (Social
Sciences)} 47, no. 2 (2019): 80--89.
\url{https://doi.org/10.13885/j.issn.1000-2804.2019.02.010}.

Liu, Hailong, Zhuoxiao Xie, and Kairong Shu. ``Networked Bodies: Human
as Viruses and Patches in Technological Systems.'' \emph{Journalism
Research}, no.5 (2021): 40--55, 122--23.

McLuhan, Marshall. \emph{Understanding Media: The Extensions of Man.}
2nd ed. New York: New American Library, 1964.

Pan, Ji and Lingyan Li. ``Media Research, Technological Innovation and
Knowledge Production: Insights from Media Archeology-Conversation with
Professor Siegfried Zielinski.'' \emph{Chinese Journal of Journalism \&
Communication} 42, no. 7 (2020): 96--113.
\url{https://doi.org/10.13495/j.cnki.cjjc.2020.07.005}.

Peters, John Durham. ``Strange Sympathies: Horizons of Media Theory in
America and Germany.'' \emph{Electronic Book Review}, June 4, 2009.
\href{http://electronicbookreview.com/essay/strange-sympathies-horizons-of-media-theory-in-america-and-germany/}{http://electronicbookreview.com/essay/strange-sympathies-horizons-of-media-theory-in-america-and-germany/}.

Pooley, Jefferson. ``The New History of Mass Communication Research.''
In \emph{The History of Media and Communication Research: Contested
Memories}, edited by David W. Park and Jefferson Pooley, 43--70. New
York: Peter Lang, 2008.

Rui, Bifeng, and Zhen Ang. ``Embodied Communication: From the
Perspective of Cognitive Linguistics.'' \emph{Modern Communication} 43,
no. 4 (2021): 33--39.
\url{https://doi.org/10.3969/j.issn.1007-8770.2021.04.007}.

Schramm, Wilbur, and William Earl Porter. \emph{Men, Women, Message, and
Media: Understanding Human Communication}. New York: Harper \& Row,
1973.

Shi, Chang. ``Yesterday Once More: The Rise of Media Archaeology and Its
Awareness of Problems.'' \emph{Journalism \& Communication} 26, no. 7
(2019): 33--53, 126--27.

Sun, Wei. ``Research Approaches to and Theoretical Innovation for Urban
Communication.'' \emph{Modern Communication} 40, no. 12 (2018): 29--40.
\url{https://doi.org/10.3969/j.issn.1007-8770.2018.12.006}.

Sun, Xinru. ``Media Anthropology as a `Cultural Method.'\,''
\emph{Nanjing Journal of Social Sciences}, no. 5 (2019): 113--20, 156.
\url{https://doi.org/10.15937/j.cnki.issn1001-8263.2019.05.015}.

Sun, Xinru, and Hong Duan. ``Rethinking `Embeddedness': Relational
Dimension of Media Anthropology.'' \emph{Nanjing Journal of Social
Sciences}, no. 9 (2020): 103--11.
\url{https://doi.org/10.15937/j.cnki.issn1001-8263.2020.09.014}.

Wang, Georgette, Vincent Shen, and Ven-hwei Lo. ``Chinese Communication
Theory Construction: Mission Impossible?'' \emph{Mass Communication
Research}, no. 70 (2002): 1--15.

Wang, Hongzhe. ``Eye of the Sky and the Deep: Media Archeology of
Crittercam.'' \emph{Film Art}, no. 3 (2018): 23--26.

Xie, Zhuoxiao. ``Bodies as Mobile Media: The Embodied Communication
Practices and Communicator's Bodies in the Cross-Border Shopping.''
\emph{Chinese Journal of Journalism \& Communication} 43, no. 3 (2021):
40--57. \url{https://doi.org/10.13495/j.cnki.cjjc.2021.03.003}.

Yu, Yelu. ``The Ice-Breaking Travel of the Science of Communication and
`Chinese Communication' in China (1982--2002).'' In \emph{The Thirty
Years of Chinese Communication Study}, edited by Yihong Wang and Yiqing
Hu, 609--19. Beijing: Encyclopedia of China Publishing House, 2010.

Yuan, Yan. ``The Materiality of Television and the Politics of Mobility:
A Media Geographic Study from Two Urban Villages.'' \emph{Journalism \&
Communication} 23, no. 6 (2016): 92--104, 128.

---------. ``When Geographers Talk about Media and Communication, What
Do They Talk About? Comments on Paul Adams' \emph{Geographies of Media
and Communication}.'' \emph{Chinese Journal of Journalism \&
Communication} 41, no. 7 (2019): 157--76.
\url{https://doi.org/10.13495/j.cnki.cjjc.2019.07.011}.

Zhang, Gehao, and Lei Zhang. ``A Dialectic Thinking between Material
Being and Anthropogenic Seeing: The Materiality Turn in Media and
Cultural Analysis.'' \emph{Global Journal of Media Studies} 6, no. 2
(2019): 103--15. \url{https://doi.org/10.16602/j.gmj.20190018}.

Zeng, Guohua. ``Media and Communication Materiality Studies: Theoretical
Origins, Research Approaches and Subareas.'' \emph{Chinese Journal of
Journalism \& Communication} 42, no. 11 (2020): 6--24.
\url{https://doi.org/10.13495/j.cnki.cjjc.2020.11.001}.



\end{hangparas}


\end{document}