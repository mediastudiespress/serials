% see the original template for more detail about bibliography, tables, etc: https://www.overleaf.com/latex/templates/handout-design-inspired-by-edward-tufte/dtsbhhkvghzz

\documentclass{tufte-handout}

%\geometry{showframe}% for debugging purposes -- displays the margins

\usepackage{amsmath}

\usepackage{hyperref}

\usepackage{fancyhdr}

\usepackage{hanging}

\hypersetup{colorlinks=true,allcolors=[RGB]{97,15,11}}

\fancyfoot[L]{\emph{History of Media Studies}, vol. 1, 2021}


% Set up the images/graphics package
\usepackage{graphicx}
\setkeys{Gin}{width=\linewidth,totalheight=\textheight,keepaspectratio}
\graphicspath{{graphics/}}

\title[Collective Identity and the History of Communication Studies]{Collective Identity and the History of Communication Studies} % longtitle shouldn't be necessary

% The following package makes prettier tables.  We're all about the bling!
\usepackage{booktabs}

% The units package provides nice, non-stacked fractions and better spacing
% for units.
\usepackage{units}

% The fancyvrb package lets us customize the formatting of verbatim
% environments.  We use a slightly smaller font.
\usepackage{fancyvrb}
\fvset{fontsize=\normalsize}

% Small sections of multiple columns
\usepackage{multicol}

% Provides paragraphs of dummy text
\usepackage{lipsum}

% These commands are used to pretty-print LaTeX commands
\newcommand{\doccmd}[1]{\texttt{\textbackslash#1}}% command name -- adds backslash automatically
\newcommand{\docopt}[1]{\ensuremath{\langle}\textrm{\textit{#1}}\ensuremath{\rangle}}% optional command argument
\newcommand{\docarg}[1]{\textrm{\textit{#1}}}% (required) command argument
\newenvironment{docspec}{\begin{quote}\noindent}{\end{quote}}% command specification environment
\newcommand{\docenv}[1]{\textsf{#1}}% environment name
\newcommand{\docpkg}[1]{\texttt{#1}}% package name
\newcommand{\doccls}[1]{\texttt{#1}}% document class name
\newcommand{\docclsopt}[1]{\texttt{#1}}% document class option name


\begin{document}

\begin{titlepage}

\begin{fullwidth}
\noindent\LARGE\emph{Launch essay
} \hspace{85mm}\includegraphics[height=1cm]{logo3.png}\\
\noindent\hrulefill\\
\vspace*{1em}
\noindent{\Huge{Collective Identity and the History of\\\noindent Communication Studies \par}}

\vspace*{1.5em}

\noindent\LARGE{Maria Löblich}\par}\marginnote{\emph{Maria Löblich, ``Collective Identity and the History of Communication Studies ,'' \emph{History of Media Studies} 1 (2021), \href{https://doi.org/10.32376/d895a0ea.cf521edc}{https://doi.org/ 10.32376/d895a0ea.cf521edc}.} \vspace*{0.75em}}
\vspace*{0.5em}
\noindent{{\large\emph{Frei Universität Berlin}, \href{mailto:maria.loeblich@fu-berlin.de}{maria.loeblich@fu-berlin.de}\par}} \marginnote{\href{https://creativecommons.org/licenses/by-nc/4.0/}{\includegraphics[height=0.5cm]{by-nc.png}}}

% \vspace*{0.75em} % second author

% \noindent{\LARGE{<<author 2 name>>}\par}
% \vspace*{0.5em}
% \noindent{{\large\emph{<<author 2 affiliation>>}, \href{mailto:<<author 2 email>>}{<<author 2 email>>}\par}}

% \vspace*{0.75em} % third author

% \noindent{\LARGE{<<author 3 name>>}\par}
% \vspace*{0.5em}
% \noindent{{\large\emph{<<author 3 affiliation>>}, \href{mailto:<<author 3 email>>}{<<author 3 email>>}\par}}

\end{fullwidth}

\vspace*{1em}


\newthought{This essay explores} the role of collective identity in the history of
communication studies. Collective identities are ideas about social
belonging and values, about who we are and who the others are.
Collective identities involve ideas of homogeneity and difference and
involve issues of power and resource
distribution.\footnote{Jan Assmann, \emph{Das kulturelle Gedächtnis. Schrift, Erinnerung und
  politische Identität in frühen Hochkulturen} (München: C.H. Beck,
  1992), 131; Shmuel Noah Eisenstadt and Bernhard Giesen, ``The
  Construction of Collective Identity,'' \emph{European Journal of
  Sociology} 36, no. 1 (1995): 77; Bonnie J. Dow, ``The Lessons of
  History: Women's Studies in Communication Approaches 40,''
  \emph{Women's Studies in Communication} 37 (2014): 260.
} There are at least
two aspects of the historical relation between communication studies and
collective identity: (1) The history of communication studies is linked
to public communication, where collective identities are discussed,
defined, and ascribed to groups or collectives in society or in the
world. Scholarly work in the past studied, tacitly reproduced, or
questioned collective identity constructs made by the media. (2)
Communication studies worldwide have been carriers of a
``we-consciousness,''\footnote{Assmann, \emph{Das kulturelle Gedächtnis}, 131.
} precondition
for academic community-building and legitimacy work. Moreover,
communication studies have been the object of identity ascriptions from
the outside.

Some historians and sociologists have criticized the concept of
collective identity for the ideologically charged, manipulative use of
this term in the twentieth century, for the lack of a theory, and for
suggesting collective identity is something
``natural.''\footnote{Eisenstadt and Giesen, ``The Construction of Collective Identity,''
  74.
} In communication
studies, some have noted critically that some historical accounts have
been whitewashing and mystifying the field's historiography. According
to Simonson and Park, ``representatives of the dominant paradigms
offered their own historical accounts'' when defining the significance
of classic works or great male
scholars.\footnote{Peter Simonson and David W. Park, ``Introduction. On the History of
  Communication Study,'' in \emph{The International History of
  Communication Study}, eds. Peter Simonson and David W. Park (New York:
  Routledge, 2016), 4--5; cf. Jefferson Pooley and David W. Park,
  ``Introduction,'' in \emph{The History of Media and Communication
  Research. Contested Memories}, eds. David W. Park and Jefferson Pooley
  (New York: Lang, 2008), 1--4.
} Wahl-Jorgensen has
written about the ``tale of coherence'' that those accounts provided by
glossing over separations and differences which were typi-

\enlargethispage{2\baselineskip}

\vspace*{2em}

\noindent{\emph{History of Media Studies}, vol. 1, 2021}




 \end{titlepage}


\noindent cal for the
development of ``the messy field of mass communication
study.''\footnote{Karin Wahl-Jorgensen, ``Rebellion and Ritual in Disciplinary Histories
  of U.S. Mass Communication Study: Looking for `The Reflexive
  Turn,'\,'' \emph{Mass Communication \& Society} 3, no. 1 (2000): 90,
  98.
}

What is this ``plastic phrase,''\footnote{Dietmar Rost, ``Reviewed Work(s): Kollektive Identität. Heimliche
  Quellen einer unheimlichen Konjunktur by Lutz Niethammer,``
  \emph{Historical Social Research}. 28, no. 4 (2003): 188.
}
collective identity, worth? This essay does not intend to prescribe a
disciplinary collective identity but rather to suggest reflecting on the
analytical potential of this concept. Regarding theory, this essay is
based on the sociology of Pierre Bourdieu and the works of Jan Assmann,
the German Egyptologist. Their theoretical tools (personal and
collective identity, cultural memory, symbolic power, habitus, capital,
and field) lead to an understanding of collective identity as a social
construct that contains selected and formed ideas about a collective as
well as its memories of the past. Collective identities are constructed
in symbolic practices, rooted in habitus--capital--field constellations,
and linked to power. They may lead to inclusion and exclusion.
Collective identity reveals how communication studies were historically
tied to symbolic systems in society and what degree of autonomy they
had.\footnote{Assmann, \emph{Das kulturelle Gedächtnis}, 132; Jan Assmann and John
  Czaplicka, ``Collective Memory and Cultural Identity,'' \emph{New
  German Critique} 65 (1995): 130; Pierre Bourdieu, \emph{Language and
  Symbolic Power} (Cambridge: Polity Press, 1992), 167.
} The aim of the historiography
suggested here is to determine which symbolic formation of a ``we'' and
``the others'' the discipline has contributed to over the course of
time.\footnote{Eisenstadt and Giesen, \emph{The Construction of Collective Identity},
  74.
} This applies to groups in
society and to the discipline itself.

Studying the exclusion or marginalization of theories and groups in the
history of the field helps to discover and recover forgotten or
marginalized alternative ways of thinking and approaches to media and
communication. Moreover, it enables us to reflect on the discipline's
involvement in political and manipulatory uses of the normative concepts
of collective identities. An example of such an involvement is German
communication studies and its past in National
Socialism.\footnote{Christopher Simpson, ``Elisabeth Noelle-Neumann's `Spiral of Silence'
  and the Historical Context of Communication Theory,'' \emph{Journal of
  Communication} 46 (1996).
}

The concept of collective identity is not a typical concept deployed in
research on inclusion and exclusion in the discipline---for example, in
feminist and cosmopolitan approaches. But it could complement and, to
some extent, bundle such perspectives because there is the shared
concern to reveal and discuss ``asymmetries for exclusion or inclusion
of scholarly voices'' and other voices in
society.\footnote{Hanan Badr and Sarah Anne Ganter, ``Towards Cosmopolitan Media and
  Communication Studies: Bringing Diverse Epistemic Perspectives into
  the Field.'' \emph{Global Media Journal,} \emph{German Edition} 11,
  no. 1 (2021): 2; cf. Lana F. Rakow, ``Feminist Historiography and the
  Field,'' in \emph{The History of Media and Communication Research:
  Contested Memories}, eds. David W. Park and Jefferson Pooley (New
  York: Peter Lang, 2008); Martina Thiele, ``Female Academics in
  Communication Science and the Post-War Reconstruction Generation in
  Austria and Germany,'' in \emph{The International History of
  Communication Study,} eds. Peter Simonson and David W. Park (New York:
  Routledge, 2016).
}

\hypertarget{historical-perspectives-on-collective-identity-and-communication-studies}{%
\section{Historical Perspectives on Collective Identity and
Communication
Studies}\label{historical-perspectives-on-collective-identity-and-communication-studies}}

There are at least two ways the usage of the term collective identity
can be distinguished. In the first, which involves \emph{reconstructive}
understanding, collective identity describes the idea a group develops
of itself by identifying similar self-views and worldviews, experiences,
and expectations on the part of single members of the group. In the
second, there is a \emph{prescriptive} use of the term. It ascribes
common characteristics, historical continuity, and coherence to members
of a group. However, this group may be merely imagined. Collective
identity in this case is a construct detached from the memories, values,
and orientations of individuals. Normative collective identity concepts
are predetermined to serve political
interests.\footnote{Assmann, \emph{Das kulturelle Gedächtnis}, 132; Jürgen Straub,
  ``Personale und kollektive Identität. Zur Analyse eines theoretischen
  Begriffs,'' in \emph{Identitäten. Erinnerung, Geschichte, Identität},
  eds. Aleida Assmann and Heidrun Friese (Frankfurt am Main: Suhrkamp,
  1998), 98--99, 102.
}

These two perspectives can be discussed in terms of what their (1)
\emph{analytical value} is for historical studies on communication
studies. (2) Moreover, knowing that there are reconstructive and
prescriptive ways of understanding collective identity can be helpful to
classify findings. In this case, prescriptive and reconstructive
perspectives are the \emph{result of historical analysis}. (1) The
\emph{analytical} value of a prescriptive usage is low because this
usage aims at pushing a certain normative idea of a collective.
Therefore, I suggest applying the reconstructive perspective as an
analytical tool. There are two subject matters which can be studied by
applying the reconstructive perspective: firstly, collective identities
in studies about public communication and secondly, collective
identities of communication studies. These two subject matters are
described below. (2) A reconstructive understanding will have to
classify in which way the discipline has used the concept of collective
identity at different moments in history. Has it been the reconstructive
or the prescriptive way, or yet another way, and what were the reasons
for this? The reconstructive or the prescriptive use of collective
identity therefore \emph{might be the result of historical analysis}.

Arguing that a prescriptive perspective on collective identity is a
normative one, shall not point to the conclusion that my whole essay is
free of normative charge. Focusing on issues such as exclusion and
inclusion, like I do here, is, of course, a normative decision.

\hypertarget{subject-matter-1-collective-identities-in-studies-about-public-communication}{%
\section{Subject Matter 1: Collective Identities in Studies about
Public\\\noindent Communication}\label{subject-matter-1-collective-identities-in-studies-about-public-communication}}

In terms of collective identity construction, the media have held
``positions of power'' because, depending on the particular media
system, they have shaped prevailing societal representations of groups
and perpetuated cognitive representations of those
groups.\footnote{Jake Harwood and Abhik Roy, ``Social Identity Theory and Mass
  Communication Research,'' in \emph{Language as Social Action.
  Intergroup Communication}, eds. Jake Harwood and Howard Giles (New
  York: Peter Lang, 2005), 191.
} Communication studies'
work gives insights into how the discipline has been linked to
collective identity construction in public communication. But the
patterns of social categorization and boundary construction that have
appeared in publications are not the only objects of historiography.
Such a history could also reveal to what extent research fields such as
media usage, media effects, and media content reproduced or questioned
the dominant principles of division perpetuated by the media or by
politics. Are there recurring ideas, approaches, and methodological
principles within the ``specific symbolic
universe''\footnote{Bourdieu, \emph{Language and Symbolic Power}, 124.
} of communication
studies that legitimated or opposed societal exclusion and inclusion
(e.g., regarding class, geography/``first'' and ``third'' worlds,
nations, gender, ethnicity)? Using stereotypes has always been integral
to the symbolic practices of media audiences in Europe---for instance,
cinema audiences. Women, children, or subordinate classes have been
ascribed characteristics such as being emotional, pleasure-seeking, and
suggestible. Such descriptions were incorporated into early theories on
cinema.\footnote{Richard Butsch, ``Audiences and Publics, Media and Public Spheres.''
  In \emph{The Handbook of Media Audiences}, ed. Virginia Nightingale
  (Chichester and Malden, MA: Wiley-Blackwell, 2011), 156.
} Audience studies about
socialist countries largely applied citizen/consumer categorizations
derived from Western systems. The experiences and living conditions of
users in the East were obviously ``never made part of this
narrative.''\footnote{Irena Reifová, ``A Study in the History of Meaning-making: Watching
  Socialist Television Serials in the Former Czechoslovakia,''
  \emph{European Journal of Communication} 30, no. 1 (2015), 80.
} Dichotomies such as
citizen/consumer or active/passive defined what was the established
ideal at a given moment of history: ``What is good, deserves reward,
power, privilege.''\footnote{Richard Butsch, \emph{The Making of American Audiences. From State to
  Television, 1750--1990} (Cambridge: Cambridge University Press, 2000),
  2.
} They are
linked to hegemonic symbolic
systems.\footnote{Bourdieu, \emph{Language and Symbolic Power}, 132.
} To understand the
mechanisms of symbolic power, analysis must include who published where,
with what habitus, social position, and geopolitical location: Who got
to speak for the field and shaped its contours, in which local,
national, and global settings?\footnote{Badr and Ganter, ``Towards Cosmopolitan Media and Communication
  Studies,'' 2.
}

\hypertarget{subject-matter-2-collective-identities-of-communication-studies}{%
\section{Subject Matter 2: Collective Identities of Communication
Studies}\label{subject-matter-2-collective-identities-of-communication-studies}}

This perspective sheds light on the way communication studies
collectives constructed a ``we'' for themselves that, for some time, was
binding. It gives insights into what past they invented for themselves
and which members of these collectives held symbolic power in such
self-definitions. ``Great men, great events, great places'' have been
used, for instance, in the North American and Western European context
to form a historical canon.\footnote{Rakow, ``Feminist Historiography and the Field,'' 115; cf. Katharina
  Wischmeyer, ``\,`Ungleich unter Gleichen`. Frauen am Berliner Institut
  für Publizistik und der Fall Elke Baur,'' in \emph{`Regierungszeit des
  Mittelbaus'? Annäherungen an die Berliner Publizistikwissenschaft nach
  der Studentenbewegung,} eds. Maria Löblich and Niklas Venema (Köln:
  Halem, 2020).
}
Collective identity construction in the academic world may serve the
purpose of self-reflection (one's own cognitive and social principles),
of integrating members, and of legitimating the field to the outside.
Furthermore, this perspective reveals structures and practices regarding
boundary construction and exclusion. Social categories such as class,
ethnicity, and gender charged with specific value judgments at different
moments in history shaped appointment and employment procedures,
curriculum building, distribution of seats and positions in committees
and associations, and informal academic
networks.\footnote{Dow, ``The Lessons of History''; Martina Thiele, ``Female Academics in
  Communication Science.''
} What were the
consequences for the institutionalization of perspectives, ideas, and
theories? Furthermore, the field was influenced by larger collective
identity processes regarding nations and regions. The regionalization of
social sciences in Latin America would serve as an example. When ``the
notion of the Latin American unity'' was no longer based on shared
colonization aftereffects and common languages but on the idea to ``make
sense of each nation or sub-region,'' promoted by political movements, a
new ideal of ``(Latin) Americanism'' also shaped ideas regarding
cooperation and integration of social scientific communities in
different Latin American
countries.\footnote{Gustavo Sorá and Alejandro Blanco, ``Unity and Fragmentation in the
  Social Sciences in Latin America,'' in \emph{The Social and Human
  Sciences in Global Power Relations. Socio-Historical Studies of the
  Social and Human Sciences}, eds. Johan Heilbron, Gustavo Sorá and
  Thibaud Boncourt (Cham: Springer International Publishing, 2009), 128.
}

\hypertarget{an-example-from-germany}{%
\section{An Example from Germany}\label{an-example-from-germany}}

The German media discourse on East Germany provides an example of how a
prescriptive collective identity in public communication was reproduced
in scholarly works. A one-sided, negative interpretation of the German
Democratic Republic (GDR) was constructed in the national media after
1989 and has largely prevailed until today. It typically reduced the
socialist country to dictatorship, to persons killed at the borders, and
to the Fall of the Berlin Wall.\footnote{Hans-Jörg Stiehler, ``\,`Eine eigenartige Wendung.' Warum die
  überregionale Presse in Ostdeutschland scheitert. Gespräch mit
  Hans-Jörg Stiehler,'' in \emph{Wie die Medien zur Freiheit kamen. Zum
  Wandel der ostdeutschen Medienlandschaft seit dem Untergang der DDR,}
  eds. Michael Haller and Lutz Mükke (Köln: Halem, 2010), 254; Michael
  Meyen, ``\emph{Wir haben freier gelebt.`` Die DDR im kollektiven
  Gedächtnis der Deutschen} (Bielefeld: transcript, 2013).
}
The events of 1989 were framed as the East German way out of
``dictatorship'' and into ``freedom,'' for the most part excluding the
everyday experiences in the GDR. The German reunification was recounted
as a story of success.\footnote{Markus Böick, Constantin Goschler, and Ralph Jessen, ``Die deutsche
  Einheit als Geschichte der Gegenwart: Einleitung,'' in \emph{Jahrbuch
  Deutsche Einheit} \emph{2020} (Berlin: Ch. Links, 2020); Martin
  Sabrow, ``\,`1989' als Erzählung'' (Bonn: Bundeszentrale für
  politische Bildung, 2021).
} East
Germans became the ``others,'' whose behavior had to be explained to
``us,'' the majoritarian West German
society.\footnote{Thomas Ahbe, ``Die Ost-Diskurse als Strukturen der Nobilitierung und
  Marginalisierung von Wissen,'' in \emph{Die Ostdeutschen in den
  Medien. Das Bild von den Anderen nach 1990}, eds\emph{.} Thomas Ahbe,
  Rainer Gries, and Wolfgang Schmale (Leipzig: Leipziger
  Universitätsverlag, 2009), 59--112.
} Even today, critics say
that media coverage about East Germany is pejorative and unbalanced. Its
issues have been right-wing radicalism, under-education in terms of
democracy, and the inability to cope with a market society. The national
press located in West Germany and under West German ownership held the
interpretive primacy of these ``delegitimating
discourses.''\footnote{Stiehler, ``\,`Eine eigenartige Wendung,'\,'' 254.
}

Central elements of these discourses appeared in communication studies.
Insofar as communication studies after the political turn dealt with
East Germany at all, publications focused on press concentration and on
the adjustment of East German journalists to the requirements and norms
of the Western media system. Some examples demonstrate that these
sources are worth historicizing. They contain similar descriptions of
``the other'' German. Methodically, they did not aim at including the
remembrances and experiences of East Germans. ``In the communities of
the former GDR,'' reads one study, ``. . . there is a lack of democratic
and participatory traditions.''\footnote{Beate Schneider and Dieter Stürzebecher, \emph{Wenn das Blatt sich
  wendet. Die Tagespresse in den neuen Bundesländern} (Baden-Baden:
  Nomos, 1998), 212.
}
East German local communication, reads another, was characterized by ``a
fixation on the state and
authority.''\footnote{Schneider and Stürzebecher, \emph{Wenn das Blatt sich wendet,} 220.
} Another study
examined to what extent East German journalists had the ``capability to
investigate.''\footnote{Claudia Mast, Klaus Haasis, and Matthias Weigert, \emph{Medien und
  Journalismus im Umbruch. Konzepte und Erfahrungen von
  Medienunternehmen, Verbänden und Redakteuren in den neuen
  Bundesländern}, in \emph{Journalismus in den}}
  It also
investigated their ``profiles of
performance''\textsuperscript{29} and whether they
could ``cope'' with the ``density of work'' that the market organization
involved.\textsuperscript{30} Book titles at the end
of the second decade after reunification still reproduced the dominant
narratives regarding 1989 and 1990 (e.g., \emph{How the Media Came to
Freedom}\textsuperscript{31} and \emph{Media Freedom
after the Political Turn}\textsuperscript{32}).

The\marginnote{\emph{neuen Ländern: ein
  Berufsstand. zwischen Aufbruch und Anpassung}, eds. Frank Böckelmann,
  Claudia Mast and Beate Schneider (Konstanz: Universitätsverlag, 1994),
  256.
} social\marginnote{\textsuperscript{29} Mast, Haasis, and Weigert, \emph{Medien und Journalismus im Umbruch},
  291.
} field\marginnote{\textsuperscript{30} Mast, Haasis, and Weigert, \emph{Medien und Journalismus im Umbruch},
  293.
} structure\marginnote{\textsuperscript{31} Michael Haller and Lutz Mükke (eds.), \emph{Wie die Medien zur
  Freiheit kamen. Zum Wandel der ostdeutschen Medienlandschaft seit dem
  Untergang der DDR} (Köln: Halem, 2010).
} of\marginnote{\textsuperscript{32} Marcel Machill, Markus Beiler and Johannes Gerstenberg,
  \emph{Medienfreiheit nach der Wende Entwicklung von Medienlandschaft,
  Medienpolitik und Journalismus in Ostdeutschland} (Konstanz: UVK,
  2010).
}\setcounter{footnote}{32} communication studies and the habitus of
representatives, which can only be described roughly here, help to
understand these ascriptions. Throughout the 1990s, new institutes were
founded in East Germany. West German scholars were appointed as
professors. GDR professors and most academic staff had to leave the
institute in Leipzig, which had been the center for journalism education
in the GDR.\footnote{Michael Meyen, ``Der Ost-West-Gipfel vom Mai 1990,'' in
  \emph{Biografisches Lexikon der Kommunikationswissenschaft,} eds.
  Michael Meyen and Thomas Wiedemann (Köln: Halem, 2020).
} Communication studies
after 1989 has been primarily involved in quantitative mass
communication research. This paradigm had been dominant in the Federal
Republic of Germany since the 1960s. It led the discipline in West
Germany out of its crisis after National Socialism by demonstrating
allegiance to the United States, the alleged role model of democracy.
Walter J. Schütz (1930--2013, a founding member of the West German
communication studies' association in 1963, federal ministry official
involved in grant decisions, newspaper statistician), remembered that he
opposed several times the admission of those people ``who had
represented the system of media control'' to the
association.\footnote{Walter J. Schütz, ``Ich habe immer von Selbstausbeutung gelebt,'' in
  ``\emph{Ich habe dieses Fach erfunden.'' Wie die
  Kommunikationswissenschaft an die deutschsprachigen Universitäten
  kam,} eds. Michael Meyen and Maria Löblich (Köln: Halem, 2007), 55.
} Altogether, there
were limits in the interest of researching GDR media and East German
media development.\footnote{ Mandy Tröger, \emph{Pressefrühling und Profit}. \emph{Wie westdeutsche
  Verlage 1989/1990 den Osten eroberten} (Köln: Halem, 2019), 35.}

\hypertarget{conclusion}{%
\section{Conclusion}\label{conclusion}}

Communication studies, viewed through the collective identity lens, has
its historical narratives about ``we'' and ``the others.'' These
narratives are featured in publications, for instance, about media
audiences. Ideas of homogeneity and difference have also been developed
for the scholarly community itself. They may enable integration (and
communication studies' favorable presentation in the public), but may
also lead to exclusion and marginalization. Using the theories of Pierre
Bourdieu and Jan Assmann, communication (and communication research)
related to collective identities can be understood as social constructs
which are formed by symbolic power, habitus, capital, field structures,
personal identity, and cultural memory.

The aim of the historiography suggested here is to determine to which
symbolic formation of a ``we'' and ``the others,'' in and outside the
media, the discipline has contributed. As such formations have their
normative side, such a historiography helps to understand particular
power relations within and outside the field. It reveals, furthermore,
what degree of autonomy communication studies had and which dependencies
it shaped in the course of time.





\newpage

\section{Bibliography}\label{bibliography}

\begin{hangparas}{.25in}{1} 



Ahbe, Thomas. ``Die Ost-Diskurse als Strukturen der Nobilitierung und
Marginalisierung von Wissen,`` in \emph{Die Ostdeutschen in den Medien.
Das Bild von den Anderen nach 1990,} edited by Thomas Ahbe, Rainer
Gries, and Wolfgang Schmale, 59--12. Leipzig: Leipziger
Universitätsverlag, 2009.

Assmann, Jan. \emph{Das kulturelle Gedächtnis. Schrift, Erinnerung und
politische Identität in frühen Hochkulturen}. München: C. H. Beck, 1992.

Assmann, Jan, and John Czaplicka. ``Collective Memory and Cultural
Identity.'' \emph{New German Critique} 65 (1995): 125--33.
\url{https://doi.org/10.2307/488538}.

Badr, Hanan and Sarah Anne Ganter. ``Towards Cosmopolitan Media and
Communication Studies: Bringing Diverse Epistemic Perspectives into the
Field.'' \emph{Global Media Journal}. \emph{German Edition} 11, no. 1
(2021). \url{https://doi.org/10.22032/dbt.491642021}.

Böick, Markus, Constantin Goschler, and Ralph Jessen. ``Die deutsche
Einheit als Geschichte der Gegenwart: Einleitung.'' In \emph{Jahrbuch
Deutsche Einheit} \emph{2020}. 9-23. Berlin: Christoph Links, 2020.

Bourdieu, Pierre. \emph{Language and Symbolic Power}. Cambridge: Polity
Press, 1992.

Butsch, Richard. \emph{The Making of American Audiences. From State to
Television, 1750-1990.} Cambridge: Cambridge University Press, 2000.

Butsch, Richard. ``Audiences and Publics, Media and Public Spheres.'' In
\emph{The Handbook of Media Audiences}, edited by Virginia Nightingale,
149--68. Malden, MA: Wiley-Blackwell, 2011.

Dow, Bonnie J. ``The Lessons of History: Women's Studies in
Communication Approaches 40.'' \emph{Women's Studies in Communication}
37 (2014): 259--61. \url{https://doi.org/10.1080/07491409.2014.955431}.

Eisenstadt, Shmuel Noah, and Bernhard Giesen. ``The Construction of
Collective Identity.'' \emph{European Journal of Sociology} 36, no. 1
(1995): 72--102. \url{https://doi.org/10.1017/S0003975600007116}.

Haller, Michael, and Lutz Mükke, eds. \emph{Wie die Medien zur Freiheit
kamen. Zum Wandel der ostdeutschen Medienlandschaft seit dem Untergang
der DDR}. Köln: Halem, 2010.

Harwood, Jake, and Abhik Roy. ``Social Identity Theory and Mass
Communication Research.'' In \emph{Language as Social Action. Intergroup
communication}, edited by Jake Harwood and Howard Giles, 189--211. New
York: Peter Lang, 2005.

Machill, Marcel, Markus Beiler, and Johannes Gerstenberg.
\emph{Medienfreiheit nach der Wende Entwicklung von Medienlandschaft,
Medienpolitik und Journalismus in Ostdeutschland}. Konstanz: UVK, 2010.

Mast, Claudia, Klaus Haasis, and Matthias Weigert. ``Medien und
Journalismus im Umbruch. Konzepte und Erfahrungen von Medienunternehmen,
Verbänden und Redakteuren in den neuen Bundesländern.'' In
\emph{Journalismus in den neuen Ländern: ein Berufsstand. zwischen
Aufbruch und Anpassung}, edited by Frank Böckelmann, Claudia Mast and
Beate Schneider, 231--450. Konstanz: Universitätsverlag, 1994.

Meyen, Michael. ``\emph{Wir haben freier gelebt.'' Die DDR im
kollektiven Gedächtnis der Deutschen}. Transcript. Bielefeld, 2013.

Meyen, Michael. ``Der Ost-West-Gipfel vom Mai 1990.'' In
\emph{Biografisches Lexikon der Kommunikationswissenschaft,} edited by
Michael Meyen and Thomas Wiedemann. Köln: Halem, 2020.
\url{http://blexkom.halemverlag.de/ost-west-gipfel}.

Pooley, Jefferson, and David W. Park. ``Introduction.'' In \emph{The
History of Media and Communication Research. Contested Memories}, edited
by David W. Park and Jefferson Pooley, 1--15. New York: Peter Lang,
2008.

Rakow, Lana F. ``Feminist Historiography and the Field.'' In \emph{The
History of Media and Communication Research. Contested Memories}, edited
by David W. Park and Jefferson Pooley, 113--39. New York: Peter Lang,
2008.

Reifová, Irena. ``A Study in the History of Meaning-making: Watching
Socialist Television Serials in the Former Czechoslovakia.''
\emph{European Journal of Communication} 30, no. 1 (2015): 79--94.

Rost, Dietmar. ``Reviewed Work(s): Kollektive Identität. Heimliche
Quellen einer unheimlichen Konjunktur by Lutz Niethammer.''
\emph{Historical Social Research} 28, no. 4 (2003): 188--202.

Sabrow, Martin. \emph{``1989'' als Erzählung}. Bonn: Bundeszentrale für
politische Bildung, 2021.
\url{http://www.bpb.de/apuz/295464/1989-als-erzaehlung}.

Schneider, Beate, and Dieter Stürzebecher. \emph{Wenn das Blatt sich
wendet. Die Tagespresse in den neuen Bundesländern}. Baden-Baden: Nomos,
1998.

Schütz, Walter J. ``Ich habe immer von Selbstausbeutung gelebt.'' In
``\emph{Ich habe dieses Fach erfunden.'' Wie die
Kommunikationswissenschaft an die deutschsprachigen Universitäten kam,}
edited by Michael Meyen and Maria Löblich, 33--58. Köln: Halem, 2007.

Simonson, Peter, and David W. Park. ``Introduction: On the History of
Communication Study.'' In \emph{The International History of
Communication Study}, edited by Peter Simonson and David W. Park, 1--22.
New York: Routledge, 2016.

Simpson, Christopher. ``Elisabeth Noelle-Neumann's `Spiral of Silence'
and the Historical Context of Communication Theory.'' \emph{Journal of
Communication} 46 (1996): 149--73.
\url{https://doi.org/10.1111/j.1460-2466.1996.tb01494.x}.

Sorá, Gustavo, and Alejandro Blanco. ``Unity and Fragmentation in the
Social Sciences in Latin America.'' In \emph{The Social and Human
Sciences in Global Power Relations. Socio-Historical Studies of the
Social and Human Sciences}, edited by Johan Heilbron, Gustavo Sorá and
Thibaud Boncourt, 127--52. Cham: Springer International Publishing,
2009.

Stiehler, Hans-Jörg. ``\,`Eine eigenartige Wendung.' Warum die
überregionale Presse in Ostdeutschland scheitert. Gespräch mit Hans-Jörg
Stiehler.'' In \emph{Wie die Medien zur Freiheit kamen. Zum Wandel der
ostdeutschen Medienlandschaft seit dem Untergang der DDR,} edited by
Michael Haller and Lutz Mükke, 248--60. Köln: Halem, 2010.

Straub, Jürgen. ``Personale und kollektive Identität. Zur Analyse eines
theoretischen Begriffs.'' In \emph{Identitäten: Erinnerung, Geschichte,
Identität}, edited by Aleida Assmann and Heidrun Friese, 73--104.
Frankfurt am Main: Suhrkamp, 1998.

Thiele, Martina. ``Female Academics in Communication Science and the
Post-War Reconstruction Generation in Austria and Germany.'' In
\emph{The International History of Communication Study,} edited by Peter
Simonson and David W. Park, 130--50. New York: Routledge, 2016.

Tröger, Mandy. \emph{Pressefrühling und Profit}. \emph{Wie westdeutsche
Verlage 1989/1990 den Osten eroberten}. Köln: Halem, 2019.

Wahl-Jorgensen, Karin. ``Rebellion and Ritual in Disciplinary Histories
of U.S. Mass Communication Study: Looking for `The Reflexive Turn.'\,''
\emph{Mass Communication \& Society} 3, no. 1 (2000): 87--115.
\url{https://doi.org/10.1207/S15327825MCS0301_05}.

Wendelin, Manuel, and Elisabeth Noelle-Neumann. \emph{Biografisches
Lexikon der Kommunikationswissenschaft,} edited by Michael Meyen and
Thomas Wiedemann. Köln: Halem, 2013.
\url{http://blexkom.halemverlag.de/elisabeth-noelle-neumann}.

Wischmeyer, Katharina. ``\,`Ungleich unter Gleichen.' Frauen am Berliner
Institut für Publizistik und der Fall Elke Baur.'' In
\emph{``Regierungszeit des Mittelbaus?'' Annäherungen an die Berliner
Publizistikwissenschaft nach der Studentenbewegung}, edited by Maria
Löblich and Niklas Venema, 431--50. Köln: Halem, 2020.



\end{hangparas}


\end{document}