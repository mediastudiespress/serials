% see the original template for more detail about bibliography, tables, etc: https://www.overleaf.com/latex/templates/handout-design-inspired-by-edward-tufte/dtsbhhkvghzz

\documentclass{tufte-handout}

%\geometry{showframe}% for debugging purposes -- displays the margins

\usepackage{amsmath}

\usepackage{hyperref}

\usepackage{fancyhdr}

\usepackage{hanging}

\hypersetup{colorlinks=true,allcolors=[RGB]{97,15,11}}

\fancyfoot[L]{\emph{History of Media Studies}, vol. 2, 2022}


% Set up the images/graphics package
\usepackage{graphicx}
\setkeys{Gin}{width=\linewidth,totalheight=\textheight,keepaspectratio}
\graphicspath{{graphics/}}

\title[Antonio Pasquali]{Antonio Pasquali. Una práctica intelectual entre América Latina y Europa (1979–1989)} % longtitle shouldn't be necessary

% The following package makes prettier tables.  We're all about the bling!
\usepackage{booktabs}

% The units package provides nice, non-stacked fractions and better spacing
% for units.
\usepackage{units}

% The fancyvrb package lets us customize the formatting of verbatim
% environments.  We use a slightly smaller font.
\usepackage{fancyvrb}
\fvset{fontsize=\normalsize}

% Small sections of multiple columns
\usepackage{multicol}

% Provides paragraphs of dummy text
\usepackage{lipsum}

% These commands are used to pretty-print LaTeX commands
\newcommand{\doccmd}[1]{\texttt{\textbackslash#1}}% command name -- adds backslash automatically
\newcommand{\docopt}[1]{\ensuremath{\langle}\textrm{\textit{#1}}\ensuremath{\rangle}}% optional command argument
\newcommand{\docarg}[1]{\textrm{\textit{#1}}}% (required) command argument
\newenvironment{docspec}{\begin{quote}\noindent}{\end{quote}}% command specification environment
\newcommand{\docenv}[1]{\textsf{#1}}% environment name
\newcommand{\docpkg}[1]{\texttt{#1}}% package name
\newcommand{\doccls}[1]{\texttt{#1}}% document class name
\newcommand{\docclsopt}[1]{\texttt{#1}}% document class option name


\begin{document}

\begin{titlepage}

\begin{fullwidth}
\noindent\LARGE\emph{Exclusions in the History of Media Studies
} \hspace{25mm}\includegraphics[height=1cm]{logo3.png}\\
\noindent\hrulefill\\
\vspace*{1em}
\noindent{\Huge{Antonio Pasquali. Una práctica intelectual entre América Latina y Europa (1979–1989)\par}}

\vspace*{1.5em}

\noindent\LARGE{Emiliano Sánchez Narvarte}\par}\marginnote{\emph{Emiliano Sánchez Narvarte, ``Antonio Pasquali: una práctica intelectual entre América Latina y Europa (1979–1989),'' \emph{History of Media Studies} 2 (2022), \href{https://doi.org/10.32376/d895a0ea.0444937b}{https://doi.org/ 10.32376/d895a0ea.0444937b}.} \vspace*{0.75em}}
\vspace*{0.5em}
\noindent{{\large\emph{Universidad Nacional de La Plata},\par}} \marginnote{\href{https://creativecommons.org/licenses/by-nc/4.0/}{\includegraphics[height=0.5cm]{by-nc.png}}}
\noindent{{\large \href{mailto:sancheznarvarteemiliano@gmail.com}{sancheznarvarteemiliano@gmail.com}\par}}

% \vspace*{0.75em} % second author

% \noindent{\LARGE{<<author 2 name>>}\par}
% \vspace*{0.5em}
% \noindent{{\large\emph{<<author 2 affiliation>>}, \href{mailto:<<author 2 email>>}{<<author 2 email>>}\par}}

% \vspace*{0.75em} % third author

% \noindent{\LARGE{<<author 3 name>>}\par}
% \vspace*{0.5em}
% \noindent{{\large\emph{<<author 3 affiliation>>}, \href{mailto:<<author 3 email>>}{<<author 3 email>>}\par}}

\end{fullwidth}

\vspace*{2.5em}


\hypertarget{resumen}{%
\section{Resumen}\label{resumen}}

El objetivo de este trabajo es analizar cómo la participación de Antonio
Pasquali en organismos internacionales dedicados a la cultura
---principalmente desde la UNESCO--- potenció la presencia de los
estudios latinoamericanos de comunicación en el debate internacional.
Pensar su trayectoria en espacios de investigación y cooperación en
comunicación transnacionales, nos va a permitir explorar las condiciones
de producción de diálogos interinstitucionales que viabilizaron la
formación de asociaciones regionales, como la Asociación Latinoamericana
de Investigadores de la Comunicación (ALAIC). Para cumplir con ese
objetivo, vamos a pensar a Pasquali como un mediador intelectual que
operó como conector entre espacios de producción intelectual
transnacionales y con tradiciones diversas. Seguir la trayectoria
académica e institucional de Pasquali entre 1979 y 1989, dada su
conexión con espacios múltiples de organización cultural e investigación
comunicacional, permitirá analizar las tensiones que atravesaron el
proceso de institucionalización de los estudios de comunicación
latinoamericanos a lo largo de los años ochenta del siglo XX.

\hypertarget{abstract}{%
\section{Abstract}\label{abstract}}

The objective of this work is to analyze how Antonio Pasquali's\\\noindent
participation in international organizations dedicated to culture ---


\enlargethispage{2\baselineskip}

\vspace*{2em}

\noindent{\emph{History of Media Studies}, vol. 2, 2022}


 \end{titlepage}

% \vspace*{2em} | to use if abstract spills over


\noindent mainly UNESCO --- enhanced the presence of Latin American communication
studies in international debates. Thinking about his trajectory in
transnational communication research and related spaces of cooperation
will allow us to explore conditions that produced inter-institutional
dialogues that helped form regional associations, such as the Latin
American Association of Communication Researchers (ALAIC). To fulfill
this objective, the essay casts Pasquali as an intellectual mediator who
operated as a connector between transnational spaces of intellectual
production and diverse traditions. Following the academic and
institutional trajectories of Pasquali between 1979-1989, given their
connection to multiple spaces of cultural organization and
communicational research, illuminates tensions involved in the process
through which Latin American communication studies were
institutionalized in the 1980s.

\vspace*{2em}

\enlargethispage{\baselineskip}


\hypertarget{introduccin}{%
\section{Introducción}\label{introduccin}}

\newthought{Antonio Pasquali} (1929-2019) es considerado uno de los teóricos e
intelectuales más destacados de los estudios en comunicación en América
Latina.\footnote{Migdalia Pineda de Alcázar, ``Antonio Pasquali: la
  vigencia de su pensamiento cuarenta años después'', en \emph{Travesía
  intelectual de Antonio Pasquali. A propósito de los 50 años de
  Comunicación y cultura de masas}, ed. por Marcelino Bisbal y Andrés
  Cañizález (Caracas: UCAB, 2014), 21-30; Elizabeth Safar, ``Una
  constante en la obra de Antonio Pasquali: el servicio público de
  radiotelevisión'', en \emph{Travesía intelectual de Antonio Pasquali},
  47-58; Jesús María Aguirre y Gustavo Hernández Díaz,
  \emph{Diccionario: investigadores venezolanos de la comunicación}
  (Caracas: UCAB, 2018); León Hernández, \emph{Pasquali. El último
  libro, la última entrevista y el último banquete} (Caracas: UCAB,
  2019).} Además de constituirse en una referencia académica para las
escuelas y facultades de comunicación de la región, entre 1978 y 1989
Pasquali ocupó distintos cargos de gestión en la Organización de las
Naciones Unidas para la Educación, la Ciencia y la Cultura (UNESCO).
Durante la dirección general de Amadou-Mahtar M'Bow en la UNESCO, entre
1978 y 1983 fue Sub-director General Adjunto del sector de la Cultura y
Comunicación, en 1984 ocupó la Sub-dirección General del Sector de las
Comunicaciones y, finalmente, entre 1986 y 1989 fue Coordinador Regional
de la UNESCO para América Latina y el Caribe, y Director del Centro
Regional de Educación Superior para América Latina y el Caribe (CRESALC)
con sede en Caracas.

Dada su notable trayectoria en el campo de los estudios en comunicación
y cultura en América Latina, el objetivo de este trabajo es analizar
cómo la participación de Antonio Pasquali en organismos supranacionales
dedicados a la cultura ---principalmente desde la UNESCO--- potenció la
presencia de los estudios latinoamericanos de comunicación en el debate
internacional. Pensar su trayectoria en espacios de investigación y
cooperación en comunicación internacionales, nos va a permitir explorar
las condiciones de producción de diálogos interinstitucionales que
viabilizaron la formación de asociaciones regionales, como la Asociación
Latinoamericana de Investigadores de la Comunicación (ALAIC).

Para cumplir con ese objetivo retomamos la conceptualización de
``mediador intelectual'' que Mariano Zarowsky utilizó en su análisis
sobre Armand Mattelart,\footnote{Mariano Zarowsky, \emph{Del laboratorio
  chileno a la comunicación-mundo. Un itinerario intelectual de Armand
  Mattelart} (Buenos Aires: Biblos, 2013).} para pensar a Antonio
Pasquali como \emph{un agente que operó como conector entre espacios de
producción intelectual transnacionales y con tradiciones diversas}. Con
la idea de ``mediador'' se problematizarán, complementariamente, los
procesos de circulación internacional de las ideas entre Latinoamérica y
Europa en los dos sentidos del movimiento: del centro a la periferia y
de la periferia hacia el centro, movimiento que, consideramos, pone de
relieve distintas redes internacionales de sociabilidad
intelectual.\footnote{Carlos Altamirano, ``Elites culturales en el siglo
  XX latinoamericano'', \emph{en Historia de los intelectuales en
  América Latina. II. Los avatares de la ``ciudad letrada'' en el siglo
  XX}, ed. por Carlos Altamirano (Buenos Aires: Katz, 2010).} Un modo de
pensar lo transnacional, de acuerdo a Johan Heilbron, Nicolas Guilhot y
Laurent Jeanpierre, es estudiar ``las conexiones y la circulación de las
ideas, personas y productos culturales a través de las fronteras
nacionales''.\footnote{Johan Heilbron, Nicolas Guilhot y Laurent
  Jeanpierre, ``Toward a transnational history of the social sciences'',
  \emph{Journal of the History of the Behavioral Sciences} 44, n.° 2
  (primavera 2008): 146--160.} Desde esta perspectiva, lo transnacional
se constituye como un enfoque histórico particular que se centra en los
movimientos, flujos, circulación e intersección de personas, ideas y
bienes a través de fronteras políticas y culturales.\footnote{Joy
  Damousi y Mariano Ben Plotkin, eds., \emph{The Transnational
  Unconscious Essays in the History of Psychoanalysis and
  Transnationalism} (Nueva York: Palgrave Macmillan, 2008); Gisèle
  Sapiro, Marco Santoro y Patrick Baert, \emph{Ideas on the Move in the
  Social Sciences and Humanities. The International Circulation of
  Paradigms and Theorists} (Basingstoke: Palgrave Macmillan, 2020).}

Raymond Williams plantea que al estudiar cualquier periodo pasado, lo
más difícil de aprehender es esa ``sensación vívida'' de la experiencia
particular: identificar ``cómo se combinaban las actividades específicas
en un modo de pensar y vivir''.\footnote{Raymond Williams, \emph{La
  larga revolución} (Buenos Aires: Nueva Visión, 2003), 56.} Para
trabajar sobre ello, tomamos la decisión teórico-metodológica de
realizar análisis documentales y elaborar entrevistas. Los testimonios,
sigue Williams, son claves para intentar aprehender y analizar las
``sensaciones'' y la ``experiencia concreta'' a través de las cuales los
agentes elaboraron ciertas reflexiones. La utilización del testimonio
oral se presenta como una instancia para dar cuenta, retrospectivamente,
del ``registro reflexivo sobre una acción'', según Anthony
Giddens,\footnote{Anthony Giddens, \emph{La constitución de la sociedad.
  Bases para la teoría de la estructuración} (Buenos Aires: Amorrortu,
  2006), 24.} porque permite destacar la vivencia de los actores
respecto a su propia práctica e itinerario vital, esto es, incorporar el
sentido subjetivo que estos le daban en cada momento a su acción. Según
afirma el historiador François Dosse, la utilización del testimonio oral
permite que emerja ``el momento de subjetivación del instante'' que da
cuenta de una dimensión de ``intensa afectividad''.\footnote{François
  Dosse, \emph{La marcha de las ideas. Historia de los intelectuales,
  historia intelectual} (Valencia: Universitat de Valéncia, 2006), 273.}

Seguir la trayectoria académica e institucional de Pasquali en este
periodo, dada su conexión con espacios múltiples de organización
cultural e investigación comunicacional, permitirá analizar las
tensiones que atravesaron el proceso de institucionalización de los
estudios en comunicación latinoamericanos a lo largo de los años
ochenta. Para dicho propósito, analizaremos la conformación de
encuentros y congresos académicos en la región, fundamentalmente los
debates que se produjeron en 1980 en la XII Asamblea General y
Conferencia de la Association Internacionale des Études et Recherches
Sur L'Information et la Communication (AIERI) que organizó en Caracas el
Instituto de Investigaciones de la Comunicación (ININCO), como así
también los debates intelectuales suscitados en las revistas de
investigación más importantes de esa época, como \emph{Chasqui} y
\emph{Comunicación y Cultura}.

En un contexto integral, que incluye a otros académicos como Juan
Somavía y Fernando Reyes Matta, entre otros, nos preguntamos acerca del
papel que jugaron los intelectuales latinoamericanos en la producción de
las ideas en comunicación. Es decir, indagar la posición que ocupó
Venezuela en particular y América Latina en general, como polo de
producción de saberes en comunicación, en el marco de los debates
internacionales sobre la circulación y el flujo de la información.

\hypertarget{de-la-academia-a-la-unesco}{%
\section{De la academia a la
UNESCO}\label{de-la-academia-a-la-unesco}}

Hacia finales de los años setenta, Antonio Pasquali era considerado como
uno de los más importantes teóricos de la comunicación en América
Latina. Sus libros \emph{Comunicación y cultura de masas} {[}EBUC,
Caracas, 1964{]}, \emph{El aparato singular} {[}UCV, Caracas, 1967{]} y
\emph{Comprender la comunicación} {[}Monte Ávila, Caracas, 1978{]}, le
dieron visibilidad a nivel regional; dictaba, además, cursos y
seminarios en distintas universidades sudamericanas. Al mismo tiempo,
era un referente en términos de elaboración de políticas culturales y
comunicacionales a partir de su experiencia en Venezuela como director
del Comité de Radio y Televisión durante el primer gobierno de Carlos
Andrés Pérez (1974-1979), en el que se diagramaron políticas vinculadas
a la democratización de la producción cultural y comunicacional.

El proyecto de una radiodifusión y televisión del Estado venezolano se
denominó Proyecto RATELVE. La orientación general de la propuesta del
equipo liderado por Pasquali\footnote{Además de Pasquali, entre quienes
  participaron se encuentran los referentes teóricos Héctor Mujica,
  Oswaldo Capriles, Raúl Agudo Freites y Aníbal Gómez, Francisco Tugues
  del Consejo Nacional de Investigaciones Científicas y Tecnológicas,
  Ovidio Pérez Morales, sacerdote de Caracas, y Hely Santeliz, como
  representante de las Fuerzas Armadas.} entendía a la comunicación en
estrecha relación con la organización política de lo social. El Estado
debía convertirse en el rector de la comunidad para defender y promover
la libertad, la igualdad, el desarrollo y la independencia del pueblo.
La clave pasaba por entender a la radiodifusión como un servicio público
definido como aquel que, administrado por el Estado, debía servir a los
intereses colectivos, ``armonizando el uso de la radio y la televisión
con los objetivos de desarrollo nacional''.\footnote{Consejo Nacional de
  la Cultura, \emph{Proyecto RATELVE} (Caracas: Suma, 1977), 41.} Para
ser un ``auténtico servicio público'', la radiodifusión debía adaptarse
a las necesidades sociales y culturales de la comunidad. La necesidad
fundamental, según el proyecto, era que todos los ciudadanos tuvieran
acceso al uso de la radiodifusión como un derecho otorgado por el Estado
con fines educativos, informativos y entretenimiento, orientado a la
integración nacional y cultural del país.\footnote{Consejo Nacional de
  la Cultura, \emph{Proyecto...}, 44.}

Entre 1975 y 1976, Pasquali no solo era una figura destacada a nivel
nacional sino que articulaba espacios locales vinculados a la
investigación sobre políticas de comunicación con los organismos
internacionales. Desde enero de 1975 había consolidado su vínculo con la
UNESCO y se comenzó a ``estudiar la posibilidad y conveniencia de asumir
un encargo regional''.\footnote{Oficio n.° IIC- 158/75, 20 de enero de
  1975, expediente Pasquali Greco, Antonio Arnaldo, periodo A,
  clasificación D.67. Archivo Histórico de la Facultad de Humanidades y
  Educación (AH-FHyE-UCV).} En paralelo, en 1976 Pasquali participó como
delegado del gobierno venezolano en la Reunión de Expertos
gubernamentales que tuvo lugar en la sede de la UNESCO en París para
producir el documento ``Acceso y participación de las masas populares en
la cultura''. En calidad de experto internacional y simultáneamente como
``hombre de Estado'', el mismo año participó en la Reunión Preparatoria
de la Conferencia Intergubernamental sobre Políticas Culturales a
desarrollarse en Jamaica.\footnote{Oficio n.° IIC-512/76, 29 de
  septiembre de 1976, expediente Pasquali Greco, Antonio Arnaldo,
  periodo A, clasificación D.67. (AH-FHyE-UCV).} Uno de los espacios
claves desde los cuales se conectaba lo nacional y lo internacional, era
el ININCO, que había sido fundado por Pasquali en 1974 en Caracas, y fue
su primer director hasta 1978. Elizabeth Safar, por entonces miembro del
ININCO, recuerda que los primeros proyectos de investigación habían sido
realizados para la UNESCO, por ejemplo, estudios sobre políticas de
comunicación, informes para el Centro Interamericano de Adiestramiento
en Comunicación para Planificación Familiar y Población, análisis para
el Instituto Nacional de la Vivienda e investigaciones sobre la
legislación de la comunicación en el país.\footnote{Elizabeth Safar,
  entrevistada por el autor, 10 de marzo de 2016, Caracas, Venezuela.}

Tras una serie de reuniones con el organismo, el 14 de septiembre de
1978 Pasquali asumió formalmente el cargo de Sub-director General
Adjunto para el Sector de Cultura y Comunicación. Amadou-Mahtar M'Bow,
director general de la UNESCO, consideraba que la incorporación de
Pasquali se debía a su experiencia como ``presidente del Comité para el
Estudio de la Política de Radiodifusión del Estado y como miembro
principal del {[}Consejo Nacional de la Cultura{]} CONAC''. También, se
estimaban sus relaciones con el organismo desde 1976, dado que Pasquali
había sido ``consultor para las conferencias sobre políticas de
comunicación y políticas culturales en América Latina y el
Caribe''.\footnote{Amadou-Mahtar M'Bow, ``Sous-Directeuer général
  adjoint (Programme), Secteur de la culture et de la communication'',
  archivo personal de Antonio Pasquali, París, 20 de septiembre de 1978.}
El teórico contaba con el apoyo del gobierno venezolano. En una carta
dirigida a Pasquali, el Ministro de Estado para Ciencia y Tecnología,
José Luis Salcedo-Bastardo, le manifestaba a aquel que ``estaba seguro''
que no ``defraudaría el apoyo'' que le había brindado ``el Presidente
Carlos Andrés Pérez, ni la confianza que siempre ha{[}bía{]} tenido en
sus méritos personales y docentes''.\footnote{República de Venezuela,
  Ministro de Estado para Asuntos Científicos, Tecnológicos y
  Culturales, ``Carta del Ministro de Estado José Luis Salcedo-Bastardo
  a Antonio Pasquali'', archivo personal de Antonio Pasquali, 25 de
  julio de 1978.}

Entendemos que la \emph{expertise} desarrollada por Pasquali en su
trabajo sobre las relaciones entre medios de comunicación y la
formulación de políticas públicas, se había vuelto un capital valorado
por la UNESCO dado el proceso que estaba atravesando la institución. Era
un contexto, sostienen Fernando Quirós y Francisco Sierra
Caballero,\footnote{Fernando Quirós y Francisco Sierra Caballero, eds.,
  \emph{El espíritu MacBride. Neocolonialismo, Comunicación-Mundo y
  alternativas democráticas} (Quito: Ediciones CIESPAL, 2016), 12.} en
el que desde la Conferencia de Argel de 1973, el Movimiento de Países No
Alineados había comenzado a actuar ``de forma concertada en la ONU y sus
organismos especializados'' para incorporar, en la agenda del debate
internacional, dos proyectos: uno vinculado a el Nuevo Orden Económico
Internacional y otro sobre el Nuevo Orden Internacional de la
Información (NOII). Los proyectos, que contaron con el apoyo del Grupo
de los 77,\footnote{El grupo incluía por entonces a 77 países en vías de
  desarrollo con el objetivo de cooperar en las discusiones que se
  desarrollaban al interior de la Organización de las Naciones Unidas.}
la Unión Soviética y los países socialistas, hicieron de la UNESCO un
escenario de disputa geopolítica.

En 1977, en el marco de la XIX Conferencia General en Nairobi, la UNESCO
había creado la Comisión Internacional de Estudio de los Problemas de
Comunicación\footnote{Presidida por Sean MacBride, de la misma
  participaban referentes del periodismo y la investigación de distintos
  países, en especial del ``tercer mundo''. Entre varios otros,
  participaron Elebe MaEkonzo (Zaire), Akporuaro Omu (Nigeria) como
  miembro de la Coordinación de la Información de los Países No
  Alineados, Leonid Zamietin (URSS), embajador, periodista y miembro del
  Soviet Supremo, el escritor y periodista colombiano Gabriel García
  Márquez, y el investigador chileno Juan Somavía, como Director
  Ejecutivo del Instituto Latinoamericano de Estudios Transnacionales
  (ILET).} con el objetivo de analizar el problema de la ``difusión
internacional de la información'' y de la ``dependencia en materia de
comunicación''. Se trataba de una situación, argumentaba la Comisión, en
la que se estaba ``poniendo en tela de juicio'' el ``orden mundial de la
información''.\footnote{UNESCO (Organización de las Naciones Unidas para
  la Educación, la Ciencia y la Cultura), \emph{Informe provisional
  sobre los problemas de comunicación en la sociedad moderna} (París:
  UNESCO, 1978), 11. Sobre los debates y las posiciones de la UNESCO en
  el Nuevo Orden Mundial de la Información y Comunicación, véase Esteban
  López-Escobar, \emph{Análisis del Nuevo Orden Internacional de la
  Información} (Pamplona: Ediciones Universidad de Navarra, 1978);
  Marcial Murciano, ``El Informe MacBride: la búsqueda imposible del
  consenso entre norte/sur y este/oeste'', \emph{Anàlisi}, n.° 3 (1981):
  109-119; Raquel Salinas, \emph{Agencias transnacionales de información
  y el tercer mundo} (Quito: The Quito Times, 1984); Alcira Argumedo,
  \emph{Los laberintos de la crisis. América Latina: poder transnacional
  y comunicaciones} (Buenos Aires: ILET-Folios Ediciones, 1984); Josep
  Gifreu, \emph{El debate internacional de la comunicación} (Pamplona:
  Ariel, 1986); Javier Esteinou Madrid, ``El rescate del Informe Mc
  Bride y la construcción de un nuevo orden mundial de la información'',
  \emph{Razón y Palabra}, n.° 39 (2004); Enrique Sánchez Ruiz,
  ``Actualidad del Informe Mac-} Al momento del
ingreso de Pasquali, la Comisión ya se había reunido en tres
oportunidades entre 1977 y 1978. En dichas reuniones se habían
establecido las ``preguntas principales'' que orientarían el trabajo y
las investigaciones: ``cómo garantizar el derecho a comunicar'', qué se
entendía por ``nuevo orden mundial de la información'' y qué ``por
circulación libre y equilibrada de la información''.\textsuperscript{21} Estas cuestiones que ocupaban a la UNESCO se
vinculaban a problemas que un sector de la intelectualidad en
comunicación de América Latina venía trabajando y desplegando. Sobre
todo entre los investigadores nucleados en el Instituto Latinoamericano
de Estudios Transnacionales (ILET) en México y del ININCO en Venezuela,
en el seminario sobre el papel de la información en el nuevo orden
internacional que se realizó en mayo de 1976 en México. Allí, tal como
relata Oswaldo Capriles ---que participó como miembro del ININCO---, los
tópicos centrales se vincularon a la ``\emph{nueva estructura jurídica
para las comunicaciones internacionales}'' y al ``\emph{papel de la
información en el nuevo orden internacional}''.\textsuperscript{22} Allí se denunciaba que los ``procesos
académicos'' reproducían ``conceptos y principios sobre la Información y
la Comunicación Social'' del ``primer mundo'' y no hacían ``énfasis'' en
las producciones académicas locales, es decir, en el ``aporte conceptual
del Tercer Mundo''.\textsuperscript{23}

\hypertarget{institucionalizar-la-investigacin-en-comunicacin-en-amrica-latina}{%
\section{Institucionalizar la investigación en comunicación en\\\noindent
América
Latina}\label{institucionalizar-la-investigacin-en-comunicacin-en-amrica-latina}}

A través de distintos encuentros y reuniones, como las organizadas por
la AIERI, se potenció la convergencia entre investigadores. Elizabeth
Safar\marginnote{Bride, a 25 años de su publicación'',
  \emph{Anuario ININCO} 1, n.° 17 (2005); Guillermo Mastrini y Diego de
  Charras, ``20 años no es nada: del NOMIC a la CMSI'', \emph{Anuario
  ININCO} 2, n.° 17 (2005); Andrés Cañizález, ``Veinticinco años del
  Informe Mac Bride. Releyendo el gran inventario de la comunicación'',
  \emph{Temas de Comunicación}, n.° 13, (2006): 15-26; Fernando Quirós,
  ``El debate sobre la información, la comunicación y el desarrollo en
  la UNESCO durante el siglo XX'', \emph{Commons} 2, n.° 2 (2013): 7-38;
  Quirós y Sierra Caballero, \emph{El espíritu}...}\marginnote{\textsuperscript{21} UNESCO,
  \emph{Informe}..., 17.}\marginnote{\textsuperscript{22} Oswaldo
  Capriles, \emph{Elementos para un análisis crítico del Nuevo Orden
  Internacional de la Información o de la Comunicación} (trabajo de
  ascenso, Caracas: UCV, 1979).}\marginnote{}{\textsuperscript{23}\setcounter{footnote}{23} ILET (Instituto Latinoamericano de Estudios
  Transnacionales), ``La información en el nuevo orden internacional.
  Recomendaciones para la acción'', \emph{Comunicación}, n.° 18 (1978):
  90.} recuerda que en la XI Conferencia de la AIERI celebrada en
septiembre de 1978 en Varsovia, se produjeron dos hechos de relevancia.
En primer lugar, con la \emph{mediación} de Pasquali desde su posición
al interior de la UNESCO, la Asamblea General aprobó que el ININCO y la
Universidad Central de Venezuela organizaran la XII Conferencia de la
AIERI y que la sede del evento fuera Caracas. En segundo lugar, un
conjunto de investigadores latinoamericanos, entre quienes se
encontraban Pasquali, Elizabeth Safar, Marco Ordoñez, Aníbal Gómez y
Patricia Anzola, se reunieron a discutir sobre la necesidad de fundar
una asociación de investigadores a nivel regional.\footnote{Elizabeth
  Safar, entrevistada por el autor, 10 de marzo de 2016, Caracas,
  Venezuela.}

La formación de la ALAIC fue así traccionada por fuerzas diferentes. Un
proceso regional de circulación de investigadores con preocupaciones
vinculadas a la relación entre medios masivos, cultura y procesos
políticos, en un marco en el que la UNESCO venía promoviendo y
articulando instituciones e investigaciones en la Comisión MacBride.
Pasquali operó como conector entre ambos procesos. Aníbal Gómez, primer
director de la ALAIC, consideraba que había sido un hecho ``clave'' la
llegada de Pasquali a la UNESCO porque había potenciado la posición
latinoamericana en las ``redes internacionales de investigadores'' de la
comunicación.\footnote{Andrés Cañizález, ``Yo no creía que ALAIC iba a
  durar por mucho tiempo (entrevista a Aníbal Gómez)'', en \emph{Del
  mimeógrafo a las redes digitales. Narrativas, testimonios y análisis
  del campo comunicacional en el 40 aniversario de ALAIC}, org. por
  Delia Crovi Druetta y Gustavo Cimadevilla (Ciudad de México: Ediciones
  La Biblioteca, 2018), 101-108. Para profundizar sobre el proceso de
  fundación de la ALAIC en el marco de la institucionalización del campo
  de la comunicación en América Latina, remitimos a José Marques de Melo
  y Maria Cristina Gobbi, \emph{Pensamento Comunicacional
  Latino-Americano. Da pesquisa denúncia ao pragmatismo utópico} (São
  Paulo: UMESP, 2004); José Marques de Melo, ``Los tiempos heroicos: la
  formación de la comunidad latinoamericana de Ciencias de la
  Comunicación'', \emph{ALAIC}, n.° 1 (2004); José Marques de Melo,
  \emph{História das ciências da comunicação} (Río de Janeiro: Mauad,
  2008); Centro Interdisciplinario Boliviano de Estudios de la
  Comunicación, \emph{Asociación Latinoamericana de Investigadores de la
  Comunicación, 1978-1998. Contribuciones para una memoria
  institucional} (La Paz:} En efecto, a la par de los
encuentros realizados en América Latina, se habían ido conformando
asociaciones en distintos países. La primera de ellas, creada en 1972,
había sido la Associação Brasileira de Pesquisa e Ensino da Comunicação
(ABEPEC). Para 1977 se había fundado la Asociación Venezolana de
Investigadores de la Comunicación (AVIC). Ese mismo año surgieron la
Asociación Colombiana de Investigadores de la Comunicación (ACICOM) y la
Sociedade Brasileira de Estudos Interdisciplinares da Comunicação. En
1979 se fundó la Asociación Mexicana de Investigadores de la
Comunicación.

La formación de ALAIC da cuenta de un momento de cristalización de ese
proceso heterogéneo y relativamente disperso que había caracterizado la
emergencia de espacios de investigación en comunicación en la región
hacia finales de la década de los setenta. La asamblea convocada por la
AVIC y el ININCO que se realizó en noviembre de 1978 en Caracas, logró
articular a referentes de la región como Mario Kaplún, Luiz Gonzaga
Motta, Marco Ordoñez, Fernando Reyes Matta y Oswaldo Capriles, entre
otros.\textsuperscript{26} Tal como quedó asentado en el Acta
Constitutiva de la ALAIC, se postulaba como objetivo establecer un
``organismo internacional científico-gremial sin fines de lucro,
destinado a agrupar a Asociaciones,\newpage 

\noindent Instituciones e Individuos dedicados
a la investigación\marginnote{CIBEC, 1998); Luis Ramiro Beltrán, ``ALAIC: el
  albergue de la inquietud (entrevista)'', \emph{ALAIC}, n.° 1 (2004);
  Raúl Fuentes Navarro, ``La institucionalización académica de las
  ciencias de la comunicación: campos, disciplinas, profesiones'', en
  \emph{Campo académico de la comunicación: hacia una reconstrucción
  reflexiva}, coord. por Jesús Galindo y Carlos Luna (Ciudad de México:
  ITESO, 1995); Raúl Fuentes Navarro, coord., \emph{Instituciones y
  redes académicas para el estudio de la comunicación en América Latina}
  (Guadalajara: ITESO, 2006); Maria Cristina Gobbi, \emph{A batalha pela
  hegemonia comunicacional na América Latina. 30 anos da ALAIC} (São
  Paulo: Universidade Metodista de São Paulo, 2008); César Bolaño, Delia
  Crovi Druetta y Gustavo Cimadevilla, eds., \emph{La contribución de
  América Latina al campo de la comunicación} (Buenos Aires: Prometeo,
  2015); Delia Crovi Druetta y Gustavo Cimadevilla, orgs., \emph{Del
  mimeógrafo a las redes digitales. Narrativas, testimonios y análisis
  del campo comunicacional en el 40 aniversario de ALAIC} (Ciudad de
  México: Ediciones La Biblioteca, 2018).}\marginnote{\textsuperscript{26}\setcounter{footnote}{26} ``Acta constitutiva Asociación Latinoamericana de
  Investigadores de la Comunicación'', archivo personal de Elizabeth
  Safar, Caracas, Venezuela, 17 de noviembre de 1978. Además de los que
  ya indicamos, según el documento participaron miembros del ILET, del
  Centro Internacional de Estudios Superiores de Comunicación para
  América Latina (CIESPAL), de la ACICOM, de la ABEPEC y, de México, del
  Consejo Nacional para la Enseñanza y la Investigación en Ciencias de
  la Comunicación (CONEICC).} científica en materia de comunicación''.


Si las redes nacionales de investigación en comunicación lograron
instituir una asociación regional se debió también al impulso y a la
legitimación que otorgaba la UNESCO en tanto organismo internacional.
Los argumentos expresados desde París validaban la formación de espacios
en distintas latitudes sosteniendo la necesidad de cooperación y
desarrollo más allá de las fronteras nacionales. Esto se puede observar
en el hecho de que unos días después de la asamblea fundacional de la
ALAIC, la UNESCO organizó en Panamá una reunión en la que participaron
una veintena de investigadores latinoamericanos de la comunicación y
expertos del organismo con el objetivo de consolidar los estudios en la
región. Se afirmaba que en el encuentro se había considerado a la ALAIC
como ``un instrumento imprescindible y un aliado de la UNESCO'' para
``asegurar en América Latina la promoción y la coordinación de los
programas de investigación en comunicaciones''.\footnote{José Martínez
  Terrero, ``Investigación para la toma de decisiones en políticas de
  comunicación'', \emph{Comunicación}, n.° 22 (1979): 114.} La nueva
asociación debía ser un nodo estratégico de construcción de enlaces
entre investigadores latinoamericanos. En los Estatutos de la ALAIC se
puede leer un programa de acción para una franja de la intelectualidad
de la comunicación nucleada en institutos y centros que estaban
dispersos por Latinoamérica. En un contexto de persecución y represión
que alcanzaba también a la producción científica y cultural en los
países donde gobernaban dictaduras militares, la conformación de la
asociación también puede ser pensada como una apuesta de reorganización
de la actividad intelectual especializada en comunicación que buscaba
---como se afirma en sus estatutos--- garantizar la ``libertad
científica para los investigadores de la comunicación en América
Latina'' y abrir espacios de edición, circulación y difusión de la
``documentación científica sobre la especialidad'', preferentemente la
que se producía en la región.\footnote{``Asociación Latinoamericana de
  Investigadores de la Comunicación (A.L.A.I.C). Estatutos'', archivo
  personal de Elizabeth Safar, Caracas, Venezuela, 17 de noviembre de
  1978.} Era, también, una posibilidad de reordenar un mercado
especializado que en el cono sur había sido coartado por los gobiernos
militares y, para un sector de la intelectualidad, se convertía en una
oportunidad para incorporarse como productores de saberes.

La ALAIC pretendió establecer acuerdos con organizaciones de la sociedad
civil y formar ``cuadros de investigación de la comunicación'' para que
se reunieran en torno a los proyectos sociales y culturales
``conducentes a cambiar'' las sociedades latinoamericanas. En el
estatuto, los fundadores de la ALAIC se declaraban interpelados por los
``grandes problemas estratégicos como el Nuevo Orden Informativo
Internacional'' y el desarrollo de ``políticas de comunicación que vayan
en beneficio de los sectores mayoritarios de la sociedad
latinoamericana''. Se postulaba que había que acompañar
``especialmente'' los debates y las discusiones que en esta materia se
llevaban adelante en la UNESCO. Una de las cuestiones que se planteaban
era la necesidad de repensar los programas de las instituciones
académicas especializadas en periodismo y comunicación en América
Latina. El dilema central era cómo reorientar la producción de saberes y
teorías que ---se caracterizaba--- no daban cuenta de la especificidad
de la región. Con ese objetivo se reunieron en diciembre de 1979, en
Caracas, representantes de distintas universidades, escuelas y centros
de comunicación latinoamericanos.\footnote{Participaron miembros de la
  Universidad de Guayaquil (Ecuador), la Universidad de Lima (Perú), la
  Universidad del Salvador (Argentina), la Universidad Central de
  Venezuela y la Universidad Católica Andrés Bello (Venezuela), de la
  Asociación Colombiana de Facultades de Comunicación y la Universidad
  de Oriente (Colombia), de la ABEPEC (Brasil), de la Universidad del
  Norte (Chile), de la Universidad Nacional Autónoma de México y el
  CONEICC (México), de la Universidad Central del Este (República
  Dominicana), del CIESPAL y de las fundaciones Konrad Adenauer y
  Friedrich Ebert.} Se privilegiaba como horizonte de estudio pensar a
los medios de comunicación como ``servicio público'', la defensa del
derecho a la comunicación, el desarrollo de los mecanismos de acceso y
participación y la ``democratización de las estructuras
comunicacionales''. En esta línea, a principios de 1980, a través de su
Escuela de Comunicación Social y del ININCO, la Facultad de Humanidades
y Educación de la Universidad Central de Venezuela inauguró la Maestría
en Comunicación Social ``Políticas y Planificación de la Comunicación
Social en América Latina''.\footnote{Entre las asignaturas de la
  maestría, que tenía una duración total de dos años, se encontraban
  ``Metodología de la Investigación para Comunicaciones Sociales'',
  ``Comunicación, Desarrollo y Dependencia'', ``Diseño de Políticas y
  Planificación de las Comunicaciones Sociales'', ``Legislación de las
  Comunicaciones'', ``Tecnología, Tenencia y Uso de los Medios en
  América Latina'' y ``Diseño de Modelos para Políticas Nacionales de
  Comunicación''. Una versión del programa de la maestría fue publicada
  en ``Maestría en Comunicación Social'', \emph{ININCO}, n.° 1 (1980):
  80.}

Desde nuestra perspectiva y objetivos, queremos destacar la posición
central que adquiría entonces Venezuela como polo de producción de
saberes en comunicación en la región, potenciada por el rol de Antonio
Pasquali en una serie de instituciones y redes transnacionales como la
comisión de la UNESCO. Esta posición le permitió a la Universidad
Central de Venezuela reunir en la maestría en políticas y planificación
de la comunicación a la elite intelectual del campo de la comunicación a
nivel internacional. Además de los referentes locales, entre ellos
Pasquali, dictaron cursos y conferencias Tapio Varis y Kaarle
Nordenstreng (Finlandia), Herbert Schiller, Everett Rogers y Albert
Hester (Estados Unidos), Luis Ramiro Beltrán (Bolivia), Juan Somavía
(Chile), Marco Ordoñez (Ecuador), Armand Mattelart y Michèle Mattelart
(Francia).

\hypertarget{la-aieri-en-caracas-entre-el-informe-macbride-y-las-democracias-latinoamericanas}{%
\section{La AIERI en Caracas. Entre el informe MacBride y las\\\noindent
democracias
latinoamericanas}\label{la-aieri-en-caracas-entre-el-informe-macbride-y-las-democracias-latinoamericanas}}

Entre el 25 y el 29 de agosto de 1980, se desarrolló la XII Asamblea
General y Conferencia de la AIERI/IAMCR en la Universidad Central de
Venezuela. Se estimaba que habían participado más de 300 investigadores
de unos 60 países.\footnote{José Ignacio Rey, ``Encuentro de
  Investigadores y Nuevo Orden Informativo Internacional'',
  \emph{Comunicación}, n.° 30-31 (1981): 32.} Según el testimonio
retrospectivo de Elizabeth Safar, Pasquali planteó en la conferencia de
apertura los lineamientos generales que se iban a discutir a lo largo de
las jornadas: afirmó que el ``tema central'' que convocaba por entonces
a los investigadores de la comunicación era el ``Nuevo Orden Informativo
Internacional''.\footnote{Elizabeth Safar, entrevistada por el autor, 10
  de marzo de 2016, Caracas, Venezuela.} Las expectativas que Pasquali y
todos los miembros del ININCO habían depositado en el evento se
vinculaban a que el Instituto venía discutiendo con ``la directiva de la
AIERI'' la necesidad de desplazar el eje del debate desde la cuestión
``este-oeste'' hacia las problemáticas ``norte-sur''. Este giro para
pensar ``las estructuras internacionales de la comunicación'' era una de
las ``razones básicas'' por las cuales el Instituto había asumido la
responsabilidad de organizar el evento.\footnote{``Conferencia Mundial
  de AIERI en Caracas'', \emph{ININCO}, n.° 1 (1980): 6.} El
desplazamiento se vinculaba con la presencia ineludible de referentes de
la intelectualidad del ``tercer mundo'' en el debate sobre el flujo
internacional de la comunicación y, según la posición del ININCO, ello
permitía disputar los ejes y la orientación de las discusiones.

La Asamblea de Caracas se presentaba como una oportunidad para dar
visibilidad al ``tan esperado Informe MacBride'', ``pieza clave para la
definición de las políticas internacionales de comunicación''. El
temario que organizaba el evento había quedado definido en torno al
análisis de los desarrollos tecnológicos y sus implicaciones políticas,
la reflexión acerca de las nuevas demandas para la formación profesional
del periodista en un entorno de ``transformaciones comunicacionales'' y,
por último, problematizar la relación entre medios, sociedad y Estado en
el mapa más amplio de las relaciones transnacionales.\footnote{``Conferencia...'',
  7.} Entre los prestigiosos referentes que participaron en la asamblea,
estuvieron Elisabeth Noelle-Neumann (Alemania), Ithiel de Sola Pool,
Herbert Schiller y Everett Rogers (Estados Unidos), Cees Hamelink
(Holanda), Tomas Szecsko (Hungría), Peter Schenkel (Austria), Dallas
Smythe (Canadá), Francesco Fattorello (Italia) y Kjeld Veirup
(Dinamarca). Entre los ``latinoamericanos'' asistieron Héctor Schmucler
(Argentina), Luis Ramiro Beltrán (Bolivia), Jesús Martín-Barbero
(España) y Oswaldo Capriles (Venezuela). La polémica giró respecto a dos
tópicos: las conclusiones de \emph{Un solo mundo, voces
múltiples}\footnote{Con ese título y editado de forma conjunta entre el
  Fondo de Cultura Económica de México y la UNESCO, en 1980 fue
  publicado el informe final de la Comisión Internacional sobre
  Problemas de Comunicación presidida por Sean MacBride.} y acerca del
debate sobre la relación entre investigación, tecnología y comunicación.

El ``Informe MacBride'' dejaba planteado que la comunicación era ``un
derecho fundamental'' que debía ser garantizado en todas las naciones.
Este énfasis se revelaba estratégico para postular la necesidad de
``democratizar las comunicaciones'':\footnote{UNESCO, \emph{Un solo
  mundo, voces múltiples} (Ciudad de México: Fondo de Cultura Económica,
  1980)\emph{,} 451.} si la producción y circulación de cultura y
comunicación estaba concentrada, el derecho relativo a buscar, recibir y
difundir información, no podría cumplirse. Se volvía indispensable que
los países ``formularan unas políticas globales de comunicación ligadas
a la totalidad de los objetivos de desarrollo social, cultural y
económico''. Para el proceso de ``democratización'', la ``explosión de
la tecnología de la comunicación'' ofrecía grandes posibilidades si se
tomaba como base el principio de la participación de la sociedad en los
nuevos procesos sociales.

Ante los planteos del Informe emergieron posicionamientos antagónicos:
para los norteamericanos y europeos, era un trabajo ``excesivamente
politizado y tercermundista''; para los latinoamericanos, ``sólo una
buena plataforma para seguir avanzando'';\footnote{Rey,
  ``Encuentro...'', 35.} era acusado de ser ``demasiado complaciente
frente a los sistemas tradicionales'' sobre las cuestiones de la
libertad de prensa y de la ``tecnología importada'' por los
latinoamericanos. Por su parte, los delegados de la Unión Soviética,
Líbano, Nigeria y la India, reflexionaba Peter Schenkel en la revista
\emph{Chasqui},\footnote{Peter Schenkel, ``El Informe Mac Bride: entre
  la realidad y la utopía'', \emph{Chasqui}, n.° 1 (II Época, 1981): 86.
  En ese entonces, Schenkel era miembro de la Fundación Friedrich Ebert
  y del CIESPAL.} habían criticado que el Informe socavaba el NOII al
haber subestimado ``la función de la cultura de masas occidental'', y
que algunos de sus planteamientos parecían ``inaplicables en muchas
partes del mundo'' si no cambiaban las relaciones de poder. Los
investigadores norteamericanos, además, sostenían que era una propuesta
que ``iba demasiado lejos'' y trataba de proponer una ``tiranía estatal
sobre los medios''.\footnote{Las polémicas suscitadas en Caracas se
  trasladarían unos meses después a Belgrado. En la XXI Conferencia
  General de la UNESCO, si bien fueron adoptadas las recomendaciones de
  \emph{Un solo mundo, voces múltiples}, se produjeron ``arduos
  debates'' y tensiones. Gifreu, \emph{El debate}..., 147-149.} Sobre el
papel de la investigación en el marco de la ``nueva estructura de la
comunicación internacional'', los referentes europeos y norteamericanos
asumieron una posición que veía, en la efectiva incorporación de los
desarrollos tecnológicos a los procesos sociales, posibilidades de
garantizar mayor libertad en el plano comunicacional y cultural.
Hamelink consideraba que era necesario revisar los ``viejos paradigmas''
para que la investigación pudiera dirigir ``exitosamente'' a la
tecnología hacia los fines que demandaba la sociedad.\footnote{Cees
  Hamelink, ``Nuevas estructuras de la comunicación internacional: el
  papel de la investigación (selección)'', \emph{Comunicación}, n.°
  30-31 (1981): 37.} Por su parte, Sola Pool planteaba que la ``actual
revolución de las comunicaciones'' permitiría un ``flujo libre sin
restricción'', ``mayor diversidad'' y que tal como se estaban
reestructurando los medios masivos, ``las comunicaciones serían menos
controladas, más libres y dejarían de tener una sola
dirección''.\footnote{Ithiel Sola Pool, ``La nueva estructura de la
  comunicación internacional: el papel de la investigación
  (selección)'', \emph{Comunicación}, n.° 30-31 (1981): 38.}

A diferencia de los planteos de algunos referentes europeos y
norteamericanos, la relación entre investigación, tecnología y
comunicación era, para un sector de la intelectualidad
``tercermundista'', un problema eminentemente político. Las ponencias de
Capriles, Martín-Barbero y Schmucler, con énfasis diversos, cuestionaban
que la comunicación y las tecnologías fueran pensadas con una excesiva
autonomía de las condiciones sociales, políticas y culturales. Capriles
criticaba que tanto las Políticas Nacionales de Comunicación como el
NOII revelaban una ``insuficiencia'' para inscribir a la comunicación en
``análisis globales y macroestructurales''.\footnote{Oswaldo Capriles,
  ``De las Políticas Nacionales de la Comunicación al Nuevo Orden
  Internacional de la Información: algunas lecciones para la
  investigación'' (ponencia, Caracas: UCV, 1980): 69.} Esta carencia
tenía que ver con ``la no comprensión de los fenómenos de la
comunicación e información'' en el ``plano de la reproducción de los
procesos productivos y las relaciones de dominación''. Asimismo,
Martín-Barbero consideraba que ante una reflexión dominante que seguía
pensando a la técnica desde una ``concepción instrumentalista'', era
necesario combatirla con un pensamiento que la reinscribiera en el
``peso histórico'' y en su ``entramado político''.\footnote{Jesús
  Martín-Barbero, ``Retos a la investigación de la comunicación en
  América Latina'', \emph{Comunicación y Cultura}, n.° 9 (1983): 100.}
Schmucler, en tanto, planteaba un aspecto que se había vuelto
problemático a lo largo de la Asamblea: la relación entre comunicación y
democracia. El investigador afirmaba que una comunicación
``verdaderamente democrática'' solo podía ser alcanzada en el marco de
la democratización del conjunto social.\footnote{Héctor Schmucler, ``La
  investigación sobre comunicación en América Latina en la hora de las
  computadoras'' (ponencia, Caracas: UCV, 1980), 13.} La postura de
Schmucler era representativa de la posición de distintos investigadores
de la región. En la crónica del evento publicada por José Ignacio Rey,
se afirmaba que ``los latinoamericanos'' habían propuesto la relación
entre comunicación y democracia como tema central para la siguiente
asamblea de la AEIRI/IAMCR a realizarse en 1982 en París. Para la mayor
parte de los investigadores de Norteamérica y Europa, ``la democracia
era un tema político y conflictivo'' y la investigación científica debía
quedar situada por ``encima'' de esas ``polémicas''. Tras intensos
debates, sigue Rey, el tema fue aprobado por una ``amplia
mayoría''.\footnote{Rey, ``Encuentro...'', 34.}

Las discusiones que emergieron a lo largo de la conferencia de la AIERI
en Caracas, permiten afirmar dos cuestiones respecto al lugar de la
intelectualidad ``tercermundista'' en el debate internacional: por un
lado, que la Universidad Central de Venezuela y el ININCO se habían
convertido en un polo destacado de reunión de especialistas a nivel
internacional. Producto de las elaboraciones teóricas y de la
experiencia de los académicos locales en los procesos vinculados a la
formulación de políticas estatales de comunicación, Venezuela se había
vuelto un escenario de relevancia para el debate intelectual a escala
mundial. Por otro, y en consonancia con lo anterior, la presión que
habían ejercido los intelectuales latinoamericanos en la organización de
los temas centrales para la siguiente asamblea de la AIERI/IAMCR,
permite leer una posición que ---no sin conflicto--- lograba instituir
la agenda de problemas que merecían ser estudiados y reflexionados, al
interior de una asociación que se presentaba como el máximo organismo
bajo el que se nucleaban los investigadores de la comunicación a nivel
internacional.

\hypertarget{la-relacin-estado-y-comunicacin-entre-la-traicin-y-el-fracaso}{%
\section{La relación Estado y comunicación. Entre la traición y el
fracaso}\label{la-relacin-estado-y-comunicacin-entre-la-traicin-y-el-fracaso}}

Entre finales de los años setenta y principio de los ochenta se había
empezado a incorporar el término ``democracia'' a las discusiones sobre
las Políticas Nacionales de Comunicación. Lo que unos años antes
aparecía como una preocupación relativamente aislada y emergente en un
sector de la investigación, ahora cobraba una fuerza que
transversalizaba las discusiones académicas de la comunicación. Esto
representaba un giro al interior de la intelectualidad latinoamericana.
Si hacia mediados de los setenta un sector de los especialistas e
investigadores de la comunicación había visto en los procesos políticos
la posibilidad de encarnar la democratización de las comunicaciones,
ahora de lo que se trataba era de democratizar la sociedad. En un
sentido más amplio, la pregunta tuvo un desplazamiento clave: desde qué
políticas para democratizar la comunicación hacia la \emph{redefinición
del papel del Estado} en América Latina.

Este proceso se comprende en el contexto de las diferentes experiencias
políticas que atravesaron a la región. Para 1976, cuando se celebró la
Conferencia Intergubernamental sobre Políticas de Comunicación en
América Latina y el Caribe, con excepción de Venezuela, Colombia, México
y Costa Rica, casi toda Latinoamérica estaba gobernada por dictaduras
militares. Se produjo una heterogeneidad de posiciones en la
intelectualidad. Una franja se radicalizó ideológicamente cuando
percibió que las políticas llevadas adelante por los gobiernos de sus
países, en especial los venezolanos, abrían una posibilidad de
democratizar el modo de producción cultural y comunicacional. Como
sostenía Oswaldo Capriles, había una oportunidad de operar el pasaje
desde ``las exigencias teóricas de los expertos a la práctica
política''.\footnote{Capriles, \emph{Elementos}..., 250.} Mientras que
otro sector, fundamentalmente de Chile y Argentina, se radicalizó
políticamente con una activa participación en la lucha revolucionaria.
Intelectuales como Schmucler, Nicolás Casullo y Rubén Sergio Caletti,
entre otros, tras la experiencia exiliar se propusieron rediscutir en
términos ``histórico-culturales y teóricos el fin de una época: el
vanguardismo, los autoritarismos antidemocráticos, la crítica a las
armas''.\footnote{Nicolás Casullo, citado en María Verónica Gago,
  \emph{Controversia, una lengua del exilio} (Buenos Aires: Biblioteca
  Nacional, 2012), 83.}

Los intercambios sobre estas cuestiones se produjeron y circularon en
publicaciones especializadas y en distintos encuentros de investigadores
en la región. La relación entre comunicación y democracia fue el tópico
elegido para la reunión constitutiva del grupo de comunicación del
Consejo Latinoamericano de Ciencias Sociales (CLACSO), que se realizó en
marzo de 1981 en Colombia.\footnote{Allí participaron, entre otros,
  Patricia Terrero, Giselle Munizaga, Luiz Gonzaga Motta, Oswaldo
  Capriles, Fátima Fernández, Elizabeth Fox, Reyes Matta, Ana María
  Nethol, Alcira Argumedo y Héctor Schmucler.} Allí se discutieron las
distintas experiencias político-culturales latinoamericanas y se debatió
sobre los ejes que debía tomar la investigación en comunicación. Se
consideraba necesario repensar las relaciones de poder y superar los
``análisis mecanicistas del funcionamiento de la sociedad'', en especial
aquellos estudios que entendían al receptor como ``un mero objeto
pasivo''. Había que indagar la relación entre prácticas de comunicación
y movimientos populares y pensar su ``papel sustantivo'' en la conquista
de ``una comunicación democrática en el continente''.\footnote{Héctor
  Schmucler y Elizabeth Fox, \emph{Comunicación y democracia en América
  Latina} (Lima: DESCO, 1982), 14.} Era un desplazamiento que,
complementariamente, permitía ``resemantizar'' conceptos que antes
habían sido entendidos como ``trampas'' de las empresas transnacionales
de la comunicación: ``libertad de prensa'', ``libertad de expresión'' y
``objetividad'' eran ahora demandas legítimas de una sociedad que quería
estar informada y procuraba decir su propia palabra.

Lo que empezaba a cuestionarse eran los \emph{repertorios de acción} de
la intelectualidad latinoamericana para intervenir en el debate público.
Repertorios que condensaban experiencias, \emph{habitus} intelectuales,
filiaciones políticas y escenarios nacionales que expresaban una trama
de diferencias. En otras palabras, se estaban rediscutiendo las
modalidades de intervención intelectual a la luz de las experiencias
políticas de las dictaduras. Este era un dilema que progresivamente
había ocupado la agenda académica y se vinculaba a la experiencia en la
formulación de regulaciones para el sistema de medios. En
\emph{Comprender la comunicación}, Pasquali se había interrogado si la
intelectualidad debía seguir por el camino de la formulación de
Políticas Nacionales de Comunicación, considerando las dificultades para
la aplicación efectiva de las regulaciones.

El debate respecto a las políticas de comunicación, no solo como campo
de investigación sino como escenario de intervención intelectual, se
materializó en distintos puntos de la región. En un artículo publicado
en \emph{Chasqui}, el investigador brasilero Luiz Gonzaga Motta sostenía
al respecto que las experiencias de las políticas de comunicación no
habían cambiado nada e incluso habían terminado por favorecer a los
gobiernos autoritarios. Según Gonzaga Motta, las dictaduras habían
incorporado a sus programas de gobierno consideraciones que las
``fuerzas progresistas'' del continente propusieron para democratizar la
comunicación. El ``fracaso'' del encuentro en Costa Rica de 1976 se
expresaba, continuaba el investigador, en que todos los países habían
aumentado el control gubernamental en la comunicación con políticas que
servían a la ``dominación, la legitimización política y a la
consolidación de los regímenes militares''.\footnote{Luiz Gonzaga Motta,
  ``Costa Rica: seis años después'', \emph{Chasqui}, n.° 3 (II Época,
  1982): 17.}

De todos modos, referentes como Pasquali y Beltrán se posicionaban de
forma diferenciada respecto a las políticas de comunicación. Pasquali se
incorporó a los debates promovidos por \emph{Chasqui} que, entre 1982 y
1983, dedicó varios números a cuestiones sobre políticas de
comunicación, comunicación popular y alternativa y sobre comunicación y
democracia. En 1983, Pasquali afirmaba que si bien en América Latina
``prevalecían las dictaduras militares'', el problema de la censura y de
los controles estaba vinculado tanto al accionar estatal como a la
``manipulación practicada por la iniciativa privada'' que ``había
alcanzado una dimensión multinacional'', por lo que no debían
descartarse la producción de regulaciones para el mercado
comunicacional.\footnote{Antonio Pasquali, ``¿Contradicción entre
  libertad y equilibrio informativo?'', \emph{Chasqui}, n.° 6 (II Época,
  1983): 28.} Por su parte, Beltrán no creía que las recomendaciones de
Costa Rica se hubieran convertido en ``instrumentos de control
represivo''. Entendía que los gobiernos dictatoriales no buscaban la
democratización sino establecer regulaciones ``caprichosas y dispersas''
para dar la impresión de que respetaban la libertad de información. En
cuanto al lugar del Estado en la formulación de políticas de
comunicación, planteaba que si bien ya no era un ``actor principal'', sí
era un ``árbitro inevitable e indispensable como respaldo a la
aplicación de políticas''. La sugerencia de Beltrán, aun cuando
pareciera ``impracticable'', era la formación de consejos nacionales
``pluralistas'', distanciándose así de la estatización.\footnote{Luis
  Ramiro Beltrán, ``No renunciemos jamás a la utopía (entrevista
  exclusiva de Patricia Anzola'', \emph{Chasqui}, n.° 3 (II Época,
  1982): 10.}

¿Se trataba de abandonar la vía de las Políticas Nacionales de
Comunicación? ¿Qué otros caminos debían tomar los intelectuales de la
comunicación? Gonzaga Motta exhortaba a los ``profesionales teóricos y
prácticos'' de la comunicación a reorientar sus luchas partiendo de
experiencias concretas y buscar modalidades alternativas de
democratización. Para el investigador brasilero se había producido un
claro problema: los intelectuales se habían abstraído de la ``cuestión
política'' y al caer en la ``trampa del tecnicismo'', la transformación
de la comunicación se había tornado ``un fin en sí mismo''.\footnote{Gonzaga
  Motta, ``Costa...'', 17.} Un sector de los agentes del campo convocaba
abiertamente a abandonar la discusión y a pensar diferentes
alternativas. Uno de los proyectos revisteriles donde se condensó esta
posición con más claridad fue en la revista \emph{Comunicación y
Cultura}.\footnote{Para un itinerario de conjunto de la revista
  \emph{Comunicación y Cultura}, véanse Víctor Lenarduzzi, \emph{Revista
  Comunicación y Cultura. Itinerarios, ideas y pasiones} (Buenos Aires:
  Eudeba, 1998) y Zarowsky, \emph{Del laboratorio}...} Entre principios
y mediados de los años ochenta le dedicó espacio a discutir en torno a
``los límites del debate internacional de la comunicación''. Las
críticas elaboradas hacían énfasis en la problemática relación entre
transformación comunicacional y cambio social; la crisis del tercer
mundo como zona estratégica desde la cual construir proyectos políticos
y los tópicos ``silenciados'' en el debate del Nuevo Orden Mundial de la
Información y Comunicación. Schmucler afirmaba que la crisis no era
simplemente política sino la de un ``modelo de entender'' la sociedad
que hacía aparecer al tercer mundo como una zona ``homogénea''. Esto
habilitaba a realizar operaciones ideológicas de ``izquierda'' y
``progresistas'' frente a un otro que era presentado como lo
``reaccionario'' a ser doblegado. Según Schmucler, la paradoja de que
Estados Unidos se retirara de la UNESCO,\footnote{Sobre el retiro de
  Estados Unidos de la UNESCO volveremos más adelante.} ofrecía el
``insólito hecho'' de que, sin sus aportes económicos, ese camino de
``transformación'' ya no podía realizarse.\footnote{Héctor Schmucler,
  ``Año mundial de la comunicación. Con penas y sin gloria'',
  \emph{Comunicación y Cultura}, n.° 11 (1984).} Nicolás Casullo
sostenía que en los encuentros ``de los defensores del NOII'' habían
surgido importantes contradicciones que por lo general habían sido
``poco analizadas''. En particular se volvían evidentes las
``\emph{diferencias intratercer-mundo}''.\footnote{Nicolás Casullo,
  ``1980: la UNESCO discute el informe MacBride'', \emph{Comunicación y
  Cultura}, n.° 11 (1984): 133.} No se había formulado con ``claridad la
índole de los proyectos nacionales'' que pudieran devenir en órdenes
informativos de ``real democracia y participación de los pueblos''. La
``secundarización'' de las especificidades nacionales había traído
consigo que el debate silenciara las heterogéneas realidades políticas
latinoamericanas, que iban desde gobiernos más o menos progresistas y
democráticos a feroces dictaduras militares. Por su parte, Rubén Sergio
Caletti consideraba que en los debates del NOII no se habían distinguido
``formas concretas de Estado, de relación Estado-sociedad y del papel
mismo de la información en ese vínculo''.\footnote{Rubén Sergio Caletti,
  ``El nuevo orden informativo: un fantasma del viejo pasado'',
  \emph{Comunicación y Cultura}, n.° 13 (1985): 123.} Esta obliteración
había tenido como ``efecto negativo'' que se impidiera la discusión
sobre el ``significado real del nuevo orden''. Dadas estas condiciones
generales de agotamiento del contexto que lo había visto nacer y de las
tensiones internas, finalizaba, ``el NOII ha muerto''.\footnote{Caletti,
  ``El nuevo...'', 124.}

En el marco de estos debates y a la luz de su propia trayectoria al
interior de la UNESCO, se puede leer en el posicionamiento de Pasquali
una apuesta al desarrollo de estrategias de cooperación internacional.
El investigador venezolano consideraba que las nuevas discusiones sobre
el derecho a la comunicación tendían a priorizar el ``derecho de los
públicos'' por sobre los propietarios de los medios masivos. Si bien los
estados nacionales ya no eran instituciones que garantizaban el derecho
a la comunicación, Pasquali entendía que, antes que ``abandonar'' la
apuesta intelectual de formular regulaciones para el sector de la
cultura y la comunicación, se debían profundizar y fortalecer los lazos
institucionales a nivel transnacional. Esto habilitaría un proceso de
producción, circulación y consumo de bienes simbólicos que fomentaría el
desarrollo social y cultural de los países del tercer mundo.\footnote{Pasquali,
  ``¿Contradicción...?''.}

\hypertarget{el-giro-estratgico-hacia-las-polticas-culturales}{%
\section{El giro estratégico hacia las políticas
culturales}\label{el-giro-estratgico-hacia-las-polticas-culturales}}

De forma paralela a estos debates, se fueron organizando planes de
cooperación promovidos por la UNESCO como el Programa Internacional para
el Desarrollo de la Comunicación (PIDC). En la XXI Conferencia de
Belgrado se dictaminó que el PIDC tenía por objetivo ``estimular,
especialmente entre los países en desarrollo, acuerdos relativos al
intercambio de informaciones, programas y experiencias'', como así
también fomentar ``la cooperación y la coproducción entre organismos de
radiodifusión y de televisión''.\footnote{UNESCO, ``21C/Resoluciones,
  3/07'', París: UNESCO, 1980.} Teniendo este programa como guía, en
1980 se conformó la Asociación de Radio y Televisión Estatales de
América Latina (ARTEAL), a la que se adscribieron Costa Rica, Cuba,
Honduras, Panamá, Perú y Venezuela. En 1984 la asociación cobró mayor
potencia y visibilidad luego de un acuerdo entre el gobierno de Costa
Rica con la UNESCO. En enero de ese año, el ministro costarricense
Armando Vargas Araya\footnote{Vargas Araya fue ministro de Comunicación
  e Información, entre 1982 y 1986, durante el gobierno de Luis Alberto
  Monge Álvarez, representante del Partido Liberación Nacional.} junto a
Pasquali ---como \emph{representante} de la UNESCO--- suscribieron el
memorándum ``Reforzamiento de Cooperación e Intercambio entre
Televisoras y Radioemisoras de Servicio Público de América Latina y el
Caribe''.\footnote{Antonio Pasquali y Armando Vargas Araya, \emph{De la
  marginalidad al rescate: los servicios públicos de radiodifusión en la
  América Latina} (San José de Costa Rica: EUNED, 1990), 181.} En la
visión retrospectiva de Pasquali, un plan de características
trasnacionales era una opción para fortalecer ``desde arriba'' los
sistemas públicos de radio y televisión, e incluso para promover una
mayor independencia del poder gubernamental. Si ``desde abajo'', con el
debate social y político en los congresos nacionales de América Latina
no se habían logrado las transformaciones, una opción era interpelar a
los estados desde un organismo regional que, al suscribirse al
memorándum, los orientara a cumplir con ciertas demandas.\footnote{Antonio
  Pasquali, entrevistado por el autor, 11 de marzo de 2016, Caracas,
  Venezuela.}

A lo largo de 1984, los miembros de ARTEAL optaron por modificar su
nominación hacia la de Unión Latinoamericana y Caribeña de Radiodifusión
(ULCRA), con el objetivo de ampliar el espectro geopolítico y adherir a
organismos del ``tercer sector'', como las instituciones universitarias
y religiosas.\footnote{Pasquali y Vargas Araya, \emph{De la
  marginalidad...}, 177-204.} La nueva institución incorporó a
representantes estatales de la Argentina, Bolivia, Brasil, El Salvador,
México, Nicaragua y República Dominicana. Pasquali fue una figura
destacada en este proceso, primero como Coordinador Regional de la
UNESCO para América Latina y el Caribe y, desde 1986, como director del
CRESALC.\footnote{El CRESALC comenzó su actividad en 1976 y buscaba
  asistir a los estados miembros de la UNESCO a desarrollar las
  instituciones de enseñanza superior e instar a la cooperación técnica
  entre los países de la región.} Su posición al interior de la UNESCO
le permitió motorizar a la Unión en el marco de los convenios de
colaboración sur/sur de Cooperación Técnica de los Países en Desarrollo,
que habilitaba el intercambio de experiencias entre instituciones de los
países de la región. La ULCRA fue constituida como una institución
cooperativa con sede en Costa Rica y representó a más de 200 organismos
de comunicación audiovisual y de radiotelevisión de servicio
público.\footnote{Progresivamente se fueron incorporando, entre varios
  otros, Argentina Televisora Color, Buenos Aires; el Instituto Cubano
  de Radio y Televisión, La Habana; el Sistema Nacional de Radio y
  Televisión Cultural, de Costa Rica; la Universidad de Costa Rica, con
  el canal 15 y la Radio UCR; la Asociación Católica Latinoamericana
  para la Radio y Televisión; la World Association for Christian
  Communication/Región América Latina y Caribe; la Radio Sutatenza, de
  Colombia; el CIESPAL, y los canales 2 y 6 del Sistema Sandinista de
  Televisión, de Nicaragua.} En el proyecto de la ULCRA se puede leer
que la cuestión política de la comunicación se fue desplazando desde una
perspectiva de servicio público más vinculada con los problemas
nacionales hacia una concepción amplia de ``\emph{integración}
regional''.\footnote{Antonio Pasquali, ``Industrias culturales en
  América Latina {[}1988{]}'', en \emph{El orden reina. Escritos sobre
  comunicaciones}, Antonio Pasquali (Venezuela: Monte Ávila Editores,
  1991), 213.} A pesar del ``auge privatizador'', argumentaba Pasquali
como coordinador de investigación de la ULCRA en octubre de 1986 en un
encuentro en México, era ``imperativo luchar por la creación,
fortalecimiento y mejoría de los servicios públicos que estimularan el
cooperativismo y la coproducción regional''. En ese marco y a diferencia
de lo postulado en los tiempos del RATELVE, el ``proyecto de
integración'' hacía énfasis en la ``independencia'' y en la
``pluralización'' del servicio público, concebido con autonomía tanto
del poder político como del económico.\footnote{Antonio Pasquali, ``Qué
  es una radiodifusión de servicio público {[}1986{]}'', en \emph{El
  orden reina}, 153.} Financiado por el erario público, debía responder
a las necesidades de la sociedad y se debía promover la presencia
``activa y concreta del usuario y de sus libres asociaciones en la
gestión de las empresas de radiodifusión''.

La ULCRA debía asegurar el intercambio de la producción entre los países
miembros y con otras regiones del mundo. En un contexto en el que se
producía una ``reconfiguración de la dependencia'' hacia las áreas de la
comunicación, la información y la cultura, afirmaba Pasquali, era
necesario construir ``sólidos mecanismos de intercambio regional'' y
``crear industrias culturales propias y capaces de exportar calidad''.
Como medidas concretas, la ULCRA organizó en 1986 en Guadalajara,
México, el I Mercado Latinoamericano del Audiovisual. Allí participaron
76 instituciones de 22 países en el que además de formalizarse acuerdos
bilaterales de coproducción, intercambio, cooperación y apoyo mutuo, se
compraron y vendieron ``horas de programas de televisión'' y de
``programas de radio''. El II Mercado, organizado al año siguiente
también en Guadalajara, según el testimonio retrospectivo de
Pasquali,\footnote{Antonio Pasquali, entrevistado por el autor, 1 de
  marzo de 2016, Caracas, Venezuela.} fue entendido por los referentes
de la ULCRA como la ``confirmación'' del rol estratégico de la
institución como enlace para fomentar el intercambio de producciones
culturales en la región.

Desde la Unión se entendía que el sur del continente estaba atravesado
por un proceso de fuerte atomización de su sistema de medios, por la
desconexión entre sí y dependencia masiva de la importación de
programación. Para contrarrestar este estado de situación y consolidar
las industrias culturales, la ULCRA pretendió reducir la dependencia de
la importación y promover la circulación de programas y servicios
nacionales y regionales. La ULCRA presentó un análisis pormenorizado de
la situación de los medios de comunicación en el continente y una
propuesta política de cómo reorganizar el sistema de medios ante el
estado de la cuestión. Consideraba fundamental estructurar un ``Espacio
Audiovisual Latinoamericano'' para el cual era ineludible el
establecimiento de ``políticas nacionales de información/comunicación''
que resguardaran la pluralidad cultural y política para ``reafirmar la
soberanía nacional y movilizar la participación de la
comunidad''.\footnote{Pasquali y Vargas Araya, \emph{De la
  marginalidad...}, 153.} Por eso se volvía ``urgente'' desarrollar
sistemas e infraestructuras de comunicación e impulsar medidas
tendientes a fomentar la distribución regional de ``bienes audiovisuales
y establecer normas comunes''.

En cuanto a su posicionamiento teórico, el proyecto de la ULCRA
expresaba dos desplazamientos en lo referente a la formulación de
regulaciones y estrategias para el sector de la comunicación y la
cultura. Como advertimos anteriormente, el primero tenía que ver con un
cambio de concepción en relación con la función del Estado. La
investigadora Elizabeth Fox afirma que tras los regímenes militares y su
accionar represivo sobre las instituciones políticas y culturales,
cualquier responsabilidad gubernamental en lo referente a la
radiodifusión o la prensa, volvió ``sospechosa'' la acción
estatal.\footnote{Elizabeth Fox, ed., \emph{Medios de comunicación y
  política en América Latina: la lucha por la democracia} (México:
  Gustavo Gili, 1988), 23.} Eso generaba la necesidad de formar redes de
integración regional en términos culturales, es decir, construir un
mecanismo de circulación de productos audiovisuales que se complementara
con los intercambios estrictamente económicos entre los países de
América Latina y el Caribe. Al mismo tiempo en que se procuraba la
articulación interna, un ``bloque cultural'' buscaba constituirse para
competir en el mercado internacional de bienes simbólicos. En esta
concepción se producía un segundo desplazamiento: si históricamente la
industria cultural había representado el carácter dominante de la lógica
del capitalismo y era una acción política y comercial para ``alienar'' a
las masas, era necesario repensarlas conceptual y políticamente. El
objetivo, se afirmaba desde la ULCRA, era entender estratégicamente a
las industrias culturales como posibilidad de revitalizar las economías
regionales y ``como instrumento de desarrollo económico, progreso social
y consolidación democrática''.

También se puede leer un reposicionamiento intelectual de Pasquali
respecto a las industrias culturales, inscrito en un desplazamiento más
amplio de los estudios en comunicación latinoamericanos. En términos
particulares marcaba una ruptura con su conceptualización de los años
sesenta sobre la relación entre economía y cultura, mediada por sus
lecturas de la Escuela de Frankfurt. Hacia 1988, en Caracas, durante una
conferencia en el Centro de Estudios Latinoamericanos Rómulo Gallegos,
Pasquali afirmaba que toda ``invectiva abstracta en contra de las
industrias culturales'' en el tercer mundo se volvía un ``discurso
reaccionario y de renuncia''. Se trataba de que los países de la región
se convirtieran en ``productores de bienes y servicios culturales,
rentables y exportables''.\footnote{Pasquali, ``Industrias...'', 219.}
Según Pasquali, esto le iba permitir a la región incorporarse al
``\emph{pool} de productores y emisores culturales'', distribuir riesgos
y beneficios para generar mejores plataformas para los productores
culturales locales. En ese marco, la comunicación ya no podía pensarse
de manera aislada: era una dimensión clave al interior de una política
cultural que ---en línea con la propuesta de \emph{desarrollo}
históricamente estipulada por la UNESCO--- debía promoverse desde el
Estado en función de su horizonte estratégico para la organización de
los procesos económicos y sociales. Se trataba, según recuerda Pasquali,
de iniciar un proceso o un ``camino'' mediante programas y regulaciones
estatales de democratización educativa, cultural y
comunicacional.\footnote{Antonio Pasquali, entrevistado por el autor, 1
  de marzo de 2016, Caracas, Venezuela.}

\hypertarget{palabras-finales}{%
\section{Palabras finales}\label{palabras-finales}}

En este artículo pretendimos analizar la participación de Pasquali como
\emph{mediador intelectual} entre los procesos de producción de
conoc- imiento en comunicación que se estaban efectuando en América Latina
y las preocupaciones político-culturales llevadas adelante por
organismos supranacionales, como la UNESCO. De forma complementaria,
consideramos que Venezuela en particular y América Latina en general se
convirtieron en un destacado polo de producción de saberes en
comunicación, en el marco de los debates internacionales sobre la
circulación y el flujo de la información. Planteamos que la
participación de Pasquali como \emph{mediador} entre los dilemas que
atravesaban a los investigadores de la comunicación y los organismos de
cooperación internacional, fueron claves en el proceso de
institucionalización de las redes de investigación en comunicación que
estaban dispersas en América Latina, que llegó a su punto más alto con
la fundación de la ALAIC.

En este contexto, en América Latina las discusiones sobre el NOII
tuvieron unas dinámicas propias y diferentes respecto a las que
traccionaban a la UNESCO. Los intelectuales latinoamericanos se
vincularon a los ``problemas de la comunicación'' no simplemente como
``teóricos'' sino como \emph{militantes políticos}. Este carácter
implicó que expertos, profesionales e investigadores se relacionaran de
formas diversas con los procesos políticos, de modos más o menos
orgánicos: por ejemplo, Fernando Reyes Matta fue asesor en temas de
comunicación del Ministerio de Relaciones Exteriores de Chile durante el
gobierno de la Unidad Popular y Juan Somavía había sido representante de
la comitiva chilena en el Pacto Andino.\footnote{Daniel Badenes, ``Ya no
  alcanza con las matrices ligadas al pensamiento occidental (entrevista
  a Fernando Reyes Matta)'', \emph{ALAIC} 15, n.° 29 (2019).} Además,
como vimos, un sector de la intelectualidad venezolana como Antonio
Pasquali, Oswaldo Capriles y Héctor Mujica, fueron miembros del Comité
de Radio y Televisión durante el gobierno de Carlos Pérez. La compresión
política de lo comunicacional se anudaba al hecho de que una franja de
la intelectualidad se incorporó activamente a distintas esferas del
Estado y a organizaciones políticas entre finales de los años sesenta y
mediados de los setenta.

Por otro lado, observamos que, al tiempo que se fueron
transnacionalizando los itinerarios personales hacia mediados de la
década de los setenta, producto de exilios y persecuciones
estatal-policiales en el cono sur, se fue transnacionalizando el debate.
Es decir, comenzaron a circular análisis sobre experiencias políticas y
comunicacionales ``nacionales'' en las tramas de la región, conectadas
con un mapa de problemas vinculado al mercado comunicacional
latinoamericano o al orden informativo en un sentido más amplio, a nivel
internacional. El problema de la circulación de la información, el rol
del Estado y la democratización de la comunicación no surgió simplemente
como una ``cuestión teórica'' al interior de la academia sino que se fue
configurando como campo de discusión producto de una \emph{experiencia
vital} que conectó a distintas franjas de la intelectualidad de la
comunicación. Desde las redes intelectuales y los proyectos culturales
que se fueron gestando ---a través de la UNESCO--- como la ULCRA y los
encuentros organizados en el Instituto para América Latina, se apostaba
a finales de los años ochenta por un reordenamiento de los intelectuales
del campo y producir trabajos de investigación y cooperación regional
como modalidad de responder ``con verdaderos planes de desarrollo''
socioculturales a los nuevos procesos de ``mundialización''.\footnote{Rafael
  Roncagliolo, ``Desafíos a la investigación'', \emph{Chasqui}, n.° 31
  (II Época, 1989): 54.}

De todos modos, y aun cuando varios de los ``planes'' se fueron llevando
a cabo, los proyectos regionales tropezarían con unas condiciones
políticas y económicas de gran precariedad. Las redes se volvieron
insostenibles en un marco de debilitamiento político y económico de los
organismos internacionales que años antes habían sido garantía de
financiamiento. La emergencia de políticas neoliberales que no tenían
interés en fortalecer los servicios públicos de radio y televisión de la
región se complementó con la progresiva disminución de los acuerdos
multilaterales. Así, una franja de los intelectuales de la comunicación
observó, siguiendo las palabras de Antonio Pasquali, que las políticas
vinculadas a la producción cultural y comunicacional se fueron volviendo
un escenario de intervención cada vez más hostil y problemático, un
``trabajo sucio'' en tiempos en los que comenzaba la hegemonía de las
tendencias ``privatizadoras''.\footnote{Antonio Pasquali, ``Comunicación
  y Cultura'', en \emph{El orden reina,}163.}

\vspace*{2em}


\section{Bibliography}\label{bibliography}

\begin{hangparas}{.25in}{1} 



Aguirre, Jesús María y Gustavo Hernández Díaz. \emph{Diccionario:
investigadores venezolanos de la comunicación}. Caracas:
ABediciones-UCAB, 2018.

Altamirano, Carlos. ``Elites culturales en el siglo XX
latinoamericano'', en \emph{Historia de los intelectuales en América
Latina. II. Los avatares de la ``ciudad letrada'' en el siglo XX},
editado por Carlos Altamirano. Buenos Aires: Katz, 2010.

Argumedo, Alcira. \emph{Los laberintos de la crisis. América Latina:
poder transnacional y comunicaciones}. Buenos Aires: ILET-Folios
Ediciones, 1984.

Badenes, Daniel. ``Ya no alcanza con las matrices ligadas al pensamiento
occidental (entrevista a Fernando Reyes Matta)''. \emph{ALAIC} 15, n.°
29 (2019): 252-259.

Beltrán, Luis Ramiro. ``ALAIC: el albergue de la inquietud
(entrevista)''. \emph{ALAIC}, n.° 1 (2004).

Beltrán, Luis Ramiro. ``No renunciemos jamás a la utopía (entrevista
exclusiva de Patricia Anzola)''. \emph{Chasqui}, n.° 3 (II Época, 1982):
6-13.

Bisbal, Marcelino y Andrés Cañizález, eds. \emph{Travesía intelectual de
Antonio Pasquali. A propósito de los 50 años de} Comunicación y cultura
de masas. Caracas: UCAB, 2014.

Bolaño, César, Delia Crovi Druetta y Gustavo Cimadevilla, eds. \emph{La
contribución de América Latina al campo de la comunicación}. Buenos
Aires: Prometeo, 2015.

Caletti, Rubén Sergio. ``El nuevo orden informativo: un fantasma del
viejo pasado''. \emph{Comunicación y Cultura}, n.° 13 (1985): 117-124.

Cañizález, Andrés. ``Veinticinco años del Informe Mac Bride. Releyendo
el gran inventario de la comunicación''. \emph{Temas de Comunicación},
n.° 13 (2006): 15-26.

Cañizález, Andrés. ``Yo no creía que ALAIC iba a durar por mucho tiempo
(entrevista a Aníbal Gómez)''. En \emph{Del mimeógrafo a las redes
digitales. Narrativas, testimonios y análisis del campo comunicacional
en el 40 aniversario de ALAIC}, organizado por Delia Crovi Druetta y
Gustavo Cimadevilla, 101-108. Ciudad de México: Ediciones La Biblioteca,
2018.

Capriles, Oswaldo. ``De las Políticas Nacionales de la Comunicación al
Nuevo Orden Internacional de la Información: algunas lecciones para la
investigación''. Ponencia para la Conferencia Científica AIERI-IAMCR.
Caracas: UCV, 1980.

Capriles, Oswaldo. \emph{Elementos para un análisis crítico del Nuevo
Orden Internacional de la Información o de la Comunicación}. Trabajo
presentado para ascender a la categoría de Profesor Agregado. Caracas:
UCV, 1979.

Casullo, Nicolás. ``1980: la UNESCO discute el informe MacBride''.
\emph{Comunicación y Cultura}, n.° 11 (1984): 132-138.

Centro Interdisciplinario Boliviano de Estudios de la Comunicación.
\emph{Asociación Latinoamericana de Investigadores de la Comunicación,
1978-1998. Contribuciones para una memoria institucional}. La Paz:
CIBEC, 1978.

``Conferencia Mundial de AIERI en Caracas''. \emph{ININCO}, n.° 1
(1980): 6-8.

Consejo Nacional de la Cultura. \emph{Proyecto RATELVE}. Caracas: Suma,
1977.

Crovi Druetta, Delia y Gustavo Cimadevilla, orgs. \emph{Del mimeógrafo a
las redes digitales. Narrativas, testimonios y análisis del campo
comunicacional en el 40 aniversario de ALAIC}. Ciudad de México:
Ediciones La Biblioteca, 2018.

Damousi, Joy y Mariano Ben Plotkin, eds. \emph{The Transnational
Unconscious Essays in the History of Psychoanalysis and
Transnationalism}. Nueva York: Palgrave Macmillan, 2008.

Dosse, François. \emph{La marcha de las ideas. Historia de los
intelectuales, historia intelectual}. Valencia: Universitat de Valéncia,
2006.

Esteinou Madrid, Javier. ``El rescate del Informe Mc Bride y la
construcción de un nuevo orden mundial de la información''. \emph{Razón
y Palabra}, n.° 39 (2004).

Fox, Elizabeth, ed. \emph{Medios de comunicación y Política en América
Latina: la lucha por la democracia}. Gustavo Gili: México, 1988.

Fuentes Navarro, Raúl, coord. \emph{Instituciones y redes académicas
para el estudio de la Comunicación en América Latina}. Guadalajara:
ITESO, 2006.

Fuentes Navarro, Raúl. ``La institucionalización académica de las
ciencias de la comunicación: campos, disciplinas, profesiones''. En
\emph{Campo académico de la comunicación: hacia una reconstrucción
reflexiva}, coordinado por Jesús Galindo y Carlos Luna. Ciudad de
México: ITESO, 1995.

Gago, María Verónica. \emph{Controversia, una lengua del exilio}. Buenos
Aires: Biblioteca Nacional, 2012.

Giddens, Anthony. \emph{La constitución de la sociedad. Bases para la
teoría de la estructuración}. Buenos Aires: Amorrortu, 2006.

Gifreu, Josep. \emph{El debate internacional de la comunicación}.
Pamplona: Ariel, 1986.

Gobbi, Maria Cristina\emph{. A batalha pela hegemonia comunicacional na
América Latina. 30 anos da ALAIC}. São Paulo: Universidade Metodista de
São Paulo, 2008.

Gonzaga Motta, Luiz. ``Costa Rica: seis años después''. \emph{Chasqui},
n.° 3 (II Época, 1982): 14-19.

Hamelink, Cees. ``Nuevas estructuras de la comunicación internacional:
el papel de la investigación (selección)''. \emph{Comunicación}, n.°
30-31 (1981): 35-40.

Heilbron, Johan, Nicolas Guilhot y Laurent Jeanpierre. ``Toward a
transnational history of the social sciences''. \emph{Journal of the
History of the Behavioral Sciences} 44, n.° 2 (primavera 2008):
146--160.

Hernández, León\emph{. Pasquali. El último libro, la última entrevista y
el último banquete}. Caracas: ABediciones, 2019.

ILET (Instituto Latinoamericano de Estudios Transnacionales). ``La
información en el nuevo orden internacional. Recomendaciones para la
acción''. \emph{Comunicación}, n.° 18 (1978): 86-95.

Lenarduzzi, Víctor. \emph{Revista} Comunicación y Cultura\emph{.
Itinerarios, ideas y pasiones}. Buenos Aires: Eudeba, 1998.

Lechnert, Norbert. \emph{Las sombras del mañana. La dimensión subjetiva
de la política}. Santiago de Chile: LOM, 2002.

López-Escobar, Esteban. \emph{Análisis del Nuevo Orden Internacional de
la Información}. Pamplona: Ediciones Universidad de Navarra, 1978.

``Maestría en Comunicación Social''. \emph{ININCO}, n.° 1 (1980): 80.

Maíz, Claudio y Álvaro Fernández Bravo, eds. \emph{Episodios en la
formación de redes culturales en América Latina}. Buenos Aires:
Prometeo, 2009.

Marques de Melo, José. \emph{História das ciências da comunicação}. Río
de Janeiro: Mauad, 2008.

Marques de Melo, José. ``Los tiempos heroicos: la formación de la
comunidad latinoamericana de Ciencias de la Comunicación''.
\emph{ALAIC}, n.° 1 (2004).

Marques de Melo, José y Maria Cristina Gobbi, orgs. \emph{Pensamento
Comunicacional Latino-Americano. Da pesquisa denúncia ao pragmatismo
utópico}. São Paulo: UMESP, 2004.

Martín-Barbero, Jesús. ``Retos a la investigación de la comunicación en
América Latina''. \emph{Comunicación y Cultura}, n.° 9 (1983): 99-114.

Martínez Terrero, José. ``Investigación para la toma de decisiones en
políticas de comunicación''. \emph{Comunicación}, n.° 22 (1979):
114-126.

Mastrini, Guillermo y Diego de Charras. ``20 años no es nada: del NOMIC
a la CMSI''. \emph{Anuario ININCO} 2, n.° 17 (2005).

Mujica, Héctor. ``Políticas de comunicación y planificación (Lección
inaugural de la `Maestría en Políticas de Comunicación')''.
\emph{Comunicación}, n.° 28-29 (1980): 48-57.

Murciano, Marcial. ``El Informe MacBride: la búsqueda imposible del
consenso entre norte/sur y este/oeste''. \emph{Anàlisi}, n.° 3 (1981):
109-119.

Pasquali, Antonio. ``Comunicación y Cultura {[}1986{]}''. En Pasquali,
\emph{El orden reina. Escritos sobre comunicaciones}.

Pasquali, Antonio. ``¿Contradicción entre libertad y equilibrio
informativo?''. \emph{Chasqui}, n.° 6 (II Época, 1983): 26-31.

Pasquali, Antonio. ``De la academia a la acción''. En \emph{América
Latina: las comunicaciones de cara al 2000}, editado por Pedro
Goicochea. Lima: IPAL, 1991.

Pasquali, Antonio. \emph{El orden reina. Escritos sobre comunicaciones}.
Venezuela: Monte Ávila Editores, 1991.

Pasquali, Antonio. ``Industrias culturales en América Latina
{[}1988{]}''. En Pasquali, \emph{El orden reina. Escritos sobre
comunicaciones}.

Pasquali, Antonio. \emph{La comunicación cercenada. El caso Venezuela}.
Venezuela: Monte Ávila, 1990.

Pasquali, Antonio. ``Qué es una radiodifusión de servicio público
{[}1986{]}''. En Pasquali, \emph{El orden reina. Escritos sobre
comunicaciones}.

Pasquali, Antonio y Armando Vargas Araya, eds. \emph{De la marginalidad
al rescate: los servicios públicos de radiodifusión en la América
Latina}. San José de Costa Rica: EUNED, 1990.

Pineda de Alcázar, Migdalia. ``Antonio Pasquali: la vigencia de su
pensamiento cuarenta años después''. En Bisbal y Cañizález,
\emph{Travesía intelectual de Antonio Pasquali}, 21-30.

Quirós, Fernando. ``El debate sobre la información, la comunicación y el
desarrollo en la UNESCO durante el siglo XX''. \emph{Commons} 2, n.° 2
(2013): 7-38.

Quirós, Fernando y Francisco Sierra Caballero, eds. \emph{El espíritu
MacBride. Neocolonialismo, Comunicación-Mundo y alternativas
democráticas}. Quito: Ediciones CIESPAL, 2016.

Rey, José Ignacio. ``Encuentro de Investigadores y Nuevo Orden
Informativo Internacional''. \emph{Comunicación}, n.° 30-31 (1981):
32-40.

Reyes Matta, Fernando, coord. \emph{Comunicación alternativa y búsquedas
democráticas}. Ciudad de México: ILET, 1983.

Roncagliolo, Rafael. ``Desafíos a la investigación'', \emph{Chasqui},
n.° 31 (II Época, 1989): 51-55.

Safar, Elizabeth. ``Una constante en la obra de Antonio Pasquali: el
servicio público de radiotelevisión''. En Bisbal y Cañizález,
\emph{Travesía intelectual de Antonio Pasquali}, 47-58.

Salinas, Raquel. \emph{Agencias transnacionales de información y el
tercer mundo}. Quito: The Quito Times, 1984.

Sánchez Ruiz, Enrique. ``Actualidad del Informe MacBride, a 25 años de
su publicación''. \emph{Anuario ININCO} 1, n.° 17 (2005).

Sapiro, Gisèle, Marco Santoro y Patrick Baert. \emph{Ideas on the Move
in the Social Sciences and Humanities. The International Circulation of
Paradigms and Theorists}. Basingstoke: Palgrave Macmillan, 2020.

Schenkel, Peter. ``El Informe Mac Bride: entre la realidad y la
utopía''. \emph{Chasqui}, n.° 1, (II Época, 1981): 81-86.

Schmucler, Héctor. ``Año mundial de la comunicación. Con penas y sin
gloria''. \emph{Comunicación y Cultura}, n.° 11 (1984): 3-8.

Schmucler, Héctor. ``La investigación sobre comunicación en América
Latina en la hora de las computadoras''. Ponencia para la Conferencia
Científica AIERI-IAMCR. Caracas: UCV, 1980.

Schmucler, Héctor. ``Un proyecto de comunicación/cultura''.
\emph{Comunicación y cultura}, n.° 12 (1984): 3-8.

Schmucler, Héctor y Elizabeth Fox. \emph{Comunicación y democracia en
América Latina}. Lima: DESCO, 1982.

Simpson Grinberg, Máximo, comp. \emph{Comunicación alternativa y cambio
social. I. América Latina}. Ciudad de México: UNAM, 1981.

Sola Pool, Ithiel. ``La nueva estructura de la comunicación
internacional: el papel de la investigación (selección)''.
\emph{Comunicación}, n.° 30-31 (1981): 35-40.

Torres, William. ``Investigar la comunicación y formar comunicadores en
América Latina hoy. Una conversación con Jesús Martín-Barbero''. En
\emph{La contribución de América Latina al campo de la comunicación},
editado por César Bolaño, Delia Crovi Druetta y Gustavo Cimadevilla.
Buenos Aires: Prometeo, 2015.

UNESCO (Organización de las Naciones Unidas para la Educación, la
Ciencia y la Cultura). ``21C/Resoluciones, 3/07'', París: UNESCO, 1980.

UNESCO (Organización de las Naciones Unidas para la Educación, la
Ciencia y la Cultura). \emph{Informe provisional sobre los problemas de
comunicación en la sociedad moderna}. París: UNESCO, 1978.

UNESCO (Organización de las Naciones Unidas para la Educación, la
Ciencia y la Cultura). \emph{Un solo mundo, voces múltiples.} Ciudad de
México: Fondo de Cultura Económica, 1980.

\pagebreak Williams, Raymond. \emph{La larga revolución}. Buenos Aires: Nueva
Visión, 2003.

Zarowsky, Mariano. \emph{Del laboratorio chileno a la
comunicación-mundo. Un itinerario intelectual de Armand Mattelart}.
Buenos Aires: Biblos, 2013.



\end{hangparas}


\end{document}