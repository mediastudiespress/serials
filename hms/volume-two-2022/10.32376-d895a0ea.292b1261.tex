% see the original template for more detail about bibliography, tables, etc: https://www.overleaf.com/latex/templates/handout-design-inspired-by-edward-tufte/dtsbhhkvghzz

\documentclass{tufte-handout}

%\geometry{showframe}% for debugging purposes -- displays the margins

\usepackage{amsmath}

\usepackage{hyperref}

\usepackage{fancyhdr}

\usepackage{hanging}

\hypersetup{colorlinks=true,allcolors=[RGB]{97,15,11}}

\fancyfoot[L]{\emph{History of Media Studies}, vol. 2, 2022}


% Set up the images/graphics package
\usepackage{graphicx}
\setkeys{Gin}{width=\linewidth,totalheight=\textheight,keepaspectratio}
\graphicspath{{graphics/}}

\title[Matrices y vertientes de pensamiento]{Matrices y vertientes de pensamiento sobre los medios indígenas en América Latina} % longtitle shouldn't be necessary

% The following package makes prettier tables.  We're all about the bling!
\usepackage{booktabs}

% The units package provides nice, non-stacked fractions and better spacing
% for units.
\usepackage{units}

% The fancyvrb package lets us customize the formatting of verbatim
% environments.  We use a slightly smaller font.
\usepackage{fancyvrb}
\fvset{fontsize=\normalsize}

% Small sections of multiple columns
\usepackage{multicol}

% Provides paragraphs of dummy text
\usepackage{lipsum}

% These commands are used to pretty-print LaTeX commands
\newcommand{\doccmd}[1]{\texttt{\textbackslash#1}}% command name -- adds backslash automatically
\newcommand{\docopt}[1]{\ensuremath{\langle}\textrm{\textit{#1}}\ensuremath{\rangle}}% optional command argument
\newcommand{\docarg}[1]{\textrm{\textit{#1}}}% (required) command argument
\newenvironment{docspec}{\begin{quote}\noindent}{\end{quote}}% command specification environment
\newcommand{\docenv}[1]{\textsf{#1}}% environment name
\newcommand{\docpkg}[1]{\texttt{#1}}% package name
\newcommand{\doccls}[1]{\texttt{#1}}% document class name
\newcommand{\docclsopt}[1]{\texttt{#1}}% document class option name


\begin{document}

\begin{titlepage}

\begin{fullwidth}
\noindent\LARGE\emph{Exclusions in the History of Media studies
} \hspace{25mm}\includegraphics[height=1cm]{logo3.png}\\
\noindent\hrulefill\\
\vspace*{1em}
\noindent{\Huge{Matrices y vertientes de pensamiento sobre los medios indígenas en América Latina\par}}

\vspace*{1.5em}

\noindent\LARGE{María Magdalena Doyle} \href{https://orcid.org/0000-0002-1847-3514}{\includegraphics[height=0.5cm]{orcid.png}}\par}\marginnote{\emph{María Magdalena Doyle, ``Matrices y vertientes de pensamiento sobre los medios indígenas en América Latina,'' \emph{History of Media Studies} 2 (2022), \href{https://doi.org/10.32376/d895a0ea.292b1261}{https://doi.org/ 10.32376/d895a0ea.292b1261}.} \vspace*{0.75em}}
\vspace*{0.5em}
\noindent{{\large\emph{National University of Córdoba}, \href{mailto:magdalenadoyle@gmail.com}{magdalenadoyle@gmail.com}\par}} \marginnote{\href{https://creativecommons.org/licenses/by-nc/4.0/}{\includegraphics[height=0.5cm]{by-nc.png}}}

% \vspace*{0.75em} % second author

% \noindent{\LARGE{<<author 2 name>>}\par}
% \vspace*{0.5em}
% \noindent{{\large\emph{<<author 2 affiliation>>}, \href{mailto:<<author 2 email>>}{<<author 2 email>>}\par}}

% \vspace*{0.75em} % third author

% \noindent{\LARGE{<<author 3 name>>}\par}
% \vspace*{0.5em}
% \noindent{{\large\emph{<<author 3 affiliation>>}, \href{mailto:<<author 3 email>>}{<<author 3 email>>}\par}}

\end{fullwidth}

\vspace*{1em}


\newthought{\hypertarget{resumen}{%
\section{Resumen}\label{resumen}}
Este artículo analiza} la trayectoria investigativa que, dentro del campo
científico sobre comunicación en América Latina, se ha ido configurando
en relación con las prácticas de comunicación que diversos grupos
indígenas han desarrollado desde sus propios medios en distintas
regiones. A partir de la categoría epistémica ``matrices de
pensamiento'', se abordan las trayectorias de los estudios sobre
comunicación indígena. Con ello, se da cuenta del modo en que los
abordajes y sus marcos conceptuales se fueron emplazando en los
distintos momentos históricos en los que emergieron, se actualizaron, se
adaptaron o se enriquecieron, de forma que la controversia teórica se
engarza con debates políticos sustantivos respecto de las identidades
indígenas y de los medios.

\hypertarget{abstract}{%
\section{Abstract}\label{abstract}}

This article analyzes the investigative trajectory that, within the
scientific field on communication in Latin America, has been configured
in relation to the communication practices that different indigenous
groups have developed from their own media in different regions. From
the epistemic category "matrices of thought", the trajectories of
studies on indigenous communication are analyzed. Thus, it 

\enlargethispage{2\baselineskip}

\vspace*{2em}

\noindent{\emph{History of Media Studies}, vol. 2, 2022}




 \end{titlepage}

\noindent shows the way
in which the approaches and their conceptual frameworks were located in
the different historical moments in which they emerged, were updated,
adapted or enriched, so that the theoretical controversy is linked with
substantive political debates referred to indigenous and media
identities.

\vspace*{2em}

\hypertarget{introduccin}{%
\section{Introducción}\label{introduccin}}

\newthought{A partir del año 2000}, hemos asistido en América Latina a una creciente
multiplicación de investigaciones que, desde el campo de los estudios
sobre comunicación, analizan los medios y las prácticas de comunicación
mediatizada de los pueblos indígenas de esta región del continente, en
general vinculando esas prácticas a diversos procesos de luchas por
derechos que protagonizan esos pueblos. Sin embargo, la trayectoria
investigativa sobre estas experiencias y prácticas se remonta a muchas
décadas atrás, al menos hasta la década de los setenta, y los modos de
abordaje, enfoques teóricos y conceptualizaciones se han ido
transformando y diversificando a lo largo de los años.

La propuesta de este artículo es reconstruir y analizar el tejido de
significados y referencias de que está hecha esa trayectoria conceptual,
emplazando sus reconfiguraciones en una doble inscripción: por un lado,
en los tiempos de las matrices de pensamiento que fueron delineando de
modo más general al ``campo científico'' de la comunicación en América
Latina\footnote{Maria Immacolata Vassallo de Lopes, ``El campo de la
  comunicación: sobre su estatuto disciplinar'', \emph{Oficios
  Terrestres}, n.º 7 (La Plata: UNLP): 74-83,
  \url{http://sedici.unlp.edu.ar/handle/10915/47380}.} y, por otro,
partimos de asumir la intrínseca articulación de esa historia conceptual
con los tiempos largos\footnote{Aníbal Ford, \emph{Navegaciones.
  Comunicación, cultura y crisis} (Buenos Aires: Amorrortu, 1994).} de
los procesos políticos/económicos/socioculturales de transfiguración de
las identificaciones vinculadas a la indigenidad en este continente. En
ese sentido, proponemos analizar el movimiento de gestación de los
desarrollos, las conceptualizaciones sobre la comunicación indígena
mediática, estableciendo vínculos con los contextos históricos en que
ciertas marcas de la indigenidad se activaron o no, y con las
disputas/tensiones que esas marcas vehiculizaron.

En términos de nuestra propuesta analítica, organizamos esta
sistematización en tres periodos diferenciados, asumiendo que toda
clasificación resulta en algún punto arbitraria, pero a la vez permite
ordenar y reconocer continuidades y rupturas en esta trayectoria
investigativa:

\begin{itemize}
\item
  Primer periodo: abarca textos publicados entre inicios de 1970 y
  mediados de los ochenta. Durante estos años se produjeron los primeros
  trabajos que abordaron el vínculo entre poblaciones indígenas y medios
  educativos, y populares. Estos análisis están signados por la tensión
  teórica y política entre desarrollo y dependencia, resultando de ello
  dos tipos de enfoques en los trabajos: algunos centrados en dar cuenta
  de la capacidad de esos medios para mejorar los niveles de vida de la
  población (enfoque del desarrollo), y otros analizando la capacidad de
  esos medios para movilizarles políticamente y dotarles de voz (enfoque
  de la dependencia).
\item
  Segundo periodo: abarca trabajos publicados entre la segunda mitad de
  la década de los ochenta y la primera mitad de los noventa. Como
  veremos, este constituye un periodo bisagra tanto de los análisis de
  la etapa anterior como de las primeras reflexiones sobre las
  especificidades culturales de los usos de la tecnología en el marco de
  la puesta en foco de las identidades étnicas. Esta etapa marca el
  inicio del proceso de profesionalización del campo de la comunicación
  en general, y ello incide en las condiciones de enunciación de los
  estudios sobre esta temática vinculada a medios indígenas (por
  ejemplo, se comienzan a producir tesis de posgrado sobre el tema).
\item
  Tercer periodo: abarca trabajos publicados entre la segunda mitad de
  la década de los noventa y hasta la actualidad. En este periodo,
  encontramos más claramente delineadas dos vertientes en los estudios
  sobre esta temática: por un lado, análisis que se focalizan en los
  usos de las tecnologías de la comunicación digitales por pueblos
  indígenas y exploran las potencialidades democratizadoras de esas
  tecnologías y, por otro, trabajos que exploran la complejidad de las
  articulaciones entre luchas políticas, procesos de identificación
  étnicas y medios de comunicación, atendiendo tanto a los marcos de
  desigualdades expresivas como a las mutuas y complejas relaciones de
  performatividad entre esas tres dimensiones (identidades, medios y
  política).
\end{itemize}

Este artículo, elaborado centralmente a partir de la sistematización y
análisis bibliográfico sobre el tema, es parte de la investigación que
desarrollamos en el Doctorado en Antropología en la Universidad de
Buenos Aires\footnote{Magdalena Doyle, ``El derecho a la comunicación de
  los pueblos originarios. Límites y posibilidades de las
  reivindicaciones indígenas en relación al sistema de medios de
  comunicación en Argentina'' (tesis doctoral en Antropología,
  Universidad de Buenos Aires, 2017).} y se inscribe también en una
trayectoria de colaboración activista en emisoras de pueblos indígenas
en Argentina.

Para este trabajo en particular, relevamos materiales que constituyen el
corpus de análisis, conformado de acuerdo a los siguientes criterios:
artículos publicados en revistas científicas y libros que den cuenta de
investigaciones sobre prácticas, medios, experiencias de comunicación
indígena en América Latina; tesis de posgrado publicadas sobre esa
temática; ponencias que son parte de actas de congresos; todo ello en un
recorte de tiempo que abarca los tres periodos que enunciamos
previamente. Si bien asumimos que, como todo corpus, posee un grado de
arbitrariedad, procuramos construir uno lo suficientemente amplio y
diverso que permita dar cuenta de las trayectorias, vertientes y
rupturas en el abordaje sobre esta temática.

Antes de avanzar en el análisis de esos trabajos, nos interesa definir
brevemente la categoría de matrices de pensamiento, que constituye el
punto de partida y el marco de este abordaje.

\hypertarget{sobre-matrices-y-vertientes-de-pensamiento}{%
\section{Sobre matrices y vertientes de
pensamiento}\label{sobre-matrices-y-vertientes-de-pensamiento}}

\enlargethispage{\baselineskip}

Las grandes corrientes de las ciencias humanísticas y sociales, afirma
la socióloga argentina Alcira Argumedo, están intrínsecamente vinculadas
a proyectos históricos y políticos de los momentos en que emergen y se
desarrollan. Y la configuración del campo científico de la comunicación
no ha sido una excepción.

Al interior del pensamiento sobre lo histórico y lo social, afirma
Argumedo, existe una multiplicidad de corrientes teóricas. El carácter
complejo y polémico de dicho campo se debe a la forma en que el mismo se
constituye y reconstituye en una relación históricamente condicionada
con los procesos políticos-culturales, ``como expresión de visiones del
mundo que impregnan los más diversos aspectos del acontecer de las
sociedades''.\footnote{Alcira Argumedo, \emph{Los silencios y las voces
  en América Latina. Notas sobre el pensamiento nacional y popular}
  (Buenos Aires: Ediciones del Pensamiento Nacional, 2004).} A su vez,
ello no implica plantear dicotomías que imposibiliten el diálogo entre
las distintas corrientes de pensamiento, ya que estas se vertebran en
marcos más amplios, en concepciones culturales y modos de percibir el
mundo que les otorgan sus significaciones esenciales al margen de sus
especificidades, lo cual da la posibilidad de encontrar puntos de
acuerdo y líneas de continuidad. Para el análisis de esa compleja
configuración de las trayectorias investigativas en ciencias sociales y
humanas, Argumedo propone la categoría de matrices de pensamiento,
entendiéndolas como ``la articulación de un conjunto de categorías y
valores constitutivos que conforman la trama lógico conceptual básica y
establecen los fundamentos de una determinada corriente de
pensamiento''.\footnote{Ibíd., 79.} Una matriz de pensamiento define, de
una u otra manera, una concepción acerca de las sociedades; del devenir
histórico; elementos para la comprensión de los fenómenos del presente;
que formula planteos acerca de los actores protagónicos del devenir
histórico y social; hipótesis referidas a los fenómenos políticos,
sociales, económicos y culturales, y fundamentos para optar entre
valores o intereses en conflicto. En coherencia con estas nociones, y
con una concepción acerca de qué es la ciencia social, cuáles son sus
formas de objetividad y conocimiento del objeto de estudio, cobran
sentido los distintos conceptos y metodologías de abordaje. En síntesis,
las matrices de pensamiento son sistematizaciones conceptuales de
determinados modos de percibir el mundo (o aspectos de él).

Tal como plantea Argumedo, la definición de las matrices de pensamiento
permite detectar las líneas de continuidad o ruptura de los conceptos,
valores, propuestas de las principales corrientes de pensamiento en un
determinado campo de investigación.\footnote{Alcira Argumedo, ``Las
  matrices del pensamiento teórico-político'', en \emph{Antología del
  pensamiento crítico argentino contemporáneo}, coord. de Sergio
  Caggiano y Alejandro Grimson (Buenos Aires: CLACSO, agosto de 2015),
  141.} A su vez, la autora propone la noción de ``vertientes'' para
referir a las diversas expresiones o modos particulares de desarrollo
teórico que se procesan dentro de las coordenadas impuestas por la
articulación conceptual de una matriz. Estas vertientes, explica,
constituyen ramificaciones de un tronco común y reconocen una misma
matriz, no obstante sus múltiples matices, sus características
particulares, sus eventuales contradicciones o los grados de
refinamiento y actualización alcanzados por cada una de ellas.

A continuación nos detendremos en los periodos históricos definidos
anteriormente, para dar cuenta de las conceptualizaciones en torno a la
comunicación indígena mediática presentes en ellos. Entendemos que estas
conceptualizaciones constituyen vertientes que se desarrollaron
alrededor de matrices teóricas más amplias del campo de la comunicación.
Y que, a la vez, no pueden comprenderse con autonomía de los modelos de
sociedad en el marco de los cuales se fueron transformando las luchas
políticas y los procesos identificatorios de los pueblos indígenas.

\hypertarget{primer-periodo-tensiones-fundantes-de-los-estudios-sobre-comunicacin-indgena-en-amrica-latina}{%
\section{Primer periodo: tensiones fundantes de los estudios sobre\\\noindent
comunicación indígena en América
Latina}\label{primer-periodo-tensiones-fundantes-de-los-estudios-sobre-comunicacin-indgena-en-amrica-latina}}

Los años sesenta fueron de gran convulsión política, cultural y
económica en todo el mundo. La posguerra, la guerra fría, la alianza por
el progreso, el movimiento de países no alineados y el movimiento de
países tercermundistas en su lucha por la descolonización, fueron
influencias claves en el desarrollo de los estudios sobre comunicación,
sobre todo en América Latina. Distintos autores que historizaron y
cartografiaron el campo, tal como Raúl Fuentes Navarro,\footnote{Raúl
  Fuentes Navarro, ``La investigación de la comunicación en América
  Latina: condiciones y perspectivas para el siglo XXI'', \emph{Oficios
  Terrestres}, n.º 6 (1999): 56-67,
  \url{http://sedici.unlp.edu.ar/handle/10915/38040}.} Jesús
Martín-Barbero\footnote{Jesús Martín-Barbero, \emph{Oficio del
  cartógrafo. Travesías latinoamericanas de la comunicación en la
  cultura} (Chile: Fondo de Cultura Económica, 2002).} o Luciano
Sanguinetti,\footnote{Luciano Sanguinetti, \emph{Comunicación y medios.
  Claves para pensar y enseñar una teoría latinoamericana sobre
  comunicación} (La Plata: Facultad de periodismo y comunicación
  social-UNLP, 2001).} aluden a dos matrices preponderantes en esta
región del continente, durante aquellos años. Por un lado, están las
investigaciones que analizaban la relación entre la comunicación y el
poder en los procesos de dominación de las masas y dependencia cultural,
que tenían a los medios masivos de comunicación como protagonistas. Por
otro, surgen aquellas matrices que afirman la posibilidad de lograr el
desarrollo de los pueblos utilizando a los medios de comunicación como
instrumentos de difusión de los avances modernos, para superar el
subdesarrollo:

\begin{quote}
En las múltiples intervenciones de su carrera, Martín-Barbero expresó su
visión de que dos concepciones fundacionales opuestas caracterizaban los
estudios de comunicación latinoamericanos, y la necesidad de superarlas:
``Por un lado, estaba el paradigma Funcional'', que relacionaba el
estudio de comunicación a la difusión de innovaciones; y ``por otro
lado, estaba la Teoría de la Dependencia'', que afirmaba que la
comunicación de masas formaba ``parte del proceso que incluía la
dominación que debían soportar los países latinoamericanos''. Los
argumentos detrás de la defensa de Martín-Barbero de su posición sobre
las estrategias del campo para hacer frente a las configuraciones
cambiantes de las dimensiones comunicación-cultura-política de las
sociedades latinoamericanas, se volvieron progresivamente explícitos y
coherentes a medida que debatía, a veces ferozmente, con líderes y
seguidores de otras escuelas de pensamiento, especialmente algunos
economistas políticos críticos y posmodernistas idealistas. La historia
del campo que trazó, así como el enfoque teórico y metodológico que
sostuvo, se convirtió en una de las fuentes más influyentes de la
``escuela latinoamericana de pensamiento''. Dio forma tanto a las
prácticas de investigación como a la formación académica de
profesionales que intentaban ``comprender el papel que jugaron los
procesos de comunicación y los medios de comunicación en los cambios que
se estaban produciendo en América Latina'' (y en otros lugares) antes y
después del cambio de siglo.\footnote{Raúl Fuentes Navarro,
  ``Communication Research in Latin America: Will the ``Nocturnal Map''
  Survive or Fade Away?'', \emph{History of Media Studies} 1 (2021); la
  traducción es propia.}
\end{quote}

A su vez, atravesada originalmente también por esta tensión entre el
desarrollo y la dependencia, durante este periodo nació una línea de
investigación y trabajo en comunicación en América Latina que se
constituyó en matriz para pensar los medios de comunicación, llamada
``comunicación popular''. Esta matriz, que en términos académicos tuvo
siempre cierta marginalidad, surgió en íntima vinculación con el proceso
de transformación de la Iglesia católica en Latinoamérica, luego del
Concilio Vaticano II, y sus pronunciamientos en relación a los sectores
oprimidos. Desde esa concepción, el oprimido es el sujeto sin voz y, en
ese marco, la comunicación comenzó a ser considerada una práctica
liberadora.\footnote{Giselle Munizaga y Anny Rivera, \emph{La
  investigación en comunicación social en Chile} (Lima: Desc, 1983): 26.}
Con ello, muchos investigadores de la comunicación comenzaron a plantear
que la democratización de las relaciones en las sociedades
latinoamericanas vendría de recuperar las formas de comunicación de esos
sectores populares que se organizaban para enfrentar al opresor, y lo
hacían partiendo de modelos de comunicación participativa y
horizontal.\footnote{Robert White, ``La teoría de la comunicación en
  América Latina. Una visión europea de sus contribuciones'',
  \emph{Telos}, n.º 19 (1989): 43-54.}
  
Tal como plantea María Cristina Mata,\footnote{María Cristina Mata,
  ``Comunicación popular: continuidades, transformaciones y desafíos'',
  \emph{Oficios Terrestres}, n.º 26 (abril 2011): 1-22.} si bien
inicialmente las experiencias de comunicación popular eran pensadas como
los espacios desde los cual dar o dotar de voz a los sujetos oprimidos,
al pueblo, en los primeros años de la década de los setenta,

\begin{quote}
\dots la idea de una mayoría ``sin voz'' a la cual debía dársele, fue
discutida y revisada. Se argumentaba que postular que los explotados y
marginados no tenían voz era desconocer una palabra que se revelaba en
sus prácticas, en su capacidad de organización y de lucha... Pero
también desconocer que esa palabra era modo de vivir, de imaginar,\newpage de
soñar, de pensar. Una palabra dominada y resistente, por eso fragmentada
y contradictoria, en la que estaban inscriptas tradiciones e
historias.\footnote{Mata, ``Comunicación popular'', 13.}
\end{quote}

Alrededor de esta matriz y, por lo tanto, emplazados también en las
tensiones entre el desarrollo y la dependencia, a principios de la
década del setenta comenzaron a producirse trabajos que reflexionaron
sobre los vínculos de los pueblos indígenas y estos medios de
comunicación, en aquel momento denominados populares, mineros,
educativos o campesinos.

Además de su configuración en el marco de los estudios sobre
comunicación popular, estos trabajos también se vieron influenciados por
su desarrollo en un contexto político que, aunque con matices, era común
a los distintos países de la región: el indigenismo integracionista de
las políticas orientadas a la población indígena implementadas en
América Latina a partir de la década de los años cuarenta, con el cual
se institucionalizó un impulso homogenizador tendiente a la
invisibilización de dichos pueblos.\footnote{Héctor Díaz Polanco,
  ``Derechos indígenas en la actualidad'', \emph{Boletín de Antropología
  Americana}, n.º 33 (1998): 91-99,
  \url{https://www.jstor.org/stable/40978130}.} Cabe recordar que este
indigenismo integracionista fue una expresión de los proyectos
desarrollistas y modernizadores en el área de la economía, la política y
la cultura, que orientaron el destino de los países latinoamericanos
centralmente durante las décadas de los sesenta y setenta. Desde esta
concepción se afirmaba que para transformar a los países periféricos en
naciones desarrolladas y con mayor autonomía era necesario incrementar
la intervención del Estado en los distintos planos de la vida
socioeconómica de cada país. Se trataba de medidas de corte asistencial
que apuntaban, en el fondo, a garantizar la sobrevivencia del
capitalismo con nuevos instrumentos de intervención e interpretación,
capaces de solucionar las contradicciones más apremiantes de la
acumulación de capital. Desde este lugar, se planteaba que la población
indígena, esencialmente campesina en aquellos años, debía ser sacada de
su atraso y su refugio en la tradición, objetivo frente al cual la
lengua, la cultura y todos aquellos ``elementos o señales de identidad
indígena no eran más que obstáculos a superar''.\footnote{Álvaro Bello,
  \emph{Etnicidad y ciudadanía en América Latina. La acción colectiva de
  los pueblos indígenas} (Chile: CEPAL / Sociedad Alemana de Cooperación
  Técnica, 2004): 62.} Este objetivo asimilacionista impulsó, entre
otros, muchos de los programas de alfabetización de las comunidades.

Asimismo, a fines de los años cincuenta algunos gobiernos de la región
---como el caso de Bolivia (1953), Chile (1962), Ecuador (1964) o Perú
(1969)--- impulsaron, con diferencias en cada caso, reformas agrarias
que contribuyeron a cambiar en alguna medida las condiciones de vida de
los campesinos y, en muchos casos, fueron la plataforma para el
desarrollo de organizaciones campesinas e indígenas. Muchas de estas
organizaciones surgidas en los años cincuenta estuvieron vinculadas
desde su origen a intelectuales y organizaciones de la izquierda
latinoamericana. Por lo general, en esos sectores ---salvo el caso de
algunos pensadores como, por ejemplo, José Mariátegui en Perú o Guzmán
Böckler en Guatemala--- existieron ciertas reticencias para aceptar el
hecho de una movilización política indígena como una entidad propia, con
lo cual la tendencia fue integrar las luchas indígenas y las reflexiones
sobre las mismas en la lógica de la lucha campesina.

Al mismo tiempo, y tensionando esas tendencias que desde ámbitos
académicos y políticos tendían a la invisibilización de la población
indígena, durante aquellos años comenzó a expresarse una crítica entre
científicos sociales latinoamericanos respecto del indigenismo
asimilacionista al que aludimos. Una expresión de estos cambios fueron
los documentos conocidos como Declaraciones de Barbados I, II y III,
dadas a conocer en 1971, 1979 y 1994,

\begin{quote}
\ldots en las cuales un grupo de antropólogos de toda América Latina
cuestionábamos las políticas indigenistas vigentes y demandábamos la
liberación del indígena a través de su autogestión, autodeterminación y
la configuración de autonomías. Nuestros documentos propusieron la
redefinición de los estados en términos étnicamente plurales, lo que
provocó la reacción antagónica de los ideólogos de la homogeneización
cultural y política. Los rígidos paradigmas de índole economicista que
tipificaban ese momento histórico de la reflexión social y política en
México, impidieron que la toma de conciencia respecto a las dimensiones
de la cuestión étnica ocurriera en forma simultánea a la de otros países
de América Latina.\footnote{Miguel Bartolomé, ``Pluralismo cultural y
  redefinición del Estado en México'', \emph{Série Antropologia}, n.º
  210 (1996): 8.}
\end{quote}

Estos planteos fueron parte de los puntapiés iniciales para las
transformaciones ocurridas en el siguiente periodo respecto de los modos
de pensar, nombrar y actuar las adscripciones identitarias étnicas en
relación con los pueblos indígenas. Pero habría de pasar más de una
década desde la primera Declaración hasta que, con los retornos de las
democracias en distintos países de América Latina, este debate se
extendiese en la academia y en otros ámbitos políticos.\footnote{Bello,
  ``Etnicidad y ciudadanía''.}

Aquel escenario complejo, entonces, junto a las tensiones que
configuraban el campo de estudios sobre comunicación, constituyeron el
marco de emergencia de las primeras publicaciones académicas que
aludieron a la participación de población indígena en medios: en algunos
casos poniendo en cuestión y en otros reproduciendo los sentidos
hegemónicos y las disputas políticas en torno a los pueblos indígenas a
que aludimos antes. Y, en el mismo sentido, esta diversidad de
vertientes tuvo su correlato en los distintos modos en que dichos
trabajos conceptualizaron a los medios y reflexionaron sobre su rol
respecto de los procesos de integración o expresión de esos pueblos.

Respecto de su circulación, los trabajos de este periodo se publicaron
centralmente en formato de artículos que se difundían en las principales
publicaciones sobre comunicación/educación popular creadas en los años
setenta y principios de los ochenta, algunas impulsadas por centros de
investigación en comunicación y otras por la Iglesia católica o grupos
de laicos. La revista que publicó el mayor número de trabajos sobre esta
temática durante aquellos años fue \emph{Chasqui},\footnote{Jerry
  O'Sullivan, ``Informe de la investigación exploratoria sobre nuevas
  posibilidades del radio en la Tarahumara'', \emph{Chasqui. Revista
  Latinoamericana de Comunicación}, n.º 9 (1975): 27; Margarita Nolasco
  Armas, ``Educación y medios de comunicación masiva'', \emph{Chasqui.
  Revista Latinoamericana de Comunicación}, n.º 5 (1974): 25-38; Ray
  Chesterfield y Kenneth Renddle, ``El uso de los canales indígenas de
  comunicación en el desarrollo rural venezolano'', \emph{Chasqui.
  Revista Latinoamericana de Comunicación}, n.º 12 (1976): 11-18; Xavier
  Albó y Néstor Quiroga, ``La radio como expresión libre del Aymara'',
  \emph{Chasqui. Revista Latinoamericana de Comunicación}, n.º 7 (1974):
  109-26; Xavier Albó, ``Idiomas, escuelas y radios en Bolivia'',
  \emph{Chasqui. Revista Latinoamericana de Comunicación}, n.º 6 (1974):
  92-130.} una publicación del Centro Internacional de Estudios
Superiores de Comunicación para América Latina (CIESPAL), que comenzó a
editarse en 1972 en Ecuador. También publicaron artículos sobre esta
temática la revista \emph{Nueva Sociedad},\footnote{Orlando Encinas
  Valverde, ``La radio al servicio de la liberación indígena: Radio
  Mezquital. La radio como instrumento de apoyo en programas de
  desarrollo integral'', \emph{Nueva Sociedad}, n.º 25 (julio-agosto
  1976): 85-94.} una publicación de ciencias sociales perteneciente a la
Fundación Friedrich Ebert que se publica desde 1972, y la revista
\emph{Comunicación y Cultura} y la \emph{Cultura Popular}. \emph{Revista
Latinoamericana de Educación Popular},\footnote{Antonio Oseguera, ``Una
  experiencia de comunicación educativa para el desarrollo rural'',
  \emph{Comunicación y Cultura}, n.º 8 (julio 1982): 33-38; Alfonso
  Gumucio Dagron, ``El papel político de las radios mineras'',
  \emph{Comunicación y Cultura}, n.º 8. (julio 1982): 89-100.} creada en
1981 por la Comisión Evangélica Latinoamericana de Educación Cristiana.
También pueden hallarse publicaciones no periódicas del CIESPAL en las
cuales se presentan sistematizaciones de experiencias de
comunicación/educación popular y, en menor medida, se publicaron
artículos sobre esta temática.\footnote{Francisco Santos Velasco y
  Francisco Reyes Ruiz, \emph{Una experiencia de comunicación educativa
  a través de la radio en la zona lacustre de Pátzcuaro} (Ecuador:
  CIESPAL / CREFAL, 1983).} Del análisis de esos trabajos, emerge la
tensión transversal al campo de la comunicación en América Latina.

Por un lado, nos encontramos con textos que adscriben a la perspectiva
según la cual los medios de comunicación son \emph{herramientas al
servicio del desarrollo} de los pueblos indígenas, de la reversión de su
situación de ``estancamiento cultural''\footnote{O'Sullivan, ``Informe
  de la investigación''.} y paliativos de las dificultades que las
instituciones de la educación formal tenían para educar a la población
de las comunidades indígenas.\footnote{Nolasco Armas, ``Educación y
  medios''; O'Sullivan, ``Informe de la investigación''; Chesterfield y
  Renddle, ``El uso de los canales indígenas''; Santos Velasco y Reyes
  Ruiz, \emph{Una experiencia de comunicación}.} En estos textos se
parte de entender a la política en tanto ordenamiento
jurídico-institucional particular, se afirma una jerarquía de saberes y
valores culturales, y la comunicación se entiende como aquello que
efectúen unos ``medios técnicos con capacidad de intervención en el
espacio social a través de la diseminación de mensajes que refieren, en
términos generales... a los valores que se desprenden de ese
ordenamiento y de esas instituciones''.\footnote{Sergio Caletti,
  \emph{Comunicación, política y espacio público. Notas para repensar la
  democracia en la sociedad contemporánea. Borradores de trabajo}
  (Buenos Aires: UBA, 1998-2002): 28.} Es decir, los medios de
comunicación se piensan, en esos textos, en tanto herramientas que
permitirían coadyuvar en la integración de campesinos a la condición
ciudadana y al mercado, como parte de una lógica modernizante y
republicana democrática.

Por otro, esos artículos coexisten, durante el mismo periodo y en muchos
casos en las mismas revistas, con otros textos que analizan los procesos
políticos vinculados a la indigenidad con una fuerte impronta de la
clave analítica de las luchas de clase\footnote{Albó y Quiroga, ``La
  radio como expresión''; Albó, ``Idiomas, escuelas y radio''; Encinas
  Valverde, ``La radio al servicio''; Oseguera, ``Una experiencia de
  comunicación''; Gumucio Dagron, ``El papel político''.} y, en ese
sentido, el sujeto de estos textos son los sectores populares (sobre
todo campesinos o mineros) y la comunicación es pensada en vínculo con
los procesos de liberación de la palabra de los sectores oprimidos. Tal
como afirma María Cristina Mata,\footnote{Mata, ``Comunicación
  popular''; Oseguera, ``Una experiencia de comunicación''.} si bien
desde esta perspectiva inicialmente se pensaban los espacios de
comunicación popular como lugares desde los cual dotar de voz a los
sujetos, luego ello fue discutido y revisado y comenzó a plantearse la
necesidad de reconocer la palabra de los sectores oprimidos en sus
prácticas, en su capacidad de organización y de lucha, y de asumir que
en esa palabra, dominada y resistente, estaban inscriptas tradiciones e
historias. En el marco de esas discusiones y ese reconocimiento,
comenzaron a surgir trabajos que problematizaron la histórica opresión a
que fueron sometidos los indígenas, y pensaron a los espacios de
comunicación en tanto ámbitos para la revalorización de las expresiones
culturales de estos pueblos. Así, por ejemplo, encontramos textos
publicados en la revista \emph{Chasqui} donde se afirma que ``el único
camino viable para que los sectores nativos oprimidos puedan luchar por
sus \emph{reivindicaciones} es adquirir el nivel técnico del
dominante... y ello sólo es posible con el dominio oral y escrito del
castellano''.\footnote{Albó, ``Idiomas, escuelas y radio'', 104.}
Además, afirman que ``el idioma nativo debe ser una meta... para que el
hombre quechua y aymara logre realmente comunicar lo que es y siente a
propios y extraños''\footnote{Ibíd., 105.} y que ``la radio está
ayudando a revitalizar una actitud emocional de aprecio por lo propio en
los grupos más oprimidos''.\footnote{Ibíd., 110.}

Una cuestión común a los trabajos de esta época es que, en términos
metodológicos, se trata de análisis centrados en el abordaje descriptivo
o exploratorio de experiencias de las cuales los autores de los textos
habían sido parte o en las que habían colaborado. Respecto de los
enfoques metodológicos y técnicas trabajadas, algunas publicaciones
explicitan haber retomado el enfoque dialógico de Paulo Freire y el
trabajo en talleres o seminarios durante los cuales se produjeron
debates colectivos y conversaciones informales con integrantes de las
experiencias descriptas.\footnote{O'Sullivan, ``Informe de la
  investigación''; el trabajo de Albó y Quiroga, ``La radio como
  expresión'', argumenta sobre el aporte de la ``pedagogía del diálogo''
  de Paulo Freire para fomentar la ``participación'' indígena en las
  radios; Santos Velasco y Reyes Ruiz, \emph{Una experiencia de
  comunicación.}}

En otros trabajos, en cambio, no se mencionan el enfoque \\\noindent metodológico ni
las técnicas sino que se describen las experiencias y se argumenta su
importancia al interior de las comunidades que se
inscribieron.\footnote{Nolasco Armas, ``Educación y medios'';
  Chesterfield y Renddle, ``El uso de los canales indígenas''; Albó,
  ``Idiomas, escuelas y radio''; Encinas Valverde, ``La radio al
  servicio''; el texto de Gumucio Dagron, ``El papel político'', explica
  el papel de las radios mineras en la resistencia al golpe de estado de
  1980 en Bolivia, y retoma fragmentos textuales de emisiones de las
  radios, para dar cuenta del modo en que se trataba el tema en esos
  medios.} El objetivo de esos trabajos es, centralmente, visibilizar
las experiencias y analizar sus roles y relevancia, antes que dar cuenta
de los procesos de investigación o intervención que motivaron los
textos. Esta escasa alusión a las definiciones metodológicas tal vez se
deba, hipotetizamos, a una entonces incipiente institucionalización del
campo en general y de sus pautas de circulación de los trabajos
publicados; al hecho de que muchos de los textos estaban vinculados a
experiencias de intervención y no investigación estrictamente, y a la no
exigencia de requisitos vinculados a la explicitación de los marcos
metodológicos de los trabajos en las revistas académicas durante ese
periodo.

\newpage

En síntesis, en el marco de las matrices teóricas predominantes en el
periodo y del contexto político y cultural al que hicimos referencia,
gran parte de las investigaciones y reflexiones producidas en los años
setenta y principios de los ochenta sobre las experiencias de
comunicación con participación indígena, pensaron estas prácticas en
tanto comunicación popular o comunicación rural, y los espacios en que
ellas se desplegaban como radios mineras o radios campesinas, en general
asumiéndolos al servicio de la educación de la población. Sin embargo,
si bien es posible hablar de tendencias predominantes, también es
importante, incluso para comprender los posteriores desarrollos
investigativos en la materia, reconocer la existencia de las distintas
vertientes de esas matrices y las tensiones que atravesaron estas
producciones sobre experiencias de comunicación con participación de
población indígena.

Alrededor de estas tensiones comienzan a producirse trabajos donde se
introduce, entre las dimensiones de análisis de estas prácticas de
comunicación, algunos ejes tales como los vínculos entre las dinámicas
sociolingüísticas de un pueblo indígena particular y los modos de
constituirse en emisores y audiencia de radios,\footnote{Albó,
  ``Idiomas, escuelas y radio''.} o entre el uso de la radio y las
formas de organización política de cada pueblo.

De este modo, se daban los primeros pasos en la búsqueda por comprender
los procesos reivindicatorios emprendidos por los pueblos indígenas y su
vinculación con las posibilidades y los modos de expresión pública de
los sujetos en una particular articulación de configuraciones culturales
y trayectorias y coyunturas político-económicas.

\hypertarget{segundo-periodo-entre-los-viejos-interrogantes-y-las-nuevas-conceptualizaciones}{%
\section{Segundo periodo: entre los viejos interrogantes y las nuevas\\\noindent
conceptualizaciones}\label{segundo-periodo-entre-los-viejos-interrogantes-y-las-nuevas-conceptualizaciones}}

Desde mediados de la década de los ochenta, una serie de factores
contribuyó a modificar poco a poco el marco político de las luchas
indígenas en el continente y, con ello, también las investigaciones
sobre las experiencias de comunicación indígena que surgían en torno a
aquellas luchas. Si bien no vamos a extendernos en este punto, es
posible afirmar que desde la década de los ochenta se transformaron
paulatinamente las relaciones entre los estados nacionales de América
Latina y los pueblos indígenas, y se comenzaron a transformar los modos
de pensar, nombrar y actuar las adscripciones identitarias étnicas en
relación a estos grupos. Se trató de un proceso
político/cultural/económico que, en términos generales, se puede definir
como la tendencia a promover, al menos discursivamente, el respeto a
diferencias antes invisibilizadas. Y, como parte de ello, se fue
produciendo lentamente la conversión de los pueblos indígenas en sujetos
de derecho internacional.\footnote{Claudia Briones, ``Viviendo a la
  sombra de naciones sin sombra: poéticas y políticas de (auto)
  marcación de `lo indígena' en las disputas contemporáneas por el
  derecho a una educación intercultural'', en \emph{Interculturalidad y
  política. Desafíos y posibilidades}, ed. de Norma Fuller (Lima: Red
  para el Desarrollo de las Ciencias Sociales en el Perú, 2002),
  381-417.} Tal como afirma Claudia Briones, el logro de estos derechos
y legitimidades no ha sido fruto de concesiones estatales sino de arduas
luchas indígenas, como parte de un proceso etnopolítico de emergencia
indígena que tuvo lugar centralmente desde fines de los ochenta y
durante la década de los noventa.\footnote{José Bengoa, ``¿Una segunda
  etapa de la emergencia indígena en América Latina?'', \emph{Cuadernos
  de Antropología Social}, n.º 29 (2009): 7-22.} Este proceso, muy
complejo en relación a la diversidad de actores y factores que
confluyeron, como hito histórico estuvo vinculado a la conmemoración del
quinto centenario de la llegada de los colonizadores europeos a este
continente. En aquel momento, muchos grupos indígenas se negaron a
aceptar las conmemoraciones de esta fecha y lo transformaron en símbolo
de resistencia y reconstrucción de sus identidades étnicas.\footnote{Ídem.}
En ese escenario y como parte de esas disputas, comenzaron a crearse
desde mediados de los años ochenta diversas experiencias de medios de
comunicación gestionados por personas u organizaciones indígenas. Estos
espacios fueron surgiendo como herramientas al servicio de luchas en las
que la indigenidad de los sujetos se constituyó como el elemento
articulador y reivindicatorio central.\footnote{Magdalena Doyle,
  ``Debates and indigenous demands on rights to communication in Latin
  America'', \emph{Temas Antropológicos}, n.º 37 (2015): 89-118.} Un
ejemplo pionero de este uso de medios en el marco de luchas donde las
identificaciones éticas constituyeron una dimensión articuladora central
desde los años noventa, fue el del movimiento zapatista, que avanzaron
en el uso intensivo y amplio de la Internet para difundir su lucha a
nivel mundial y crear solidaridades en relación a sus causas.\footnote{Jorge
  Augurto y Jahve Mescco, ``La comunicación indígena como dinamizadora
  de la comunicación para el cambio social'' (ponencia, Asociación
  Latinoamericana de Investigadores de la Comunicación, 11 de mayo de
  2012).}

En el apartado siguiente veremos cómo, articulados con esa emergencia y
visibilización de los espacios de comunicación que adscriben a la
indigenidad, desde la segunda mitad de la década de los noventa comenzó
a crecer poco a poco el número de investigaciones dedicadas a estudiar,
desde diversas perspectivas, el uso de medios de comunicación por parte
de pueblos indígenas, con el foco puesto en reflexionar sobre la
especificidad de la noción de comunicación indígena, esto es, sobre las
marcas identitarias étnicas de esas experiencias.

Sin embargo, este periodo analizado constituye aún un momento bisagra
entre la etapa siguiente y la posterior: todavía el número de trabajos
es acotado y, en términos de matrices de pensamiento en que se
inscriben, la mayor parte de los textos sobre experiencias de medios
indígenas adscribe aún a los estudios sobre comunicación popular, tal
como ocurría en el periodo anterior. A la vez, comienzan a desarrollarse
investigaciones desde otras matrices, como los estudios sobre recepción.
Y, de modo transversal, en todos los trabajos se problematiza de forma
explícita la cuestión de la identidad étnica como dimensión cultural
clave para comprender estas prácticas de comunicación. También,
alrededor de la incipiente profesionalización del campo de la
comunicación en general,\footnote{Raúl Fuentes Navarro, \emph{Un campo
  cargado de futuro. El estudio de la comunicación en América Latina}
  (México: FELAFACS, 1991).} en este periodo se produce la primera tesis
de posgrado que hallamos sobre el estudio de prácticas de comunicación
mediatizadas atendiendo a rasgos culturales asociados a identidades
indígenas.\footnote{Nos referimos a la tesis de Maestría en Comunicación
  de Inés Cornejo Portugal. Accedimos a tres artículos académicos
  producidos a partir de dicha tesis que citamos párrafos más adelante.
  Inés Cornejo Portugal, ``La voz de la Mixteca y la comunidad receptora
  de la mixteca oaxaqueña'' (tesis de Maestría, Universidad
  Iberoamericana Ciudad de México, 1991).}

Respecto de los trabajos inscriptos en la matriz sobre comunicación
popular, en ellos se analizan sobre las posibilidades y modalidades de
expresión de estos pueblos haciendo foco en los vínculos entre las
condiciones socioeconómicas que estos comparten con otros sectores
oprimidos y las identificaciones étnicas.\footnote{Fernando Hernández,
  ``Video y cultura zapoteca'', \emph{Materiales para la comunicación
  popular}, n.º 9 (julio 1987): 12-17; Luis López, ``Cuatro propuestas
  desde la radio para que los `quiechuas' retomen el control de su
  propia cultura'', \emph{ALRED}, n.º 23 (septiembre 1997): 41-45; Kurt
  Hein, ``Baha'i community radio interacts with campesinos of Ecuador'',
  \emph{Media Development}, n.º 2 (1989): 39-41; Alan O'Connor,
  ``People's radio in Latin America, a new assessment'', \emph{Media
  Development} 36 (julio 1989): 47-53; Luis Ramiro Beltrán y Jaime
  Reyes, ``Radio popular en Bolivia: la lucha de obreros y campesinos
  para democratizar la comunicación'', \emph{Diálogos de la
  Comunicación,} n.º 35 (1993): 14-31; Cristina Romo, \emph{La otra
  radio. Voces débiles, voces de esperanza} (México: Fundación Manuel
  Buendía / Instituto Mexicano de la Radio, 1990).} En uno de esos
trabajos se afirma, por ejemplo, que el uso de las radios contribuía a
``recuperar la autoestima... de los quiechuahablantes difundiendo sus
valores, su cosmovisión y su tecnología, y demostrando {[}\emph{en el
espacio de lo público}{]} que esos indígenas tan denigrados, son
portadores de conocimientos que nos pueden ayudar a vivir mejor a
todos''.\footnote{López, ``Cuatro propuestas'', 42.}

De este modo, los textos inscriptos en esta matriz comparten con los
trabajos del periodo anterior un abordaje de los espacios y prácticas de
comunicación que hace énfasis en el vínculo entre esos medios y las
prácticas y los proyectos políticos de los pueblos, donde la liberación
de la palabra aparece como condición de ese proceso. Y, al mismo tiempo,
avanzan en una reflexión sobre los modos en que se producen los vínculos
entre las experiencias de comunicación mediática de los pueblos,
comunidades y organizaciones indígenas y los proyectos políticos
reivindicatorios vinculados a la etnicidad de los sujetos. En estos
casos, los medios de comunicación pertenecientes a los pueblos indígenas
son pensados en tanto ámbitos de liberación y legitimación pública de la
palabra, de organización política de las comunidades y de configuración
de aquellas redes que comienzan a cobrar centralidad para los
movimientos indígenas de América Latina en aquel periodo.

En términos metodológicos, algunos trabajos comparten con la etapa
anterior el objetivo de la descripción de experiencias de comunicación
particulares a partir de intervenciones de los investigadores en ellas,
así como la argumentación en torno a la centralidad de esos espacios
para favorecer la participación y expresión indígena.\footnote{Hernández,
  ``Video y cultura''; López, ``Cuatro propuestas''; Hein, ``Baha'i
  community''.}

Otros, en cambio, consisten en sistematizaciones de la diversidad de
experiencias de comunicación con participación indígena en una región,
buscando de algún modo comenzar a caracterizar el tipo de prácticas, sus
objetivos político-comunicacionales, sus trayectorias, sus audiencias,
entre otros.\footnote{Beltrán y Reyes, ``Radio popular en Bolivia'';
  Romo, \emph{La otra radio}.}

Simultáneamente, durante aquellos años cobró preponderancia en América
Latina otra matriz de investigación denominada ``estudios de
recepción'': aquella que, bajo la influencia de la hipótesis de ``los
usos y gratificaciones'', llevó adelante un desplazamiento teórico y
metodológico respecto de las matrices preponderantes en la década de los
setenta y ochenta. Tal como plantea Guillermo Sunkel, ``la preocupación,
(dominante en los 70) por el análisis ideológico del mensaje comienza a
ser desplazada por una temática radicalmente distinta: la recepción
crítica''.\footnote{Guillermo Sunkel, introducción a \emph{El consumo
  cultural en América Latina. Construcción teórica y líneas de
  investigación,} coord. de Guillermo Sunkel (Bogotá: Convenio Andrés
  Bello, 1999): 11-26.} El autor afirma que dicha noción de recepción
crítica viene a constituirse en una suerte de estrategia frente a lo que
se concibe como la poderosa influencia de la televisión.

Como parte de esa matriz, hacia principios de los años noventa se
realizaron algunos estudios de recepción que buscaron indagar ``el papel
que cumple la radio {[}\emph{indigenista}{]} en la población de su
cobertura y la respuesta de ésta frente a un medio de comunicación que
no surge de su propia cultura''.\footnote{Inés Cornejo Portugal, ``La
  voz de la Mixteca: diagnóstico y perspectivas'', en \emph{Radio
  regional y rural en México. Enlace de mil voces}, comp. de Jorge
  Martínez Lugo, Inés Cornejo Portugal y Etelvina Hernández Aguirre
  (México: Universidad Iberoamericana, 1992), 38.} Una referencia clave
de esa línea de investigación es la tesis de posgrado de Inés Cornejo
Portugal,\footnote{Inés Cornejo Portugal. ``Presencia de `La voz de la
  Mixteca' en la comunidad receptora de Tlaxiaco, Oaxaca'', en
  \emph{Radiodifusión en México. Historias. Programas. Audiencias},
  comp. de Francisco Aceves, Pablo Arredondo y Carlos Luna (México:
  Universidad de Guadalajara, 1991), 155-68; Cornejo Portugal, ``La voz
  de la Mixteca''.} quien realizó una encuesta exploratoria para
estudiar a la comunidad receptora de emisoras indigenistas de México a
partir de conocer los hábitos de la audiencia: el perfil del auditorio y
el ``uso que dan a la radio en términos de su propia identidad cultural,
su música, sus costumbres, y especialmente de su lengua, es decir, cómo
se concibe a la radio desde la cultura indígena''.\footnote{Cornejo
  Portugal, ``La voz de la Mixteca'', 38.} En términos generales, se
trata de trabajos producidos desde una concepción de las identidades
indígenas asumidas como una serie de rasgos culturales que determinarían
unos modos de apropiación de las tecnologías de la comunicación.

A modo de cierre de este apartado, es posible afirmar, como anunciamos
previamente, que este periodo comprendido entre fines de los años
ochenta y mediados de los noventa se constituye en una bisagra entre dos
tipos de trabajos. Por un lado, aquellas primeras reflexiones e
indagaciones del periodo anterior, formuladas desde la matriz de los
estudios sobre comunicación popular, que pensaban a la comunicación en
vínculo con los procesos políticos de liberación de la palabra de los
pueblos y que estaban, como dijimos, atravesados por la tensión entre la
preeminencia de la identidad de clase para pensar las prácticas
comunicacionales y la necesidad de comenzar a reflexionar ciertas
dimensiones de estas prácticas que no siempre podían adscribirse a
dichas identidades de clase. Por otro, se encuentran los textos
producidos a partir de fines de la década de los noventa, donde la
identidad étnica adquiere preeminencia para pensar a estos sujetos y sus
prácticas, al punto que conlleva en muchos casos la omisión de la
permanente interacción de estas adscripciones con otras que configuran
también la identidad de los sujetos, como la pertenencia a un
determinado sector socioeconómico.

\newpage

\hypertarget{tercer-periodo-nuevas-vertientes-y-tensiones-en-el-anlisis-de-la-relacin-entre-medios-polticas-e-identidades}{%
\section{Tercer periodo: nuevas vertientes y tensiones en el análisis
de\\\nnoindent la relación entre medios, políticas e
identidades}\label{tercer-periodo-nuevas-vertientes-y-tensiones-en-el-anlisis-de-la-relacin-entre-medios-polticas-e-identidades}}

En las últimas décadas asistimos a un notable incremento de los trabajos
que abordan a estas experiencias desde el campo de la comunicación, de
la mano con la creciente emergencia de medios de comunicación de pueblos
indígenas.\footnote{Doyle, ``Debates''.}

Además, una cuestión a mencionar es que muchos de estos trabajos se
realizaron en el marco de investigaciones de posgrado en distintos
países de América Latina y de equipos de investigación consolidados en
las universidades y con financiamiento nacional o internacional, lo cual
confirma la tendencia a la profesionalización del campo a la cual
referimos en el periodo anterior.

A su vez, veremos que las particularidades preponderantes de estos
trabajos nos remiten a nuevas matrices de pensamiento sobre
medios/tecnologías, políticas e identidades que fueron parte del campo
científico de la comunicación en el continente durante este periodo.

En términos de caracterización general de ese campo en América Latina,
Raúl Fuentes Navarro describe la década de los noventa e inicios del año
2000 como una época marcada por el ``abandono de las premisas críticas,
sea ante la adopción de la `inevitable vigencia' de las leyes del
mercado también en el ámbito de la investigación, sea ante la dispersión
de enfoques sobre las múltiples `mediaciones' culturales de las
prácticas sociales''.\footnote{Fuentes Navarro, ``La investigación de la
  comunicación'', 57.}

Predominan, durante ese periodo, las temáticas sobre globalización y las
tecnologías digitales, como las que se centran en identidades
microsociales. En las investigaciones latinoamericanas sobre
comunicación parece haberse perdido, señala el autor, la profundidad
ideológica, lo cual produjo la ``caída de la utopía'' según la cual la
comunicación debía contribuir a la democratización de las sociedades.

En diálogo con ello, identificamos que un elemento común de parte de los
trabajos que analizan la comunicación indígena en este periodo es que se
centran, principalmente, en abordar las transformaciones que los usos de
las tecnologías de la comunicación y la información generan en las
identidades, subjetividades políticas, culturas y condiciones de vida de
distintos grupos indígenas.

Sin embargo, más allá de esa caracterización muy genérica, en los
trabajos que son foco de este periodo identificamos dos vertientes
claramente diferenciadas:

\emph{a) Sobre brechas digitales y la promesa de la democratización
mediada tecnológicamente}. Una perspectiva extendida para el estudio de
las prácticas de comunicación masiva de los pueblos indígenas se
inscribe en la esperanza democratizadora que desde distintos ámbitos se
ha depositado durante esos años en la sociedad de la información,
asumiendo que los medios, en particular las tecnologías vinculadas a la
Internet, habilitarían nuevas modalidades de participación política, de
empoderamiento de los grupos indígenas.\footnote{Marck Becker y
  Guillermo Delgado, ``Latin America: The Internet and indigenous
  texts'', \emph{Cultural Survival} 21, n.º 4 (1997);
  Sami Ayriwa Pilco, ``La red de Internet y los pueblos indígenas de
  América Latina: experiencias y perspectivas'' (tesis de Maestría en
  Guion para Documental y Ficción, University of Bergen, 2000); Isabel
  Hernández y Silvia Calcagno, ``Los pueblos indígenas y la sociedad de
  la información en América Latina y el Caribe. Un marco para la
  acción'', informe de investigación producida para la CEPAL y el
  Instituto para la Conectividad en las Américas (Chile: CEPAL, 2003);
  Gloria Monasterios, ``Usos de Internet por organizaciones indígenas
  (OI) de Abya Yala: para una alternativa en políticas
  comunicacionales'', \emph{Revista Comunicación}, n.º 22 (segundo
  trimestre 2003): 60-69; Eduardo Sandoval Forero y Laura Mota Díaz,
  ``Indígenas y democracia en las tecnologías de la información y la
  comunicación (TICs)'' (ponencia, Universidad Nacional de Colombia,
  2007); Alejandra Aguilar Pinto, ``Identidade/diversidade cultural no
  ciberespaco: práticas informacionais e de inclusão digital nas
  comunidades indígenas, o caso dos Kariri Xocó e Pankararu no Brasil''
  (tesis doctoral, Universidade de Brasília, 2010); Jenny Arévalo
  Mosquera, ``Tejiendo en la red de Pueblos Indígenas y TIC: la
  presencia de la CONAIE en el ciberespacio'' (tesis de Maestría de
  Comunicación y Sociedad con mención en políticas públicas para
  Internet, FLACSO, 2010); Ramiro
  Catalán Pesce, ``Los desafíos de la inclusión digital étnica'',
  \emph{Diálogos de la Comunicación}, n.º 82 (sep.-dic. 2010): 1-7; Ivania
  Dos Santos Neves y Helen Alonso Monarcha, ``Inclusão digital indígena
  no Brasil: verdades e mentiras'', \emph{Diálogos de la Comunicación},
  n.º 86 (ene.-jul. 2013): 1-20.} Desde
ese lugar y mediante entrevistas y análisis de contenidos en sitios web
de pueblos indígenas, realizan abordajes en los cuales dan cuenta de las
desigualdades relacionadas al acceso y uso de las tecnologías de la
información y la comunicación (TIC) por parte de esos pueblos
---desigualdades que coinciden en llamar, con matices en su definición,
brecha digital---; describen casos de organizaciones o comunidades que
han utilizado esas tecnologías en sus procesos de lucha, y, comenzando
con esos casos que darían cuenta del potencial de las TIC cuando los
indígenas logran acceder a ellas, se proponen políticas públicas
tendientes a propiciar la participación de los pueblos indígenas de cada
región en la sociedad de la información.

También hallamos trabajos que se focalizan en dar cuenta de los modos de
uso de las tecnologías y las desigualdades que ello genera, así como las
condiciones legales para la reversión de esas desigualdades en el caso
de grupos indígenas. Encontramos, por un lado, abordajes que se centran
en la caracterización de la estructura de propiedad del sistema de
medios de un país y la participación de los indígenas en él\footnote{Raúl
  Borja, \emph{Comunicación social y pueblos indígenas del Ecuador}
  (Ecuador: Abya Yala, 1998); Javier Esteinou Madrid, ``Las etnias y el
  acceso a los medios de comunicación en México'', \emph{Interacción.
  Revista de comunicación educativa}, n.º 26 (2001);
  Javier Esteinou Madrid y Margarita Lorea Chávez y Peniche, ``La
  reforma del Estado y el acceso de los pueblos indios a los medios de
  comunicación'', \emph{Economía, Sociedad y Territorio}, n.º 12 (2002); Miguel
  Fuenmayor y Oscar José Antepaz, ``La comunicación radial intercultural
  bilingüe en el Zulia'' (ponencia, Asociación de Investigadores
  Venezolanos de la Comunicación, 2009).}
y que realizan un aporte clave al caracterizar parte de las
desigualdades expresivas que configuran los sistemas de medios en
distintos países de América Latina.

Particularmente, aquí encontramos trabajos motivados por analizar las
demandas de medios de comunicación propios planteadas por el Ejército
Zapatista de Liberación Nacional (EZLN) desde su primer alzamiento en
1994 y los acuerdos a los que el EZLN llegó con el Estado mexicano en
esta materia.\textsuperscript{53}
Estas publicaciones indagan las razones por las cuales aquellas demandas
y acuerdos no se plasmaron en efectivas políticas públicas que
garantizaran el acceso de los pueblos indígenas a medios masivos de
comunicación propios.\textsuperscript{54}

Motivados en gran medida por este mismo movimiento indígena, encontramos
también investigaciones que aluden a los medios (sobre todo a la
Internet) como herramientas de difusión de mensajes al servicio de las
luchas de estos pueblos, específicamente como plataformas para
transmitir mensajes que tendrían un impacto en la opinión
pública.\textsuperscript{55}
Estos trabajos se desarrollan a partir de la caracterización y los
análisis de contenidos en sitios web de pueblos indígenas.

\emph{b) Políticas de identidad, desigualdades expresivas y luchas por
el espacio público}. Finalmente, desde inicios y sobre todo mediados de
la segunda década del siglo XXI, enmarcados en un enfoque cultural y
político sobre los medios y en el interés por indagar los procesos de
configuración de identificaciones en el espacio público mediatizado, se\marginnote{\textsuperscript{53} Por ejemplo: Esteinou Madrid, ``Las etnias'' y
  Esteinou Madrid y Loera Chávez y Peniche, ``La reforma del Estado''.}\marginnote{\textsuperscript{54} Dado el recorte temático de este
  artículo, focalizado en analizar investigaciones que hayan estudiado
  el uso de medios por parte de pueblos indígenas, no incorporamos en el
  corpus trabajos que hayan abordado las representaciones sobre pueblos
  indígenas construidas por medios comerciales. Sin embargo, respecto
  del EZLN hay diversos trabajos que abordaron esa temática: para un
  análisis al respecto, puede verse por ejemplo el libro \emph{Chiapas,
  la comunicación enmascarada}, escrito por Raúl Trejo Delarbre, donde
  el autor analiza, esencialmente, el tratamiento que la radio, la
  televisión y la prensa realizaron del levantamiento zapatista de 1994.
  Raúl Trejo Delarbre, \emph{Chiapas. La comunicación enmascarada: los
  medios y el pasamontañas} (México: Diana, 1994).}\marginnote{\textsuperscript{55}\setcounter{footnote}{55} Markus Itturriaga, ``The War of Ink and Internet'',
  \emph{Chancellor's Honors Program Projects} (Knoxville: University of
  Tennessee, 1996); Anne C.
  Barnhart-Park, ``alt.Indigenous.electronic-(re)sources'' (ponencia,
  Latin American Studies Association, 16 al 18 de marzo de 2000);
  Fernando Camacho Padilla, ``La cibermovilización como estrategia de
  resistencia: el caso de los pueblos originarios'', \emph{Revista
  Chilena de Antropología Visual}, n.º 5 (2005): 28-43.}
produjeron trabajos que aportan pautas teóricas y metodológicas
centrales para el abordaje de este objeto.\footnote{Nos referimos, por
  ejemplo, a los siguientes trabajos: Oscar Espinosa, ``Los pueblos
  indígenas de la Amazonía peruana y el uso político de los medios de
  comunicación'', \emph{América Latina Hoy. Revista de Ciencias
  Sociales}, n.º 19 (1998): 91-100; José
  Ramos Rodríguez, ``Los programas de avisos en las radiodifusoras
  indigenistas de México: espacios de reproducción de la etnicidad''
  (ponencia, Asociación Latinoamericana de Investigadores de la
  Comunicación\emph{,} Bolivia, 12 al 15 de junio de 2002); José Ramos
  Rodríguez, ``Inequidad mediática y multiculturalidad: alcances y
  límites de la participación indígena en la radio oficial del estado de
  Puebla, México'' (ponencia, Latin American Studies Association, Río de
  Janeiro, 11 a 14 de junio de 2009); Juan Salazar, ``Prácticas de
  auto-representación y los dilemas de la autodeterminación: el cara y
  sello de los derechos a la comunicación Mapuche'', en
  \emph{Aproximaciones a la cuestión mapuche en Chile, una mirada}}
Esta vertiente de investigación sobre la comunicación indígena se va
consolidando como la más extendida durante la última década en América
Latina.

Muchos de estos textos se inscriben en el consenso que adquirieron, en
la antropología y en las ciencias sociales, las teorías performativas de
la identidad (desde las cuales se asume el carácter descentrado de los
sujetos y se piensan las identidades en tanto fragmentadas, flexibles y
disputadas). En ese marco, cobran centralidad nociones como
``negociación de identidades'' o ``políticas de identidad'', ya que los
autores coinciden en afirmar que los medios se constituyen en espacios
fundamentales de negociación en las disputas político-identitarias
libradas por los pueblos indígenas.

De este modo, a la vez que reconocen y analizan extensamente los
contextos de desigualdades expresivas en que los medios hegemónicos han
sumido a los indígenas, se abordan ---desde enfoques epistemológicos
vinculados a las diversas visiones de los estudios decoloniales y las
epistemologías del sur; abordajes teóricos transdisciplinares, y
propuestas metodológicas que priorizan la perspectiva de los actores e
incluso la producción colaborativa de conocimiento, recuperando los
sentidos y narraciones que cada experiencia construye sobre sí misma---
las prácticas de comunicación mediática que estos pueblos desarrollan
para disputar sentidos en torno a la indigeneidad. Entendemos que este
tipo de abordaje se inscribe en la preeminencia de la perspectiva
decolonial surgida en la década de los noventa, a la cual alude Erick R.
Torrico Villanueva, y que según el autor ha abierto una ruta intelectual
que en América Latina viene removiendo los cimientos del conocimiento
establecido para reinterpretarlos desde un lugar de enunciación
distinto:

\begin{quote}
Se trata, pues, de un pensamiento crítico remozado y propositivo que no
solo extiende los objetivos de reconocimiento, justicia y
democratización a los planos epistemológico y teórico, además del
político, sino que a la vez propone una reconsideración de las
historias, los saberes y los haceres situados junto a una
reontologización de sus protagonistas marginalizados.\textsuperscript{57}
\end{quote}

Desde esos posicionamientos epistémicos, los trabajos a los que
referimos en este apartado ponen en cuestión concepciones instrumentales
en relación a los medios y, en términos generales, los modos de pensar
los vínculos entre la mediatización y la política, planteando que ``el
espacio mediático y la apropiación de herramientas de comunicación son
algunos de los ámbitos en los que los pueblos indígenas están ejerciendo
un papel activo, renovador y de redefinición de lo que son, han sido y
quieren ser, planteando nuevos proyectos\newpage \noindent civilizatorios\marginnote{\emph{desde
  la historia y las ciencias sociales}, ed. de Claudio Barrientos
  (Santiago de Chile: RIL Editores, 2014), 145-65; Claudia Magallanes
  Blanco et al., ``Memoria e imaginarios en el discurso mediático
  indígena: producciones radiofónicas de Oaxaca'', \emph{Realis} 3, n.º
  2 (2013): 156-77;
  Claudia Magallanes Blanco y José Ramos Rodríguez, coords.,
  \emph{Miradas propias. Pueblos indígenas, comunicación y medios en la
  sociedad global} (México: Universidad Iberoamericana Puebla, 2016);
  Antoni Castells I Talens, ``¿Ni indígena ni comunitaria? La radio
  indigenista en tiempos neoindigenistas'', \emph{Comunicación y
  Sociedad}, n.º 15 (junio 2011): 123-42;
  Antoni Castells I Talens, ``Radio y nacionalismo iconográfico en
  México: la negociación discursiva de una identidad maya'', \emph{Signo
  y Pensamiento} 27, n.º 53 (jul.-dic. 2008): 230-45;
  Liliana Lizondo, ``Comunicación con identidad o comunicación
  comunitaria. El caso de la FM `La voz indígena'\,'' (tesis de Maestría
  en Planificación y Gestión de Procesos Comunicacionales, Universidad
  Nacional de La Plata, 2015); Elena Nava Morales, ``Radio Totopo y
  comunalidad: una experiencia de comunicación indígena en Oaxaca'', en
  \emph{Miradas propias. Pueblos indígenas, comunicación y medios en la
  sociedad global}, coord. de Claudia Magallanes Blanco y José Ramos
  Rodríguez (México: Universidad Iberoamericana Puebla, 2016), 215-32.}\marginnote{\textsuperscript{57}\setcounter{footnote}{75} Erick R.
  Torrico Villanueva, ``Comunicación organizacional y decolonialidad:
  desafíos para una intersección factible'', \emph{Organicom}, n.º 37
  (sep.-dic. 2021): 16.} que son negados
por la civilización eurocéntrica y moderna''.\footnote{Magallanes Blanco
  et al., ``Memoria e imaginarios'', 161.}

Entonces, en términos generales, estos trabajos problematizan dos
dimensiones clave de estas prácticas y experiencias de comunicación
indígena: por un lado, qué sentidos sobre las identidades étnicas se
construyen desde esos espacios de comunicación indígena y con qué otros
actores se disputan esos sentidos y, por otro, abordan la complejidad de
las relaciones entre el uso de medios y las luchas indígenas, dando
cuenta de que esa relación no refiere solo a luchas que se dan
\emph{desde} el espacio público mediatizado sino también \emph{por} ese
espacio en términos de derechos a la comunicación.

\hypertarget{a-modo-de-cierre}{%
\section{A modo de cierre}\label{a-modo-de-cierre}}

Las trayectorias de investigación sobre la temática que nos ocupa han
ido transformándose, en diálogo con el creciente desarrollo y
visibilización de las experiencias de comunicación indígena y los
procesos etnopolíticos en que ese desarrollo va teniendo lugar, y
también alrededor de las (re)configuraciones de las matrices de
pensamiento que se han desarrollado en el campo científico de la
comunicación en América Latina durante los periodos analizados.

En ese sentido, es posible notar cómo fueron transformándose los modos
de nombrar tanto a los sujetos de estas prácticas de comunicación
---desde un primer momento en que primaron identidades de clase hasta
los últimos periodos en que se abordan las identidades étnicas como
tramas que configuran sentidos y prácticas en torno a los medios y
tecnologías---; como los modos de analizar a los medios y las
tecnologías en el marco de los procesos políticos identitarios que
protagonizan esos sujetos ---desde concepciones difusionistas e
instrumentalistas a entenderlos como espacios que tienen inscriptas
condiciones desiguales de expresión, pero que a la vez constituyen
ámbitos clave para la transformación de esas condiciones---.

Durante los últimos años es posible notar, en las trayectorias
investigativas sobre esta temática, transformaciones teóricas y
metodológicas donde prima la transdisciplinariedad y los enfoques
decoloniales como opción para el estudio cada vez más complejo de este
fenómeno.

Tal como anticipamos en la introducción de este texto, el análisis que
aquí propusimos se fundamenta en el convencimiento de que la comprensión
de la cultura contemporánea requiere articular sistemáticamente el
análisis de los procesos históricos donde emergen y se desarrollan las
técnicas y prácticas comunicativas, con el análisis de las nociones
provenientes de las áreas del conocimiento que posibilitaron dicha
emergencia.

En ese sentido esperamos haber contribuido al análisis del tejido de
significados y referencias de que está hecha la trayectoria conceptual
sobre la comunicación indígena, asumiendo que ``poner en historia los
términos en que se formulan los debates es ya una forma de acceso a los
combates, a los conflictos y luchas que atraviesan los discursos y las
cosas''.\footnote{Jesús Martín-Barbero, \emph{De los medios a las
  mediaciones. Comunicación, cultura y hegemonía} (Barcelona: Gustavo
  Gili, 1987): 32.}







\section{Bibliography}\label{bibliography}

\begin{hangparas}{.25in}{1} 



Aguilar Pinto, Alejandra. ``Identidade/diversidade cultural no
ciberespaco: práticas informacionais e de inclusão digital nas
comunidades indígenas, o caso dos Kariri Xocó e Pankararu no Brasil''.
Tesis doctoral, Universidade de Brasília, 2010.

Albó, Xavier. ``Idiomas, escuelas y radios en Bolivia''. \emph{Chasqui.
Revista Latinoamericana de Comunicación}, n.º 6 (1974): 92-130.

Albó, Xavier y Néstor Quiroga. ``La radio como expresión libre del
Aymara''. \emph{Chasqui. Revista Latinoamericana de Comunicación}, n.º 7
(1974): 109-26.

Arévalo Mosquera, Jenny. ``Tejiendo en la red de Pueblos Indígenas y
TIC: la presencia de la CONAIE en el ciberespacio''. Tesis de Maestría
de Comunicación y Sociedad con mención en políticas públicas para
Internet, FLACSO, 2010. Acceso el 29 de junio de 2021,
\url{https://repositorio.flacsoandes.edu.ec/handle/10469/2484}.

Argumedo, Alcira. ``Las matrices del pensamiento teórico-político''.
\emph{Antología del pensamiento crítico argentino contemporáneo},
coordinado por Sergio Caggiano y Alejandro Grimson, 129-56. Buenos
Aires: CLACSO, agosto de 2015.

Argumedo, Alcira. \emph{Los silencios y las voces en América Latina.
Notas sobre el pensamiento nacional y popular.} Buenos Aires: Ediciones
del Pensamiento Nacional, 2004.

Augurto, Jorge y Jahve Messco. ``La comunicación indígena como
dinamizadora de la comunicación para el cambio social''. Ponencia
presentada en el XI Congreso Latinoamericano de Investigadores de la
Comunicación Social de la Asociación Latinoamericana de Investigadores
de la Comunicación. Montevideo, mayo 2012. \href{http://servindi.org/pdf/ALAIC\_comunicaci\%C3\%B3nindigena2012.pdf}{http://servindi.org/pdf/ALAIC\_comunicaci\%C3\%B3nindigena2012.pdf}.

Ayriwa Pilco, Sami. ``La red de Internet y los pueblos indígenas de
América Latina: experiencias y perspectivas''. Tesis de Maestría en
Guion para Documental y Ficción, University of Bergen, 2000.

Barnhart-Park, Anne C. ``alt.Indigenous.electronic-(re)sources''.
Ponencia presentada en el Annual Meeting of the Latin American Studies
Association. Miami, 16 al 18 de marzo de 2000.

Bartolomé, Miguel. ``Pluralismo cultural y redefinición del Estado en
México''. \emph{Série Antropologia}, n.° 210 (1996): 1-16.

Becker, Marck y Guillermo Delgado. ``Latin America: The Internet and
indigenous texts''. \emph{Cultural Survival} 21, n.º 4 (1997). Acceso el
25 de junio de 2021, \href{https://www.culturalsurvival.org/publications/cultural-survival-quarterly/latin-america-internet-and-indigenous-texts}{https://www.culturalsurvival.org/publications/cultural-survival-quarterly/latin-america-internet-and-indigenous-texts}.

Bello, Andres. \emph{Etnicidad y ciudadanía en América Latina. La acción
colectiva de los pueblos indígenas.} Chile: CEPAL / Sociedad Alemana de
Cooperación Técnica, 2004.

Beltrán, Luis Ramiro y Jaime Reyes. ``Radio popular en Bolivia: la lucha
de obreros y campesinos para democratizar la comunicación''.
\emph{Diálogos de la Comunicación,} n.º 35 (1992): 14-31.

Bengoa, José. ``¿Una segunda etapa de la emergencia indígena en América
Latina?''. \emph{Cuadernos de Antropología Social,} n.º 29 (2009): 7-22.

Borja, Raúl. \emph{Comunicación social y pueblos indígenas del Ecuador}.
Ecuador: Abya Yala, 1998.

Briones, Claudia. \emph{La alteridad del ``cuarto mundo''. Una
deconstrucción antropológica de la diferencia}. Buenos Aires: Ediciones
del Sol, 1998.

Briones, Claudia. ``Viviendo a la sombra de naciones sin sombra:
poéticas y políticas de (auto)marcación de `lo indígena' en las disputas
contemporáneas por el derecho a una educación intercultural''. En
\emph{Interculturalidad y política. Desafíos y posibilidades}, editado
por Norma Fuller, 381-417. Lima: Red para el Desarrollo de las Ciencias
Sociales en el Perú, 2002.

Caletti, Sergio. \emph{Comunicación, política y espacio público. Notas
para repensar la democracia en la sociedad contemporánea. Borradores de
trabajo}. Buenos Aires: UBA, 1998-2002.

Camacho Padilla, Fernando. ``La cibermovilización como estrategia de
resistencia: el caso de los pueblos originarios''. \emph{Revista Chilena
de Antropología Visual}, n.º 5 (2005): 28-43. Acceso el 23 de junio de
2021, \url{http://www.rchav.cl/2005_5_art03_camacho.html}.

Castells I Talens, Antoni. ``¿Ni indígena ni comunitaria? La radio
indigenista en tiempos neoindigenistas''. \emph{Comunicación y
Sociedad}, n.º 15 (junio 2011): 123-42. Acceso el 29 de junio de 2021,
\url{http://www.scielo.org.mx/scielo.php?script=sci_arttext\&pid=S0188-252X2011000100006}.

Castells I Talens, Antoni. ``Radio y nacionalismo iconográfico en
México: la negociación discursiva de una identidad maya''. \emph{Signo y
Pensamiento} 27, n.º 53 (julio-diciembre 2008): 230-45. Acceso el 29 de
junio de 2021,
\url{http://revistas.javeriana.edu.co/index.php/signoypensamiento/article/view/4566/3531}.

Catalán Pesce, Ramiro. ``Los desafíos de la inclusión digital étnica''.
\emph{Diálogos de la Comunicación}, n.º 82 (septiembre-diciembre 2010):
1-7. Acceso el 26 de junio de 2021,
\url{https://dialnet.unirioja.es/servlet/articulo?codigo=3728256}.

Chesterfield, Ray y Kenneth Renddle. ``El uso de los canales indígenas
de comunicación en el desarrollo rural venezolano''. \emph{Chasqui.
Revista Latinoamericana de Comunicación}, n.º 12 (1976): 11-18.

Cornejo Portugal, Inés. ``La voz de la Mixteca: diagnóstico y
perspectivas''. En \emph{Radio regional y rural en México. Enlace de mil
voces}, compilado por Jorge Martínez Lugo, Inés Cornejo Portugal y
Etelvina Hernández Aguirre. México: Universidad Iberoamericana, 1992:
37-64.

Cornejo Portugal, Inés. ``Presencia de `La voz de la Mixteca' en la
comunidad receptora de Tlaxiaco, Oaxaca''. En \emph{Radiodifusión en
México. Historias. Programas. Audiencias}, compilado por Francisco
Aceves, Pablo Arredondo y Carlos Luna, 155-68. México: Universidad de
Guadalajara, 1991.

Díaz Polanco, Héctor. ``Derechos indígenas en la actualidad''.
\emph{Boletín de Antropología Americana}, n.º 33 (1998): 91-99.
\url{https://www.jstor.org/stable/40978130}.

Dos Santos Neves, Ivania y Helen Alonso Monarcha. ``Inclusão digital
indígena no Brasil: verdades e mentiras''. \emph{Diálogos de la
Comunicación}, n.º 86 (enero-julio 2013): 1-20. Acceso el 26 de junio de
2021, \url{https://dialnet.unirioja.es/servlet/articulo?codigo=6845034}.

Doyle, Magdalena. ``Debates and indigenous demands on rights to
communication in Latin America''. \emph{Temas Antropológicos,} n.º 37
(2015): 89-118. \url{https://dialnet.unirioja.es/servlet/articulo?codigo=5267277}.

Doyle, Magdalena. ``El derecho a la comunicación de los pueblos
originarios. Límites y posibilidades de las reivindicaciones indígenas
en relación al sistema de medios de comunicación en Argentina''. Tesis
doctoral en Antropología, Universidad de Buenos Aires, 2017.

Doyle, Magdalena. ``Los medios masivos de comunicación en las luchas de
los pueblos indígenas. Abordajes desde los estudios sobre comunicación
en América Latina''. Tesis de Maestría en Comunicación y Cultura
Contemporánea, Universidad Nacional de Córdoba, 2013.

Encinas Valverde, Orlando. ``La radio al servicio de la liberación
indígena: Radio Mezquital. La radio como instrumento de apoyo en
programas de desarrollo integral''. \emph{Nueva Sociedad}, n.º 25
(julio-agosto 1976): 85-94.

Espinosa, Oscar. ``Los pueblos indígenas de la Amazonía peruana y el uso
político de los medios de comunicación''. \emph{América Latina Hoy.
Revista de Ciencias Sociales}, n.º 19 (1998): 91-100. Acceso el 29 de
junio de 2021,
\url{https://revistas.usal.es/index.php/1130-2887/article/view/2258}.

Esteinou Madrid, Javier. ``Las etnias y el acceso a los medios de
comunicación en México''. \emph{Interacción. Revista de comunicación
educativa}, n.º 26 (2001). Acceso el 28 de junio de 2021,
\url{http://interaccion.cedal.org.co/26/lasetnias.htm}

Esteinou Madrid, Javier y Margarita Lorea Chávez y Peniche. ``La reforma
del Estado y el acceso de los pueblos indios a los medios de
comunicación''. \emph{Economía, Sociedad y Territorio}, n.º 12 (2002).
Acceso el 4 de mayo de 2022, \url{https://est.cmq.edu.mx/index.php/est/article/view/340}.

Ford, Aníbal. \emph{Navegaciones. Comunicación, cultura y crisis}.
Buenos Aires: Amorrortu, 1994.

Fuenmayor, Miguel y Oscar José Antepaz. ``La comunicación radial
intercultural bilingüe en el Zulia''. Ponencia presentada en el II
Congreso Invecom de la Asociación de Investigadores Venezolanos de la
Comunicación. Venezuela, 2009. Acceso el 27 de junio de 2021,
\url{http://congresoinvecom.org/index.php/invecom2009/invecom2009/paper/view/46}.

Fuentes Navarro, Raúl. ``Communication Research in Latin America: Will
the ``Nocturnal Map'' Survive or Fade Away?''. \emph{History of Media
Studies} 1 (2021): 1-7. https://doi.org/10.32376/d895a0ea.a5f7f735.

Fuentes Navarro, Raúl. ``La investigación de la comunicación en América
Latina: condiciones y perspectivas para el siglo XXI''. \emph{Oficios
Terrestres}, n.º 6 (1999): 56-67.
\url{http://sedici.unlp.edu.ar/handle/10915/38040}.

Fuentes Navarro, Raúl. \emph{Un campo cargado de futuro. El estudio de
la comunicación en América Latina}. México: FELAFACS, 1991.

Gumucio Dagron, Alfonso. ``El papel político de las radios mineras''.
\emph{Comunicación y Cultura,} n.º 8 (julio 1982): 89-100.

Gumucio Dagron, Alfonso. \emph{Haciendo olas: historias de la
comunicación participativa para el cambio social.} Nueva York:
Rockefeller Foundation, 2001.

Hasen, Felipe. ``Lago Neltume: la filmación inmersa en un conflicto
ambiental como herramienta de resistencia''. \emph{Revista Chilena de
Antropología Visual}, n.º 17 (junio 2011): 146-62.
\url{http://www.rchav.cl/2011_17_etn02_hasen.html}.

Hein, Kurt. ``Baha'i community radio interacts with campesinos of
Ecuador''. \emph{Media Development}, n.º 2 (1989): 39-41.

Hernández, Fernando. ``Video y cultura zapoteca''. \emph{Materiales para
la comunicación popular}, n.º 9 (julio 1987): 12-17.

Hernández, Isabel y Silvia Calcagno. ``Los pueblos indígenas y la
sociedad de la información en América Latina y el Caribe. Un marco para
la acción''. Informe de investigación producida para la CEPAL y el
Instituto para la Conectividad en las Américas. Chile: CEPAL, 2003.
Acceso el 29 de junio de 2021,
\url{http://lanic.utexas.edu/project/etext/llilas/claspo/workingpapers/indigenas.pdf}.

Itturriaga, Markus. ``The War of Ink and Internet''. \emph{Chancellor's
Honors Program Projects}. Knoxville: University of Tennessee, 1996.
Acceso el 26 de junio de 2021, \url{http://trace.tennessee.edu/utk\_chanhonoproj/167}.


Lizondo, Liliana. ``Comunicación con identidad o comunicación
comunitaria. El caso de la FM `La voz indígena'\,''. Tesis de Maestría
en Planificación y Gestión de Procesos Comunicacionales, Universidad
Nacional de La Plata, 2015.

López, Luis. ``Cuatro propuestas desde la radio para que los `quiechuas'
retomen el control de su propia cultura''. \emph{ALRED}, n.º 23
(septiembre 1997): 41-45.

Magallanes Blanco, Claudia y José Ramos Rodríguez, coords. \emph{Miradas
propias. Pueblos indígenas, comunicación y medios en la sociedad
global.} México: Universidad Iberoamericana Puebla, 2016.

Magallanes Blanco, Claudia et al. ``Memoria e imaginarios en el discurso
mediático indígena: producciones radiofónicas de Oaxaca''. \emph{Realis}
3, n.º 2 (2013): 156-77. Acceso el 29 de junio de 2021,
\url{https://periodicos.ufpe.br/revistas/realis/article/download/8796/8771}.

Martín-Barbero, Jesús. \emph{De los medios a las mediaciones.
Comunicación, cultura y hegemonía.} Barcelona: Gustavo Gili, 1987.

Martín-Barbero, Jesús. \emph{Oficio del cartógrafo. Travesías
latinoamericanas de la comunicación en la cultura}. Chile: Fondo de
Cultura Económica, 2002.

Mata, María Cristina. ``Comunicación popular: continuidades,
transformaciones y desafíos''. \emph{Oficios Terrestres}, n.º 26 (abril
2011): 1-22.

Monasterios, Gloria. ``Usos de Internet por organizaciones indígenas
(OI) de Abya Yala: para una alternativa en políticas comunicacionales''.
\emph{Revista Comunicación}, n.º 22 (2003): 60-69.

Munizaga, Giselle y Anny Rivera. \emph{La investigación en comunicación
social en Chile}. Lima: Desc, 1983.

Nava Morales, Elena. ``Radio Totopo y comunalidad: una experiencia de
comunicación indígena en Oaxaca''. En \emph{Miradas propias. Pueblos
indígenas, comunicación y medios en la sociedad global}, coordinado por
Claudia Magallanes Blanco y José Ramos Rodríguez, 215-32. México:
Universidad Iberoamericana Puebla, 2016.

Nolasco Armas, Margarita. ``Educación y medios de comunicación masiva''.
\emph{Chasqui. Revista Latinoamericana de Comunicación}, n.º 5 (1974):
25-38.

O'Connor, Alan. ``People's radio in Latin America, a new assessment''.
\emph{Media Development} 36 (julio 1989): 47-53.

O'Sullivan, Jerry. ``Informe de la investigación exploratoria sobre
nuevas posibilidades del radio en la Tarahumara''. \emph{Chasqui.
Revista Latinoamericana de Comunicación}, n.º 9 (1975): 25-67.

Oseguera, Antonio. ``Una experiencia de comunicación educativa para el
desarrollo rural''. \emph{Comunicación y Cultura,} n.º 8 (julio 1982):
33-38.

Ramos Rodríguez, José. ``Inequidad mediática y multiculturalidad:
alcances y límites de la participación indígena en la radio oficial del
estado de Puebla, México''. Ponencia presentada en el XXVIII
International Congress de la Latin American Studies Association. Río de
Janeiro, 11 a 14 de junio de 2009.

Ramos Rodríguez, José. ``Los programas de avisos en las radiodifusoras
indigenistas de México: espacios de reproducción de la etnicidad''.
Ponencia presentada en el VI Congreso de la Asociación Latinoamericana
de Investigadores de la Comunicación. Bolivia, 12 al 15 de junio de
2002.

Romo, Cristina. \emph{La otra radio. Voces débiles, voces de esperanza}.
México: Fundación Manuel Buendía / Instituto Mexicano de la Radio, 1990.

Salazar, Juan. ``Prácticas de auto-representación y los dilemas de la
autodeterminación: el cara y sello de los derechos a la comunicación
Mapuche''. En \emph{Aproximaciones a la cuestión mapuche en Chile, una
mirada desde la historia y las ciencias sociales}, editado por Claudio
Barrientos, 145-65. Santiago de Chile: RIL Editores, 2014.

Sandoval Forero, Eduardo y Laura Mota Díaz. ``Indígenas y democracia en
las tecnologías de la información y la comunicación (TICs)''. Ponencia
presentada en las VI Jornadas latinoamericanas de estudios sociales de
la ciencia y la tecnología de la Universidad Nacional de Colombia.
Bogotá, 2007.

Sanguinetti, Luciano. \emph{Comunicación y medios. Claves para pensar y
enseñar una teoría latinoamericana sobre comunicación}. La Plata:
Universidad Nacional de La Plata, 2001.

Santos Velasco, Francisco y Francisco Reyes Ruiz. \emph{Una experiencia
de comunicación educativa a través de la radio en la zona lacustre de
Pátzcuaro}. Ecuador: CIESPAL / CREFAL, 1983.

Stavenhagen, Rodolfo. \emph{Los pueblos indígenas y sus derechos.
Informes Temáticos del Relator Especial sobre la situación de los
Derechos Humanos y las Libertades Fundamentales de los Pueblos Indígenas
del Consejo de Derechos Humanos de la Organización de las Naciones
Unidas.} México: UNESCO, 2007.

Sunkel, Guillermo, coord. \emph{El consumo cultural en América Latina.
Construcción teórica y líneas de investigación}. Bogotá: Convenio Andrés
Bello, 1999.

Torrico Villanueva, Erick R. ``Comunicación organizacional y
decolonialidad: desafíos para una intersección factible''.
\emph{Organicom}, n.º 37 (septiembre-diciembre 2021): 14-22.

Trejo Delarbre, Raúl. \emph{Chiapas. La comunicación enmascarada: los
medios y el pasamontañas}. México: Diana, 1994.

Vassallo de Lopes, Maria Immacolata. ``El campo de la comunicación:
sobre su estatuto disciplinar''. \emph{Oficios Terrestres},
n.\textsuperscript{o} 7 (2000): 74-83.
\url{http://sedici.unlp.edu.ar/handle/10915/47380}.

White, Robert. ``La teoría de la comunicación en América Latina. Una
visión europea de sus contribuciones''. \emph{Telos}
n.\textsuperscript{o} 19 (1989): 43-54.



\end{hangparas}


\end{document}