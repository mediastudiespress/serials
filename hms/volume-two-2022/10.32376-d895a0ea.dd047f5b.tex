% see the original template for more detail about bibliography, tables, etc: https://www.overleaf.com/latex/templates/handout-design-inspired-by-edward-tufte/dtsbhhkvghzz

\documentclass{tufte-handout}

%\geometry{showframe}% for debugging purposes -- displays the margins

\usepackage{amsmath}

\usepackage{tabu}
\usepackage{longtable}

\usepackage{hyperref}


\usepackage{amssymb}
\DeclareUnicodeCharacter{2264}{\leq}

\usepackage{fancyhdr}

\usepackage{hanging}

\hypersetup{colorlinks=true,allcolors=[RGB]{97,15,11}}

\fancyfoot[L]{\emph{History of Media Studies}, vol. 2, 2022}


% Set up the images/graphics package
\usepackage{graphicx}
\setkeys{Gin}{width=\linewidth,totalheight=\textheight,keepaspectratio}
\graphicspath{{graphics/}}

\title[Inequality]{Inequality: The Blind Spot of Western Communication Studies} % longtitle shouldn't be necessary

% The following package makes prettier tables.  We're all about the bling!
\usepackage{booktabs}

% The units package provides nice, non-stacked fractions and better spacing
% for units.
\usepackage{units}

% The fancyvrb package lets us customize the formatting of verbatim
% environments.  We use a slightly smaller font.
\usepackage{fancyvrb}
\fvset{fontsize=\normalsize}

% Small sections of multiple columns
\usepackage{multicol}

% Provides paragraphs of dummy text
\usepackage{lipsum}

% These commands are used to pretty-print LaTeX commands
\newcommand{\doccmd}[1]{\texttt{\textbackslash#1}}% command name -- adds backslash automatically
\newcommand{\docopt}[1]{\ensuremath{\langle}\textrm{\textit{#1}}\ensuremath{\rangle}}% optional command argument
\newcommand{\docarg}[1]{\textrm{\textit{#1}}}% (required) command argument
\newenvironment{docspec}{\begin{quote}\noindent}{\end{quote}}% command specification environment
\newcommand{\docenv}[1]{\textsf{#1}}% environment name
\newcommand{\docpkg}[1]{\texttt{#1}}% package name
\newcommand{\doccls}[1]{\texttt{#1}}% document class name
\newcommand{\docclsopt}[1]{\texttt{#1}}% document class option name


\begin{document}

\begin{titlepage}

\begin{fullwidth}
\noindent\LARGE\emph{Exclusions in the History of Media Studies
} \hspace{25mm}\includegraphics[height=1cm]{logo3.png}\\
\noindent\hrulefill\\
\vspace*{1em}
\noindent{\Huge{Inequality: The Blind Spot of Western\\\noindent Communication Studies\par}

\vspace*{1.5em}

\noindent\LARGE{Boris Mance} \href{https://orcid.org/0000-0002-8864-1683}{\includegraphics[height=0.5cm]{orcid.png}}\par}\marginnote{\emph{Boris Mance and Sašo Slaček Brlek, ``Inequality: The Blind Spot of Western Communication Studies,'' \emph{History of Media Studies} 2 (2022), \href{https://doi.org/10.32376/d895a0ea.dd047f5b}{https://doi.org/ 10.32376/d895a0ea.dd047f5b}.} \vspace*{0.75em}}
\vspace*{0.5em}
\noindent{{\large\emph{University of Ljubljana}, \href{mailto:boris.mance@fdv.uni-lj.si}{boris.mance@fdv.uni-lj.si}\par}} \marginnote{\href{https://creativecommons.org/licenses/by-nc/4.0/}{\includegraphics[height=0.5cm]{by-nc.png}}}

\vspace*{0.75em} 

\noindent{\LARGE{Sašo Slaček Brlek} \href{https://orcid.org/0000-0001-6270-3092}{\includegraphics[height=0.5cm]{orcid.png}}\par}
\vspace*{0.5em}
\noindent{{\large\emph{University of Ljubljana}, \href{mailto:saso.brlek-slacek@fdv.uni-lj.si}{saso.brlek-slacek@fdv.uni-lj.si}\par}}

% \vspace*{0.75em} % third author

% \noindent{\LARGE{<<author 3 name>>}\par}
\vspace*{0.5em}
% \noindent{{\large\emph{<<author 3 affiliation>>}, \href{mailto:<<author 3 email>>}{<<author 3 email>>}\par}}

\end{fullwidth}



\hypertarget{abstract}{%
\section{Abstract}\label{abstract}}

Our study focuses on the prevalence of conceptualizations of
communicative inequality in the field of communication studies after the
end of World War II. While communication studies has adopted and been
influenced by conceptualizations of inequality from related disciplines
and fields, conceptualizations of communicative inequality seem to have
played only a marginal role. By means of a network analysis conducted on
a corpus of more than fifteen thousand articles published in eight
prominent international journals in the field\footnote{\emph{Journal of
  Communication}; \emph{Communication Monographs}; \emph{Public Opinion
  Quarterly}; \emph{Journalism \& Mass Communication Quarterly};
  \emph{Communication Research}; \emph{Media, Culture \& Society};
  \emph{European Journal of Communication;} and \emph{Discourse \&
  Society.}} between 1945 and 2018, this study aims to map the
prominence and adoption of different conceptualizations of communication
inequality. With the tools of network analysis, the study identifies
particular conceptualizations by tracing the most co-occuring cited
authors associated with a particular conceptualization across time. We
identify four distinct clusters of conceptualizations: modernization
theory, cultural imperialism, knowledge gap, and digital divide.
Historically, approaches to communication inequality have been divided
either along ideological lines---largely defined by support for
(modernization theory) or opposition to (cultural imperialism) US
foreign policy---or in terms of different levels of communication
inequality. While both modernization and cultural imperialism focus on
international communication inequality, the knowledge gap tradition
focuses

\enlargethispage{2\baselineskip}

\vspace*{2em}

\noindent{\emph{History of Media Studies}, vol. 2, 2022}


 \end{titlepage}



\noindent on interpersonal differences. We argue that the dominant
approaches and paradigmatic shifts in conceptualizations of
communication inequality have largely been driven by forces outside of
communication studies. Modernization, which dominated the period until
the late 1970s, grew from US interests in securing hegemony in the third
world. Critiques of cultural imperialism emerged during the 1970s as a
direct challenge to modernization theory, connected strongly to
third-world opposition to US hegemony. The notion of a digital divide,
which has become the predominant conceptualization of communication
inequality since 2000, stems largely from the concerns of the US
Commerce Department's National Telecommunications and Information
Administration for providing ``universal service'' to US citizens, while
the knowledge gap tradition relates to the effectiveness of top-down
communication campaigns.

\hypertarget{resumen}{%
\section{Resumen}\label{resumen}}

Nuestro estudio se centra en las conceptualizaciones sobre la
desigualdad comunicativa en el campo de los estudios de comunicación que
han predominado tras el final de la segunda guerra mundial. Mientras que
las conceptualizaciones sobre la desigualdad que provienen de
disciplinas y campos afines han sido adoptadas o han influenciado a los
estudios de comunicación, las conceptualizaciones sobre la desigualdad
comunicativa parecen haber desempeñado solo un papel marginal. Mediante
un análisis de redes a un corpus de más de quince mil artículos
publicados entre 1945 y 2018 en ocho revistas internacionales de
prestigio en el campo, este estudio pretende mapear la relevancia y
adopción de distintas conceptualizaciones sobre la desigualdad
comunicativa. Con el uso de herramientas de análisis de redes, el
estudio identifica conceptualizaciones particulares al rastrear a través
del tiempo a los autores citados más co-ocurrentes en asociación con una
conceptualización particular. Identificamos cuatro grupos de
conceptualizaciones: teoría de la modernización, imperialismo cultural,
brecha del conocimiento y brecha digital. Históricamente, los enfoques
de la desigualdad en la comunicación se han dividido en función de
líneas ideológicas ---definidas en gran medida por el apoyo (teoría de
la modernización) o la oposición (imperialismo cultural) a la política
exterior de Estados Unidos--- o en términos de diferentes niveles de
desigualdad en la comunicación. Mientras que la modernización y el
imperialismo cultural se centran en la desigualdad comunicativa
internacional, la tradición de la brecha de conocimiento se centra en
las diferencias interpersonales. Sostenemos que los enfoques dominantes
y los cambios paradigmáticos en las conceptualizaciones sobre la
desigualdad comunicativa han sido impulsados por fuerzas ajenas a los
estudios de comunicación. La modernización, que dominó el periodo hasta
finales de la década de los setenta, nació de los intereses de Estados
Unidos por asegurar la hegemonía en el tercer mundo. Las críticas al
imperialismo cultural surgieron en los años setenta como un desafío
directo a la teoría de la modernización, conectada fuertemente con la
oposición del tercer mundo a la hegemonía estadounidense. La noción de
brecha digital, la conceptualización predominante de la desigualdad en
la comunicación desde el año 2000, proviene en gran medida de las
preocupaciones de la Administración Nacional de Telecomunicaciones e
Información del Departamento de Comercio de Estados Unidos por
proporcionar un ``servicio universal'' a los ciudadanos estadounidenses,
en tanto que la tradición de la brecha de conocimiento se relaciona con
la eficacia de las campañas de comunicación de arriba hacia abajo.

\vspace*{2em}




\newthought{Let us say} you were faced with the challenge of correctly positioning
social scientists within different theoretical traditions or paradigms.
The caveat is that you know nothing about them and are allowed to ask
only a single question. Arguably, the best choice would be to query them
about their views on inequality. If someone responds that the root cause
of inequality is the exploitation of workers, made possible through the
private property of the means of production, you would be quite
justified in guessing that this person is not a huge fan of Friedrich
Hayek or Talcott Parsons. If, on the other hand, the answer is that
income and profit are just rewards of an unerring market, distributed
according to merit, and that attempts to interfere with this mechanism
of distribution would only bring calamity upon a society so foolish as
to attempt it, then you would know better that to invite this person to
join your monthly reading circle of \emph{Das Kapital}.

Inequality is one of the key concepts in the social sciences around
which the most fruitful academic debates have historically arisen,
compared and related to concepts such as power, ideology, values,
rights, and culture. Debates about the causes and consequences of
inequality in society helped to articulate the most distinctive schools
of thought in the social sciences, such as the functionalist and Marxist
paradigms in sociology. These ideas have been central to critical
theories related to class, race, gender, or sexual orientation, as we
find in trajectories of thought inspired by postcolonialism, feminist
theories, and LGBTQI critique. The fundamental point of divergence that
constitutes the most prominent paradigms lies in their
conceptualizations of inequality; while functionalism views inequality
not only as inevitable but as a desirable and therefore functional
element in the formation and evolution of (modern) society, critical
schools of thought view inequality as dysfunctional, or rather, as a
functional tool of the dominant class in its reproduction of power and
maintenance of the status quo.

The situation within communication studies, however, seems to be
different. We can certainly observe the influence of economic and
sociological theories and critiques of inequality like Marxism or the
social stratification theory of Pierre Bourdieu, as well as other
critical traditions, which are most prominent in analyses of the ways in
which dominant power relations and hierarchies are reproduced in mass
media content. Examples include critical discourse analysis, framing
analysis, and ideology critique.\footnote{Boris Mance, ``The Changing
  Role and Patterns of Critical Communication Scholarship in the
  Academic Journal Publishing System,'' (doctoral thesis, University of
  Ljubljana, 2020).} However, all of these approaches deal with the ways
social inequalities are represented or reinforced \emph{through} the
mass media; they do not focus their attention on the ways inequalities
are present \emph{in} systems of mass communication---that is, on
communication inequalities.

In order to track the prominence, historical development, and relations
between different conceptualizations of communication inequality, our
study traces selected keywords as indicators of conceptualizations of
communication inequality. Concepts enable adherents to communicate
through the established linguistic system of codes, or a ``shared body
of words and meanings,''\footnote{Raymond Williams, \emph{Keywords: A
  Vocabulary of Culture and Society} (1976; repr., New York: Oxford
  University Press, 2015).} and evolve out of the need for more precise,
practical, and economical communication among members of a scientific
community. As such, concepts serve as mechanisms of reduction, each
denoting a set of arguments that a community agrees upon and that
distinguishes it from the concepts of other communities. Accordingly,
the analysis of conceptualizations of communication inequality can also
give insight into the prominence, historical development, and relations
between different paradigms within communication studies.

\hypertarget{institutionalization-of-administrative-and-critical-paradigms-in-communication-research}{%
\section{Institutionalization of Administrative and Critical
Paradigms in Communication
Research}\label{institutionalization-of-administrative-and-critical-paradigms-in-communication-research}}

The institutionalization of the administrative paradigm in the US was
largely supported by government and private actors, particularly the
Rockefeller Foundation,\footnote{Todd Gitlin, ``Media Sociology: The
  Dominant Paradigm,'' \emph{Theory and Society} 6, no. 2 (1978): 228.}
the Ford Foundation,\footnote{Peter Simonson and John Durham Peters,
  ``Communication and Media Studies, History to 1968'' in \emph{The
  International Encyclopedia of Communication}, ed. Wolfgang Donsbach
  (Malden, MA: Wiley-Blackwell, 2008).} and various branches of the US
administration,\textsuperscript{6} such
as the Army, the CIA, and the State Department.\textsuperscript{7} By privileging research on media effects, media and
communication research was gradually detached from other fields of
communication research such as journalism and speech.\textsuperscript{8}

\newpage

Government\marginnote{\textsuperscript{6} Brett Gary, ``Communication Research, the
  Rockefeller Foundation, and Mobilization for the War on Words,
  1938--1944,'' \emph{Journal of Communication} 46, no. 3 (1996).}\marginnote{\textsuperscript{7} Jefferson
  Pooley, ``The New History of Mass Communication Research,'' in
  \emph{The History of Media and Communication Research: Contested
  Memories}, ed. David Park and Jefferson Pooley (New York: Peter Lang,
  2008).}\marginnote{\textsuperscript{8}\setcounter{footnote}{8} Juha
  Koivisto, \emph{Mapping Communication and Media Research} (Tampere,
  Finland: Tampere University Press, 2010).} funding shaped the research programs of the most influential
research institutes, which in turn shaped the emerging field of
communication studies. Between 1950 and 1951, Lazarsfeld's Bureau for
Applied Social Research (BASR), for example, gained 75 percent of its
funding from government sources, primarily from the US Air Force's Human
Resources Research Institute and the Department of State's Voice of
America program. Between 1952 and 1953, the number rose to 84
percent.\footnote{Timothy Glander, \emph{Origins of Mass Communications
  Research During the American Cold War: Educational Effects and
  Contemporary Implications} (Mahwah, NJ and London: Lawrence Erlbaum
  Associates, 2010), 124.} BASR was no exception in this regard, as at
the same time, Hadley Cantril's Institute for International Social
Research (IISR), as well as the Center for International Studies
(CENIS), gained more than 75 percent of their income from contracts with
the government.\footnote{Christopher Simpson, \emph{Science of Coercion:
  Communication Research \& Psychological Warfare, 1945}--\emph{1960}
  (New York: Oxford University Press, 1994), 4.} At this time (1952), a
whopping 96 percent of federal funds for social sciences were provided
by the military.\footnote{Simpson, \emph{Science of Coercion}, 52.} The
influence of the military on the early institutionalization of the field
can be seen as well in the fact that the four presidents of the
International Communication Association (ICA) between 1953 and 1962 all
had military backgrounds, while a third of POQ editors and editorial
board members were financially dependent upon psychological warfare
contracts.\footnote{Simpson, \emph{Science of Coercion}, 43.}

These circumstances shaped the understanding of communication in general
and communication inequality in particular. The interests of the US
government---and particularly the military---to use communication in
order to reach its goals favored an understanding of communication as
propaganda, reflected in a strong research focus on media effects, as
well as the role of mass media as a tool of ``modernization'' in the
developing world. In such a conception, communication inequality between
the elites, who determine the goals and methods of communication
campaigns, and the masses, who are the intended site for the production
of media effects, is implicitly assumed and thereby naturalized. Those
who receive such messages are not meant to actively participate in
processes of mass communication; they are meant to be persuaded. The
effectiveness of persuasion can then be measured by the emerging
``science'' of public opinion polling. As Simpson has eloquently put it:

\begin{quote}
As will become apparent, the ``dominant paradigm'' of the period proved
to be in substantial part a paradigm of dominance, in which the
appropriateness and inevitability of elite control of communication was
taken as a given. As a practical matter, the key academic journals of
the day demonstrated only a secondary interest in what communication
``is.'' Instead, they concentrated on how modern technology could be
used by elites to manage social change, extract political concessions,
or win purchasing decisions from targeted audiences. Their studies
emphasized those aspects of communication that were of greatest
practical interest to the public and private agencies that were
underwriting most of the research. This orientation reduced the
extraordinarily complex, inherently communal social process of
communication to simple models based on the dynamics of transmission of
persuasive---and, in the final analysis, coercive---messages.\footnote{Simpson,
  \emph{Science of Coercion}, 62.}
\end{quote}

The decisive impetus for the rise of the critical paradigm, on the other
hand, arrived from labor militancy and new social movements in the late
1960s, drawing on critique of the structural political and economic
determinism of mass communication, historical materialism, and a
questioning of the ``hegemonic status of logical
positivism.''\footnote{Kaarle Nordenstreng, ``Ferment in the Field:
  Notes on the Evolution of Communication Studies and Its Disciplinary
  Nature,'' \emph{Javnost---The Public} 11, no. 3 (2004).} While the
administrative paradigm naturalized (communication) inequality, the
critical paradigm embraced demystification and struggle against social
inequalities, focusing on the ways the mass media serves to obfuscate
and reproduce dominant power relations. Critical research had scant
opportunities to obtain significant funding from national political and
economic institutions.\footnote{William H. Melody and Robin E. Mansell,
  ``The Debate over Critical vs. Administrative Research: Circularity or
  Challenge,'' \emph{Journal of Communication} 33, no. 3 (1983).} Key
material resources for critical research were easier to obtain at the
international level, where national economic and political interest had
less influence,\footnote{Dallas W. Smythe and Tran Van Dinh, ``On
  Critical and Administrative Research: A New Critical Analysis,''
  \emph{Journal of Communication} 33, no. 3 (1983).} as for instance,
with UNESCO's New World Information and Communication Order (NWICO)
initiative and the International Association for the Study of Mass
Communication (IAMCR), which actively supported progressive thought by
establishing a politico-economic section.\footnote{Kaarle Nordenstreng,
  ``Being (Truly) Critical in Media and Communication Studies:
  Reflections of a Media Scholar between Science and Politics,''
  \emph{Javnost---The Public} 23, no.1 (2016).}

\hypertarget{contested-concepts}{%
\subsection{Contested
Concepts}\label{contested-concepts}}

The issues associated with conceptualizations have long preoccupied
scholars. Numerous studies have qualitatively examined contested
concepts, such as ``capitalism,''\footnote{Luc Boltanski and Eve
  Chiapello, \emph{The New Spirit of Capitalism} (London: Verso, 2007).}
``alienation,''\footnote{Williams, \emph{Keywords}.} ``ideology,''
\footnote{John Downey and Jason Toynbee, ``Ideology: Towards Renewal of
  a Critical Concept,'' \emph{Media, Culture \& Society} 38, no. 8
  (2016).} ``hegemony,'' \footnote{Joe L. Kincheloe and Peter L.
  McLaren, ``Rethinking Critical Theory and Qualitative Research,'' in
  \emph{Landscape of Qualitative Research: Theories and Issues,} ed.
  Norman K. Denzin and Yvonna S. Lincoln (London: SAGE, 1998).} ``public
opinion,''\footnote{Slavko Splichal, \emph{Public Opinion:}
  \emph{Developments and Controversies in Twentieth Century} (Lanham,
  MD: Rowman \& Littlefield, 1999).} or ``public sphere,''\footnote{Natalie
  Fenton, ``Fake Democracy: The Limits of Public Sphere Theory,''
  \emph{Javnost---The Public} 25, no. 1--2 (2018).} and found that the
original critical concepts have become neutralized. To trace
conceptualizations of inequality, we believe quantitative analysis is
more appropriate because it allows us to identify the relationships
between conceptualizations and scholarly communities that appear to be
(or actually are) unrelated and/or have been labeled differently.

The use of keywords as indicators of prominence of particular strands of
research also has a long history in communication research, with
indicator selection based on deductive and inductive approaches. The
former uses corpora (e.g., the population of research papers) to
identify the most prominent keywords and is particularly popular for
identifying trends in longitudinal research.\textsuperscript{24} The latter method, in which
specific concepts are selected in advance, is better suited to
identifying the dynamics of less prominent or non-mainstream research.
Such methods have been used, for example, to examine the development of
concepts in communication research and\newpage

\noindent have found an increase in
cognitive\marginnote{\textsuperscript{24}\setcounter{footnote}{24} Mark A. Hamilton
  and Kristine L. Nowak, ``Information Systems Concepts across Two
  Decades: An Empirical Analysis of Trends in Theory, Methods, Process,
  and Research Domains,'' \emph{Journal of Communication} 55, no. 3
  (2005); Julian Lin and Seow Ting Lee, ``Mapping Twelve Years of
  Communication Scholarship: Themes and Concepts in the Journal of
  Communication'' in \emph{The Outreach of Digital Libraries: A
  Globalized Resource Network}, Lecture Notes in Computer Science 7634
  (Berlin and Heidelberg: Springer, 2012).} conceptualizations and a decrease in behavioral
conceptualizations.\footnote{Hamilton and Nowak, ``Information Systems
  Concepts,'' 529--53.}

Since Eugene Garfield's institutionalization of citation indexing in
1964, bibliographic analysis, which views citations as indicators of the
significance of scientific ideas, has become more important for
identifying scientific communities.\footnote{Robert K. Merton and Norman
  William Storer, \emph{The Sociology of Science: Theoretical and
  Empirical Investigations} (Chicago: University of Chicago Press,
  1998).} In particular, its extension, co-citation analysis, has been
used to map the connections of co-cited authors or works and to identify
the structural patterns of scientific ideas. These analyses have been
used in the field to detect, for example, differences between journals
in co-citation patterns and topics,\footnote{Ronald E. Rice et al.,
  ``What's in a Name? Bibliometric Analysis of Forty Years of the
  \emph{Journal of Broadcasting} (\emph{\& Electronic Media}),''
  \emph{Journal of Broadcasting \& Electronic Media} 40, no. 4 (1996).}
or gender bias in citations, also known as the Matilda
effect.\footnote{Silvia Knobloch-Westerwick and Carroll J. Glynn, ``The
  Matilda Effect---Role Congruity Effects on Scholarly Communication: A
  Citation Analysis of Communication Research and Journal of
  Communication Articles,'' \emph{Communication Research} 40, no. 1
  (2013).}

The digitization of content and the availability of bibliometric
(meta)data, as well as the advancement of network analysis methods and
community detection algorithms, have proven productive in tracing the
long-term dynamics of conceptualizations, such as examining the
genealogy of the concept of social capital across
disciplines,\footnote{Chul-joo Lee and Sohn Dongyoung, ``Mapping the
  Social Capital Research in Communication,'' \emph{Journalism \& Mass
  Communication Quarterly} 93, no. 4 (2016).} different strands of
thought in agenda-setting research,\footnote{Zixue Tai, ``The Structure
  of Knowledge and Dynamics of Scholarly Communication in Agenda Setting
  Research, 1996--2005,'' \emph{Journal of Communication} 59, no. 3
  (2009).} reconceptualizations of the concept of the public
sphere,\footnote{Adrian Rauchfleisch, ``The Public Sphere as an
  Essentially Contested Concept: A Co-citation Analysis of the Last
  Twenty Years of Public Sphere Research,'' \emph{Communication and the
  Public} 2, no. 1 (2017).} and/or mapping critical approaches in
communication research.\footnote{Slavko Splichal and Boris Mance,
  ``Paradigm(s) Lost? Islands of Critical Media Research in
  Communication Journals,'' \emph{Journal of Communication} 68, no. 2
  (2018).}

To explore the presence of communication inequalities in the scholarly
literature, we have conducted a network analysis on a corpus of more
than fifteen thousand articles published in eight prominent journals in
the field. We will attempt to map the prominence and adoption of
different conceptualizations of communication inequality in
communication studies. With the tools of network analysis, we attempt to
identify particular conceptualizations by tracing the most co-occuring
words and cited authors associated with a particular conceptualization
across time. The contribution of the present paper to the existing body
of research can be found in our identification of the dynamics and
prominence of particular conceptualizations of communication inequality
appearing in the period following the end of World War II. Furthermore,
by coupling the concepts with the co-occurring cited authors in articles
and treating them with the tools of network analysis, we identify
relations between different conceptualizations of inequality, paradigms
affiliated with particular conceptualizations, as well as relations
between those paradigms.

\hypertarget{defining-communication-inequality-and-identifying-keywords}{%
\subsection{Defining Communication Inequality and
Identifying
Keywords}\label{defining-communication-inequality-and-identifying-keywords}}

As Yu notes, research on information and communication inequalities is
highly fragmented and there is a notable lack of integrative research
and theory building.\footnote{Liangzhi Yu, ``The Divided Views of the
  Information and Digital Divides: A Call for Integrative Theories of
  Information Inequality,'' \emph{Journal of Information Science} 37,
  no. 6 (2011).} This leads to the fact that there is no single term
used in the scholarly literature to describe the concept of
communication inequality. Rather, there is a multiplicity of approaches,
theories, and concepts dealing with its various aspects. To further
complicate matters, the terms used to describe various aspects of
communication inequality are historically specific, some of them
emerging only relatively recently (for example, the term ``digital
divide'' ), possibly replacing other terms and concepts, while some
concepts were prominent in earlier eras and have since almost
disappeared from use (for example, Lerner's modernization theory),
possibly being replaced by other terms and concepts.

In order to be able to identify keywords for our analysis, we need first
of all to construct a broad enough definition of communication
inequality to take account of this paradigmatic fragmentation. For this
purpose, we define communication inequality as differences in the access
to and capacity to use the means of producing, manipulating, and
receiving information. These differences can be observed on several
levels: between individuals, between social groups, or at the level of
whole societies, nations, or regions of the world. These levels are
closely related to those identified by Yu (see Figure 1); however, with
the addition of the international dimension, which featured prominently
in the critical literature during the 1970s and 1980s, these differences
can be related closely to non-aligned efforts at addressing perceived
imbalances in global information and communication flows (the debates on
the New World Information and Communication Order, or NWICO).

\vspace*{2em}

\begin{figure}
    \centering
    \includegraphics{mance-figure-one.png}
    \caption{\emph{Theoretical Perspectives of Information Inequality.}
Source: Liangzhi Yu, ``Understanding Information Inequality: Making
Sense of the Literature of the Information and Digital
Divides,''~\emph{Journal of Librarianship and Information Science} 38,
no. 4 (2006): 229--52, \url{https://doi.org/10.1177/0961000606070600}.}
    \label{fig:one}
\end{figure}


Based on this definition, we have used a two-pronged approach to create
a list of keywords for the analysis. First, we have created keywords by
combining the words information, communication, media, digital,
internet, and ICT with terms designating inequality, such as inequality,
gap, divide, poverty, access, rich, poor, haves, have nots, etc.
Secondly, we have identified specific vocabularies used by various
research traditions to refer to information and communication
inequalities using a literature review. To begin with, we have
identified the relevant schools of thought or research traditions,
relying mostly on Yu and Donsbach.\footnote{Wolfgang Donsbach, \emph{The
  International Encyclopedia of Communication} (Malden, MA: Blackwell
  Publishing, 2008).} In addition to the approaches identified by
Yu,\footnote{Yu, ``Divided.''} we have added the three most prominent
paradigms within development communication (modernization theory,
cultural imperialism, and participatory communication), as they deal
with international communicative inequalities, and public sphere theory.
We then conducted a literature review of each of the identified research
traditions in order to identify specific terminology used to describe
information and communication inequalities within those traditions.

In this way we have created a list of 144 potential keywords. In the
second step, we have reduced the number of keywords by eliminating those
that are very infrequent in our corpus (n $ \leq $ 10), as we are interested in
identifying systematic patterns and not terms that appear only
idiosyncratically. Forty-eight keywords have met this criterion. In the
third step, we have reviewed the results and eliminated those keywords
that lack specificity (i.e., are not used exclusively to refer to
information and communication inequality). When keywords were found to
be too broad, we have attempted to modify them to increase specificity
when possible (e.g., substituting ``unequal media access'' for the
non-specific ``media access''); however, none of these modified keywords
met the frequency criteria (n  10) to be included in the
analysis. In the case of the keyword ``modernization,'' we have added
the requirement that either ``Lerner'' or ``Rogers'' (the two most
prominent authors in this tradition) co-occur in the same article in
order to achieve specificity.

In this way, we have reduced the initial list of potential keywords to
twenty-four (see Table 1). In addition, we have searched for articles in
the Web of Science ``Communication'' category containing any of the
twenty-four keywords in their topics in order to supplement the sample
of prominent keywords by any that we might have missed. By extracting
the common phrases appearing in the titles, abstracts, and keywords of
8,024 resulting articles, the noun phrase detection analysis in the
CiteSpace software package yielded 1,018 results (Appendix E ). Manual
inspection of phrases appearing at least ten times did not yield any
additional keywords. We are therefore confident that our list adequately
represents the most relevant keywords referring to communication
inequality.

\newpage

\hypertarget{table-1-final-list-of-keywords-and-their-frequencies}{%
\subsection{Table 1. Final List of Keywords and Their
Frequencies}\label{table-1-final-list-of-keywords-and-their-frequencies}}

\tabulinesep=1.1mm
{\begin{longtabu} to 1.05\textwidth { X[l] X[c] } 
\emph{Keyword} & \emph{N} \\
\endfirsthead
\emph{Keyword} & \emph{N} \\
\endhead
knowledge gap & 270 \\
cultural imperialism & 213 \\
digital divide & 198 \\
(``modernization'' AND ``Lerner'') OR (``modernization'' AND ``Rogers'')
& 149 \\
world system OR world-system & 123 \\
media imperialism & 122 \\
cultural domination & 99 \\
information gap & 87 \\
NWICO OR New World Information and Communication Order & 71 \\
communication gap & 49 \\
digital inequalit* & 42 \\
New International Information Order & 42 \\
information poverty & 41 \\
participatory communication & 33 \\
information poor & 30 \\
counterpublic & 29 \\
digital inclusion & 25 \\
equitable access & 21 \\
information inequalit* & 16 \\
access to ICT OR access to the ICT & 12 \\
balanced flow of information & 12 \\
communication inequalit* & 11 \\
digital exclusion & 11 \\
electronic colonialism & 11 \\
\end{longtabu}}

\vspace*{2em}

\hypertarget{method}{%
\section{Method}\label{method}}

\hypertarget{sampling}{%
\subsection{Sampling}\label{sampling}}

In order to identify the structural and dynamic features of
conceptualizations of communication inequalities, our analysis detects
the patterns of co-occurrences of indicators---selected terms/concepts
and cited authors---in the articles published in eight elite scholarly
journals whose research scope is focused on the problems of media and
mass communication and which are indexed in the SSCI bibliographic
database Clarivate Analytics: Web of Science (WOS). The labor-intensive
process of cleaning and standardizing more than five hundred thousand
cited authors' entries, and acquisition and preparation of more than
fifteen thousand articles, limited the number of journals in the sample.
The selection of journals seeked to reflect: a) the journal's
prominence, indicated by relatively high impact factor, and b)
distinctiveness of research communities, operationalized by affiliations
with professional communication organizations, reflecting broader and
substantial trends in communication research, thus providing a
relatively long publication period for analysis (the case of US based
journals). Apart from the commercial journal \emph{Communication
Research}, the sampled US-based journals include association journals,
such as \emph{Journalism and Mass Communication Quarterly}, published by
the Association for Education in Journalism and Mass Communication
(AEJMC); \emph{Communication Monographs}, published by the National
Communication Association (NCA); \emph{Public Opinion Quarterly},
published by the American Association for Public Opinion Research
(AAPOR); and the \emph{Journal of Communication}, published by the
International Communication Association (ICA). The European journals
included are: the \emph{European Journal of Communication}, oriented to
publishing European research; \emph{Media, Culture \& Society}, which is
critical in scope; and \emph{Discourse \& Society}, which relates to a
specific epistemological community. These three journals appeared later
than their US counterparts, together with the institutionalization of
communication research in Europe. The population of journals that deal
with communication cannot be defined, as research published in journals
from other research fields such as sociology, psychology, linguistics,
computer science, etc., also includes certain aspects of communication.
On the other hand, the population of communication journals categorized
in the SSCI database includes some very specialized journals that were
considered to be less relevant to this particular research problem, such
as \emph{New Media \& Society}.

The data range covers all published articles from 1945 through 2018,
although some journals commenced publication later in this period.
Included are only research and review articles and proceedings papers,
as they contain relevant research outputs. Publications such as
introductions/prefaces, book and software reviews, editorial material,
abstracts, meeting abstracts, letters, forums, biographical items,
obituaries, corrections, additions, and retraction notices were omitted
from the analysis.

The study was conducted using selected journals from the North Atlantic
region. It is therefore not representative of the entire field of
communication research over the seventy-year period, nor of the entire
North Atlantic region. The study is also limited by not including
monographic publications. Despite its limitations, the study presents
the main paradigms and conceptualizations of the communication of
inequality in Western Anglophone journals that can be considered
mainstream. However, further research on a different set of journals,
especially outside of the main Anglo-saxon tradition, would be welcome
as it might uncover how conceptualizations of communication inequality
differ among scholarly traditions and social contexts. A further
limitation of the study arises from the fact that conceptualizations of
communication inequality were operationalized via specific keywords.
This has meant that conceptualizations of communication inequality that
are not or cannot be tied to specific keywords were not registered. For
example, one could claim that concepts of capitalism, commodification,
and social class, which are central to political economy of
communication, necessarily imply communication inequality; however, as
they do not explicitly refer to it, they were not included as
indicators. In this sense, further qualitative work into
conceptualizations of communication inequality would enhance and
complement our quantitative findings.

\hypertarget{data-preparation-analysis-and-visualizations}{%
\subsection{Data Preparation, Analysis, and
Visualizations}\label{data-preparation-analysis-and-visualizations}}

The automated search for selected terms was performed on the population
of 15,180 articles using Boolean operators in search queries, where some
were adapted to include and/or exclude certain words or phrases. The
corpus was cleaned of the ``left-leaning'' and ``right-leaning''
characters, such as keys and symbols (. , ! ? ; / « » '' `' ; :), as
they could mask terms from the search algorithm. All articles containing
at least one mention of a term were included in the analysis.

Since the metadata of all publications published in the selected
journals in the WOS database is not standardized to the fullest extent,
the data files were exported and cleaned, including authors in 457,228
cited references. Special attention was devoted to standardization of
synonyms (different name records belonging to the same author) and
homonyms (exact name records belonging to different authors). The cases
of cited works with multiple authorships contain only the first author,
due to limitations of the WOS database which could not be overcome.

The sample consisted of those articles which contain at least one term,
and 1,180 of such articles contained 48,549 references pertaining to
20,249 unique authors. Distinguishing among texts where a particular
term/concept obtains central position and those where the term is
briefly mentioned is difficult. Since authors differ in their writing
styles, the richness of their (English) vocabulary, publication length,
etc., the relevance of the term in a particular article can not be
determined by its frequency. Therefore, all articles containing a
specific term were treated equally in the analysis (even when the term
appeared only once). On the other hand, more weight was given to
instances where a particular author held more references---as this
information is a better indicator of the centrality of someone's ideas
in the analyzed text.

By identifying the articles that contained the selected keywords, we
were able to select all authors appearing in the references in those
articles and generate networks containing only co-occuring terms and
cited authors in the same article. In this process, the width of links
contains the weighted information on co-occurrences of units, where
links created between the same node---so-called loops---were removed.

The visualization layout of the network is performed with Gephi software
with the use of a MultiGravity ForceAtlas 2 algorithm, which distributes
the nodes within the network according to vector, produced by two
opposing forces: the attraction force, based on the edge weights of the
nodes, and the repulsion force, generated by adjacent nodes.\footnote{Mathieu
  Jacomy, Tommaso Venturini, Sebastien Heymann, and Mathieu Bastian,
  ``ForceAtlas2: A Continuous Graph Layout Algorithm for Handy Network
  Visualization Designed for the Gephi Software,'' \emph{PLoS ONE} 9,
  no. 6 (2014): e98679.} In this way, nodes co-occurring more frequently
and co-occurring with a greater number of other nodes are drawn closer
together, while those co-occurring more seldomly are drawn further
apart. In the resulting graphs, the size of node labels is
proportionally dimensioned to the number of all edges connected to the
node. Analysis is conducted on the whole corpora and segmented according
to the three identified periods and journals. Visualizations conducted
of particular journals holding lower explanatory value are located in
the Appendices (A--E).

Clusters, or modules of nodes within the network, are detected on the
grounds of structural properties of the network---with their (uneven)
distribution based on the number and weight strength of edges connecting
them. Modules contain nodes with a higher density of stronger edges
among themselves, separated from neighboring clusters by a weaker and
lesser number of edges.\footnote{Vincent D. Blondel et al., ``Fast
  Unfolding of Communities in Large Networks,'' \emph{Journal of
  Statistical Mechanics: Theory and Experiment} 2008, no. 10 (October
  2008).} Modularity index designates the strength of division of a
network into modules, where a higher modularity index indicates higher
fragmentation of the network (more dense links between particular groups
of nodes and weaker links between neighboring groups). The resolution
parameter, on the other hand, adjusts the algorithm's sensitivity and
defines the number of modules.\footnote{Renaud Lambiotte, Jean-Charles
  Delvenne, and Mauricio Barahona, ``Laplacian Dynamics and Multiscale
  Modular Structure in Networks,'' \emph{IEEE Transactions on Network
  Science and Engineering} 1, no. 2 (2015).}

\hypertarget{results}{%
\section{Results}\label{results}}

\hypertarget{four-central-conceptualizations-of-communication-inequalities}{%
\subsection{Four Central Conceptualizations of
Communication
Inequalities}\label{four-central-conceptualizations-of-communication-inequalities}}

To better understand how the concepts of inequality under study are
conceptualized, we examined the relationships between the keywords under
study and the authors cited by identifying subtle structural properties
of the network using a clustering procedure. By reducing the entire
network to the most prominent nodes (a minimum degree value was set at
1,280, obtaining 0.37\% of nodes with highest values), the network of
keywords and cited authors obtained from 1,180 articles formed four
clusters, mirroring four main conceptualizations/traditions of research
on communication inequalities: ``cultural imperialism cluster'' (green,
42.1\% of units), ``knowledge gap cluster'' (purple, 35.5\%),
``modernization cluster'' (blue, 11.8\%), and ``digital divide cluster''
(orange, 10.5\%) (Figure 2).

\begin{figure}
    \centering
    \includegraphics{mance-figure-two.png}
    \caption{\emph{Clusters in the reduced network of 1,180 articles
published between 1945--2018}; min. degree = 1,280; resolution = 1.0;
modularity = 0.264; nodes visible = 76 (0.37\%). Cultural imperialism
cluster (green), knowledge gap cluster (purple), modernization cluster
(blue) and digital divide cluster (orange)}
    \label{fig:two}
\end{figure}


The mapped clusters reveal that the knowledge gap theories and theories
of cultural imperialism are distanced furthest. Clusters formed around
concepts and cited authors pertaining to theories of modernization,
positioned between knowledge gap theory and theories stemming from
critical tradition, show an overlap which indicates mutual patterns of
co-citation.\footnote{When clustering procedures are performed on a
  relatively small number of units, cases where an author has been cited
  with multiple works in a single article may have a stronger influence
  on cluster membership, due to the different treatment of edge weights
  by the clustering and spatialization algorithms. Albert Bandura and
  Ronald E. Rice are examples where the placement of the spatialization
  algorithm and the cluster membership differ.} This may indicate either
a cohabitation of different scientific currents or a contestation
between critical and dominant theoretical currents in the articles where
the terms/concepts co-occur together. The clustering does not
necessarily indicate a similarity of ideas, as it may just as well
indicate critique, i.e., negative citations. For example, the inclusion
of Ithiel de Sola Pool in the cultural imperialism cluster (albeit on
the periphery) alongside other, more prolific critical authors may be
due to criticism,\footnote{Peter Golding, ``Media Role in National
  Development,'' \emph{Journal of Communication} 24, no. 3 (1974).} or
references to research work which the author acknowledges but does not
endorse.\footnote{Vincent Mosco and Andrew Herman, ``Communication,
  Domination and Resistance,''~\emph{Media, Culture \& Society}~2, no. 4
  (1980).}

The relative distance of the digital divide cluster from others suggests
its conceptual distinctiveness, which to some extent can be attributed
to the fact that it is a relatively new concept, emerging only with the
advent of the Internet in the second half of the period under study, but
drawing on three distinctive and prior perspectives: knowledge gap
theory, critical tradition, and research on information richness.

The analysis of the entire corpora, containing all available
information, provides a general perspective. However, it obscures the
influence of specific journals with their unequal number of articles.
Therefore, in order to minimize bias and gain better insight into the
conceptualization of the digital divide, separate analyses are conducted
for individual journals and presented in the section below concerning
the era of the digital divide (2000--2018).

\hypertarget{three-eras-in-conceptualizing-communication-inequality}{%
\subsection{Three Eras in Conceptualizing
Communication
Inequality}\label{three-eras-in-conceptualizing-communication-inequality}}

In this section we will be adopting a diachronic/historical perspective
and examining how the prominence of various conceptualizations of
communication inequality has changed throughout the history of
(institutionalized) communication studies. We will attempt to
contextualize our results by connecting them with literature from the
history of communication studies that identifies the dominant social
interests that shaped the field of communication research and the ways
it conceptualizes communication inequality.

The significant variability of the articles that mentioned the analyzed
terms throughout the analyzed time period (Figure 3) indicates that the
terms used to describe communication inequality are historically
specific, not only in the case of the ``digital divide,'' a term that
for obvious reasons could not have been in circulation in the 1960s, but
also in the case of other terms like ``modernization,'' which has been
prominent in an earlier era and has since gone out of favor. Based on
this analysis, we propose that it is reasonable to talk about three eras
in the post-war conceptualization of communication inequality: the era
of modernization (1945--1978), the era of ideological struggle
(1979--1999), and the era of the digital divide (2000--2018).


\begin{figure}
    \centering
    \includegraphics{mance-figure-three.png}
    \caption{Chronological distribution of articles containing most
prominent terms of communication inequality (normalized by the number of
published articles and smoothened according to three-year average). The
cultural imperialism line groups keywords: cultural imperialism world
system, media imperialism, cultural domination, and NWICO. The knowledge
gap line groups keywords: knowledge gap, information gap, and
communication gap.}
    \label{fig:three}
\end{figure}


\hypertarget{the-era-of-modernization-19451978}{%
\subsection{The Era of Modernization
(1945--1978)}\label{the-era-of-modernization-19451978}}

The first era began in the early sixties, as terms designating
communication inequality were almost absent before that time. Its main
characteristic is the hegemony of the term ``modernization'' to describe
communication inequality, hence we call this era the "era of
modernization." Coined by Daniel Lerner, the term modernization implies
that societies can be hierarchically ordered, with ``traditional''
societies representing a lower stage of social evolution than ``modern''
ones, with the US occupying the pinnacle of this hierarchy.\footnote{Daniel
  Lerner, \emph{The Passing of Traditional Society: Modernizing the
  Middle East} (New York: The Free Press of Glencoe, 1958).} The term
modernization is therefore a projection of US hegemony toward the third
world, its prevalence in scholarly literature reflecting the impact of
geopolitics on scholarly literature. The relationship between so-called
``traditional'' and ``modern'' or ``developing'' and ``developed''
countries was of key geopolitical importance because winning the hearts
and minds of people in the third world emerged as one of the key goals
of US foreign policy.\footnote{Gilbert Rist, \emph{The History of
  Development: From Western Origins to Global Faith,} 3rd ed. (London
  and New York: Zed Books, 2008), 69--79; Robert K. Olson, \emph{US
  Foreign Policy and the New International Economic Order: Negotiating
  Global Problems, 1974--1981} (Boulder, CO: Westview Press, 1981).}
Already in his 1949 inaugural address, when he laid out his Manichean
view of the world, with its split between the irreconcilable forces of
``communism'' and ``democracy,'' President Truman named development of
less developed countries as the fourth pillar of defense of the free
world, alongside the Marshall plan, the United Nations, and
NATO.\footnote{Harry S. Truman, ``The Inaugural Address of Harry S.
  Truman,'' transcript of inaugural address delivered in Washington, DC,
  January 20, 1949.}
  
\begin{figure}
    \centering
    \includegraphics{mance-figure-four.png}
    \caption{Era of modernization visualized with reduced networks of
concepts (red) and cited authors (black) irrespective of journal
affiliation, 1945--1978; n of articles = 118; min. degree = 123; nodes
visible = 30 (2.65\%).}
    \label{fig:four}
\end{figure}


In modernization theory, communication inequality serves as an
explanatory variable of social evolution, as Lerner believes that the
presence of mass media is a key factor driving the transition from
``traditional'' to ``modern'' societies, while the relative absence of
mass mediated content in a country can hamper its social
evolution.\footnote{Lerner, \emph{Passing}.} In this model, unequal
global information and communication flows (from developed to developing
countries) are assumed to be desirable, as the influx of media content
and foreign investment in media infrastructure will help to
``modernize'' backwards societies. In this way, modernization theory
expressed and legitimized US interests in increasing the reach and
effectiveness of their propaganda throughout the third world. In fact,
the empirical material for Lerner's \emph{The Passing of Traditional
Society}, the foundational text of modernization theory, was gathered as
part of a government contract aimed at assessing the effectiveness of
the State Department's Voice of America radio program in the Middle
East.\footnote{Hemant Shah explains that as the Voice of America was
  facing scrutiny in Congress, the government had no reliable data with
  which to demonstrate its effectiveness. The task of producing such
  data was given to Leo Löwenthal, who engaged the Bureau of Applied
  Social Science Research, where Lerner was employed at the time. The
  contract stated that the main goal of the project, commencing in 1949,
  was to ``gain insights into the comparative effectiveness of the
  propaganda struggle between East and West {[}through{]} comparisons of
  reliability and popularity of VOA, BBC, USSR.'' Shah, \emph{The
  Production of Modernization: Daniel Lerner, Mass Media, and the
  Passing of Traditional Society} (Philadelphia: Temple University
  Press, 2011), 13. Löwenthal later explained his views on the role of
  communication science and particularly opinion measurement in
  evaluating the effectiveness of ``psychological warfare'' in an
  article published in \emph{POQ}; see Klapper and Löwenthal, ``The
  Contributions of Opinion Research to the Evaluation of Psychological
  Warfare,'' \emph{Public Opinion Quarterly} 15, no. 4 (1951).} Lerner
himself believed the role of international communications should be to
align the peoples of the ``Free World'' with US leadership and heaped
scorn on ``neutralists,'' who refused to take sides in the cold war,
regarding them as either ``privatized apathetics or apoplectic
antagonists'' to the world hegemon.\textsuperscript{47}

While modernization is clearly the hegemonic conceptualization of
communication inequality during this era, we can see the knowledge gap
cluster emerge and gain popularity especially during the second half of
the 1970s. First formulated by Tichenor, the knowledge gap hypothesis
posits that socioeconomic status is a mediating variable in acquiring
information transmitted through the mass media.\textsuperscript{48} Building on earlier research findings on
the uneven spread of information in targeted communication campaigns,
the focus of knowledge gap research was on social conflict, community
social structure, and information control, which could optimize
mobilization functions exercised by campaign planners, policy makers,
and other advocates and actors.\textsuperscript{49}
  
\newpage

The\marginnote{\textsuperscript{47} Daniel Lerner,
  ``International Coalitions and Communications Content: The Case of
  Neutralism,'' \emph{Public Opinion Quarterly} 16, no. 4 (1952): 687.}\marginnote{\textsuperscript{48} Philip
  J.Tichenor, George A. Donahue, and Clarice N. Olien, ``Mass Media Flow
  and Differential Growth in Knowledge,'' \emph{Public Opinion
  Quarterly} 34, no. 2 (1970).}\marginnote{\textsuperscript{49}\setcounter{footnote}{49} Katam Viswanath and John R.
  Finnegan, ``The Knowledge Gap Hypothesis: Twenty-Five Years Later,''
  \emph{Annals of the International Communication Association} 19, no. 1
  (1996): 205.} hegemony of modernization theory in conceptualizing communication
inequalities is symptomatic for the overwhelming influence of the US
government and select private interests in determining the research
agenda of communication studies in particular and social sciences in
general. The era of modernization coincides with the
institutionalization of the dominant or administrative paradigm in mass
communication and media studies in the United States, embodied in the
media effects tradition, which was initiated and supported by the
``funding fathers'': the Rockefeller Foundation,\footnote{Gitlin,
  ``Media Sociology.''} the Ford Foundation,\footnote{Simonson and
  Peters, ``Communication and Media Studies.''} and various branches of
the US administration, such as the Army, the CIA, and the State
Department.\footnote{Gary, ``Communication Research''; Simonson and
  Peters, ``Communication and Media Studies''; Pooley, ``New History.''}

In this paradigm, communication inequality could emerge as a problem in
only a very limited sense. In order for propaganda to produce its
results and for the mass media to produce ``modernization'' of the
developing world, the persuasive messages needed to reach their intended
audiences, and these audiences needed to be open to persuasion. However,
the problem presented itself quite differently in the developing world
as opposed to the US. In the developing world, underdeveloped
communication infrastructure presented the main barrier to the
effectiveness of persuasive communication, while in the US, the main
issue was how audiences were receiving and processing information.
Hence, Charles Glock argued for communication studies to adopt a
two-pronged approach: ``In the United States, where most of the mass
media are accessible to everyone, the questions of exposure and
resistances to exposure require primary attention. In a country, such as
Jordan, where the distribution of the mass media is limited,
accessibility and restrictions on accessibility are currently more
crucial problems.''\footnote{Charles Y. Glock, ``The Comparative Study
  of Communications and Opinion Formation,'' \emph{Public Opinion
  Quarterly} 16, no. 4 (1952): 515.} Our results indicate that
communication studies did indeed develop according to this two-pronged
approach, with modernization theory focusing attention on limited media
access in developing countries and the knowledge gap hypothesis emerging
to understand issues of exposure and ``resistance'' to exposure within
the US.

\hypertarget{the-era-of-ideological-struggle-19791999}{%
\subsection{The Era of Ideological Struggle
(1979--1999)}\label{the-era-of-ideological-struggle-19791999}}

The second era begins in the second half of the 1970s. We have chosen
1978 as the cut-off point, as it represents the sharpest decline in the
dominance of modernization theory and the starting point of the
ascendance of theories of cultural imperialism (defined by the keywords
``media and cultural imperialism,'' ``world-system'' and ``cultural
domination''), which have emerged to directly challenge modernization
theory. That is why we call this era the ``era of ideological
struggle,'' as paradigms are divided on the basis of political ideology,
with modernization remaining to be aligned with US foreign policy goals,
and the cultural imperialism cluster emerging to challenge and oppose US
hegemony.

While the struggle over international communication inequalities is the
defining feature of this era, we can also notice the continuing presence
of the knowledge gap cluster, which develops relatively independently
from both the modernization and cultural imperialism traditions. In this
sense, another line dividing scholarly communities emerges: While the
modernization and cultural imperialism clusters both engage with
international communication inequality, albeit from opposing ideological
positions, the knowledge gap cluster engages with communication
inequality primarily on an interpersonal level. Unlike the case of
international communication inequalities, where our analysis revealed a
polarization between the critical and administrative paradigms, research
into interpersonal communication inequality does not exhibit such
ideological polarization. Rather, throughout this era, the knowledge gap
cluster is gaining in prominence and is consolidating as a separate
research tradition around a group of key authors---the most prominent
being Phillip J. Tichenor.

\begin{figure}
    \centering
    \includegraphics{mance-figure-five.png}
    \caption{Era of ideological struggle visualized with reduced networks
of concepts (red) and cited authors (black) irrespective of journal
affiliation, 1979--1999; n of articles = 486; min. degree = 833; nodes
visible = 30 (0.39\%).}
    \label{fig:five}
\end{figure}


The sharp decline in the prevalence of modernization theory in the late
1970s took place during the crisis of US hegemony,\footnote{See Giovanni
  Arrighi, \emph{Adam Smith in Beijing} (London and New York: Verso,
  2007); Robert Brenner, \emph{Economics of Global Turbulence: The
  Advanced Capitalist Economies from Long Boom to Long Downturn,
  1945--2005} (London: Verso, 2006), 143--80.} which gave rise to
challenges to the dominant theories in the social sciences in general,
as well as in communication studies in particular. Consequently, the
promises of development and modernization, which informed the dominant
conceptualizations of communication inequality in the preceding era,
started to lose their power. They were becoming harder to defend as the
gap between rich and poor nations was widening,\footnote{Karl P.
  Sauvant, ``Toward the New International Economic Order,'' in \emph{The
  New International Economic Order: Confrontation of Cooperation between
  North and South?}, ed. Karl P. Sauvant and Hajo Hasenpflug (Boulder,
  CO: Westview Press, 1977), 4.} while rural populations in the third
world stubbornly refused to be ``modernized.'' Consequently, scholars
began to modify Lerner's highly exaggerated claims regarding the
effectiveness of mass media by theorizing limited media effects and
including theories like diffusion of innovation to explain the limited
success of modernization campaigns. However, lacking the practical
success of modernization campaigns and facing increasing scholarly
scrutiny and critique, modernization theory was slowly abandoned even by
its erstwhile proponents.\footnote{See Everett M. Rogers,
  ``Communication and Development: The Passing of the Dominant
  Paradigm,'' \emph{Communication Research} 3, no. 2 (1976).}

Theoretically, opposition to US hegemony took the form of interest in
the notion of imperialism, as well as the highly related strands of
dependency and world-systems theory. While theories of development and
modernization assumed that each country individually followed a
predetermined evolutionary path from tradition to modernity, the
critical theoretical traditions recast global economic inequalities not
as different stages of social evolution on the path to the pinnacle of
US-style capitalism, but as inequitable relations between a center and
periphery of a globalized economic system---between exploiter and
exploited, colonizer and colonized. These perspectives are inspired by
and tightly interwoven with the process of decolonization and the voices
from the periphery who drew attention to the continuing impact of
colonialism on colonial subjectivity and culture.\footnote{Frantz Fanon,
  \emph{Wretched of the Earth}, trans. Constance Farrington (New York:
  Grove Press, 1963).} Critiques of US hegemony also extended to the
economic dominance of the US and its multinationals,\footnote{Kwame
  Nkrumah, \emph{Neo-Colonialism: The Last Stage of Capitalism} (London:
  Thomas Nelson \& Sons, 1965).} as well as to the dominance of the
center of the global economy over a subordianted periphery, as theorized
by the ``Latin American school of development,''\footnote{Cristóbal Kay,
  \emph{Latin American Theories of Development and Underdevelopment}
  (London and New York: Routledge, 1989), 2.} which offered alternative
theoretical viewpoints to the predominant neoclassical and Keynesian
analyses of development and modernization.

\enlargethispage{\baselineskip}

While these theories emerged from the field of economics and political
economy, they soon became very influential in the field of
communication, which can be seen in the way critical communication
scholars adopted the vocabulary of imperialism and world-systems theory
to describe communication inequalities.\footnote{E.g., Herbert I.
  Schiller, \emph{Mass Communication and American Empire} (Boston:
  Beacon Press, 1969), 5--19.} Theories of the cultural imperialism
cluster attempted to recast global inequalities between societies not as
stages of social evolution, but through an antagonistic relationship
between exploiters and exploited. Global communication inequality is
reframed as a mechanism of ideological and cultural domination of the
informationally poor by the informationally rich, rather than as a
benevolent tool of social advancement.\footnote{See Ariel Dorfman and
  Armand Mattelart, \emph{How to Read Donald Duck} (1971; repr., New
  York and London: OR Books, 2018).} Critical scholars have begun to
question unbalanced global information and communication flows and the
power of Western media corporations both internationally and
domestically.

Challenges to the hegemony of modernization theory came from several
sources. The first came from within the rich capitalist countries, where
labor and racial conditions contributed to protests and strikes and the
rise of new social movements, which were further fuelled by
international events such as the Vietnam War. These circumstances
brought about profound changes in academia as well, with greater
emphasis applied to the social relevance of research, to political
economy, and to critique of the hegemonic status of logical
positivism.\footnote{Nordenstreng, ``Ferment in the Field,'' 7.} While
we cannot speak of an institutionalization of the critical paradigm
within the US academy, the expansion and professionalization of the
field, which allowed for some degree of autonomy from the government,
did open a space for critical and even radical voices, like Herb
Schiller, the most prominent author of the cultural imperialism cluster.
In Europe we do see steps toward institutionalizing the critical
paradigm, particularly in the United Kingdom. The legacy of British
colonialism became an object of the emerging paradigm, which nurtured a
critical imperative based on historical criticism and personified by
authors such as Stuart Hall, Raymond Williams, and Richard Hoggart; and
by institutions such as the Centre for Contemporary Cultural Studies at
the University of Birmingham and the Glasgow Media Group; but also by
newly founded journals such as \emph{Screen} and \emph{Media},
\emph{Culture \& Society}, which are part of our sample. In our
analysis, scholars from the cultural studies (Stuart Hall) and political
economy of media (Graham Murdock) traditions emerge as prominent names
in the cultural imperialism cluster.

The second challenge to modernization theory resulted from the
internationalization of communication studies, as authors from the
periphery of global capitalism came to prominence. The name of Armand
Mattelart, who was active in Chile until the coup of 1973, emerges as
one of the prominent names in the cultural imperialism cluster. While
Mattelart moved to France after the coup, his research interests
continued to be shaped by the Chilean anti-imperialist experience,
focusing on the global dominance of the US and its
multinationals.\footnote{Armand Mattelart, ``Cultural Imperialism in the
  Multinationals' Age,'' \emph{Instant Research on Peace and Violence}
  6, no. 4 (1976); Armand Mattelart, \emph{Multinational Corporations
  and the Control of Culture: The Ideological Apparatuses of
  Imperialism} (Brighton, UK: Harvester, 1979).}

The third challenge arose from non-aligned international efforts which
sought to oppose ``cultural imperialism'' by establishing a New World
Information and Communication Order.\footnote{See Tran Van Dinh,
  ``Non-Alignment and Cultural Imperialism: The Black Scholar,''
  \emph{Journal of Black Studies and Research} 8, no. 3 (1976).} Several
authors in the cultural imperialism cluster were either directly
involved in the NWICO initiative (Kaarle Nordenstreng, primarily as
president of the International Organization of Journalists) or have
explicitly supported the initiative like Herb Schiller did. UNESCO
proved to be a fertile ground for the growth of the critical paradigm,
as it provided a site for critically reflecting on the national and
international interests of ``developed'' (e.g., the member states of the
Trilateral Commission) versus ``less developed'' countries (e.g., the
Non-Aligned Movement {[}NAM{]}).\footnote{Smythe and Van Dinh,
  ``Critical and Administrative Research,'' 125.} Through the UNESCO
critical tradition, which addressed issues of inequality in information
flows, the concepts of ``development'' and ``modernization'' were
challenged as ways to cover up problems rooted in
colonialism.\footnote{Sunril Amrith and Glenda Sluga, ``New Histories of
  the United Nations,'' \emph{Journal of World History} 19, no. 3
  (2008): 252.}

The connections between critical scholarship and the NWICO initiative
can be seen in their shared vocabulary---for example, the NAM's use of
``imperialism'' to describe ``alien ideological domination over the
peoples of the developing world,''\footnote{Documents of the Fourth
  Summit Conference of Heads of State or Government of the Non-Aligned
  Movement (September 1973).} with imperialism and cultural domination
also being some of the most prominent keywords of the cultural
imperialism cluster. It is also notable that the initial surge of the
cultural imperialism cluster in our sample (from 1974 to 1986) roughly
corresponds to the era of NWICO: from 1973, when its agenda was first
coherently articulated at the Fourth Summit of the Non-Aligned Movemtent
in Algiers, to 1985, when it was crippled by the US, UK, and Singapore
leaving UNESCO in protest against it.

\hypertarget{the-era-of-the-digital-divide-20002018}{%
\subsection{The Era of the Digital Divide
(2000--2018)}\label{the-era-of-the-digital-divide-20002018}}

The final era, which we call the ``era of the digital divide,'' begins
around the turn of the millennium with the emergence of the term digital
divide, which becomes the most commonly used term to describe
communication inequality in the last years of this era. In 2000 we also
saw the cultural imperialism cluster reach its peak and then quickly
decline, while still remaining relevant and even increasing in
prominence again towards the end of the analyzed period. Throughout this
era, the knowledge gap cluster does not exhibit a trend, but remains
consistently relevant. Modernization, on the other hand, has now almost
completely disappeared as a way to make sense of communication
inequality. It exhibits a short resurgence in 2010, but this is due to
interest in the history of the concept rather than proof of its
continuing relevance.\footnote{Thomas L. Jacobson, ``Amartya Sen's
  Capabilities Approach and Communication for Development and Social
  Change,'' \emph{Journal of Communication} 66, no. 5 (2016); Angel
  Barbas and John Postill, ``Communication Activism as a School of
  Politics: Lessons from Spain's Indignados Movement,''~\emph{Journal of
  Communication} 67, no. 5 (2017); Evan Elkins, ``Powered by Netflix:
  Speed Test Services and Video-On-Demand's Global Development
  Projects,'' \emph{Media, Culture \& Society} 40, no. 6 (2018).}
  
\begin{figure}
    \centering
    \includegraphics{mance-figure-six.png}
    \caption{Era of the digital divide visualized with reduced networks of
concepts (red) and cited authors (black) irrespective of journal
affiliation, 2000--2018; n of articles = 576; min. degree = 1,284; nodes
visible = 30 (0.22\%).}
    \label{fig:six}
\end{figure}

In this era the digital divide cluster is most closely associated with
the knowledge gap cluster. This finding is in line with reviews of the
literature on the digital divide, which point to the knowledge gap
tradition as its precursor,\footnote{Matthew S. Eastin, Vincent
  Cicchirillo, and Amanda Mabry, ``Extending the Digital Divide
  Conversation: Examining the Knowledge Gap Through Media
  Expectancies,'' \emph{Journal of Broadcasting \& Electronic Media} 59,
  no. 3 (2015): 419.} while the origin of the term is most often traced
back to the National Telecommunications and Information Administration
of the US Department of Commerce and its concern for providing
``universal service'' to US citizens. According to this origin, the
major focus of digital divide scholarship has been on unequal access to
the internet, with less attention on unequal digital skills and outcomes
of internet use.\textsuperscript{70} The
focus of digital divide research has been\marginnote{\textsuperscript{70}\setcounter{footnote}{70} Anique Scheerder, Alexander van Deursen, and
  Jan van Dijk, ``Determinants of Internet skills, Uses and Outcomes: A
  Systematic Review of the Second- and Third-level Digital Divide,''
  \emph{Telematics and Informatics} 34, no. 8 (2017): 1608­--09.} primarily on individual
characteristics, as ``the most common determinants studied across all
digital divides are sociodemographic and socioeconomic.''\footnote{Scheerder,
  van Deursen, and van Dijk, ``Determinants,'' 1614.} Structural factors
shaping the digital divide, like ``social and cultural differentiation
or individualization, rising income differentials, privatization and
cutbacks in social and public services,''\footnote{Jan Van Dijk and
  Kenneth Hacker, ``The Digital Divide as a Complex and Dynamic
  Phenomenon,'' \emph{The Information Society} 19, no. 4 (2003): 324.}
have been receiving much less attention. Similarly, political-economic
analyses focusing on the ways that the privatized, commercialized, and
monopolized nature of the internet is shaping communication inequalities
are quite rare in this tradition.\footnote{Ilse Mariën and Jernej Amon
  Prodnik, ``Digital Inclusion and User (Dis)empowerment: A Critical
  Perspective,'' \emph{info} 16, no. 6 (2014).}

These findings point toward associating the term digital divide with the
administrative paradigm. The ongoing prominence of the cultural
imperialism cluster likewise hints at the fact that critical scholars
have not found in the digital divide a neutral term that can be used to
describe aspects of global communication inequalities, suggesting that
the term digital divide remains slanted towards analyses that do not
take into account the structural factors shaping communication
inequality, which have traditionally been the focus of the political
economy of media approach. Reviewing several early monographs on the
digital divide, Graham Murdock points out that they do not pay ``much
attention to corporations as key actors in shaping e-Societies, which is
odd since there is plenty of evidence to show that their concerted
lobbying, extensive public relations activities and the well-oiled
revolving doors connecting cabinet rooms to boardrooms have moved them
to the center of policy formation.''\footnote{Graham Murdock, ``Review
  Article: Debating Digital Divides,'' \emph{European Journal of
  Communication} 17, no. 3 (2002): 389.} Other critical scholars claim
that the concept of the digital divide carries a ``new moral authority
which legitimates intervention in the affairs of places which are deemed
to be on the wrong side of the divide,''\footnote{Veva Leye, ``UNESCO,
  ICT Corporations and the Passion of ICT for Development: Modernization
  Resurrected,'' \emph{Media, Culture \& Society} 29, no. 6 (2007): 979.}
or believe it to be closely aligned with modernization
theory.\footnote{Elkins, "Powered by Netflix"; Leye, "UNESCO."}

It is therefore unclear whether---given its still increasing
popularity---the term digital divide will come to encompass all aspects
of communication inequality and be adopted by various scholarly
communities, or whether these communities will remain divided and the
study of communication inequality continue to be fragmented, both along
normative/ideological lines and along different aspects of the problem
(most notably between interpersonal and intergroup differences,
international communication inequality, and structural aspects like
commodification).

To further shed light on these questions, we analyze how the term
digital divide is represented in the different analyzed scholarly
journals, mirroring the debates of different scholarly communities.

\hypertarget{digital-divide-a-shared-term-dividing-scholarly-communities}{%
\subsection{Digital Divide: A Shared Term
Dividing Scholarly
Communities}\label{digital-divide-a-shared-term-dividing-scholarly-communities}}

Further introspection on the conceptual differences in adopting the
concept of digital divide within particular research communities is
offered in analyses performed on specific journals where the differences
are most evident (Figures 7--9, Appendices A--E).

The notion of the digital divide has been taken up differently by
different communities in communication research, with the differences
being most evident in three journals (if we leave out of this analysis
perhaps the most specific discourse analytic approach contained in
\emph{Discourse \& Society}, Appendix E). While digital divide functions
as the most prominent term in the European critical journal \emph{Media,
Culture \& Society} (Figure 7) and gains roughly the same prominence as
the knowledge gap in the \emph{Journal of Communication} (Figure 9), its
prominence is relatively low compared to the term knowledge gap in
\emph{Public Opinion Quarterly} (Figure 8). Based on the cited authors
most associated with the concept, we find that there is almost no
overlap between the three journals, suggesting that the research is
unrelated and therefore differently conceptualized.

In fifty-nine articles published in \emph{Media, Culture \& Society}
(6.5\% of all articles published in the journal within the period), the
concept is theoretically grounded in the traditions of the critical
paradigm, namely the research on theories of democracy and the public
sphere, French structuralism, and political economy of media.

\begin{figure}
    \centering
    \includegraphics{mance-figure-seven.png}
    \caption{Reduced network of keywords (red) and cited authors (black) in
articles sampled in \emph{Media, Culture \& Society;} min. degree = 614;
nodes visible = 26 (0.39\%).}
    \label{fig:seven}
\end{figure}


On the contrary, seldom occurrences in eight (1.2\% of all articles
published in the journal within the period) articles published in
\emph{Public Opinion Quarterly} and the pattern of cited authors point
to the absence of theoretical reflections on the concept, since the
majority of authors with the strongest links are affiliated with the
field of social psychology and/or methodologies of social surveys.
Digital divide thus appears in the context of measuring the impact of
the internet on civic engagement,\footnote{Kent M. Jennings and Vicki
  Zeitner, ``Internet Use and Civic Engagement: A Longitudinal
  Analysis,'' \emph{Public Opinion Quarterly} 67, no. 3 (2003).}
development of web surveys as investigative tools,\footnote{Mick P.
  Couper, ``Web Surveys: A Review of Issues and Approaches,''
  \emph{Public Opinion Quarterly} 64, no. 4 (2000).} or comparisons of
web and telephone surveys.\footnote{Scott Fricker, Mirta Galesic, Roger
  Tourangeau, and Ting Yan, ``An Experimental Comparison of Web and
  Telephone Surveys,'' \emph{Public Opinion Quarterly} 69, no. 3 (2005).}
  
\begin{figure}
    \centering
    \includegraphics{mance-figure-eight.png}
    \caption{\emph{Theoretical Perspectives of Information Inequality.}
Source: Liangzhi Yu, ``Understanding Information Inequality: Making
Sense of the Literature of the Information and Digital
Divides,''~\emph{Journal of Librarianship and Information Science} 38,
no. 4 (2006): 229--52, \url{https://doi.org/10.1177/0961000606070600}.}
    \label{fig:eight}
\end{figure}


The third distinctive conceptualization of the digital divide,
identified in forty-three articles (5.5\% of all articles published in
the journal within the period) from the \emph{Journal of Communication},
stemming from the works of Eszter Hargittai (19) and strongly associated
with political science- and/or political communication-affiliated
researchers. The relative closeness between the concepts of the digital
divide and the knowledge gap (Figure 9) argues for the importance of the
``second level'' conceptualization of the digital divide,\footnote{Eszter
  Hargittai, ``Second-level Digital Divide: Differences in People's
  Online Skills,'' \emph{First Monday} 74 (2002).} which is grounded in
the arguments that the binarization of inequality into haves and
have-nots has been oversimplified and traditional socioeconomic and
demographic indicators are of insufficient explanatory power.\footnote{Margaret
  Richardson, C. Kay Weaver, and Theodore E. Zorn, ``\,`Getting on':
  Older New Zealanders' Perceptions of Computing,'' \emph{New Media \&
  Society} 7, no. 2 (2005).} Research on the digital divide, which falls
into this second category, uses the segmentation of social classes along
their identified ability to communicate digitally, and shifts the
analytical focus from a critical consideration of structural social
inequalities to the individual, giving the concept a strong resemblance
to ``knowledge divide'' theory.\footnote{Eastin, Cicchirillo, and Mabry,
  ``Extending the Digital Divide''; Colin Sparks, ``What is the `Digital
  Divide' and Why is it Important?'' \emph{Javnost---The Public} 20, no.
  2 (2013); Sharon Strover, ``The US Digital Divide: A Call for a New
  Philosophy,'' \emph{Critical Studies in Media Communication} 31, no. 2
  (2014).}
  
  \begin{figure}
    \centering
    \includegraphics{mance-figure-nine.png}
    \caption{Reduced network of keywords (red) and cited authors (black) in
articles sampled in \emph{Journal of Communication;} min. degree = 649;
nodes visible = 31 (0.47\%).}
    \label{fig:nine}
\end{figure}


\hypertarget{summary-and-discussion}{%
\section{Summary and Discussion}\label{summary-and-discussion}}

We have identified four distinct clusters of conceptualizations of
communication inequality. The first to emerge was modernization theory,
which was closely tied to US foreign policy interests in influencing the
hearts and minds of peoples in the third world. Modernization theory was
part of the administrative paradigm, sharing its conceptualization of
communication as a tool of power and its research focus on media
effects. Theories of cultural imperialism emerged in the late 1960s to
directly challenge modernization theory and were tied to changes in
global geopolitics (crisis of US hegemony, NWICO), domestic politics in
the US and Europe (social movements, labor militancy), the
institutionalization of the critical paradigm, as well as the
internationalization of communication studies. The antagonism between
the two was constitutive for the critique of cultural imperialism and
created a relationship of ``dialogical-dialectical
coherence.''\footnote{Robert T. Craig, ``Communication Theory as a
  Field,'' \emph{Communication Theory} 9, no. 2 (1999): 123--24.}

Parallel to these two approaches, the knowledge gap tradition emerged
and began to consolidate in the 1970s. While modernization and cultural
imperialism focused on international communication inequality, the
knowledge gap tradition focused on the interpersonal differences in
acquiring and processing mass-mediated information, particularly the
mediating role of sociodemographic variables on the effectiveness of
targeted information campaigns. With its focus on media effects and the
effectiveness of communication campaigns, we can assign the knowledge
gap tradition to the administrative paradigm.

Finally, the notion of the digital divide emerged around the turn of the
millenium and quickly became the most frequently used concept referring
to communication inequality. However, the notion of the digital divide
was developed with little historical context and with only weak
references to previous theoretical and empirical research into the
concept of communication inequality. As the name already implies, the
digital divide seems to be a \emph{sui generis} phenomenon, not a
manifestation of communication inequality in digital information and
communication technologies.

The case of the digital divide illustrates how the dominant approaches
and paradigmatic shifts in conceptualizations of communication
inequality continue to be driven by developments outside of
communication studies and informed by theories from other fields, when
not lacking a theoretical basis altogether. Modernization grew out of US
interests in securing hegemony in the third world and was theoretically
informed largely by economic historians like Cyril Black and Walt Rostow
who advanced social evolutionist ideas.\footnote{Rist, \emph{History of
  Development}, 93--108.} Concepts of the knowledge, communication and
information gap have gained prominence as a way of explaining
sociodemographic barriers to the effectiveness of persuasive
communication. The challenge to the idea of modernization was driven by
third world opposition to US hegemony and was theoretically informed by
critical political-economic ideas (primarily dependency theory,
world-systems theory and the marxist notion of imperialism). The notion
of a digital divide emerged from the concern of the US Commerce
Department's National Telecommunications and Information Administration
(NTIA) for measuring to what degree the goal of ``universal service''
was being achieved by US telecommunications policy.\footnote{National
  Telecommunications and Information Administration (NTIA),
  \emph{Falling through the Net: A Survey of the ``Have Nots'' in Rural
  and Urban America} (Washington, DC: US Department of Commerce, July
  1995).}

This lack of a solid theoretical foundation (of integrative empirical
research and theory building) is not accidental; it is a direct
consequence of the instrumentalisation of the field of communication. As
research goals within communication science were largely determined by
the administrative interests of the US government, conceptualizations of
communication inequality have remained largely implicit and unexamined
because it was never in the interest of the holders of political and
economic power to have that very power questioned. Hence, the dominant
paradigm in communication naturalized the structural inequalities of a
society deeply divided along the lines of class, race, and gender, and
implicitly assumed that communication must be, by its nature, unequal
and imbalanced: originating from social elites and distributed to the
large mass of society, whose members are seen as passive recipients of
media messages and targets of media effects. The same implicit view
colored the understanding of international communication, where the
foreign policy goals of the US determined research efforts. In
modernization theory, the history of imperialism and colonialism was
entirely removed from view, and global inequalities between countries
were reframed as different stages of social evolution. Finally, digital
divide scholarship only rarely questions the way global
telecommunications policy---which has led to the internet developing as
a commodified, commercialized, and increasingly monopolized system---has
reinforced and reshaped old inequalities and created new ones in the
process.




\section{Bibliography}\label{bibliography}

\begin{hangparas}{.25in}{1} 



Amrith, Sunril, and Glenda Sluga. ``New Histories of the United
Nations.'' \emph{Journal of World History} 19, no. 3 (2008): 251--74.
\url{http://www.jstor.org/stable/40542615}.

Barbas, Angel, and John Postill. ``Communication Activism as a School of
Politics: Lessons from Spain's Indignados Movement.''~\emph{Journal of
Communication} 67, no. 5 (2017): 646--64.
\url{https://doi.org/10.1111/jcom.12321}.

Blondel, Vincent D., Jean-Loup Guillaume, Renaud Lambiotte, and Etienne
Lefebvre. ``Fast Unfolding of Communities in Large Networks.''
\emph{Journal of Statistical Mechanics: Theory and Experiment,} no. 1
(2008). \url{https://arxiv.org/pdf/0803.0476.pdf}.

Brenner, Robert. \emph{Economics of Global Turbulence: The Advanced
Capitalist Economies from Long Boom to Long Downturn, 1945--2005}.
London: Verso, 2006.

Boltanski, Luc, and Eve Chiapello. \emph{The New Spirit of Capitalism}.
London: Verso, 2007.

Couper, Mick P. ``Web Surveys: A Review of Issues and Approaches.''
\emph{Public Opinion Quarterly} 64, no. 4 (2000): 464--94.
\url{https://doi.org/10.1086/318641}.

Craig, Robert T. ``Communication Theory as a Field.''
\emph{Communication Theory} 9, no. 2 (1999): 119--61.
\url{https://doi.org/10.1111/j.1468-2885.1999.tb00355.x}.

Documents of the Fourth Summit Conference of Heads of State or
Government of the Non-Aligned Movement (Algiers, Algeria, September
5--9, 1973).
\url{http://cns.miis.edu/nam/documents/Official_Document/4th_Summit_FD_Algiers_Declaration_1973_Whole.pdf}.

Donsbach, Wolfgang. \emph{The International Encyclopedia of
Communication}. Malden, MA: Blackwell Publishing, 2008.

Dorfman, Ariel, and Armand Mattelart. \emph{How to Read Donald Duck}.
1971. Reprint, New York: OR Books, 2018.

Downey, John, and Jason Toynbee. ``Ideology: Towards Renewal of a
Critical Concept.'' \emph{Media, Culture \& Society 38,} no. 8 (2016):
1261--71.

Eastin, Matthew S., Vincent Cicchirillo, and Amanda Mabry. ``Extending
the Digital Divide Conversation: Examining the Knowledge Gap Through
Media Expectancies.'' \emph{Journal of Broadcasting \& Electronic Media}
59, no. 3 (2015): 416--37.
\url{https://doi.org/10.1080/08838151.2015.1054994}.

Elkins, Evan. ``Powered by Netflix: Speed Test Services and
Video-On-Demand's Global Development Projects.'' \emph{Media, Culture \&
Society} 40, no. 6 (2018): 838--55.
\url{https://doi.org/10.1177/0163443718754649}.

Fanon, Frantz. \emph{Wretched of the Earth}. Translated by Constance
Farrington. New York: Grove Press, 1963.

Fenton, Natalie. ``Fake Democracy: The Limits of Public Sphere Theory.''
\emph{Javnost---The Public} 25, no. 1--2 (2018): 28--34.

Fricker, Scott, Mirta Galesic, Roger Tourangeau, and Ting Yan. ``An
Experimental Comparison of Web and Telephone Surveys\emph{.}''
\emph{Public Opinion Quarterly} 69, no. 3 (2005): 370--92.
\url{https://doi.org/10.1093/poq/nfi027}.

Gary, Brett. ``Communication Research, the Rockefeller Foundation, and
Mobilization for the War on Words, 1938--1944.'' \emph{Journal of
Communication} 46, no. 3 (1996): 124--48.
\url{https://doi.org/10.1111/j.1460-2466.1996.tb01493.x}.

Gitlin, Todd. ``Media Sociology: The Dominant Paradigm.'' \emph{Theory
and Society} 6, no. 2 (1978): 205--53.

Glander, Timothy. \emph{Origins of Mass Communications Research During
the American Cold War: Educational Effects and Contemporary
Implications}. Mahwah, NJ: Lawrence Erlbaum Associates, 2000.

Glock, Charles Y. ``The Comparative Study of Communications and Opinion
Formation.'' \emph{Public Opinion Quarterly} 16, no. 4 (1952): 512--23.

Golding, Peter.~``Media Role in National Development.'' \emph{Journal of
Communication} 24, no.3 (1974):
39--53\emph{.}~\url{https://doi.org/10.1111/j.1460-2466.1974.tb00387.x}.

Hargittai, Eszter. ``Second-level Digital Divide: Differences in
People's Online Skills.'' \emph{First Monday} 74 (2002).
\url{https://firstmonday.org/ojs/index.php/fm/article/view/942}.

Hamilton, Mark A., and Kristine L. Nowak. ``Information Systems Concepts
across Two Decades: An Empirical Analysis of Trends in Theory, Methods,
Process, and Research Domains.'' \emph{Journal of Communication} 55, no.
3 (2005): 529--53.
\url{https://doi.org/10.1111/j.1460-2466.2005.tb02684.x}.

Jacobson, Thomas L. ``Amartya Sen's Capabilities Approach and
Communication for Development and Social Change.'' \emph{Journal of
Communication} 66, no. 5
(2016):~789--810.~\url{https://doi.org/10.1111/jcom.12252}.

Jacomy, Mathieu, Tommaso Venturini, Sebastien Heymann, and Mathieu
Bastian. ``ForceAtlas2: A Continuous Graph Layout Algorithm for Handy
Network Visualization Designed for the Gephi Software." \emph{PLoS ONE}
9, no. 6 (2014): e98679.
\url{https://doi.org/10.1371/journal.pone.0098679}.

Jennings, Kent M., and Vicki Zeitner. ``Internet Use and Civic
Engagement: A Longitudinal Analysis.'' \emph{Public Opinion Quarterly}
67, no. 3 (2003): 311--34. \url{https://doi.org/10.1086/376947}.

Kay, Cristóbal. \emph{Latin American Theories of Development and
Underdevelopment.} London: Routledge, 1989.

Kincheloe, Joe L., and Peter L. McLaren. ``Rethinking Critical Theory
and Qualitative Research.'' In \emph{Landscape of Qualitative Research:
Theories and Issues}, edited by Norman K. Denzin and Yvonna S. Lincoln,
237\emph{--}67. London: SAGE, 1998.

Klapper, Joseph T., and Leo Löwenthal. ``The Contributions of Opinion
Research to the Evaluation of Psychological Warfare.'' \emph{Public
Opinion Quarterly} 15, no. 4 (1951): 651--62.
\url{http://www.jstor.org/stable/2745953}.

Knobloch-Westerwick, Silvia, and Carroll J. Glynn. ``The Matilda
Effect---Role Congruity Effects on Scholarly Communication: A Citation
Analysis of Communication Research and Journal of Communication
Articles.'' \emph{Communication Research} 40, no. 1 (2013): 3--26.
\url{https://doi.org/10.1177/0093650211418339}.

Koivisto, Juha. \emph{Mapping Communication and Media Research}.
Tampere, Finland: Tampere University Press, 2010.
\url{https://trepo.tuni.fi/bitstream/handle/10024/66336/mapping_communication_2010.pdf?sequence=1}.

Lambiotte, Renaud, Jean-Charles Delvenne, and Mauricio Barahona.
``Random Walks, Markov Processes and the Multiscale Modular Organization
of Complex Networks.'' \emph{IEEE Transactions on Network Science and
Engineering} 1, no. 2 (2015): 76--90\emph{.}
\url{https://doi.org/10.1109/tnse.2015.2391998}.

Lin, Julian, and Lee Seow Ting. ``Mapping Twelve Years of Communication
Scholarship: Themes and Concepts in the Journal of Communication.'' In
\emph{The Outreach of Digital Libraries: A Globalized Resource Network}.
Lecture Notes in Computer Science 7634. Berlin: Springer, 2012.
\url{https://doi.org/10.1007/978-3-642-34752-8_53}.

Lee, Chul-joo, and Dongyoung Sohn. ``Mapping the Social Capital Research
in Communication.'' \emph{Journalism \& Mass Communication Quarterly}
93, no. 4 (2016):
728--49. \url{https://doi.org/10.1177/1077699015610074}.

Lerner, Daniel. ``International Coalitions and Communications Content:
The Case of Neutralism.''\emph{ Public Opinion Quarterly} 16, no. 4
(1952): 681­--Lerner, Daniel. \emph{The Passing of Traditional Society:
Modernizing the Middle East}. New York: Free Press of Glencoe, 1958.

Leye, Veva. ``UNESCO, ICT Corporations and the Passion of ICT for
Development: Modernization Resurrected.'' \emph{Media, Culture \&
Society} 29, no. 6 (2007): 972--93.
\url{https://doi.org/10.1177/0163443707081711}.

Mance, Boris. ``The Changing Role and Patterns of Critical Communication
Scholarship in the Academic Journal Publishing System.'' Doctoral
thesis, University of Ljubljana, 2020.

Mariën, Ilse, and Jernej Amon Prodnik. ``Digital Inclusion and User
(Dis)empowerment: A Critical Perspective.'' \emph{info} 16, no. 6
(2014): 35--47. \url{https://doi.org/10.1108/info-07-2014-0030}.

Mattelart, Armand. ``Cultural Imperialism in the Multinationals' Age.''
\emph{Instant Research on Peace and Violence} 6, no. 4 (1976): 160--74.

Mattelart, Armand. \emph{Multinational Corporations and the Control of
Culture: The Ideological Apparatuses of Imperialism}. Brighton, UK:
Harvester, 1979.

Melody, William H., and Robin E. Mansell. ``The Debate over Critical vs.
Administrative Research: Circularity or Challenge.'' \emph{Journal of
Communication} 33, no. 3 (1983): 103--16.
\url{https://doi.org/10.1111/j.1460-2466.1983.tb02412.x}.

Merton, Robert K., and Norman William Storer. \emph{The Sociology of
Science: Theoretical and Empirical Investigations}. Chicago: University
of Chicago Press, 1998.

Murdock, Graham. ``Review Article: Debating Digital Divides.''
\emph{European Journal of Communication} 17, no. 3 (2002): 385--90.

National Telecommunications and Information Administration (NTIA).
\emph{Falling through the Net: A Survey of the "Have Nots'' in Rural and
Urban America.} Washington, DC: US Department of Commerce, July 1995.
\url{https://www.ntia.doc.gov/ntiahome/fallingthru.html}.

Nkrumah, Kwame. \emph{Neo-Colonialism: The Last Stage of Capitalism.}
London: Thomas Nelson \& Sons, 1965.

Olson, Robert K. \emph{US Foreign Policy and the New International
Economic Order: Negotiating Global Problems, 1974--1981}. Boulder, CO:
Westview Press, 1981.

Nordenstreng, Kaarle. ``Ferment in the Field: Notes on the Evolution of
Communication Studies and Its Disciplinary Nature.'' \emph{Javnost­---The
Public} 11, no. 3 (2004): 5--17.

Nordenstreng, Kaarle. ``Being (Truly) Critical in Media and
Communication Studies: Reflections of a Media Scholar between Science
and Politics.'' \emph{Javnost---The Public} 23, no. 1 (2016): 89--104.

Pooley, Jefferson. ``The New History of Mass Communication Research.''
In \emph{The History of Media and Communication Research: Contested
Memories}, edited by David Park and Jefferson Pooley, 43--69. New York:
Peter Lang, 2008.

Rauchfleisch, Adrian. ``The Public Sphere as an Essentially Contested
Concept: A Co-citation Analysis of the Last Twenty Years of Public
Sphere Research.'' \emph{Communication and the Public} 2, no. 1 (2017):
3--18. h\url{ttps://doi.org/10.1177/2057047317691054}.

Rice, Ronald E., John Chapin, Rebecca Pressman, Soyeon Park, and Edward
Funkhouser. ``What's in a Name? Bibliometric Analysis of Forty Years of
the \emph{Journal of Broadcasting} (\emph{\& Electronic Media}).''
\emph{Journal of Broadcasting \& Electronic Media} 40, no. 4 (1996):
511--39. \url{https://doi.org/10.1080/08838159609364373}.

Richardson, Margaret C., Kay Weaver, and Theodore E. Zorn. ``\,'Getting
on': Older New Zealanders' Perceptions of Computing.'' \emph{New Media
\& Society} 7, no. 2 (2005): 219--45.
\url{https://doi.org/10.1177/1461444805050763}.

Rist, Gilbert. \emph{The History of Development: From Western Origins to
Global Faith}. 3rd ed. London: Zed Books, 2008.

Rogers, Everett M. ``Communication and Development: The Passing of the
Dominant Paradigm.'' \emph{Communication Research} 3, no. 2 (1976):
213--40.

Schiller, Herbert I. \emph{Mass Communication and American Empire}.
Boston: Beacon Press, 1969.

Sauvant, Karl P. ``Toward the New International Economic Order.'' In
\emph{The New International Economic Order: Confrontation of Cooperation
between North and South?}, edited by Karl P. Sauvant and Hajo
Hasenpflug, 3--19. Boulder, CO: Westview Press, 1977.

Scheerder, Anique, Alexander van Deursen, and Jan van Dijk.
``Determinants of Internet Skills, Uses and Outcomes: A Systematic
Review of the Second- and Third-level Digital Divide.'' \emph{Telematics
and Informatics} 34, no. 8 (2017): 1607--24.

Shah, Hemant. \emph{The Production of Modernization: Daniel Lerner, Mass
Media, and the Passing of Traditional Society}. Philadelphia: Temple
University Press, 2011.

Simonson, Peter, and John Durham Peters. ``Communication and Media
Studies, History to 1968.'' In \emph{The International Encyclopedia of
Communication}, edited by Wolfgang Donsbach, 764--71. Malden, MA:
Wiley-Blackwell, 2008.

Simpson, Christopher. \emph{Science of Coercion: Communication Research
\& Psychological Warfare, 1945}--\emph{1960}. New York: Oxford
University Press, 1994.

Smythe, Dallas W., and Tran Van Dinh. ``On Critical and Administrative
Research: A New Critical Analysis.'' \emph{Journal of Communication} 33,
no. 3 (1983): 117--27.

Sparks, Colin. ``What is the `Digital Divide' and Why is It Important?''
\emph{Javnost---The Public} 20, no. 2 (2013): 27--46.
\url{https://doi.org/10.1080/13183222.2013.1100911}.

Splichal, Slavko. \emph{Public Opinion:} \emph{Developments and
Controversies in Twentieth Century.} Lanham, MD: Rowman \& Littlefield,
1999.

Splichal, Slavko, and Boris Mance. ``Paradigm(s) Lost? Islands of
Critical Media Research in Communication Journals.'' \emph{Journal of
Communication} 68, no. 2 (2018): 399--414.

Strover, Sharon. ``The US Digital Divide: A Call for a New Philosophy.''
\emph{Critical Studies in Media Communication} 31, no. 2 (2014):
114--22. \url{https://doi.org/10.1080/15295036.2014.922207}.

Tai, Zixue. ``The Structure of Knowledge and Dynamics of Scholarly
Communication in Agenda Setting Research, 1996--2005.'' \emph{Journal of
Communication} 59, no. 3 (2009): 481--513.
\url{https://doi.org/10.1111/j.1460-2466.2009.01425.x}.

Tichenor, Philip J., George A. Donahue, and Clarice N. Olien. ``Mass
Media Flow and Differential Growth in Knowledge.'' \emph{Public Opinion
Quarterly} 34, no. 2 (1970): 159--70.

Truman, Harry S. ``The Inaugural Address of Harry S. Truman.''
Transcript of inaugural address delivered in Washington, DC, January 20,
1949. \url{https://avalon.law.yale.edu/20th_century/truman.asp}.

Viswanath, Katam, and John R. Finnegan. ``The Knowledge Gap Hypothesis:
Twenty-Five Years Later.'' \emph{Annals of the International
Communication Association} 19, no. 1 (1996): 187--228.
\url{https://doi.org/10.1080/23808985.1996.11678931}.

Van Dijk, Jan, and Kenneth Hacker. ``The Digital Divide as a Complex and
Dynamic Phenomenon.'' \emph{The Information Society} 19, no. 4 (2003):
315--26. \url{https://doi.org/10.1080/01972240309487}.

Van Dinh, Tran. ``Non-Alignment and Cultural Imperialism: The Black
Scholar.'' \emph{Journal of Black Studies and Research} 8, no. 3 (1976):
39--49.

Williams, Raymond. \emph{Keywords: A Vocabulary of Culture and Society}.
1976. Reprint, New York: Oxford University Press, 2015.

Yu, Liangzhi. ``Understanding Information Inequality: Making Sense of
the Literature of the Information and Digital Divides.''~\emph{Journal
of Librarianship and Information Science} 38, no. 4 (2006): 229--52.
\url{https://doi.org/10.1177/0961000606070600}.

Yu, Liangzhi. ``The Divided Views of the Information and Digital
Divides: A Call for Integrative Theories of Information Inequality.''
\emph{Journal of Information Science} 37, no. 6 (2011): 660--79.
\url{https://doi.org/10.1177/0165551511426246}.

\vspace*{2em}

\hypertarget{disclosure-of-funding-sources}{%
\section{Disclosure of Funding
Sources}\label{disclosure-of-funding-sources}}

This work was conducted as part of the research project ``The Role of
Communication Inequalities in Disintegration of a Multinational
Society'' (J5-1793), funded by the Slovenian Research Agency (ARRS).

\newpage

\vspace*{4em}

\begin{center}
    

\noindent\huge{\textit{Appendices}}

\end{center}



\vspace*{3em}



\begin{figure}
    \centering
    \includegraphics{mance-figure-ten.png}
    \caption*{\textbf{Appendix A: }Reduced Network of Keywords (Red) and Cited Authors
(Black) in Articles Sampled in \emph{Journalism Quarterly;} min. degree
= 333; nodes visible = 31 (1.14\%)}
\end{figure}

\begin{figure}
    \centering
    \includegraphics{mance-figure-eleven.png}
    \caption*{\textbf{Appendix B:} Reduced Network of Keywords (Red) and Cited Authors
(Black) in Articles Sampled in \emph{Communication Monographs;} min.
degree = 117; nodes visible = 30 (3.44\%)}
    \label{fig:eleven}
\end{figure}


\begin{figure}
    \centering
    \includegraphics{mance-figure-twelve.png}
    \caption*{\textbf{Appendix C:} Reduced Network of Keywords (Red) and Cited Authors
(Black) in Articles Sampled in \emph{Communication Research;} min.
degree = 345; nodes visible = 30 (1.34\%)
}
    \label{fig:twelve}
\end{figure}


\begin{figure}
    \centering
    \includegraphics{mance-figure-thirteen.png}
    \caption*{\textbf{Appendix D:} Reduced Network of Keywords (Red) and Cited Authors
(Black) in Articles Sampled in \emph{European Journal of Communication;}
min. degree = 259; nodes visible = 86 (2.33\%)}
    \label{fig:thirteen}
\end{figure}


\begin{figure}
    \centering
    \includegraphics{mance-figure-fourteen.png}
    \caption*{\textbf{Appendix E:} Reduced Network of Keywords (Red) and Cited Authors
(Black) in Articles Sampled in \emph{Discourse \& Society;} min. degree
= 97; nodes visible = 32 (5.33\%)}
    \label{fig:fourteen}
\end{figure}



\newpage

\begin{fullwidth}


\textbf{Appendix F:} List of Sampled Journals with Publishing Years,
Number of Sampled Articles, and the Country of Origin

\vspace*{2em}


\tabulinesep=1.9mm
{\begin{longtabu} to 1.55\textwidth { X[l] X[c] X[c] X[c]} 
\emph{Journal name (Period)} & \emph{Founding year} & \emph{No. of articles in sample} & \emph{Country of origin}\\
\endfirsthead
\emph{Journal name (Period)} & \emph{Founding year} & \emph{No. of articles in sample} & \emph{Country of origin}\\
\endhead
JOURNALISM QUARTERLY (1924--1994) / JOURNALISM \& MASS COMMUNICATION
QUARTERLY (1995--) & 1924 & 4,322 & US \\
SPEECH MONOGRAPHS (1934--1974)/ COMMUNICATION MONOGRAPHS (1975--) & 1934
& 1,691 & US \\
PUBLIC OPINION QUARTERLY & 1936 & 2,499 & US \\
JOURNAL OF COMMUNICATION & 1950 & 2,403 & US \\
COMMUNICATION RESEARCH & 1973 & 1,346 & US \\
MEDIA, CULTURE \& SOCIETY & 1978 & 1,497 & UK \\
EUROPEAN JOURNAL OF COMMUNICATION & 1986 & 724 & UK \\
DISCOURSE \& SOCIETY & 1989 & 756 & Netherlands \\
\end{longtabu}}


\newpage

\textbf{Appendix G:} List of Noun Phrases Detected in CiteSpace Software
Package

\vspace*{1em}


\tabulinesep=1.1mm
{\begin{longtabu} to 1.25\textwidth { X[l] X[c] X[c]} 
\emph{No.} & \emph{Count} & \emph{Keyword} \\
\endfirsthead
\emph{No.} & \emph{Count} & \emph{Keyword} \\
\endhead
1 & 872 & ­­social media \\
2 & 589 & digital divide \\
3 & 318 & internet \\
4 & 276 & communication technology \\
5 & 265 & content analysis \\
6 & 214 & digital media \\
7 & 211 & social network \\
8 & 207 & journalism \\
9 & 190 & new media \\
10 & 183 & media \\
11 & 177 & internet access \\
12 & 168 & news media \\
13 & 167 & in-depth interviews \\
14 & 164 & mass media \\
15 & 153 & young people \\
16 & 145 & twitter \\
17 & 144 & communication \\
18 & 138 & health information \\
19 & 128 & political communication \\
20 & 125 & mobile phones \\
21 & 125 & new technology \\
22 & 121 & facebook \\
23 & 120 & public sphere \\
24 & 113 & public relation \\
25 & 108 & digital technology \\
26 & 107 & social networks \\
27 & 103 & public relations \\
28 & 102 & developing countries \\
29 & 95 & china \\
30 & 93 & traditional media \\
31 & 93 & gender \\
32 & 92 & practical implications \\
33 & 92 & internet user \\
34 & 89 & universal service \\
35 & 85 & everyday life \\
36 & 85 & political participation \\
37 & 85 & media literacy \\
38 & 84 & digital inequality \\
39 & 81 & television \\
40 & 81 & social media platforms \\
41 & 81 & previous research \\
42 & 79 & media system \\
43 & 78 & mobile devices \\
44 & 76 & covid-19 \\
45 & 72 & knowledge gap \\
46 & 70 & media coverage \\
47 & 68 & social capital \\
48 & 68 & digital age \\
49 & 68 & globalization \\
50 & 67 & digital inclusion \\
51 & 66 & youth \\
52 & 65 & focus group \\
53 & 64 & news \\
54 & 62 & public opinion \\
55 & 61 & case study \\
56 & 61 & big data \\
57 & 60 & health communication \\
58 & 58 & covid-19 pandemic \\
59 & 57 & democracy \\
60 & 57 & south africa \\
61 & 55 & theoretical framework \\
62 & 55 & online survey \\
63 & 54 & information technology \\
64 & 54 & social change \\
65 & 53 & european union \\
66 & 53 & broadband \\
67 & 53 & privacy \\
68 & 51 & science communication \\
69 & 51 & education \\
70 & 51 & political knowledge \\
71 & 50 & latin america \\
72 & 49 & social movements \\
73 & 49 & technology \\
74 & 46 & mobile communication \\
75 & 45 & online news \\
76 & 44 & mainstream media \\
77 & 44 & news consumption \\
78 & 44 & civil society \\
79 & 43 & research question \\
80 & 43 & digital platforms \\
81 & 43 & participation \\
82 & 43 & climate change \\
83 & 43 & fake news \\
84 & 42 & crisis communication \\
85 & 41 & digital literacy \\
86 & 41 & social networking site \\
87 & 40 & qualitative analysis \\
88 & 40 & ict \\
89 & 40 & mobile phone \\
90 & 39 & news coverage \\
91 & 39 & political information \\
92 & 39 & political economy \\
93 & 39 & media education \\
94 & 39 & online media \\
95 & 38 & icts \\
96 & 38 & spain \\
97 & 38 & information society \\
98 & 37 & information flow \\
99 & 36 & survey data \\
100 & 35 & new forms \\
101 & 35 & rural area \\
102 & 34 & mobile technology \\
103 & 34 & literature review \\
104 & 33 & public policy \\
105 & 33 & social interaction \\
106 & 33 & framing \\
107 & 32 & telecommunications \\
108 & 32 & news production \\
109 & 32 & representative sample \\
110 & 31 & research gap \\
111 & 31 & social support \\
112 & 31 & previous study \\
113 & 30 & significant difference \\
114 & 30 & semi-structured interviews \\
115 & 29 & discourse analysis \\
116 & 29 & open access \\
117 & 29 & interpersonal communication \\
118 & 29 & internet use \\
119 & 29 & news stories \\
120 & 28 & wide web \\
121 & 28 & disinformation \\
122 & 28 & global south \\
123 & 27 & identity \\
124 & 27 & social networking \\
125 & 27 & digital communication \\
126 & 27 & youtube \\
127 & 26 & cultural imperialism \\
128 & 26 & digital inequalities \\
129 & 26 & inequality \\
130 & 26 & mobile media \\
131 & 26 & thematic analysis \\
132 & 25 & universal access \\
133 & 24 & india \\
134 & 24 & information flows \\
135 & 24 & policy \\
136 & 23 & qualitative research \\
137 & 23 & communications technology \\
138 & 23 & computer-mediated communication \\
139 & 23 & media content \\
140 & 22 & social network site \\
141 & 22 & coronavirus \\
142 & 22 & transparency \\
143 & 22 & health literacy \\
144 & 22 & comparative analysis \\
145 & 22 & media effects \\
146 & 21 & children \\
147 & 21 & university students \\
148 & 21 & ethnography \\
149 & 21 & africa \\
150 & 21 & digital journalism \\
151 & 20 & empirical research \\
152 & 20 & south korea \\
153 & 20 & ethics \\
154 & 20 & surveillance \\
155 & 19 & 21st century \\
156 & 19 & newspapers \\
157 & 19 & regulation \\
158 & 18 & social networking sites \\
159 & 18 & digital skills \\
160 & 18 & artificial intelligence \\
161 & 18 & socioeconomic status \\
162 & 17 & race \\
163 & 17 & access \\
164 & 17 & young adults \\
165 & 17 & infrastructure \\
166 & 17 & empirical analysis \\
167 & 16 & general public \\
168 & 16 & rural communities \\
169 & 16 & new ways \\
170 & 16 & information \\
171 & 16 & critical discourse analysis \\
172 & 16 & communication theory \\
173 & 16 & news source \\
174 & 16 & politics \\
175 & 16 & algorithms \\
176 & 16 & health care \\
177 & 15 & public access \\
178 & 15 & survey \\
179 & 15 & online communication \\
180 & 15 & limited access \\
181 & 15 & media policy \\
182 & 14 & main objective \\
183 & 14 & crucial role \\
184 & 14 & personal information \\
185 & 14 & conceptual framework \\
186 & 14 & everyday lives \\
187 & 14 & civic engagement \\
188 & 14 & video games \\
189 & 14 & new zealand \\
190 & 14 & web 2.0 \\
191 & 14 & advertising \\
192 & 14 & social inclusion \\
193 & 14 & mobile internet \\
194 & 13 & digital environment \\
195 & 13 & daily lives \\
196 & 13 & web site \\
197 & 13 & empirical evidence \\
198 & 13 & journalism study \\
199 & 13 & new challenge \\
200 & 13 & virtual reality \\
201 & 13 & european countries \\
202 & 13 & growing body \\
203 & 12 & qualitative interviews \\
204 & 12 & media environment \\
205 & 12 & significant role \\
206 & 12 & mass communication \\
207 & 12 & activism \\
208 & 12 & search engine \\
209 & 12 & new opportunities \\
210 & 12 & digital era \\
211 & 12 & network analysis \\
212 & 12 & political engagement \\
213 & 12 & knowledge gaps \\
214 & 12 & migration \\
215 & 12 & access to information \\
216 & 12 & international communication \\
217 & 12 & national trends survey \\
218 & 11 & knowledge gap hypothesis \\
219 & 11 & online activity \\
220 & 11 & mobile communications \\
221 & 11 & strategic communication \\
222 & 11 & citizen journalism \\
223 & 11 & citizenship \\
224 & 11 & pandemics \\
225 & 11 & audiences \\
226 & 11 & convergence \\
227 & 11 & election campaigns \\
228 & 11 & digital \\
229 & 10 & mental health \\
230 & 10 & telecommunications policy \\
231 & 10 & power relation \\
232 & 10 & online communities \\
233 & 10 & public relations practitioners \\
234 & 10 & macbride report \\
235 & 10 & media representation \\
236 & 10 & television news \\
237 & 10 & international news \\
238 & 10 & existing literature \\
239 & 10 & significant impact \\
240 & 10 & virtual environment \\
241 & 10 & culture \\
242 & 10 & business model \\
243 & 10 & economic development \\
244 & 10 & political actors \\
245 & 10 & college students \\
\end{longtabu}}


\end{fullwidth}


\end{hangparas}


\end{document}