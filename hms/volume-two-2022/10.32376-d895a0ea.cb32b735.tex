% see the original template for more detail about bibliography, tables, etc: https://www.overleaf.com/latex/templates/handout-design-inspired-by-edward-tufte/dtsbhhkvghzz

\documentclass{tufte-handout}

%\geometry{showframe}% for debugging purposes -- displays the margins

\usepackage{amsmath}

\usepackage{hyperref}

\usepackage{fancyhdr}

\usepackage{hanging}

\hypersetup{colorlinks=true,allcolors=[RGB]{97,15,11}}

\fancyfoot[L]{\emph{History of Media Studies}, vol. 2, 2022}


% Set up the images/graphics package
\usepackage{graphicx}
\setkeys{Gin}{width=\linewidth,totalheight=\textheight,keepaspectratio}
\graphicspath{{graphics/}}

\title[Exclusiones/Exclusions]{Exclusiones/Exclusions: El papel de la historia en saldar la deuda histórica del campo} % longtitle shouldn't be necessary

% The following package makes prettier tables.  We're all about the bling!
\usepackage{booktabs}

% The units package provides nice, non-stacked fractions and better spacing
% for units.
\usepackage{units}

% The fancyvrb package lets us customize the formatting of verbatim
% environments.  We use a slightly smaller font.
\usepackage{fancyvrb}
\fvset{fontsize=\normalsize}

% Small sections of multiple columns
\usepackage{multicol}

% Provides paragraphs of dummy text
\usepackage{lipsum}

% These commands are used to pretty-print LaTeX commands
\newcommand{\doccmd}[1]{\texttt{\textbackslash#1}}% command name -- adds backslash automatically
\newcommand{\docopt}[1]{\ensuremath{\langle}\textrm{\textit{#1}}\ensuremath{\rangle}}% optional command argument
\newcommand{\docarg}[1]{\textrm{\textit{#1}}}% (required) command argument
\newenvironment{docspec}{\begin{quote}\noindent}{\end{quote}}% command specification environment
\newcommand{\docenv}[1]{\textsf{#1}}% environment name
\newcommand{\docpkg}[1]{\texttt{#1}}% package name
\newcommand{\doccls}[1]{\texttt{#1}}% document class name
\newcommand{\docclsopt}[1]{\texttt{#1}}% document class option name


\begin{document}

\begin{titlepage}

\begin{fullwidth}
\noindent\LARGE\emph{Exclusions in the History of Media Studies
} \hspace{25mm}\includegraphics[height=1cm]{logo3.png}\\
\noindent\hrulefill\\
\vspace*{1em}
\noindent{\Huge{Exclusiones/Exclusions: El papel de la\\\noindent historia en saldar la deuda histórica del campo\par}}

\vspace*{1.5em}

\noindent\LARGE{Peter Simonson} \href{https://orcid.org/0000-0001-7156-467X}{\includegraphics[height=0.5cm]{orcid.png}}\par}\marginnote{\emph{Peter Simonson, David W. Park, and Jefferosn Pooley, ``Exclusiones/Exclusions: El papel de la historia en saldar la deuda histórica del campo,'' \emph{History of Media Studies} 2 (2022), \href{https://doi.org/10.32376/d895a0ea.cb32b735}{https://doi.org/ 10.32376/d895a0ea.cb32b735}.} \vspace*{0.75em}}
\vspace*{0.5em}
\noindent{{\large\emph{University of Colorado Boulder}, \href{mailto:peter.simonson@colorado.edu}{peter.simonson@colorado.edu}\par}} \marginnote{\href{https://creativecommons.org/licenses/by-nc/4.0/}{\includegraphics[height=0.5cm]{by-nc.png}}}

\vspace*{0.75em} 

\noindent{\LARGE{David W. Park}  \href{https://orcid.org/0000-0001-7019-1525}{\includegraphics[height=0.5cm]{orcid.png}}\par}
\vspace*{0.5em}
\noindent{{\large\emph{Lake Forest College}, \href{mailto:park@lfc.edu}{park@lfc.edu}\par}}

\vspace*{0.75em} % third author

\noindent{\LARGE{Jefferson Pooley} \href{https://orcid.org/0000-0002-3674-1930}{\includegraphics[height=0.5cm]{orcid.png}}\par}
\vspace*{0.5em}
\noindent{{\large\emph{Muhlenberg College}, \href{mailto:pooley@muhlenberg.edu}{pooley@muhlenberg.edu}\par}}

\end{fullwidth}


\newthought{\emph{Traducción de} William Quinn}



\hypertarget{resumen}{%
\section{Resumen}\label{resumen}}

En esta introducción para la sección especial sobre
``\href{https://hms.mediastudies.press/exclusions}{Exclusiones en la
historia de los estudios de los medios de comunicación}'', comenzamos
llamando la atención sobre el papel constitutivo que la exclusión ha
desempeñado en la historiografía de los estudios de los medios de
comunicación. Las exclusiones relacionadas con el género, la raza, la
lengua, el colonialismo, la ubicación geopolítica y los privilegios
sancionados institucionalmente desempeñan un papel importante en la
configuración de los relatos formales e informales del pasado de
nuestros campos. El proyecto de reversión y recuperación se basa en el
pensamiento poscolonial y decolonial, en las críticas raciales y
afrocéntricas, en los estudios feministas y en la crítica geopolítica.
Uno de los objetivos es provincializar gran parte de la historiografía
de los estudios de los medios de comunicación. Inspirándonos en los
movimientos críticos contemporáneos que se interrelacionan
profundamente, identificamos cuatro tareas urgentes para la historia de
los estudios de los medios de comunicación: 1) poner en evidencia el
estado actual de los campos académicos, 

\enlargethispage{2\baselineskip}

\vspace*{2em}

\noindent{\emph{History of Media Studies}, vol. 2, 2022}




 \end{titlepage}



\noindent 2) crear colaboraciones
internacionales que reconfiguren lo que se ha considerado como
``centros'' y ``periferias'' en los estudios sobre los medios de
comunicación, 3) encontrar formas de resistencia a la creciente
hegemonía del inglés en el sistema global de conocimiento, y 4) apoyar
una infraestructura de publicación abierta y sin ánimo de lucro.
Proponemos que una historiografía informada por la comprensión
constitutiva y contingente de la exclusión representa un importante
camino a seguir para la historia de los estudios de los medios de
comunicación.

\vspace*{2em}


\newthought{Las historias y las} realidades actuales de los campos académicos se
constituyen mutuamente. Los patrones de exclusión que nos llegaron como
legado de nuestros diversos pasados suelen persistir, como vicios, en
nuestras prácticas actuales como profesionales dedicados a la
investigación, docencia, administración, y también como colegas. Estos
patrones asumen formas muy conocidas y otras nuevas, a la vez que los
agentes de cambio hacen el trabajo pesado de señalarlos y buscar
alternativas más inclusivas. Esto aplica de manera general a las
prácticas académicas, pero en esta introducción y en la sección especial
que sigue, queremos enfocarnos en un conjunto particular de exclusiones:
las vinculadas a las memorias colectivas de los pasados de nuestros
campos y a las historias formales que se han escrito sobre ellos. Si no
se hacen esfuerzos explícitos, las exclusiones implicadas con el género,
la raza, la lengua, el colonialismo, la ubicación geopolítica y el
privilegio avalado institucionalmente se reproducirán en los relatos
formales e informales que se hagan de los pasados de nuestros campos.
Asimismo, si hemos de comprender las inequidades que condicionan los
campos académicos contemporáneos, necesitamos hacer más por iluminar
exactamente cómo surgieron y se perpetuaron históricamente, y
necesitamos prestar más atención a los lugares, las personas y los temas
que han sido marginados de nuestras memorias colectivas. Eso
precisamente era la intención de una preconferencia virtual de la
Asociación Internacional de Comunicación (ICA) celebrada en mayo de
2021,
\href{https://hms.mediastudies.press/pub/schedule/release/12}{``Exclusions
in the History and Historiography of Communication Studies'',}
(``Exclusiones en la historia y la historiografía de los estudios de
comunicación'') donde se leyeron y comentaron borradores de los ensayos
publicados aquí. Esta introducción tiene el propósito de ubicar en su
perspectiva histórica esa preconferencia y otros esfuerzos afines, y de
situar los ensayos dentro de la historiografía reciente de los estudios
de los medios y la comunicación.

\newpage

En la reunión de 2021 se vio reflejado un momento más amplio de la
historia de las ciencias sociales y humanidades (CSH), mismas que se
encuentran insertas desde luego en las sociedades más amplias que las
configuran. Ese momento, cuya historia se contará algún día, lleva las
huellas del reconocimiento generalizado que se está llevando a cabo ante
las hegemonías e inequidades sistémicas presentes en el trabajo
académico. La palabra clave aquí es \emph{generalizado}. Los miembros de
grupos marginados dentro de las CSH siempre han experimentado la
inequidad, como se ha señalado en críticas incisivas publicadas desde la
década de los sesenta y setenta, si no es que desde antes. En estas
intervenciones añejas se han planteado múltiples interrogatorios:
¿cuáles temas, métodos y paradigmas ocupan un lugar central, y cuáles
han sido ignorados o marginados? ¿cómo es que investigadores, así como
la investigación misma, de Estados Unidos y otros países del Norte
Global acumulan ventajas sistemáticas que dejan marginado o de plano
invisibilizado al Sur Global? ¿y cómo es que quienes se han dedicado a
la academia minorizados acumulan a su vez desventajas
sistemáticas?\footnote{La acumulación de ventajas y desventajas fue un
  tema propuesto por Robert K. Merton y aplicado a muchos contextos,
  entre ellos al famoso concepto del ``efecto Mateo''. Merton escribe:
  ``El concepto de la ventaja acumulativa nos hace ver las maneras en
  que las ventajas comparativas iniciales de la capacidad formada, la
  ubicación estructural y los recursos disponibles desencadenan
  incrementos sucesivos de ventaja, de tal modo que se abren cada vez
  más las brechas entre quienes tienen y quienes no tienen en el campo
  de la ciencia (al igual que en otros ámbitos de la vida social) hasta
  que la dinámica quede atenuada por procesos compensatorios''. Robert
  K. Merton, ``The Matthew Effect in Science, II Cumulative Advantage
  and the Symbolism of Intellectual Property'', \emph{Isis} 79,
  n.\textsuperscript{o} 4 (1988): 606.}¿cómo es que quedan marginados en
las redes sociales de hombres blancos que ostentan posiciones de poder?
¿cómo se les carga trabajo adicional, típicamente sin reconocimiento?
¿cómo quedan asignados a ámbitos de conocimiento donde otras personas
como ellos han sido excluidas de los cánones de los textos clásicos y
los ''padres fundadores'' del campo? ¿cómo han sido relegados a publicar
trabajos que luego no se citan o se quedan en el olvido? Como lo
esbozaremos aquí, se trata de prácticas sociales que han sido señaladas
y criticadas por más de medio siglo, pero muy pocos han tenido oídos
para escuchar estos señalamientos. Más recientemente, sin embargo, las
intervenciones críticas han tomado más ímpetu ---entrelazadas e
imposibles de ignorar por parte de los que ocupan los centros
hegemónicos de los campos---.

\hypertarget{una-breve-genealoguxeda-del-presente}{%
\section{Una breve genealogía del
presente}\label{una-breve-genealoguxeda-del-presente}}

Es importante señalar las líneas de intervención que nos han traído a
esta coyuntura, en parte porque apenas si se han reconocido en los
esfuerzos actuales por saldar la deuda histórica del campo. Uno de los
vectores está conformado por las críticas geopolíticas de las CSH
expresadas por primera vez a finales de los años sesenta y en los
setenta. Los relatos de imperialismo, neocolonialismo y dependencia
académicos se abrevaron de versiones más amplias de la teoría de la
dependencia y de las refutaciones de los planteamientos dominantes de la
modernización.\footnote{Véanse por ejemplo Syed Hussein Alatas,
  ``Academic Imperialism'', conferencia magistral pronunciada ante} Estos
relatos iban de la mano de los esfuerzos realizados en el Sur Global por
desarrollar ciencias sociales ``indígenas'', es\marginnote{la
  Conferencia Regional para el Sureste de Asia de la Asociación
  Internacional de Sociología (1969), reimpresa en \emph{Reflections on
  Alternative Discourses for Southeast Asia} , ed. de Syed Farid Alatas
  (Singapur: Centre for Advanced Studies, 2001); Philip G. Altbach,
  ``Servitude of the Mind?: Education, Dependency, and Neocolonialism'',
  \emph{Teachers College Record} 79, n.\textsuperscript{o} 2 (1977);
  Frederick H. Gareau, ``Another Type of Third World Dependency: The
  Social Sciences'', \emph{International Sociology} 3,
  n.\textsuperscript{o} 2 (1988), y Syed Farid Alatas, ``Academic
  Dependency and the Global Division of Labour in the Social Sciences'',
  \emph{Current Sociology} 51, n.\textsuperscript{o} 6 (2003).} decir, conocimientos
construidos a partir de epistemologías desarrolladas desde las
tradiciones culturales locales, en vez de refractadas a través de los
paradigmas dominantes del Norte.\footnote{Juan Eugenio Corradi,
  ``Cultural Dependence and the Sociology of Knowledge: The Latin
  American Case'', en \emph{Ideology and Social Change in Latin
  America}, ed. de June Nash, Juan Eugenio Corradi y Hobart Spaulding,
  Jr. (Nueva York: Gordon and Breach, 1977); Guillermo Boils Morales,
  ``Bibliografía sobre ciencias sociales en América Latina'',
  \emph{Revista Mexicana de Sociología} 40 (1978); Oladimeji I. Alo,
  ``Contemporary Convergence in Sociological Theories: The Relevance of
  the African Thought-System in Theory Formation'', \emph{Présence
  Africaine}, n.\textsuperscript{o} 126 (1983). Utilizamos la palabra
  \emph{indígena} en minúscula para referirnos a sistemas de
  conocimiento generados dentro de una cierta región, por lo general
  marginada en términos geopolíticos, e \emph{Indígena} en mayúscula
  para referirnos a culturas y pueblos cuyas historias remontan a las
  sociedades precoloniales o anteriores a la llegada de los colonos.}
Cuando en los años noventa y a inicios del año 2000 las personas que
estudiaban los medios en Europa Occidental y Estados Unidos hicieron
esfuerzos encaminados a ``desoccidentalizar'' e ``internacionalizar''
los estudios de medios, rara vez vinculaban sus esfuerzos a estas
tradiciones más amplias. Como observa Wendy Willems, sus intenciones
``tenían que ver más con extender la cobertura de las indagaciones
académicas de los medios y la comunicación a países que no solían
incluirse en el canon occidental que con cuestionar la centralidad de la
teoría occidental''.\footnote{Wendy Willems, ``Provincializing Hegemonic
  Histories of Media and Communication Studies: Toward a Genealogy of
  Epistemic Resistance in Africa'', \emph{Communication Theory} 24,
  n.\textsuperscript{o} 4 (2014): 416. Véase también Afonso de
  Albuquerque y Thaiane Moreira de Oliveira, ``Thinking the Recolonial
  in Communication Studies: Reflections from Latin America'',
  \emph{Comunicação, Mídia, e Consumo} 18, n.\textsuperscript{o} 51
  (2021).} Consecuencia irónica de los llamados a ``desoccidentalizar''
el campo era que se quedaban ocultadas tradiciones intelectuales de
África y otras regiones. Cuando provienen de estudiosos aventajados del
Norte Global, los llamados a desoccidentalizar amenazan con borrar las
historias locales, reproduciendo sin querer patrones coloniales añejos.
De hecho, estamos conscientes de que esta misma introducción puede caer
en la misma dinámica, por lo que tenemos muy presente el riesgo.

En las décadas más recientes, las críticas geopolíticas de las CSH
globales han tomado bastante ímpetu fuera de los estudios de
comunicación y medios y, más en estos últimos años todavía, dentro de
ellos. Los pensadores poscoloniales y decoloniales han provincializado
las aspiraciones universalistas de la modernidad europea, incluidas sus
normas de racionalidad y conocimiento, vinculándolas a ``las historias
más generales de colonialismo, imperio y esclavización'', en palabras de
Gurminder K. Bhambra.\footnote{Gurminder K. Bhambra, ``Postcolonial and
  Decolonial Dialogues'', \emph{Postcolonial Studies} 17,
  n.\textsuperscript{o} 2 (2014): 115. Para un volumen
  interdisciplinario clave de América Latina, véase Edgardo Lander, ed.
  \emph{La colonialidad del saber: eurocentrismo y ciencias sociales}
  (Buenos Aires: CLASCO, 2000), y para un relato útil del
  poscolonialismo y/versus decolonialidad desde los horizontes de los
  estudios de medios y comunicación, véase Sinfree Makoni y Katherine A.
  Masters, ``Decolonization and Globalization in Communication
  Studies'', en \emph{Oxford Research Encyclopedia of Communication},
  ed. de Jon Nussbaum (Oxford: Oxford University Press, 2021).} Estas
críticas se traslapan con tradiciones intelectuales Indígenas, bien
establecidas y crecientes, de activismo, análisis crítico y formas de
conocimiento y cultura que representan alternativas contrahegemónicas al
pensamiento colonial occidental.\textsuperscript{6} Del pensamiento decolonial e indígena ha surgido una
nueva ola de teoría académica de la dependencia.\textsuperscript{7}
Dicha ola se suma al coro de voces que denuncian los impactos de la
globalización neoliberal en la producción académica y en las jerarquías
de estatus geopolítico. Como se ha demostrado en múltiples estudios, el
\emph{ranking} global de universidades, revistas especializadas y
factores de impacto favorece a Estados Unidos y Europa
Occidental.\textsuperscript{8} Estos \emph{rankings} se vinculan a su vez a la creciente
hegemonía del inglés como \emph{lingua franca} para las ciencias
sociales internacionales y las presiones resultantes por publicar en
inglés, situación que empieza a cuestionarse.\textsuperscript{9} Los métodos
cuantitativos de análisis de redes a gran escala han fortalecido estos
esfuerzos recientes al proporcionar nuevas herramientas para destacar
las inequidades sistémicas entre los centros y las periferias globales
en términos de publicaciones, tasas\marginnote{\textsuperscript{6} Véase por ejemplo Bagele
  Chilisa, \emph{Indigenous Research Methodologies} (Thousand Oaks, CA:
  SAGE, 2012).} de\marginnote{\textsuperscript{7} Sintetizada en
  Caroline M. Schöpf, ``The Coloniality of Global Knowledge Production:
  Theorizing the Mechanisms of Academic Dependency'', \emph{Social
  Transformations} 8, n.\textsuperscript{o} 2 (2020), y Jinba Tenzin y
  Chenpang Lee, ``Are We Still Dependent? Academic Dependency Theory
  after 20 Years'', \emph{Journal of Historical Sociology} 35 (2022).} citación,\marginnote{\textsuperscript{8} Márton Demeter, ``The Winner Takes it All:
  International Inequality in Communication and Media Studies Today'',
  \emph{Journalism and Mass Communication Quarterly} 96,
  n.\textsuperscript{o} 1 (2019); Afonso de Albuquerque et al.,
  ``Structural Limits to the De-Westernization of the Communication
  Field: The Editorial Board in Clarivate's \emph{JCR} System'',
  \emph{Communication, Culture \& Critique} 13, n.\textsuperscript{o} 2
  (2020).} integración\marginnote{\textsuperscript{9}\setcounter{footnote}{9} Robert Phillipson
  y Tove Skutnabb-Kangas, ``Communicating in `Global English': Promoting
  Linguistic Human Rights or Complicit with Linguicism and Linguistic
  Imperialism'', en \emph{The Handbook of Global Interventions in
  Communication Theory}, ed. de Yoshitaka Miike y Jing Yin (Nueva York:
  Routledge, 2022). Volvemos más adelante al tema de la hegemonía del
  inglés y otros trabajos recientes en los que se aborda.} de juntas
editoriales y asociaciones profesionales internacionales.\footnote{Véase
  por ejemplo Márton Demeter, \emph{Academic Knowledge Production and
  the Global South: Questioning Inequality and Under-Representation}
  (Cham, Suiza: Palgrave Macmillan, 2020). Para un análisis reciente del
  campo de la comunicación, véase Brian Ekdale et al., ``Geographic
  Disparities in Knowledge Production: A Big Data Analysis of
  Peer-Reviewed Communication Publications from 1990 to 2019'',
  \emph{International Journal of Communication} 16 (2022).}

Entrecruzándose con estas críticas geopolíticas globales, en los años
sesenta y setenta estudiosos negros empezaron a desarrollar críticas
afrocéntricas y de otras perspectivas raciales con respecto a las formas
euronorteamericanas del conocimiento que predominaban. En Estados Unidos
se establecieron los programas de estudios negros a finales de los
sesenta, tanto en los Colegios y Universidades Históricamente Negros
(HBCU) como en instituciones mayoritariamente blancas.\footnote{Fabio
  Rojas, \emph{From Black Power to Black Studies: How a Radical Social
  Movement Became an Academic Discipline} (Baltimore: Johns Hopkins
  University Press, 2007).} Su fundación se debió a una mezcla de
oposición intelectual y política a las epistemologías y metodologías
dominantes, un compromiso con una teoría del conocimiento como vehículo
para el cambio social, y una exigencia de que la educación superior
prestara un servicio más eficaz a las comunidades negras.\footnote{James
  E. Turner, ``Foreword: Africana Studies and Epistemology, a Discourse
  in the Sociology of Knowledge'', en \emph{The Next Decade: Theoretical
  and Research Issues in Africana Studies}, ed. de James E. Turner
  (Ithaca, NY: Cornell University Africana Studies and Research Center,
  1984).} Por ejemplo, el Instituto para un Mundo Negro, ubicado en
Atlanta, atrajo a intelectuales de la diáspora negra, entre quienes se
encontraban Sylvia Wynter y C.L.R. James de la región caribeña. El
instituto impulsó un esfuerzo transnacional alineado con el llamado
lanzado por Franz Fanon en 1965 de ``probar nuevos conceptos y tratar de
poner en pie un hombre nuevo''.\footnote{Derrick E. White, \emph{The
  Challenge of Blackness: The Institute of the Black World and Political
  Activism in the 1970s} (Gainesville, FL: University Press of Florida,
  2011), 146, 232n25, citando el libro de Fanon \emph{Wretched of the
  Earth}.} El concepto de diáspora africana, que tenía sus raíces en los
panafricanismos de Marcus Garvey y W.E.B. DuBois, surgió en una reunión
del Congreso Internacional de Historiadores Africanos celebrada en
Tanzania en 1965.\textsuperscript{14} El movimiento
tomó impulso en el campo de la comunicación oral en Estados Unidos,
donde en 1968 estudiosos afroamericanos formaron un Bloque Negro dentro
de la Asociación de Comunicación Oral de América (Speech Association of
America), y en 1972 celebraron una Conferencia de Comunicación
Negra.\textsuperscript{15} En la década de
los ochenta algunos de sus integrantes organizarían congresos mundiales
sobre la Comunicación Negra que convocaban a un elenco internacional de
investigadores. Uno de los líderes del Bloque Negro fue Arthur L. Smith,
quien adoptó el nombre de Molefi Asante al visitar la Universidad de
Ghana en 1973 y desarrolló algunas de las primeras y más influyentes
críticas afrocéntricas de las teorías occidentales de la
comunicación.\textsuperscript{16} Más
allá de los estudios de la comunicación, la crítica y teórica cultural
Sylvia Wynter, la socióloga Patricia Hill Collins y el filósofo Charles
W. Mills publicaron en los años ochenta y noventa obras que resultarían
fundamentales para la siguiente generación de críticas raciales de las
formas europeas del conocimiento que predominaban, y de las comunidades
académicas que las perpetuaban.\textsuperscript{17}

Desde otros ámbitos sociales e intelectuales, las investigadoras
feministas expusieron sistemas de exclusión en las CSH, organizados por
los ejes de género, sexualidad e interseccionalidad. Esa historia
empieza\marginnote{\textsuperscript{14} Joseph E. Harris, introducción a \emph{Global
  Dimensions of the African Diaspora}, segunda edición, ed. de Joseph E.
  Harris (Washington: Howard University Press, 1993), 4.} a configurarse en los sesenta y setenta al arraigarse los
movimientos de\marginnote{\textsuperscript{15} Jack L. Daniel, \emph{Changing the Players and the Game:
  A Personal Account of the Speech Communication Association Black
  Caucus Origins} (Annandale, VA: Speech Communication Association,
  1995). La Asociación de Comunicación Oral de América (The Speech
  Association of America) tras un cambio intermedio de nombre, en 1972
  se convirtió en la Asociación Nacional de Comunicación (The National
  Communication Association, o NCA), de Estados Unidos.} mujeres\marginnote{\textsuperscript{16} Alton Hornsby, ``Molefi Kete Asante/Arthur Lee
  Smith Jr. (1942- )'', \emph{BlackPast,} 20 julio 2007. Sobre Asante en
  los contextos de los estudios de comunicación, véanse Armond Towns,
  ``Against the `Vocation of Autopsy': Blackness and/in US Communication
  Histories'', \emph{History of Media Studies} 1 (2021), y Ronald L.
  Jackson II y Sonja M. Brown Givens, \emph{Black Pioneers in
  Communication Research} (Thousand Oaks, CA: SAGE, 2016), 11--38.} y\marginnote{\textsuperscript{17}\setcounter{footnote}{17} Por ejemplo, Sylvia Wynter,
  ``The Ceremony Must be Found: After Humanism'', \emph{boundary 2} 12,
  n.\textsuperscript{o} 3/13, n.\textsuperscript{o} 1 (1984); Patricia
  Hill Collins, ``Learning from the Outsider Within: The Sociological
  Significance of Black Feminist Thought'', \emph{Social Problems} 33,
  n.\textsuperscript{o} 6 (1986); Charles W. Mills, \emph{The Racial
  Contract} (Ithaca, NY: Cornell University Press, 1997).} la segunda ola del feminismo en varias regiones
del mundo. Estos acontecimientos se dieron de manera diversa en
diferentes culturas y contextos nacionales, y su historia global en la
academia está pendiente de escribirse. En algunos espacios y durante la
década de los setenta, empezó a aumentar el número de mujeres que
ocupaban puestos académicos. Formaron bloques de mujeres dentro de las
asociaciones profesionales, y estos bloques, junto con los del
movimiento LGBTQ emergente, comenzaron a desafiar la segregación por
género en las conferencias académicas a la vez que brindaban redes de
apoyo para compartir experiencias de marginación, inequidad y lucha
cotidiana.\footnote{Véase por ejemplo Pamela Roby, ``Women and the ASA:
  Degendering Organizational Structures and Processes, 1964--1974'',
  \emph{The American Sociologist} 23 (1992). Sobre el caso de la NCA de
  Estados Unidos, véase Charles E. Morris III y Catherine Helen
  Palczewski, ``Sexing Communication: Hearing, Feeling, Remembering
  Sex/Gender and Sexuality in the NCA'', en \emph{A Century of
  Communication Studies: The Unfinished Conversation}, ed. de Pat J.
  Gehrke y William M. Keith (Nueva York: Routledge, 2015).}

En los años setenta y ochenta, feministas de todos los campos de las CSH
lanzaron críticas incisivas por los presupuestos que se daban por
sentados en sus disciplinas y buscaron reorientar los objetos y procesos
de conocimiento. De ahí surgió un interés por investigar de manera
sistemática las dinámicas de la producción del conocimiento dentro de
los campos académicos, aplicándose conceptos como los ``conocimientos
situados'' de Donna Haraway y el ``efecto Matilda'' de Margaret
Rossiter, que resultaron ser poderosas herramientas
transdisciplinares.\footnote{Donna Haraway, ``Situated Knowledges: The
  Science Question in Feminism and the Privilege of Partial
  Perspective'', \emph{Feminist Studies} 14, n.\textsuperscript{o} 3
  (1988); Margaret W. Rossiter, ``The Matilda Effect in Science'',
  \emph{Social Studies of Science} 23, n.\textsuperscript{o} 2 (1993).
  El trabajo de Rossiter representa una poderosa expansión del concepto
  del Efecto Mateo, propuesto por Robert K. Merton y citado arriba en la
  nota al pie 1.} Sue Curry Jansen aprovechó estas ideas para abrir
espacios críticos donde se planteaba el género como elemento
constitutivo, y no contingente, de la historia de los campos de la
comunicación y su (re)producción dinámica del conocimiento y poder en la
actualidad.\footnote{Sue Curry Jansen, ``\,`The Future is Not What it
  Used to Be': Gender, History, and Communication Studies'',
  \emph{Communication Theory} 3, n.\textsuperscript{o} 2 (1993).} Al
mismo tiempo, a finales de los ochenta, retomando lo que Patricia Hill
Collins llamara el reconocimiento añejo de ``la naturaleza imbricada de
raza, género y clase'' en el pensamiento feminista negro, Kimberlé
Crenshaw formuló el concepto de interseccionalidad que ha servido como
lente esencial para investigar los enmarañamientos de género, raza,
clase y otras posicionalidades.\textsuperscript{21}

¿Cómo han incidido las críticas geopolíticas, raciales y feministas de
los campos de conocimiento en las historias escritas de los estudios de
medios y comunicación? Su impacto ha sido lento e indirecto. Primero,
cabe señalar que antes de la década de los noventa no existía un
conjunto significativo de trabajos sobre la historia de estos campos
sino solo esfuerzos aislados y el mito de los ``padres fundadores'' y
los hijos críticos que los mataron de manera simbólica.\textsuperscript{22} En una revisión sistemática de la literatura en
inglés dos décadas después se encontró un abrumador enfoque geográfico
en las personas e ideas provenientes de Estados Unidos, Canadá, Reino
Unido y Alemania.\textsuperscript{23}
El Sur Global y otras regiones apenas figuraban para los estudiosos de
habla inglesa. Si bien no contamos con un análisis así de sistemático
del género y la raza de quienes merecieron mención\marginnote{\textsuperscript{21} Collins, ``Learning from the
  Outsider Within'', S19; Kimberlé Crenshaw, ``Demarginalizing the
  Intersection of Race and Sex: A Black Feminist Critique of
  Antidiscrimination Doctrine, Feminist Theory and Antiracist
  Politics'', \emph{University of Chicago Legal Forum} 1989 (1989).} en\marginnote{\textsuperscript{22} Jefferson
  Pooley, ``The New History of Mass Communication Research'', en
  \emph{The History of Media and Communication Research: Contested
  Memories}, ed. de David W. Park y Jefferson Pooley (Nueva York: Peter
  Lang, 2008); Peter Simonson y David W. Park, introducción ``On the
  History of Communication Study'' a \emph{The International History of
  Communication Study}, eds. Peter Simonson y David W. Park (Nueva York:
  Routledge, 2016).} los\marginnote{\textsuperscript{23}\setcounter{footnote}{23} Jefferson Pooley y David W. Park,
  ``Communication Research'', en \emph{The Handbook of Communication
  History}, ed. de Peter Simonson et al. (Nueva York: Routledge, 2013).} escritos
existentes, nos atrevemos a afirmar que la literatura histórica resulta
aún más abrumadoramente enfocada en los hombres blancos de ascendencia
europea. Además, existe un conjunto pequeño, pero creciente, de trabajos
sobre integrantes de grupos minorizados. La historia de las mujeres y el
género es la más desarrollada y remonta a los años noventa, aunque el
registro dista de estar completo y le falta un enfoque internacional y
comparativo más amplio.\footnote{Para América Latina véase Yamila Heram
  y Santiago Gándara, \emph{Pioneras en los estudios de comunicación}
  (Buenos Aires: TeseoPress, 2021), y Clemencia Rodríguez et al., eds.,
  \emph{Mujeres de la comunicación} (Bogotá: Friedrich Ebert Stiftung,
  2020). Para un excelente estudio transnacional véase Elisabeth Klaus y
  Josef Seethaler, eds., \emph{What Do We Really Know about Herta
  Herzog?} (Fráncfort del Meno: Peter Lang, 2016). Para trabajos
  recientes en inglés que incluyen bibliografías selectas de la
  literatura, véanse Leonarda García-Jiménez y Esperanza Herrero,
  ``Narrating the Field Through Some Female Voices: Women's Experiences
  and Stories in Academia'', \emph{Communication Theory} 32,
  n.\textsuperscript{o} 2 (2022), y Karen Lee Ashcraft y Peter Simonson,
  ``Gender, Work, and the History of Communication Research: Figures,
  Formations, and Flows'', en Simonson y Park, \emph{The International
  History}.} Luego está la literatura creciente sobre la historia de los
estudiosos negros y las estructuras racializadas de los estudios de
comunicación y medios, si bien de nuevo queda mucho trabajo por
hacer.\footnote{Para una bibliografía selecta de trabajos en el contexto
  de Estados Unidos, véase Armond Towns, ``Against the `Vocation of
  Autopsy'\,''. Véanse también Dhanveer Singh Brar y Ashwani Sharma,
  ``What is This `Black` in Black Studies? From Black British Cultural
  Studies to Black Critical Thought in UK Arts and Higher Education'',
  \emph{New Formations} n.\textsuperscript{o} 99 (2019); Jeffrey S.
  Wilkinson, William R. Davie y Angeline J. Taylor, ``Journalism
  Education in}

Hasta hace poco, las críticas geopolíticas rara vez ofrecían guías
explícitas para escribir las historias del campo, aunque existe un
conjunto significativo de trabajos que podrían apoyar el esfuerzo
reorientando nuestro imaginario global más allá de su centro tradicional
en Estados Unidos. En las últimas dos décadas se han visto tendencias
hacia marcos más transnacionales y hasta globales para abarcar el
desarrollo histórico de los estudios de comunicación y
medios.\textsuperscript{26} Dichas tendencias se enmarcan dentro de movimientos
más amplios en la historia y sociología de las CSH.\textsuperscript{27} Los marcos transnacionales y globales
abren la posibilidad de mapear las líneas de la hegemonía de Estados
Unidos tras la segunda guerra mundial y, a la vez, de provincializar la
versión norteamericana del campo. Resulta crucial reconocer que existen
tradiciones alternativas de educación e investigación del periodismo, el
cine, la radio, la televisión y otras formas de lo que en América Latina
llegó a conocerse como comunicación social. Ver la comunicación como un
campo norteamericano ---lo suelen hacer tanto los entusiastas como los
críticos--- implica ignorar, por ejemplo, la \emph{Zeitungswissenschaft}
alemana y la \emph{Publizistik} europea de la posguerra, pues ambas
brindaban modelos alternativos de alcance transnacional.\textsuperscript{28} Asimismo, implica ignorar las tradiciones católicas que
han dejado una huella profunda, como el caso de la educación jesuita en
América Latina.\textsuperscript{29} Por otro lado, existe una larga tradición de
investigación sobre la comunicación desde perspectivas marxistas; tiene
sus raíces intelectuales en los movimientos independentistas del siglo
XIX, luego tuvo un impulso fuerte gracias a la revolución cubana de 1959
y se enriqueció en los años sesenta y setenta con lecturas de Antonio
Gramsci y la teoría de la dependencia.\textsuperscript{30}
Aunque menos desarrolladas que las de América Latina, hay tradiciones
intelectuales análogas en el África post-independencia y en el mundo
árabe, forjadas a su vez por el rechazo de los paradigmas de la
modernización y el impulso hacia alternativas comprometidas con la
indigenización, la africanización y la unión panárabe.\textsuperscript{31} La Organización de
las Naciones Unidas para la Educación, la Ciencia y la Cultura (UNESCO),\marginnote{Black and White: A 50-Year Journey Toward Diversity'',
  \emph{Journalism \& Mass Communication Educator} 75,
  n.\textsuperscript{o} 4 (2020); Julian Henriques y David Morley, eds.,
  \emph{Stuart Hall: Conversations, Projects and Legacies} (Londres:
  Goldsmiths Press, 2017); Nova Gordon Bell, ``Towards an Integrated
  Caribbean Paradigm in Communication Thought: Confronting Academic
  Dependence in Media Research'', en \emph{Re-imagining Communication in
  Africa and the Caribbean}, ed. de Hopeton S. Dunn et al. (Cham, Suiza:
  Palgrave Macmillan), y Terje Skjerdal y Keyan Tomaselli,
  ``Trajectories of Communication Studies in Sub-Saharan Africa'', en
  Simonson y Park, \emph{The International History}.}
el\marginnote{\textsuperscript{26} Para ejemplos, véanse Simonson y Park, \emph{The
  International History}; Erick R. Torrico Villanueva, \emph{La
  comunicación: pensada desde América Latina (1960--2009)} (Salamanca:
  Comunicación Social, 2016), y Stefanie Averbeck-Lietz, ed.,
  \emph{Kommunikationswissenschaft im Internationalen Vergleich:
  Transnationale Perspektiven} (Wiesbaden: Springer Fachmedien
  Wiesbaden, 2017).} movimiento\marginnote{\textsuperscript{27} Véase Johan
  Heilbron, Nicolas Guilhot y Laurent Jeanpierre, ``Toward a
  Transnational History of the Social Sciences'', \emph{Journal of the
  History of Behavioral Sciences} 44, n.\textsuperscript{o} 2 (2008);
  Neus Rotger, Diana Roig-Sanz y Marta Puxan-Oliva, ``Introduction:
  Towards a Cross-Disciplinary History of the Global in the Humanities
  and Social Sciences'', \emph{Journal of Global History} 14,
  n.\textsuperscript{o} 3 (2019).} de\marginnote{\textsuperscript{28} Es
  amplia la literatura sobre la historia de la investigación alemana de
  la comunicación y sus precursores históricos, tanto en alemán como en
  inglés. Para empezar con bibliografías extensas, véanse Erik Vroons,
  ``Communication Studies in Europe: A Sketch of the Situation about
  1955'', \emph{Gazette} 67, n.\textsuperscript{o} 6 (2005); Maria
  Löblich, ``German \emph{Publizistikwissenschaft} and Its Shift from a
  Humanistic to an Empirical Social Science'', \emph{European Journal of
  Communication} 22, n.\textsuperscript{o} 1 (2007); Thomas Wiedemann,
  ``Practical Orientation as a Survival Strategy: The Development of
  \emph{Publizistikwissenschaft} by Walter Hagemann'', en Simonson y
  Park, \emph{The International History}; Averbeck-Lietz,
  \emph{Kommunikationswissenschaft im Internationalen Vergleich;} Thomas
  Wiedemann, Michael Meyen e Iván Lacasa-Mas, ``100 Years of
  Communication Study in} países no alineados y la Asociación Internacional de
Estudios en Comunicación Social (AIECS) jugaron papeles globales en la
circulación de ideas y el desarrollo de redes sociales entre
investigadores del Sur Global y aliados izquierdistas en el Norte; la
antigua República Democrática Alemana desempeñó un rol similar, junto
con otros países del antiguo bloque soviético que mantuvieron una fuerte
presencia en la AEICS durante la guerra fría, así como sus propias
tradiciones dentro del diálogo transnacional cargado de tensión
política, sin hablar de las nuevas complicaciones que surgieron después
de 1989.\textsuperscript{32} Si hemos de desarrollar una comprensión más compleja de
las líneas contemporáneas de hegemonía geopolítica y exclusión en los
estudios de comunicación y medios, debemos incorporar todo lo que
sabemos acerca de las historias globales heterogéneas de dichos campos y
gran parte de ellas no se encuentra publicada en inglés.

\hypertarget{saldar-la-deuda-histuxf3rica-ahora}{%
\section{Saldar la deuda histórica
ahora}\label{saldar-la-deuda-histuxf3rica-ahora}}

En el momento contemporáneo se han enredado las críticas geopolíticas,
raciales, feministas y con perspectiva de género que han circulado
durante décadas, a la vez que se han convertido en una reivindicación
cada vez más difícil de ignorar por parte de quienes ocupan los centros
hegemónicos. Los estudios de medios y comunicación, junto con otros
muchos campos académicos, han tenido que lidiar con un reconocimiento
tardío de los patrones sistémicos de exclusión e injusticia implicados
en su misma constitución como campos. Esto se ha visto impulsado en
parte por el desarrollo constante de líneas de pensamiento añejas. Si
bien antes ciertas líneas de crítica geopolítica y racial podían hacer
caso omiso de la perspectiva de género, y los feminismos blancos podían
desatender cuestiones de raza, ahora estas líneas se animan mutuamente y
se enriquecen de otras corrientes de teoría social crítica. Juntas nos
brindan nuevos vocabularios y sensibilidades intelectuales, en tanto que
el reconocimiento que ocurre en el campo se está viendo impulsado por
otros reconocimientos más grandes que se han dado a nivel de sociedad y
a escala global. Para nombrar solo algunos: el movimiento Black Lives
Matter que surgió en 2013 en Estados Unidos ya se había convertido en
fenómeno global en 2016, poniendo sobre la mesa formas persistentes de
racismo en todo el mundo, y en mayo de 2020 llegó a otro nivel con el
asesinato de George Floyd. En la primavera de 2015 se lanzó en Sudáfrica
la campaña \#RhodesMustFall para protestar la persistencia del legado
colonialista en la educación superior; la campaña dio impulso a un
creciente movimiento multilateral que pretende decolonizar e indigenizar
el conocimiento en el mundo. Las\marginnote{Europe: Karl Bücher's Impact on the
  Discipline's Reflexive Project'', \emph{Studies in Communication and
  Media} 7, n.\textsuperscript{o} 1 (2018). Sobre la asimilación del
  modelo de ciencias periodísticas alemanas en Japón, véase Fabian
  Schäfer, \emph{Public Opinion, Propaganda, Ideology: Theories on the
  Press and Its Social Function in Interwar Japan, 1918--1937} (Leiden:
  Brill, 2012).}\marginnote{\textsuperscript{29} Ira Wagman, ``Remarkable Invention!''
  \emph{History of Media Studies} 1 (2021). Para América Latina, véase
  el rico estudio del ITESO, Universidad Jesuita de Guadalajara
  (México), en Graciela Bernal Loaiza, ed., \emph{50 años en la
  formación universitaria de comunicadores, 1967--2008: génesis,
  desarrollo y perspectivas} (Guadalajara: ITESO, 2018), y la reseña
  biográfica del fundador de los estudios de comunicación en
  instituciones jesuitas del país, con una reimpresión de su carta de
  1960 que versa sobre el tema, en Luis Sánchez Villaseñor, S.J.,
  \emph{José Sánchez Villaseñor, S.J., 1911/1961: notas biográficas}
  (Guadalajara: ITESO, 1997). Sobre Venezuela, véase José Martínez
  Terrero, S.J., ``Los jesuitas de Venezuela en la comunicación
  social'', \emph{Temas de Comunicación. Universidad Católica Andrés
  Bello} 1 (1992).}\marginnote{\textsuperscript{30} Existe una rica
  literatura historiográfica sobre América Latina. Para empezar, véanse
  Mariano Zarowsky, ``Communication Studies in Argentina in the 1960s
  and `70s: Specialized Knowledge and Intellectual Intervention Between
  the Local and the Global'', \emph{History of Media Studies} 1 (2021);
  Torrico Villanueva, \emph{La comunicación}; Raúl Fuentes-Navarro,
  ``Institutionalization and Internationalization of the Field of
  Communication Studies in Mexico and Latin America'', en Simonson y
  Park, \emph{The International History}; Maria Immacolata Vassallo de
  Lopes y Richard Romancini, ``History of Communication Study in Brazil:
  The Institutionalization of an Interdisciplinary Field'', en Simonson
  y Park, \emph{The International History}. Desde los noventa, Fuentes
  Navarro ha producido una extensa obra sobre la historia del campo en
  México y América Latina, misma que sirve como una guía estupenda.}\marginnote{\textsuperscript{31} Para el
  África subsahariana, véanse Willems, ``Provincializing Hegemonic
  Histories''; Skjerdal y Tomaselli, ``Trajectories of Communication'';
  Mohammad Musa, ``Looking Backward, Looking Forward: African Media
  Studies and} Marchas de Mujeres de enero de 2017,
organizadas para el día posterior a la inauguración de Donald Trump,
galvanizó la atención del mundo sobre temas de género y los nuevos
embates a los derechos y el bienestar de mujeres y gente LGBTQ. Para el
otoño de 2017, \#MeToo, iniciado en 2006 por la activista negra Tarana
Burke, tomó vuelo cuando lo enarbolaron actrices blancas de Hollywood.
Rápidamente se volvió global, convirtiéndose en grito de guerra para los
esfuerzos por visibilizar y combatir el fenómeno ubicuo, a menudo
silenciado, del abuso y acoso sexual, junto con los sistemas
patriarcales de poder que lo fomentaban. Simultáneamente, ante el
surgimiento de nuevos etnonacionalismos y supremacías blancos vinculados
a populismos derechistas, los liberales y progresistas voltearon su
mirada al poder y privilegio sistémicos de la blanquitud.

Estas son algunas de las corrientes que han impulsado las críticas
recientes de los estudios de comunicación y medios con respecto al
género, la raza, la sexualidad, el lenguaje, la colonialidad y la
ubicación geopolítica. Son demasiadas para quedar resumidas aquí, por lo
que solo señalaremos algunos hilos clave. En su último número de
``Ferment in the Field'' (``Fermento en el campo''), la \emph{Journal of
Communication} publicó el ensayo ``\#CommunicationSoWhite'', un análisis
crítico de los patrones racializados de publicación en revistas de la
ICA y de la Asociación Nacional de Comunicación (NCA) de Estados Unidos,
que ha detonado conversaciones a nivel internacional.\textsuperscript{33} En otros trabajos se ha examinado lo que Vicki Mayer et al.
llaman ``la terca persistencia del patriarcado en los estudios de
comunicación''.\textsuperscript{34} Los latinoamericanos han llevado la batuta en las críticas
decoloniales del campo, extendiendo una tradición de más de cinco
décadas.\textsuperscript{35} Forman parte de una nueva ola de esfuerzos decolonizantes que
atraviesan subcampos de los estudios de comunicación y medios, y abarcan
distintas regiones del mundo, las que, a su vez, se entrelazan con el
pensamiento reciente sobre el proyecto de
``desoccidentalización''.\textsuperscript{36} Las intervenciones adoptan formas diversas, incluyendo
una renovada atención crítica a las políticas y prácticas excluyentes de
las sedes de las conferencias y su preferencia por investigadores
provenientes de instituciones ricas del Norte Global.\textsuperscript{37} Los esfuerzos son variados y van en
aumento.

\hypertarget{exigencias-las-necesidades-del-momento}{%
\section{Exigencias: las necesidades del
momento}\label{exigencias-las-necesidades-del-momento}}

En esta empresa crítica multidimensional, que \emph{History of Media
Studies} apoya en su totalidad, quedan pendientes muchos tipos de
trabajo. En el sentido más amplio, necesitamos expandir nuestras
imaginaciones cosmopolitas a la vez que habitamos los lugares
particulares de cincuenta años de crítica y activismo que han arrojado
luz sobre las\marginnote{the Question of Power'', \emph{Journal of African Media
  Studies} 1, n.\textsuperscript{o} 1 (2009), y Eddah M. Mutua, Bala A.
  Musa y Charles Okigbo, ``(Re)visiting African Communication
  Scholarship: Critical Perspectives on Research and Theory'',
  \emph{Review of Communication} 22, n.\textsuperscript{o} 1 (2022).
  Para el mundo árabe, véanse Mohammad I. Ayish, ``Communication Studies
  in the Arab World'', en Simonson y Park, \emph{The International
  History}, y Carola Richter y Hanan Badr, ``Die Entwicklung der
  Kommunikationsforschung und -wissenschaft in Ägypten. Transnationale
  Zirkulationen im Kontext von Kolonialismus und Globalisierung'', en
  \emph{Kommunikationswissenschaft im Internationalen Vergleich:
  Transnationale Perspektiven}, ed. de Stefanie Averbeck-Lietz
  (Wiesbaden: Springer Fachmedien Wiesbaden, 2017).}\marginnote{\textsuperscript{32} Michael Meyen, ``IAMCR on the East-West Battlefield: A
  Study on the GDR's Attempts to Use the Association for Diplomatic
  Purposes'', \emph{International Journal of Communication} 8 (2014);
  Zrinjka Peruško y Dina Vozab, ``The Field of Communication in Croatia:
  Toward a Comparative History of Communication Studies in Central and
  Eastern Europe'', en Simonson y Park, \emph{The International
  History}; Maureen C. Minielli et al., eds., \emph{Media and Public
  Relations Research in Post-Socialist Countries} (Lanham, MD: Lexington
  Books, 2021).}\marginnote{\textsuperscript{33} Paula
  Chakravartty et al., ``\#CommunicationSoWhite'',~\emph{Journal of
  Communication} 68, n.\textsuperscript{o} 2 (2018). Casi de inmediato
  se empezó a citar el ensayo de forma extensa a nivel internacional, lo
  que inspiró una preconferencia de la ICA que se convirtió en un número
  especial: Eve Ng, Khadijah Costley White y Anamik Saha,
  ``\#CommunicationSoWhite: Race and Power in the Academy and Beyond'',
  \emph{Communication, Culture \& Critique} 13, n.\textsuperscript{o} 2
  (2020).}\marginnote{\textsuperscript{34} Vicki Mayer et al., ``How Do We Intervene in
  the Stubborn Persistence of Patriarchy in Communication Scholarship'',
  en \emph{Interventions: Communication Theory and Practice}, ed. de D.
  Travers Scott y Adrienne Shaw (Nueva York: Peter Lang, 2018). Véanse
  también Rodríguez et al., \emph{Mujeres de la comunicación}; Sabine
  Trepte y Laura Loths, ``National and Gender Diversity in
  Communication: A Content Analysis of Six Journals Between 2006 and
  2016'', \emph{Annals of the International Communication Association}
  44, n.\textsuperscript{o} 4 (2020), y Xinyi Wang et al., ``Gendered
  Citation Practices in the Field of Communication'', \emph{Annals of
  the International Communication Association} 45, n.\textsuperscript{o}
  2 (2021).} inequidades y la violencia epistemológica de los sistemas
dominantes del conocimiento. Esto implica reconocer cómo estas
inequidades se (re)producen en textos, encuentros y prácticas
particulares. Simultáneamente debemos explorar cómo esas instancias
concretas se vinculan comunicativamente o corresponden a situaciones
análogas en otros lugares y tiempos, lo que nos exigirá reconocer las
conexiones entre los campos académicos y la sociedad, con una
apreciación minuciosa de las distintas orientaciones normativas que
surgen en cada ámbito. La crisis del periodo actual brinda una
oportunidad inusual para hacer ajustes en los niveles más fundamentales
de la práctica. El trabajo de las personas históricamente otrerizadas,
ayudadas por aliados, ha puesto sobre la mesa los problemas y
posibilidades del momento. Hace tiempo que quienes pertenecemos a grupos
históricamente privilegiados tendríamos que haber hecho más para sacar
adelante este proceso ---reconociendo el peligro de que en nuestros
intentos simplemente volvamos a configurar las líneas tradicionales del
poder y de la ignorancia privilegiada---. Hará falta algún tipo de
división flexible de trabajo en esta empresa, vinculada a
posicionalidades, lugares institucionales, pericias y capacidades.

Sentimos que entre las muchas necesidades del presente hay cuatro en
particular que se intentaron (y se siguen intentando) abordar en la
preconferencia de 2021, en esta sección especial de ensayos y en la
revista \emph{History of Media Studies}. Ni de lejos responden a la
amplia gama de cuestiones señaladas en las críticas basadas en la
geopolítica, la colonialidad, la raza, el género, la sexualidad y la
discapacidad. No obstante, creemos que podrán contribuir al esfuerzo más
amplio, cada una a su manera. Ellas son: (1) la necesidad de remarcar
con mayor nitidez el trasfondo histórico del estado actual de los campos
académicos, (2) la necesidad de construir colaboraciones internacionales
que reconfiguren lo que se ha acostumbrado señalar como los ``centros''
y ``periferias'' en los estudios de medios, (3) la necesidad de
encontrar maneras de poner resistencia a la creciente hegemonía del
inglés en el sistema global de conocimiento, y (4) la necesidad de
apoyar una infraestructura de publicación abierta y sin fines de lucro.
Ninguna de estas necesidades es fácil de atender. Hay resistencia a cada
una de ellas integrada en las prácticas e instituciones que estructuran
nuestros campos en la actualidad. Y ninguna ha recibido la debida
atención en el momento actual en que se intenta saldar la deuda
histórica.

\hypertarget{rauxedces-histuxf3ricas}{%
\subsection{Raíces
históricas}\label{rauxedces-histuxf3ricas}}

Primero, la necesidad de remarcar con mayor nitidez el trasfondo
histórico del presente: con notables excepciones, las críticas recientes
del campo\marginnote{\textsuperscript{35} Erick R. Torrico Villanueva, ``La \emph{comunicología
  de liberación}, otra fuente para el pensamiento decolonial: una
  aproximación a las ideas de Luis Ramiro Beltrán'', \emph{Quórum
  Académico} 7 n.\textsuperscript{o} 1 (2010); Tanius Karam, ``Tensiones
  para un giro decolonial en el pensamiento comunicológico: abriendo la
  discusión'', \emph{Chasqui. Revista Latinoamericana de Comunicación}
  133 (2016); Francisco Sierra Caballero y Claudio Maldonado Rivera,
  eds., \emph{Comunicación, decolonialidad y Buen Vivir} (Quito:
  Ediciones CIESPAL, 2016); Francisco Sierra Caballero, Claudio
  Maldonado Rivera y Carlos del Valle, ``Nueva comunicología
  latinoamericana y giro decolonial: continuidades y rupturas'',
  \emph{Cuadernos de Información y Comunicación} 25 (2020); Alejandro
  Barranquero y Juan Ramos-Martín, ``Luis Ramiro Beltrán and Theorizing
  Horizontal and Decolonial Communication'', en \emph{The Handbook of
  Global Interventions in Communication Theory}, ed. de Yoshitaka Miike
  y Jing Yin (Nueva York: Routledge, 2022); Claudia Magallanes-Blanco,
  ``Media and Communication Studies: What is there to
  Decolonize?''~\emph{Communication Theory} 32, n.\textsuperscript{o} 2
  (2022).}\marginnote{\textsuperscript{36} Sobre los esfuerzos decolonizantes
  que atraviesan subcampos de los estudios de comunicación y medios,
  véanse Antje Glück, ``De-Westernization and Decolonization in Media
  Studies'', en \emph{Oxford Research Encyclopedia of Communication,}
  ed. de Jon Nussbaum (Oxford: Oxford University Press, 2018); Joëlle M.
  Cruz y Chigorzirim Utah Sodeke, ``Debunking Eurocentrism in
  Organizational Communication Theory: Marginality and Liquidities in
  Postcolonial Contexts'', \emph{Communication Theory} 31,
  n.\textsuperscript{o} 3 (2021); Mohan Dutta et al., ``Decolonizing
  Open Science: Southern Interventions'',~\emph{Journal of
  Communication} 71, n.\textsuperscript{o} 5 (2021); Bruce Mutsvairo et
  al., ``Ontologies of Journalism in the Global South'',
  \emph{Journalism \& Mass Communication Quarterly} 98,
  n.\textsuperscript{o} 4 (2021); C.S.H.N. Murthy, ``Unbearable
  Lightness? Maybe Because of the Irrelevance/Incommensurability of
  Western Theories? An Enigma of Indian Media Research'',
  \emph{International Communication Gazette} 78, n.\textsuperscript{o} 7
  (2016), y Makoni y Masters, ``Decolonization and Globalization''. Para
  más sobre el proyecto de ``desoccidentalización'', véase Silvio
  Waisbord, ``What is Next for De-Westernizing Communication Studies?''
  \emph{Journal of Multicultural Discourses}, publicación anticipada en
  línea (2022).}\marginnote{\textsuperscript{37} Eve Ng y
  Paula Gardner, ``Location, Location, Location? The Politics of ICA
  Conference Venues'', \emph{Communication, Culture \& Critique} 13,
  n.\textsuperscript{o} 2 (2020).} y de las disciplinas de la investigación de comunicación y
medios en su conjunto son de enfoque marcadamente
contemporáneo.\textsuperscript{38} Cabe notar que con este presentismo
no se trata del universalismo atemporal de las ciencias sociales
positivistas y postpositivistas, pues surgió de los análisis de
estudiosos críticos ampliamente comprometidos con las formas dialécticas
y situadas de historicidad. Autores de intervenciones recientes
reconocerían que los fenómenos que critican surgieron de procesos
históricos que les dieron forma. No obstante, en sus obras por lo
general no han tenido el cuidado de contemplar las dinámicas históricas
de la exclusión y la marginación presentes en la misma construcción y
evolución de los campos de medios y comunicación. Una explicación de
este descuido de la dimensión histórica es la división del trabajo
experto que caracteriza a la academia moderna. Otra es la urgencia
apremiante del presente. De ahí que críticas que en otros aspectos
resultan bastante convincentes ofrezcan los mismos relatos superficiales
del pasado de los campos que se reproducen en los libros de texto desde
que Wilbur Schramm propuso su mito de los ``cuatro fundadores'' en los
años sesenta, seguidos por los asesinatos simbólicos realizados por los
estudiosos críticos en los años setenta y ochenta.\textsuperscript{39} Es decir, los críticos corren el riesgo de
reproducir los clichés historiográficos que, en su engañosa simplicidad,
han contribuido a las estrechas autoconcepciones que se manejan en los
campos.

\enlargethispage{\baselineskip}

A fin de contrarrestar este presentismo, en el momento actual de
reconocimiento se requieren por lo menos tres tipos de especificidad
histórica. Primero, necesitamos analizar críticamente las dinámicas por
las que la heteromasculinidad euronorteamericana blanca capturó y
mantuvo el centro hegemónico del campo a partir del periodo de
entreguerras. Asimismo, necesitamos hacer más por recuperar y centrar
las experiencias de los integrantes minorizados del campo y resistirnos
a aceptar los recuerdos colectivos totalizantes que irónicamente ocultan
sus maniobras complejas ante la hegemonía masculinista blanca, así como
sus aportes a la producción histórica del campo. Por último, necesitamos
resistirnos a aceptar las narrativas por igual totalizantes en las que
se caracteriza la investigación de comunicación y medios como ``un campo
norteamericano'' y se pasan por alto las ricas historias de las
indagaciones en América Latina, África, Asia del Este y Europa ---cada
región con sus tradiciones intelectuales propias y sus complejas
maniobras ante la hegemonía de la investigación de corte
norteamericano---. La historiografía revisionista, así como los relatos
más establecidos publicados en lenguas que no son inglés, ya ofrecen
recursos para los tres tipos de especificidad, pero nos falta más
investigación sobre lugares geopolíticos y grupos sociales específicos,
con todo y sus imbricaciones con las historias que ya se han contado de
manera más cabal.

\hypertarget{colaboraciones-internacionales}{%
\subsection{Colaboraciones
internacionales}\marginnote{\textsuperscript{38} Entre las excepciones notables se encuentran
  Amin Alhassan, ``The Canonic Economy of Communication and Culture: The
  Centrality of the Postcolonial Margins'', \emph{Canadian Journal of
  Communication} 32, no.1 (2007); Roopali Mukherjee, ``Of Experts and
  Tokens: Mapping a Critical Race Archaeology of Communication'',
  \emph{Communication, Culture and Critique} 13, n.\textsuperscript{o} 2
  (2020), y Afonso de Albuquerque, ``The Institutional Basis of
  Anglophone Western Centrality'', \emph{Media, Culture \& Society} 43,
  n.\textsuperscript{o} 1 (2021).}\marginnote{\textsuperscript{39}\setcounter{footnote}{39} Pooley,
  ``The New History''.}\label{colaboraciones-internacionales}}

Segundo, debemos seguir construyendo colaboraciones internacionales más
robustas e incluyentes. A diferencia de la relativa falta de historia en
las críticas contemporáneas, existen muchos ejemplos de analistas que
trascienden las fronteras nacionales al rastrear los patrones de género,
raciales, coloniales y geopolíticos que siguen estructurando nuestros
campos. Vemos esfuerzos importantes al interior de nuestras asociaciones
profesionales, en los congresos grandes y más pequeños, en números
especiales de revistas especializadas y en investigaciones y
publicaciones individuales. Sin embargo, hay que reconocer a la vez la
persistencia de hábitos y estructuras institucionales añejos que
complican nuestros esfuerzos. No sorprende que algunos resultan
especialmente evidentes en el campo de Estados Unidos, por ser el centro
geopolítico de habla inglesa (al menos así lo vemos nosotros), con toda
la arrogancia e ignorancia que eso conlleva. De una estrechez
implacable, la NCA juega un papel importante en este sentido; hasta su
nombre carente de especificidad parece querer adjudicarse primacía sobre
las demás asociaciones académicas de alcance nacional. Ciertamente, sus
integrantes ven las cosas de otro modo y han hecho un trabajo esencial a
nombre de la NCA en sus revistas, congresos, listas de distribución y
otras conversaciones en redes sociales al abordar temas de blanquitud,
raza, género y hasta cierto punto colonialidad. Sin embargo, estos
mismos integrantes tienden a enfocarse en el ámbito de Estados Unidos en
su investigación y redes sociales; suelen ser además monolingües y a
menudo inconscientes de cómo su propia particularidad cultural incide en
sus análisis críticos. En las revistas de la NCA colaboran casi
exclusivamente personas ubicadas en Estados Unidos, tanto en las juntas
editoriales como entre los autores.\footnote{Fue el caso inclusive de un
  reciente número especial publicado en dos volúmenes sobre los estudios
  africanos de comunicación: no se incluían aportes de estudiosos
  radicados en universidades africanas, pero sí se publicaron ensayos
  insignes escritos por académicos africanos radicados en Estados
  Unidos. Véase Godfried A. Asante y Jenna N. Hanchey, eds.,
  ``(Re)Theorizing Communication Studies from African Perspectives'',
  Partes I y II, números especiales, \emph{Review of Communication} 21,
  n.\textsuperscript{o} 4 (2021) y 22, n.\textsuperscript{o} 1 (2022),
  respectivamente.} Existen desde luego en otras partes del mundo el
mismo tipo de mecanismos que estructuran culturas, redes e instituciones
intelectuales bajo una lógica nacional y pueden tener el mismo efecto de
impedir la colaboración internacional. Además, existen los grupos
minorizados dentro de todas las regiones del mundo y quedan excluidos de
participar de manera plena y equitativa debido a los sistemas de género,
raza, etnicidad, sexualidad y dis/capacidad. A lo que vamos es que
necesitamos concertar experiencias y capacidades diversas, provenientes
de diferentes partes del mundo, a fin de comprender la constitución de
sistemas globales de conocimientos y necesitamos crear espacios más
equitativos para que los estudiosos minorizados dentro de las regiones
del mundo escriban desde sus propios lugares y en sus propios términos.

\hypertarget{contra-la-hegemonuxeda-del-ingluxe9s}{%
\subsection{Contra la hegemonía del
inglés}\label{contra-la-hegemonuxeda-del-ingluxe9s}}

Tercero, si hemos de llevar adelante el actual reconocimiento y
desarrollar a nivel global un campo más equitativo y cosmopolita,
deberemos encontrar maneras de poner resistencia a la hegemonía del
inglés y la investigación en lengua inglesa. Se trata evidentemente de
un factor más que limita las colaboraciones internacionales más
robustas, aunque es más que eso. Como afirma Afonso de Albuquerque, a
partir de la década de los noventa el inglés se ha vuelto más poderoso a
escala global en los estudios de comunicación y medios, impulsado por
``el surgimiento de un orden mundial unipolar'' y la aceleración de la
globalización neoliberal.\footnote{De Albuquerque, ``The Institutional
  Basis'': 181.} Conforme el inglés consolidaba su lugar como la
\emph{lingua franca} indiscutible de las ciencias sociales
internacionales, se privilegiaban los hablantes nativos del inglés a la
vez que se abrían espacios para estudiosos provenientes de países ricos
del norte de Europa y otros que habían contado con los medios para
dominar el idioma.\footnote{Ana Cristina Suzina, ``English as
  \emph{Lingua Franca.} Or the Sterilisation of Scientific Work'',
  \emph{Media, Culture \& Society} 43, n.\textsuperscript{o} 1 (2021).}
La hegemonía lingüística agrega otra capa de poder y privilegio para las
revistas de Estados Unidos y Reino Unido, con sus juntas editoriales de
habla inglesa, que dominan los \emph{rankings} en los campos de la
comunicación y las ciencias sociales.\footnote{Demeter, ``The Winner
  Takes It All''.} El acelerante dominio del inglés ha contribuido a
invisibilizar en Estados Unidos y Europa la robusta tradición
latinoamericana de estudios de comunicación.\footnote{Sarah Ann Ganter y
  Félix Ortega, ``The Invisibility of Latin American Scholarship in
  European Media and Communication Studies: Challenges and Opportunities
  of De-Westernization and Academic Cosmopolitanism'',
  \emph{International Journal of Communication} 13 (2019); Florencia
  Enghel y Martín Becerra, ``Here and There: (Re)Situating Latin America
  in International Communication Theory'', \emph{Communication Theory}
  28, n.\textsuperscript{o} 2 (2018).} Se puede abordar la hegemonía
global del inglés como imperialismo lingüístico, vinculado a lo que los
lingüistas han denominado como \emph{lingüicismo} ---prácticas con
fundamento ideológico que perpetúan la repartición desigual de recursos
y poder entre grupos con base en el idioma---.\footnote{Robert
  Phillipson, \emph{Linguistic Imperialism Continued} (Londres:
  Routledge, 2009); Phillipson y Skutnabb-Kangas, ``Communicating in
  `Global English'\,''.} Toca encontrar maneras de abordar el
imperialismo lingüístico sin marginar a las personas que viven en países
de habla inglesa y que han carecido del capital cultural para aprender
una segunda lengua.

Aceptar como condición inalterable el dominio de la producción académica
anglófona sería una claudicación. La traducción y la interpretación son
herramientas necesarias para el trabajo que proponemos. Aunque ambas
pueden resultar imprácticas por caras, se están desarrollando
herramientas prometedoras de traducción automática, como DeepL, y muchas
plataformas que facilitan las reuniones en tiempo real (incluyendo Zoom)
permiten la interpretación simultánea. Por supuesto, no existe una
solución técnica fácil para el problema del lingüicismo, pero uno de los
obstáculos más relevantes a la superación del problema ---el supuesto de
que la traducción e interpretación son innecesarias o inalcanzables---
resulta manifiestamente falso.

\hypertarget{publicaciuxf3n-de-acceso-abierto}{%
\subsection{Publicación de acceso
abierto}\label{publicaciuxf3n-de-acceso-abierto}}

En cuarto y último lugar, necesitamos promover el acceso igualitario a
la producción académica, para lectores y autores por igual. La industria
editorial académica, dominada por un puñado de empresas gigantes del
Norte Global, ha ayudado a agudizar la geografía de exclusión al
restringir el acceso a quienes puedan pagar las onerosas cuotas de
acceso. Las editoriales de acceso abierto sin fines de lucro, que no
cobran por publicar ni por leer artículos, representan una respuesta
importante a este régimen cerrado de conocimiento. En este sentido, los
campos de los estudios de medios y comunicación deben seguir la
iniciativa de una tradición que ya está bien arraigada en América
Latina, de publicar en revistas de acceso abierto.\footnote{Dominique
  Babini, ``Toward a Global Open-Access Scholarly Communications System:
  A Developing Region Perspective'', en \emph{Reassembling Scholarly
  Communications: Histories, Infrastructures, and Global Politics of
  Open Access}, ed. de Martin Paul Eve y Jonathan Gray (Cambridge: MIT
  Press, 2020); Michelli Pereira da Costa y Fernando César Lima Leite,
  ``Open Access in the World and Latin America: A Review Since the
  Budapest Open Access Initiative'', \emph{Transinformação} 28,
  n.\textsuperscript{o} 1 (2016).} Las instituciones europeas y
norteamericanas han facilitado ---conscientemente o no--- la cínica
cooptación del movimiento de acceso abierto por parte de las editoriales
comerciales.\footnote{Marcel Knöchelmann, ``The Democratisation Myth:
  Open Access and the Solidification of Epistemic Injustices'',
  \emph{Science \& Technology Studies} 34, n.\textsuperscript{o} 2
  (2021); Richard Poynder, ``Open Access: Could Defeat Be Snatched from
  the Jaws of Victory?'' \emph{Open and Shut?}, 18 noviembre 2019.} Su
táctica ha sido cambiar las cuotas extorsionistas de subscripción por
cargos usurarios para publicar artículos, esto es, poner barreras de
pago para autores en vez de para lectores.\footnote{Audrey C. Smith et
  al., ``Assessing the Effect of Article Processing Charges on the
  Geographic Diversity of Authors Using Elsevier's `Mirror Journal'
  System'', \emph{Quantitative Science Studies} 2, n.\textsuperscript{o}
  4 (2021); Alicia Kowaltowski, Michel Naslavsky y Mayana Zatz, ``Open
  Access Is Closed to Middle-Income Countries'', \emph{Times Higher
  Education}, 14 abril 2022.} Los académicos latinoamericanos han
lanzado una campaña global contra este régimen corporativo de acceso
abierto basado en cuotas, argumentando que las cuotas de más de 3,000
dólares para publicar un artículo constituyen una exclusión \emph{de
facto} del Sur Global.\footnote{Arianna Becerril-García, ``The
  Commercial Model of Academic Publishing Underscoring Plan S Weakens
  the Existing Open Access Ecosystem in Latin America'', \emph{LSE
  Impact} (blog), 20 de mayo de 2020; Kathleen Shearer y Arianna
  Becerril-García, ``Decolonizing Scholarly Communications through
  Bibliodiversity'', \emph{Zenodo}, 7 de enero de 2021.} El modelo
latinoamericano sin fines de lucro y sin cuotas ofrece una alternativa
floreciente, apoyada por el financiamiento colectivo. Últimamente,
personas que se dedican a la investigación de medios y comunicación en
América Latina y otras regiones han difundido súplicas a campos
específicos con el fin de frenar la transición al acceso abierto
comercial que excluye a autores y amenaza con agravar las inequidades
globales de conocimiento.\footnote{Thaiane Moreira de Oliveira et al.,
  ``Toward an Inclusive Agenda of Open Science for Communication
  Research: A Latin American Approach'', \emph{Journal of Communication}
  71, n.\textsuperscript{o} 5 (2021) ; Dutta et al., ``Decolonizing Open
  Science''.} En plan de contraataque se propone crear colaboraciones
internacionales multilingües entre revistas de acceso abierto que no
cobran cuotas; \emph{History of Media Studies} actualmente está
piloteando la propuesta con
\href{http://www.comunicacionysociedad.cucsh.udg.mx/index.php/comsoc}{\emph{Comunicación
y Sociedad}} y
\href{https://www.revistas.usp.br/matrizes/}{\emph{MATRIZes}}\emph{.}

\hypertarget{registros-de-exclusiuxf3n}{%
\section{Registros de exclusión}\label{registros-de-exclusiuxf3n}}

Se organizó el año pasado la
\href{https://hms.mediastudies.press/pub/schedule/}{Preconferencia} ICA
precisamente con estas propuestas en mente: mapear la exclusión y llevar
adelante el trabajo de recuperación. La reunión virtual de dos días
convocó a dos docenas de estudiosos de todo el mundo para un diálogo en
torno a los trabajos que, en forma revisada, se publican en esta sección
especial. Se pensaba desde un principio que la preconferencia y la
revista estarían vinculadas. Un objetivo era plasmar nuestra visión para
\emph{History of Media Studies} como un sitio donde se amplíe la
historiografía. La
\href{https://hms.mediastudies.press/pub/precon-cfp/}{convocatoria},
lanzada en español, portugués e inglés, explicitó el objetivo de lograr
la participación de investigadores de América Latina y que se trabajara
dentro de las tradiciones de la región. Casi la mitad de los trabajos
estaba escrita en español, la otra mitad en inglés. Quienes organizaron
el evento se valieron de DeepL para generar traducciones funcionales que
los participantes pudieran leer por anticipado. La preconferencia misma
luego se valió de intérpretes simultáneos en vivo para facilitar lo que
resultó ser un diálogo bilingüe apasionante. La viveza del intercambio
nos dejó motivados y convencidos de que la nueva revista podría, con
mucho trabajo y colaboración, plantear como su pilar central el
descentramiento de las historias de los campos. Todos caímos en la
cuenta de que el formato virtual fue un elemento clave para el
intercambio intelectual. Los altos costos de los viajes a congresos
representan, a fin de cuentas, un mecanismo importante de exclusión, en
particular a lo largo de las fallas que separan a Norte y Sur. Se
ahorraron estos costos y, con fondos procurados junto con el cobro
renunciable de un modesto pago de inscripción a la conferencia, se logró
pagar los servicios de los intérpretes profesionales ---cuyo trabajo
dependía a su vez del \emph{software} de bajo costo para conferencias
que permitió que el grupo se reuniera de manera remota---.

Fue la experiencia de la preconferencia lo que impulsó a \emph{History
of Media Studies} a lanzarse como revista multilingüe, con manuscritos
\href{https://hms.mediastudies.press/author-guidelines}{aceptados tanto
en español como en inglés}. Guiados por nuestra
\href{https://hms.mediastudies.press/editorial}{Junta Editorial},
pensamos abrirnos con el tiempo a otros idiomas, priorizando literaturas
y tradiciones lingüísticas excluidas por la acelerante hegemonía de la
lengua inglesa en nuestros campos. De los siete artículos de esta
sección especial, tres se publican en español y cuatro en inglés, y esta
introducción aparece en ambas lenguas. Los trabajos abordan una variedad
de exclusiones, atravesando geografías y ámbitos intelectuales. Hay
dimensiones cruciales de exclusión y oclusión, sobre todo en las líneas
de raza y género, que se tratan solo indirectamente en esta colección,
lo que refleja el enfoque de los trabajos que se presentaron como fruto
de la preconferencia. En ese sentido vemos la sección especial como un
primer paso, una especie de pagaré de la historiografía incluyente del
campo que la revista pretende incubar. Cada uno de los artículos, con su
enfoque particular, representa el tipo de trabajo que pensamos publicar
en los números futuros ---investigación en la que se aplica la
sensibilidad histórica para el reconocimiento del campo por sus
omisiones e inequidades---.

El aporte de Sarah Cordonnier aborda el campo mismo como objeto de
exclusión. Ella observa que los estudios de medios, comunicación y cine
se establecieron de maneras distintas en todo el mundo. De ahí que las
historias institucionales e intelectuales de estos campos son muy
diferentes según el contexto nacional y regional. Sin embargo, lo que
comparten estas diversas formaciones de los estudios de medios es la
experiencia de la marginalidad. Los estudiosos de comunicación, en un
campo nacional tras otro, han sido relegados a la periferia de sus
respectivas universidades, donde han tenido que conformarse con un
estatus inferior. En el artículo de Cordonnier se registra este patrón
de estigmatización, rastreable en parte al crecimiento institucional
acelerado y tardío de los campos. La autora lanza un llamado a los
campos a revertir su postura defensiva ---a acoger las mismas
condiciones, incluyendo la proximidad a la vida cotidiana y la tremenda
heterogeneidad de las prácticas de conocimiento, que sirvieran para
socavar su legitimidad---.

El artículo de Daniel Horacio Cabrera Altieri sobre el ``imaginario
textil'' puede leerse como una extensión y profundización de la
conclusión de Cordonnier. Cabrera Alitieri plantea la práctica y la
metáfora del tejido como una alternativa por largo tiempo enterrada y
con perspectiva de género distinta de las concepciones de la
comunicación fundadas en imágenes de transporte y de redes que han
predominado en el campo organizado. La preocupación de los teóricos de
la comunicación por la racionalidad discursiva y por el despliegue
progresivo de nuevos medios, ha ayudado a ocultar una rica tradición
alterna que Cabrera Altieri desarrolla prestando particular atención a
las culturas indígenas de América Latina. Su proyecto consiste en
recuperar la memoria y excavar los trazos subterráneos, excluidos por la
``amnesia textil'' de los campos. Como imaginario rival, el textil
sugiere una ética de cuidado, de entretejido, centrada en el tejido
social.

Una fuente largamente suprimida de concepciones alternativas de la
comunicación, incluyendo la metáfora fundamental del tejido, es la
experiencia indígena en América Latina. Como lo describe María Magdalena
Doyle en su aportación, la indigeneidad fue un área descuidada de
estudio dentro de las emergentes disciplinas nacionales de comunicación
en la región---al igual, cabe señalarlo, que en otras partes del
mundo---. A partir de los años setenta, los estudiosos latinoamericanos
empezaron a estudiar la comunicación y las prácticas mediáticas de los
pueblos indígenas, pero típicamente a través de los lentes antagonistas
de la modernización o la lucha de clases. Doyle mapea un cambio de
enfoque de la investigación a partir de mediados de los años ochenta,
reflejo en parte del incipiente reconocimiento de la identidad
particular de los pueblos indígenas en los ámbitos nacionales e
internacionales. Para la década de los noventa surgió una corriente de
trabajo en la que se empezó a plasmar un imaginario decolonial con base
en la comunicación y la lucha política de los indígenas, una que brindó
epistemologías alternativas, mismas que se han desarrollado en una
creciente producción académica.

El trabajo de Emiliano Sánchez Narvarte sigue otro hilo dentro de la
investigación latinoamericana de la comunicación: la interacción de la
región con las organizaciones internacionales y la circulación
transnacional de la investigación. A partir de finales de los años
setenta, el estudioso venezolano Antonio Pasquali desempeñó una serie de
puestos en la UNESCO. Desde ahí, y por medio de su densa red de vínculos
con otros investigadores latinoamericanos, Pasquali ayudó a poner a la
región en contacto con instituciones como la UNESCO y la AIECS, que se
dedicaban en aquel entonces a desafiar la estrechez no reconocida de la
investigación de la comunicación que se realizaba en Estados Unidos.
Sánchez Narvarte traza la política de la comunicación en América Latina
entre 1979 y 1989 y coloca a Pasquali como el \emph{mediador
intelectual},\footnote{El concepto de \emph{mediador} intelectual se
  toma de Mariano Zarowsky, \emph{Del laboratorio chileno a la
  comunicación--mundo. Un itinerario intelectual de Armand Mattelart}
  (Buenos Aires: Biblos, 2013).} cuyo papel conectivo ayudó a su vez a
consolidar la conciencia regional del campo en nuevos espacios como la
Asociación Latinoamericana de Investigadores de la Comunicación.

En los años setenta y ochenta, la UNESCO y la AIECS ayudaron a coordinar
a estudiosos críticos de todo el mundo en lo que fue, para algunos
investigadores por lo menos, un proyecto deliberado encaminado a
construir alternativas a la tradición estadounidense de los efectos de
los medios. En su artículo, Maria Löblich, Niklas Venema y Elisa Pollack
relatan el apogeo y la caída de la investigación crítica en el
invernadero de la guerra fría que fuera Berlín Occidental. Como secuela
de los acontecimientos de 1968, estudiantes izquierdistas de la Freie
Universität ayudaron a apoyar nuevas contrataciones y un currículo
renovado en el que se mezclaba la teoría crítica con el desarrollo de
habilidades. Utilizando el marco de la sociología de la ciencia de
Pierre Bourdieu, Löblich, Venema y Pollack dan cuenta de cómo la
política febril y la retórica enaltecida del anticomunismo pronto
impulsaron al gobierno de Berlín Occidental a realizar una reingeniería
que en efecto dio fin a la fugaz tradición crítica de la universidad.

Asimismo, la política de la producción académica sobre la comunicación
constituye el enfoque de Angela Xiao Wu, en su relato de cómo la
disciplina en China se distinguió de sus pares al apropiarse de la
cibernética y la teoría de sistemas en la década postMao de los ochenta.
Los estudiosos del periodismo, en particular, integraron la cibernética
con el cientificismo de la dialéctica de la naturaleza de Friedrich
Engels para crear lo que Wu llama ``periodismo de sistemas''. La medida
de las noticias no era su correspondencia a la realidad sino sus aportes
a la estabilidad global del sistema. Para principios de los años
noventa, el campo chino, \emph{xiwen chuanbo} (estudio de periodismo
comunicación) se designó como disciplina de primer nivel, debido en
parte a la improbable aleación de Engels y la teoría de sistemas ---una
adaptación creativa, como lo demuestra Wu, a las condiciones locales
complejas---. Cabe señalar que con dichas condiciones locales se
rescatan precisamente el tipo de contextos geográficos y geopolíticos
que han quedado excluidos o marginados en gran parte de la
historiografía existente.

En la última aportación de la sección especial, Boris Mance y Sašo
Slaček Brlek se basan en un análisis cuantitativo de la red de ocho
revistas de lengua inglesa para mapear cómo los campos de la
comunicación abordan la inequidad. Como lo demuestran los autores, el
tema ha quedado relegado a los márgenes de las disciplinas desde la
segunda guerra mundial. En los escasos abordajes de la inequidad que
aparecen en la literatura, se ha configurado en función de contextos más
amplios más allá del campo, como la rivalidad de la guerra fría o,
después, la política de Estados Unidos con respecto a la Internet. Mance
y Brlek concluyen que la inequidad, como tema de investigación, se ha
domesticado, se le han extirpado las garras ---efecto secundario, según
los autores, de los vínculos estrechos del campo institucionalizado con
los intereses administrativos de los estados poderosos como Estados
Unidos---.

Los trabajos recolectados aquí apuntan hacia el doble carácter de las
exclusiones de los campos. Estos patrones de omisión y comisión son, en
un primer registro, \emph{constitutivos} de las formaciones
disciplinarias que se nos han legado y que hemos reproducido. Es decir,
los estudios de medios, cine y comunicación fueron configurados de
manera fundamental por silenciamientos, privilegios, olvidos e
impugnaciones. El centro no marcado y la periferia excluida, en un
sentido importante, se han cocreado. Sin embargo, en un segundo registro
estas exclusiones representan acontecimientos \emph{contingentes}. No
existe una ley férrea de la dependencia académica, ni hay una
trayectoria predeterminada de sobreextensión hegemónica.

Tomando en cuenta la conversación que se tuvo en la preconferencia, así
como los artículos reunidos aquí, proponemos que una historiografía
influida por ambos registros ---el constitutivo y el contingente---
podría contribuir a un reconocimiento tentativo del campo con su pasado
y su presente. Al enmarcar estas exclusiones como constitutivas se
descartan ciertas soluciones fáciles consistentes en la simple
inclusividad; en lugar de eso, se nos invita a considerar todas las
maneras en que estas y otras exclusiones han servido para centrar
ciertos problemas, teorías, lenguajes, naciones, identidades sociales y
lugares de publicación, y para excluir o marginar otros que se plantean
como de un valor diferenciado menor, de un estatus inferior, como Otro y
más. Al enmarcarlas como constitutivas también se arroja luz sobre la
manera en que dichas exclusiones se ejecutan performativamente de manera
permanente a través de toda una gama de prácticas sociales y
epistemológicas, por medio de las cuales el campo se (re)produce.

Por su parte, la promesa del marco de la contingencia es la de cultivar
una sensibilidad ante las diversas formaciones, literaturas y
conocimientos alternativos que, desde los inicios múltiples de los
campos, siempre han existido a la sombra de los ámbitos mejor
financiados, más visibles y lingüísticamente privilegiados. En este
segundo registro, la historiografía de los estudios de medios podría
ayudar a impedir una consecuencia no intencionada de algunas
intervenciones críticas recientes. Al repetir ciertos clichés
históricos, inclusive con la intención de tumbarlos, corremos el riesgo
de dejar invisibilizadas muchas de las tradiciones heterodoxas y de
oposición que existen en los campos, enterradas bajo el olvido crónico
que se pretende revertir con los trabajos de esta sección. Mientras nos
esforcemos por levantar una carpa más incluyente para el estudio de los
medios, las historias de nuestros campos podrán ayudar a abonar el suelo
bajo esa carpa. Ese trabajo apenas inicia.\footnote{Gracias a Karen Lee
  Ashcraft por sus lúcidos comentarios sobre un borrador anterior de
  esta introducción, a Raúl Fuentes Navarro por su orientación sobre la
  formación en comunicación en las instituciones jesuitas de América
  Latina y a Joëlle M. Cruz por sus sugerencias sobre la
  contextualización de los estudios negros en Estados Unidos en relación
  con la evolución intelectual en África.}







\section{Bibliography}\label{bibliography}

\begin{hangparas}{.25in}{1} 



Alatas, Syed Farid. ``Academic Dependency and the Global Division of
Labour in the Social Sciences''. \emph{Current Sociology} 51,
n.\textsuperscript{o} 6 (2003):
599--613.~\url{https://doi.org/10.1177/00113921030516003}.

Alatas, Syed Hussein. ``Academic Imperialism''. Conferencia magistral
pronunciada ante la Conferencia Regional para el Sureste de Asia de la
Asociación Internacional de Sociología, reimpresa en \emph{Reflections
on Alternative Discourses for Southeast Asia}, editado por Syed Farid
Alatas, 32--46. Singapur: Centre for Advanced Studies, 2001.

Alhassan, Amin. ``The Canonic Economy of Communication and Culture: The
Centrality of the Postcolonial Margins''. \emph{Canadian Journal of
Communication} 32, n.\textsuperscript{o} 1 (2007): 103--18.
\url{https://doi.org/10.22230/cjc.2007v32n1a1803}.

Alo, Oladimeji I. ``Contemporary Convergence in Sociological Theories:
The Relevance of the African Thought System in Theory Formation''.
\emph{Présence Africaine}, n.\textsuperscript{o} 126 (1983): 34--57.
\\\hspace{.21in}\href{http://www.jstor.org/stable/3539695}{https://www.jstor.org/stable/24351389}.

Altbach, Philip G. ``Servitude of the Mind? Education, Dependency, and
Neocolonialism''. \emph{Teachers College Record} 79,
n.\textsuperscript{o} 2 (1977):
1--11.~\url{https://doi.org/10.1177/016146817707900201}.

Asante, Godfried A. y Jenna N. Hanchey, eds. ``(Re)Theorizing
Communication Studies from African Perspectives, Part I''. Número
especial, \emph{Review of Communication} 21, n.\textsuperscript{o} 4
(2021). \url{https://www.tandfonline.com/toc/rroc20/21/4}.

Asante, Godfried A. y Jenna N. Hanchey, eds. ``(Re)Theorizing
Communication Studies from African Perspectives, Part II''. Número
especial, \emph{Review of Communication} 22, n.\textsuperscript{o} 1
(2022). \url{https://www.tandfonline.com/toc/rroc20/22/1}.

Ashcraft, Karen Lee y Peter Simonson. ``Gender, Work, and the History of
Communication Research: Figures, Formations, and Flows''. En Simonson y
Park, \emph{The International History}, 47--68.

Averbeck-Lietz, Stefanie, ed. \emph{Kommunikationswissenschaft im
Internationalen Vergleich: Transnationale Perspektiven}. Wiesbaden:
Springer Fachmedien Wiesbaden, 2017.

Ayish, Mohammad I. ``Communication Studies in the Arab World''. En
Simonson y Park, \emph{The International History}, 474--93.

Babini, Dominique. ``Toward a Global Open-Access Scholarly
Communications System: A Developing Region Perspective''. En
\emph{Reassembling Scholarly Communications: Histories, Infrastructures,
and Global Politics of Open Access}, editado por Martin Paul Eve y
Jonathan Gray, 331--41. Cambridge: MIT Press, 2020.
\url{https://doi.org/10.7551/mitpress/11885.003.0033}.

Barranquero, Alejandro y Juan Ramos-Martín. ``Luis Ramiro Beltrán and
Theorizing Horizontal and Decolonial Communication''. En \emph{The
Handbook of Global Interventions in Communication Theory}, editado por
Yoshitaka Miike y Jing Yin, 298--309. Nueva York: Routledge, 2022.

Becerril-García, Arianna. ``The Commercial Model of Academic Publishing
Underscoring Plan S Weakens the Existing Open Access Ecosystem in Latin
America''. \emph{LSE Impact} (blog), 20 de mayo de 2020.
\href{https://blogs.lse.ac.uk/impactofsocialsciences/2020/05/20/the-commercial-model-of-academic-publishing-underscoring-plan-s-weakens-the-existing-open-access-ecosystem-in-latin-america/}{https://blogs.lse.ac.uk/impactofsocialsciences/2020/05/20/the-commercial-model-of-academic-publishing-underscoring-plan-s-weakens-the-existing-open-access-ecosystem-in-latin-america/}.

Bell, Nova Gordon. ``Towards an Integrated Caribbean Paradigm in
Communication Thought: Confronting Academic Dependence in Media
Research''. En \emph{Re-imagining Communication in Africa and the
Caribbean}, editado por Hopeton S. Dunn et al., 51--74. Cham, Suiza:
Palgrave Macmillan. \url{https://doi.org/10.1007/978-3-030-54169-9_4}.

Bernal Loaiza, Graciela, ed. \emph{50 años en la formación universitaria
de comunicadores, 1967--2017: génesis, desarrollo y perspectivas}.
Guadalajara: ITESO, 2018.

Bhambra, Gurminder K. ``Postcolonial and Decolonial Dialogues''.
\emph{Postcolonial Studies} 17, n.\textsuperscript{o} 2 (2014): 115--21.
\url{https://doi.org/10.1080/13688790.2014.966414}.

Boils Morales, Guillermo. ``Bibliografía sobre ciencias sociales en
América Latina''. \emph{Revista Mexicana de Sociología} 40 (1978):
349--78. .

Brar, Dhanveer Singh y Ashwani Sharma. ``What is This `Black` in Black
Studies? From Black British Cultural Studies to Black Critical Thought
in UK Arts and Higher Education''. \emph{New Formations}
n.\textsuperscript{o} 99 (2019): 88---109.
\url{http://dx.doi.org/10.3898/NEWF:99.05.2019}.

Chakravartty, Paula et al. ``\#CommunicationSoWhite''. \emph{Journal of
Communication} 68, n.\textsuperscript{o} 2 (2018):
254--66.~\url{https://doi.org/10.1093/joc/jqy003}.

Chilisa, Bagele. \emph{Indigenous Research Methodologies.} Thousand
Oaks, CA: SAGE, 2012.

Collins, Patricia Hill. ``Learning from the Outsider Within: The
Sociological Significance of Black Feminist Thought''. \emph{Social
Problems} 33, n.\textsuperscript{o} 6 (1986): S14--S32.
\url{https://doi.org/10.2307/800672}.

Corradi, Juan Eugenio. ``Cultural Dependence and the Sociology of
Knowledge: The Latin American Case''. En \emph{Ideology and Social
Change in Latin America}, editado por June Nash, Juan Eugenio Corradi y
Hobart Spaulding, Jr., 7--30. Nueva York: Gordon and Breach, 1977.

Costa, Michelli Pereira da y Fernando César Lima Leite. ``Open Access in
the World and Latin America: A Review Since the Budapest Open Access
Initiative''. \emph{Transinformação} 28, n.\textsuperscript{o} 1 (2016):
33--46. \url{https://doi.org/10.1590/2318-08892016002800003}.

Crenshaw, Kimberlé. ``Demarginalizing the Intersection of Race and Sex:
Black Feminist Critique of Antidiscrimination Doctrine, Feminist Theory,
and Antiracist Politics''. \emph{University of Chicago Legal Forum} 1989
(1989): 139--68.

Cruz, Joëlle M. y Chigorzirim Utah Sodeke. ``Debunking Eurocentrism in
Organizational Communication Theory: Marginality and Liquidities in
Postcolonial Contexts''. \emph{Communication Theory} 31,
n.\textsuperscript{o} 3 (2021): 528--48.
\url{https://doi.org/10.1093/ct/qtz038}.

Daniel, Jack L. \emph{Changing the Players and the Game: A Personal
Account of the Speech Communication Association Black Caucus Origins}.
Annandale, VA: Speech Communication Association, 1995.

De Albuquerque, Afonso. ``The Institutional Basis of Anglophone Western
Centrality''. \emph{Media, Culture \& Society} 43, n.\textsuperscript{o}
1 (2021): 180--88. \url{https://doi.org/10.1177/0163443720957893}.

De Albuquerque, Afonso y Thaiane Moreira de Oliveira. ``Thinking the
Recolonial in Communication Studies: Reflections from Latin America''.
\emph{Comunicação, Mídia, e Consumo} 18, n.\textsuperscript{o} 51
(2021). \url{http://dx.doi.org/10.18568/CMC.V18I51.2521}.

De Albuquerque, Afonso et al. ``Structural Limits to the
De-Westernization of the Communication Field: The Editorial Board in
Clarivate's \emph{JCR} System''. \emph{Communication, Culture \&
Critique} 13, n.\textsuperscript{o} 2 (2020): 185--203.
\url{https://doi.org/10.1093/ccc/tcaa015}.

De Oliveira, Thaiane Moreira et al. ``Toward an Inclusive Agenda of Open
Science for Communication Research: A Latin American Approach''.
\emph{Journal of Communication} 71, n.\textsuperscript{o} 5 (2021):
785--802. \url{https://doi.org/10.1093/joc/jqab025}.

Demeter, Márton. \emph{Academic Knowledge Production and the Global
South: Questioning Inequality and Under-Representation}. Cham, Suiza:
Palgrave Macmillan, 2020.

Demeter, Márton. ``The Winner Takes It All: International Inequality in
Communication and Media Studies Today''. \emph{Journalism \& Mass
Communication Quarterly} 96, n.\textsuperscript{o} 1 (2019): 37--59.
\url{https://doi.org/10.1177\%2F1077699018792270}.

Dorsten, Aimee-Marie. ``Women in Communication Research''. En \emph{The
International Encyclopedia of Communication Theory and Philosophy},
editado por Klaus Bruhn Jensen y Robert T. Craig. Walden, MA: Wiley
Blackwell, 2016.
\url{https://doi.org/10.1002/9781118766804.wbiect106}\emph{.}

Dutta, Mohan et al. ``Decolonizing Open Science: Southern
Interventions''. \emph{Journal of Communication} 71,
n.\textsuperscript{o} 5 (2021): 803--26.
\url{https://doi.org/10.1093/joc/jqab027}.

Ekdale, Brian et al. ``Geographic Disparities in Knowledge Production: A
Big Data Analysis of Peer-Reviewed Communication Publications from 1990
to 2019''. \emph{International Journal of Communication} 16 (2022):
2498--525. \url{https://ijoc.org/index.php/ijoc/article/view/18386}.

Enghel, Florencia y Martín Becerra. ``Here and There: (Re)Situating
Latin America in International Communication Theory''.
\emph{Communication Theory} 28, n.\textsuperscript{o} 2 (2018): 111--30.
\url{https://doi.org/10.1093/ct/qty005}.

Fuentes-Navarro, Raúl. ``Institutionalization and Internationalization
of the Field of Communication Studies in Mexico and Latin America''. En
Simonson y Park, \emph{The International History}, 325--45.

Ganter, Sarah Ann y Félix Ortega. ``The Invisibility of Latin American
Scholarship in European Media and Communication Studies: Challenges and
Opportunities of De-Westernization and Academic Cosmopolitanism''.
\emph{International Journal of Communication} 13 (2019): 68--91.
\url{https://ijoc.org/index.php/ijoc/article/view/8449}.

García-Jiménez, Leonarda y Esperanza Herrero. ``Narrating the Field
Through Some Female Voices: Women's Experiences and Stories in
Academia''. \emph{Communication Theory} 32, n.\textsuperscript{o} 2
(2022): 289--97. \url{https://doi.org/10.1093/ct/qtac002}.

García-Jiménez, Leonarda y Peter Simonson, eds. Sección especial sobre
``Female Roles, Contributions, and Invisibilities in the Field of
Communication''. \emph{Revista Mediterránea de Comunicación} 12,
n.\textsuperscript{o} 2 (2021): 17--113.
\url{https://doi.org/10.14198/MEDCOM.20163}.

Gareau, Frederick H. ``Another Type of Third World Dependency: The
Social Sciences''. \emph{International Sociology} 3,
n.\textsuperscript{o} 2 (1988): 171--78.
\url{https://doi.org/10.1177/026858088003002005}.

Glück, Antje. ``De-Westernization and Decolonization in Media Studies''.
En \emph{Oxford Research Encyclopedia of Communication}, editado por Jon
Nussbaum. Oxford: Oxford University Press, 2018.
\url{https://doi.org/10.1093/acrefore/9780190228613.013.898}

Haraway, Donna. ``Situated Knowledges: The Science Question in Feminism
and the Privilege of Partial Perspective''. \emph{Feminist Studies} 14,
n.\textsuperscript{o} 3 (1988): 575--99.
\url{https://doi.org/10.2307/3178066}.

Harris, Joseph E., ed. \emph{Global Dimensions of the African Diaspora}.
Segunda edición. Washington: Howard University Press, 1993.

Heilbron, Johan, Nicolas Guilhot y Laurent Jeanpierre. ``Toward a
Transnational History of the Social Sciences''. \emph{Journal of the
History of the Behavioral Sciences} 44, n.\textsuperscript{o} 2 (2008):
146--60. \url{https://doi.org/10.1002/jhbs.20302}.

Heilbron, Johan, Gustavo Sorá y Thibaud Boncourt, eds. \emph{The Social
and Human Sciences in Global Power Relations}. Cham, Suiza: Palgrave
Macmillan, 2018. \url{https://doi.org/10.1007/978-3-319-73299-2}.

Henriques, Julian y David Morley, eds. \emph{Stuart Hall: Conversations,
Projects and Legacies.} Londres: Goldsmiths Press, 2017.

Heram, Yamila y Santiago Gándara. \emph{Pioneras en los estudios de
comunicación}. Buenos Aires: TeseoPress, 2021.

Hornsby, Alton. ``Molefi Kete Asante/Arthur Lee Smith Jr. (1942-- )''.
\emph{BlackPast}, 20 julio 2007.
\href{https://www.blackpast.org/african-american-history/asante-molefi-kete-arthur-lee-smith-jr-1942-2/}{https://www.blackpast.org/african-american-history/asante-molefi-kete-arthur-lee-smith-jr-1942-2/}.

Jackson II, Ronald L. y Sonja M. Brown Givens. \emph{Black Pioneers in
Communication Research}. Thousand Oaks, CA: SAGE, 2016.

Jansen, Sue Curry. ``\,`The Future is Not What it Used to Be': Gender,
History, and Communication Studies''. \emph{Communication Theory} 3,
n.\textsuperscript{o} 2 (1993): 136--48.
\url{https://doi.org/10.1111/j.1468-2885.1993.tb00063.x}.

Karam, Tanius. ``Tensiones para un giro decolonial en el pensamiento
comunicológico: abriendo la discusión''. \emph{Chasqui. Revista
Latinoamericana de Comunicación} 133 (2016), 247--64.
\url{https://www.redalyc.org/articulo.oa?id=16057383017}.

Klaus, Elisabeth y Josef Seethaler, eds. \emph{What Do We Really Know
about Herta Herzog?} Fráncfort del Meno: Peter Lang, 2016.

Knöchelmann, Marcel. ``The Democratisation Myth: Open Access and the
Solidification of Epistemic Injustices''. \emph{Science \& Technology
Studies} 34, n.\textsuperscript{o} 2 (2021): 65--89.
\url{https://doi.org/10.23987/sts.94964}.

Kowaltowski, Alicia, Michel Naslavsky y Mayana Zatz. ``Open Access Is
Closed to Middle-Income Countries''. \emph{Times Higher Education}, 14
abril 2022.
\url{https://www.timeshighereducation.com/opinion/open-access-closed-middle-income-countries}.

Lander, Edgardo, ed. \emph{La colonialidad del saber: eurocentrismo y
ciencias sociales}. Buenos Aires: CLASCO, 2000.

Löblich, Maria. ``German \emph{Publizistikwissenschaft} and Its Shift
from a Humanistic to an Empirical Social Science''. \emph{European
Journal of Communication} 22, n.\textsuperscript{o} 1 (2007): 69--88.
\url{https://doi.org/10.1177\%2F0267323107073748}.

Magallanes-Blanco, Claudia. ``Media and Communication Studies: What is
there to Decolonize?''~\emph{Communication Theory} 32,
n.\textsuperscript{o} 2 (2022):
267--72.~\url{https://doi.org/10.1093/ct/qtac003}.

Makoni, Sinfree y Katherine A. Masters. ``Decolonization and
Globalization in Communication Studies''.~En \emph{Oxford Research
Encyclopedia of Communication}, editado por Jon Nussbaum. Oxford: Oxford
University Press, 2021.
\url{https://doi.org/10.1093/acrefore/9780190228613.013.1152}.

Martínez Terrero, S.J., José. "Los jesuitas de Venezuela en la
comunicación social''. \emph{Temas de Comunicación. Universidad Católica
Andrés Bello} 1 (1992): 31--46.

Mayer, Vicki et al. ``How Do We Intervene in the Stubborn Persistence of
Patriarchy in Communication Scholarship?'' En \emph{Interventions:
Communication Theory and Practice}, editado por D. Travers Scott y
Adrienne Shaw, 53--64. Nueva York: Peter Lang, 2018.

Merton, Robert K. ``The Matthew Effect in Science, II: Cumulative
Advantage and the Symbolism of Intellectual Property''. \emph{Isis} 79,
n.\textsuperscript{o} 4 (1988): 606--23. .

Meyen, Michael. ``IAMCR on the East-West Battlefield: A Study on the
GDR's Attempts to Use the Association for Diplomatic Purposes''.
\emph{International Journal of Communication} 8 (2014): 2071--89.
\url{https://ijoc.org/index.php/ijoc/article/view/2443}.

Miike, Yoshitaka y Jing Yin, eds. \emph{The Handbook of Global
Interventions in Communication Theory}. Nueva York: Routledge, 2022.

Mills, Charles W. \emph{The Racial Contract}. Ithaca, NY: Cornell
University Press, 1997.

Minielli, Maureen C. et al., eds. \emph{Media and Public Relations
Research in Post-Socialist Countries}. Lanham, MD: Lexington Books,
2021.

Morris III, Charles E. y Catherine Helen Palczewski. ``Sexing
Communication: Hearing, Feeling, Remembering Sex/Gender and Sexuality in
the NCA''. En \emph{A Century of Communication Studies: The Unfinished
Conversation}, editado por Pat J. Gehrke y William M. Keith, 128--65.
Nueva York: Routledge, 2015.

Mukherjee, Roopali. ``Of Experts and Tokens: Mapping a Critical Race
Archaeology of Communication''. \emph{Communication, Culture and
Critique} 13, n.\textsuperscript{o} 2 (2020): 152--67.
\url{https://doi.org/10.1093/ccc/tcaa009}.

Murthy, C.S.H.N. ``Unbearable Lightness? Maybe Because of the
Irrelevance/Incommensurability of Western Theories? An Enigma of Indian
Media Research''. \emph{International Communication Gazette} 78,
n.\textsuperscript{o} 7 (2016): 636--42.
\url{https://doi.org/10.1177/1748048516655713}.

Musa, Mohammad. ``Looking Backward, Looking Forward: African Media
Studies and the Question of Power''. \emph{Journal of African Media
Studies} 1, n.\textsuperscript{o} 1 (2009): 35--54.
\url{https://doi.org/10.1386/jams.1.1.35_1}.

Mutsvairo, Bruce et al. ``Different, But the Same: How the Global South
is Challenging the Hegemonic Epistemologies and Ontologies of
Westernized/Western-Centric Journalism Studies''. \emph{Journalism \&
Mass Communication Quarterly} 98, n.\textsuperscript{o} 4 (2021):
996--1016. \url{https://doi.org/10.1177\%2F10776990211048883}.

Mutua, Eddah M., Bala A. Musa y Charles Okigbo. ``(Re)visiting African
Communication Scholarship: Critical Perspectives on Research and
Theory''. \emph{Review of Communication} 22, n.\textsuperscript{o} 1
(2022): 76--92. \url{https://doi.org/10.1080/15358593.2021.2025413}.

Ng, Eve y Paula Gardner. ``Location, Location, Location? The Politics of
ICA Conference Venues''. \emph{Communication, Culture \& Critique} 13,
n.\textsuperscript{o} 2 (2020): 265--69.
\url{https://doi.org/10.1093/ccc/tcaa006}.

Ng, Eve, Khadijah Costley White y Anamik Saha. ``\#CommunicationSoWhite:
Race and Power in the Academy and Beyond''. \emph{Communication, Culture
\& Critique} 13, n.\textsuperscript{o} 2 (2020): 143--51.
\url{https://doi.org/10.1093/ccc/tcaa011}.

Peruško, Zrinjka y Dina Vozab. ``The Field of Communication in Croatia:
Toward a Comparative History of Communication Studies in Central and
Eastern Europe''. En Simonson y Park, \emph{The International History},
213--34.

Phillipson, Robert. \emph{Linguistic Imperialism Continued.} Londres:
Routledge, 2009.

Phillipson, Robert y Tove Skutnabb-Kangas. ``Communicating in `Global
English': Promoting Linguistic Human Rights or Complicit with Linguicism
and Linguistic Imperialism''. En \emph{The Handbook of Global
Interventions in Communication Theory}, editado por Yoshitaka Miike y
Jing Yin, 425--39. Nueva York: Routledge, 2022.

Pooley, Jefferson. ``The New History of Mass Communication Research''.
En \emph{The History of Media and Communication Research: Contested
Memories}, editado por David W. Park y Jefferson Pooley, 43--69. Nueva
York: Peter Lang, 2008.

Pooley, Jefferson y David W. Park. ``Communication Research''. En
\emph{The Handbook of Communication History}, editado por Peter Simonson
et al., 76--90. Nueva York: Routledge, 2013.

Poynder, Richard. ``Open Access: Could Defeat Be Snatched from the Jaws
of Victory?'' \emph{Open and Shut}?, 18 noviembre 2019.
\href{https://poynder.blogspot.com/2019/11/open-access-could-defeat-be-snatched.html}{https://poynder.blogspot.com/2019/11/open-access-could-defeat-be-snatched.html}.

Richter, Carola y Hanan Badr. ``Die Entwicklung der
Kommunikationsforschung und -wissenschaft in Ägypten. Transnationale
Zirkulationen im Kontext von Kolonialismus und Globalisierung''. En
\emph{Kommunikationswissenschaft im Internationalen Vergleich:
Transnationale Perspektiven}, editado por Stefanie Averbeck-Lietz,
383--408. Wiesbaden: Springer Fachmedien Wiesbaden, 2017.

Roby, Pamela. ``Women and the ASA: Degendering Organizational Structures
and Processes, 1964--1974''. \emph{The American Sociologist} 23 (1992):
18--48. \url{https://doi.org/10.1007/BF02691878}.

Rodríguez, Clemencia et al., eds. \emph{Mujeres de la comunicación.}
Bogotá: Friedrich Ebert Stiftung, 2020.

Rojas, Fabio. \emph{From Black Power to Black Studies: How a Radical
Social Movement Became an Academic Discipline}. Baltimore: Johns Hopkins
University Press, 2007.

Rossiter, Margaret W. ``The Matthew Matilda Effect in Science''.
\emph{Social Studies of Science} 23, n.\textsuperscript{o} 2 (1993):
325--41. \url{https://doi.org/10.1177/030631293023002004}.

Rotger, Neus, Diana Roig-Sanz y Marta Puxan-Oliva. ``Introduction:
Towards a Cross-Disciplinary History of the Global in the Humanities and
Social Sciences''. \emph{Journal of Global History} 14,
n.\textsuperscript{o} 3 (2019): 325-\/--34.

Sánchez Villaseñor, S.J., Luis. \emph{José Sánchez Villaseñor, S.J.,
1911/1961: notas biográficas}. Guadalajara: ITESO, 1997.

Sapiro, Gisèle, Marco Santoro y Patrick Baert, eds. \emph{Ideas on the
Move in the Social Sciences and Humanities: The International
Circulation of Paradigms and Theorists}. Cham, Suiza: Springer, 2020.
\url{https://doi.org/10.1007/978-3-030-35024-6}.

Schäfer, Fabian. \emph{Public Opinion, Propaganda, Ideology: Theories on
the Press and Its Social Function in Interwar Japan, 1918--1937}.
Leiden: Brill, 2012.

Schöpf, Caroline M. ``The Coloniality of Global Knowledge Production:
Theorizing the Mechanisms of Academic Dependency''. \emph{Social
Transformations} 8, n.\textsuperscript{o} 2 (2020): 5--46.
\url{https://doi.org/10.1111/johs.12355}.

Shearer, Kathleen y Arianna Becerril-García. ``Decolonizing Scholarly
Communications through Bibliodiversity''. \emph{Zenodo}, 7 de enero de
2021. \url{https://doi.org/10.5281/zenodo.4423997}.

Sierra Caballero Francisco y Claudio Maldonado Rivera, eds.
\emph{Comunicación, decolonialidad y Buen Vivir}. Quito: Ediciones
CIESPAL, 2016.

Sierra Caballero, Francisco, Claudio Maldonado Rivera y Carlos del
Valle. ``Nueva comunicología latinoamericana y giro decolonial:
continuidades y rupturas''. \emph{Cuadernos de Información y
Comunicación} 25 (2020): 225--42.
\url{http://dx.doi.org/10.5209/ciyc.68236}.

Simonson, Peter y David W. Park, eds. \emph{The International History of
Communication Study}. Nueva York: Routledge, 2016.

Simonson, Peter y David W. Park. Introducción ``On the History of
Communication Study'' a \emph{The International History of Communication
Study}, editado por Peter Simonson y David W. Park, 1--22. Nueva York:
Routledge, 2016.

Skjerdal, Terje y Keyan Tomaselli. ``Trajectories of Communication
Studies in Sub-Saharan Africa''. En Simonson y Park, \emph{The
International History}, 455--73.

Smith, Audrey C. et al. ``Assessing the Effect of Article Processing
Charges on the Geographic Diversity of Authors Using Elsevier's `Mirror
Journal' System''. \emph{Quantitative Science Studies} 2,
n.\textsuperscript{o} 4 (2021): 1123--43.
\url{https://doi.org/10.1162/qss_a_00157}.

Solovey, Mark y Christian Dayé, eds. \emph{Cold War Social Science:
Transnational Entanglements}. Cham, Suiza: Palgrave Macmillan, 2021.
\url{https://doi.org/10.1007/978-3-030-70246-5}.

Suzina, Ana Cristina. ``English as \emph{Lingua Franca.} Or the
Sterilisation of Scientific Work''. \emph{Media, Culture \& Society} 43,
n.\textsuperscript{o} 1 (2021): 171--79.
\url{https://doi.org/10.1177\%2F0163443720957906}.

Tenzin, Jinba y Chenpang Lee. ``Are We Still Dependent? Academic
Dependency Theory after 20 Years''. \emph{Journal of Historical
Sociology} 35 (2022): 2--13. \url{https://doi.org/10.1111/johs.12355}.

Torrico Villanueva, Erick R. \emph{La comunicación: pensada desde
América Latina (1960--2009)}. Salamanca: Comunicación Social, 2016.

Torrico Villanueva, Erick R. ``La \emph{comunicología de liberación},
otra fuente para el pensamiento decolonial: una aproximación a las ideas
de Luis Ramiro Beltrán''. \emph{Quórum Académico} 7,
n.\textsuperscript{o} 1 (2010): 65--77.
\url{http://revistas.luz.edu.ve/index.php/quac/article/viewFile/5046/4901}.

Towns, Armond. ``Against the `Vocation of Autopsy': Blackness and/in US
Communication Histories''. \emph{History of Media Studies} 1 (2021).
\url{https://doi.org/10.32376/d895a0ea.89f81da7}.

Trepte, Sabine y Laura Loths. ``National and Gender Diversity in
Communication: A Content Analysis of Six Journals Between 2006 and
2016''. \emph{Annals of the International Communication Association} 44,
n.\textsuperscript{o} 4 (2020): 289--311.
\url{https://doi.org/10.1080/23808985.2020.1804434}.

Turner, James E. ``Foreword: Africana Studies and Epistemology, a
Discourse in the Sociology of Knowledge''. En \emph{The Next Decade:
Theoretical and Research Issues in Africana Studies}, editado por James
E. Turner, v--xxv. Ithaca, NY: Cornell University Africana Studies and
Research Center, 1984.

Vanderstraeten, Raf y Joshua Eykens. ``Communalism and Internationalism:
Publication Norms and Structures in International Social Science''.
\emph{Serendipities: Journal for the Sociology and History of the Social
Sciences} 3, n.\textsuperscript{o} 1 (2018): 14--28.
\url{https://doi.org/10.25364/11.3:2018.1.2}.

Vassallo de Lopes, Maria Immacolata y Richard Romancini. ``History of
Communication Study in Brazil: The Institutionalization of an
Interdisciplinary Field''. En Simonson y Park, \emph{The International
History}, 346--66.

Vroons, Erik. ``Communication Studies in Europe: A Sketch of the
Situation about 1955''. \emph{Gazette} 67, n.\textsuperscript{o} 6
(2005): 495--522.
\href{file:///C:/Users/simonsop/Downloads/https/doi.org/10.1177/0016549205057541}{https//doi.org/10.1177/0016549205057541}.

Wagman, Ira. ``Remarkable Invention!'' \emph{History of Media Studies} 1
(2021). \url{https://doi.org/10.32376/d895a0ea.ef8f548f}.

Waisbord, Silvio. ``What is Next for De-Westernizing Communication
Studies?'' \emph{Journal of Multicultural Discourses}. Publicación
anticipada en línea (2022).
\url{https://doi.org/10.1080/17447143.2022.2041645}.

Wang, Xinyi et al. ``Gendered Citation Practices in the Field of
Communication''. \emph{Annals of the International Communication
Association} 45, n.\textsuperscript{o} 2 (2021): 134--53.
\url{https://doi.org/10.1080/23808985.2021.1960180}.

White, Derrick E. \emph{The Challenge of Blackness: The Institute of the
Black World and Political Activism in the 1970s}. Gainesville:
University Press of Florida, 2011.

Wiedemann, Thomas. ``Practical Orientation as a Survival Strategy: The
Development of \emph{Publizistikwissenschaft} by Walter Hagemann''. En
Simonson y Park, \emph{The International History}, 109--29.

Wiedemann, Thomas, Michael Meyen e Iván Lacasa-Mas. ``100 Years
Communication Study in Europe: Karl Bücher's Impact on the Discipline's
Reflexive Project''. \emph{Studies in Communication and Media} 7,
n.\textsuperscript{o} 1 (2018).

Wilkinson, Jeffrey S., William R. Davie y Angeline J. Taylor.
``Journalism Education in Black and White: A 50-Year Journey Toward
Diversity''. \emph{Journalism \& Mass Communication Educator} 75,
n.\textsuperscript{o} 4 (2020): 362--74.
\url{https://doi.org/10.1177\%2F1077695820935324}.

Willems, Wendy. ``Provincializing Hegemonic Histories of Media and
Communication Studies: Toward a Genealogy of Epistemic Resistance in
Africa''. \emph{Communication Theory} 24, n.\textsuperscript{o} 4
(2014): 415--34. \url{https://doi.org/10.1111/comt.12043}.

Willems, Wendy. ``Unearthing Bundles of Baffling Silences: The Entangled
and Racialized Global Histories of Media and Media Studies''.
\emph{History of Media Studies} 1 (2021).
\url{https://doi.org/10.32376/d895a0ea.52801916}.

Wynter, Sylvia. ``The Ceremony Must be Found: After Humanism''.
\emph{boundary 2} 12, n.\textsuperscript{o} 3/13, n.\textsuperscript{o}
1 (1984): 19--70. \url{https://doi.org/10.2307/302808}.

Zarowsky, Mariano. ``Communication Studies in Argentina in the 1960s and
`70s: Specialized Knowledge and Intellectual Intervention Between the
Local and the Global''. \emph{History of Media Studies} 1 (2021).
\url{https://doi.org/10.32376/d895a0ea.42a0a7aa}.

Zarowsky, Mariano. \emph{Del laboratorio chileno a la
comunicación--mundo. Un itinerario intelectual} \emph{de Armand
Mattelart}. Buenos Aires: Biblos, 2013.



\end{hangparas}


\end{document}