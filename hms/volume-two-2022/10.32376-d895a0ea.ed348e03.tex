% see the original template for more detail about bibliography, tables, etc: https://www.overleaf.com/latex/templates/handout-design-inspired-by-edward-tufte/dtsbhhkvghzz

\documentclass{tufte-handout}

%\geometry{showframe}% for debugging purposes -- displays the margins

\usepackage{amsmath}

\usepackage{hyperref}

\usepackage{fancyhdr}

\usepackage{hanging}

\hypersetup{colorlinks=true,allcolors=[RGB]{97,15,11}}

\fancyfoot[L]{\emph{History of Media Studies}, vol. 2, 2022}


% Set up the images/graphics package
\usepackage{graphicx}
\setkeys{Gin}{width=\linewidth,totalheight=\textheight,keepaspectratio}
\graphicspath{{graphics/}}

\title[Exclusions/Exclusiones]{Exclusions/Exclusiones: The Role for History in the Field’s Reckoning} % longtitle shouldn't be necessary

% The following package makes prettier tables.  We're all about the bling!
\usepackage{booktabs}

% The units package provides nice, non-stacked fractions and better spacing
% for units.
\usepackage{units}

% The fancyvrb package lets us customize the formatting of verbatim
% environments.  We use a slightly smaller font.
\usepackage{fancyvrb}
\fvset{fontsize=\normalsize}

% Small sections of multiple columns
\usepackage{multicol}

% Provides paragraphs of dummy text
\usepackage{lipsum}

% These commands are used to pretty-print LaTeX commands
\newcommand{\doccmd}[1]{\texttt{\textbackslash#1}}% command name -- adds backslash automatically
\newcommand{\docopt}[1]{\ensuremath{\langle}\textrm{\textit{#1}}\ensuremath{\rangle}}% optional command argument
\newcommand{\docarg}[1]{\textrm{\textit{#1}}}% (required) command argument
\newenvironment{docspec}{\begin{quote}\noindent}{\end{quote}}% command specification environment
\newcommand{\docenv}[1]{\textsf{#1}}% environment name
\newcommand{\docpkg}[1]{\texttt{#1}}% package name
\newcommand{\doccls}[1]{\texttt{#1}}% document class name
\newcommand{\docclsopt}[1]{\texttt{#1}}% document class option name


\begin{document}

\begin{titlepage}

\begin{fullwidth}
\noindent\LARGE\emph{Exclusions in the History of Media Studies
} \hspace{25mm}\includegraphics[height=1cm]{logo3.png}\\
\noindent\hrulefill\\
\vspace*{1em}
\noindent{\Huge{Exclusions/Exclusiones: The Role for\\\noindent History in the Field’s Reckoning\par}}

\vspace*{1.5em}

\noindent\LARGE{Peter Simonson} \href{https://orcid.org/0000-0001-7156-467X}{\includegraphics[height=0.5cm]{orcid.png}}\par}\marginnote{\emph{Peter Simonson, David W. Park, and Jefferson Pooley ``Exclusions/Exclusiones: The Role for History in the Field’s Reckoning,'' \emph{History of Media Studies} 2 (2022), \href{https://doi.org/10.32376/d895a0ea.ed348e03}{https://doi.org/ 10.32376/d895a0ea.ed348e03}.} \vspace*{0.75em}}
\vspace*{0.5em}
\noindent{{\large\emph{University of Colorado Boulder}, \href{mailto:peter.simonson@colorado.edu}{peter.simonson@colorado.edu}\par}} \marginnote{\href{https://creativecommons.org/licenses/by-nc/4.0/}{\includegraphics[height=0.5cm]{by-nc.png}}}

\vspace*{0.75em} 

\noindent{\LARGE{David W. Park} \href{https://orcid.org/0000-0001-7019-1525}{\includegraphics[height=0.5cm]{orcid.png}}\par}
\vspace*{0.5em}
\noindent{{\large\emph{Lake Forest College}, \href{mailto:park@lfc.edu}{park@lfc.edu}\par}}

\vspace*{0.75em} % third author

\noindent{\LARGE{Jefferson Pooley} \href{https://orcid.org/0000-0002-3674-1930}{\includegraphics[height=0.5cm]{orcid.png}}\par}
\vspace*{0.5em}
\noindent{{\large\emph{Muhlenberg College}, \href{mailto:pooley@muhlenberg.edu}{pooley@muhlenberg.edu}\par}}

\end{fullwidth}

\vspace*{1em}



\hypertarget{abstract}{%
\section{Abstract}\label{abstract}}

In this introduction for the special section on
``\href{https://hms.mediastudies.press/exclusions}{Exclusions in the
History of Media Studies},'' we begin by calling attention to the
constituting role that exclusion has played in the historiography of
media studies. Exclusions linked to gender, race, language, colonialism,
geopolitical location, and institutionally sanctioned privilege play
substantial roles in shaping formal and informal accounts of our fields'
pasts. The project of reversal and recovery builds on postcolonial and
decolonial thought, Afrocentric and racial critiques, feminist
scholarship, and geopolitically informed critique. One aim is to
provincialize much of the historiography of media studies. Drawing
inspiration from the deeply inter-animating contemporary critical
movements, we identify four pressing tasks for the history of media
studies: 1) to throw the present state of academic fields into sharper
historical relief, 2) to build international collaborations that
refigure what have been taken to be the ``centers'' and ``peripheries''
in media studies, 3) to find ways to resist the growing hegemony of
English in the global knowledge system, and 4) to support an open and
nonprofit publishing infrastructure. We propose that a historiography



\enlargethispage{5\baselineskip}


\vspace*{2em}

\noindent{\emph{History of Media Studies}, vol. 2, 2022}




 \end{titlepage}

\noindent informed by constitutive and contingent understandings of exclusion
represents an important way forward for the history of media studies.

\vspace*{2em}


\newthought{The histories and} present realities of academic fields are mutually
constituting. Patterns of exclusion we inherit from our pasts have
pernicious ways of persisting in our current practices as scholars,
teachers, administrators, and colleagues. They take both familiar and
newer forms, even as agents of change do the hard work of calling
attention to them and finding new, more inclusive ways forward. This
applies across academic practices, but in this introduction and the
special section that follows, we want to focus on one set of exclusions:
those tied to collective memories of our fields' pasts and to the formal
histories written about them. Without conscious efforts, exclusions
linked to gender, race, language, colonialism, geopolitical location,
and institutionally sanctioned privilege all reproduce themselves in
formal and informal accounts of our fields' pasts. Moreover, if we are
to understand the inequities that shape contemporary academic fields, we
need to do more to illuminate how they were set in motion and
perpetuated historically, and we need to focus more attention on
locations, people, and topics that have been marginalized in our
collective memories. That was the aim of a virtual preconference of the
International Communication Association (ICA) held in May 2021,
\href{https://hms.mediastudies.press/pub/schedule/release/12}{``Exclusions
in the History and Historiography of Communication Studies,''} where
early versions of the seven essays published here were read and
discussed. This introduction aims to throw that preconference and
related efforts into historical perspective and to situate the essays
within the recent historiography of media and communication studies.

The 2021 gathering reflected a larger moment in the history of the
social sciences and humanities (SSH), which are of course always
embedded within the broader societies that shape them. That moment,
whose history will one day be told, is marked by widespread reckonings
with the hegemonies and systemic inequities in academic work. The key
word is \emph{widespread.} Members of marginalized groups within the SSH
disciplines have always experienced inequity, as exposed by pointed
critiques since at least the 1960s and 1970s. These longstanding
interventions have interrogated multiple targets. What topics, methods,
and paradigms are considered central and which ones ignored or
marginalized? How do scholars and scholarship from the US and other
countries in the Global North accumulate systematic advantages that
marginalize or outright ignore the Global South? How do scholars
minoritized by gender, race, sexuality, ethnicity, language, and other
means in turn accumulate systematic disadvantages?\footnote{The
  accumulation of advantage and disadvantage was a theme pioneered by
  Robert K. Merton and captured, among other ways, in his much-used
  concept of ``the Matthew Effect.'' As he writes, ``The concept of
  cumulative advantage directs our attention to the ways in which
  initial comparative advantages of trained capacity, structural
  location, and available resources make for successive increments of
  advantage such that the gaps between the haves and have-nots in
  science (as in other domains of social life) widen until dampened by
  countervailing processes.'' Merton, ``The Matthew Effect in Science,
  II: Cumulative Advantage and the Symbolism of Intellectual Property,''
  \emph{Isis} 79, no. 4 (1988): 606.} How are they marginalized in the
social networks of well-placed white men, burdened with extra and
typically unrecognized labor, thrown into realms of knowledge where
people like them have been excluded from canons of classic texts and
``founding fathers'' of the field, and relegated to publishing work that
is uncited or otherwise ignored? As we will sketch, these are all social
practices that have been exposed and critiqued for more than half a
century, yet too few had ears to hear them. Recently, however, critical
interventions have gathered new force---intertwined with one another and
all but impossible for those occupying the fields' hegemonic centers to
ignore.

\hypertarget{a-brief-genealogy-of-the-present}{%
\section{A Brief Genealogy of the
Present}\label{a-brief-genealogy-of-the-present}}

It is important to highlight the lines of intervention that have brought
us to this juncture, in part because they have been rarely acknowledged
in the current reckonings. One vector consists of geopolitical critiques
of the social sciences and humanities first articulated in the late
1960s and 1970s. Accounts of academic imperialism, neo-colonialism, and
dependency drew upon broader versions of dependency theory and
refutations of dominant understandings of modernization.\footnote{See,
  for instance, Syed Hussein Alatas, ``Academic Imperialism,'' keynote
  address delivered before the International Sociology Association
  Regional Conference for Southeast Asia (1969), reprinted in Syed Farid
  Alatas, ed., \emph{Reflections on Alternative Discourses for Southeast
  Asia} (Singapore: Centre for Advanced Studies, 2001); Philip G.
  Altbach, ``Servitude of the Mind?: Education, Dependency, and
  Neocolonialism,''~\emph{Teachers College Record}~79, no. 2 (1977);
  Frederick H. Gareau, ``Another Type of Third World Dependency: The
  Social Sciences,'' \emph{International Sociology}~3, no. 2 (1988); and
  Syed Farid Alatas, ``Academic Dependency and the Global Division of
  Labour in the Social Sciences,''~\emph{Current Sociology}~51, no. 6
  (2003).} These paralleled efforts in the Global South to develop
``indigenous'' social science, referring to knowledge built from
epistemologies developed through local cultural traditions, rather than
refracted through the dominant paradigms of the North.\footnote{Juan
  Eugenio Corradi, ``Cultural Dependence and the Sociology of Knowledge:
  The Latin American Case,'' in \emph{Ideology and Social Change in
  Latin America}, ed. June Nash, Juan Corradi, and Hobart Spaulding Jr.
  (New York: Gordon and Breach, 1977); Guillermo Boils Morales,
  ``Bibliografía sobre ciencias sociales en América} When
media scholars in Western Europe and the US made efforts in the 1990s
and 2000s to ``de-Westernize'' and ``internationalize'' media studies,
they rarely connected their efforts to these broader traditions. As
Wendy Willems has observed, their aims were ``more about extending the
coverage of academic inquiry on media and communication to countries not
ordinarily included in the Western canon than about questioning the
centrality of Western theory.''\textsuperscript{4} One ironic
consequence of calls to ``de-Westernize'' the field was to occlude
existing intellectual traditions in Africa and elsewhere. When they come
from advantaged scholars in the Global North, calls to de-Westernize
risk erasing local histories in ways that reproduce longstanding
colonial patterns. We are aware, indeed, that this Introduction could
fall into the same pattern, a danger we have in front of mind.

In recent decades, geopolitical critiques of global social science and
humanities have gathered considerable momentum outside communication and
media studies and, more recently, within it. Postcolonial and decolonial
thinkers have provincialized the universal aspirations of\marginnote{Latina,''~\emph{Revista Mexicana de Sociología}~40 (1978); Oladimeji
  I. Alo, ``Contemporary Convergence in Sociological Theories: The
  Relevance of the African Thought-System in Theory
  Formation,''~\emph{Présence Africaine}, no. 126 (1983). We use the
  lower-case \emph{indigenous} to refer to knowledge systems generated
  within a particular, typically geopolitically marginalized region, and
  capitalize \emph{Indigenous} to refer to cultures and peoples who
  trace their histories to pre-colonial or pre-settler societies.} European\marginnote{\textsuperscript{4} Wendy Willems,
  ``Provincializing Hegemonic Histories of Media and Communication
  Studies: Toward a Genealogy of Epistemic Resistance in Africa,''
  \emph{Communication Theory} 24, no. 4 (2014): 416. See also Afonso de
  Albuquerque and Thaiane de Oliveira, ``Thinking the Recolonial in
  Communication Studies: Reflections from Latin America,''
  \emph{Comunicação, Mídia e Consumo} 18, no. 51 (2021).}\setcounter{footnote}{4}
modernity, including its standards of rationality and knowledge,
connecting them to ``the broader histories of colonialism, empire, and
enslavement,'' as Gurminder Bhambra writes.\footnote{Gurminder K.
  Bhambra, ``Postcolonial and Decolonial Dialogues,'' \emph{Postcolonial
  Studies} 17, no. 2 (2014): 115. For a key interdisciplinary Latin
  American volume, see Edgardo Lander, ed., \emph{La colonialidad del
  saber: Eurocentrismo y ciencias sociales} (Buenos Aires: CLASCO,
  2000); and for a useful account of postcolonialism and/versus
  decoloniality from the horizons of media and communication studies,
  see Sinfree Makoni and Katherine A. Masters, ``Decolonization and
  Globalization in Communication Studies,''~in \emph{Oxford Research
  Encyclopedia of Communication}, ed. Jon Nussbaum (Oxford: Oxford
  University Press, 2021).} These critiques overlap with established and
growing Indigenous intellectual traditions of activism, critical
analysis, and forms of knowledge and culture that stand as
counter-hegemonic alternatives to Western, colonial thought.\footnote{See,
  for instance, Bagele Chilisa, \emph{Indigenous Research Methodologies}
  (Thousand Oaks, CA: SAGE, 2012).} Decolonial and Indigenous thinking
have also informed a new wave of academic dependency theory.\footnote{Dependency
  theory is synthesized in Caroline M Schöpf, ``The Coloniality of
  Global Knowledge Production: Theorizing the Mechanisms of Academic
  Dependency,'' \emph{Social Transformations} 8, no. 2 (2020); and Jinba
  Tenzin and Chenpang Lee, ``Are We Still Dependent? Academic Dependency
  Theory after 20 Years,'' \emph{Journal of Historical Sociology} 35
  (2022).} That wave has joined the chorus of voices calling out the
impacts of neoliberal globalization on academic production and
geopolitical status hierarchies. As multiple studies have shown, global
rankings of universities, journals, and impact factors all favor the US
and Western Europe.\footnote{Márton Demeter, ``The Winner Takes It All:
  International Inequality in Communication and Media Studies Today,''
  \emph{Journalism and Mass Communication Quarterly} 96, no. 1 (2019);
  Afonso de Albuquerque et al., ``Structural Limits to the
  De-Westernization of the Communication Field: The Editorial Board in
  Clarivate's \emph{JCR} System,'' \emph{Communication, Culture \&
  Critique} 13, no. 2 (2020).} Those rankings are also tied to the
hegemony of English as the lingua franca for international social
science and the attendant pressures to publish in English, the target of
growing but still too-limited scrutiny.\textsuperscript{9} Large-scale quantitative network-analysis
methods have aided these recent efforts by providing new tools for
illuminating systemic inequities among global centers and peripheries in
publishing, citation rates, editorial board membership, and
international professional associations.\textsuperscript{10}

Intersecting with these global geopolitical critiques, Black scholars
began developing Afrocentric and other racial critiques of dominant,
Euro-North American forms of knowledge in the 1960s and 1970s. In the
US, Black Studies programs were established in the late 1960s at both
Historically Black Colleges and Universities (HBCUs) and at
predominantly white institutions.\textsuperscript{11} Their founding was driven by a mix of intellectual and
political opposition to dominant epistemologies and methodologies,
commitment to a theory of knowledge as a vehicle for social change, and
a demand that higher education better serve Black
communities.\textsuperscript{12} The Atlanta-based Institute for a
Black World, for example, drew together intellectuals from the Black
diaspora, including the Caribbean scholars Sylvia Wynter and C. L. R.
James. The Institute advanced a transnational effort that aligned with
Franz Fanon's 1965 call to ``work out new concepts, and try to set afoot
a new man.''\textsuperscript{13} The concept of
an African diaspora, which had roots in the pan-Africanisms of Marcus
Garvey and W.E.B. DuBois, emerged from a 1965 meeting in Tanzania of the
International Congress of African Historians.\textsuperscript{14} The movement took hold in US speech communication,
where African American scholars formed a Black Caucus in the Speech
Association of America in 1968 and held a Black Communication Conference
in 1972.\textsuperscript{15} In the 1980s, some of its members would organize World
Congresses on Black Communication that drew an international array of
scholars. One of the leaders of the Black Caucus was Arthur L. Smith,
who adopted the\marginnote{\textsuperscript{9} Robert Phillipson and
  Tove Skutnabb-Kangas, ``Communicating in `Global English': Promoting
  Linguistic Human Rights or Complicit with Linguicism and Linguistic
  Imperialism,'' in \emph{The Handbook of Global Interventions in
  Communication Theory}, ed. Yoshitaka Miike and Jing Yin (New York:
  Routledge, 2022). We return to the topic of English hegemony and other
  recent work on it below.} name\marginnote{\textsuperscript{10} See, for instance,
  Márton Demeter, \emph{Academic Knowledge Production and the Global
  South: Questioning Inequality and Under-Representation} (Cham,
  Switzerland: Palgrave Macmillan, 2020). For a recent analysis of the
  communication field, see Brian Ekdale et al., ``Geographic Disparities
  in Knowledge Production: A Big Data Analysis of Peer-Reviewed
  Communication Publications from 1990 to 2019,'' \emph{International
  Journal of Communication} 16 (2022).} Molefi\marginnote{\textsuperscript{11} Fabio Rojas, \emph{From Black
  Power to Black Studies: How a Radical Social Movement Became an
  Academic Discipline} (Baltimore: Johns Hopkins University Press,
  2007).} Asante\marginnote{\textsuperscript{12} James E. Turner, ``Foreword: Africana Studies and
  Epistemology, a Discourse in the Sociology of Knowledge,'' in
  \emph{The Next Decade: Theoretical and Research Issues in Africana
  Studies}, ed. James E. Turner (Ithaca, NY: Cornell University Africana
  Studies and Research Center, 1984).} while\marginnote{\textsuperscript{13} Derrick E. White, \emph{The Challenge of
  Blackness: The Institute of the Black World and Political Activism in
  the 1970s} (Gainesville, FL: University Press of Florida, 2011), 146,
  232n25, quoting Fanon's \emph{Wretched of the Earth}.} visiting\marginnote{\textsuperscript{14} Joseph E. Harris,
  ``Introduction,'' in \emph{Global Dimensions of the African Diaspora,}
  ed. Joseph E. Harris, 2nd ed. (Washington: Howard University Press,
  1993), 4.} the\marginnote{\textsuperscript{15} Jack L. Daniel, \emph{Changing the Players and the
  Game: A Personal Account of the Speech Communication Association Black
  Caucus Origins} (Annandale, VA: Speech Communication Association,
  1995). The Speech Association of America, after an intervening name
  change, became the (US) National Communication Association (NCA) in
  1997.}\setcounter{footnote}{15} University of
Ghana in 1973, and developed some of the earliest and most influential
Afrocentric critiques of Western theories of communication.\footnote{Alton
  Hornsby, ``Molefi Kete Asante/Arthur Lee Smith Jr. (1942--),''
  \emph{BlackPast,} July 20, 2007. On Asante within the contexts of
  communication studies, see Armond Towns, ``Against the `Vocation of
  Autopsy': Blackness and/in US Communication Histories,'' \emph{History
  of Media Studies} 1 (2021); and Ronald L. Jackson II and Sonja M.
  Brown Givens, \emph{Black Pioneers in Communication Research}
  (Thousand Oaks, CA: SAGE, 2016), 11--38.} Beyond communication
studies, the critic and cultural theorist Sylvia Wynter, the sociologist
Patricia Hill Collins, and the philosopher Charles W. Mills all
published work in the 1980s and '90s that would prove foundational to
the next generation of racial critiques of dominant, European forms of
knowledge and the academic communities that perpetuated them.\textsuperscript{17}

Arising from other social and intellectual quarters, feminist scholars
exposed systems of exclusion in the social sciences and humanities
organized around gender, sexuality, and intersectionality. That story
begins taking shape in the 1960s and 1970s, as women's movements and
second wave feminism took root around the world. Those developments
unfolded differently across cultures and national contexts, and their
global history in academia has not yet been written. In some contexts,
women began entering the professoriate in increasing numbers in the
1970s. They formed women's caucuses within professional associations,
which along with emerging LGBTQ caucuses, began to challenge the
gendering of academic conferences while providing supportive networks to
share experiences of marginalization, inequity, and everyday
struggle.\textsuperscript{18} In the 1970s
and 1980s, feminists across the social sciences and humanities offered
trenchant critiques of the taken-for-granted assumptions of their fields
and began reorienting the objects and processes of knowing. This fed
concerted attention to the social dynamics of knowledge production
within academic fields, with concepts like Donna Haraway's situated
knowledges and Margaret Rossiter's Matilda Effect providing powerful
cross-disciplinary tools.\textsuperscript{19} Sue Curry Jansen drew upon these insights to open
critical space for understanding gender not as incidental to, but
constitutive of, the history of the communication fields and their
dynamic (re)production of knowledge and power in the present.\textsuperscript{20} At the same time, in the late 1980s, building on what
Patricia Hill Collins called the longstanding recognition of ``the
interlocking nature of race, gender, and class oppression'' in Black
feminist thought, Kimberlé Crenshaw formulated the concept of
intersectionality that has been such a crucial lens for investigating
the entanglements of gender, race, class, and other
positionalities.\textsuperscript{21}

How have geopolitical, racial, and feminist critiques of knowledge
fields shaped written histories of media and communication studies?
Slowly and indirectly. First, it is important to remember that, until
the 1990s, there was no significant body of work on the history of those
fields, only scattered efforts and a mythos of ``founding fathers'' and
the critical sons who symbolically killed them.\textsuperscript{22} A\marginnote{\textsuperscript{17} For
  instance, Sylvia Wynter, ``The Ceremony Must be Found: After
  Humanism,'' \emph{boundary 2} 12, no. 3/13, no. 1 (1984); Patricia
  Hill Collins, ``Learning from the Outsider Within: The Sociological
  Significance of Black Feminist Thought,'' \emph{Social Problems} 33,
  no. 6 (1986); Charles W. Mills, \emph{The Racial Contract} (Ithaca,
  NY: Cornell University Press, 1997).} systematic\marginnote{\textsuperscript{18} See, for instance, Pamela Roby, ``Women and the ASA:
  Degendering Organizational Structures and Processes, 1964--1974,''
  \emph{The American Sociologist 23} (1992). For the case of the US
  National Communication Association, see Charles E. Morris III and
  Catherine Helen Palczewski, ``Sexing Communication: Hearing, Feeling,
  Remembering Sex/Gender and Sexuality in the NCA,'' in \emph{A Century
  of Communication Studies: The Unfinished Conversation}, ed. Pat J.
  Gehrke and William M. Keith (New York: Routledge, 2015).} review\marginnote{\textsuperscript{19} Donna Haraway, ``Situated Knowledges:
  The Science Question in Feminism and the Privilege of Partial
  Perspective,'' \emph{Feminist Studies} 14, no. 3 (1988); Margaret W.
  Rossiter, ``The Matilda Effect in Science,''~\emph{Social Studies of
  Science}~23, no. 2 (1993). Rossiter's was a powerful expansion of
  Robert K. Merton's concept of the Matthew Effect, referenced above in
  footnote 1.} of\marginnote{\textsuperscript{20} Sue
  Curry Jansen, ``\,`The Future is Not What it Used to Be': Gender,
  History, and Communication Studies,'' \emph{Communication Theory} 3,
  no. 2 (1993).} the\marginnote{\textsuperscript{21} Collins, ``Learning from the Outsider
  Within,'' S19; Kimberlé Crenshaw, ``Demarginalizing the Intersection
  of Race and Sex: A Black Feminist Critique of Antidiscrimination
  Doctrine, Feminist Theory and Antiracist Politics,'' \emph{University
  of Chicago Legal Forum} 1989 (1989).} English-language\marginnote{\textsuperscript{22} Jefferson
  Pooley, ``The New History of Mass Communication Research,'' in
  \emph{The History of Media and Communication Research: Contested
  Memories}, ed. David W. Park and Jefferson Pooley (New York: Peter
  Lang, 2008); Peter Simonson and David W. Park, ``Introduction: On the
  History of Communication Study,'' in \emph{The International History
  of Communication Study}, ed. Peter Simonson and David W. Park (New
  York: Routledge, 2016).}\setcounter{footnote}{22}
literature two decades later revealed an overwhelming geographical focus
on people and ideas from the US, Canada, the United Kingdom, and
Germany.\footnote{Jefferson Pooley and David W. Park, ``Communication
  Research,'' in \emph{The Handbook of Communication History}, ed. Peter
  Simonson et al. (New York: Routledge, 2013).} The Global South and
other regions barely registered for Anglophone scholars. While we don't
have a similarly systematic analysis of the gender and race of those
featured in the extant writings, we can confidently conclude that the
historical literature is even more overwhelmingly focused on white men
of European descent. At the same time, we have a small but growing body
of work on members of minoritized groups. The history of women and
gender is best developed and dates to the 1990s, though the record is
far from complete and in need of broader international and comparative
focus.\textsuperscript{24} There is also a growing literature on the
history of Black scholars and the racialized structures of communication
and media studies, though again a great deal more work needs to be
done.\textsuperscript{25}

Until recently, geopolitical critiques have rarely been explicit guides
for writing histories of the field, but there is a significant body of
work that could aid the effort by reorienting our global imaginary
beyond its traditional US center. Over the last two decades, there have
been trends toward more transnational and even global frameworks for
understanding the historical development of communication and media
studies.\textsuperscript{26} These are of a piece with broader developments in
the history and sociology of the social sciences and
humanities.\textsuperscript{27} Transnational and global frameworks afford the possibilities
to both map lines of post--World War II US hegemony and to provincialize
the US version of the field. It is crucial to recognize that there have
been alternative traditions of education and research on journalism,
film, radio, television, and other forms of what in Latin America came
to be called \emph{comunicación social}. To designate communication an
American field---a tendency of boosters and critics alike---is to
ignore, for instance, German \emph{Zeitungswissenschaft} and postwar
European \emph{Publizistik}, both of which provided alternative models
with transnational reach.\textsuperscript{28} It is also to
ignore Catholic traditions that have deeply shaped, among other things,
Jesuit education in Latin America.\textsuperscript{29}
There is, moreover, a long tradition of critical and Marxian-informed
communication inquiry in Latin America, with intellectual roots in
nineteenth-century independence movements, catalyzed by the 1959 Cuban
Revolution and shaped in the 1960s and 1970s by readings of Gramsci and
dependency theory.\textsuperscript{30} Though less
developed than Latin America's, there are analogous intellectual
traditions in post-independence Africa and the Arab world, also forged
through rejections of modernization paradigms, which advance
alternatives in the name of indigenization, Africanization, and pan-Arab
unity.\textsuperscript{31} UNESCO, the Non-Aligned
Movement, and the International Association\marginnote{\textsuperscript{24} For Latin America, see Yamila Heram and Santiago
  Gándara, \emph{Pioneras en los estudios latinoamericanos de
  comunicación} (Buenos Aires: TeseoPress, 2021); and Clemencia
  Rodríguez et al., eds., \emph{Mujeres de la comunicación} (Bogotá:
  Friedrich Ebert Stiftung, 2020). For an excellent transnational study
  of an influential figure, see Elisabeth Klaus and Josef Seethaler,
  eds., \emph{What Do We Really Know about Herta Herzog?} (Frankfurt am
  Main: Peter Lang, 2016). For recent English-language work that
  includes select bibliographies of the literature, see Leonarda
  García-Jimenez and Esperanza Herrero, ``Narrating the Field Through
  Some Female Voices: Women's Experiences and Stories in Academia,''
  \emph{Communication Theory} 32, no. 2 (2022); and Karen Lee Ashcraft
  and Peter Simonson, ``Gender, Work, and the History of Communication
  Research: Figures, Formations, and Flows,'' in \emph{The International
  History of Communication}, ed. Peter Simonson and David Park (New
  York: Routledge, 2016).} of\marginnote{\textsuperscript{25} For a select bibliography of work in the US context, see
  Armond Towns, ``Against the `Vocation of Autopsy.'\,'' See also
  Dhanveer Singh Brar and Ashwani Sharma, ``What is This `Black' in
  Black Studies? From Black British Cultural Studies to Black Critical
  Thought in UK Arts and Higher Education,'' \emph{New Formations}, no.
  99 (2019); Jeffrey S. Wilkinson, William R. Davie, and Angeline J.
  Taylor, ``Journalism Education in Black and White: A 50-Year Journey
  Toward Diversity,'' \emph{Journalism \& Mass Communication Educator}
  75, no. 4 (2020); Julian Henriques and David Morley, eds.,
  \emph{Stuart Hall: Conversations, Projects and Legacies} (London:
  Goldsmiths Press, 2017); Nova Gordon Bell, ``Towards an Integrated
  Caribbean Paradigm in Communication Thought: Confronting Academic
  Dependence in Media Research,'' in \emph{Re-imagining Communication in
  Africa and the Caribbean}, ed. Hopeton S. Dunn et al. (Cham,
  Switzerland: Palgrave Macmillan, 2021); and Terje Skjerdal and Keyan
  Tomaselli, ``Trajectories of Communication Studies in Sub-Saharan
  Africa,'' in \emph{The International History of Communication Study},
  ed. Peter Simonson and David W. Park (New York: Routledge, 2016).} Media\marginnote{\textsuperscript{26} See, for instance, Peter Simonson and David W. Park,
  eds., \emph{The International History of Communication Study} (New
  York: Routledge, 2016); Erick Torrico Villanueva, \emph{La
  Comunicación: Pensada Desde América Latina (1960--2009)} (Salamanca:
  Comunicación Social, 2016); and Stefanie Averbeck-Lietz, ed.,
  \emph{Kommunikationswissenschaft im internationalen Vergleich:
  Transnationale Perspektiven} (Wiesbaden: Springer Fachmedien
  Wiesbaden, 2017).} and Communication
Research (IAMCR) all played global roles in circulating ideas and
developing social networks among scholars from the Global South and
leftist allies in the North. So too did the former German Democratic
Republic, which like other countries from the former Soviet Bloc,
maintained a strong presence in IAMCR during the Cold War and maintained
their own distinct traditions in politically charged transnational
dialogue, with new entanglements after 1989.\textsuperscript{32} If we are to develop
more complex understandings of contemporary lines of geopolitical
hegemony and exclusion in communication and media studies, we need to
incorporate all that we know about the heterogeneous global histories of
those fields, much of it not published in English.

\hypertarget{the-current-reckoning}{%
\section{\texorpdfstring{The Current Reckoning
}{The Current Reckoning }}\label{the-current-reckoning}}

In the contemporary moment, the geopolitical, racial, and feminist and
gender-based critiques that have circulated for decades have both
entangled themselves together and become much more difficult for those
in the hegemonic centers to ignore. Media and communication studies,
along with many other academic fields, have faced a belated reckoning
with the systemic patterns of exclusion and injustice that have helped
constitute them. This has been driven in part by the ongoing development
of longstanding lines of thought. While in earlier moments certain lines
of geopolitical and racial critique could be inattentive to gender, and
white feminisms inattentive to race, they have become deeply
interanimating and enriched by other strands of critical social theory.
Together they provide new vocabularies and intellectual sensibilities,
but the reckoning of the field is driven by the much larger political
reckonings that have occurred at the societal and global level. To note
a few: the Black Lives Matter movement that began in the US in 2013 and
grew into a global phenomenon by 2016 brought new attention to ongoing
forms of systemic racism around the world and reached a new level with
George Floyd's murder in May 2020. In the spring of 2015, the
\#RhodesMustFall campaign was launched in South Africa, in protest of
the continued legacy of colonialism in higher education. The campaign
generated momentum for a growing, multi-pronged movement to decolonize
and Indigenize knowledge around the world. The Women's Marches of
January 2017, held the day after Donald Trump's inauguration, galvanized
worldwide attention on gender and new assaults on the rights and
wellbeing of women and LGBTQ folks. By the fall of 2017, \#MeToo, which
Black activist Tarana Burke had started in 2006, took hold when white
Hollywood actresses picked it up. It quickly went global, providing a
rallying cry for efforts to publicize and combat the\marginnote{\textsuperscript{27} See Johan Heilbron, Nicolas Guilhot, and Laurent
  Jeanpierre, ``Toward a Transnational History of the Social Sciences,''
  \emph{Journal of the History of Behavioral Sciences} 44, no. 2 (2008);
  Neus Rotger, Diana Roig-Sanz, and Marta Puxan-Oliva, ``Introduction:
  Towards a Cross-Disciplinary History of the Global in the Humanities
  and Social Sciences,'' \emph{Journal of Global History} 14, no. 3
  (2019).} ubiquitous\marginnote{\textsuperscript{28} The literature on the history of
  German communication research and its historical precursors is
  extensive, in both German and English. For a start, with rich
  bibliographies, see Erik Vroons, ``Communication Studies in Europe: A
  Sketch of the Situation about 1955,'' \emph{Gazette} 67, no. 6 (2005);
  Maria Löblich, ``German \emph{Publizistikwissenschaft} and Its Shift
  from a Humanistic to an Empirical Social Science,'' \emph{European
  Journal of Communication} 22, no. 1 (2007); Thomas Wiedemann,
  ``Practical Orientation as a Survival Strategy: The Development of
  \emph{Publizistikwissenschaft} by Walter Hagemann,'' in \emph{The
  International History of Communication Study}, ed. Peter Simonson and
  David W. Park (New York: Routledge, 2016); Averbeck-Lietz, ed.,
  \emph{Kommunikationswissenschaft im internationalen Vergleich;} Thomas
  Wiedemann, Michael Meyen, and Iván Lacasa-Mas, ``100 Years of
  Communication Study in Europe: Karl Bücher's Impact on the
  Discipline's Reflexive Project,'' \emph{Studies in Communication and
  Media} 7, no. 1 (2018). For the uptake of the German model of
  newspaper science in Japan, see Fabian Schäfer, \emph{Public Opinion,
  Propaganda, Ideology: Theories on the Press and Its Social Function in
  Interwar Japan, 1918--1937} (Leiden: Brill, 2012).} but\marginnote{\textsuperscript{29} Ira Wagman, ``Remarkable
  Invention!'' \emph{History of Media Studies} 1 (2021). For Latin
  America, see the rich study of ITESO, Universidad Jesuita de
  Guadalajara (Mexico) in Graciela Bernal Loaiza, ed., \emph{50 años en
  la formación universitaria de comunicadores, 1967--2008: Génesis,
  desarrollo y perspectivas} (Guadalajara: ITESO, 2018); and the
  biographical account of the founder of Jesuit communication study in
  that country, with a reprint of his 1960 letter on the subject, in
  Luis Sánchez Villaseñor, \emph{José Sánchez Villaseñor, S.J.,
  1911/1961: Notas biográficas} (Guadalajara: ITESO, 1997). On
  Venezuela, see José Martínez Terrero, ``Los Jesuitas de Venezuela en
  la Comunicación Social,'' \emph{Temas de comunicación,} no. 1 (1992).}
often silenced phenomena of sexual abuse and harassment, together with
the gendered systems of power that perpetuated it. Meanwhile, faced with
renewed ethno-nationalisms and white supremacies tied to right-wing
populisms, liberals and progressives turned new attention to the
systemic power and privilege of whiteness.

These are some of the currents that have helped energize recent
critiques of communication and media studies in relation to gender,
race, sexuality, language, coloniality, and geopolitical location. They
are too widespread to begin to summarize, but we note some key threads.
In its latest ``Ferment in the Field'' issue, the \emph{Journal of
Communication} published the essay ``\#CommunicationSoWhite,'' a
critical analysis of racialized publishing patterns in the journals of
the ICA and the US National Communication Association (NCA), which has
galvanized conversations internationally.\textsuperscript{33} Others
have scrutinized what Vicki Mayer and her co-authors call ``the stubborn
persistence of patriarchy in communication studies.''\textsuperscript{34} Latin Americans have led the way on
decolonial critiques of the field, extending a tradition that dates back
more than five decades.\textsuperscript{35} They are
part of a recent wave of decolonizing efforts that traverse subfields of
communication and media studies and extend across world regions, which
in turn intersect with recent thinking about the ``de-Westernization''
project.\textsuperscript{36} The interventions take many forms, including
renewed critical attention to the politics and exclusionary practices of
conference locations and their privileging of scholars from wealthier
institutions in the Global North.\textsuperscript{37}
The efforts are manifold and growing.

\hypertarget{exigencies-the-needs-of-the-moment}{%
\section{Exigencies: The Needs of the
Moment}\label{exigencies-the-needs-of-the-moment}}

In this multi-dimensional critical endeavor, which \emph{History of
Media Studies} fully supports, there are many kinds of work that remain
to be done. In the broadest sense, we need to both expand our
cosmopolitan imaginations and dwell in the particulars of fifty years of
critique and activism that have thrown light on the inequities and
epistemological violence of dominant systems of knowledge. This means
recognizing how these inequities are re/produced in particular texts,
encounters, and practices. At the same time, we should explore how those
discrete instances are communicatively linked to---or find analogues
in---other places and times. To do so will require that we recognize the
connections between academic fields and society, with a fine-grained
appreciation of the distinct normative orientations that emerge from
each. The crisis of the present moment brings with it an uncommon
opportunity to make adjustments at the most fundamental levels of
practice. The work of historically Othered people,\marginnote{\textsuperscript{30} There is a rich historiographical literature
  on Latin America. For a start, see Mariano Zarowsky, ``Communication
  Studies in Argentina in the 1960s and '70s: Specialized Knowledge and
  Intellectual Intervention Between the Local and the Global,''
  \emph{History of Media Studies} 1 (2021); Torrico Villanueva, \emph{La
  comunicación}; Raúl Fuentes-Navarro, ``Institutionalization and
  Internationalization of the Field of Communication Studies in Mexico
  and Latin America,'' in \emph{The International History of
  Communication Study}, ed. Peter Simonson and David W. Park (New York:
  Routledge, 2016); Maria Immacolata Vassallo de Lopes and Richard
  Romancini, ``History of Communication Study in Brazil: The
  Institutionalization of an Interdisciplinary Field,'' in \emph{The
  International History of Communication Study}, ed. Peter Simonson and
  David W. Park (New York: Routledge, 2016). Fuentes Navarro has written
  extensively about the history of the field in Mexico and Latin America
  since the 1990s, and his work is a superb guide.} aided\marginnote{\textsuperscript{31} For sub-Saharan Africa, see Willems, ``Provincializing
  Hegemonic Histories''; Skjerdal and Tomaselli, ``Trajectories of
  Communication in Sub-Saharan Africa''; Mohammad Musa, ``Looking
  Backward, Looking Forward: African Media Studies and the Question of
  Power,'' \emph{Journal of African Media Studies} 1, no. 1 (2009); and
  Eddah M. Mutua, Bala A. Musa, and Charles Okigbo, ``(Re)visiting
  African Communication Scholarship: Critical Perspectives on Research
  and Theory,'' \emph{Review of Communication} 22, no. 1 (2022). For the
  Arab world, see Mohammad I. Ayish, ``Communication Studies in the Arab
  World,'' in \emph{The International History of Communication Study},
  ed. Peter Simonson and David W. Park (New York: Routledge, 2016); and
  Carolan Richter and Hanan Badr, ``Die Entwicklung der
  Kommunikationsforschung und -wissenschaft in Ägypten: Transnationale
  Zirkulationen im Kontext von Kolonialismus und Globalisierung,'' in
  \emph{Kommunikationswissenschaft im internationalen Vergleich:
  Transnationale Perspektiven}, ed. Stefanie Averbeck-Lietz (Wiesbaden:
  Springer Fachmedien Wiesbaden, 2017).} by\marginnote{\textsuperscript{32} Michael Meyen,
  ``IAMCR on the East-West Battlefield: A Study on the GDR's Attempts to
  Use the Association for Diplomatic Purposes,'' \emph{International
  Journal of Communication} 8 (2014);} allies, has
brought the problems and possibilities of the moment to light. It is
past time that more of us who are members of historically privileged
groups do what we can to help move the process forward---recognizing the
danger that, as we try, we may simply reconfigure traditional lines of
power and privileged ignorance. There will need to be some kind of
flexible division of labor in this work, linked to positionalities,
institutional locations, expertise, and capacities.

We feel that, among the many needs of the present, there are four in
particular that the 2021 preconference, this special section of essays,
and the \emph{History of Media Studies} journal have all tried (and are
trying) to address. They do not begin to speak to the range of issues
brought out by critiques based on geopolitics, coloniality, race,
gender, sexuality, and disability. But we believe that they can
contribute to the much broader effort in their own ways. They are: (1)
the need to throw the present state of academic fields into sharper
historical relief, (2) the need to build international collaborations
that refigure what have been taken to be the ``centers'' and
``peripheries'' in media studies, (3) the need to find ways to resist
the growing hegemony of English in the global knowledge system, and (4)
the need to support an open and nonprofit publishing infrastructure.
None of these needs is easily satisfied. Resistance to each is built
into the practices and institutions that currently structure our fields.
And none of them has received the attention that it deserves in the
present moment of reckoning.

\hypertarget{historical-roots}{%
\subsection{Historical
Roots}\label{historical-roots}}

First, the need to cast the present in fuller historical relief: While
there are exceptions, recent critiques of the field, like the
disciplines of communication and media research as a whole, are
strikingly contemporary in their focus.\textsuperscript{38} This presentism, it should be
noted, is not the timeless universalism of positivist and
post-positivist social science, produced as it is by critical scholars
broadly committed to dialectical and situated forms of historicity. The
authors of recent interventions would agree that the phenomena they
critique all trail histories that have shaped them. Yet, in general,
their works have not carefully attended to the historical dynamics of
exclusion and marginalization, as grounded in the media and
communication fields' emergence and evolution. One explanation for this
neglect of history is the division of expert labor that characterizes
the modern academy. Another is the pressing urgency of the present. The
result is that otherwise persuasive critiques, in many cases, invoke
two-dimensional accounts of the fields' past that have populated
textbooks\marginnote{Zrinjka Peruško and Dina Vozab,
  ``The Field of Communication in Croatia: Toward a Comparative History
  of Communication Studies in Central and Eastern Europe,'' in \emph{The
  International History of Communication Study}, ed. Peter Simonson and
  David W. Park (New York: Routledge, 2016); Maureen C. Minielli et al.,
  eds., \emph{Media and Public Relations Research in Post-Socialist
  Countries} (Lanham, MD: Lexington Books, 2021).} since\marginnote{\textsuperscript{33} Paula Chakravartty,
  Rachel Kuo, Victoria Grubbs, and Charlton McIlwain,
  ``\#CommunicationSoWhite,''~\emph{Journal of Communication} 68, no. 2
  (2018). The essay was, almost immediately, widely and internationally
  cited, and inspired an ICA preconference which led to a special issue:
  Eve Ng, Khadijah Costley White, and Anamik Saha,
  ``\#CommunicationSoWhite: Race and Power in the Academy and Beyond,''
  \emph{Communication, Culture \& Critique} 13, no. 2 (2020).} Wilbur\marginnote{\textsuperscript{34} Vicki
  Mayer et al., ``How Do We Intervene in the Stubborn Persistence of
  Patriarchy in Communication Scholarship?'' in \emph{Interventions:
  Communication Theory and Practice}, ed. D. Travers Scott and Adrienne
  Shaw (New York: Peter Lang, 2018). See also Clemencia Rodríguez et
  al., \emph{Mujeres de la comunicación}; Sabine Trepte and Laura Loths,
  ``National and Gender Diversity in Communication: A Content Analysis
  of Six Journals between 2006 and 2016,'' \emph{Annals of the
  International Communication Association} 44, no. 4 (2020); and Xinyi
  Wang et al., ``Gendered Citation Practices in the Field of
  Communication,'' \emph{Annals of the International Communication
  Association} 45, no. 2 (2021).} Schramm\marginnote{\textsuperscript{35} Erick R. Torrico Villanueva, ``La
  Comunicología de Liberación, otra fuente para el pensamiento
  decolonial: Una aproximación a las ideas de Luis Ramiro Beltrán,''
  \emph{Quórum Académico} 7 no. 1 (2010); Tanius Karam, ``Tensiones para
  un giro decolonial en el pensamiento comunicológico: Abriendo la
  discusión,'' \emph{Chasqui: Revista Latinoamericana de Comunicación} 133 (2016); Francisco Sierra Caballero and Claudio Maldonado Rivera,
  eds., \emph{Comunicación, decolonialidad y buen vivir} (Quito:
  Ediciones CIESPAL, 2016); Francisco Sierra Caballero, Claudio
  Maldonado, and Carlos del Valle, ``Nueva Comunicología Latinoamericana
  y Giro Decolonial: Continuidades y rupturas,'' \emph{Cuadernos de
  Información y Comunicación}} advanced his ``four founders'' myth in
the 1960s, followed by critical scholars' symbolic slayings in the 1970s
and 1980s.\footnote{Jefferson Pooley, ``The New History of Mass
  Communication Research,'' in \emph{The History of Media and
  Communication Research: Contested Memories}, ed. David W. Park and
  Jefferson Pooley (New York: Peter Lang, 2008).} In other words,
critics risk reinforcing the historiographical cliches that, in their
misleading simplicity, have contributed to the fields' narrow
self-conceptions.

To counter this presentism, the current moment of reckoning requires at
least three kinds of historical specificity. We need, first, to
critically analyze the dynamics through which white Euro-American
hetero-masculinity captured and maintained the hegemonic center of the
field from the interwar period on. In addition, we need to do more to
recover and center the experiences of minoritized members of the field,
resisting totalizing collective memories that ironically occlude their
complex negotiations of the white masculinist hegemony and contributions
to the production of the field historically. Finally, we need to resist
similarly totalizing narratives that call communication and media
research ``an American field'' and fail to recognize the rich histories
of inquiries in Latin America, Africa, East Asia, and Europe---with
their own intellectual traditions and complex negotiations of the
hegemony of US-style research. Revisionist historiography, as well as
longer-standing accounts published in languages besides English, already
furnish material for all three kinds of specificity, but we need more
research on specific geopolitical locations and social groups, as well
as their entanglements with histories that have been better told.

\hypertarget{international-collaborations}{%
\subsection{International
Collaborations}\label{international-collaborations}}

Second, we should continue to build more robust and inclusive
international collaborations. In contrast with the relative paucity of
history within contemporary critiques, there are many examples of
scholars working across national borders to scrutinize the gendered,
racial, colonial, and geopolitical patterns that continue to structure
our fields. We see important efforts within professional associations,
large and smaller conferences, journal special issues, editorial boards,
and individual research projects and publications. At the same time,
there are longstanding habits and institutional structures that impede
our efforts here. Unsurprisingly, some are particularly evident in the
US field, a function of operating in the English-speaking geopolitical
center (or so we think), with all the arrogance and obliviousness that
carries. The relentlessly parochial National Communication Association
(NCA) plays a significant role, with its unmarked name itself implying
sole dominion over nation-based scholarly societies. To its members'
credit, NCA journals, conferences, listservs, and social media
conversations have all done vital work in focusing attention on
whiteness,\marginnote{25 (2020); Alejandro Barranquero and Juan
  Ramos-Martín, ``Luis Ramiro Beltrán and Theorizing Horizontal and
  Decolonial Communication,'' in \emph{The Handbook of Global
  Interventions in Communication Theory}, ed. Yoshitaka Miike and Jing
  Yin (New York: Routledge, 2022); Claudia Magallanes-Blanco, ``Media
  and Communication Studies: What is There to
  Decolonize?''~\emph{Communication Theory} 32, no. 2 (2022).} race,\marginnote{\textsuperscript{36} Regarding decolonizing efforts across subfields of
  communication and media studies, see Antje Glück, ``De-Westernization
  and Decolonization in Media Studies,'' in \emph{Oxford Research
  Encyclopedia of Communication,} ed. Jon Nussbaum (Oxford: Oxford
  University Press, 2018); Joëlle M. Cruz and Chigorzirim Utah Sodeke,
  ``Debunking Eurocentrism in Organizational Communication Theory:
  Marginality and Liquidities in Postcolonial Contexts,''
  \emph{Communication Theory} 31, no. 3 (2021); Mohan Dutta et al.,
  ``Decolonizing Open Science: Southern Interventions,''~\emph{Journal
  of Communication} 71, no. 5 (2021); Bruce Mutsvairo et al.,
  ``Ontologies of Journalism in the Global South,'' \emph{Journalism \&
  Mass Communication Quarterly} 98, no. 4 (2021); C. S. H. N. Murthy,
  ``Unbearable Lightness? Maybe Because of the
  Irrelevance/Incommensurability of Western Theories? An Enigma of
  Indian Media Research,'' \emph{International Communication Gazette}
  78, no. 7 (2016); and Makoni and Masters, ``Decolonization and
  Globalization.'' For more on the ``de-Westernizing project,'' see
  Silvio Waisbord, ``What is Next for De-Westernizing Communication
  Studies?'' \emph{Journal of Multicultural Discourses} (2022): advance
  online publication.} gender,\marginnote{\textsuperscript{37} Eve Ng and Paula Gardner,
  ``Location, Location, Location? The Politics of ICA Conference
  Venues,'' \emph{Communication, Culture \& Critique} 13, no. 2 (2020).} and\marginnote{\textsuperscript{38} Among the notable
  exceptions are Amin Alhassan, ``The Canonic Economy of Communication
  and Culture: The Centrality of the Postcolonial Margins,''
  \emph{Canadian Journal of Communication} 32, no. 1 (2007); Roopali
  Mukherjee, ``Of Experts and Tokens: Mapping a Critical Race
  Archaeology of Communication,'' \emph{Communication, Culture and
  Critique} 13, no. 2 (2020); and Afonso de Albuquerque, ``The
  Institutional Basis of Anglophone Western Centrality,'' \emph{Media,
  Culture \& Society} 43, no. 1 (2021).} to a certain extent, coloniality.
Nevertheless, its members tend to be US-focused in their research and
social networks, monolingual, and often unaware of how their own
cultural particularity shapes their critical analyses. NCA journals are
overwhelmingly populated by US-based scholars, on editorial boards and
among authors.\textsuperscript{39} There are of course
nation-based structuring mechanisms elsewhere in the world as well,
organized through intellectual cultures, networks, and institutions.
They too can impede international collaboration. There are, moreover,
minoritized groups within every world region, excluded from full and
equal participation by systems of gender, race, ethnicity, sexuality,
and dis/ability. The larger point is that we need to pool experience and
expertise from around the world to understand the constitution of global
knowledge systems, and we need to create more equitable space for
minoritized scholars within world regions to write from their own
places, on their own terms.

\hypertarget{against-the-hegemony-of-english}{%
\subsection{Against the Hegemony of
English}\label{against-the-hegemony-of-english}}

Third, if we are to both advance the recent critical reckoning and
develop a more equitable, cosmopolitan field globally, we need to find
ways to push against the hegemony of English and English-language
scholarship. This is of course one factor that limits more robust
international collaborations, but it is more than that. As Afonso de
Albuquerque argues, English has grown more powerful in global
communication and media studies since the 1990s, tied to ``the rise of a
unipolar world order'' and the acceleration of neoliberal
globalization.\textsuperscript{40} As the language solidified its
place as the unquestioned lingua franca of international social science,
it elevated native English speakers, while also making space for
scholars from wealthy Northern European countries and others privileged
enough to know the language well.\textsuperscript{41} The
linguistic hegemony adds another layer of power and privilege to the
journals in the US and UK, with their English-speaking editorial boards,
which dominate global rankings in communication and the social
sciences.\textsuperscript{42} The
accelerating dominance of English has helped render the robust tradition
of Latin American communication scholarship virtually invisible in the
US and Europe.\textsuperscript{43} The global hegemony of English can also be seen as a
linguistic imperialism, tied to what linguists have termed
\emph{linguicism---}ideologically informed practices that perpetuate
unequal divisions of resources and power among groups on the basis of
language.\textsuperscript{44} We need to find ways to
address linguistic imperialism without re-marginalizing those in
English-speaking countries without the cultural capital to have learned
second languages.

To\marginnote{\textsuperscript{39} This was even true in a recent, important,
  two-volume special issue on African communication studies, which did
  not include contributions by scholars working in African universities,
  though there were signal essays by US-based African scholars. See
  Godfried A. Asante and Jenna N. Hanchey, eds., ``(Re)Theorizing
  Communication Studies from African Perspectives, Part I,'' special
  issue, \emph{Review of Communication} 21, no. 4 (2021); and Godfried
  A. Asante and Jenna N. Hanchey, eds., ``(Re)Theorizing Communication
  Studies from African Perspectives, Part II,'' special issue,
  \emph{Review of Communication} 22, no. 1 (2022).} take\marginnote{\textsuperscript{40} Albuquerque, ``The Institutional Basis of
  Anglophone Western Centrality,'' 181.} the\marginnote{\textsuperscript{41} Ana Cristina Suzina,
  ``English as \emph{Lingua Franca.} Or the Sterilisation of Scientific
  Work,'' \emph{Media, Culture \& Society} 43, no. 1 (2021).} dominance\marginnote{\textsuperscript{42} Demeter, ``The Winner Takes It All.''} of\marginnote{\textsuperscript{43} Sarah Ann Ganter and Félix Ortega, ``The
  Invisibility of Latin American Scholarship in European Media and
  Communication Studies: Challenges and Opportunities of
  De-Westernization and Academic Cosmopolitanism,'' \emph{International
  Journal of Communication} 13 (2019); Florencia Enghel and Martín
  Becerra, ``Here and There: (Re)Situating Latin America in
  International Communication Theory,'' \emph{Communication Theory} 28,
  no. 2 (2018).} Anglophone\marginnote{\textsuperscript{44} Robert Phillipson, \emph{Linguistic Imperialism
  Continued} (London: Routledge, 2009); Phillipson and Skutnabb-Kangas,
  ``Communicating in `Global English.'\,''}\setcounter{footnote}{44} scholarship as an unalterable given
is to concede too much. Translation and interpretation are necessary
tools in the work we pursue. Though both can quickly become
prohibitively expensive, new automated translation tools such as DeepL
show great promise, and many software platforms that support real-time
meetings (including Zoom) allow for simultaneous interpretation. There
is of course no easy technical solution to the problem of linguicism,
but one of the most substantial obstacles to overcoming this
problem---the presumption that translation and interpretation are
unnecessary or completely out of reach---is demonstrably false.

\hypertarget{open-access-publishing}{%
\subsection{Open Access
Publishing}\label{open-access-publishing}}

Fourth and finally, we need to promote egalitarian access to
scholarship, for readers \emph{and} for authors. The scholarly
publishing industry, dominated by a handful of giant firms from the
Global North, has helped sharpen the geography of exclusion by
restricting access to those who can pay the steep entrance fees.
Nonprofit open access (OA) publishing, free of article charges, is an
important response to this closed regime of knowledge. In this effort,
the media and communication studies fields should follow the lead of the
well-established Latin American tradition of OA journal
publishing.\footnote{Dominique Babini, ``Toward a Global Open-Access
  Scholarly Communications System: A Developing Region Perspective,'' in
  \emph{Reassembling Scholarly Communications: Histories,
  Infrastructures, and Global Politics of Open Access}, ed. Martin Paul
  Eve and Jonathan Gray (Cambridge: MIT Press, 2020); Michelli Pereira
  da Costa and Fernando César Lima Leite, ``Open Access in the World and
  Latin America: A Review Since the Budapest Open Access Initiative,''
  \emph{Transinformação} 28, no. 1 (2016).} European and North American
institutions have enabled---wittingly or otherwise---the commercial
publishers' cynical cooptation of the OA movement.\footnote{Marcel
  Knöchelmann, ``The Democratisation Myth: Open Access and the
  Solidification of Epistemic Injustices,''~\emph{Science \& Technology
  Studies}~34, no. 2 (2021); Richard Poynder, "Open Access: Could Defeat
  Be Snatched from the Jaws of Victory?" \emph{Open and Shut?} (blog),
  November 18, 2019.} Their tack has been to swap out extortionate
subscription fees for usurious article charges, erecting author paywalls
in place of reader paywalls.\textsuperscript{47} Latin American scholars have led a global campaign
against this corporate, fee-based OA regime, citing the \$3,000+ article
fees as a de facto exclusion of the Global South.\textsuperscript{48} The nonprofit, fee-free Latin American model is a robust and
thriving alternative, supported by collective funding. Media and
communication scholars from Latin America and elsewhere have recently
issued field-specific pleas to reverse the momentum behind commercial,
author-excluding open access, which promises to deepen global knowledge
inequalities.\textsuperscript{49} One counter-move is to
forge international, multi-lingual collaborations among no-fee OA
journals, an idea that \emph{HMS} is now piloting with
\href{http://www.comunicacionysociedad.cucsh.udg.mx/index.php/comsoc}{\emph{Comunicación
y Sociedad}} and
\href{https://www.revistas.usp.br/matrizes/}{\emph{MATRIZes}}\emph{.}

\hypertarget{registers-of-exclusion}{%
\section{Registers of Exclusion}\label{registers-of-exclusion}}

Last year's \href{https://hms.mediastudies.press/pub/schedule/}{ICA
Preconference} was organized with these aims in mind: to chart exclusion
and to continue the work of recovery. The two-day virtual gathering
convened two dozen scholars from around the world for\marginnote{\textsuperscript{47} Audrey C. Smith et al.,
  ``Assessing the Effect of Article Processing Charges on the Geographic
  Diversity of Authors Using Elsevier's `Mirror Journal' System,''
  \emph{Quantitative Science Studies} 2, no. 4 (2021); Alicia
  Kowaltowski, Michel Naslavsky, and Mayana Zatz, ``Open Access Is
  Closed to Middle-Income Countries,'' \emph{Times Higher Education},
  April 14, 2022.} a\marginnote{\textsuperscript{48} Arianna
  Becerril-García, ``The Commercial Model of Academic Publishing
  Underscoring Plan S Weakens the Existing Open Access Ecosystem in
  Latin America,'' \emph{LSE Impact Blog}, May 20, 2020; Kathleen
  Shearer and Arianna Becerril-García, ``Decolonizing Scholarly
  Communications through Bibliodiversity,'' preprint submitted January
  7, 2021.} dialogue\marginnote{\textsuperscript{49} Oliveira et al., ``Towards an Inclusive Agenda'';
  Dutta et al., ``Decolonizing Open Science.''}\setcounter{footnote}{49} on
papers that, in revised form, appear in this special section. The
preconference and journal were, from the beginning, meant to be linked.
One goal was to make manifest our vision for \emph{History of Media
Studies}, as a site of historiographical broadening. The
\href{https://hms.mediastudies.press/pub/precon-cfp/}{call for papers},
issued in Spanish, Portuguese, and English, noted the aim to engage
Latin American scholars and traditions in particular. Roughly half the
papers were written in Spanish, the other half in English. Organizers
used DeepL to generate very serviceable translations for participants to
read ahead of time. The preconference event itself then included live,
simultaneous interpreters supporting what was an electrifying, bilingual
conversation. The vibrancy of the exchange was momentum-generating proof
that the new journal could, with lots of work and collaboration, make
de-centering the fields' histories its editorial pillar. It was lost on
no one that the virtual format was a crucial \emph{enabler} of
intellectual exchange. The high cost of conference travel, after all, is
a major mechanism of exclusion, typically along South-North lines. Those
costs were saved, with the result that grants and a modest, waivable
conference fee could support the professional interpreters
instead---whose work, in turn, depended on the low-cost conference
software that brought the group together.

It was the example of the preconference that prompted \emph{History of
Media Studies} to launch as a multilingual journal, with manuscripts
\href{https://hms.mediastudies.press/author-guidelines}{accepted in both
Spanish and English}. Guided by our international
\href{https://hms.mediastudies.press/editorial}{Editorial Board}, we
plan to support additional languages over time, with priority granted to
linguistic literatures and traditions excluded by the fields'
accelerating English-language hegemony. Of the seven articles in this
special section, three are published in Spanish, four in English, with
this introduction appearing in both languages. The papers address a
variety of exclusions, across geographies and intellectual domains.
Crucial dimensions of exclusion and occlusion, notably along lines of
race and gender, are treated only indirectly in this collection, which
reflects the emphasis of papers that were submitted out of the
preconference. In that respect, we view the special section as a first
step, a promissory note of sorts, toward the inclusive historiography of
the field that the journal seeks to incubate. Each of the papers, in its
particular focus, represents the kind of work we aim to publish in
future volumes---scholarship that brings historical sensitivity to bear
on the field's reckoning with its oversights and inequities.

Sarah Cordonnier's contribution takes the field itself as the object of
exclusion. Media, communication, and film studies were established, she
observes, in diverse ways across the globe. The fields' institutional
and intellectual histories, accordingly, look strikingly different
according to national and regional context. Yet these various media
studies formations share the experience of marginality. Communication
scholars, in one national field after another, have been relegated to
the low-status periphery of their host universities. Cordonnier's
article registers this pattern of stigmatization, traceable in part to
the fields' rapid and late-arriving institutional growth. The paper
issues a call for the fields to reverse their defensive posture---to
embrace the very conditions, including proximity to everyday life and
the sheer heterogeneity of knowledge practices, that have served to sap
their legitimacy.

Daniel Horacio Cabrera Altieri's article on the ``textile imaginary''
can be read as an extension, and also a deepening, of Cordonnier's
concluding point. Cabrera Altieri positions the practice and metaphor of
weaving as a long-submerged and differently gendered alternative to the
transport- and network-oriented conceptions of communication that have
dominated the organized field. Communication theorists' preoccupation
with discursive rationality, and with the progressive unfolding of new
media, has helped to obscure a rich alter-tradition, which Cabrera
Altieri draws out through particular attention to Latin American
Indigenous cultures. His project is to recover the memory, and to
excavate subterranean traces, excluded by the fields' ``textile
amnesia.'' As a rival imaginary, the textile suggests a care-oriented
ethic of interweaving, one which centers on the social fabric.

One long-suppressed source for alternative conceptions of communication,
including the grounding metaphor of textile, is the Indigenous
experience in Latin America. As María Magdalena Doyle describes in her
contribution, Indigeneity was a neglected area of study within the
emerging national disciplines of communication in the region---and, we
would add, elsewhere in the world. Starting in the 1970s, Latin American
scholars began to study Indigenous peoples' communication and media
practices, but typically through the dueling prisms of modernization or
class struggle. Doyle charts a shift in scholarship around the
mid-1980s, a reflection in part of the dawning recognition of Indigenous
peoples' distinctive identity, in national and international arenas. By
the 1990s, a strand of work had begun to articulate a decolonial
imaginary on the basis of Indigenous communication and political
struggle, one that furnished---as a growing body of scholarship draws
out---alternative epistemologies.

Emiliano Sánchez Narvarte's paper follows another thread in Latin
American communication research, the region's engagement with
international organizations and the transnational circulation of
research. Beginning in the late 1970s, Venezuelan scholar Antonio
Pasquali took on a series of posts at UNESCO. From this perch, and by
way of his dense web of ties with other Latin American researchers,
Pasquali helped to connect the region to institutions, like UNESCO and
the International Association of Media \& Communication Research
(IAMCR), that were, at the time, engaged in challenging the unmarked
parochialism of US communication research. Sánchez Narvarte draws out
the politics of communication in Latin America between 1979 and 1989 and
positions Pasquali as an \emph{intellectual mediator} (\emph{mediador
intelectual}),\footnote{The \emph{mediador intelectual} concept is drawn
  from Mariano Zarowsky, \emph{Del laboratorio chileno a la
  comunicación--mundo: Un itinerario intelectual de Armand Mattelart}
  (Buenos Aires: Biblos, 2013).} whose connective role helped, in turn,
to solidify the field's regional consciousness in new spaces like the
Asociación Latinoamericana de Investigadores de la Comunicación (ALAIC).

In the 1970s and 1980s, UNESCO and IAMCR helped braid critical scholars
from around the world in what was, for some researchers at least, a
self-conscious project to build alternatives to the US media effects
tradition. In their article, Maria Löblich, Niklas Venema, and Elisa
Pollack chronicle the rise and fall of critical research in the Cold War
hothouse of West Berlin. In the wake of 1968, leftist students at Freie
Universität helped support new hires and an overhauled curriculum that
mixed critical theory with skills training. Drawing on Pierre Bourdieu's
sociology-of-science framework, Löblich, Venema, and Pollack recount how
the charged politics and heightened rhetoric of anticommunism soon led
the West Berlin government to engineer a restructuring, one that had the
effect of shuttering the university's short-lived critical tradition.

The politics of communication scholarship are Angela Xiao Wu's focus
too, in her account of the Chinese discipline's distinctive embrace of
cybernetics and systems theory in the post-Mao 1980s. Journalism
scholars in particular joined cybernetics with Friedrich Engels'
dialectics-of-nature scientism into what Wu calls ``systems
journalism.'' The measure of news was not correspondence to reality, but
instead its contributions to the overall system's stability. By the
early 1990s, the Chinese field, \emph{xiwen chuanbo} (``journalism
communication study"), was anointed a first-tier discipline, partly
owing to the improbable amalgam of Engels and systems theory---a
creative adaptation, Wu shows, to complex local conditions. It bears
pointing out that these local conditions pick up on just the kind of
geographical and geopolitical contexts that have been excluded or
marginalized in much of the extant historiography.

In the special section's final contribution, Boris Mance and Sašo Slaček
Brlek draw on a quantitative network analysis of eight English-language
journals to chart the communication fields' treatment of inequality. The
topic, they show, has been relegated to the disciplines' margins since
World War II. The spare treatment of inequality, such as it was, has
tended to track broader contexts beyond the field, including the Cold
War contest or, later, the US government's internet policy. Mance and
Brlek conclude that inequality, as a research topic, has been
domesticated, even de-fanged---a byproduct, they argue, of the
mainstream field's close ties to the administrative interests of
powerful states like the US.

The papers collected here gesture at the double character of the fields'
exclusions. These patterns of omission and commission are, at one
register, \emph{constitutive} of the disciplinary formations that we
have inherited and reproduced. Media, film, and communication studies
were, in other words, shaped in fundamental ways by silencings,
entitlements, forgettings, and contestations. The unmarked center and
the excluded periphery have, in an important sense, co-created one
another. At a second register, however, these exclusions represent
\emph{contingent} developments. There is no iron law of academic
dependency, no pre-fated course of hegemonic overspread.

Informed by the preconference conversation and the articles gathered
here, our view is that a historiography informed by both registers---the
constitutive and the contingent---could contribute to the fields'
tentative reckoning with their pasts and presents. To frame these
exclusions as constitutive is to head off any easy solutions in the form
of mere inclusivity; rather, it is to invite us to consider all the ways
in which these and other exclusions have functioned to center certain
problems, theories, methods, languages, nations, social identities, and
publication venues; and to exclude or marginalize others that are cast
as differentially less valuable, lower status, Other, and more. To frame
them as constitutive is also to draw attention to how those exclusions
are performatively enacted on an ongoing basis through the full range of
practices, social and epistemological, through which the field
(re)produces itself.

The promise of the contingency frame, in turn, is to cultivate a
sensitivity to the many alternative formations, literatures, and ways of
knowing that have, from the fields' various beginnings, always shadowed
the better-funded, more visible, and linguistically privileged domains.
In this second register, the historiography of media studies might help
to head off an unintended consequence of some recent critical
interventions. By repeating stock historical tropes, even with the aim
of toppling them, the risk is that the fields' many heterodox and
oppositional traditions will remain invisible, buried by the patterned
forgetting that this section's papers seek to reverse. As we work to
support a more inclusive canopy for the study of media, our fields'
histories could help fertilize the ground beneath. That work has only
begun.






\section{Bibliography}\label{bibliography}

\begin{hangparas}{.25in}{1} 



Alatas, Syed Farid. ``Academic Dependency and the Global Division of
Labour in the Social Sciences.''~\emph{Current Sociology}~51, no. 6
(2003): 599--613. \url{https://doi.org/10.1177/00113921030516003}.

Alatas, Syed Hussein. ``Academic Imperialism.'' Keynote address
delivered before the International Sociology Association Regional
Conference for Southeast Asia (1969). Reprinted in \emph{Reflections on
Alternative Discourses for Southeast Asia,} edited by Syed Farid Alatas,
32--46. Singapore: Centre for Advanced Studies, 2001.

Albuquerque, Afonso de. ``The Institutional Basis of Anglophone Western
Centrality.'' \emph{Media, Culture \& Society} 43, no. 1 (2021):
180--88. \url{https://doi.org/10.1177/0163443720957893}.

Albuquerque, Afonso de, and Thaiane de Oliveira. ``Thinking the
Recolonial in Communication Studies: Reflections from Latin America.''
\emph{Comunicação, Mídia e Consumo} 18, no. 51 (2021).
\url{http://dx.doi.org/10.18568/CMC.V18I51.2521}.

Albuquerque, Afonso de, Thaiane Moreira de Oliveira, Marcelo Alves dos
Santos Junior, and Sofia Oliveira Firmo de Albuquerque. ``Structural
Limits to the De-Westernization of the Communication Field: The
Editorial Board in Clarivate's \emph{JCR} System.'' \emph{Communication,
Culture \& Critique} 13, no. 2 (2020): 185--203.
\url{https://doi.org/10.1093/ccc/tcaa015}.

Alhassan, Amin. ``The Canonic Economy of Communication and Culture: The
Centrality of the Postcolonial Margins.'' \emph{Canadian Journal of
Communication} 32, no. 1 (2007): 103--18.
\url{https://doi.org/10.22230/cjc.2007v32n1a1803}.

Alo, Oladimeji I. ``Contemporary Convergence in Sociological Theories:
The Relevance of the African Thought System in Theory Formation.''
\emph{Présence Africaine}, no. 126 (1983): 34--57.\\\hspace{0.21in}
\href{http://www.jstor.org/stable/3539695}{https://www.jstor.org/stable/24351389}.

Altbach, Philip G. ``Servitude of the Mind? Education, Dependency, and
Neocolonialism.''~\emph{Teachers College Record}~79, no. 2 (1977):
1--11.~\url{https://doi.org/10.1177/016146817707900201}.

Asante, Godfried A., and Jenna N. Hanchey, eds. ``(Re)Theorizing
Communication Studies from African Perspectives, Part I.'' Special
issue, \emph{Review of Communication} 21, no. 4 (2021).
\url{https://www.tandfonline.com/toc/rroc20/21/4}.

Asante, Godfried A., and Jenna N. Hanchey, eds. ``(Re)Theorizing
Communication Studies from African Perspectives, Part II.'' Special
issue, \emph{Review of Communication} 22, no. 1 (2022).
\url{https://www.tandfonline.com/toc/rroc20/22/1}.

Ashcraft, Karen Lee, and Peter Simonson. ``Gender, Work, and the History
of Communication Research: Figures, Formations, and Flows.'' In
\emph{The International History of Communication}, edited by Peter
Simonson and David W. Park, 47--68. New York: Routledge, 2016.

Averbeck-Lietz, Stefanie, ed. \emph{Kommunikationswissenschaft im
internationalen Vergleich: Transnationale Perspektiven}. Wiesbaden:
Springer Fachmedien Wiesbaden, 2017.

Ayish, Mohammad I. ``Communication Studies in the Arab World.'' In
\emph{The International History of Communication Study}, edited by Peter
Simonson and David W. Park, 474--93. New York: Routledge, 2016.

Babini, Dominique. ``Toward a Global Open-Access Scholarly
Communications System: A Developing Region Perspective.'' In
\emph{Reassembling Scholarly Communications: Histories, Infrastructures,
and Global Politics of Open Access}, edited by Martin Paul Eve and
Jonathan Gray, 331--41. Cambridge: MIT Press, 2020.
\url{https://doi.org/10.7551/mitpress/11885.003.0033}.

Barranquero, Alejandro, and Juan Ramos-Martín. ``Luis Ramiro Beltrán and
Theorizing Horizontal and Decolonial Communication.'' In \emph{The
Handbook of Global Interventions in Communication Theory}, edited by
Yoshitaka Miike and Jing Yin, 298--309. New York: Routledge, 2022.

Becerril-García, Arianna. ``The Commercial Model of Academic Publishing
Underscoring Plan S Weakens the Existing Open Access Ecosystem in Latin
America.'' \emph{LSE Impact Blog}, May 20, 2020.
\href{https://blogs.lse.ac.uk/impactofsocialsciences/2020/05/20/the-commercial-model-of-academic-publishing-underscoring-plan-s-weakens-the-existing-open-access-ecosystem-in-latin-america/}{https://blogs.lse.ac.uk/impactofsocialsciences/2020/05/20/the-commercial-model-of-academic-publishing-underscoring-plan-s-weakens-the-existing-open-access-ecosystem-in-latin-america/}

Bell, Nova Gordon. ``Towards an Integrated Caribbean Paradigm in
Communication Thought: Confronting Academic Dependence in Media
Research.'' In \emph{Re-imagining Communication in Africa and the
Caribbean}, edited by Hopeton S. Dunn, Dumisani Moyo, William O.
Lesitaokana, and Shanade Bianca Barnabas, 51--74. Cham, Switzerland:
Palgrave Macmillan, 2021.
\url{https://doi.org/10.1007/978-3-030-54169-9_4}.

Bernal Loaiza, Graciela, ed. \emph{50 años en la formación universitaria
de comunicadores, 1967-2017: Génesis, desarrollo y perspectivas}.
Guadalajara: ITESO, 2018.

Bhambra, Gurminder K. ``Postcolonial and Decolonial Dialogues.''
\emph{Postcolonial Studies} 17, no. 2 (2014): 115--21.
\url{https://doi.org/10.1080/13688790.2014.966414}.

Boils Morales, Guillermo. ``Bibliografía sobre ciencias sociales en
América Latina.''~\emph{Revista Mexicana de Sociología}~40 (1978):
349--78. https://www.jstor.org/stable/3539695.

Brar, Dhanveer Singh, and Ashwani Sharma. ``What is This `Black' in
Black Studies? From Black British Cultural Studies to Black Critical
Thought in UK Arts and Higher Education.'' \emph{New Formations}, no. 99
(2019): 88--109. \url{http://dx.doi.org/10.3898/NEWF:99.05.2019}.

Chakravarty, Paula, Rachel Kuo, Victoria Grubbs, and Charlton McIlwain.
``\#CommunicationSoWhite.'' \emph{Journal of Communication} 68, no. 2
(2018): 254--66.~\url{https://doi.org/10.1093/joc/jqy003}.

Chilisa, Bagele. \emph{Indigenous Research Methodologies.} Thousand
Oaks, CA: SAGE, 2012.

Collins, Patricia Hill. ``Learning from the Outsider Within: The
Sociological Significance of Black Feminist Thought.'' \emph{Social
Problems} 33, no. 6 (1986): S14--S32.
\url{https://doi.org/10.2307/800672}.

Corradi, Juan Eugenio. ``Cultural Dependence and the Sociology of
Knowledge: The Latin American Case.'' In \emph{Ideology and Social
Change in Latin America}, edited by June Nash, Juan Corradi, and Hobart
Spaulding Jr., 7--30. New York: Gordon and Breach, 1977.

Crenshaw, Kimberlé. ``Demarginalizing the Intersection of Race and Sex:
Black Feminist Critique of Antidiscrimination Doctrine, Feminist Theory,
and Antiracist Politics.'' \emph{University of Chicago Legal Forum} 1989
(1989): 139--68.

Cruz, Joëlle M., and Chigorzirim Utah Sodeke. ``Debunking Eurocentrism
in Organizational Communication Theory: Marginality and Liquidities in
Postcolonial Contexts.'' \emph{Communication Theory} 31, no. 3 (2021):
528--48. \url{https://doi.org/10.1093/ct/qtz038}.

Da Costa, Michelli Pereira, and Fernando César Lima Leite. ``Open Access
in the World and Latin America: A Review Since the Budapest Open Access
Initiative.'' \emph{Transinformação} 28, no. 1 (2016): 33--46.
\url{https://doi.org/10.1590/2318-08892016002800003}.

Daniel, Jack L. \emph{Changing the Players and the Game: A Personal
Account of the Speech Communication Association Black Caucus Origins}.
Annandale, VA: Speech Communication Association, 1995.

Demeter, Márton. ``The Winner Takes It All: International Inequality in
Communication and Media Studies Today.'' \emph{Journalism \& Mass
Communication Quarterly} 96, no. 1 (2019): 37--59.
\url{https://doi.org/10.1177/1077699018792270}.

Demeter, Márton. \emph{Academic Knowledge Production and the Global
South: Questioning Inequality and Under-Representation}. Cham,
Switzerland: Palgrave Macmillan, 2020.

Dorsten, Aimee-Marie. ``Women in Communication Research.'' In \emph{The
International Encyclopedia of Communication Theory and Philosophy},
edited by Klaus Bruhn Jensen and Robert T. Craig. Walden, MA: Wiley
Blackwell, 2016.
\url{https://doi.org/10.1002/9781118766804.wbiect106}\emph{.}

Dutta, Mohan, Srividya Ramasubramanian, Mereana Barrett, Christine
Ellers, Divina Sarwatay, Preeti Raghunath, Satveer Kaur, et al.
``Decolonizing Open Science: Southern Interventions.'' \emph{Journal of
Communication} 71, no. 5 (2021): 803--26.
\url{https://doi.org/10.1093/joc/jqab027}.

Ekdale, Brian, Abby Rinaldi, Mir Ashfaquzzaman, Mehrnaz Khanjani,
Frankline Matanji, Ryan Stoldt, and Melissa Tully. ``Geographic
Disparities in Knowledge Production: A Big Data Analysis of
Peer-Reviewed Communication Publications from 1990 to 2019.''
\emph{International Journal of Communication} 16 (2022): 2498--525.
\url{https://ijoc.org/index.php/ijoc/article/view/18386}.

Enghel, Florencia, and Martín Becerra. ``Here and There: (Re)Situating
Latin America in International Communication Theory.''
\emph{Communication Theory} 28, no. 2 (2018): 111--30.
\url{https://doi.org/10.1093/ct/qty005}.

Fuentes-Navarro, Raúl. ``Institutionalization and Internationalization
of the Field of Communication Studies in Mexico and Latin America.'' In
\emph{The International History of Communication Study}, edited by Peter
Simonson and David W. Park, 325--45. New York: Routledge, 2016.

Ganter, Sarah Ann, and Félix Ortega. ``The Invisibility of Latin
American Scholarship in European Media and Communication Studies:
Challenges and Opportunities of De-Westernization and Academic
Cosmopolitanism.'' \emph{International Journal of Communication} 13
(2019): 68--91. \url{https://ijoc.org/index.php/ijoc/article/view/8449}.

García-Jimenez, Leonarda, and Esperanza Herrero. ``Narrating the Field
Through Some Female Voices: Women's Experiences and Stories in
Academia.'' \emph{Communication Theory} 32, no. 2 (2022): 289--97.
\url{https://doi.org/10.1093/ct/qtac002}.

García Jimenez, Leonarda, and Peter Simonson. ``Female Roles,
Contributions, and Invisibilities in the Field of Communication.''
Introduction to special section, \emph{Revista Mediterránea de
Comunicación} 12, no. 2 (2021): 17--113.
\url{https://doi.org/10.14198/MEDCOM.20163}.

Gareau, Frederick H. ``Another Type of Third World Dependency: The
Social Sciences.''~\emph{International Sociology}~3, no. 2 (1988):
171--78. \url{https://doi.org/10.1177/026858088003002005}.

Glück, Antje. ``De-Westernization and Decolonization in Media Studies.''
In \emph{Oxford Research Encyclopedia of Communication}, edited by Jon
Nussbaum. Oxford: Oxford University Press, 2018.
\url{https://doi.org/10.1093/acrefore/9780190228613.013.898}.

Haraway, Donna. ``Situated Knowledges: The Science Question in Feminism
and the Privilege of Partial Perspective.'' \emph{Feminist Studies} 14,
no. 3 (1988): 575--99. \url{https://doi.org/10.2307/3178066}.

Harris, Joseph E. ``Introduction.'' In \emph{Global Dimensions of the
African Diaspora,} edited by Joseph E. Harris, 3--10. 2nd ed.
Washington: Howard University Press, 1993.

Heilbron, Johan, Nicolas Guilhot, and Laurent Jeanpierre. ``Toward a
Transnational History of the Social Sciences.'' \emph{Journal of the
History of the Behavioral Sciences} 44, no. 2 (2008): 146--60.
\url{https://doi.org/10.1002/jhbs.20302}.

Heilbron, Johan, Gustavo Sorá, and Thibaud Boncourt, eds. \emph{The
Social and Human Sciences in Global Power Relations}. Cham, Switzerland:
Palgrave Macmillan, 2018.
\url{https://doi.org/10.1007/978-3-319-73299-2}.

Henriques, Julian, and David Morley, eds. \emph{Stuart Hall:
Conversations, Projects and Legacies.} London: Goldsmiths Press, 2017.

Heram, Yamila, and Santiago Gándara. \emph{Pioneras en los estudios
latinoamericanos de comunicación}. Buenos Aires: TeseoPress, 2021.

Hornsby, Alton. ``Molefi Kete Asante/Arthur Lee Smith Jr. (1942--).''
\emph{BlackPast,} July 20, 2007.
\href{https://www.blackpast.org/african-american-history/asante-molefi-kete-arthur-lee-smith-jr-1942-2/}{https://www.blackpast.org/african-american-history/asante-molefi-kete-arthur-lee-smith-jr-1942-2/}.

Jackson, Ronald L., II, and Sonja M. Brown Givens. \emph{Black Pioneers
in Communication Research}. Thousand Oaks, CA: SAGE, 2016.

Jansen, Sue Curry. ``\,`The Future is Not What it Used to Be': Gender,
History, and Communication Studies.'' \emph{Communication Theory} 3, no.
2 (1993): 136--48.
\url{https://doi.org/10.1111/j.1468-2885.1993.tb00063.x}.

Karam, Tanius. ``Tensiones para un giro decolonial en el pensamiento
comunicológico: Abriendo la discusión.'' \emph{Chasqui: Revista
Latinoamericana de Comunicación} 133 (2016): 247--64.
\url{https://www.redalyc.org/articulo.oa?id=16057383017}.

Klaus, Elisabeth, and Josef Seethaler, eds. \emph{What Do We Really Know
about Herta Herzog?} Frankfurt am Main: Peter Lang, 2016.

Knöchelmann, Marcel. ``The Democratisation Myth: Open Access and the
Solidification of Epistemic Injustices.''~\emph{Science \& Technology
Studies}~34, no. 2 (2021): 65--89.
\url{https://doi.org/10.23987/sts.94964}.

Kowaltowski, Alicia, Michel Naslavsky, and Mayana Zatz. ``Open Access Is
Closed to Middle-Income Countries.'' \emph{Times Higher Education},
April 14, 2022.
\url{https://www.timeshighereducation.com/opinion/open-access-closed-middle-income-countries}.

Lander, Edgardo, ed. \emph{La colonialidad del saber: Eurocentrismo y
ciencias sociales}. Buenos Aires: CLASCO, 2000.

Löblich, Maria. ``German \emph{Publizistikwissenschaft} and Its Shift
from a Humanistic to an Empirical Social Science.'' \emph{European
Journal of Communication} 22, no. 1 (2007): 69--88.
\url{https://doi.org/10.1177/0267323107073748}.

Magallanes-Blanco, Claudia. ``Media and Communication Studies: What Is
There to Decolonize?''~\emph{Communication Theory} 32, no. 2 (2022):
267--72. \url{https://doi.org/10.1093/ct/qtac003}.

Makoni, Sinfree, and Katherine A. Masters. ``Decolonization and
Globalization in Communication Studies.''~In \emph{Oxford Research
Encyclopedia of Communication}, edited by Jon Nussbaum. Oxford: Oxford
University Press, 2021.
\url{https://doi.org/10.1093/acrefore/9780190228613.013.1152}.

Martínez Terrero, José. "Los Jesuitas de Venezuela en la Comunicación
Social." \emph{Temas de comunicación}, no. 1 (1992): 31--46.

Mayer, Vicki, Andrea L. Press, Deb Verhoeven, and Jonathan Sterne. ``How
Do We Intervene in the Stubborn Persistence of Patriarchy in
Communication Scholarship?'' In \emph{Interventions: Communication
Theory and Practice}, edited by D. Travers Scott and Adrienne Shaw,
53--64. New York: Peter Lang, 2018.

Merton, Robert K. ``The Matthew Effect in Science, II: Cumulative
Advantage and the Symbolism of Intellectual Property.'' \emph{Isis} 79,
no. 4 (1988): 606--23. https://www.jstor.org/stable/234750.

Meyen, Michael. ``IAMCR on the East-West Battlefield: A Study on the
GDR's Attempts to Use the Association for Diplomatic Purposes.''
\emph{International Journal of Communication} 8 (2014): 2071--89.
\url{https://ijoc.org/index.php/ijoc/article/view/2443}.

Miike, Yoshitaka, and Jing Yin, eds. \emph{The Handbook of Global
Interventions in Communication Theory}. New York: Routledge, 2022.

Mills, Charles W. \emph{The Racial Contract}. Ithaca, NY: Cornell
University Press, 1997.

Minielli, Maureen C., Marta N. Lukacovic, Sergei A. Samoilenko, Michael
R. Finch, and Deborrah Uecke, eds. \emph{Media and Public Relations
Research in Post-Socialist Countries}. Lanham, MD: Lexington Books,
2021.

Morris, Charles E., III, and Catherine Helen Palczewski. ``Sexing
Communication: Hearing, Feeling, Remembering Sex/Gender and Sexuality in
the NCA.'' In \emph{A Century of Communication Studies: The Unfinished
Conversation}, edited by Pat J. Gehrke and William M. Keith, 128--65.
New York: Routledge, 2015.

Mukherjee, Roopali. ``Of Experts and Tokens: Mapping a Critical Race
Archaeology of Communication.'' \emph{Communication, Culture and
Critique} 13, no. 2 (2020): 152--67.
\url{https://doi.org/10.1093/ccc/tcaa009}.

Murthy, C. S. H. N. ``Unbearable Lightness? Maybe Because of the
Irrelevance/Incommensurability of Western Theories? An Enigma of Indian
Media Research.'' \emph{International Communication Gazette} 78, no. 7
(2016): 636--42. \url{https://doi.org/10.1177/1748048516655713}.

Musa, Mohammad. ``Looking Backward, Looking Forward: African Media
Studies and the Question of Power.'' \emph{Journal of African Media
Studies} 1, no. 1 (2009): 35--54.
\url{https://doi.org/10.1386/jams.1.1.35_1}.

Mustvairo, Bruce, Eddy Borges-Rey, Saba Bebawi, Mireya Márquez-Ramírez,
Claudia Mellado, Hayes Mawindi Mabweazara, Márton Demeter, et al.
``Different, But the Same: How the Global South is Challenging the
Hegemonic Epistemologies and Ontologies of Westernized/Western-Centric
Journalism Studies.'' \emph{Journalism \& Mass Communication Quarterly}
98, no. 4 (2021): 996--1016.
\url{https://doi.org/10.1177/10776990211048883}.

Mutua, Eddah M., Bala A. Musa, and Charles Okigbo. ``(Re)visiting
African Communication Scholarship: Critical Perspectives on Research and
Theory.'' \emph{Review of Communication} 22, no. 1 (2022): 76--92.
https://doi.org/10.1080/15358593.2021.2025413.

Ng, Eve, and Paula Gardner. ``Location, Location, Location? The Politics
of ICA Conference Venues.'' \emph{Communication, Culture \& Critique}
13, no. 2 (2020): 265--69. \url{https://doi.org/10.1093/ccc/tcaa006}.

Ng, Eve, Khadijah Costley White, and Anamik Saha.
``\#CommunicationSoWhite: Race and Power in the Academy and Beyond.''
\emph{Communication, Culture \& Critique} 13, no. 2 (2020): 143--51.
\url{https://doi.org/10.1093/ccc/tcaa011}.

Oliveira, Thaiane Moreira, Francisco Paulo Jamil Marques, Augusto Veloso
Leão, Afonso de Albuquerque, José Luiz Aidar Prado, Rafael Grohmann,
Anne Clinio, Denise Cogo, and Liziane Soares Guazina. ``Toward an
Inclusive Agenda of Open Science for Communication Research: A Latin
American Approach.'' \emph{Journal of Communication} 71 (2021):
785--802. \url{https://doi.org/10.1093/joc/jqab025}.

Peruško, Zrinjka, and Dina Vozab. ``The Field of Communication in
Croatia: Toward a Comparative History of Communication Studies in
Central and Eastern Europe.'' In \emph{The International History of
Communication Study}, edited by Peter Simonson and David W. Park,
213--234. New York: Routledge, 2016.

Phillipson, Robert. \emph{Linguistic Imperialism Continued.} London:
Routledge, 2009.

Phillipson, Robert, and Tove Skutnabb-Kangas. ``Communicating in `Global
English': Promoting Linguistic Human Rights or Complicit with Linguicism
and Linguistic Imperialism.'' In \emph{The Handbook of Global
Interventions in Communication Theory}, edited by Yoshitaka Miike and
Jing Yin, 425--39. New York: Routledge, 2022.

Pooley, Jefferson. ``The New History of Mass Communication Research.''
In \emph{The History of Media and Communication Research: Contested
Memories}, edited by David W. Park and Jefferson Pooley, 43--69. New
York: Peter Lang, 2008.

Pooley, Jefferson, and David W. Park. ``Communication Research.'' In
\emph{The Handbook of Communication History}, edited by Peter Simonson,
Janice Peck, Robert T. Craig, and John P. Jackson Jr., 76--90. New York:
Routledge, 2013.

Poynder, Richard. ``Open Access: Could Defeat Be Snatched from the Jaws
of Victory?'' \emph{Open and Shut}? (blog), November 18, 2019.
\href{https://poynder.blogspot.com/2019/11/open-access-could-defeat-be-snatched.html}{https://poynder.blogspot.com/2019/11/open-access-could-defeat-be-snatched.html}.

Richter, Carola, and Hanan Badr. ``Die Entwicklung der
Kommunikationsforschung und -wissenschaft in Ägypten: Transnationale
Zirkulationen im Kontext von Kolonialismus und Globalisierung.'' In
\emph{Kommunikationswissenschaft im internationalen Vergleich:
Transnationale Perspektiven}, edited by Stefanie Averbeck-Lietz,
383--408. Wiesbaden: Springer Fachmedien Wiesbaden, 2017.

Roby, Pamela. ``Women and the ASA: Degendering Organizational Structures
and Processes, 1964--1974.'' \emph{The American Sociologist} 23 (1992):
18--48. \url{https://doi.org/10.1007/BF02691878}.

Rodríguez, Clemencia, Claudia Magallanes Blanco, Amparo Marroquín
Parducci, and Omar Rincón, eds. \emph{Mujeres de la comunicación.}
Bogotá: Friedrich Ebert Stiftung, 2020.

Rojas, Fabio. \emph{From Black Power to Black Studies: How a Radical
Social Movement Became an Academic Discipline}. Baltimore: Johns Hopkins
University Press, 2007.

Rossiter, Margaret W. ``The \sout{Matthew} Matilda Effect in
Science.''~\emph{Social Studies of Science}~23, no. 2 (1993): 325--41.
\url{https://doi.org/10.1177/030631293023002004}.

Rotger, Neus, Diana Roig-Sanz, and Marta Puxan-Oliva. ``Introduction:
Towards a Cross-Disciplinary History of the Global in the Humanities and
Social Sciences.'' \emph{Journal of Global History} 14, no. 3 (2019):
325­--34.

Sánchez Villaseñor, Luis. \emph{José Sánchez Villaseñor, S.J.,
1911--1961: Notas biográficas}. Guadalajara: ITESO, 1997.

Sapiro, Gisèle, Marco Santoro, and Patrick Baert, eds. \emph{Ideas on
the Move in the Social Sciences and Humanities: The International
Circulation of Paradigms and Theorists}. Cham, Switzerland: Springer,
2020. \url{https://doi.org/10.1007/978-3-030-35024-6}.

Schäfer, Fabian. \emph{Public Opinion, Propaganda, Ideology: Theories on
the Press and Its Social Function in Interwar Japan, 1918--1937}.
Leiden: Brill, 2012.

Schöpf, Caroline M. ``The Coloniality of Global Knowledge Production:
Theorizing the Mechanisms of Academic Dependency.'' \emph{Social
Transformations} 8, no. 2 (2020): 5--46.
\url{https://doi.org/10.1111/johs.12355}.

Shearer, Kathleen, and Arianna Becerril-García. ``Decolonizing Scholarly
Communications through Bibliodiversity.'' Preprint submitted January 7,
2021. \url{https://doi.org/10.5281/zenodo.4423997}.

Sierra Caballero, Francisco, Claudio Maldonado, and Carlos del Valle.
``Nueva Comunicología Latinoamericana y Giro Decolonial: Continuidades y
rupturas.'' \emph{Cuadernos de Información y Comunicación} 25 (2020):
225--42. \url{http://dx.doi.org/10.5209/ciyc.68236}.

Sierra Caballero, Francisco, and Claudio Maldonado Rivera, eds.
\emph{Comunicación, decolonialidad y buen vivir}. Quito: Ediciones
CIESPAL, 2016.

Simonson, Peter, and David W. Park, eds. \emph{The International History
of Communication Study}. New York: Routledge, 2016.

Simonson, Peter, and David W. Park. ``Introduction: On the History of
Communication Study.'' In \emph{The International History of
Communication Study}, edited by Peter Simonson and David W. Park, 1--22.
New York: Routledge, 2016.

Skjerdal, Terje, and Keyan Tomaselli. ``Trajectories of Communication
Studies in Sub-Saharan Africa.'' In \emph{The International History of
Communication Study}, edited by Peter Simonson and David W. Park,
455--73. New York: Routledge, 2016.

Smith, Audrey C., Leandra Merz, Jesse B. Borden, Chris K. Gulick, Akhil
R. Kshirsagar, and Emilio M. Bruna. ``Assessing the Effect of Article
Processing Charges on the Geographic Diversity of Authors Using
Elsevier's `Mirror Journal' System.'' \emph{Quantitative Science
Studies} 2, no. 4 (2021): 1123--43.
\url{https://doi.org/10.1162/qss_a_00157}.

Solovey, Mark, and Christian Dayé, eds. \emph{Cold War Social Science:
Transnational Entanglements}. Cham, Switzerland: Palgrave Macmillan,
2021. \url{https://doi.org/10.1007/978-3-030-70246-5}.

Suzina, Ana Cristina. ``English as \emph{Lingua Franca.} Or the
Sterilisation of Scientific Work.'' \emph{Media, Culture \& Society} 43,
no. 1 (2021): 171--79. \url{https://doi.org/10.1177/0163443720957906}.

Tenzin, Jinba, and Chenpang Lee. ``Are We Still Dependent? Academic
Dependency Theory after 20 Years.'' \emph{Journal of Historical
Sociology} 35 (2022): 2--13. \url{https://doi.org/10.1111/johs.12355}.

Torrico Villanueva, Erick R. ``La Comunicología de Liberación, otra
fuente para el pensamiento decolonial: Una aproximación a las ideas de
Luis Ramiro Beltrán.'' \emph{Quórum Académico} 7, no. 1 (2010): 65--77.
\url{http://revistas.luz.edu.ve/index.php/quac/article/viewFile/5046/4901}.

Torrico Villanueva, Erick R. \emph{La comunicación: Pensada desde
América Latina (1960--2009)}. Salamanca: Comunicación Social, 2016.

Towns, Armond. ``Against the `Vocation of Autopsy': Blackness and/in US
Communication Histories.'' \emph{History of Media Studies} 1 (2021).
\url{https://doi.org/10.32376/d895a0ea.89f81da7}.

Trepte, Sabine, and Laura Loths. ``National and Gender Diversity in
Communication: A Content Analysis of Six Journals between 2006 and
2016.'' \emph{Annals of the International Communication Association} 44,
no. 4 (2020): 289--311.
\url{https://doi.org/10.1080/23808985.2020.1804434}.

Turner, James E. ``Foreword: Africana Studies and Epistemology, a
Discourse in the Sociology of Knowledge.'' In \emph{The Next Decade:
Theoretical and Research Issues in Africana Studies}, edited by James E.
Turner, v--xxv. Ithaca, NY: Cornell University Africana Studies and
Research Center, 1984.

Vanderstraeten, Raf, and Joshua Eykens. ``Communalism and
Internationalism: Publication Norms and Structures in International
Social Science.'' \emph{Serendipities: Journal for the Sociology and
History of the Social Sciences} 3, no. 1 (2018): 14--28.
\url{https://doi.org/10.25364/11.3:2018.1.2}.

Vassallo de Lopes, Maria Immacolata, and Richard Romancini. ``History of
Communication Study in Brazil: The Institutionalization of an
Interdisciplinary Field.'' In \emph{The International History of
Communication Study}, edited by Peter Simonson and David W. Park,
346--66. New York: Routledge, 2016.

Vroons, Erik. ``Communication Studies in Europe: A Sketch of the
Situation about 1955,'' \emph{Gazette} 67, no. 6 (2005): 495--522.
https//doi.org/10.1177/0016549205057541.

Wagman, Ira. ``Remarkable Invention!'' \emph{History of Media Studies} 1
(2021). \url{https://doi.org/10.32376/d895a0ea.ef8f548f}.

Waisbord, Silvio. ``What is Next for De-Westernizing Communication
Studies?'' \emph{Journal of Multicultural Discourses} (2022): Advance
online publication. \url{https://doi.org/10.1080/17447143.2022.2041645}.

Wang, Xinyi, Jordan D. Dworkin, Dale Zhou, Jennifer Stiso, Erika B.
Falk, Dani S. Bassett, Perry Zurn, and David Lydon-Staley. ``Gendered
Citation Practices in the Field of Communication.'' \emph{Annals of the
International Communication Association} 45, no. 2 (2021): 134--53.
\url{https://doi.org/10.1080/23808985.2021.1960180}.

White, Derrick E. \emph{The Challenge of Blackness: The Institute of the
Black World and Political Activism in the 1970s}. Gainesville:
University Press of Florida, 2011.

Wiedemann, Thomas. ``Practical Orientation as a Survival Strategy: The
Development of \emph{Publizistikwissenschaft} by Walter Hagemann.'' In
\emph{The International History of Communication Study}, edited by Peter
Simonson and David W. Park, 109--29. New York: Routledge, 2016.

Wiedemann, Thomas, Michael Meyen, and Iván Lacasa-Mas. ``100 Years of
Communication Study in Europe: Karl Bücher's Impact on the Discipline's
Reflexive Project.'' \emph{Studies in Communication and Media} 7, no. 1
(2018). https://doi.org/10.5771/2192-4007-2018-1-7.

Wilkinson, Jeffrey S., William R. Davie, and Angeline J. Taylor.
``Journalism Education in Black and White: A 50-Year Journey Toward
Diversity.'' \emph{Journalism \& Mass Communication Educator} 75, no. 4
(2020): 362--74. \url{https://doi.org/10.1177/1077695820935324}.

Willems, Wendy. ``Unearthing Bundles of Baffling Silences: The Entangled
and Racialized Global Histories of Media and Media Studies.''
\emph{History of Media Studies} 1 (2021).
\url{https://doi.org/10.32376/d895a0ea.52801916}.

Willems, Wendy. ``Provincializing Hegemonic Histories of Media and
Communication Studies: Toward a Genealogy of Epistemic Resistance in
Africa.'' \emph{Communication Theory} 24, no. 4 (2014): 415--34.
\url{https://doi.org/10.1111/comt.12043}.

Wynter, Sylvia. ``The Ceremony Must be Found: After Humanism.''
\emph{boundary 2} 12, no. 3/13, no. 1 (1984): 19--70.
\url{https://doi.org/10.2307/302808}.

Zarowsky, Mariano.~\emph{Del laboratorio Chileno a la
comunicación--mundo: Un itinerario intelectual} \emph{de Armand
Mattelart}. Buenos Aires: Biblos, 2013.

Zarowsky, Mariano. ``Communication Studies in Argentina in the 1960s and
'70s: Specialized Knowledge and Intellectual Intervention Between the
Local and the Global.'' \emph{History of Media Studies} 1 (2021).
\url{https://doi.org/10.32376/d895a0ea.42a0a7aa}.



\end{hangparas}


\vspace{3em}

\hypertarget{acknowledgments}{%
\section{Acknowledgments}\label{acknowledgments}}

Thanks to Karen Lee Ashcraft for her insightful comments on an earlier
draft of this Introduction, to Raúl Fuentes Navarro for guidance on
Jesuit communication education in Latin America, and to Joëlle Cruz for
suggestions on contextualizing Black Studies in the US in relation to
intellectual developments in Africa.




\end{document}