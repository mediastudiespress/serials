% see the original template for more detail about bibliography, tables, etc: https://www.overleaf.com/latex/templates/handout-design-inspired-by-edward-tufte/dtsbhhkvghzz

\documentclass{tufte-handout}

%\geometry{showframe}% for debugging purposes -- displays the margins

\usepackage{amsmath}

\usepackage{hyperref}

\usepackage{fancyhdr}

\usepackage{hanging}

\hypersetup{colorlinks=true,allcolors=[RGB]{97,15,11}}

\fancyfoot[L]{\emph{History of Media Studies}, vol. 2, 2022}


% Set up the images/graphics package
\usepackage{graphicx}
\setkeys{Gin}{width=\linewidth,totalheight=\textheight,keepaspectratio}
\graphicspath{{graphics/}}

\title[Constituted and Constituting Exclusions]{Constituted and Constituting Exclusions in Communication Studies} % longtitle shouldn't be necessary

% The following package makes prettier tables.  We're all about the bling!
\usepackage{booktabs}

% The units package provides nice, non-stacked fractions and better spacing
% for units.
\usepackage{units}

% The fancyvrb package lets us customize the formatting of verbatim
% environments.  We use a slightly smaller font.
\usepackage{fancyvrb}
\fvset{fontsize=\normalsize}

% Small sections of multiple columns
\usepackage{multicol}

% Provides paragraphs of dummy text
\usepackage{lipsum}

% These commands are used to pretty-print LaTeX commands
\newcommand{\doccmd}[1]{\texttt{\textbackslash#1}}% command name -- adds backslash automatically
\newcommand{\docopt}[1]{\ensuremath{\langle}\textrm{\textit{#1}}\ensuremath{\rangle}}% optional command argument
\newcommand{\docarg}[1]{\textrm{\textit{#1}}}% (required) command argument
\newenvironment{docspec}{\begin{quote}\noindent}{\end{quote}}% command specification environment
\newcommand{\docenv}[1]{\textsf{#1}}% environment name
\newcommand{\docpkg}[1]{\texttt{#1}}% package name
\newcommand{\doccls}[1]{\texttt{#1}}% document class name
\newcommand{\docclsopt}[1]{\texttt{#1}}% document class option name


\begin{document}

\begin{titlepage}

\begin{fullwidth}
\noindent\LARGE\emph{Exclusions in the History of Media Studies
} \hspace{25mm}\includegraphics[height=1cm]{logo3.png}\\
\noindent\hrulefill\\
\vspace*{1em}
\noindent{\Huge{Constituted and Constituting Exclusions in Communication Studies\par}}

\vspace*{1.5em}

\noindent\LARGE{Sarah Cordonnier} \href{https://orcid.org/00000-0003-2442-5356}{\includegraphics[height=0.5cm]{orcid.png}}\par}\marginnote{\emph{Sarah Cordonnier, ``Constituted and Constituting Exclusions in Communication Studies,'' \emph{History of Media Studies} 2 (2022), \href{https://doi.org/10.32376/d895a0ea.d2f41c66}{https://doi.org/ 10.32376/d895a0ea.d2f41c66}.} \vspace*{0.75em}}
\vspace*{0.5em}
\noindent{{\large\emph{Université Lumière Lyon 2}, \href{mailto:sarah.cordonnier@gmail.com}{sarah.cordonnier@gmail.com}\par}} \marginnote{\href{https://creativecommons.org/licenses/by-nc/4.0/}{\includegraphics[height=0.5cm]{by-nc.png}}}

% \vspace*{0.75em} % second author

% \noindent{\LARGE{<<author 2 name>>}\par}
% \vspace*{0.5em}
% \noindent{{\large\emph{<<author 2 affiliation>>}, \href{mailto:<<author 2 email>>}{<<author 2 email>>}\par}}

% \vspace*{0.75em} % third author

% \noindent{\LARGE{<<author 3 name>>}\par}
% \vspace*{0.5em}
% \noindent{{\large\emph{<<author 3 affiliation>>}, \href{mailto:<<author 3 email>>}{<<author 3 email>>}\par}}

\end{fullwidth}

\vspace*{1em}


\newthought{\hypertarget{abstract}{%
\section{Abstract}\label{abstract}}}

\noindent Despite their explosive growth during the past decades, the sciences
dedicated to communication are largely marginalized in the academic
communities worldwide: Typically, their publications are not read, their
theories ignored, their curricula not proposed in prestigious
universities. The stigma these ``fragmented'' disciplines suffer from
has important and interesting consequences for their development and
beyond, for the contemporary understanding of the role of scientific
knowledge in society. How does invisibilization affect the identity of
our discipline(s) and of the researchers who inhabit it? What does it
say about the intervention of social values into supposedly ``neutral''
scientific fields? The analysis of this partly documented situation
relies upon a discursive approach requiring historical resources and, at
the same time, a precise attention to heterogenous experiences that do
not ``fit'' in historically situated normative academic frames. The
approach shows how constituted and constituting exclusions of
communication studies are linked by fundamental questions about where
science ``is'' and what it ``does,'' epistemologically, practically, and
politically. Identifying and acknowledging the power of sociocultural
categories in structuring scientific practices leads us to put the
authority, usefulness, and legitimacy of contemporary science in its
proper intellectual, institutional, and sociocultural place.





\enlargethispage{2\baselineskip}

\vspace*{5em}

\noindent{\emph{History of Media Studies}, vol. 2, 2022}



 \end{titlepage}

\hypertarget{a-deafening-silence-ignoring-the-sciences-dedicated-to-communication}{%
\section{A Deafening Silence: Ignoring the Sciences Dedicated to
\\\noindent Communication}\label{a-deafening-silence-ignoring-the-sciences-dedicated-to-communication}}

``After learning that my PhD is in Communication, a sociologist I know,
one of some repute, replied with surprise, `But I thought
you were so
sophisticated!'\,''\footnote{Thomas Streeter, ``For the Study of
  Communication and against the Discipline of Communication,''
  \emph{Communication Theory} 5, no. 2 (May 1995): 117.} This anecdote
is a textbook case of the kind of exclusion described by Erving Goffman
as stigmatization: ``While the stranger is present before us, evidence
can arise of his possessing an attribute that makes him different from
others in the category of persons available for him to be, and of a less
desirable kind. . . . He is thus reduced in our minds from a whole and
usual person to a tainted, discounted one. Such an attribute is a
stigma.''\footnote{Erving Goffman, \emph{Stigma: Notes on the Management
  of Spoiled Identity} (Englewood Cliffs, NJ: Prentice-Hall, 1963), 12.}
  
\enlargethispage{\baselineskip}

The story of Streeter would probably not surprise the reader, as it is
common sense among the communities of \emph{sciences dedicated to
communication}\footnote{I use this generic locution to include all the
  names given to this field or discipline around the world, as the
  designation and thematic circumscriptions differ from country to
  country: media studies, communication studies, communication science,
  communication and media research, information and communication
  sciences, etc.: ``Few academic units in communication used the same
  name (e.g., journalism vs. {[}mass{]} communication; communication vs.
  speech), so the lack of a `common denominator,' within individual
  countries and internationally, may continue to impede further
  progress.'' Juha Koivisto and Peter D. Thomas, \emph{Mapping
  Communication and Media Research: Conjunctures, Institutions,
  Challenges} (Tampere: Tampere University Press, 2010), 29.} that they
are largely and durably kept away from the ``conversation of
disciplines.''\footnote{Robert T. Craig, ``Communication in the
  Conversation of Disciplines,'' \emph{Russian Journal of Communication}
  1, no. 1 (2008): 8.} Typically, their publications are not read, their
theories and methodologies ignored, their curricula not proposed in
prestigious universities, their PhD students not hired in other
departments, except sometimes for practical and technical teachings,
etc. Indeed, despite their ``unstoppable growth . . . at all levels of
academic activity'' during the past decades,\footnote{Koivisto and
  Thomas, \emph{Mapping}, 21--22.} these disciplines are largely
marginalized in the academic communities worldwide.

Such constant unawareness is remarkable, especially if we consider, by
contrast, the extent of the work carried out for decades now by
researchers in these disciplines, the number of journals and books that
support them, the considerable development of curriculums, etc.
Therefore, the intensity of the exclusion indicates that we are not
facing a coincidence, but a regulated eviction from the ``order of
discourse.''\footnote{Michel Foucault, ``The Discourse on Language,'' in
  \emph{The Archaeology of Knowledge}, trans. A. M. Sheridan Smith (New
  York: Pantheon Books, 1972), 215--37.}

This situation is only partly documented, for various, intertwined
reasons that can be sorted into three general categories. First, because
its perception mainly relies upon individual experiences and casual
interactions such as the one quoted above, narrated by Thomas Streeter.
Second, because it is not pleasant for us to think about it, as it
concerns and affects us directly: In this case, we are both researchers
and objects of the research. And third, because we lack a coherent
analytical and theoretical toolbox to properly analyze this process of
exclusion. Indeed, phenomena like silence or silencing, invisibility or
invisibilization, and finally, exclusion raise tricky methodological
questions, as they are intangible, disorganized, and/or do not fit into
the main, dominant, explicit, obvious analytical frameworks. As such,
they cannot be accounted for by using standardized methods, and any
``demonstration'' about them can easily be rejected as not objective, as
scientifically suspicious.

The difficulty of explaining the issue is precisely part of the problem.
It further hinders our ability to recognize that the exclusion of
sciences dedicated to communication has important and interesting
consequences for the development of the discipline, and beyond, for our
contemporary conception of the role of scientific knowledge in society.
That is why it is relevant to do justice to these omissions by studying
them thoroughly, in a situated way, and from an explained
positionality.\footnote{Gayatri Chakravorty Spivak, ``Can the Subaltern
  Speak?'' in \emph{Marxism and the Interpretation of Culture}, ed. Cary
  Nelson and Lawrence Grossberg (Chicago: University of Illinois Press,
  1988), 271--313.} The present article is dedicated to this program,
with the help of discourse analysis.

Peter Simonson suggests: ``As one small way of blending the history of
media research with ongoing work in the field (be it theoretical,
empirical, interpretative, or critical), I would advocate recycling
well-known and overlooked phrases from the past (``rhetorical
commonplaces of the field'') in our discourses today.''\footnote{Peter
  Simonson, ``Writing Figures into the Field: William McPhee and the
  Parts Played by People in Our Histories of Media Research,'' in
  \emph{The History of Media and Communication Research: Contested
  Memories}, ed. David Park and Jefferson Pooley (New York: Peter Lang,
  2008): 311.} One such commonplace is the ``too frequent
introspection,''\footnote{``Communications and media research has often
  been on the defensive, and given to perhaps too much and too frequent
  introspection, resulting in innumerable reconsiderations, reviews,
  turning points, crossroads, ferments in the field and many other
  pauses to consider where we are going and why.'' Peter Golding, Helena
  Sousa, and Karin Raeymaeckers, introduction to ``Future Priorities in
  European Media and Communication Research,'' ed. Peter Golding, Helena
  Sousa, and Karin Raeymaeckers, special issue, \emph{European Journal
  of Communication} 31, no. 1 (February 2016): 3.} ``a periodic
existential questioning, too often discussed,''\footnote{Yves Jeanneret,
  ``La prétention sémiotique dans la communication" {[}Semiotical
  pretention in communication studies{]},~\emph{Semen} 23~(2007).} an
endless ritornello of the same laments:

\begin{quote}
The field {[}of communication studies{]} has problems relating to its
historical \emph{identity}: its \emph{short tradition} as an academic
discipline, the external influences coming from the media industry and
the state, its legitimacy deficit, its diffuse research topic
(``communication''), the \emph{heterogeneous} academic backgrounds of
its scholars, and the fact of being ``scattered'' all over places at
universities. In Germany, as well as in America, these characteristics
lead within the field to a ``lack of consensus'' on its subject matters
and to difficulties in shaping a self-conception.\footnote{Maria Löblich
  and Andreas M. Scheu, ``Writing the History of Communication Studies:
  A Sociology of Science Approach,'' \emph{Communication Theory} 21, no.
  1 (February 2011): 2 (italics added).}
\end{quote}

\noindent Such self-critical reflections appear, at least latently, in all of the
discipline's monumental texts---as, for instance, in the so well known
``Ferments in the Field'':

\begin{quote}
In the call for contributions, we noted that ``Bernard Berelson's
much-cited lament on `The State of Communication Research' nearly a
quarter century ago marked not the `withering away' but, on the
contrary, the emergence of a vital new discipline.''\footnote{George
  Gerbner and Marsha Siefert, introduction to ``Ferment in the Field,''
  special issue, \emph{Journal of Communication} 33, no. 3 (September
  1983): 4--5 (italics added). As soon as 1959, Bernard Berelson wrote
  that after a creative period, communication research was ``withering
  away.'' Bernard Berelson, ``The State of Communication Research,''
  \emph{The Public Opinion Quarterly} 23, no. 1 (1959): 1--6.}
\end{quote}

\noindent This rhetorical commonplace will not lead here to further regrets or
complaints but to analysis. Following Erving Goffman: ``The stigmatized
individual tends to hold the same beliefs about identity that we {[}the
normal{]} do; this is a pivotal fact.''\footnote{Goffman, \emph{Stigma},
  16.} It is quite enlightening to consider that the stigma coming from
outside the sciences dedicated to communication is also \emph{endorsed}
internally. From this perspective, the lack of recognition of the
discipline creates a paradoxical and very original ``self-constitutive
discourse,''\footnote{Dominique Maingueneau, ``Analysing
  Self-Constituting Discourses,'' \emph{Discourse Studies} 1, no. 2 (May
  1999): 175--99.} because exclusion is both constituted and
constituting. In other words, this felt incompleteness, this
self-devaluation, is a pivotal component of the ``identity'' of the
discipline, what distinguishes it from all other disciplines.

With adequate theoretical, methodological, and reflexive tools, it is
then possible to construct more fruitful questions: How does
invisibilization affect the identity of our ``fragmented'' discipline(s)
and of the researchers who inhabit it? What does it say about the
intervention of social values into supposedly ``neutral'' scientific
fields? What first-hand knowledge and experience of the effects of
exclusion does our peripheral status give us, and how can it be turned
into clues for a better understanding of the common underlying patterns
of exclusion? This is what I will address in the following pages, with
the hope that my proposition can lead to scientific, but also practical
and political, discussions about the role of knowledge in promoting more
inclusive spaces, especially in the academic fields.

\hypertarget{methodology-looking-for-the-international-disciplinary-territory-of-the-sciences-dedicated-to-communication}{%
\section{Methodology: Looking for the International Disciplinary
Territory of the Sciences Dedicated to
Communication}\label{methodology-looking-for-the-international-disciplinary-territory-of-the-sciences-dedicated-to-communication}}

The material I will discuss here has been systematically collected
during ten years of inquiries,\footnote{Sarah Cordonnier, \emph{La
  production d'un territoire scientifique international: Les sciences
  consacrées à la communication à la croisée des circulations
  rhétoriques, institutionnelles et biographiques} {[}The production of
  an international scientific territory: The sciences dedicated to
  communication at the crossroads of rhetorical, institutional, and
  biographical circulations{]} (habilitation thesis, Sorbonne
  Université, 2018).} preceded by personal experiences of this
``disciplinary territory.'' I first witnessed various exclusions in my
own French national context, without giving them special attention. But
then, through a continuous collaboration with German colleagues since
2008, I noticed a curious phenomenon: Although our respective
disciplines (\emph{Sciences de l'information et de la communication} in
France, \emph{Medien-} and \emph{Kommunikationswissenschaft} in Germany)
developed in intellectual, institutional, historical, and cultural
contexts that are very different in many respects,\footnote{Sarah
  Cordonnier and Hedwig Wagner, ``Déployer l'interculturalité: Les
  étudiants, un vecteur pour la réflexion académique sur
  l'interculturel; Le cas des sciences consacrées à la communication et
  aux médias en France et en Allemagne {[}Deploying interculturality:
  Students as a vector for academic reflection on interculturality; The
  case of communication and media sciences in France and Germany{]},
  \emph{Interkulturelle Kompetenz in deutsch-französischen
  Studiengängen} {[}Key competences for higher education and
  employability{]}, ed. Gundula Hiller et al. (Wiesbaden: Springer VS,
  2017): 221--34.} I heard and then read the same laments about the
discipline(s) in France and in Germany,\footnote{They are synthesized in
  the citation of Löblich \& Scheu above. For Germany, see also Nathalie
  Huber, \emph{Kommunikationswissenschaft als Beruf: Zum
  Selbstverständnis von Professoren des Faches im deutschsprachigen
  Raum} {[}Communication science as a profession: On the self-perception
  of professors of the discipline in German-speaking countries{]} (Köln:
  Herbert von Halem} especially about their youth, their lack of
legitimacy. The applied nature of the research was considered a problem
for Kommunikationswissenschaft, while Medienwissenschaft was sometimes
derided as a bad imitation of philosophy.

In order to understand the role and importance of disciplinary identity
in my field, and to see if my experience, localized in two countries,
could be extended to the international level, I began collecting
``fragments from daily life''\textsuperscript{18} through observations,
interactions with colleagues, and reading of various documents. As my
object of study was still under construction and did not fit within the
established theoretical and methodological frames of epistemology or
sociology of sciences, Rosalind Gill's approach was instrumental in
inspiring my own method and allowing me to make sense of this blurry
configuration. Gill writes:

\begin{quote}
I begin {[}the article{]} from experiences in the Academy---experiences
that are often kept secret or silenced, that do not have ``proper
channels'' of communication. My ``data'' are entirely unscientific, but
nevertheless, I contend, they\marginnote{Verlag, 2010), esp. 19--40. For France, see also
  Robert Boure, ``Les sciences de l'information et de la communication
  au risque de l'expertise? Sur et sous des pratiques scientifiques"
  {[}Information and communication sciences at the risk of expertise? On
  and beneath scientific practices{]}, \emph{Réseaux} 15, no. 82--83
  (1997): 233--53.} tell\marginnote{\textsuperscript{18} Rosalind Gill, ``Breaking the
  Silence: The Hidden Injuries of Neo-Liberal Academia,''
  \emph{Feministische Studien} 34 (2016): 43.}\setcounter{footnote}{18} us something real and significant
about our own workplaces. They consist of conversations and e-mails from
friends or colleagues, University memos, letters from journal editors
and other ``fragments'' from daily life.\footnote{Gill, ``Breaking,''
  43.}
\end{quote}

\noindent Because of their number and coherence, I would not say that data like
Gill's are ``entirely unscientific,'' despite their heterogenous formats
and origins. A few years ago, when I read for the first time the
``back-handed compliment'' I mentioned in the beginning of this text, I
strongly related with Streeter's story.\footnote{Streeter, ``For the
  Study,'' 127.} I, as many colleagues with whom I had informal
discussions, had also experienced this kind of contemptuous evaluation
of our scientific skills and purpose, and the ways in which the
disciplinary affiliation negatively ``contaminates'' one's whole
academic identity (reduced forever to the discipline in which the PhD
has been done). Streeter encourages fellow media scholars to go beyond
merely noting this phenomenon, to interrogate it and to ``extend
{[}their reflections{]} beyond the level of shop talk and academic
gossip.''\footnote{Streeter, 117.}

Toward that end, these fragmented observations are a crucial first
step---though it is by no means a simple one. From a practical
standpoint, such data are hard to collect. Such remarks are often made
in passing, in the elusive interactions of the daily life. And more
generally, the (lack of) consideration for the sciences dedicated to
communication remains latent. Furthermore, due to the nature of stigma,
those who are afflicted with it, no matter the number or consistency of
their testimonies, will have a hard time deconstructing it.\footnote{``Certainly,
  members of our field are well aware that we are not always taken
  seriously by scholars in other disciplines (DeFleur 1998; Downing
  2006; Livingstone, 2009; Rogers 1997; Streeter 1995); as McChesney
  (1993, 100) colorfully described it, `communication is seemingly
  regarded by the pooh-bahs in history, political science, and sociology
  as having roughly the same intellectual merit as, say, driver's
  education.'\,'' Mel Stanfill, ``Finding Birds of a Feather: Multiple
  Memberships and Diversity without Divisiveness in Communication
  Research,'' \emph{Communication Theory} 22, no. 1 (February 2012): 17.}
Outside the group, stigma will most frequently be denied. And when
``unveiling'' it inside the group, it is quite hard to be heard in a
dispassionate way, as it can be perceived as a threat to one's identity,
and/or to individual or collective strategies to cope with or to conceal
the stigma. For instance, over the years, my research about the
specificities of sciences dedicated to communication has many times been
seen as complaining too much: ``too much . . . self-pity, prone to lead
to accusations of being `cry-babies' . . . more irony and humor would be
useful,"\footnote{Making jokes is one of the tactics at the disposal for
  the stigmatized, following Goffman.} to cite just one such ``Reviewer
\#2'' assessment. It has also been seen as too implicated: ``It seems to
me that the author identifies herself/himself too closely and
passionately with the `sciences dedicated to communication.'\,'' It is
indeed a common problem for research about exclusion of all kinds to be
seen as non-objective, by contrast with research about and from ``the
normal.''\footnote{Donna Haraway, ``Situated Knowledges: The Science
  Question in Feminism and the Privilege of Partial Perspective,''
  \emph{Feminist Studies} 14, no. 3 (1988): 575--99.} Ironically, such
objections are typically based on normative, ideological assessments
rather than scientific ones. Thus, speaking out is a contradictory
strategy to the one wanting to maintain a low profile, aiming ``to
convince the public to use a softer social label for the category in
question.''\footnote{Goffman, \emph{Stigma}, 36.} (I will come back to
the two-faceted problem of personal implication and complaint in the
last part of this article, as it is also a common critique addressed to
the sciences dedicated to communication in general.)

The collection and analysis of situated cases, even if they are
heterogenous, blurry, or fragmentary, brings attention to the
phenomenon. For me, they are not a result, but the precursor to a
collective discussion. To conduct this exploration in a systematic way,
I first gathered a corpus of twenty documents (books and journal issues)
containing three hundred texts published over a period of thirty years
(1983--2016) and dealing with the sciences dedicated to communication as
a discipline at a global level and/or in international
comparison.\footnote{For a full list of the texts, see Sarah Cordonnier,
  ``Les sciences consacrées à la communication, laboratoire
  disciplinaire? Analyses exploratoires d'un discours `international'"
  {[}The sciences dedicated to communication as disciplinary laboratory?
  Exploratory analysis of an "international" discourse{]},~\emph{Revue
  française des sciences de l'information et de la communication} 10
  (2017).} I went in search of an international disciplinary discourse
that would give me access to the main issues under discussion within the
discipline. My approach for locating ``rhetorical commonplaces'' was
genealogical: I started with the most recent publications, and from
these identified previous, but still ``activated'' (quoted), resources.
Thus, the corpus is composed mainly of \emph{collective} works (from
which at least one contribution is cited in other works) and also of
pivotal individual texts that appear frequently. The number of citations
for a given work attests to its visibility and importance within the
scholarly conversation.\footnote{François Heinderyckx, Margaux Hardy,
  and Marc Vanholsbeeck, ``Les revues scientifiques en
  information-communication: L'ère des mutations?" {[}Scientific
  journals in information and communication sciences: An era of
  change?{]}, \emph{Questions de communication} 21 (2012).} This corpus
is both a (random in a way) coring and a curated sample of texts
allowing me to work on the coherence of an ``international''
disciplinary community through comparison, quantitative analysis, and
observation of temporal continuities. I subjected these texts to several
types of analysis (thematic, editorial, enunciative, quantitative,
contextualization, cross-citation, etc.), but for the purpose of the
present text, I will just take some examples and citations from this
corpus.

Both sets of collected material---daily observations and corpus
texts---granted me access to what Goffman described as ``a publication
of some kind which gives voice to shared feelings, consolidating and
stabilizing for the reader his sense of the realness of `his'
{[}stigmatized{]} group and his attachment to it. Here the ideology of
the members is formulated---their complaints, their aspirations, their
politics.''\footnote{Goffman, \emph{Stigma}, 36.}

In conducting this inquiry, I contributed to other kinds of exclusion.
Indeed, by definition, my identification of ``the discipline'' with its
most common international conceptions has led me to ignore (or to be
unable to reach, due to language, online accessibility, etc.)
marginalized discourses within the discipline.\footnote{I mention this
  problem in the conclusion of another article: Sarah Cordonnier,
  ``Looking Back Together to Become `Contemporaries in Discipline,'\,''
  \emph{History of Media Studies} 1 (2021). Author nationalities in my
  corpus can be found in Cordonnier, ``Laboratoire disciplinaire,'' Fig.
  1,
  \url{https://journals.openedition.org/rfsic/docannexe/image/2750/img-1.jpg};
  and the countries and regions analyzed in Fig. 2,
  \url{https://journals.openedition.org/rfsic/docannexe/image/2750/img-2.jpg}.}
That is why the material I share here comes primarily from France
(especially the daily observations) and deals almost only with Western
countries, whose role is dominant in the current processes of academic
globalization. In order to better understand the complex processes of
homogenization and heterogenization\footnote{Arjun Appadurai,
  \emph{Modernity at Large: Cultural Dimensions of Globalization}
  (Minneapolis: University of Minnesota Press, 1996).} in globalized
academia, and in the sciences dedicated to communication at an
international level, further studies would be required.

\hypertarget{discipline-matters-but-why-and-how}{%
\section{Discipline Matters! But Why, and
How?}\label{discipline-matters-but-why-and-how}}

At this point, some readers probably think that my research could be
interesting in a speculative manner but easily fixable in practice, and
not of their direct concern. It was indeed suggested many times
\emph{not} to consider communication studies as a discipline, because of
all the previously mentioned ``problems.'' The ``field or discipline''
debate is another rhetorical commonplace.\footnote{Kaarle Nordenstreng,
  ``Discipline or Field? Soul-Searching in Communication Research,''
  \emph{Nordicom Review} 27 (2007): 211--22.} In our contemporary world,
it would be better, and more satisfying, to think of communication as a
``post-discipline.''\footnote{Silvio R. Waisbord, \emph{Communication. A
  Post-Discipline} (Cambridge: Polity, 2019).}

This perspective is rich but does not cover the entire scope of the
issue. In this sense, it resembles many other works which are likewise
seductive and ``modern'' but lack a firm conceptual basis about what
discipline is. Indeed, ``disciplines'' have been studied many times as a
theoretical object, and just as frequently, they have been disqualified
from a practical standpoint as a useless, artificial, restricting
framework. Neither epistemology, nor sociology of sciences, nor
laboratory anthropology can fully confront this paradoxical object, from
which academics repeatedly and unsuccessfully tried to escape. In both
cases, scientific and practical, the categories, scales, factors, and
focuses of the study are most often very confused or unstable.

\hypertarget{commonplace-impasses-in-the-study-of-discipline}{%
\subsection{Commonplace Impasses in the Study
of ``Discipline''
}\label{commonplace-impasses-in-the-study-of-discipline}}




In many cases, the works on discipline start from a \emph{normative}
ideal of scientific knowledge rather than from the reality of
practices---or, on the contrary, remain too close to those practices
without putting them into broader historical contexts. That is why, for
instance, they often implicitly activate the ``invisible college''
model,\footnote{Diana Crane, \emph{Invisible Colleges: Diffusion of
  Knowledge in Scientific Communities} (Chicago and London: University
  of Chicago Press, 1975): 172.} where characteristics of knowledge and
characteristics of scientific communities are ``glued'' together in a
(chrono)logical movement: first the idea and a very small group of
bright people; then development (``normal science'' and groups of
rank-and-file collaborators); and finally, specialization, followed by
decline.

This narrative which sticks chronological temporality and logical
development together is more of a fiction than an analysis. Following
Michel Foucault, this is even a mistake:

\begin{quote}
Their chronology {[}of thresholds of scientificity{]}, in fact, is
neither regular nor homogeneous. The discursive formations do not cross
them at regular intervals, or at the same time, thus dividing up the
history of human knowledge (\emph{connaissances}) into different ages. \ldots Moreover: each discursive formation does not pass through these
different thresholds in turn, as through the natural stages of
biological maturation, in which the only variable is the latency period
or the length of the intervals. They are, in fact, events whose
dispersion is not evolutive: their unique order is one of the
characteristics of each discursive formation.\footnote{Foucault,
  \emph{Archaeology}, 206.}
\end{quote}

\noindent Empirically, no discipline follows a (chrono)logical pattern, which was
besides conceived by Crane to explain domains of knowledge (intellectual
contexts) and not disciplines. Indeed, in their cultural structures,
\emph{all} the disciplines lack coherence; all of them bring together
heterogenous traditions, theories, methods, objects, etc.\footnote{About
  his discipline, sociology, Abbott writes that it is ``the most general
  of the social sciences, or, to put it less politely, the least
  defined. . . . There is indeed not one sociology but many. . . .
  Sociology, in short, is irremediably interstitial.'' Andrew Abbott,
  \emph{Chaos of Disciplines} (Chicago: The University of Chicago Press,
  2000), 3, 4. This topic is widely covered in many other disciplines as
  well. Of anthropology, for example, Clifford Geertz writes: ``The idea
  of a discipline . . . fits anthropology none too well. At once broad
  and general, wildly aspiring (`The Study of Man'), and particular and
  miscellaneous, strangely obsessive (puberty rites, gift exchange, kin
  terminology), it has always had, both to itself and to outsiders, a
  blurry image. Neither method nor subject matter very exactly defines
  it.'' Clifford Geertz, \emph{After the Fact: Two Countries, Four
  Decades, One Anthropologist} (Cambridge: Harvard University Press,
  1995): 96--97.} ``Discipline'' is a useful category, but its
scientific use has to be more accurate.

In his foundational study on the ``chaos of disciplines,'' Abbott
recalls that the claim for interdisciplinarity emerges almost
simultaneously with the constitution of disciplines: ``indeed, the long
history and stability of interdisciplinarity---unsuspected by its
current publicists---raise the interesting question for why
interdisciplinarity has \emph{not} transformed the intellectual
system.''\footnote{Abbott, \emph{Chaos}, 134.} He states that academic
organization will remain essentially disciplinary for the foreseeable
future, no matter how many claims for interdisciplinarity,
trans-disciplinarity, or post-disciplinarity are made. The reason for it
is that ``the disciplines constitute the macrostructure of the
\emph{labor market} for faculty. . . . Careers remain within discipline
much more than within university'';\footnote{Abbott, \emph{Chaos}, 126
  (italics added).} ``the credential system {[}of disciplinary PhDs{]}
dictates the disciplinary labor markets.''\footnote{Abbott, 141.}

\hypertarget{a-communicational-perspective-on-discipline-effectiveness}{%
\subsection{A Communicational Perspective on
Discipline
Effectiveness}\label{a-communicational-perspective-on-discipline-effectiveness}}

Extending the consideration of disciplines to their organizational
consequences, it is then no surprise that ``the field {[}of
communication{]} has successfully addressed institutional challenges
despite intellectual nebulousness and astounding diversity.''\footnote{Waisbord,
  \emph{Communication}, 124.} The discipline is not a stable norm to
which local practices can easily be attached. It is not equivalent to
``science'' or ``theory''; it is not focused only on intellectual
dimensions. The usual exegetes of the disciplines are led to ``not see''
what the erroneous (chrono)logical model hinders, in terms of
understanding as well as in terms of normative dominations and
exclusions.

The discipline operates forms of belonging, constraint, and scientific
creativity which depend on different local, national, and international
situations, and on their interpretation by various actors, having
various positions, representations, and knowledge. It is adjusted to the
internal expectations of the activity and to the collectives that take
charge of it. But it also concerns the legitimacy and the visibility of
the people and their productions in and outside the scientific worlds.

It is then especially relevant to consider the ``rhetorical resources
for constructing and legitimizing disciplines,'' as articulated by
Robert T. Craig. Craig divides these into three different contexts:
intellectual, institutional, and sociocultural. The latter, which has
``a primary role,'' consists of ``ordinary concepts and practices more
or less deeply ingrained in the cultural belief systems and habits of
the general society.''\footnote{Craig, ``Communication in the
  Conversation,'' 8--9.} Sociocultural contexts are not supposed to be
rational, objective, nor scientific. But this does not at all mean that
academic environments themselves are preserved from commonsense
conceptions: Sociocultural contexts are just as manifest there.
Disciplinary intellectual and institutional contexts in particular come
under, and carry on, ``ordinary concepts'' and ``cultural belief
systems,'' establishing what Goffman calls ``the normal'' without
allowing for its explication by analytical tools. So, while blurred and
disqualifying representations of the sciences dedicated to communication
have their deeper roots in sociocultural contexts, they are more often
reflected in the more frequently observed institutional and, this latter
especially, intellectual contexts, as we will see now with the support
of examples from my corpus.

\hypertarget{making-sense-of-a-stigmatized-identity}{%
\section{Making Sense of a Stigmatized
Identity}\label{making-sense-of-a-stigmatized-identity}}

\hypertarget{intellectual-contexts-there-are-none-so-deaf-as-those-who-will-not-hear}{%
\subsection{Intellectual Contexts: There are
None so Deaf as Those Who Will\\\noindent Not
Hear}\label{intellectual-contexts-there-are-none-so-deaf-as-those-who-will-not-hear}}

\enlargethispage{\baselineskip}

The ``intellectual context'' is the more obvious in commonsense
conceptions of the discipline. It consists in ``classic and current
texts, theories, problems, methods, and modes of analysis,''\footnote{Craig,
  ``Communication in the Conversation,'' 8.} or, in the words of Abbott,
in ``research practices, evidentiary conventions, rhetorical strategies,
genres, canonical works, and the like.''\footnote{Abbott, \emph{Chaos},
  140.}

In the case of the sciences dedicated to communication, this
intellectual context does not seem to be clear at all: There are
literally countless internal debates about the theoretical and
methodological coherence of our discipline (or field). This situation
partly results from a defective conception of what ``discipline'' is.
Internationally, within the scope of my inquiry, administrative
recognition of the discipline came ``first,'' or, at least, without an
intellectual context that would be considered, internally and
externally, as coherent. This ``unique order'' is conceived as a
paradox---as we can see in the following citation, whose interest does
not reside in its singularity, but in its similarity with many others:

\begin{quote}
We argue that the ``field'' is defined on a social and institutional
level, not at the level of ``basic concepts'' or disciplinarily, and not
even in terms of a supposed common object of study. These perspectives
more often than not are less reflective of any real intellectual
coordinates than they are expressions of particular institutional and
historical conditions, hypostasized into institutional forms, which then
react back upon the organisation of ongoing study and research. We agree
with Peters when he argues that ``\,`Communication' has come to be
administratively, not conceptually defined'' (Peters 1986, p.
528).\footnote{Koivisto and Thomas, \emph{Mapping}, 40--41.}
\end{quote}

\noindent The intellectual discomfort resulting from this specific history, and
the associated difficulty in endorsing it as such, do not at all prevent
the productivity of the discipline. But they contribute to hiding
another important characteristic of this intellectual context: the
almost complete ignorance, on the part of those \emph{outside} the
discipline, of the research produced within it.

Today, this lack of recognition can be felt, for instance, when we
repeatedly read books or articles published in other disciplines, where
what we consider to be basic knowledge about a topic widely covered in
our discipline is lacking. Such silence is hard to analyze
systematically, but bibliometric studies can allow us to approach it in
a quantitative way. James R. Beniger, for instance, argues that
``bibliometric studies over the past fifteen years depict the field of
communication as an intellectual ghetto, one that rarely cites outside
itself and is even more rarely cited by other disciplines.''\footnote{James
  R. Beniger, ``Communication: Embrace the Subject, not the Field,''
  \emph{Journal of Communication} 43, no. 3 (1993): 19, quoted in
  Streeter, ``For the Study,'' 122--3.} But that quantitative approach
remains hard, too, because the sciences dedicated to communication are
also rarely included in comparative or interdisciplinary studies,
whether in history or in sociology of sciences:

\begin{quote}
{[}A number of factors{]} give rise to persistent prestige gaps between
communication studies and its neighbors. Hard data are hard to come
by---ironically because communication research is typically excluded
from reputation studies and was only recently recognized as a doctoral
field by the US National Research Council. In the single study that has
included communication,\footnote{D. J. Downey et al., ``The Status of
  Sociology within the Academy: Where We are, Why We're There, and How
  to Change It,'' \emph{The American Sociologist} 39, no. 2 (2008).} the
US academic deans surveyed judged communication to have the
\emph{lowest} prestige among the twenty-five disciplines
named.\footnote{Jefferson D. Pooley, \emph{James W. Carey and
  Communication Research: Reputation at the University's Margins} (New
  York: Peter Lang, 2016), xvii.}
\end{quote}

\noindent In an even more structural way, many journals relevant for the
discipline are absent from reference citation databases.\footnote{``In
  contrast to other disciplines, such as political science, sociology,
  or psychology, important national journals are missing outside the US.
  ISI should consider including other major national or regional
  journals---for instance, the \emph{Asian Journal of Communication},
  \emph{Communications}, \emph{Keio}, \emph{Nordicom}, or
  \emph{Publizistik}. Such an adjustment of the ISI journal sample would
  improve the average national diversity of all communication
  journals.'' Edmund Lauf, ``National Diversity of Major International
  Journals in the Field of Communication,'' \emph{Journal of
  Communication} 55, no. 1 (March 2005): 148.}

The ability to quantify, in this way, the extent to which the discipline
is or is not present within these databases and discourses offers
interesting clues about exclusion, which can be supplemented by the
numerous, even if scattered, testimonies about adjacent scholarship's
relative neglect of the discipline's contributions with regard to a
given object of study:

\begin{figure}
    \centering
    \includegraphics[width=9cm]{tweet-one.jpeg}
    \caption{Katy Pearce (@katypearce), ``I just skimmed a 2021 article from sociologists on a social media topic,'' Twitter, February 9, 2022, 3:25 p.m., \href{https://twitter.com/katypearce/status/1491508668478607360?t=TVSnTHkyXjG6jGCvsLXbuw&s=03}{https://twitter.com/katypearce/status/} \href{https://twitter.com/katypearce/status/1491508668478607360}{1491508668478607360}.}
    \label{fig:one}
\end{figure}

\begin{figure}
    \centering
    \includegraphics[width=9cm]{tweet-two.jpeg}
    \caption{Jessica L. Beyer (@jlbeyer), ``I see this in political science too,'' Twitter, February 10, 2022, 5:14 a.m., \href{https://twitter.com/jlbeyer/status/1491626631005294597}{https://twitter.com/jlbeyer/status/} \href{https://twitter.com/jlbeyer/status/1491626631005294597}{1491626631005294597}.}
    \label{fig:two}
\end{figure}


\begin{figure}
    \centering
    \includegraphics[width=9cm]{tweet-three.jpeg}
    \caption{Emily Ryalls (@ProfRyalls), ``@ProfJillian and I had this convo a few years ago!'' Twitter, February 10, 2022, 6:49 a.m., \href{https://twitter.com/ProfRyalls/status/1491650489892564997}{https://twitter.com/ProfRyalls/status/} \href{https://twitter.com/ProfRyalls/status/1491650489892564997}{1491650489892564997}.}
    \label{fig:three}
\end{figure}

\begin{quote}
Issues relating to the media are today being studied in many different
disciplines, independent of what has been done, or is being done, by
researchers in media and communication.\footnote{Ulla Carlsson, ``Has
  Media and Communication Research Become Invisible? Some Reflections
  from a Scandinavian Horizon,'' \emph{Gazette} (Leiden, Netherlands)
  67, no. 6 (December 2005): 545.}
\end{quote}

\noindent In the same way, researchers who are considered to be prominent within
the discipline are often unknown outside it:

\begin{quote}
Carey was a giant figure within communication research, and his name is
still getting regularly invoked. Perhaps more surprising is his
invisibility outside the field. Even scholars working in cognate areas
like film studies or the sociology of culture are ignorant of Carey and
his work. Bring him up, and you are likely to get blank stares or
puzzled allusions to comedic acting.\footnote{Pooley, \emph{James W.
  Carey}, xi.}
\end{quote}

\noindent The accumulation of indices contributes to a better understanding of how
``in topographic terms, communication studies sits in a depression,
surrounded---if not by peaks---then by the foothills of the social
sciences and humanities. The metaphor, overwrought as it is, helps to
vivify the effects of prestige on the circulation of ideas.''\footnote{Pooley,
  xii.}

To make sense of this problematic (absence of) recognition, colleagues
sometimes give the argument of the ``poor quality'' of research, of the
``intellectual poverty''\footnote{John Durham Peters, ``Institutional
  Sources of Intellectual Poverty in Communication Research,''
  \emph{Communication Research} 13, no. 4 (October 1986): 527--59.} of
this yet already institutionalized discipline. This self-disqualifying
assessment is of an intellectual nature, which corresponds to what we
usually consider as the ``normal level'' for such reflection. And yet,
this cannot be the explanation for such silence because our research is
neither criticized nor even discussed. To make this argument would imply
that the work is at least consulted and reported on; to the contrary, it
is completely overlooked, even by researchers working on ``our''
research objects. I consider this to be a fundamental characteristic of
the intellectual context of the sciences dedicated to communication,
which corresponds quite well to what Goffman describes:

\begin{quote}
We {[}the normals{]} may try to act as if he {[}the stigmatized{]} were
a ``non-person'' and not present at all as someone of whom ritual notice
is to be taken. He, in turn, is likely to go along with these
strategies, at least initially.\footnote{Goffman, \emph{Stigma}, 29.}
\end{quote}

\noindent How does the stereotypical and stigmatized identity of the sciences
dedicated to communication turn them into ``non-disciplines''? I will
examine that through the following citation, which comes from a text
otherwise dedicated to a subtle analysis of what disciplines operate in
social sciences. It contains one of the few explicit mentions of the
sciences dedicated to communication that I have found outside the
discipline, and it condenses a lot of those latent positions about them.

\begin{quote}
Aborted disciplines and disciplines that are linked to social practices
rather than bodies of knowledge are excellent topics for reflection: The
communication sciences, for example, are defined by the existence of
diverse and evolving forms of technologies that allow the transmission
and reception of messages but do not result in the coupling of an
object, a method and a community. They offer the example of an
institutionalization independent of the emergence of a disciplinary
matrix, even in the weakest sense of the notion developed by Thomas
Kuhn.\footnote{Jean-Louis Fabiani, ``À quoi sert la notion de
  discipline?" {[}What is the purpose of the concept of discipline?{]},
  in \emph{Qu'est-ce qu'une discipline?}, ed. Jean Boutier, Jean-Claude
  Passeron, and Jacques Revel (Paris: Éditions de l'EHESS, 2006), 34 (my
  translation).}
\end{quote}

\noindent For Fabiani (and undoubtedly for others, beginning with the editors of
the text), a banal and imprecise link with ancillary technical objects
is enough to give an account of what the sciences dedicated to
communication are---that is:

\begin{enumerate}
\item
  mere social practices, which would not be mediated by ``bodies of
  knowledge''; and
\item
  a counter-example that reinforces other researchers in the certainty
  that their own discipline is not ``linked to social
  practices''---which is quite surprising: What is the mission of social
  sciences, if not to pay attention to ``social practices'' which are
  necessarily diverse and evolving?
\end{enumerate}

Just a few pages above this extract, Fabiani writes that ``the notion of
discipline {[}is{]} a convenient descriptor of the composite practices
inscribed under the name of science'';\footnote{Fabiani, ``À quoi
  sert,'' 15.} but ``to the naked eye, it is the diversity {[}of
sociology{]} that strikes, or even the cacophony, or at least the low
degree of paradigmatic integration'';\footnote{Fabiani, 23.} etc. And
yet, when it comes to communication sciences, the ``absence of a
disciplinary matrix,''\footnote{Fabiani, 34.} rather than being a
richness and/or a commonplace feature shared by all social sciences,
becomes a problem. The rich analytical frameworks developed by Fabiani
are valid for other disciplines but would be impossible to apply here.
For Fabiani, it would be due to the ``independent'' institutionalization
of the sciences dedicated to communication. But only the differential in
\emph{sociocultural} acknowledgment can analytically explain this
distortion of the analytical tools themselves depending on the
discipline under consideration.

Instead of enduring the exclusion, and taking on the stigma piecemeal,
this analysis should lead us to change our perspective. \emph{Internally
and practically}, why would we have to prove the coherence of our
discipline, and the quality of all the works produced within it, when
others do not? Why would it be important to be heard, and recognized, as
widely as possible in the academic community? And, \emph{externally and
theoretically}, what does this silence say about the sociocultural power
of categories (like ``discipline'') within contemporary academia, and
about the differentiated intellectual legitimacy of scientific
knowledge?

\hypertarget{institutional-contexts-successes-hidden-by-evictions}{%
\subsection{Institutional Contexts: Successes
Hidden by
Evictions}\label{institutional-contexts-successes-hidden-by-evictions}}

The \emph{academic} \emph{settlement}\footnote{Abbott, \emph{Chaos},
  136ff.} of a discipline is fundamental for the perpetuation and
development of a domain of knowledge. Indeed, as one of its most
paradoxical characteristics, the discipline is at the same time a
framework to produce new knowledge, \emph{and} ``an operation of
stabilization of communicational and pedagogical devices allowing the
reproduction of a state of knowledge.''\footnote{Fabiani, ``À quoi
  sert,'' 15.} Institutional contexts allow the pedagogical and
administrative identification of a discipline by various immediate and
distant audiences (from future students and their families, media,
university administration, etc. to, as we have seen before, the academic
labor market). These contexts are tangible, visible, and yet generally
neglected in works about discipline---maybe because they seem too
trivial or even a bit contemptible: the realm of academic policies
rather than a topic for ``pure'' researchers.

Sciences dedicated to communication have widely achieved their
installation in institutional contexts, in such a fast and exponential
way that Koivisto and Thomas call it an ``international enigma'':

\begin{quote}
The growth of communications and media research in the post-war period
may constitute an ``international enigma.'' Countries and traditions
quite distant from and sometimes resistant to the dominant Anglophone
models, such as France, or those with strong traditional academic
structures and traditions that are often resistant to change, such as
Germany, display similar features of unstoppable growth of
communications research and study, at all levels of academic
activity.\footnote{Koivisto and Thomas, \emph{Mapping}, 20--21.}
\end{quote}

 \enlargethispage{-\baselineskip}
 
If we look closer, this undeniable institutional presence does not come
with an unequivocal recognition, as the departments of communication are
often relegated to the campus peripheries---whether it be the physical
outer edges of given campus or the fringes on a national or symbolic
scale. I have observed the sciences' exclusion from prestigious places
in academia in France, the US,\footnote{``Everybody is always asking
  what communication is and why they don't have communication
  departments at Harvard, Princeton, and Yale.'' Michael Meyen,
  ``Fifty-Seven Interviews with ICA Fellows: Byron Reeves,''
  \emph{International Journal of Communication} 6 (2012): 1781. ``As a
  discipline, we continue to struggle not to be seen as only a service
  department handling a lot of undergraduates and not necessarily as
  belonging at the table with all the other major players on campus. We
  have no presence at the Ivy League schools. That's where many people
  get their models of who `belongs' in academia.'' Judee Burgoon, cited
  in Michael Meyen, ``International Communication Association Fellows: A
  Collective Biography,'' \emph{International Journal of Communication}
  6 (2012): 2384.} Great Britain,\textsuperscript{61} and Germany. And it has surely happened
elsewhere, though I am unable to confirm this due to the scope of my
investigation (see ``Methodology''). The situation in the US is well
described in the following citation:

\begin{quote}
\emph{On the campus periphery:} Though some speech-oriented
communication departments are housed within their universities' arts and
sciences faculties, most US communication programs exist as stand-alone
schools or colleges. In practice this means that most programs are
segregated from the other social science and humanities disciplines in
both administrative and physical terms. The arts and sciences faculties,
especially for their constituent scholars, remain the symbolic (and
often geographic) center of the US university, committed (in theory at
least) to the academy's traditional truth-seeking mission. By contrast,
professional units like communication---but also education, business,
and architecture---are often\marginnote{\textsuperscript{61} ``The division of British
  universities according to their age is important for the reason that a
  university's reputation and prestige are often defined by its
  historical status, and the quality of its teaching and research is
  often seen as correlating with its age and traditions. This has
  serious implications in the discipline of communication and media
  studies, since these subjects have not often been favoured in the
  traditional universities in the past. Recent developments, however,
  such as the establishment of media research institutes at such
  prestigious universities as the London School of Economics and Oxford
  University in the past five years, suggest that this tendency may be
  changing, as communication and media and studies moves from the
  institutional `periphery' to the `centre.'\,'' Koivisto and Thomas,
  \emph{Mapping}, 86.}\setcounter{footnote}{61} viewed as questionably academic impostors
that threaten to corrode the university tradition. Stand-alone
communication programs, housed in their own buildings on the edge of
campus, act as a brick-and-mortar drag on the discipline's legitimacy.

\emph{Midwestern state universities:} For some of the same reasons, most
early programs were established in large Midwestern land-grant
universities, like Illinois, Iowa and Michigan State. {[}\ldots{]} With
only a pair of exceptions, the elite private universities on the Eastern
seaboard have shunned the discipline altogether.\footnote{Pooley,
  \emph{James W. Carey}, xv--xvi.}
\end{quote}

The partial exclusion of sciences dedicated to communication from
institutional contexts takes various forms, depending on national
academic histories. But everywhere and overall, it contributes to
concealing the fact that these disciplines have mainly achieved their
academic installation. Instead of eliciting satisfaction or even pride,
this institutional accomplishment seems to reinforce suspicion or even
rejection of the discipline from the outside. And internally, it raises
endless existential questions about its identity: ``Despite this
success---or rather, perhaps precisely due to it---this area of
scholarly activity lacks any clear scientific identity.''\footnote{Koivisto
  and Thomas, \emph{Mapping}, 194.}

With some rare exceptions,\footnote{Wolfgang Donsbach, ``The Identity of
  Communication Research,'' \emph{Journal of Communication} 56, no. 3
  (2006): 437--48. François Heinderyckx, ``The Academic Identity Crisis
  of the European Communication Researcher,'' in \emph{Media
  Technologies and Democracy in an Enlarged Europe}, ed. Nico Carpentier
  et al\emph{.}, 357--62 (Tartu: University of Tartu Press, 2007).
  Leonarda García-Jiménez and Susana Martínez Guillem, ``Does
  Communication Studies Have an Identity? Setting the Bases for
  Contemporary Research,'' \emph{Catalan Journal of Communication \&
  Cultural Studies} 1, no. 1 (August 2009): 15--27.} the continuous
evocation of this ``problem of identity'' lacks a focus on institutional
contexts (except when the institutional achievement is taken as an
argument to diminish the discipline). It mainly consists either in
mentioning the ``immaturity'' or ``backwardness'' of the sciences
dedicated to communication by (often approximate) comparison with more
established disciplines; or in lamenting its ``fragmentation,'' the
absence of founding fathers or proximity with professional or practical
knowledge. In the latter case, this is done without comparison to other
disciplines, which nevertheless present the same characteristics (see
``Discipline Matters'').

Once again, our analysis should lead us to change this cumbersome
narrative, and even turn it into a productive understanding of the
contemporary academia. First, instead of letting the stigma get in the
way, the successful academic settlement of the sciences dedicated to
communication should be strongly emphasized and taken as what it is: an
interesting singularity. Indeed, the creation of a new discipline is
very uncommon after the first settlement of disciplines at the end of
the nineteenth and beginning of the twentieth century:

\begin{quote}
Recalling my earlier argument about institutionalization . . . we can
see that there is one central social structure signifying full
disciplinarity. That is reciprocity in acceptance of PhD faculty. Border
fields often employ faculty of diverse disciplines. We can think of them
as having become true disciplines in the social structural sense once
they hire mainly PhDs in their own field\emph{.} Communication is an
excellent example, reaching disciplinary status, in this sense, only
very recently. (American studies is still trying.)\footnote{Abbott,
  \emph{Chaos}, 139 (italics added).}
\end{quote}

And second, the singularity of the sciences dedicated to communication
leads toward a precise conceptualization of the discipline as a
historically-situated intellectual and institutional frame. The
unquestioned, reproduced (chrono)logical model of disciplines, going
from discovery to institutionalization through intellectual coherence,
is normative and even ideological, while its rhetorical strength is
nonetheless producing effects. In this sense, we can look at discipline
as an \emph{apparatus}. For Foucault, an apparatus consists in a system
of relations between heterogenous elements, resulting in ``a sort of . .
. formation which has as its major function at a given historical moment
that of responding to an \emph{urgent need}. The apparatus thus has a
dominant strategic function.''\footnote{Michel Foucault, ``The
  Confession of the Flesh,'' in \emph{Power/Knowledge: Selected
  Interviews and Other Writings, 1972}--\emph{1977}, ed. Colin Gordon
  (New York: Pantheon, 1980): 195 (italics in the original).} The
disciplinary system\footnote{Abbott, \emph{Chaos}, 122ff\emph{.}}
emerging between the middle of the nineteenth century and the beginning
of the twentieth century responded to such an urgent need of
organization, classification, and autonomy of sciences during the
constitution of the Western modern academia, which is parallel to the
constitution of the modern nation-state.\footnote{See esp. Johan
  Heilbron, ``The Social Sciences as an Emerging Global Field,''
  \emph{Current Sociology} 62, no. 5 (September 2014).} This particular
urgent need no longer exists, and contexts have strongly changed
since---but the disciplinary apparatus remains.

Sciences dedicated to communication thrived after WWII in an already
constituted disciplinary landscape. It was no longer possible to back
scientific practice with the ``purity'' and the authority of science
understood as the production of a knowledge independent of contingencies
(the famous ivory tower)---notably because older disciplines knew all
too well how to make their own practical and vocational origins
disappear.\footnote{See esp. Abbott, \emph{Chaos}, 145.} Intellectual
and institutional territories had already been claimed by other
disciplines, which understandably wished to preserve them from any
encroachment.

For instance, even today in France, communication studies researchers
cannot make their career at the prestigious Center for National
Scientific Research (CNRS), despite the center's creation of a
Communication Studies department in 2008. During a committee meeting
which took place in 2011, it was said that ``according to the members of
the national committee, the level of the candidates is lower in the
Communication department than in the other Social Sciences
departments.''\footnote{``Compte-rendu des élus du Conseil scientifique
  du CNRS des 14 \& 15 novembre 2011,'' retrieved May 20, 2021 (my
  translation). The report of this meeting is still partly available
  here:
  \href{https://sncs.fr/2011/11/21/compte-rendu-des-elus-du-conseil-scientifique-du-cnrs-des-14-15-novembre-2011/}{https://sncs.fr/2011/11/21/compte-rendu-des-elus-du-conseil-scientifique-} \href{https://sncs.fr/2011/11/21/compte-rendu-des-elus-du-conseil-scientifique-du-cnrs-des-14-15-novembre-2011/}{du-cnrs-des-14-15-novembre-2011/}.}
No one on the committee objected to this assumption, even as it would
have been quite difficult for any of them (none of whom represented the
sciences dedicated to communication) to compare ``levels'' in such a
general manner, especially with respect to applicants, not hired
researchers. Despite its lack of coherence, rationality, or logic, this
opinion, of a \emph{sociocultural} nature, had consequences at an
institutional level: It was one of the reasons given to the suppression
of the department after only four years of existence (2008--2012). The
intellectual definition of the discipline (existing in French
universities since the middle of the 1970s\footnote{Stefanie
  Averbeck-Lietz, Fabien Bonnet, Sarah Cordonnier, and Carsten Wilhelm,
  ``Communication Studies in France: Looking for a~`Terre du milieu'?''
  \emph{Publizistik} (2019): 1--18.}) has been carefully kept silent
during the discussion, but the ``cultural beliefs'' constituting the
sociocultural context were pregnant---and all the stronger since the
negative opinions given without further precaution were confused by the
stakeholders with an informed reflexive evaluation, without any
(cognitive) price to pay in this well-tuned assembly.

The symbolic, intellectual, and institutional resources available for
the sciences dedicated to communication to build a ``disciplinary
identity'' and a disciplinary culture are, by definition, not the same
as those of older disciplines. They mostly developed at the peripheries,
from vocational demand in new fields, and in strong relation with public
and private commissioning of surveys. The disciplinary apparatus still
applies to them and constrains them, but it is also amended by their
emergence in a more open, transnational, technological society, and in a
more massified, standardized, and professionalized academic system.
Their late foundation allows a greater lability in the theoretical and
methodological constructions, a more reflective and distanced attention
towards the norms that are imposed on the scientific production and the
categories that circumscribe it, a better acceptance of the
heterogeneity of knowledge formats, and a more horizontal way of
interacting with the so-called ``society of knowledge.'' The
constitutive intellectual-institutional exclusions with which the
sciences dedicated to communication are confronted tend to mask these
rather positive aspects.

\hypertarget{from-exclusion-to-questioning-the-normal}{%
\section{From Exclusion to Questioning the
``Normal''}\label{from-exclusion-to-questioning-the-normal}}

Sciences dedicated to communication present huge national differences in
both intellectual and institutional contexts.\footnote{Peter Simonson
  and David W. Park, eds., \emph{The International History of
  Communication Study} (New York and London: Routledge, 2016). Stefanie
  Averbeck\textsc{-}Lietz, ed., \emph{Kommunikationswissenschaft im
  internationalen Vergleich: Transnationale Perspektiven} (Wiesbaden:
  Springer VS, 2017).} And yet, from a \emph{sociocultural} perspective,
they seem to suffer from a fairly homogenous (lack of) consideration
throughout the world, or, at least, in the Western countries (where I
found my examples, and which are the main source of the ``international
disciplinary discourse'' identified in my corpus). In this regard, the
observation of exclusions provides a better understanding of academic
globalization in general. Academic globalization is not primarily a
unifying process in the production and circulation of knowledge, but
rather a complex movement combining ``cultural homogenization and
cultural heterogenization'' with ``fundamental disjunctures''\footnote{Appadurai,
  \emph{Modernity}, 32ff.} between intellectual, institutional, and
sociocultural contexts---or, to say it differently, between ``four
dimensions: the subject of study, the body of evidence, analytical
frameworks, and academic cultures.''\footnote{Silvio R. Waisbord and
  Claudia Mellado, ``De‐westernizing Communication Studies: A
  Reassessment,'' \emph{Communication Theory} 24 (2014): 363.}

The development of disciplines dedicated to communication both resulted
from and contributes to transformations of the academic field. Goffman
notes that the stigmatized individual ``can come to re-assess the
limitations of normals.''\footnote{Goffman, \emph{Stigma}, 21.} If not
addressed directly, the stigmatized identity of the sciences dedicated
to communication is a weakness, for the discipline and for its members.
But if the causes and consequences of exclusion are better understood,
disadvantages can also be turned into a strength: analytical tools to
shed light on these rampant, sometimes sclerosing, homogenization
processes.

Indeed, my material shows that instead of spreading scientific knowledge
in all its cultural and local diversity, the globalization of academia
mainly happens through:

\begin{enumerate}
\item
  circulating frames (or apparatus) that are detached from their
  original conditions of production, whether that be disciplines or, as
  we will now examine, the professional practices of research in
  academia;
\item
  broadcasting sociocultural beliefs like competition and international
  hierarchies at all levels and, as we will also examine, the place of
  scientific knowledge in contemporary globalized societies.
\end{enumerate}

\hypertarget{institutional-contexts-promoting-more-inclusive-professional-practices-in-the-academic-fields}{%
\subsection{Institutional Contexts: Promoting
More Inclusive Professional\\\noindent Practices in the Academic
Fields}\label{institutional-contexts-promoting-more-inclusive-professional-practices-in-the-academic-fields}}

The imaginary in scholarly life is characterized by ``a liberal dream of
personal, autonomous epic, which has aroused generations of enthusiasm
for scientific research and shaped countless vocations.''\footnote{Judith
  Schlanger, \emph{La vocation} {[}The calling{]}, (Paris: Seuil, 1997):
  226.} Abbott gives a useful description of what he calls
``professional purity'' in the university:

\begin{quote}
In general, professionals who are doing what the public imagines to be
the most basic professional functions are of relatively low status in
the eyes of professionals themselves. It is the ``professionals'
professionals'' who are of high status. The same process happens in
academic life, \emph{perhaps so obviously that we never think to comment
on it}. Professors give highest prestige to people who in fact do as
little teaching as possible. Such people emphasize research, a purely
professional activity. . . . In short, \emph{academics like other
professionals are subject to a ``regression'' into professional purity}.
The intellectual consequences of academic regression of this kind are
considerable. First, regression explains why to academics themselves the
chief intellectual structures of disciplines are not applied
disciplinary practices (like teaching writing), but rather the research
practices and rhetorical strategies.\footnote{Abbott, \emph{Chaos}, 146
  (italics added).}
\end{quote}

This type of narrative is paradoxical: ``Being hard-working,
self-motivating and enterprising subjects is what constitutes academics
as so perfectly emblematic of this neoliberal moment'' of
academia;\footnote{Gill, ``Breaking,'' 51.} but it also corresponds less
and less to actual professional practices in a university context
characterized by an increasing scarcity of material, organizational, and
symbolic resources (especially positions, time, etc.). The pursuit of
prestige or ``dominant'' positions---described at length by Pierre
Bourdieu and since become common sense---may still drive some (aspiring)
researchers. But their daily tasks go far beyond this single point. For
a vast majority of academics, especially without a tenure position, the
``pure production of knowledge'' is only one activity among many others:
teaching, of course, but also reviewing, auditing, assessing,
communicating, organizing events, attending meetings, filling out forms,
answering emails, etc. These activities require interactions with a wide
variety of collaborators: students, academics, administrative staff,
professionals from other sectors, etc. And the meaning of these
activities profoundly evolves, as ``academics work in universities that
no longer envision their primary objective as the production and
dissemination of knowledge for its \emph{intrinsic} use value. Instead,
we work in institutions that conceptualize knowledge production as
necessarily part of the production of exchange values.''\footnote{Lawrence
  D. Berg, Edward H. Huijbens, and Henrik Gutzon Larsen, ``Producing
  Anxiety in the Neoliberal University,'' \emph{The Canadian Geographer}
  60, no. 2 (2016): 178.}

The disjunction between the idealized and actual profession, and the
resulting contradictory injunctions\footnote{Dominique Vinck,
  ``L'activité de recherche en situation d'injonctions contradictoires"
  {[}Research activity in a situation of contradictory injunctions{]},
  in \emph{Les études de sciences: Pour une réflexivité
  institutionnelle}, ed. Joëlle Le Marec (Paris: Éditions des archives
  contemporaines, 2010), 65--80.} with which researchers are confronted,
are growing into ``inter-academic power struggles exacerbated by the
constant squeeze from neoliberal institutions, such as universities that
behave increasingly like corporations.''\footnote{Christian Fuchs and
  Jack Linchuan Qiu, ``Ferments in the Field: Introductory Reflections
  on the Past, Present and Future of Communication Studies,''
  \emph{Journal of Communication} 68, no. 2 (April 2018): 222.} For the
academics, it creates isolation, endless competition at all levels, and
poor working conditions, to the great detriment of all.\footnote{See,
  for instance, Gill, ``Breaking.''} It also hinders the production and
social dissemination of scientific knowledge: Can the exhausted
precarious worker invest the time necessary for a quality and meaningful
investigation? Can the elite researcher achieving ``professional
purity,'' in Abbott's words, be in tune with the social phenomena he
claims to report on?

In the long run, giving ``highest prestige to people who in fact do as
little teaching as possible'' is deleterious. Even if legitimacy seems
to reside solely in the intellectual contexts, art for art's sake,
science for science's sake, cannot be a viable stand. On the contrary,
this artificial separation, within the international context of
``excellence'' academic policies, contributes to more and more rigidity
in the disciplinary norms---that is, in the ways we define, produce, and
circulate ``scientific knowledge.'' When disciplinary apparatus was put
in place, the missions and strategical choices of academics went beyond
that and intervened in a different historical context.

Rather than being individually subjected to the impossible and
alienating injunction to distinguish the scientific from the other
aspects of the profession, the constituted and constituting exclusions
of the sciences dedicated to communication require us to rethink and
reverse the relations between all these aspects, not as contingent but
as necessary and meaningful in our collective organizations. Attending
to professional practices in the sciences dedicated to communication
could allow us to amend the narrative of ``professional purity,'' as
described by Andrew Abbott.

By choosing the sciences dedicated to communication, a discipline that
is so often despised and/or ignored, and so anchored in practical
teaching and applied research, researchers would likely encounter a
problem of personal positioning in academia: They would have ``played
the game wrong'' from the outset, as the overall situation provides very
little room for a classical valorizing narrative based on individual
talent, prestige, or ``professional purity.''

In our ``recent'' discipline, a lot of now prominent scholars have had
to take on a large number of very trivial tasks besides ``pure
production of knowledge,'' and to do that very strategically to
establish and defend their department, laboratory, journal, curriculum,
and/or professional network.

\begin{quote}
As the material should show, the difference in status between the
world-leading communication researchers and an unknown professor from
far away didn't play any role at all. ICA Fellows are by no means an
``ultra-elite,'' such as Nobel laureates in science (Zuckerman, 1972).
In fact, quite the opposite is true. Courtesy, curiosity, and the appeal
of doing something completely new are part of the fellows' habitus. Most
of the interviewees seemed to be glad to get the possibility to talk
about themselves and their work.\footnote{Meyen, ``Fifty-Seven
  Interviews,'' 1456.}
\end{quote}

The discipline's global sociocultural contexts have probably left their
mark on its professional culture and ethos through informal
transmissions. That is why we are now well placed to defend the
idea---so crucial today but expressed repeatedly during the past
decades---that ``occasional critical reflection on the pressures and
limits within which we think and work is an intellectual, not just
practical, necessity, and should be integrated with the specifics of our
work.''\footnote{Streeter, ``For the Study,'' 118.} We are well placed
to rethink the historically situated tasks, roles, and practical
activities of academics in contemporary academic frames. Students'
demographics and needs, regional economic activities, practical
knowledge, and local administrative matters are equally important, and
thus worthy of academic consideration and reflexivity.\footnote{``For
  all the interest in reflexivity in recent decades, the experiences of
  academics have somehow largely escaped critical attention. It is as if
  the parameters for reflexivity are bounded by the individual study,
  leaving the institutional context in which academic knowledge is
  produced simply as a taken for granted backdrop.'' Gill, ``Breaking,''
  40.}

From our ``excluded'' position, we could then be able to explicitly
reintegrate a variety of heterogenous, living, informal, caring,
relational, subaltern practices in our job description, rather than
letting them to the more dominated members of the field.

\hypertarget{intellectual-contexts-reassessing-relations-between-scientific-knowledge-and-common-sense-in-contemporary-societies}{%
\subsection{Intellectual Contexts: Reassessing
Relations between Scientific\\\noindent Knowledge and Common Sense in Contemporary
Societies}\label{intellectual-contexts-reassessing-relations-between-scientific-knowledge-and-common-sense-in-contemporary-societies}}

Unveiling the discrepancies between actual daily practices and academic
prestige has to do with an issue which is both epistemological and
practical: the complex relations between scholarly knowledge and common
sense, which are the underlying thread of this text.

The researchers' \emph{exteriority} to their object for study is the
basis of the scientific posture, both from a cognitive point of view and
for the social legitimization of sciences. It is crucial because it is
linked to the very status of the produced knowledge. This topic is
treated continuously in the epistemology and methodology of the social
sciences since the emergence of disciplines at the end of the
19\textsuperscript{th} century, through frequent comparisons with the
natural sciences. It constitutes a rhetorical commonplace today. But in
the social sciences, exteriority appears to be difficult to maintain,
due to logical, methodological and social reasons. Indeed, ``the thought
objects constructed by the social scientists refer to and are founded
upon the thought objects constructed by the common-sense thought of man
living his everyday life among his fellowmen.''\footnote{Alfred Schütz,
  ``Common-Sense and Scientific Interpretation of Human Action,''
  \emph{Philosophy and Phenomenological Research} 14, no. 1 (1953): 3.}
Science does not intervene in a knowledge void. And especially now, in
``modern societies,'' scientific knowledge is also part of daily,
``lay,'' life: ``notions coined in the metalanguages of the social
sciences routinely reenter the universe of actions they were initially
formulated to describe or account for. {[}\ldots{]} \emph{Sociological
knowledge spirals in and out of the universe of social life,
reconstructing both itself and that universe as an integral part of that
process.}''\footnote{Anthony Giddens, \emph{The Consequences of
  Modernity} (Stanford: Stanford University Press, 1990), 15--16
  (italics in the original).}

From their beginning, the sciences dedicated to communication have been
in close contact with practices of everyday life and ``commonsense
thought'' (through reception studies, etc.), as well as with different
professional fields that are themselves particularly nourished by
scholarly knowledge (media, culture and digital sectors, organizational
communication, etc.). In this context, it is particularly interesting to
come back to one of the fiercest, most hostile accusations leveled at
the sciences dedicated to communication (along with their having a
``diffuse research topic''\footnote{Löblich and Scheu, ``Writing,'' 2.}
and no defined method): the fact that they are ``linked to social
practices rather than bodies of knowledge.''\footnote{Fabiani, ``À quoi
  sert,'' 34.} In other words, they are too receptive to ``external
influences coming from the media industry and the state,''\footnote{Löblich
  and Scheu, ``Writing,'' 2.} ``too close'' or over-indulgent to the
professional fields, suspiciously involved in the objects of their study
(see Reviewer \#2 above).

Such reactions are visceral rather than argumented. The relationship
between common sense and scientific knowledge is not questioned, but
reified and normalized. We can see where this logic breaks down if we
try to apply them to another (less stigmatized) discipline, or if we
consider what those reactions call for: To be valid, must knowledge be
produced from the ``ivory tower'' and/or have no social relevance?
Should researchers have no interest in the topics they are studying?
Could mere contact with social objects, or an appreciation of the
specific knowledges of people under study, taint the ``purity'' of the
scientific knowledge?

The problem of the (absence of) separation between scientific and
everyday knowledge goes both ways: Scientific knowledge is involved in
society, and the daily practices of academics are also fed by
imaginaries, rhetorical commonplaces, ``cultural belief systems and
habits of the general society.''\footnote{Craig, ``Communication in the
  Conversation,'' 9.}

From their specific history and positionality, the sciences dedicated to
communication are well placed to question in new terms the contemporary
relevance of the separation between science and other knowledge as it
was established more than a century ago. Some very creative propositions
have emerged in this space, such as Robert T. Craig's argument that the
sciences dedicated to communication be considered as a practical
discipline: ``A practical discipline cultivates a practice by engaging
critically and constructively with the normative metadiscourse that
constitutes and regulates the practice in society. Practical inquiry
itself is a metadiscursive practice that emerges from, reconstructs, and
potentially influences ordinary metadiscourse.''\footnote{Robert T.
  Craig, ``For a Practical Discipline,'' \emph{Journal of
  Communication}, 68, no. 2 (April 2018): 291.}

The sciences dedicated to communication contribute in their own way to
showing how the impartiality of scientific knowledge is a social
construct, drawing attention to practices that are neglected in everyday
social science discourse and giving importance to the material,
technical, social, and symbolic mediations at work in the scientific
construction of an empirical object. ``Contrary to what one might think,
this particularity has in fact a double advantage: it forces the
researcher to construct his or her research object, and it offers him or
her a completely unique relationship with the field. On the condition,
obviously, that one takes the measure of this
particularity.''\footnote{Jean Davallon, ``Objet concret, objet
  scientifique, objet de recherche" {[}Concrete object, scientific
  object, research object{]}, \emph{Hermès} 38 (2004): 32 (my
  translation).}

The resistance it has faced from ``the normal'' has led, by now, to the
discipline's stigmatization rather than to acceptance by other
disciplines or even shifts in their research practices. But if we manage
to get out of a defensive and justificatory posture, we can begin to
demand that knowledges of all kinds (scientific, practical,
professional, or otherwise) be respected and included, rather than
excluded, invisibilized or delegitimized. Rather than diminishing the
value of scientific knowledge, such a shift would place the authority,
usefulness, and legitimacy of science in its proper situated,
intellectual, symbolic, and sociocultural place: among others, at their
service---not ``above'' them, with an undue rhetorical authority.

\hypertarget{conclusion}{%
\section{Conclusion}\label{conclusion}}

Individually, we do not have to endorse the disciplinary (intellectual)
identity all the time, let alone champion it in every situation. We are
entitled to stay ``within'' or to go ``outside'' its intellectual
bounds---if a boundary can be identified at all! If stigmatized, we can
use the variety of tactics that Goffman described to protect ourselves.
And fortunately, despite negative representations of the discipline in
sociocultural contexts, many colleagues from other disciplines are
willing to work with us. Misunderstandings, the impression that one is
being instrumentalized, and other misadventures of the sort, are part of
the usual difficulties of interdisciplinarity and must be dealt with on
a project by project basis. None of this requires us to have a clear
idea of what the discipline ``is.'' At the intellectual level,
``discipline'' is a diffuse but powerful category, especially as it is
often neglected and confused in common parlance and in epistemology. The
main constraint of the discipline is institutional and is manifest at
rare but crucial moments, notably in the labor market. That is why it is
almost impossible for applicants---and consequently, concerned
supervisors---to escape it, except by leaving the academy. At this
level, the exclusion of the sciences dedicated to communication by other
disciplines is at its highest; however, knowledge provided by our
disciplines can also intervene, at this same level, in a different
way---occasioning deeper reflection on the processes of exclusion in our
work environments.

My analysis aimed to better situate the sciences dedicated to
communication in their different contexts, to better identify the
exclusions we suffer from, and ideally, to avoid reproducing these
exclusions where we would have the latitude to do so. My discursive
approach also required epistemological, methodological, and historical
resources, and at the same time, a precise attention to tenuous things:
heterogenous experiences, practices, and ideas that do not ``fit'' in
classical, normative frames. Together with other analyses, my study
contributes to the collective reflection about ``what we are''---not by
seeking a clear definition, but by identifying and acknowledging
together the power of sociocultural categories in structuring scientific
practices, in society but also in academia. Constituted and constituting
exclusions of the sciences dedicated to communication are linked by
fundamental questions about where science ``is'' and what it ``does,''
epistemologically, practically, and politically. In the normal
circumstances of everyday academic life, as long as no problem is
encountered, we do not pay attention to the differences between internal
and external constraints, as informed by intellectual, institutional,
and sociocultural factors. But a better understanding of those
differences can lead us toward an individual and collective solution to
these academic challenges. Namely, by strategically playing with the
fluidity of disciplinary identity, we may be able to respond to the
``urgent needs'' of today with new, richer, and more considered
``rhetorical commonplaces.'' Thus, the discipline---not despite its
``impurity'' but as a result of it---constitutes a space of rules, and
at the same time, of densification, evolution, and freedom.




\section{Bibliography}\label{bibliography}

\begin{hangparas}{.25in}{1} 



Abbott, Andrew. \emph{Chaos of Disciplines}. Chicago: University of
Chicago Press, 2000.

Appadurai, Arjun. \emph{Modernity at Large: Cultural Dimensions of
Globalization}. Minneapolis: University of Minnesota Press, 1996.

Averbeck-Lietz, Stefanie, ed. \emph{Kommunikationswissenschaft im
internationalen Vergleich: Transnationale Perspektiven}. Wiesbaden:
Springer VS, 2017.

Averbeck-Lietz, Stefanie, Fabien Bonnet, Sarah Cordonnier, and Carsten
Wilhelm. ``Communication Studies in France: Looking for a~`Terre du
milieu'?'' \emph{Publizistik} (2019): 1--18.
\url{https://doi.org/10.1007/s11616-019-00504-3}.

Beniger, James R. ``Communication: Embrace the Subject, Not the Field.''
\emph{Journal of Communication} 43, no. 3 (1993): 18--25.
\url{https://doi.org/10.1111/j.1460-2466.1993.tb01272.x}.

Berelson, Bernard. ``The State of Communication Research.'' \emph{Public
Opinion Quarterly} 23, no. 1 (1959): 1--6.
\url{http://www.jstor.org/stable/2746418}.

Berg, Lawrence D., Edward H. Huijbens, and Henrik Gutzon Larsen.
``Producing Anxiety in the Neoliberal University.'' \emph{The Canadian
Geographer} 60, no. 2 (2016): 168--80.
\url{https://doi.org/10.1111/cag.12261}.

Boure, Robert. ``Les sciences de l'information et de la communication au
risque de l'expertise? Sur et sous des pratiques scientifiques''
{[}Information and communication sciences at the risk of expertise? On
and beneath scientific practices{]}. \emph{Réseaux} 15, no. 82--83
(1997): 233--53. \url{https://doi.org/10.3406/reso.1997.3068}.

Carlsson, Ulla. ``Has Media and Communication Research Become Invisible?
Some Reflections from a Scandinavian Horizon.'' \emph{Gazette} (Leiden,
Netherlands) 67, no. 6 (December 2005): 543--46.
\url{https://doi.org/10.1177/0016549205057548}.

Cordonnier, Sarah. ``Les sciences consacrées à la communication,
laboratoire disciplinaire? Analyses exploratoires d'un discours
`international'\,'' {[}The sciences dedicated to communication as
disciplinary laboratory? Exploratory analysis of an ``international''
discourse{]}.~\emph{Revue française des sciences de l'information et de
la communication} 10 (2017). \url{https://doi.org/10.4000/rfsic.2750}.

Cordonnier, Sarah. \emph{La production d'un territoire scientifique
international: Les Sciences consacrées à la communication à la croisée
des circulations rhétoriques, institutionnelles et biographiques} {[}The
production of an international scientific territory: The sciences
dedicated to communication at the crossroads of rhetorical,
institutional and biographical circulations{]}. Habilitation thesis,
Sorbonne Université, 2018.

Cordonnier, Sarah. ``Looking Back Together to Become `Contemporaries in
Discipline.''\,' \emph{History of Media Studies} 1 (2021).
\url{https://doi.org/10.32376/d895a0ea.b8153251}.

Cordonnier, Sarah, and Hedwig Wagner. ``Déployer l'interculturalité: Les
étudiants, un vecteur pour la réflexion académique sur l'interculturel;
Le cas des sciences consacrées à la communication et aux médias en
France et en Allemagne'' {[}Deploying interculturality: Students as a
vector for academic reflection on interculturality; The case of
communication and media sciences in France and Germany{]}.
\emph{Interkulturelle Kompetenz in deutsch-französischen Studiengängen}
{[}Key competences for higher education and employability{]}, edited by
Gundula Hiller et al., 221--34. Wiesbaden: Springer VS, 2017.
\url{https://doi.org/10.1007/978-3-658-14480-7_12}.

Craig, Robert T. ``Communication in the Conversation of Disciplines.''
\emph{Russian Journal of Communication} 1, no. 1 (2008): 7--23.
\url{https://doi.org/10.1080/19409419.2008.10756694}.

Craig, Robert T.. ``For a Practical Discipline.'' \emph{Journal of
Communication} 68, no. 2 (April 2018): 289--97.
\url{https://doi.org/10.1093/joc/jqx013}.

Crane, Diana. \emph{Invisible Colleges: Diffusion of Knowledge in
Scientific Communities}. Chicago and London: University of Chicago
Press, 1975.

Davallon, Jean. ``Objet concret, objet scientifique, objet de
recherche'' {[}Concrete object, scientific object, research object{]}.
\emph{Hermès} 38 (2004): 30--37.
\url{https://doi.org/10.4267/2042/9421}.

Donsbach, Wolfgang. ``The Identity of Communication Research.''
\emph{Journal of Communication} 56, no. 3 (2006): 437--48.
\url{https://doi.org/10.1111/j.1460-2466.2006.00294.x}.

Fabiani, Jean-Louis. ``À quoi sert la notion de discipline~?'' {[}What
is the purpose of the concept of discipline?{]}. In \emph{Qu'est-ce
qu'une discipline?}, edited by Jean Boutier, Jean-Claude Passeron, and
Jacques Revel, 11--34. Paris: Éditions de l'EHESS, 2006.

Foucault, Michel. ``The Discourse on Language.'' In \emph{The
Archaeology of Knowledge}, translated by A. M. Sheridan Smith,
215--37\emph{.} New York: Pantheon Books, 1972.

Foucault, Michel. ``The Confession of the Flesh.'' In
\emph{Power/Knowledge: Selected Interviews and Other Writings,
1972--1977,} edited by Colin Gordon, 194--228. New York: Pantheon, 1980.

Fuchs, Christian, and Jack Linchuan Qiu. ``Ferments in the Field:
Introductory Reflections on the Past, Present and Future of
Communication Studies.'' \emph{Journal of Communication}, 68, no. 2
(April 2018): 219--32. \url{https://doi.org/10.1093/joc/jqy008}.

García-Jiménez, Leonarda, and Martínez Susana Guillem. ``Does
Communication Studies Have an Identity? Setting the Bases for
Contemporary Research.'' \emph{Catalan Journal of Communication \&
Cultural Studies} 1, no. 1 (August 2009): 15--27.
\url{https://doi.org/10.1386/cjcs.1.1.15_1}.

Geertz, Clifford. \emph{After the Fact: Two Countries, Four Decades, One
Anthropologist}. Cambridge: Harvard University Press, 1995.

Gerbner, George, and Marsha Siefert. Introduction to ``Ferment in the
Field.'' Special issue, \emph{Journal of Communication} 33, no. 3
(September 1983): 4--5.
\url{https://doi.org/10.1111/j.1460-2466.1983.tb02400.x}.

Giddens, Anthony. \emph{The Consequences of Modernity}. Stanford:
Stanford University Press, 1990.

Gill, Rosalind. ``Breaking the Silence: The Hidden Injuries of
Neo-Liberal Academia.'' \emph{Feministische Studien} 34 (2016): 39--55.
\url{https://doi.org/10.1515/fs-2016-0105}.

Goffman, Erving. \emph{Stigma: Notes on the Management of Spoiled
Identity}. Englewood Cliffs, NJ: Prentice-Hall, 1963.

Golding, Peter, Helena Sousa, and Karin Raeymaeckers. Introduction to
``Future Priorities in European Media and Communication Research,''
edited by Peter Golding, Helena Sousa, and Karin Raeymaeckers. Special
issue, \emph{European Journal of Communication} 31, no. 1 (February
2016): 3--4. \url{https://doi.org/10.1177/0267323115614193}.

Haraway, Donna. ``Situated Knowledges: The Science Question in Feminism
and the Privilege of Partial Perspective.'' \emph{Feminist Studies} 14,
no. 3 (1988): 575--99. \url{https://doi.org/10.2307/3178066}.

Heilbron, Johan. ``The Social Sciences as an Emerging Global Field.''
\emph{Current Sociology} 62, no. 5 (September 2014): 685--703.
\url{https://doi.org/10.1177/0011392113499739}.

Heinderyckx, François. ``The Academic Identity Crisis of the European
Communication Researcher.'' In \emph{Media Technologies and Democracy in
an Enlarged Europe}, edited by Nico Carpentier et al., 357--62. Tartu:
University of Tartu Press, 2007.

Heinderyckx, François, Margaux Hardy, and Marc Vanholsbeeck. ``Les
revues scientifiques en information-communication: L'ère des
mutations?'' {[}Scientific journals in information and communication
sciences: An era of change?{]}. \emph{Questions de communication} 21
(2012). \url{https://doi.org/10.4000/questionsdecommunication.6643}.

Huber, Nathalie. \emph{Kommunikationswissenschaft als Beruf: Zum
Selbstverständnis von Professoren des Faches im deutschsprachigen Raum}
{[}Communication science as a profession: On the self-perception of
professors of the discipline in German-speaking countries{]}. Köln:
Herbert von Halem Verlag, 2010.

Jeanneret, Yves. ``La prétention sémiotique dans la communication''
{[}Semiotical pretention in communication studies{]}.~\emph{Semen}
23~(2007). \url{https://doi.org/10.4000/semen.8496}.

Koivisto, Juha, and Peter D. Thomas. \emph{Mapping Communication and
Media Research: Conjunctures, Institutions, Challenges}. Tampere:
Tampere University Press, 2010.

Lauf, Edmund. ``National Diversity of Major International Journals in
the Field of Communication.'' \emph{Journal of Communication} 55, no. 1
(March 2005): 139--51.
\url{https://doi.org/10.1111/j.1460-2466.2005.tb02663.x}.

Löblich, Maria, and Andreas M. Scheu. ``Writing the History of
Communication Studies: A Sociology of Science Approach.''
\emph{Communication Theory} 21, no. 1 (February 2011): 1--22.
\url{https://doi.org/10.1111/j.1468-2885.2010.01373.x}.

Maingueneau, Dominique. ``Analysing Self-Constituting Discourses.''
\emph{Discourse Studies} 1, no. 2 (May 1999): 175--99.
\url{https://doi.org/10.1177/1461445699001002003}.

Meyen, Michael. ``Fifty-Seven Interviews with ICA Fellows.''
\emph{International Journal of Communication} 6 (2012): 1460--886.

Meyen, Michael. ``International Communication Association Fellows: A
Collective Biography.'' \emph{International Journal of Communication} 6
(2012): 2378--96.

Nordenstreng, Kaarle. ``Discipline or Field? Soul-Searching in
Communication Research.'' \emph{Nordicom Review} 27 (2007): 211--22.

Peters, John Durham. ``Institutional Sources of Intellectual Poverty in
Communication Research.'' \emph{Communication Research} 13, no. 4
(October 1986): 527--59.
\url{https://doi.org/10.1177/009365086013004002}.

Pooley, Jefferson D. \emph{James W. Carey and Communication Research:
Reputation at the University's Margins}. New York: Peter Lang, 2016.

Schlanger, Judith. \emph{La vocation} {[}The calling{]}. Paris: Seuil,
1997.

Schütz, Alfred. ``Common-Sense and Scientific Interpretation of Human
Action.'' \emph{Philosophy and Phenomenological Research} 14, no. 1
(1953): 1--38. \url{https://doi.org/10.2307/2104013}.

Simonson, Peter. ``Writing Figures into the Field: William McPhee and
the Parts Played by People in Our Histories of Media Research.'' In
\emph{The History of Media and Communication Research: Contested
Memories}, edited by David Park and Jefferson Pooley, 291--320. New
York: Peter Lang, 2008.

Simonson, Peter, and David W. Park, eds. \emph{The International History
of Communication Study}. New York and London: Routledge, 2016.

Spivak, Gayatri Chakravorty. ``Can the Subaltern Speak?'' In
\emph{Marxism and the Interpretation of Culture}, edited by Cary Nelson
and Lawrence Grossberg, 271--313. Chicago: University of Illinois Press,
1988.

Stanfill, Mel. ``Finding Birds of a Feather: Multiple Memberships and
Diversity without Divisiveness in Communication Research.''
\emph{Communication Theory} 22, no. 1 (February 2012): 1--24.
\url{https://doi.org/10.1111/j.1468-2885.2011.01395.x}.

Streeter, Thomas. ``For the Study of Communication and against the
Discipline of Communication.'' \emph{Communication Theory} 5, no. 2 (May
1995): 117--29.
\url{https://doi.org/10.1111/j.1468-2885.1995.tb00101.x}.

Vinck, Dominique. ``L'activité de recherche en situation d'injonctions
contradictoires'' {[}Research activity in a situation of contradictory
injunctions{]}. In \emph{Les études de sciences: Pour une réflexivité
institutionnelle}, edited by Joëlle Le Marec, 65--80. Paris: Éditions
des archives contemporaines, 2010.

Waisbord, Silvio R. \emph{Communication: A Post-Discipline}. Cambridge:
Polity, 2019.

Waisbord, Silvio R., and Claudia Mellado. ``De‐westernizing
Communication Studies: A Reassessment.'' \emph{Communication Theory} 24
(2014): 361--72. \url{https://doi.org/10.1111/comt.12044}.



\end{hangparas}

\vspace{2em}

\hypertarget{acknowledgements}{%
\section{Acknowledgements}\label{acknowledgements}}

I am very grateful to the editors of the journal, the anonymous
reviewers of my article, the members of the History of Media Studies
Working Group, and the copy editor of this text, for their remarks and
comments, which have contributed in many ways to the improvement of this
text.




\end{document}