% see the original template for more detail about bibliography, tables, etc: https://www.overleaf.com/latex/templates/handout-design-inspired-by-edward-tufte/dtsbhhkvghzz

\documentclass{tufte-handout}

%\geometry{showframe}% for debugging purposes -- displays the margins

\usepackage{amsmath}

\usepackage{hyperref}

\usepackage{fancyhdr}

\usepackage{hanging}

\hypersetup{colorlinks=true,allcolors=[RGB]{97,15,11}}

\fancyfoot[L]{\emph{History of Media Studies}, vol. 2, 2022}


% Set up the images/graphics package
\usepackage{graphicx}
\setkeys{Gin}{width=\linewidth,totalheight=\textheight,keepaspectratio}
\graphicspath{{graphics/}}

\title[The Textile Imaginary]{The Textile Imaginary: An Alternative Interpretation in Communication Studies} % longtitle shouldn't be necessary

% The following package makes prettier tables.  We're all about the bling!
\usepackage{booktabs}

% The units package provides nice, non-stacked fractions and better spacing
% for units.
\usepackage{units}

% The fancyvrb package lets us customize the formatting of verbatim
% environments.  We use a slightly smaller font.
\usepackage{fancyvrb}
\fvset{fontsize=\normalsize}

% Small sections of multiple columns
\usepackage{multicol}

% Provides paragraphs of dummy text
\usepackage{lipsum}

% These commands are used to pretty-print LaTeX commands
\newcommand{\doccmd}[1]{\texttt{\textbackslash#1}}% command name -- adds backslash automatically
\newcommand{\docopt}[1]{\ensuremath{\langle}\textrm{\textit{#1}}\ensuremath{\rangle}}% optional command argument
\newcommand{\docarg}[1]{\textrm{\textit{#1}}}% (required) command argument
\newenvironment{docspec}{\begin{quote}\noindent}{\end{quote}}% command specification environment
\newcommand{\docenv}[1]{\textsf{#1}}% environment name
\newcommand{\docpkg}[1]{\texttt{#1}}% package name
\newcommand{\doccls}[1]{\texttt{#1}}% document class name
\newcommand{\docclsopt}[1]{\texttt{#1}}% document class option name


\begin{document}

\begin{titlepage}

\begin{fullwidth}
\noindent\LARGE\emph{Exclusions in the History of Media Studies
} \hspace{25mm}\includegraphics[height=1cm]{logo3.png}\\
\noindent\hrulefill\\
\vspace*{1em}
\noindent{\Huge{The Textile Imaginary: An Alternative\\\noindent Interpretation in Communication Studies\par}}

\vspace*{1.5em}

\noindent\LARGE{Daniel H. Cabrera Altieri} \href{https://orcid.org/0000-0001-6781-260X}{\includegraphics[height=0.5cm]{orcid.png}}\par\marginnote{\emph{Daniel H. Cabrera Altieri, ``The Textile Imaginary: An Alternative Interpretation in Communication Studies,'' \emph{History of Media Studies} 2 (2022), \href{https://doi.org/10.32376/d895a0ea.bde56c1e}{https://doi.org/ 10.32376/d895a0ea.bde56c1e}.} \vspace*{0.75em}}
\vspace*{0.5em}
\noindent{{\large\emph{Universidad de Zaragoza}, \href{mailto:danhcab@gmail.com}{danhcab@gmail.com}\par}} \marginnote{\href{https://creativecommons.org/licenses/by-nc/4.0/}{\includegraphics[height=0.5cm]{by-nc.png}}}

% \vspace*{0.75em} % second author

% \noindent{\LARGE{<<author 2 name>>}\par}
% \vspace*{0.5em}
% \noindent{{\large\emph{<<author 2 affiliation>>}, \href{mailto:<<author 2 email>>}{<<author 2 email>>}\par}}

% \vspace*{0.75em} % third author

% \noindent{\LARGE{<<author 3 name>>}\par}
% \vspace*{0.5em}
% \noindent{{\large\emph{<<author 3 affiliation>>}, \href{mailto:<<author 3 email>>}{<<author 3 email>>}\par}}

\end{fullwidth}

\vspace*{1em}




\hypertarget{abstract}{%
\section{Abstract}\label{abstract}}
This article\footnote{Grant PID2020-113574RB-I00 funded by MCIN/AEI/ 10.13039/50110001103}
 draws attention to the relevance of weaving for an
alternative conceptualization for the theories of communication. Current
vocabulary and a series of everyday expressions keep alive a memory that
escapes the \emph{textile amnesia} which seems to characterize our times.
Textile metaphors speak of a communicational substrate that goes beyond
the narrative resource to its heuristic and cognitive dimensions. A logo-media-centric vision has obscured the role of the loom and weaving
in the history of media and, especially, the vivacity of their presence
and significance for the Global South countries. The acknowledgment of a
\emph{subterranean centrality of the textile} shows some traces for the study
of communication from the creative interlacing, care for life and of the
\emph{social fabric}.


\hypertarget{resumen}{%
\section{Resumen}\label{resumen}}

El presente artículo llama la atención sobre la relevancia del tejido
para una conceptualización alternativa de las teorías de la
comunicación. El vocabulario actual y una serie de expresiones
cotidianas mantienen viva una memoria que escapa a la \emph{amnesia
textil} que parece caracterizar nuestra época. Las metáforas textiles
hablan de un sustrato comunicacional que va más allá del recurso
narrativo hacia sus dimensiones heurísticas y cognitivas. Una visión
logomediacéntrica ha ocultado el papel del telar y los tejidos en la
historia de los medios de comunicación y, en especial, la vivacidad de
su presencia y significación para los pueblos del Sur Global. La
constatación


\enlargethispage{2\baselineskip}

\vspace*{2em}

\noindent{\emph{History of Media Studies}, vol. 2, 2022}


 \end{titlepage}

\noindent de una \emph{centralidad subterránea de lo textil} muestra
algunos trazos para el estudio de la comunicación desde el entrelazado
creador, el cuidado de la vida y del \emph{tejido social}.

\vspace*{2em}


\hypertarget{introduction}{%
\section{Introduction}\label{introduction}}

\newthought{The study of communication} has been classified from different
perspectives, giving rise to multiple assessments and evaluations in
various languages and publications.\footnote{There are multiple
  metatheoretical research studies. For the Latin American case, see
  Raúl Fuentes Navarro, \emph{``Cuatro décadas de internacionalización
  académica en el campo de estudios de la comunicación en América
  Latina'',} \emph{Anuario Electrónico de Estudios en Comunicación
  Social, Disertaciones} no. 2 (2016): 8-26,
  \url{https://doi.org/10.12804/disertaciones.09.02.2016.01}; José
  Marques de Melo, \emph{Pensamiento comunicacional latinoamericano.
  Entre el saber y el poder} (Seville, \emph{Comunicación Social},
  2009); Erick R. Torrico Villanueva, \emph{La comunicación pensada
  desde América Latina (1960-2009)) (}Salamanca: \emph{Comunicación
  Social}, 2016). There are maps of research such as the "Mapping Media
  and Communication Research" carried out by the Communication Research
  Centre of the University of Helsinki in Finland and referring to
  several European countries, the United States, Japan and Australia.
  Juha Herkman, "Current trends in media research," \emph{Nordicom
  Review} 29 :1 (2008): 145--59. Reports available on
  \url{http://www.valt.helsinki.fi/blogs/crc/en/mapping.htm}. For the
  French case, see Thierry Lancien et al\emph{., "La recherche en
  communication en France. Tendences et carences",} \emph{Recherche \&
  communication}, ed. by Thierry Lancien, \emph{MEI (Médiation et
  Information),} no. 14 (Saint-Denis: L\textquotesingle Harmattan,
  2001): 37-62, and the \emph{Journal of Communication}, "Ferment in the
  Field", special issue 33, no. 3 (1983); "Future of the Discipline",
  Special Issue 43, No. 3 (1993), and "The State of the Art in
  Communication Theory and Research", Special Issue 54, No. 4 (2004). Spanish-language quotes in the text have been translated into English by the translator.} This has generated a meta discursive approach with
various classifications of theories, among which the one put forward by
Robert T. Craig on the "\emph{traditions}" of research analyzed from a
pragmatist approach stands out.\footnote{Robert. T. Craig,
  ``Communication Theory as a Field,'' \emph{Communication Theory} 9,
  no. 2 (1999): 119--20.} The present work takes as a starting point the
consideration of the cognitive metaphors and social imaginaries with
which communication has been studied to propose an alternative
interpretation of communication centered on the metaphor and the
imaginary of the fabric.

Communication theories are interpretations that, at some point, resort
to heuristic metaphors of various kinds, among which those of
\emph{transmission} and the \emph{network} are of significance. A
documented search reveals that no interpretation has yet been made from
the imaginary of textiles in Euro-American or Global North research
centers and universities. However, weaving is not just another metaphor,
as we will see, but a fundamental one that has been undoubtedly present
since ancient times in various cultures to refer to communication in its
various aspects.

This text focuses on the theory of social imaginaries to interpret the
cognitive metaphors of communication theories. Logo media centrism in
its linearity and teleological evolutionism are discussed as a possible
cause of exclusion of textiles. The role of power looms in the
industrial revolution and their influence on digital conception using
the punch card and the change in the way of looking are presented.
Following on from that, the metaphors and textile imagery present in
today\textquotesingle s everyday language are analyzed and the case of
classical Greece as a fundamental moment in the inherited Western
Euro-American conception of weaving is discussed. Finally, a textile
interpretation of communication is proposed as a creative intertwine, a
network of care and a space of \emph{social fabric}.

This article will continue, in a future text, to analyze in more detail
the textile imaginary from the Global South, particularly from the
Andean space, to propose an alter/native understanding\textsuperscript{3} of the phenomenon of
communication and its study, communicology.

\hypertarget{communication-and-social-imaginary}{%
\section{Communication and Social
Imaginary}\label{communication-and-social-imaginary}}

Communication\marginnote{\textsuperscript{3}\setcounter{footnote}{3} "It is
  convenient to write this alter/native with a slash to mark not only
  its potential character of otherness in the face of the Communication
  that we already have installed in academic studies... but also to make
  a certain emphasis on the connection of that otherness with the
  native, the proper and differentiating." Erick R. Torrico Villanueva,
  "Decolonizing Communication," in \emph{Comunicación, decolonialidad y
  Buen Vivir,} coord. by Francisco Sierra Caballero and Claudio
  Maldonado Rivera (Quito: CIESPAL, 2016): 96.} theories have been formulated from many metaphors
converted into models and conceptual developments that have guided the
activity of researchers, media and communicators. In this section, what
is proposed is to think about the different interpretations of the
phenomenon of communication as imaginaries spoken and visualized by
metaphors that culminate with those referring to \emph{transport} and
\emph{network}.

\hypertarget{theory-metaphor-and-imaginary}{%
\subsection{Theory, Metaphor and
Imaginary}\label{theory-metaphor-and-imaginary}}

\begin{quote}
What, then, is the truth? A mobile army of metaphors, metonyms,
anthropomorphisms... and which, after prolonged use, appear to a people
to be fixed, canonical, obligatory... metaphors that have become
worn-out and without sensible force.

---Friedrich Nietzsche, \emph{On Truth and Lies in the Extra moral
Sense}.
\end{quote}

Metaphors are fundamental instruments in scientific production that
enhance analogical reasoning, the elaboration of hypotheses, the
interpretation of results and the communication of findings.\footnote{Cynthia
  Taylor and Bryan M. Dewsbury, ``On the Problem and Promise of Metaphor
  Use in Science and Science,'' \emph{Journal of Microbiology \& Biology
  Education} 19, no. 1 (2018); Federico Pérez Álvarez and Carmen
  Timoneda Gallart, ``El poder de la metáfora en la comunicación humana:
  ¿Qué hay de cierto? La metáfora en la teoría y la práctica perspectiva
  en neurociencia,'' \emph{International Journal of Developmental and
  Educational Psychology} 6, no. 1 (2014).} The heuristic capacity of
metaphorical reasoning makes it essential for doing science.\footnote{Olaf
  Jäkel, Martin Döring and Anke Beger, ``Science and metaphor: a truly
  interdisciplinary perspective: The third international metaphorik.de
  workshop,'' \emph{Metaphorik.de -- online journal on metaphor and
  metonymy}, no. 26 (2016).} Cognitive psychology states that "every
concept is the result of a long series of spontaneous analogies, and the
elements of a situation are categorized exclusively through analogies,
however trivial they may seem.''\footnote{Douglas Hofstadter and
  Emmanuel Sander, \emph{La analogía. El motor del pensamiento}, trad.
  Roberto Musa Giuliano (Barcelona: Tusquets, 2018): 73.}

In this same tradition, long ago, George Lakoff and Mark Johnson made
clear the cognitive role of metaphors in everyday life.\footnote{George
  Lakoff and Mark Johnson, \emph{Metáforas de la vida cotidiana}, trad.
  de Carmen González Marín (Cátedra: Madrid, 2009).} In both cases, in
the sciences or in everyday life, metaphor is not just a trope but a
form of knowledge or even ``the engine of thought,'' since ``without
concepts there are no thoughts and without analogies there are no
concepts.''\footnote{Hofstadter and Sander, \emph{La analogía,} 2.} We
learn the new, the unexpected, the strange, by similarities and kinship
with the known.

Although philosophy had emphasized the role of metaphor, at least since
Aristotle's \emph{Poetics}, it is Friedrich Nietzsche who underlines the
cognitive dimension of metaphor.\footnote{Daniel Innerarity, ``La
  seducción del lenguaje. Nietzsche y la metáfora,'' \emph{Contrastes:
  revista internacional de filosofía}, no. 3 (1998); Cirilo Flórez
  Miguel, ``Retórica, metáfora y concepto en Nietzsche,'' \emph{Estudios
  Nietzsche}, no. 4 (2004).} The metaphor speaks of living social and
cultural contexts, the subjects\textquotesingle{} experiences, moods,
and interests\textsuperscript{10}\footnote{Eduardo de Bustos, \emph{La metáfora. Ensayos
  transdisciplinares} (Madrid: Fondo de Cultura Económica / UNED, 2000).}.
From the theory of social imaginaries, the task of knowledge in the
social sciences is to dilute \emph{solid} concepts in their \emph{liquid
soil and in the} gaseous \emph{environment in} which they
emerged.\textsuperscript{11} Interpretation conceived
as an alchemical task that returns the determined to the indeterminate,
the defined to the indefinite, as a way of giving rise to new
interpretations of the categories with which we explain social reality
and, in this case, the phenomenon of communication. The imaginary
approach aims to question the epistemological basis of communication\marginnote{\textsuperscript{10} "It is
  convenient to write this alter/native with a slash to mark not only
  its potential character of otherness in the face of the Communication
  that we already have installed in academic studies... but also to make
  a certain emphasis on the connection of that otherness with the
  native, the proper and differentiating." Erick R. Torrico Villanueva,
  "Decolonizing Communication," in \emph{Comunicación, decolonialidad y
  Buen Vivir,} coord. by Francisco Sierra Caballero and Claudio
  Maldonado Rivera (Quito: CIESPAL, 2016): 96.}
studies\marginnote{\textsuperscript{11}\setcounter{footnote}{11} Daniel H. Cabrera Altieri, \emph{Tecnología como
  ensoñación. Ensayos sobre el imaginario tecnocomunicacional} (Temuco:
  Ediciones Universidad de la Frontera, 2022).} which, it should be remembered, ``do not delve much into
epistemology,'' although ``seas of ink have been written about the
supposed disciplinary status of communication.''\footnote{Tanius Karam,
  ``Tensiones para un giro decolonial en el pensamiento comunicológico.
  Abriendo la discusión'', \emph{Chasqui. Revista Latinoamericana de
  Comunicación}, no. 133 (2016--2017).}

Approaching theory as a host of cognitive metaphors leads us to consider
its concepts and implications as constructions inspired by the social
imaginary, on a path that interpretation can retrace, starting with an
attitude of estrangement.\footnote{Emmanuel Lizcano, \emph{Metáforas que
  nos piensan} (Madrid: Ediciones Bajo Cero / Traficantes de Sueños,
  2014): 37--71; Jean-Jacques Wunenburger, \emph{La vida de las
  imágenes}, trad. de Hugo Francisco Bauzá (Buenos Aires: UNSAM, 2005):
  21--50.} Metaphor is a wording and an image of the social
imaginary\footnote{See Gilbert Durand, \emph{Las estructuras
  antropológicas del imaginario}, trad. Víctor Goldstein (México: Fondo
  de Cultura Económica, 2004); Cornelius Castoriadis, \emph{La
  institución imaginaria de la sociedad}, trad. Antoni Vicens y Marco
  Aurelio Galmarini (Barcelona: Tusquets, 1993).}. An imaginary that
speaks of the arbitrary and creative relationships between what seeks to
understand its deeper role in the creation of universes of
meanings.\footnote{Daniel H. Cabrera Altieri, \emph{Lo tecnológico y lo
  imaginario. Las nuevas tecnologías como creencias y esperanzas
  colectivas} (Buenos Aires: Biblos, 2006).}

\hypertarget{communication-theories-and-their-metaphors}{%
\subsection{Communication Theories and Their
Metaphors}\label{communication-theories-and-their-metaphors}}

The explanations of communication contain numerous heuristic metaphors,
in most cases not taken as such, and which have been analyzed on several
occasions. For example, the ``\emph{hypodermic} needle'' and the ``magic
\emph{bullet}'' are used\footnote{José Luis Dader, ``La evolución de las
  investigaciones sobre la influencia de los medios y su primera etapa:
  Teorías del impacto directo,'' in \emph{Opinión pública y comunicación
  política}, de Alonso Muñoz et al. (Madrid: Eudema, 1990).} to indicate
an immediate \emph{effect} and thereby introduce an understanding of
communication as social \emph{health} and as \emph{war}. The ``\emph{two
step flow of communication''} that will have a great journey with the
idea of communication\footnote{Paul F. Lazarsfeld and Elihu Katz,
  \emph{La influencia personal: el individuo en el proceso de
  comunicación de masas} (Barcelona: Hispano Europea, 1979); Elihu Katz,
  ``The Two-Step Flow of Communication: An Up-To-Date Report on an
  Hypothesis,'' \emph{Political Opinion Quarterly} 21, no. 1 (1957).} as
fluid and \emph{process} and of society divided into stages with
\emph{opinion leaders} as an explanatory figure. The ``\emph{spiral} of
silence''\footnote{Elizabeth Noelle-Neumann, \emph{La espiral del
  silencio}, trad. Javier Ruíz Calderón (Barcelona: Paidós, 2003).}
presents the circularity and recursion of \emph{minority silence}. The
metaphor of the ``\emph{agenda}''\footnote{Maxwell E. McCombs and Donald
  L. Shaw, ``The Agenda-Setting Function of Mass Media,'' \emph{Public
  Opinion Quarterly} 36, no. 2 (1972).} to conceive of society as a
\emph{list} of topics of \emph{conversation}; \emph{frames}\textsuperscript{20} applied to communication where
information is a \emph{look} through a \emph{window}, a \emph{visual
frame}. And so we also have the metaphors of the
\emph{conduit}\textsuperscript{21}\emph{, the}
fluid\textsuperscript{22}, the
ecology\textsuperscript{23}\emph{, the network}\textsuperscript{24}\emph{,} the \emph{highway}.\textsuperscript{25} Expressions that
designate an approach or a theory with concepts whose metaphor is
transparent. Even the use of the word ``communication'' tends to be
connoted as concord and coexistence in clear reference, whether
consciously or not, to the etymologies of ``communication'' (the common,
to \emph{make common, communion}, \emph{community}) \emph{and
``information'' }(\emph{to} give form)\emph{ that refer to} a
metaphorical hummus\emph{, a social imaginary, which remains at the core
of current meanings.}


Klaus Krippendorff grouped together some metaphors of\\\noindent
communication\textsuperscript{26} that, based on research
and common sense, function as true theories that construct reality. He
laid them out in six central metaphors:

\begin{itemize}
\item
  \emph{Metaphors of the receptacle:} referring to ``content,\emph{''}
  ``full of meaning,'' ``meaningless,'' ``empty sentences,'' etc.
\item
  \emph{Conduit metaphors}: derived from the technologies of cable,
  tube, flow, source, channels, etc.
\item
  \emph{Metaphors of control}: causal phenomenon, environment,
  instruments, directing, active passive, improving effectiveness,
  successful communication, etc.
\item
  \emph{Metaphors\marginnote{\textsuperscript{20} Gregory
  Bateson, ``Una teoría del juego y de la fantasía'', en \emph{Pasos
  hacia una ecología de la mente}, trad. Ramón Alcalde (Buenos Aires:
  Editorial Lohlé-Lumen, 1991).} of\marginnote{\textsuperscript{21} Analyzed by Michael Reddy, ``The conduit
  metaphor: A case or Frame Conflict in our language about language'',
  en \emph{Metaphor and Thought}, ed. de Andrew Ortony (Cambridge:
  Cambridge University Press, 1993): 164-201.} transmission\marginnote{\textsuperscript{22} Vanina Papalini, ``La comunicación según las metáforas
  oceánicas,'' \emph{Razón y Palabra}, no. 78 (2011--2012).}}: of\marginnote{\textsuperscript{23} Analyzed by Carlos A. Scolari, ``Ecología de los
  medios: de la metáfora a la teoría (y más allá),'' in \emph{Ecología
  de los medios: entornos, evoluciones e interpretaciones} (Gedisa:
  Barcelona, 2015).} technologies:\marginnote{\textsuperscript{24} Analyzed by Pierre
  Musso, ``Génesis y crítica de la noción de red'', trad. de Jorge
  Márquez Valderrama, \emph{Ciencias Sociales y Educación} 2, no. 3
  (2013).} deciphering,\marginnote{\textsuperscript{25} See Patrice Flichy,
  \emph{Lo imaginario de internet}, trad. Félix de la Fuente and Mireia
  de la Fuente Rocafort (Madrid: Tecnos, 2003): 25-47.}
  coding,\marginnote{\textsuperscript{26}\setcounter{footnote}{26} Klaus Krippendorff, ``Principales metáforas de la
  comunicación y algunas reflexiones constructivistas acerca de su
  utilización,'' in \emph{Construcciones de la experiencia humana II},
  ed. Marcelo Pakman (Barcelona: Gedisa, 1997).} transmitting, encoding/decoding, etc.
\item
  \emph{Metaphors of war:} ``defensible'' claims, ``hitting the mark,''
  ``winning'' an argument, and so on
\item
  \emph{Metaphors of dance-ritual}: performance, participation, contact,
  empathy, etc.
\end{itemize}

Jean Pierre Meunier has also analyzed the explanatory use of
metaphors\footnote{Jean Pierre Meunier, ``Las metáforas de comunicación
  como metáforas que cobran realidad'', \emph{Signo y Pensamiento} 16,
  no. 30 (1997).} by understanding their code dimension as a telegraph,
their action dimension as a coded exchange, the strategy and model of
the computer in the understanding of cognitive processes. Like
others,\footnote{Juan Ramón Muñoz-Torres, ``Abuso de la metáfora y
  laxitud conceptual en comunicación,'' \emph{Mediaciones Sociales.
  Revista de Ciencias Sociales y de la Comunicación}, no, 11 (2012).}
Meunier's analysis seeks to warn about the misuse or abuse of metaphors
in theories. As it is understood here, the essential problem is not the
twisted use of metaphors, but rather that their presence, in a
transparent or hidden way, cognitively constructs a \emph{reality} on
which one acts in accordance with that same knowledge. Communication is
what their theories say and with those definitions we act, we
communicate.

\hypertarget{communication-technologies-from-transport-to-the-network}{%
\subsection{Communication Technologies: From
Transport to the
Network}\label{communication-technologies-from-transport-to-the-network}}

Metaphors in their heuristic dimension play a very important role in the
understanding of communication imaginaries, in tracing their genealogy
and in identifying the conceptual matrices from which communication
theories have been generated. In this sense, the work of Armand
Mattelart and his interpretation of communication stands out, in his
words, the ``invention of communication,'' from the archaeologies of
four histories: the domestication of \emph{flows} and society in
\emph{motion; the} conception and manufacture of a universal \emph{bond}
between humans; geopolitical \emph{space}; and the \emph{normalization}
and emergence of the individual \emph{calculable}.\footnote{Armand
  Mattelart, \emph{La invención de la comunicación}, trad. Gilles
  Multigner (Bosch: Barcelona, 1995); Armand Mattelart, \emph{La
  comunicación-mundo. Historia de las ideas y de las estrategias}, trad.
  Gilles Multigner (México: Siglo XXI, 1997).} The tracing of the ideas
and strategies of communication in the contemporary world traced from
the moment when the mass media did not yet exist allows him to conclude
that ``the biological analogy has been installed as a natural matrix, a
great unifying paradigm, to account for the functioning of communication
systems and the link that unites them to society as an organic
whole.''\footnote{Mattelart, \emph{La invención,} 370.}

John Durham Peters has made his own history of the idea of communication
where \emph{dialogue}, whose model is Plato's Socrates, and
\emph{dissemination,} according to the model of the Jesus of the
evangelists, the media as creators of \emph{ghosts}, communication with
the dead\emph{, among others, constitute true conceptual metaphorical
nodes.}\footnote{John Durham Peters, \emph{Hablar al aire. Una historia
  de la idea de comunicación}, trad. José María Ímaz (México: Fondo de
  Cultura Económica, 2014).} Far from being narrative resources,
metaphors allow us to delve into the matrices of communication, into the
space where ideas and imaginaries are knotted to feed a vision of
communication ``as a risky adventure without guarantees.'' To justify
this, he recalls the etymology of the word communication in that it is
problematic and, therefore, is not usually cited. The author dismisses
considering the Latin vocabulary \emph{communicare} as the origin and
refers to the Greek term \emph{koino}, ``more rarely cited but equally
relevant,'' which also means to make common, to communicate, to share,
``but also to pollute or soil.''\footnote{Peters, \emph{Hablar al aire,}
  329--30.}

In both cases, from two very different approaches, the task of
understanding communication and its media as a central element of
today's society is faced and, perhaps for this reason, Lucien Sfez
affirms that ``a structural convergence between the systematic use of
metaphors by the science of communication and the new fact that current
communication has also become a \emph{symbolic figure} of
importance.''\footnote{Lucien Sfez, \emph{Crítica de la comunicación},
  trad. Aníbal C. Leal (Buenos Aires: Amorrortu, 1995): 46.}

Metaphors have also played a fundamental role in organizing alternative
models of communication theories. Under the suggestive title of ``The
Telegraph and the Orchestra,''\footnote{Yves Winkin, ``El telégrafo y la
  orquesta,'' in \emph{La nueva comunicación}, ed. Gregory Bateson et
  al., trad. Jorge Fibla (Barcelona: Kairos, 1984).} Yves Winkin opposes
two ways of explaining communication.

On the one hand, Claude Shannon's ``mathematical theory of
information,'' which is postulated as the exact transmission of a
message from one point to another. The complexity of the mathematical
explanation has generated the feeling that the only element of Shannon's
theory that has been able to inherit ``to the layman in engineering is
the image of the telegraph that still permeates the original scheme. We
could thus speak \emph{of a telegraphic model of
communication}.''\footnote{Winkin, ``El telégrafo y la orquesta,'' 18.}

On the other hand, an interdisciplinary group with different university
campuses composed of Gregory Bateson, Ray Birdwhistell, Edward Hall,
Erving Goffman, Don Jackson, and Paul Watzlawick, among others,
encouraged the idea of ``communication as an integrated whole,'' that
is, as ``a permanent social process that integrates multiple modes of
behavior: speech, gesture, gaze, mimicry, etc. the inter-individual
space, etc.''\footnote{Winkin, ``El telégrafo y la orquesta,'' 22--23.}
For this reason, Winkin argues that ``the analogy of the orchestra is
intended to make us understand how each individual can be said to
participate in communication, rather than to say that it constitutes the
origin or end of it.''\footnote{It may be objected that a model is not a
  metaphor, but in this regard, it is worth remembering Max
  Black\textquotesingle s approach when he argues that "every metaphor
  is the warning of a submerged model". Max Black quoted in Andrés
  Rivadulla, ``Metáforas y modelos en ciencia y filosofía'',
  \emph{Revista de Filosofía} 31, n.º 2 (2006): 189-202.}

Winkin proposes to speak of an \emph{orchestral model} of communication
as opposed to the telegraphic linear. The truth is that
telecommunications and information technology have transformed our
entire world, including the imaginary \emph{of} and \emph{from} which
human communication is thought.\footnote{Juan Luis Pintos,
  ``Comunicación, construcción de la realidad e imaginarios sociales,''
  \emph{Utopía y Praxis Latinoamericana} 10, no. 29 (2005).}
Undoubtedly, the mathematical model of information has colonized the
metaphorical universe with an image/idea that has flooded
everything.\footnote{See Pablo Rodríguez, \emph{Historia de la
  información. Del nacimiento de la estadística y la matemática moderna
  a los medios masivos y las comunidades virtuales} (Buenos Aires:
  Capital Intelectual, 2012); James Gleick, \emph{La información.
  Historia y realidad}, trad. Juan Rabasseda and Teófilo de Lozoya
  (Barcelona: Crítica, 2011).} \emph{Point-to-point transmission,}
information transport\emph{, a} data packet, \emph{conduit and}
channel\emph{, code}, encoding \emph{and} decoding \emph{,}
\emph{accuracy} Mathematics, among others, are part of a set of
metaphors with which people, institutions, companies, governments, and
researchers reflect. A few years later, James W. Carey proposed the
famous distinction between \emph{transmission} models and
\emph{ritual}\footnote{James W. Carey, ``A Cultural Approach to
  Communication,'' in \emph{Communication as Culture: Essays on Media
  and Society} (Boston: Unwin Hyman, 1989).} with some coincidences with
what has been said, but with the aim of seeking a constitutive theory of
communication.

If we focus on current digital technologies, other metaphors stand out,
such as \emph{viral}, \emph{cloud}, \emph{network}. Viral comes directly
from marketing designed to spread information very quickly, making it
highly likely to be transmitted from person to person through electronic
means. The phenomenon of disinformation has laid bare the ambivalence of
the \emph{virus}, highlighting its role as information consciously
manipulated to provoke states of opinion or actions in a certain
direction. For its part, the \emph{cloud} has a rich cultural history in
literature, music, and art that ``goes beyond the representation of the
sublimity of computing'' and whose analysis has already been
developed\footnote{Vincent Mosco, \emph{La nube. Big Data en un mundo
  turbulento} (Barcelona: Intervención cultural / Biblioteca Buridán,
  2014): 26. See John Durham Peters, \emph{The Marvelous Clouds: Toward
  a Philosophy of Elemental Media} (Chicago: University of Chicago,
  2015).}. The metaphor of the network \emph{has been mentioned},
perhaps, the metaphor or ``fetish word,'' as Pierre Musso states, more
developed from different points of view as a global social, cultural,
and economic logic.\footnote{Manuel Castells, \emph{La sociedad de la
  información,} Vol. 1, \emph{La sociedad red}, trad. Carmen Martínez
  Gimeno and Jesús Alborés (Madrid: Alianza, 2000).}

The network metaphor has the foundational capacity of the world in which
we live. Every metaphor needs to be carefully advanced, although in this
case even more so because ``metaphors are nothing without the political
and metaphysical positions they defend.''\footnote{Castells, \emph{La
  sociedad de la información,} 47.} \emph{Netting} has an obvious origin
in the manufacture of fabrics, the net as a set of intertwined threads,
lines, and knots. In that sense, he refers to ``the mythology of
weaving.''\footnote{Musso, ``Génesis y crítica,'' 203; see Durand,
  \emph{Las estructuras}, 330--33.} However, it arrives at the present
day through the organic conception that, from Hippocrates, through Galen
and Harvey, conceives the human body as possessing a hidden
\emph{network} of nerves and veins with invisible \emph{flows} and the
brain as \emph{a mesh}.\footnote{See Robert Jastrow, \emph{El telar
  mágico. El cerebro humano y el ordenador}, trad. Domingo Santos
  (Barcelona: Salvat, 1993).} An analogy that would be used repeatedly
by René Descartes, Denis Diderot, Henri Saint-Simon, Herbert Spencer,
and even in early cybernetics and contemporary ideologues.\footnote{Musso,
  ``Génesis y crítica,'' 205.} However, Musso believes that ``the modern
concept of the network is formed in Saint-Simon\textquotesingle s
philosophy. He produces the theory of this new logical and biopolitical
vision of the network.''\footnote{Musso, ``Génesis y crítica,'' 211.}

The current network concept with channels, conduits, cables, and waves
allows for multiple, permanent and switchable connection. The industrial
revolution made it possible to invent mechanical networks, such as the
telegraph or the railway, and the transformations in computer techniques
made self-organizing networks possible.

\newpage The network has no beginning and no end and with all its mixture
``imposes itself as a \emph{technology of the spirit.}''\footnote{Sfez,
  \emph{Crítica}, 379; emphasised in the original.} This is evident, on
the one hand, in the constant use of the analogy between the mind and
the computer and, on the other, in the qualification of ``intelligent''
referring to the devices and their capabilities. The network is an
intermediate figure between the tree, which is too linear, and chaos,
disorder. Technology liberates us from linearity and at the same time
prevents us from falling into disorder. Today, the imaginary of the
network has become a universal master key. However, as obvious as it may
seem, the metaphor \emph{of the network} understood in technological
materiality and in its logic of operation is not the main point of
access to the textile imaginary of communication.

\hypertarget{digital-writing-and-textiles.}{%
\section{Digital, Writing and
Textiles}\label{digital-writing-and-textiles.}}

Interpreting the textile imaginary of communication requires, first,
thinking about the relationship between textiles and writing in order to
see the treatment that the West has given to fabric as something
traditional, not evolved or in any case as a representation of a past
within the evolution of human communication. Second, it seems to be
forgotten that the industrial revolution involved artisan workshops and
power looms. Looms that, on the other hand, not only bequeathed the
imagination of the binary with its punch card but also modified the
organization of the artisans and their perception of images in
competition with other visual technologies such as photography.

\hypertarget{writing-and-weaving-linearity-and-teleology}{%
\subsection{Writing and Weaving: Linearity and
Teleology}\label{writing-and-weaving-linearity-and-teleology}}

The history of communication seems to be structured from a teleology
centered on dominant technologies. It is told as the narrative of a
series of inventions and devices that culminates with the latest
technologies of the 21st century, in the context of the countries of the
North and whose main agents are companies, governments and, along with
them, some genius individuals and visionary professionals. In this
narrative there are no references to the South of the planet, to the
actors and social movements that only appear as the contents of
``foreign'' stories, as testimonies of ``the strange,'' of the poor and
primitive ``other'' who must be helped and given a voice. In any case,
as a place for technology transfer, a business opportunity, even as a
``development'' aid.

All of this permeates the imaginary according to which communication
evolves from primitive stages to contemporary technologies. On this
unquestioned point, primary orality (the one prior to the existence of
writing in Walter Ong\textquotesingle s sense\footnote{Walter Ong,
  \emph{Oralidad y escritura. Tecnologías de la palabra}, trad. Angélica
  Scherp (México: Fondo de Cultura Económica, 2011): 18.}) would be a
primitive mythopoetic stage superseded by writing, especially phonetics
(Greek). Orality would be the communication of traditional, mythical,
repetitive, and cyclical societies, while alphabetic writing would be
the communication of philosophy, science, linear progress.

It is not that there has not been a critique of this evolutionism that
leads from the oral to the written\footnote{Jacques Derrida, \emph{De la
  gramatología}, trad. Oscar del Barco and Conrado Ceretti (México:
  Fondo de Cultura Económica, 2017).} and from elementary techniques to
digital technologies. The problem is that it is still valid in
journalistic discourses and in public circuits, forming part of common
sense. This presupposes that communication appears as the product of an
evolution from the oral to alphabetic writing, that is, the ``complete''
writing insofar as it represents the spoken word (with the antecedents
of logo syllabic and syllabic writing). Before that, there would be
``non-writing'' such as the paintings in which primitive drawings
express signs, but not linguistic forms. These conceptions are still
applied today to the American populations descended from the
pre-Columbian peoples, who would be labeled as primitive societies,
``\emph{agraphic},'' because they did not have alphabetic writing as
European society did, but as Luis Ramiro Beltrán says, the ``majority of
the pre-Columbian native cultures were neither primitive nor
agraphic.''\footnote{Luis Ramiro Beltrán, introduction to \emph{La
  comunicación antes de Colón. Tipos y formas en Mesoamérica y los
  Andes}, ed. Luis R. Beltrán et al. (La Paz: CIBEC, 2008).}

Alphabetic writing became the sign and symbol of communication and
meaning, ignoring---or reducing to weak copies of---all other cultural
practices that express, construct, and communicate meaning. Thus, in the
pre-Columbian world there are different ``expressive features that
structure a symbolic manifestation based on the use and organization of
certain elements''\footnote{Beltrán et al., \emph{La comunicación}, 20.}
such as, for example, dance, music, poems, songs, hymns, squares,
buildings, temples, clothing, goldsmithing, drawings, among
others.\footnote{Beltrán et al., \emph{La comunicación}, 21.} Culture is
the production, circulation and consumption of meanings and human
communication is essentially symbolization and relation/interaction of
meanings between subjects. For this reason, communication and culture
cannot be separated as if they were ``distinct'' ``objects of study''
that were unrelated.\footnote{Héctor Schmucler, ``La investigación: un
  proyecto comunicación/cultura,'' in \emph{Memoria de la Comunicación}
  (Buenos Aires: Biblos, 1997).} A look at all those practices that have
enriched, for example, the legacy of the South American
peoples\footnote{Beltrán et al., \emph{La comunicación}, 47--73.}---which
are found today in almost all Latin American countries---are like those
that can be made of many other peoples. A view that questions the
historical presentation of communication\footnote{As is done, for
  example, in Francisco Sierra Caballero and Claudio Maldonado Rivera,
  coords., \emph{Comunicación, decolonialidad y Buen Vivir} (Quito:
  CIESPAL, 2016); Claudia Magallanes Blanco and José Manuel Ramos
  Rodríguez, coords., \emph{Miradas propias. Pueblos indígenas,
  comunicación y medios en la sociedad global} (Quito: CIESPAL, 2016),
  or the interventions of Alejandro Barranquero} and that does not allow it to be conceptualized in any
other way than that focused on technical means. Among all these very
different practices of communication/culture, textiles stand out.
Textile practice as a cultural practice has an alternative dimension to
communication centered on alphabetic writing. Rescuing the textile in
the substratum of the explanations of communication would lead to
reinterpreting the communicational phenomenon\marginnote{and Juan Ramos Martín,
  "Luis Ramiro Beltrán and Theorizing Horizontal and Decolonial
  Communication", \emph{in The Handbook of Global Interventions in
  Communication Theory}, ed. by Yoshitaka Miike and Jing Yin (New York:
  Routledge, 2022); Raúl Fuentes Navarro, "Latin American Interventions
  to the Practice and Theory of Communication and Social Development: On
  the Legacy of Juan Díaz-Bordenave," in Miike and Yin, The Handbook,
  and \emph{Eva González Tanco and Carlos Arcila Calderón, "Buen Vivir
  as a Critique of Communication for Development," in Miike and Yin,}
  The Handbook} from that other place so
present, for example, in the Andean textile\footnote{Beltrán et al.,
  \emph{La comunicación,} 225--51.} where the technical appears in
another way.

The communicational imaginary of orality and writing in its logocentrism
had its elective affinity with media centrism. The unconsciousness of
such a relationship became research on communication because of its
technical instruments. Photography, press, cinema, radio, then
television, computers, cell phones and the internet have made it
possible to imagine the centrality of an evolutionary conception of
technology in media logical reflection, giving way to deterministic and
teleological linear readings.

Beyond the evolution from the mythical to the logical or from
logocentrism to media centrism, it may also be that communication is
something else and that discovering it may depend on the consideration
of textiles as a weaving of technique and writing, because weaving
continues to speak of what communication is. Here we try to show that
textiles constitute an imaginary of communication that has remained
underlying, but alive and acting in the cultural substratum. A
naturalized knowledge that manifests itself in the cultural atmosphere
when talking about communication.\footnote{Daniel H. Cabrera Altieri,
  ``Exploraciones sobre el significado de la técnica y la escritura,''
  in \emph{Cosas confusas. Comprender las tecnologías y la
  comunicación}, ed. Daniel H. Cabrera Altieri (Valencia: Tirant lo
  Blanch, 2019), and Daniel H. Cabrera Altieri, ``Lo textil como vía
  para repensar la comunicación/tecnología,'' in Cabrera Altieri,
  \emph{Cosas confusas}.}

\hypertarget{capitalism-and-power-looms}{%
\subsection{Capitalism and Power
Looms}\label{capitalism-and-power-looms}}

The linearity of the interpretation of the ``media'' tends to leave
aside a technology fundamental to Western capitalism, such as automated
looms. Power looms were the instruments of the industrial revolution.
Joseph-Marie Jacquard (1752--1834) gave his name to the most famous loom
with which capitalism acquires its definitive profile. It was used ``for
fabrics with large patterns, in which all or most of the threads of the
drawing rise or fall independently of each other; in this way lines and
figures of all kinds can easily be reproduced in the
fabric.''\footnote{Thomas W. Fox, \emph{Maquinaria de tejidos}, trad.
  Francisco Madurga (Barcelona: Bosch, 1919): 440.}

Contrary to what its name may suggest, the automatic knitting machine
was the result of more than a century of various inventions and
constructions.

Contemporaneous with automatic knitting machines, Charles Babbage, an
English mathematician, designed a ``difference engine'' to construct
mathematical tables and in 1834 proposed an ``analytical engine'' to
perform a wide variety of numerical calculations. However, he wryly
noted that the machine ``would be capable of doing anything but
composing folk pieces.''\footnote{Huskey citado en Martin Davis,
  \emph{La computadora universal. De Leibniz a Turing}, trad. Ricardo
  García Pérez (Barcelona: Debate, 2002): 165.}Although the concept of
the machine was still far from a general-purpose computer like
today\textquotesingle s, it raised the idea of a mechanism in relation
to a logic of operation. On this point, Ada Lovelace stands out who, as
Babbage's assistant and enthusiast of his machine, was the one who
related the Analytical Engine to the Jacquard loom: ``we can say without
fear of being mistaken that the Analytical Engine weaves algebraic
models exactly as the Jacquard loom weaves flowers and
leaves.''\footnote{Goldstine, cited in Davis, \emph{La computadora},
  202.} Although it is also known that Babbage himself, who had a
Jacquard loom, set out to use punched cards like those of the automatic
loom in the analytical engine he had designed.

These cards, also used by pianolas and census machines at the beginning
of the twentieth century, became the first memory of the digital world.
The punch cards quickly proved themselves to be supports for a binary
language that needed to be stored. This first hard drive made the hole
"the other option" to the surface and space.

Although power looms and their influence are well-known in the history
of computing, the histories of communication techniques do not take them
into account. Among other reasons, because of the inability to see the
common imaginary that relates looms to the media, and fabrics to
communication.

The technical functioning of the loom and the human work of the textile
workshops allowed greater productivity in the automated industrial
processing of fabrics, but an important problem was the separation
between the loom builder and the weaver, because "the builder lacks
experience in the use of the loom and the weaver does not have
sufficient knowledge of mechanics to develop the ideas that occur to him
during practice.''\footnote{Fox, \emph{Maquinaria de tejidos}, 1.} In
the case of the weavers of Lyon, after a period of rejection, they
generated new modes of organization in the use of looms that led to an
increase in the human imagination and an effort to increase the
complexity of fabrics in their quest to make textiles compete with the
dominant media of the time (engraving, engraving, etching, etching,
etching, etch printing, painting) and with the new medium of
photography. The power loom allowed them a new form of digital imaging
based on textiles.\footnote{Ganaele Langlois, ``Distributed
  Intelligence: Silk Weaving and the Jacquard Mechanism,''
  \emph{Canadian Journal of Communication} 44, no. 4 (2019). See Sadie
  Plant, \emph{Ceros + unos. Mujeres digitales + la nueva cultura},
  trad. Eduardo Urios (Barcelona: Destino, 1998).}

Understanding the relationship between the power loom, i.e., the
apparatus with its complexity, the organization of artisans in its use,
the production of fabrics, even their commercialization and influence on
the economy, poses a problem common to the history of technologies and
the history of the media. The first is technological determinism which,
in the case of the media, is applied to the transformation of attention,
memory, knowledge, the formation of public opinion, changes in the
spatial-temporal scale. In these approaches, such as those of Marshall
McLuhan,\footnote{Marshall McLuhan, \emph{La galaxia Gutenberg. Génesis
  del ``Homo typographicus,''} trad. Juan Novella. (Barcelona: Círculo
  de Lectores, 1993).} the problem lies in the teleology of change that
concentrates on the means and technologies that triumph, subsuming the
explanation to an evolutionary linearity, a progressivism that focuses
on powerful media and the power relations in which they participate. The
history of the media is thus integrated into an apparently unconscious
movement that is moved by the force of its argumentation and its
efficiency and effectiveness. The case of the Jacquard loom as a complex
human-machine collaboration technology is an example of a digital media
system not yet dominant, which fosters immense creativity. As Ganaele
Langlois has analyzed, it can be an example, which leads us to think
about the opportunities to explore the potential of non-dominant media
systems that go unnoticed by media centrism.

\hypertarget{weaving-its-metaphors-and-imaginaries}{%
\section{Weaving, its Metaphors and
Imaginaries}\label{weaving-its-metaphors-and-imaginaries}}

Delving into the considerations of fabrics implies a forgetting of
textiles as communication, as the production of meaning. A kind of
amnesia that has instruments of cure when its omnipresence in the
vocabulary and expressions of everyday life is investigated, to make
transparent the ancient kinship between tissue and human communication.
It is about the possibility of focusing the reflection on communication
from subaltern practices, from crafts before technologies, from women
before men, from the South before in the Euro-American West, from the
word and the community before from technologies and companies.

\hypertarget{textile-amnesia}{%
\subsection{Textile
Amnesia?}\label{textile-amnesia}}

\begin{quote}
... The awakening of dormant memories caused by the analogy seems to be
so close to the essence of what it means to be human that it is
difficult to imagine what mental life would be like if it did not exist.

---Douglas Hofstadter and Emmanuel Sander, \emph{The Analogy. The engine
of thought}.
\end{quote}

Textiles are an activity and an imaginary that accompanies human beings
from the Neolithic period to the present day. A past that is present in
constant archaeological finds of weaving tools (spindles, looms) and,
above all, in the vocabulary of everyday life that relates various human
actions to the world of thread, weft, warp. Anthropologist Tim Ingold
defends the idea that spinning, braiding, and weaving are among the most
archaic human arts, stating that "the making and use of threads may be a
clear indication of the emergence of characteristically human life
forms.''\footnote{Tim Ingold, \emph{Líneas. Una breve historia}, trad.
  Carlos García Simón (Barcelona: Gedisa, 2015): 69. See Tim Ingold,
  ``The textility of making'', \emph{Cambridge Journal of Economics} 34,
  no. 1 (2010).}

Despite its anthropological importance, weaving enjoys a rare privilege,
the invisibility of its presence and the irrelevance of its
consideration. To the point that the central role of textiles in the
history of technology, commerce, and civilization proper has been
``overlooked.''\footnote{Virginia Postrel, \emph{El tejido de la
  civilización. Cómo los textiles dieron forma al mundo}, trad. Lorenzo
  Luengo (Madrid: Siruela, 2020): 12.} For this reason, Virginia Postrel
assures that today's society suffers from a ``textile
amnesia.''\footnote{Postrel, \emph{El tejido}, 286.} Given the abundance
of textiles, there is no awareness of how intertwined human life, in its
different dimensions, is with fabric.

Textile production is reflected in the oldest writing systems. Fabrics,
spindles, and looms, among other portraits from the textile world, are
represented in the cuneiform script of Mesopotamia, in the hieroglyphs
of Egypt and in the linear script of the Aegean, referring to a
conceptual universe close to the current one.\footnote{Agnès
  García-Ventura, ``Imágenes del universo textil en las primeras
  escrituras'', \emph{Datatèxtil}, n.º 14 (2006): 20-31,
  \url{https://raco.cat/index.php/Datatextil/article/view/278625}.}

Textiles are a phenomenon common to humanity, from intertwining and
simple knots of plant and animal threads to complex works in multiple
cultures and places around the world. The history of weaving and its
historical evolution is based on three facts: the discovery of
crossbreeding with flexible materials; its elaboration to obtain a
thread, and the necessary instruments to hold the threads with the
necessary tension.\footnote{Rita Barendse y Antonio Lobera, \emph{Manual
  de artesanía textil} (Barcelona: Alta Fulla, 1987): 9.}

The old European industrial manuals define weaving in relation to warp
and weft as ``interweaving a series of threads placed in the direction
of the length of the fabric with another series of threads placed
transversely in the direction of width.''\footnote{Fox, \emph{Maquinaria
  de tejidos,} 5.} And they divide the textile industry into three
parts: spinning (``operations necessary to obtain the fibers...''),
weaving (``manufacture of fabrics'') and dyeing and sizing (necessary
operations of beautifying and finishing fabrics).\footnote{Max Gürtler
  and W. Kind, \emph{La industria textil}, trad. Ricardo Ferrer
  (Barcelona: Labor, 1947): 17.}

From the Andean artisanal point of view, weaving has been defined as
``the interweaving of a system of threads called warp by a system of
threads called wefts, whose constant feature is the formation of the
paso or fretwork (space that is formed between the warp threads for the
passage of the weft). The other traits can always have an
exception.''\footnote{Clara M. Abal de Russo, \emph{Arte textil incaico
  en ofrendatorios de la alta cordillera andina}. \emph{Aconcagua,
  Llullaillaco, Chuscha} (Buenos Aires: CEPPA, 2010): 50.}

The interweaving and stages of textile activity come from
ancient\footnote{Véase Anni Albers, \emph{On Weaving} (Nueva Jersey:
  Princeton University Press, 2003).} times and remain alive in various
places, among which the Latin American Andean region stands out in a
special way, where weaving occupies a central place in many
communities.\footnote{Eva Fischer, \emph{Urdiendo el tejido social.
  Sociedad y producción textil en los Andes bolivianos}, trad. Eva
  Fischer (Berlín: Lit Verlag, 2008): 26.} Research in the Andean region
forces us to reconsider communication from a pre-Columbian anthropology
that is still alive in its miscegenation and hybridizations.

\hypertarget{the-language-of-weaving}{%
\subsection{The Language of
Weaving}\label{the-language-of-weaving}}

Contemporary languages refer to the ancient textile world in very
diverse contexts, although with special dedication to the expression and
understanding of human communication. Etymologies and phraseology
provide multiple examples.

Etymologies\footnote{The following resources have been used:
  \emph{Online Spanish etymological dictionary}, last modified on May
  18, 2022,} of Western languages\textsuperscript{76} recall the
intimate relationship between ``textile,'' ``text'' and ``technique.''
All derived from the Indo-European\marginnote{\url{http://etimologias.dechile.net/}; \emph{Online
  Etymology Dictionary}, accessed March 20, 2022,
  \url{https://www.etymonline.com/}; Julius Pokorny,
  \emph{Indogermanisches Etymologisches Wörterbuch}, database, accessed
  20 March 2022,
  \url{https://indo-european.info/pokorny-etymological-dictionary/}\href{https://indo-european.info/pokorny-etymological-dictionary/;https://www.perseus.tufts.edu/hopper/}{;
  Perseus} Digital Library
  (\href{https://indo-european.info/pokorny-etymological-dictionary/;https://www.perseus.tufts.edu/hopper/}{Perseus
  Hopper}), accessed March 20, 2022,
  \url{https://www.perseus.tufts.edu/hopper/}, and Edward A.
  Roberts\emph{, Indo-European Etymological Dictionary of the Spanish}
  Language (Madrid: Alianza, 1996).} root\marginnote{\textsuperscript{76}\setcounter{footnote}{76} Although
  the expressions referred to correspond to the Spanish language, there
  are equivalents in other contemporary Western languages.} \emph{teks- }(meaning weaving,
fabricating, assembling, carpentry) from which derives the Greek
\emph{tekton} (meaning of structure, construction, work and, in English,
\emph{tectonic, architect}) and \emph{téchne} (meaning technique, art
and, in English,\emph{ technique, technocrat, technology}) and through
the Latin \emph{tela} \emph{(arrives: fabric, loom, subtle); texere}
(whence weaving, braiding, interlacing), \emph{textus} (participle
fabric of \emph{texere}, whence text, pretext, hypertext).

There is also a kinship between linen and line. The Indo-European root
\emph{li-no-} through the Greek \emph{linon} and the Latin \emph{linum}
has given rise in Spanish, among others\emph{, lino, linea} and online.
And the presence of thread from the Indo-European root \emph{gwhi-}
meaning thread and filament and through the Latin \emph{filum}
(\emph{filo}, line of a contour, \emph{hilo}) passed into Spanish
\emph{filo} (and \emph{filamento, filar, filete, filigrana}), \emph{fila
(}and \emph{desfilar, enfilar),} hilo \emph{(hilar, hilvan), perfil} and
\emph{perfilar, vilo,} among others\emph{.}

Multiple expressions in Spanish, with their correlates in Romance
languages, refer to various dimensions of textiles, weaving, and
weaving. These include, for example:

\begin{itemize}
\item
  \emph{Trama} as artifice or conspiracy against someone; entanglement
  of a dramatic work or comedy, ``the plot of the story''
\item
  \emph{Urdir} such as, for example, to plot something against someone.
\item
  The use of thread in expressions such as ``\emph{tirar del hilo},''
  ``\emph{el hilo de la cuestión},'' ``\emph{no perder el hilo},''
  ``\emph{el hilo de la vida},'' ``\emph{no dar puntada sin hilo},'' in
  addition to ``\emph{el hilo},'' it is said, we can add lost, cut, take
  up again
\item
  Sayings referring to the knot such as \emph{``nudo del problema,''
  ``nudo gordiano}
\item
  Spinning in ``\emph{hilar fino}''
\item
  Weaving as ``weaving'' themes, questions, ideas, concepts; discussing,
  devising a plan
\item
  References to texture as the arrangement or structure of a work, a
  body
\item
  Needle, ``\emph{encontrar la aguja en el pajar}''
\item
  \emph{A la rueca,} like something that gets twisted.
\end{itemize}

This list of expressions and uses of language referring to the textile
world testifies to an imaginary of the meaning of life, of doing and
speaking, as well as the English expression \emph{spinning yarns} as
``to tell stories,'' and before that, rhapsody, as \emph{``to sew songs
and stories.''}

A recent dimension of textile research is to explore the role of textile
technology in the mental universes of the past, in worship, rituals,
mythology, metaphors, political rhetoric, poetry, and the language of
the sciences.\footnote{Salvatore Gaspa, Cécile Michel and Marie-Louise
  Nosch, eds., \emph{Textile Terminologies from the Orient to the
  Mediterranean and Europe, 1000 BC to 1000 AD} (Nebraska: Zea Books
  Lincoln, 2017).} Research of this kind concludes that metaphorical and
figurative textile expressions are not mere stylistic tools but are
rooted in cognitive, terminological and experiential realities of the
past and that persist to the present day in language.

Oswald Panagl believes that proof of this can be found in the English
vocabulary related to \emph{weaving, spinning, net} that is consolidated
in terms and technical expressions of the lexicon of electronic media
such as\emph{, for example}, web address, on the web, web based, web
browser, web \emph{designer, webcast, web forum, webhead, webmaster, web
page, web-site; spin doctor; network, Internet, net speak}.\footnote{Panagl
  Oswald, ``Der Text als Gewebe: Lexikalische Studien mi Sinnbezirk von
  Webstuhl und Kleid,'' in \emph{Textile Terminologies from the Orient
  to the Mediterranean and Europe, 1000 BC to 1000 AD}, ed. Salvatore
  Gaspa, Cécile Michel and Marie-Louise Nosch (Nebraska: Zea Books
  Lincoln, 2017): 419. Hay que señalar que desde el inglés pasó a las
  lenguas de un mundo ``globalizado''.} The researcher analyzes the
semantic field of weaving to argue that it has not become a dead
metaphor but has remained productive from ancient times to the present
day.

Following the expressions and words of the current vocabulary referring
to the textile world leads to making visible a silent presence as well
as active. Metaphors show the way to the imaginary of human
communication. An imaginary that has textuality as \emph{hummus}, the
intertwined creator of skilled hands, and that refers to life and its
care together with speaking, singing, texting, writing, relating to
others. Incredibly, those words that took shape and meaning from a set
of common practices in the past remain alive, but so hidden and
disconnected from their core of meaning to the point of not drawing
attention to their explanatory capacity or their heuristic force.

\hypertarget{the-greek-case-textile-imaginary-between-rhetoric-logos-and-myth}{%
\subsection{The Greek Case: Textile Imaginary
Between Rhetoric, Logos and
Myth}\label{the-greek-case-textile-imaginary-between-rhetoric-logos-and-myth}}

Ancient peoples, as noted, offer multiple experiences of weaving
practice. Here we will mention, due to their evident influence on
European culture and languages, the vocabulary and textile expressions
molded in ancient Greece and that have come down to us through myths,
philosophy, and art. The use of Greece is intended to draw attention to
a history and culture that has been explored until, in some sense, to
turn them into Western common sense. Reconsidering this sense helps us
to understand the silences and absences of which subaltern, popular and
feminine memory is made.

The references to the language of classical Greece,\footnote{Cornelius
  Castoriadis, ``Notas sobre algunos medios de la poesía'', en
  \emph{Figuras de lo pensable} (Valencia: Cátedra, 1999): 36-61. The
  author highlights the "indivisible polysemy of words and grammatical
  cases" (p. 36) of the classical Greek language, whereby it is
  sometimes impossible to translate a word with a single meaning.} its
myths and its philosophy were based on a context of daily life and
social organization where textiles occupied a fundamental place. Labor,
\footnote{Hannah Arendt, \emph{La condición humana}, trad. de Ramón Gil
  Novales (Barcelona: Paidós, 2009): 37-95.} as Hannah Arendt defines
it, is a condition of possibility of the \emph{polis}, of "action", but
which remains in the darkness of the \emph{oikos}, the place of
inequality and necessity governed by women. Within the \emph{oikos, the
gynoecium} was an architectural and symbolic space for weaving and
weaving. There, to the rhythm of a vertical loom whose warps resembled a
lyre, the skill of the weaver was expressed with both strength and
skill. While weaving, stories and myths were told, and betrayals and
values were learned, which founded the mentality of Greek society.

In that world, weaving, its making, its tools, its inputs, were a
reference for thinking about the meaning of human life. The weaving of
the threads and the rhythms of the loom created the fabrics with which
the clothes were made and the meanings of the stories and the human life
that was sheltered there were conceived. It was "\emph{a piece of cake}"
according to the famous popular expression that coincides with the idea
that the shuttle is "a friend of songs".\footnote{Diana Segarra Crespo,
  ``Coser y cantar: a propósito del tejido y la palabra en la cultura
  clásica'', en \emph{Tejer y vestir. De la Antigüedad al Islam}, ed. de
  Manuela Marín (Madrid: CSIC, 2001): 200.} Circe, the Moirae and the
Fates also spun and sang,\footnote{Ibid., 201.} because in classical
society the sacred model was not only textile but also linguistic. In
it, the young women narrated and fixed them on a textile support, which
was the feminine way.\footnote{Ibid., 217.}

The Greek textile imaginary has been analyzed in relation to
communication, \footnote{Daniel H. Cabrera Altieri, ``El imaginario
  textil griego y la comunicación'', \emph{RAE-IC, Revista de la
  Asociación Española de Investigación de la Comunicación} 1, n.º 2
  (2014): 65-73.} highlighting that rhetoric was the theory of
communication of the \emph{agora}, public and masculine, while the
weaving activity of the \emph{oikos,} domestic and feminine, only has
references in myths, they are "women\textquotesingle s stories". A
situation that would culminate in the Classical period in the opposition
between \emph{logos} and \emph{myths}, rational explanation and
demonstration as opposed to narration, storytelling, stories.\footnote{Hans
  Georg Gadamer, \emph{Mito y razón}, trad. de José Francisco Zúñiga
  García (Paidós: Barcelona, 1997): 25.}

Aristophanes in his comedy \emph{Lysistrata (411 B.C.) and Plato in} The
Politician \emph{(367-361 B.C.) are two cases where the feminine crosses
the border into politics.} \emph{Lysistrata}, a woman, takes the floor
to advise how to solve the problems of the lack of agreement of the
citizens, "the politicians", and all her recommendation consists of an
application of textile tasks.

\begin{quote}
LYSISTRATA: First of all, as is done with fleeces, it would be necessary
to detach from the city in a bath of water all the filth that it has
grasped, remove the knots and eliminate the wicked, by throwing them on
a bed of boards, and those who still stick to them and squeeze together
to get charges should be torn out with the carder and their heads cut
off; then to put the common good will in a basket, mixing all those who
have it, not excluding the Metecs and foreigners who love us, and mixing
there also those who are indebted to the public treasury, and also, by
Zeus, all the cities that have settlers from this land, understanding
that they are all to us like tufts of wool scattered on the ground, each
one in his own way. And then, taking a thread from all of them, gather
them together here and make a huge ball of them and weave from it a
cloak for the people.\footnote{Aristófanes, \emph{Lisístrata}, trad. de
  Luis M. Macía Aparicio (Madrid: Ediciones Clásicas, 1993), 575--85.}
\end{quote}

\enlargethispage{\baselineskip}

\noindent Cleaning, mixing, gathering, joining, knitting... To create the new,
that which protects: "If you had a shred of common sense, according to
our wool you would govern everything" without the need for wars or
sterile competitions between Greek cities.

Plato, in \emph{The Politician}, puts into the mouth of a foreigner a
reflection on protection and care from the semantic field of
\emph{tekton, téchne and} texere \emph{that} lead him to conclude that
"we give the name of dresses to these defenses and clothes that are made
by interweaving their own threads; and just as we then called the art of
caring for the \textquotesingle polis\textquotesingle{}
\textquotesingle politics.\textquotesingle" Policy is presented as care,
protection, covering/coverage, art/technique.\footnote{Dimitri El Murr,
  ``La Symploke Politike: Le paradigma du tissage dans le Politique de
  Platon, ou les raisons d'un paradigma Arbitraire'', \emph{Kairos}, n.º
  19 (2002): 49-95. See Cornelius Castoriadis, \emph{Sobre} El Político
  \emph{de Platón} (México: Fondo de Cultura Económica, 2002): 121-41.}
In classical Greece, weaving could be postulated as a model of political
organization, but in a comedy or in the mouth of a foreigner and, in
both cases, it was a way of sustaining something that seems improper.

In European culture, the association of weaving with women has remained
to this day, although the reality is more complex. Thomas Blisniewski,
speaking of weaving as it is represented in European art and analyzed
from an evolutionary point of view\footnote{Thomas Blisniewski,
  \emph{Las mujeres que no pierden el hilo. Retratos de mujeres que
  hilan, tejen y cosen de Rubens a Hopper} (Madrid: Maeva, 2009):
  125-48.}, notes the Jewish and Greco-Roman legacy which, through the
Christian tradition, has led Western Europe to consider that a virtuous
woman is intimately related to textile work. Western artistic
representation repeatedly emphasizes the traditional association of
women with the arts of weaving, suggesting passivity and female
dependence. However, weaving could also be considered, as has been done,
not only as a "symbol of domestic submission" but also "as a productive
industry" and perhaps, as a result, "a sign of her feminine
virtue".\footnote{Postrel, \emph{El tejido}, 59.}

It is curious that, historically, the feminine seems to find subjection
in the textile, while the textile is not necessarily resolved in the
world of women. Moreover, the industrial revolution shows how the
conversion into industry seems to liberate textiles from the feminine
world to be transformed into patriarchal technology and the productive
economy of capitalism. This is not the case in Latin America, where
textiles, the heart of culture and meaning, continue to participate in
what can be understood as communication in its relationship with others,
with the land, with tradition and religiosity.\footnote{Véase Adalid
  Contreras Baspineiro, ``Aruskipasipxañanakasakipunirakispawa'', en
  Sierra Caballero y Maldonado Rivera, \emph{Comunicación}, 59-93.}

\hypertarget{for-an-alternative-theory-of-communication}{%
\section{For an Alternative Theory of
Communication}\label{for-an-alternative-theory-of-communication}}

What has been said so far justifies postulating an alternative theory of
the communicative phenomenon centered on the metaphor of the textile and
the imaginary of the fabric as alter/native to media centrism. In its
search, three aspects can be taken into account for a conceptualization
of communication: as a creative textile intertwining, as care for life
and as the weaving of the social fabric.

\hypertarget{communication-crisscross-textile-creator}{%
\subsection{Communication: Crisscross Textile
Creator}\label{communication-crisscross-textile-creator}}

The anthropologist Cliffort Geertz affirms, following Max Weber, that
"man is an animal inserted in webs of meaning that he himself has woven"
and continues "I consider that culture is that warp and that the
analysis of culture must therefore be not an experimental science in
search of laws, but an interpretative science in search of
meanings".\footnote{Clifford Geertz, \emph{La interpretación de la
  cultura,} trad. de Alberto L. Bixio (Barcelona: Gedisa, 2003): 20.}

Following these guidelines, John B. Thompson in his study on Modernity
and the media states that "the media constitute the spinning wheels of
the modern world and, by using these media, human beings become
fabricators of meanings for their own consumption".\footnote{John B.
  Thompson, \emph{Los media y la modernidad: una teoría de los medios de
  comunicación} (Barcelona: Paidós, 1998): 26.}

These authors use the textile metaphor as a narrative device, but it can
be used as a heuristic instrument. Vilém Flusser, briefly, makes this
interpretation by emphasizing the artificial character of communication
whose aim is "to make us forget the meaningless context in which we find
ourselves completely alone and incommunicado" and continues:

\begin{quote}
Human communication weaves a veil of the coded world, a veil of art and
science, of philosophy and religion around us and weaves it ever more
densely, so that we forget our own loneliness and our death, and the
death of those we love. Communication theory deals with the artificial
fabric that makes us forget about loneliness.\footnote{John B. Thompson,
  \emph{Los media y la modernidad: una teoría de los medios de
  comunicación} (Barcelona: Paidós, 1998): 26.}
\end{quote}

In several cultures, weaving is metaphorized to understand life and its
meaning. The laborious work of making threads, from plants, animals, or
insects, and then weaving them on the loom to produce fabric. The human
being emerges from the womb joined by a cord and, in the absence of
hair, feathers or thick skin, must be sheltered and warmed. In South
America, tissue is considered an extension of the
mother\textquotesingle s womb\footnote{See Fischer, \emph{Urdiendo el
  tejido social,} 250.}. Culture with its symbols, myths, narratives,
art, images, texts, among others, is constructed and functions as the
mother\textquotesingle s womb created by the fabric of communication.

Throughout the lives of individuals and the history of society, networks
of meanings are built that give meaning to the most diverse questions,
starting with their sensations, desires, fears, concerns and hopes, as
well as responses to threats, opportunities, competitions or to justify
actions. Culture is the fabric produced to shelter and protect oneself
from the elements of meaninglessness, while communication is the action
by which it is woven. Generative action, "perpetual
intertwining,"\footnote{Roland Barthes, \emph{El placer del texto},
  trad. de Nicolás Rosa (Buenos Aires: Siglo XXI, 1974): 81.} an
ever-present movement, whether latent or visible. The central issue of
communication is the production of webs of meaning, conscious or
unconscious, implicit, or explicit. Communicating protects, gives
ontological security, contact, shelter, skin, warmth. And how do weavers
do it? They speak of weaving as a meditative act in which "the mind
enters a state that can be described as
\textquotesingle receptive.\textquotesingle{} You hear much better when
you\textquotesingle re weaving. You don\textquotesingle t question, you
don\textquotesingle t fight, you don\textquotesingle t impede.
There\textquotesingle s nothing like listening to music and weaving, or
talking and weaving, or watching a film and weaving."\footnote{Annuska
  Angulo y Miriam Mabel Martinez, \emph{El mensaje está en el tejido}
  (México: Futura Textos, 2016): 17.}

That is why they defend knitting as a thought process and not just as a
manual skill: "weaving challenges the mind".\footnote{Postrel, \emph{El
  tejido}, 92.} Communication as listening, reception of the other,
recognition and dialogue. Reception as a feminine imaginary, openness to
the other, the ability to allow oneself to be fertilized and to embrace
the other. Here "reception" is something else, it is preparing the mind
and body for what happens. Communication as preparation for the ordinary
event, for the passing of time, for the contemplation of life. However,
weaving is also what you do while doing other things. In a sense,
weaving is distracting, it helps to look in a different way, to
concentrate listening on other levels of harmony and to a relaxed
dialogue. The body entertained through the hands produces the stillness
with which another level of existence begins and develops---to use the
metaphor of video games.

\hypertarget{communication-weaving-as-care-for-life}{%
\subsection{Communication: Weaving as Care for
Life}\label{communication-weaving-as-care-for-life}}

"The thread of life" is an ancient expression and is present in very
different cultures. The thread of the Greek Moirae and the Latin Fates,
spinners of life, is present in other Indo-European cultures such as the
Hittite, the Ancient Icelandic, the Baltic, the Slavic and the Albanian,
as well as in Indian and Iranian culture\footnote{Miguel Ángel Andrés
  Toledo, \emph{El hilo de la vida y el lazo de la muerte en la
  tradición indoirania} (Valencia: Intitució Alfons el Magnànim, 2010).},
in the Inca\footnote{Shyntia Verónica Castañeda Yapura, Renato Cáceres
  Sáenz y David Peña Soria, \emph{Tejiendo la vida. Los textiles en
  Q'ero} (Lima: Ministerio de Cultura, 2018).} and Mayan
cultures.\footnote{Manuel Alberto Morales Damián, ``La tejedora, la
  muerte y la vida. Simbolismo maya del trabajo textil en el Códice
  Tro-Cortesia'', \emph{Datatèxtil}, n.º 24 (2011): 76-83,
  https://raco.cat/index.php/Datatextil/article/view/275363.}That thread
of life is the thread of meaning, the thread of fate. The thread that is
created by twisting is extended, held, and cut. Like luck, like destiny,
life hangs in the balance.

In the Latin American Andean communities, it is understood that textiles
are like the body and like life, the loom as a mother and the fabric as
a child who, like a human being, grows.\footnote{Denise Y Arnold, Juan
  de Dios Yapita y Elvira Espejo Ayca, \emph{Hilos sueltos: los Andes
  desde el textil} (La Paz, Bolivia: ILCA, 2016): 102-23; véase también
  Luis Ramiro Beltrán et al., \emph{La comunicación}, 225-51.} The
tissue is thought to braid the thread of life from the center, the
navel, circulating the flow of bodily and spiritual energy from one
place of strength to another.\footnote{Arnold, Yapita y Espejo Ayca,
  \emph{Hilos sueltos}, 55.} To communicate is to protect life, the
body, the tissue, and the fabric, the skin, to which Marshall McLuhan
and\footnote{Marshall McLuhan, \emph{Comprender los medios de
  comunicación}, trad. de Patrick Ducher (Madrid: Paidós, 1996): 136ss.}
Andrea Saltzman\textsuperscript{104} refer
from very different angles.

The \emph{gender issue} is very present. The reality is that weaving is
not a task only for women, neither in history nor today, but there is
something, as Jacques Lacan said, that "woman is primarily a
weaver".\textsuperscript{105}
In his important book on weaving, Postrel argues: "The history of
textiles is not a male or female story, nor a European, African, Asian,
or American story. It is all of that at the same time, something
cumulative\marginnote{\textsuperscript{104} Andrea Saltzman, \emph{La metáfora de la piel.
  Sobre el diseño de la vestimenta} (Buenos Aires: Paidós, 2019).} and\marginnote{\textsuperscript{105}\setcounter{footnote}{105} Jaques Lacan, \emph{El Seminario.} Libro 10. \emph{La
  angustia}, trad. de Enric Berenguer (Buenos Aires: Paidós, 2018): 221.} shared: a human story, a tapestry woven with countless
vivid threads."\footnote{Postrel, \emph{El tejido}, 287.}

The author has a romantic and "cumulative" view of history. It is true
that in ancient times men wove, the industrial textile revolution
employed men and women, in war they have all woven clothes, to mention a
few. However, that can\textquotesingle t obscure what weaving and
weaving show. All peoples have had and still have some kind of textile
interweaving activity; however, the history of weaving also shows the
differences of gender, social classes, colonial strategies, and there
are testimonies, in writing and art, of divergences, imbalances,
discrepancies and even oppositions and simple domination. The tapestry
that makes up the history of weaving has tatters, tears, invisible
threads, and threads that cover and conceal, taut threads and others
superficial. The harmony and beauty of humanity\textquotesingle s fabric
also holds injustice, pain and suffering, but if human beings continue
to weave, it may be because we continue to seek beauty and hope.
Communication, like weaving, has a history that runs parallel to the
history of human beings. Communication thus appears as something more
and different from coordination and interaction. It is, above all, a
recognition of otherness and a search for meaning. And, as such, it is
the vital and intertwined matrix that creates and protects life.
Communication humanizes, makes humans in a human world.

Weaving has a relationship with life and with the female body that goes
far beyond the survey of who does the activity and where. The fabric of
the common, the care of others, comes from a fabric that "prolongs the
body and also the care, the bond and the bond."\footnote{Mónica Nepote,
  prólogo ``Tejer las redes del cuidado'' a \emph{El mensaje está en el
  tejido}, de Annuska Angulo y Miriam Mabel Martinez (México: Futura
  Textos, 2016): 7.} The fact that after the Second World War women
stopped weaving and sewing in the domestic space, a space essentially
unrecognized, caused society to surrender to the market of disposable
fashion. For some years now, women from many parts of the world have set
up workshops in weaving and sewing, in the creation of social fabric in
poor neighborhoods or marginalized communities, and in the (self-)repair
of identities and social ties.\footnote{There are multiple references
  that can be found under various names in Internet search engines. The
  most numerous social networks of knitters is
  \url{https://www.ravelry.com/}.}

\hypertarget{communication-weaving-the-social-fabric}{%
\subsection{Communication: Weaving the Social
Fabric}\label{communication-weaving-the-social-fabric}}

The textuality of communication confronts the instrumental and
mediacentric vision with an ontological strategy of production of life
in relation to others (ancestors, contemporaries and descendants), with
geography (mother earth and all living beings) and with the beliefs and
culture of the community. This is evident in the textile activisms and
struggles that travel through the cities and countryside throughout Meso
and South America with countless community and feminist experiences.
Just to mention a few examples. The Chilean arpilleras since the
seventies of the last century that stood up to the dictatorship of
Augusto Pinochet. Chilean women, mothers, wives, daughters and
themselves politically persecuted by the repressive regime and who found
and updated an ancient Mapuche practice as a mode of existence and
resistance.\footnote{See Gaby Franger, \emph{Arpilleras: cuadros que
  hablan vida cotidiana y organización de mujeres} (Lima: Movimiento
  Manuela Ramos, 1988); Marjorie Agosín, \emph{Tapestries of Hope,
  Threads of Love-the Arpillera Movement in Chile 1974-1994} (Nuevo
  México: University of New Mexico Press, 1996).} In Peru, the Quipu
Project, with the help of digital technologies, has been established as
an organization that fights for memory and justice against the forced
sterilization carried out in rural and indigenous communities during the
government of Alberto Fujimori.\footnote{Proyecto Quipu, acceso el 20 de
  marzo de 2022,
  \url{https://interactive.quipu-project.com/\#/es/quipu/intro}} In
Colombia, textile activism multiplies throughout the territory, building
collectivity, promoting social causes and channeling complaints or
protests.\footnote{Véase Eliana Sánchez-Aldana, Tania Pérez-Bustos y
  Alexandra Chocontá-Piraquive, ``¿Qué son los activismos textiles?: una
  mirada desde los estudios feministas a catorce casos bogotanos'',
  \emph{Athenea Digital} 19, n.º 3, (noviembre 2019), e2407,
  \url{https://doi.org/10.5565/rev/athenea.2407}.} Something similar is
happening in Panama and Guatemala, with the particularity of a search
for \emph{sui generis legal protection} of communal intellectual
property.\footnote{See Gemma Celigueta Comerma y Mónica Martínez Mauri,
  ``¿Diseños mediáticos? Investigar sobre activismo indígena en Panamá,
  Guatemala y el espacio Web 2.0'', \emph{Revista Española de
  Antropología Americana} 50 (2020): 241-52.} The cases are innumerable
and in all countries, these few examples show the liveliness of the
Latin American textile activism movements that are related to those that
have been developing in the cities of Europe or the United States for a
long time, within feminist movements such as guerrilla warfare or street
art called \emph{yarn} \emph{bombing, yarn storming,} \emph{Guerrilla
crochet}, \emph{graffiti crochet}, among other names for subversive
stitching and textile rebellion.\footnote{See Samantha Close, ``Knitting
  Activism, Knitting Gender, Knitting Race'', \emph{International
  Journal of Communication} 12 (2018): 867--89.}

The different experiences of textile activism show the need to
reconsider what a communication practice is. The "models" that
communities resort to in order to think about them and the
anthropological resources of significant interrelation are not the
"media" or their technologies, but ancestral practices that are alive in
families and communities. Among them, the use of "textile" names and
practices that articulate many popular, indigenous, and peasant social
struggles stands out, with special strength in Latin America, and with
testimonies in some cities of the Global North. In all of them, the use
of the textile imaginary does not accentuate the instruments of
communication to promote and publicize the demands for justice and
equality, but rather the communal action of braiding interests and
fighting for recognition through sitting together. It emphasizes
braiding, dialoguing, and producing fabrics without social, gender, or
racial differences. Militancies generate a protective, active, vital
network that is reinforced in the collaborative tying of companionship,
sisterhood, and community.

The invisibility of textiles reduced to a subaltern practice, mostly
feminine, popular, poor, "village like" to "poor" peoples, manifests a
difficulty in communication research. The difficulty of focusing on
phenomena outside or on the margins of the dominant digital technologies
of business. In this sense, there are possibilities for an alternative
theory, but on the condition that communication studies and history are
decolonized in order to discuss the dominant techno centrism and its
teleology. Perhaps in this way, communicational blindness to textiles
could be confronted and new genealogies could be reconstructed
considering historical textile practices and those of different
communities in their own contexts. Anthropology and the history of
weaving have already done part of the work, now it\textquotesingle s
time to listen to it from the popular communication, of Buen Vivir, and
decolonial of the South.

\hypertarget{conclusion-the-subterranean-centrality-of-textiles}{%
\section{Conclusion: The Subterranean Centrality of
Textiles}\label{conclusion-the-subterranean-centrality-of-textiles}}

Faced with the question of why textiles as an interpretation of
communication, the answer could be provisionally and from the theory of
social imaginaries:

\begin{itemize}
\item
  Because references to communication and meaning supported by the
  vocabulary and metaphor of textiles are omnipresent in Western
  languages. There is something in this ubiquitous concurrence, an
  evocation of an ancestral imaginary that is taken for granted, a
  common sense of the symbolic universe of society.
\item
  Because the frequency of the textile metaphorical cognitive resource
  in communication theories is also multiple, but it is rarely made
  explicit. Similar to what happens in the language of everyday life,
  metaphor is used as a narrative device without further ado.
\item
  Because when delving into the oblivion and absence of textiles, one
  can trace a negation that goes hand in hand with the concealment of
  the feminine, the everyday, the domestic, the subaltern, the popular,
  the artisanal.
\item
  Because when studying the textile imaginary, an understanding of the
  phenomenon of communication emerges that opens up new horizons of
  questioning.
\end{itemize}

And these answers lead to the need to postulate an alter/native theory
of communication from a place different from traditional mediacentric
communicology. To study communication by fighting against the various
forms of denial of the existence of what is different from Western
Euro-American culture.

Boaventura de Sousa Santos argues that there are "five main social forms
of non-existence produced or legitimized by dominant Eurocentric reason:
the ignorant, the residual, the inferior, the local or particular, and
the unproductive."\footnote{Boaventura de Sousa Santos,
  \emph{Descolonizar el saber, reinventar el poder} (Montevideo: Trilce
  / Universidad de la República, 2010): 22.} The \emph{ignoramivity}
produced by logics of knowledge that privilege positive science and the
aesthetics of European high culture with its canons of truth and beauty.
The \emph{backwardness is} the result of the logic of linear time with
its ideas of progress, revolution, development and growth. The
\emph{inferior} is the product of a logic of social classification,
particularly racial and sexual, and the differences and hierarchies that
arise from it. A logic of dominant scale that privileges the universal
and the global and leaves aside the \emph{particular} and the
\emph{local}. And finally, the \emph{unproductive}---sterile or
laziness, for example---a consequence of a capitalist productivist logic
applied to nature and work. The understanding of communication from the
textile imaginary fulfills all these characteristics and forces us to
confront socially significant and alternative communication practices to
the "media".

This understanding of communication must be oriented, then, towards a
communicology from the South, intercultural, decolonial and subaltern,
which implies an epistemological consideration of communication and its
study, communicology. The textile imaginary deserves to be deepened in
order to rethink the appropriations of technologies from the practices
and tactics of local, community and popular communication. In this way,
what is analyzed in this article (evolutionary linearity, importance of
orality, references to traditional textile practices, the relevance of
current social and political manifestations that take textiles as a
reference) meets the testimony of the reality of Latin American
communities. This is where the study of communication can find new paths
on the margins of Western modernity.




\section{Bibliography}\label{bibliography}

\begin{hangparas}{.25in}{1} 



Abal de Russo, Clara M. \emph{Arte textil. Incaico en ofrendatorios de
la alta cordillera andina. Aconcagua, Llullaillaco, Chuscha}. Buenos
Aires: CEPPA, 2010.

Agosín, Marjorie. \emph{Tapestries of Hope, Threads of Love-the
Arpillera Movement in Chile 1974--1994}. Albuquerque, NM: University of
New Mexico Press, 1996.

Albers, Anni. \emph{On Weaving}. Princeton, NJ: Princeton University
Press, 2003.

Andrés Toledo, Miguel Ángel. \emph{El hilo de la vida y el lazo de la
muerte en la tradición indoirania}. Valencia: Intitució Alfons el
Magnànim, 2010.

Angulo, Annuska, y Miriam Mabel Martinez. \emph{El mensaje está en el
tejido}. México: Futura Textos, 2016.

Arendt, Hannah. \emph{La condición humana}. Trad. de Ramón Gil Novales.
Barcelona: Paidós, 2009.

Aristófanes. \emph{Lisístrata}. Trad. de Luis M. Macía Aparicio. Madrid:
Ediciones Clásicas, 1993.

Arnold, Denise Y., Juan de Dios Yapita, y Elvira Espejo Ayca.
\emph{Hilos sueltos: los Andes desde el textil}. La Paz, Bolivia: ILCA,
2016.

Barendse, Rita, y Antonio Lobera. \emph{Manual de artesanía textil}.
Barcelona: Alta Fulla, 1987.

Barranquero, Alejandro y Juan Ramos Martín. ``Luis Ramiro Beltrán and
Theorizing Horizontal and Decolonial Communication.'' In Yoshitaka Miike
and Jing Yin, eds. \emph{The Handbook of Global Interventions in
Communication Theory}. New York: Routledge, 2022.

Barthes, Roland. \emph{El placer del texto}. Trad. de Nicolás Rosa.
Buenos Aires: Siglo XXI, 1974.

Bateson, Gregory. ``Una teoría del juego y de la fantasía''. En
\emph{Pasos hacia una ecología de la mente}. Trad. de Ramón Alcalde.
Buenos Aires: Editorial Lohlé-Lumen, 1991..

Beltrán, Luis Ramiro et al. \emph{La comunicación antes de Colón. Tipos
y formas en Mesoamérica y los Andes}. La Paz: CIBEC, 2008.

Blisniewski, Thomas. \emph{Las mujeres que no pierden el hilo. Retratos
de mujeres que hilan, tejen y cosen de Rubens a Hopper}. Madrid: Maeva,
2009.

Cabrera Altieri, Daniel H., coord. \emph{Cosas confusas. Comprender las
tecnologías y la comunicación}. Valencia: Tirant lo Blanch, 2019.

Cabrera Altieri, Daniel H. ``El imaginario textil griego y la
comunicación''. \emph{RAE-IC, Revista de la Asociación Española de
Investigación de la Comunicación} 1, n.° 2 (2014): 65--73.

Cabrera Altieri, Daniel H. ``Exploraciones sobre el significado de la
técnica y la escritura''. En \emph{Cosas confusas. Comprender las
tecnologías y la comunicación}, editado por Cabrera Altieri, 13--29.
Valencia: Tirant lo Blanch, 2019

Cabrera Altieri, Daniel H. \emph{Lo tecnológico y lo imaginario. Las
nuevas tecnologías como creencias y esperanzas colectivas}. Buenos
Aires: Biblos, 2006.

Cabrera Altieri, Daniel H. ``Lo textil como vía para repensar la
comunicación/tecnología''. En \emph{Cosas confusas. Comprender las
tecnologías y la comunicación}, editado por Cabrera Altieri, 35--48.
Valencia: Tirant lo Blanch, 2019.

Cabrera Altieri, Daniel H. \emph{Tecnología como ensoñación. Ensayos
sobre el imaginario tecnocomunicacional}. Temuco: Ediciones Universidad
de la Frontera, 2022.
\url{http://bibliotecadigital.ufro.cl/?a=view\&item=1962}.

Carey, James W. ``A Cultural Approach to Communication.'' In
\emph{Communication as Culture: Essays on Media and Society, }13--36.
Boston: Unwin Hyman, 1989.

Castañeda Yapura, Shyntia Verónica, Renato Cáceres Sáenz y David Peña
Soria. \emph{Tejiendo la vida. Los textiles en Q'ero}. Lima: Ministerio
de Cultura, 2018.

Castells, Manuel. \emph{La sociedad de la información}. Vol. 1, \emph{La
sociedad red}. Trad. de Carmen Martínez Gimeno y Jesús Alborés. Madrid:
Alianza, 2000.

Castoriadis, Cornelius. \emph{La institución imaginaria de la sociedad}.
Trad. de Antoni Vicens y Marco Aurelio Galmarini. Barcelona: Tusquets,
1993.

Castoriadis, Cornelius. ``Notas sobre algunos medios de la poesía''. En
\emph{Figuras de lo pensable}, 36--61. Valencia: Cátedra, 1999.

Castoriadis, Cornelius. \emph{Sobre} El Político \emph{de Platón}.
México: Fondo de Cultura Económica, 2002.

Celigueta Comerma, Gemma, y Mónica Martínez Mauri. ``¿Diseños
mediáticos? Investigar sobre activismo indígena en Panamá, Guatemala y
el espacio Web 2.0''. \emph{Revista Española de Antropología Americana},
n.° 50 (2020): 241--52.

Close, Samantha. ``Knitting Activism, Knitting Gender, Knitting Race.''
\emph{International Journal of Communication} 12 (2018): 867--89.

Contreras Baspineiro, Adalid. ``Aruskipasipxañanakasakipunirakispawa''.
En \emph{Comunicación, decolonialidad y Buen Vivir}, editado por
Francisco Sierra Caballero y Claudio Maldonado Rivera, 59--93. Quito:
CIESPAL, 2016.

Craig, Robert T. ``Communication Theory as a Field.''
\emph{Communication Theory} 9, no. 2 (1999): 119--61.

Dader, José Luis. ``La evolución de las investigaciones sobre la
influencia de los medios y su primera etapa: Teorías del impacto
directo.'' En \emph{Opinión pública y comunicación política}, editado
por de Alonso Muñoz et al. Madrid: Eudema, 1990.

Davis, Martin. \emph{La computadora universal. De Leibniz a Turing}.
Trad. de Ricardo García Pérez. Barcelona: Debate, 2002.

De Bustos, Eduardo. \emph{La metáfora. Ensayos transdisciplinares}.
Madrid: Fondo de Cultura Económica / UNED, 2000.

De Sousa Santos, Boaventura. \emph{Descolonizar el saber, reinventar el
poder}. Montevideo: Trilce / Universidad de la República, 2010.

Derrida, Jacques. \emph{De la gramatología}. Trad. de Oscar del Barco y
Conrado Ceretti. México: Fondo de Cultura Económica, 2017.

Durand, Gilbert. \emph{Las estructuras antropológicas del imaginario}.
Trad. de Víctor Goldstein. México: Fondo de Cultura Económica, 2004.

El Murr, Dimitri. ``La Symploke Politike: Le paradigma du tissage dans
le Politique de Platon, ou les raisons d\textquotesingle un paradigma
Arbitraire''. \emph{Kairos}, n.º 19 (2002): 49--95.

Fischer, Eva. \emph{Urdiendo el tejido social. Sociedad y producción
textil en los Andes bolivianos}. Trad. de Eva Fischer. Berlín: Lit
Verlag, 2008.

Flichy, Patrice. \emph{Lo imaginario de internet}. Trad. de Félix de la
Fuente y Mireia de la Fuente Rocafort. Madrid: Tecnos, 2003.

Flórez Miguel, Cirilo. ``Retórica, metáfora y concepto en Nietzsche''.
\emph{Estudios Nietzsche}, n.° 4 (2004): 51--67.

Flusser, Vilém. ``¿Qué es la comunicación?''. Trad. de Victor Silva
Echeto. En \emph{Cosas confusas. Comprender las tecnologías y la
comunicación}, editado por Cabrera Altieri. Valencia: Tirant lo Blanch,
2019.

Fox, Thomas W. \emph{Maquinaria de tejidos}. Trad. de Francisco Madurga.
Barcelona: Bosch, 1919.

Franger, Gaby. \emph{Arpilleras: cuadros que hablan vida cotidiana y
organización de mujeres}. Lima: Movimiento Manuela Ramos, 1988.

Fuentes Navarro, Raúl. ``Cuatro décadas de internacionalización
académica en el campo de estudios de la comunicación en América
Latina''. \emph{Anuario Electrónico de Estudios en Comunicación Social,
Disertaciones} 9, n.° 2 (2016): 8--26.
\url{https://doi.org/10.12804/disertaciones.09.02.2016.01}

Fuentes Navarro, Raúl. ``Latin American Interventions to the Practice
and Theory of Communication and Social Development: On the Legacy of
Juan Díaz-Bordenave.'' In Yoshitaka Miike and Jing Yin, eds. \emph{The
Handbook of Global Interventions in Communication Theory}. New York:
Routledge, 2022.

Gadamer, Hans Georg. \emph{Mito y razón}. Trad. de José Francisco Zúñiga
García. Paidós: Barcelona, 1997.

García-Ventura, Agnès. ``Imágenes del universo textil en las primeras
escrituras''. \emph{Datatèxtil}, n.° 14 (2006): 20--31.
\url{https://raco.cat/index.php/Datatextil/article/view/278625}.

Gaspa, Salvatore, Cécile Michel, and Marie-Louise Nosch, eds.
\emph{Textile Terminologies from the Orient to the Mediterranean and
Europe, 1000 BC to 1000 AD}. Nebraska: Zea Books Lincoln, 2017.

Geertz, Clifford. \emph{La interpretación de la cultura}. Trad. de
Alberto L. Bixio. Barcelona: Gedisa, 2003.

Gleick, James. \emph{La información. Historia y realidad}. Trad. de Juan
Rabasseda y Teófilo de Lozoya. Barcelona: Crítica, 2011.

González Tanco, Eva, and Carlos Arcila Calderón. ``Buen Vivir as a
Critique of Communication for Development.'' In Yoshitaka Miike and Jing
Yin, eds. \emph{The Handbook of Global Interventions in Communication
Theory}. New York: Routledge, 2022.

Gürtler, Max y W. Kind. \emph{La industria textil}. Trad. de Ricardo
Ferrer. Barcelona: Labor, 1947.

Herkman, Juha. ``Current Trends in Media Research.'' \emph{Nordicom
Review} 29, no. 1 (2008): 145--59.

Hofstadter, Douglas y Emmanuel Sander. \emph{La analogía. El motor del
pensamiento}. Trad. de Roberto Musa Giuliano. Barcelona: Tusquets, 2018.

Ingold, Tim. \emph{Líneas.} \emph{Una breve historia}. Trad. de Carlos
García Simón. Barcelona: Gedisa, 2015.

Ingold, Tim. ``The Textility of Making.'' \emph{Cambridge Journal of
Economics} 34, no. 1 (2010): 91--102.
\url{https://doi.org/10.1093/cje/bep042}.

Innerarity, Daniel. ``La seducción del lenguaje. Nietzsche y la
metáfora''. \emph{Contrastes: revista internacional de filosofía}, n.º 3
(1998): 123--45.

Jäkel, Olaf, Martin Döring, and Anke Beger. ``Science and metaphor: a
truly interdisciplinary perspective. The third international
metaphorik.de workshop.'' \emph{Metaphorik.de -- online journal on
metaphor and metonymy}, no. 26 (2016). \href{http://www.metaphorik.de/fr/journal/26/science-and-metaphor-truly-interdisciplinary-perspective-third-international-metaphorikde-workshop.html}{http://www.metaphorik.de/fr/}\\\hspace{.25in}\href{http://www.metaphorik.de/fr/journal/26/science-and-metaphor-truly-interdisciplinary-perspective-third-international-metaphorikde-workshop.html}{journal/26/science-and-metaphor-truly-interdisciplinary-perspective}\\\hspace{.25in}\href{http://www.metaphorik.de/fr/journal/26/science-and-metaphor-truly-interdisciplinary-perspective-third-international-metaphorikde-workshop.html}{-third-international-metaphorikde-workshop.html}.

Jastrow, Robert. \emph{El telar mágico. El cerebro humano y el
ordenador}. Trad. de Domingo Santos. Barcelona: Salvat, 1993.

Karam, Tanius. ``Tensiones para un giro decolonial en el pensamiento
comunicológico. Abriendo la discusión''. \emph{Chasqui. Revista
Latinoamericana de Comunicación}, n.º 133 (dic. 2016--mar. 2017):
247--64.

Katz, Elihu. ``The Two-Step Flow of Communication: An Up-To-Date Report
on an Hypothesis.'' \emph{Political Opinion Quarterly} 21, no. 1 (1957):
61--78. \url{https://doi.org/10.1086/266687}.

Krippendorff, Klaus. ``Principales metáforas de la comunicación y
algunas reflexiones constructivistas acerca de su utilización'' en
\emph{Construcciones de la experiencia humana II}, editado por Marcelo
Pakman. Barcelona: Gedisa, 1997: 107-46.

Lacan, Jaques. \emph{El Seminario.} Libro 10. \emph{La angustia}. Trad.
de Enric Berenguer. Buenos Aires: Paidós, 2018.

Lakoff, George y Mark Johnson. \emph{Metáforas de la vida cotidiana}.
Trad. de Carmen González Marín. Madrid: Cátedra, 2009.

Lancien, Thierry et al. ``La recherche en communication en France.
Tendences et carences''. \emph{Recherche \& communication}, dirigido por
Thierry Lancien. \emph{MEI} \emph{(Médiation et Information)}, n.° 14.
Saint-Denis: L'Harmattan, 2001.

Langlois, Ganaele. ``Distributed Intelligence: Silk Weaving and the
Jacquard Mechanism.'' \emph{Canadian Journal of Communication} 44, no. 4
(2019): 555--66. \url{https://doi.org/10.22230/cjc.2019v44n4a3723}.

Lazarsfeld, Paul F., y Elihu Katz. \emph{La influencia personal: el
individuo en el proceso de comunicación de masas}. Barcelona: Hispano
Europea, 1979.

Lizcano, Emmanuel. \emph{Metáforas que nos piensan}. Madrid: Ediciones
Bajo Cero / Traficantes de Sueños, 2014.

Magallanes Blanco, Claudia, y José Manuel Ramos Rodríguez, coords.
\emph{Miradas propias. Pueblos indígenas, comunicación y medios en la
sociedad global}. Quito: CIESPAL, 2016.

Marques de Melo, José. \emph{Pensamiento comunicacional latinoamericano.
Entre el saber y el poder}. Sevilla: Comunicación Social, 2009.

Mattelart, Armand. \emph{La comunicación-mundo. Historia de las ideas y
de las estrategias}. Trad. de Gilles Multigner. México: Siglo XXI, 1997.

Mattelart, Armand. \emph{La invención de la comunicación}. Trad. de
Gilles Multigner. Barcelona: Bosch, 1995.

McCombs, Maxwell E., and Donald L. Shaw. ``The Agenda-Setting Function
of Mass Media.'' \emph{Public Opinion Quarterly} 36, no. 2 (1972):
176--87.

McLuhan, Marshall. \emph{Comprender los medios de comunicación}. Trad.
de Patrick Ducher. Madrid: Paidós, 1996.

McLuhan, Marshall. \emph{La galaxia Gutenberg. Génesis del ``Homo
typographicus''}. Trad. de Juan Novella. Barcelona: Círculo de Lectores,
1993.

Meunier, Jean Pierre. ``Las metáforas de comunicación como metáforas que
cobran realidad''. \emph{Signo y Pensamiento} 16, n.° 30 (1997):
115--28.
\url{https://revistas.javeriana.edu.co/index.php/signoypensamiento/article/view/5537}.

Miike, Yoshitaka and Jing Yin, eds. \emph{The Handbook of Global
Interventions in Communication Theory}. New York: Routledge, 2022.

Morales Damián, Manuel Alberto. ``La tejedora, la muerte y la vida.
Simbolismo maya del trabajo textil en el Códice Tro-Cortesia''.
\emph{Datatèxtil}, n.° 24 (2011): 76--83.
\url{https://raco.cat/index.php/Datatextil/article/view/275363}.

Mosco, Vincent. \emph{La nube. Big Data en un mundo turbulento}.
Barcelona: Intervención cultural / Biblioteca Buridán, 2014.

Muñoz-Torres, Juan Ramón. ``Abuso de la metáfora y laxitud conceptual en
comunicación''. \emph{Mediaciones Sociales. Revista de Ciencias Sociales
y de la Comunicación}, n.º 11 (2012): 3--27.
\url{http://dx.doi.org/10.5209/rev_MESO.2012.v11.41267}.

Musso, Pierre. ``Génesis y crítica de la noción de red'', trad. de Jorge
Márquez Valderrama, \emph{Ciencias Sociales y Educación} 2, n.º 3
(enero-junio 2013): 201--24.

Noelle-Neumann, Elizabeth. \emph{La espiral del silencio}. Trad. de
Javier Ruíz Calderón. Barcelona: Paidós, 2003.

Ong, Walter. \emph{Oralidad y escritura. Tecnologías de la palabra}.
Trad. de Angélica Scherp. México: Fondo de Cultura Económica, 2011.

Panagl, Oswald. ``Der Text als Gewebe: Lexikalische Studien mi
Sinnbezirk von Webstuhl und Kleid.'' In \emph{Textile Terminologies from
the Orient to the Mediterranean and Europe, 1000 BC to 1000 AD}, edited
by Salvatore Gaspa, Cécile Michel, and Marie-Louise Nosch. Nebraska: Zea
Books Lincoln, 2017.

Papalini, Vanina. ``La comunicación según las metáforas oceánicas''.
\emph{Razón y Palabra}, n.º 78 (noviembre 2011-enero 2012).
\url{http://www.razonypalabra.org.mx/varia/N78/1a\%20parte/02_Papalini_V78.pdf}.

Pérez Álvarez, Federico, y Carmen Timoneda Gallart. ``El poder de la
metáfora en la comunicación humana: ¿Qué hay de cierto? La metáfora en
la teoría y la práctica perspectiva en neurociencia''.
\emph{International Journal of Developmental and Educational Psychology}
6, no. 1 (2014): 493--500.
\url{https://doi.org/10.17060/ijodaep.2014.n1.v6.769}.

Peters, John Durham. \emph{Hablar al aire. Una historia de la idea de
comunicación}. Trad. de José María Ímaz. México: Fondo de Cultura
Económica, 2014.

Peters, John Durham. \emph{The Marvelous Clouds. Toward a Philosophy of
Elemental Media}. Chicago: University of Chicago Press, 2015.

Pintos, Juan Luis. ``Comunicación, construcción de la realidad e
imaginarios sociales''. \emph{Utopía y Praxis Latinoamericana} 10, n.º
29 (2005): 37--65.

Plant, Sadie. \emph{Ceros + unos. Mujeres digitales + la nueva cultura}.
Trad. de Eduardo Urios. Barcelona: Destino, 1998.

Postrel, Virginia. \emph{El tejido de la civilización. Cómo los textiles
dieron forma al mundo}. Trad. de Lorenzo Luengo. Madrid: Siruela, 2020.

Reddy, Michael. ``The Conduit Metaphor: A Case of Frame Conflict in our
Language about Language.'' In \emph{Metaphor and Thought}, edited by
Eden A. Ortony, 164--201. Cambridge: Cambridge University Press, 1993.

Rivadulla, Andrés. ``Metáforas y modelos en ciencia y filosofía''.
\emph{Revista de Filosofía} 31, n.° 2 (2006): 189--202.

Roberts, Edward A. \emph{Diccionario etimológico indoeuropeo de la
lengua española}. Madrid: Alianza, 1996.

Rodríguez, Pablo. \emph{Historia de la información. Del nacimiento de la
estadística y la matemática moderna a los medios masivos y las
comunidades virtuales}. Buenos Aires: Capital Intelectual, 2012.

Saltzman, Andrea. \emph{La metáfora de la piel. Sobre el diseño de la
vestimenta}. Buenos Aires: Paidós, 2019.

Sánchez-Aldana, Eliana, Tania Pérez-Bustos, y Alexandra
Chocontá-Piraquive. ``¿Qué son los activismos textiles?: una mirada
desde los estudios feministas a catorce casos bogotanos''. \emph{Athenea
Digital} 19, n.° 3 (noviembre 2019): e2407.
\url{https://doi.org/10.5565/rev/athenea.2407}.

Schmucler, Héctor. ``La investigación: un proyecto
comunicación/cultura.'' En \emph{Memoria de la Comunicación, }145--51.
Buenos Aires: Biblos, 1997.

Scolari, Carlos A. ``Ecología de los medios: de la metáfora a la teoría
(y más allá)''. En \emph{Ecología de los medios: entornos, evoluciones e
interpretaciones, }15--42. Gedisa: Barcelona, 2015.

Segarra Crespo, Diana. ``Coser y cantar: a propósito del tejido y la
palabra en la cultura clásica'', en \emph{Tejer y vestir. De la
Antigüedad al Islam}, editado por Manuela Marín. Madrid: CSIC, 2001.

Sfez, Lucien. \emph{Crítica de la comunicación}. Trad. de Aníbal C.
Leal. Buenos Aires: Amorrortu, 1995..

Sierra Caballero, Francisco, y Claudio Maldonado Rivera, coords.
\emph{Comunicación, decolonialidad y Buen Vivir}. Quito: CIESPAL, 2016.

Taylor, Cynthia, and Bryan M. Dewsbury. ``On the Problem and Promise of
Metaphor Use in Science and Science.'' \emph{Journal of Microbiology \&
Biology Education} 19, no. 1 (2018).
\url{https://doi.org/10.1128/jmbe.v19i1.1538}.

Thompson, John B. \emph{Los media y la modernidad: una teoría de los
medios de comunicación}. Barcelona: Paidós, 1998.

Torrico Villanueva, Erick R. ``Decolonizar la comunicación''. En
\emph{Comunicación, decolonialidad y Buen Vivir}, editado por Francisco
Sierra Caballero y Claudio Maldonado Rivera, 95-112. Quito: CIESPAL,
2016.

Torrico Villanueva, Erick R. \emph{La comunicación pensada desde América
Latina (1960--2009)}. Salamanca: Comunicación Social, 2016.

Winkin, Ives. ``El telégrafo y la orquesta''. En \emph{La nueva
comunicación}, de Gregory Bateson et al. Trad. de Jorge Fibla.
Barcelona: Kairós, 1984: 11--25.

Wunenburger, Jean-Jacques. \emph{La vida de las imágenes}. Trad. de Hugo
Francisco Bauzá. Buenos Aires: UNSAM, 2005.



\end{hangparas}


\end{document}