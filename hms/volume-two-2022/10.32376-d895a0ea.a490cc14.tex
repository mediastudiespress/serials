% see the original template for more detail about bibliography, tables, etc: https://www.overleaf.com/latex/templates/handout-design-inspired-by-edward-tufte/dtsbhhkvghzz

\documentclass{tufte-handout}

%\geometry{showframe}% for debugging purposes -- displays the margins

\usepackage{amsmath}

\usepackage{hyperref}

\usepackage{fancyhdr}

\usepackage{hanging}

\hypersetup{colorlinks=true,allcolors=[RGB]{97,15,11}}

\fancyfoot[L]{\emph{History of Media Studies}, vol. 2, 2022}


% Set up the images/graphics package
\usepackage{graphicx}
\setkeys{Gin}{width=\linewidth,totalheight=\textheight,keepaspectratio}
\graphicspath{{graphics/}}

\title[El imaginario textil]{El imaginario textil: una interpretación alternativa en los estudios de la comunicación} % longtitle shouldn't be necessary

% The following package makes prettier tables.  We're all about the bling!
\usepackage{booktabs}

% The units package provides nice, non-stacked fractions and better spacing
% for units.
\usepackage{units}

% The fancyvrb package lets us customize the formatting of verbatim
% environments.  We use a slightly smaller font.
\usepackage{fancyvrb}
\fvset{fontsize=\normalsize}

% Small sections of multiple columns
\usepackage{multicol}

% Provides paragraphs of dummy text
\usepackage{lipsum}

% These commands are used to pretty-print LaTeX commands
\newcommand{\doccmd}[1]{\texttt{\textbackslash#1}}% command name -- adds backslash automatically
\newcommand{\docopt}[1]{\ensuremath{\langle}\textrm{\textit{#1}}\ensuremath{\rangle}}% optional command argument
\newcommand{\docarg}[1]{\textrm{\textit{#1}}}% (required) command argument
\newenvironment{docspec}{\begin{quote}\noindent}{\end{quote}}% command specification environment
\newcommand{\docenv}[1]{\textsf{#1}}% environment name
\newcommand{\docpkg}[1]{\texttt{#1}}% package name
\newcommand{\doccls}[1]{\texttt{#1}}% document class name
\newcommand{\docclsopt}[1]{\texttt{#1}}% document class option name


\begin{document}

\begin{titlepage}

\begin{fullwidth}
\noindent\LARGE\emph{Exclusions in the History of Media Studies
} \hspace{25mm}\includegraphics[height=1cm]{logo3.png}\\
\noindent\hrulefill\\
\vspace*{1em}
\noindent{\Huge{El imaginario textil: una interpretación\\\noindent alternativa en los estudios de la comunicación\par}}

\vspace*{1.5em}

\noindent\LARGE{Daniel H. Cabrera Altieri} \href{https://orcid.org/0000-0001-6781-260X}{\includegraphics[height=0.5cm]{orcid.png}}\par}\marginnote{\emph{Daniel H. Cabrera Altieri, ``El imaginario textil: una interpretación alternativa en los estudios de la comunicación,'' \emph{History of Media Studies} 2 (2022), \href{https://doi.org/10.32376/d895a0ea.a490cc14}{https://doi.org/ 10.32376/d895a0ea.a490cc14}.} \vspace*{0.75em}}
\vspace*{0.5em}
\noindent{{\large\emph{Universidad de Zaragoza}, \href{mailto:danhcab@gmail.com}{danhcab@gmail.com}\par}} \marginnote{\href{https://creativecommons.org/licenses/by-nc/4.0/}{\includegraphics[height=0.5cm]{by-nc.png}}}

% \vspace*{0.75em} % second author

% \noindent{\LARGE{<<author 2 name>>}\par}
% \vspace*{0.5em}
% \noindent{{\large\emph{<<author 2 affiliation>>}, \href{mailto:<<author 2 email>>}{<<author 2 email>>}\par}}

% \vspace*{0.75em} % third author

% \noindent{\LARGE{<<author 3 name>>}\par}
% \vspace*{0.5em}
% \noindent{{\large\emph{<<author 3 affiliation>>}, \href{mailto:<<author 3 email>>}{<<author 3 email>>}\par}}

\end{fullwidth}

\vspace*{2.5em}


\hypertarget{resumen}{%
\section{Resumen}\label{resumen}}

El presente artículo llama la atención sobre la relevancia del tejido
para una conceptualización alternativa de las teorías de la
comunicación. El vocabulario actual y una serie de expresiones
cotidianas mantienen viva una memoria que escapa a la \emph{amnesia
textil} que parece caracterizar nuestra época. Las metáforas textiles
hablan de un sustrato comunicacional que va más allá del recurso
narrativo hacia sus dimensiones heurísticas y cognitivas. Una visión
logomediacéntrica ha ocultado el papel del telar y los tejidos en la
historia de los medios de comunicación y, en especial, la vivacidad de
su presencia y significación para los pueblos del Sur Global. La
constatación de una \emph{centralidad subterránea de lo textil} muestra
algunos trazos para el estudio de la comunicación desde el entrelazado
creador, el cuidado de la vida y del \emph{tejido social}.

\hypertarget{abstract}{%
\section{Abstract}\label{abstract}}

This article draws attention to the relevance of weaving for an
alternative conceptualization for the theories of communication. Current
vocabulary and a series of everyday expressions keep alive a memory that
escapes the \emph{textile amnesia} which seems to characterize our times.
Textile metaphors speak of a communicational substrate that goes beyond
the narrative resource to its heuristic and cognitive dimensions. A logo-media-centric vision has obscured the role of the loom and

\enlargethispage{2\baselineskip}

\vspace*{2em}

\noindent{\emph{History of Media Studies}, vol. 2, 2022}


 \end{titlepage}


\noindent weaving
in the history of media and, especially, the vivacity of their presence
and significance for the Global South countries. The acknowledgment of a
\emph{subterranean centrality of the textile} shows some traces for the study
of communication from the creative interlacing, care for life and of the
\emph{social fabric}.

\vspace*{2em}


\hypertarget{introduccin}{%
\section{Introducción}\label{introduccin}}

\newthought{El estudio de la comunicación}
  ha sido clasificado desde diferentes
perspectivas, dando lugar a múltiples balances y evaluaciones en varias
lenguas y publicaciones.\footnote{Existen múltiples investigaciones
  metateóricas. Para el caso latinoamericano, Raúl Fuentes Navarro,
  ``Cuatro décadas de internacionalización académica en el campo de
  estudios de la comunicación en América Latina'', en \emph{Anuario
  Electrónico de Estudios en Comunicación Social, Disertaciones} 9, n.º
  2 (2016): 8-26,
  \url{https://doi.org/10.12804/disertaciones.09.02.2016.01}; José
  Marques de Melo, \emph{Pensamiento comunicacional latinoamericano.
  Entre el saber y el poder} (Sevilla, Comunicación Social, 2009); Erick
  R. Torrico Villanueva, \emph{La comunicación pensada desde América
  Latina (1960-2009)} (Salamanca: Comunicación Social, 2016). Existen
  mapas de las investigaciones como el ``Mapping Media and Communication
  Research'' realizadas por el Communication Research Centre de la
  University of Helsinki en Finlandia y referidos a varios países
  europeos, Estados Unidos, Japón y Australia. Juha Herkman, ``Current
  trends in media research'', \emph{Nordicom Review} 29: 1 (2008):
  145--59. Informes disponibles en
  \url{http://www.valt.helsinki.fi/blogs/crc/en/mapping.htm}. Para el
  caso francés, véase Thierry Lancien et al., ``La recherche en
  communication en France. Tendences et carences'', \emph{Recherche \&
  communication}, dir. por Thierry Lancien, \emph{MEI (Médiation et
  Information)}, n.º 14 (Saint-Denis: L'Harmattan, 2001): 37-62, y el
  \emph{Journal of Communication}, ``Ferment in the Field'', número
  especial 33, n.º 3 (1983); ``Future of the Discipline'', número
  especial 43, n.º 3 (1993), y ``The State of the Art in Communication
  Theory and Research'', número especial 54, n.º 4 (2004).} Ello ha
generado un enfoque metadiscursivo con diversas clasificaciones de las
teorías, entre las que destaca la realizada por Robert T. Craig en torno
a las ``tradiciones'' de investigación analizadas desde un enfoque
pragmatista.\footnote{Robert. T. Craig, ``Communication Theory as a
  Field'', \emph{Communication Theory} 9, n.º 2 (1999): 119-20.} El
presente trabajo parte desde otro lugar, desde la consideración de las
metáforas cognitivas y los imaginarios sociales con los que se ha
estudiado la comunicación para proponer una interpretación alternativa
de la comunicación centrada en la metáfora y el imaginario del tejido.

Las teorías de la comunicación son interpretaciones que, en algún
momento, recurren a metáforas heurísticas de diversos tipos, entre las
que sobresalen las de la \emph{transmisión} y la \emph{red}. Una
documentada búsqueda revela que no se ha realizado ninguna
interpretación desde el imaginario de lo textil en los centros de
investigación y universidades euroamericanas o del Norte Global. Sin
embargo, el tejido no es una metáfora más, como se verá, sino una
fundamental que ha estado muy presente desde la antigüedad de diversas
culturas para referirse a la comunicación en sus diversos aspectos.

Este texto se enfoca desde la teoría de los imaginarios sociales para
interpretar las metáforas cognitivas de las teorías de la comunicación.
Se comenta el logomediacentrismo en su linealidad y evolucionismo
teleológico como posible causa de exclusión de lo textil. Se presenta el
papel de los telares mecánicos en la revolución industrial y su
influencia en la concepción digital a través del uso de la tarjeta
perforada y el cambio en la mirada. A continuación, se analizan las
metáforas y el imaginario textil presentes en el lenguaje cotidiano
actual y se comenta el caso de la Grecia clásica como un momento
fundamental en la concepción heredada occidental euroamericana del
tejido. Finalmente, se propone una interpretación textil de la
comunicación como entrelazado creador, red de cuidados y espacio de
\emph{tejido social}.

Este artículo continuará, en un próximo texto, profundizando el
imaginario textil desde el Sur Global, en particular desde el espacio
andino, para proponer una comprensión alter/nativa\footnote{``Conviene
  escribir esta alter/nativa con una barra para marcar no solo su
  carácter potencial de otredad frente a la Comunicación que ya tenemos
  instalada en los estudios académicos... sino igualmente para hacer un
  cierto énfasis en la conexión de esa otredad con lo nativo, lo propio
  y diferenciador''. Erick R. Torrico Villanueva, ``Decolonizar la
  comunicación'', en \emph{Comunicación, decolonialidad y Buen Vivir},
  coord. de Francisco Sierra Caballero y Claudio Maldonado Rivera
  (Quito: CIESPAL, 2016): 96.} del fenómeno de la comunicación y de su
estudio, la comunicología.

\hypertarget{comunicacin-e-imaginario-social}{%
\section{Comunicación e imaginario
social}\label{comunicacin-e-imaginario-social}}

Las teorías de la comunicación se han formulado desde una gran cantidad
de metáforas convertidas en modelos y desarrollos conceptuales que han
guiado la actividad de investigadores, medios de comunicación y
comunicadores. En este apartado se propone pensar las diferentes
interpretaciones del fenómeno de la comunicación como imaginarios
apalabrados y visualizados por metáforas que culminan con las referidas
al \emph{transporte} y la \emph{red}.

\hypertarget{teora-metfora-e-imaginario}{%
\subsection{Teoría, metáfora e
imaginario}\label{teora-metfora-e-imaginario}}

\begin{quote}
¿Qué es entonces la verdad? Un ejército móvil de metáforas, metonimias,
antropomorfismos... y que, después de un prolongado uso, a un pueblo le
parecen fijas, canónicas, obligatorias... metáforas que se han vuelto
gastadas y sin fuerza sensible.

---Friedrich Nietzsche, \emph{Sobre verdad y mentira en sentido
extramoral}.
\end{quote}

Las metáforas son instrumentos fundamentales en la producción científica
que potencian el razonamiento analógico, la elaboración de hipótesis, la
interpretación de resultados y la comunicación de los
hallazgos.\footnote{Cynthia Taylor y Bryan M. Dewsbury, ``On the Problem
  and Promise of Metaphor Use in Science and Science'', \emph{Journal of
  Microbiology \& Biology Education} 19, n.º 1 (marzo 2018),
  \url{https://doi.org/10.1128/jmbe.v19i1.1538}; Federico Pérez Álvarez
  y Carmen Timoneda Gallart, ``El poder de la metáfora en la
  comunicación humana: ¿Qué hay de cierto? La metáfora en la teoría y la
  práctica perspectiva en neurociencia'', \emph{International Journal of
  Developmental and Educational Psychology} 6, n.º 1 (2014): 493-500,
  \url{https://doi.org/10.17060/ijodaep.2014.n1.v6.769}.} La capacidad
heurística del razonamiento metafórico la convierte en algo esencial
para hacer ciencia.\footnote{Olaf Jäkel, Martin Döring y Anke Beger,
  ``Science and metaphor: a truly interdisciplinary perspective. The
  third international metaphorik.de workshop'', \emph{Metaphorik.de --
  online journal on metaphor and metonymy}, n.º 26 (2016).}
La psicología cognitiva afirma que ``cada concepto es el resultado de
una larga serie de analogías espontáneas y los elementos de una
situación se categorizan exclusivamente a través de analogías, por
triviales que estas puedan parecer''.\footnote{Douglas Hofstadter y
  Emmanuel Sander, \emph{La analogía. El motor del pensamiento}, trad.
  de Roberto Musa Giuliano (Barcelona: Tusquets, 2018): 73.}

En esta misma tradición, hace tiempo, George Lakoff y Mark Johnson
dejaron claro el papel cognitivo de las metáforas en la vida
cotidiana.\footnote{George Lakoff y Mark Johnson, \emph{Metáforas de la
  vida cotidiana}, trad. de Carmen González Marín (Cátedra: Madrid,
  2009).} En ambos casos, en las ciencias o en la vida cotidiana, la
metáfora no es solo un tropo sino una forma de conocimiento o, incluso,
``el motor del pensamiento'', ya que ``sin conceptos no hay pensamientos
y sin analogías no hay conceptos''.\footnote{Hofstadter y Sander,
  \emph{La analogía,} 2.} Aprendemos lo nuevo, lo inesperado, lo
extraño, por similitudes y parentescos con lo conocido.

Aunque la filosofía había destacado el papel de la metáfora, al menos
desde la \emph{Poética} de Aristóteles, es Friedrich Nietzsche quién
subraya la dimensión cognitiva de la metáfora.\textsuperscript{9} La
metáfora habla de los contextos sociales y culturales vivos, las
experiencias de los sujetos, de sus estados de ánimo y de sus
intereses.\textsuperscript{10}
Desde la teoría de los\marginnote{\textsuperscript{9} Daniel
  Innerarity, ``La seducción del lenguaje. Nietzsche y la metáfora'',
  \emph{Contrastes: revista internacional de filosofía}, n.º 3 (1998):
  123-45; Cirilo Flórez Miguel, ``Retórica, metáfora y concepto en
  Nietzsche'', \emph{Estudios Nietzsche}, n.º 4 (2004): 51-67.}\marginnote{\textsuperscript{10}\setcounter{footnote}{10} Eduardo de Bustos, \emph{La metáfora. Ensayos
  transdisciplinares} (Madrid: Fondo de Cultura Económica / UNED, 2000).} imaginarios sociales,\footnote{Véanse Gilbert
  Durand, \emph{Las estructuras antropológicas del imaginario}, trad. de
  Víctor Goldstein (México: Fondo de Cultura Económica , 2004);
  Cornelius Castoriadis, \emph{La institución imaginaria de la
  sociedad}, trad. de Antoni Vicens y Marco Aurelio Galmarini
  (Barcelona: Tusquets, 1993).} la tarea del conocimiento de las
ciencias sociales consiste en diluir los \emph{sólidos} conceptos en su
suelo \emph{líquido} y en el ambiente \emph{gaseoso} en el que
surgieron.\footnote{Daniel H. Cabrera Altieri, \emph{Tecnología como
  ensoñación. Ensayos sobre el imaginario tecnocomunicacional} (Temuco:
  Ediciones Universidad de la Frontera, 2022): 77-96,
  \url{http://bibliotecadigital.ufro.cl/?a=view\&item=1962}.} La
interpretación concebida como una tarea alquímica que regrese lo
determinado a lo indeterminado, lo definido a lo indefinido, como una
manera de hacer \emph{surgir} nuevas interpretaciones de las categorías
con las que explicamos la realidad social y, en este caso, el fenómeno
de la comunicación. El enfoque de imaginario apunta a cuestionar la base
epistemológica de los estudios de comunicación que, cabe recordar, ``no
profundizan mucho en la epistemología'', aunque ``se escriban mares de
tinta sobre el supuesto estatuto disciplinario de la
comunicación''.\footnote{Tanius Karam, ``Tensiones para un giro
  decolonial en el pensamiento comunicológico. Abriendo la discusión'',
  \emph{Chasqui. Revista Latinoamericana de Comunicación}, n.º 133 (dic.
  2016-mar. 2017): 247-64.}

Abordar la teoría como una hueste de metáforas cognitivas lleva a
considerar sus conceptos e implicaciones como construcciones inspiradas
en el imaginario social en un camino que la interpretación puede
desandar comenzando con una actitud de extrañamiento.\footnote{Emmanuel
  Lizcano, \emph{Metáforas que nos piensan} (Madrid: Ediciones Bajo Cero
  / Traficantes de Sueños, 2014): 37-71; Jean-Jacques Wunenburger,
  \emph{La vida de las imágenes}, trad. de Hugo Francisco Bauzá (Buenos
  Aires: UNSAM, 2005): 21-50.} La metáfora es un apalabramiento y una
puesta en imagen del imaginario social. Imaginario que habla de las
relaciones arbitrarias y creativas entre lo que se busca entender su
``más (\emph{meta}) allá (\emph{fora})'' en la creación de universos de
significados.\footnote{Daniel H. Cabrera Altieri, \emph{Lo tecnológico y
  lo imaginario. Las nuevas tecnologías como creencias y esperanzas
  colectivas} (Buenos Aires: Biblos, 2006): 25-85.}

\hypertarget{las-teoras-de-la-comunicacin-y-sus-metforas}{%
\subsection{Las teorías de la comunicación y
sus
metáforas}\label{las-teoras-de-la-comunicacin-y-sus-metforas}}

Las explicaciones de la comunicación contienen numerosas metáforas
heurísticas en la mayoría de los casos no tomadas como tal y que han
sido analizadas en diversas ocasiones. Se habla, por ejemplo, de la
``\emph{aguja} hipodérmica'' y la ``\emph{bala} mágica''\footnote{José
  Luis Dader, ``La evolución de las investigaciones sobre la influencia
  de los medios y su primera etapa: Teorías del impacto directo'', en
  \emph{Opinión pública y comunicación política}, de Alonso Muñoz et al.
  (Madrid: Eudema, 1990).} para indicar un \emph{efecto} inmediato y con
ello introducir una comprensión de la comunicación como \emph{salud}
social y como \emph{guerra}. El ``\emph{flujo} de la comunicación en
\emph{dos etapas}''\footnote{Paul F. Lazarsfeld y Elihu Katz, \emph{La
  influencia personal: el individuo en el proceso de comunicación de
  masas} (Barcelona: Hispano Europea, 1979); Elihu Katz, ``The Two-Step
  Flow of Communication: An Up-To-Date Report on an Hypothesis'',
  \emph{Political Opinion Quarterly} 21, n.º 1 (primavera 1957): 61-78,
  \url{https://doi.org/10.1086/266687}.} que tendrá un gran recorrido
con la idea de la comunicación como \emph{fluido} y \emph{proceso} y de
la sociedad dividida en etapas con los \emph{líderes} de opinión como
figura explicativa. La ``\emph{espiral} del silencio''\footnote{Elizabeth
  Noelle-Neumann, \emph{La espiral del silencio}, trad. de Javier Ruíz
  Calderón (Barcelona: Paidós, 2003).} presenta la circularidad y
recursividad del \emph{silencio} de las \emph{minorías}. La metáfora de
la ``\emph{agenda}''\textsuperscript{19} para concebir la sociedad
como una \emph{lista} de temas de \emph{conversación}; ``\emph{marco}''
(\emph{frames})\textsuperscript{20}
aplicadas a la comunicación donde la información es una \emph{mirada} a
través de una \emph{ventana}, un \emph{encuadre visual}. Y así tenemos
también las metáforas del \emph{conducto},\textsuperscript{21}
el \emph{fluido},\textsuperscript{22}
la \emph{ecología},\textsuperscript{23} la
\emph{red},\textsuperscript{24} la
\emph{autopista}.\textsuperscript{25} Expresiones que designan un enfoque o
una teoría con conceptos cuya metaforización resulta transparente.
Incluso el uso de la palabra ``comunicación'' tiende a connotarse como
concordia y convivencia en clara referencia, consciente o no, a las
etimologías de ``comunicación'' (\emph{lo común}, \emph{poner en común,
comunión}, \emph{comunidad}) y de ``información'' (\emph{dar forma}) que
remiten a un \emph{hummus} metafórico, un imaginario social, que aún
permanece en el núcleo de significaciones actuales.

Klaus\marginnote{\textsuperscript{19} Maxwell E. McCombs y Donald L. Shaw, ``The
  Agenda-Setting Function of Mass Media'', \emph{Public Opinion
  Quarterly} 36, n.º 2 (verano 1972): 176-87.}\marginnote{\textsuperscript{20} Gregory Bateson, ``Una teoría del juego y de la
  fantasía'', en \emph{Pasos hacia una ecología de la mente}, trad. de
  Ramón Alcalde (Buenos Aires: Editorial Lohlé-Lumen, 1991): 132-42.}\marginnote{\textsuperscript{21} Analizado por
  Michael Reddy, ``The conduit metaphor: A case or Frame Conflict in our
  language about language'', en \emph{Metaphor and Thought}, ed. de
  Andrew Ortony (Cambridge: Cambridge University Press, 1993): 164-201.}\marginnote{\textsuperscript{22} Vanina Papalini, ``La comunicación según las
  metáforas oceánicas'', \emph{Razón y Palabra}, n.º 78 (nov. 2011-ene.
  2012),
  \url{http://www.razonypalabra.org.mx/varia/N78/1a\%20parte/02_Papalini_V78.pdf}.}\marginnote{\textsuperscript{23} Analizado por Carlos A. Scolari, ``Ecología
  de los medios: de la metáfora a la teoría (y más allá)'', en
  \emph{Ecología de los medios: entornos, evoluciones e
  interpretaciones} (Gedisa: Barcelona, 2015): 15-42.}\marginnote{\textsuperscript{24} Analizado por Pierre Musso, ``Génesis y crítica de
  la noción de red'', trad. de Jorge Márquez Valderrama, \emph{Ciencias
  Sociales y Educación} 2, n.º 3 (enero-junio 2013): 201-24.}\marginnote{\textsuperscript{25}\setcounter{footnote}{25} Véase Patrice Flichy, \emph{Lo imaginario de
  internet}, trad. de Félix de la Fuente y Mireia de la Fuente Rocafort
  (Madrid: Tecnos, 2003): 25-47.} Krippendorff agrupó algunas metáforas de la
comunicación\footnote{Klaus Krippendorff, ``Principales metáforas de la
  comunicación y algunas reflexiones constructivistas acerca de su
  utilización'', en \emph{Construcciones de la experiencia humana II},
  ed. de Marcelo Pakman (Barcelona: Gedisa, 1997): 107-46.} que, desde
la investigación y el sentido común, funcionan como verdaderas teorías
constructoras de la realidad. Las expuso en seis metáforas centrales:

\begin{itemize}
\item
  \emph{metáforas del receptáculo:} referido al ``contenido'', ``lleno
  de significado'', ``carente de sentido'', ``frases vacías'', etcétera
\item
  \emph{metáforas del conducto}: derivado de las tecnologías del cable,
  el tubo, fluir, fuente, canales, etcétera
\item
  \emph{metáforas del control}: fenómeno causal, medio, instrumentos,
  dirigir, activos-pasivos, mejorar la eficacia, comunicación exitosa,
  etcétera
\item
  \emph{metáforas de la transmisión}: de las tecnologías: descifrar,
  código, transmitir, codificar/decodificar, etcétera
\item
  \emph{metáforas de la guerra:} afirmaciones ``defendibles'', ``dar en
  el blanco'', ``ganar'' una discusión, etcétera
\item
  \emph{metáforas de la danza-ritual}: \emph{performance},
  participación, contacto, empatía, etcétera
\end{itemize}

Jean Pierre Meunier también ha analizado el uso explicativo de las
metáforas\footnote{Jean Pierre Meunier, ``Las metáforas de comunicación
  como metáforas que cobran realidad'', \emph{Signo y Pensamiento} 16,
  n.º 30 (1997): 115-28,
  \url{https://revistas.javeriana.edu.co/index.php/signoypensamiento/article/view/5537}}
entendiendo su dimensión de código como un telégrafo, su dimensión de
acción como intercambio codificado, la estrategia y el modelo de la
computadora en la comprensión de los procesos cognitivos. Como han hecho
otros,\footnote{Juan Ramón Muñoz-Torres, ``Abuso de la metáfora y
  laxitud conceptual en comunicación'', \emph{Mediaciones Sociales.
  Revista de Ciencias Sociales y de la Comunicación}, n.º 11 (2012):
  3-26, \url{http://dx.doi.org/10.5209/rev_MESO.2012.v11.41267}} el
análisis de Meunier busca alertar sobre el mal uso o abuso de las
metáforas en las teorías. Tal como se entiende aquí el problema esencial
no es el uso torcido o torticero de las metáforas sino que su presencia,
de modo transparente u oculto, construye cognitivamente una
\emph{realidad} sobre la que se actúa de manera acorde con ese mismo
conocimiento. La comunicación es lo que sus teorías dicen y con esas
definiciones se actúa, se comunica.

\hypertarget{tecnologas-de-la-comunicacin-del-transporte-a-la-red}{%
\subsection{Tecnologías de la comunicación: del
transporte a la
red}\label{tecnologas-de-la-comunicacin-del-transporte-a-la-red}}

Las metáforas en su dimensión heurística tienen un papel muy importante
en la comprensión de los imaginarios de la comunicación, en el trazado
de su genealogía y en la identificación de las matrices conceptuales
desde las que se han generado las teorías de la comunicación. En este
sentido, destaca el trabajo de Armand Mattelart y su interpretación de
la comunicación, en sus palabras, la ``invención de la comunicación'',
desde las arqueologías de cuatro historias: la domesticación de los
\emph{flujos} y de la sociedad en \emph{movimiento}; la concepción y
fabricación de un \emph{vínculo} universal entre los humanos; el
\emph{espacio} geopolítico, y la \emph{normalización} y la aparición del
individuo \emph{calculable}.\footnote{Armand Mattelart, \emph{La
  invención de la comunicación}, trad. de Gilles Multigner (Bosch:
  Barcelona, 1995); Armand Mattelart, \emph{La comunicación-mundo.
  Historia de las ideas y de las estrategias}, trad. de Gilles Multigner
  (México: Siglo XXI, 1997).} El trazado de las ideas y de las
estrategias de comunicación en el mundo contemporáneo rastreadas desde
el momento en que aún no existían los medios de masas le permite
concluir que ``la analogía biológica se ha instalado como matriz
natural, gran paradigma unificador, para dar cuenta del funcionamiento
de los sistemas de comunicación y del vínculo que los une a la sociedad
como un todo orgánico''.\footnote{Mattelart, \emph{La invención,} 370.}

John Durham Peters ha realizado su propia historia de la idea de
comunicación donde el \emph{diálogo}, cuyo modelo es el Sócrates de
Platón, y la \emph{diseminación}, según el modelo del Jesús de los
evangelistas, los medios como creadores de \emph{fantasmas}, \emph{la
comunicación con los muertos}, entre otros, constituyen verdaderos nodos
metafóricos conceptuales.\footnote{John Durham Peters, \emph{Hablar al
  aire. Una historia de la idea de comunicación}, trad. de José María
  Ímaz (México: Fondo de Cultura Económica, 2014).} Lejos de ser
recursos narrativos, las metáforas permiten adentrarse en las matrices
de la comunicación, en el espacio donde ideas e imaginarios se anudan
para alimentar una visión de la comunicación ``como una aventura
arriesgada y sin garantías''. Para justificarlo, recuerda la etimología
de la palabra comunicación en lo que tiene de problemático y, que por lo
tanto, no se suele citar. El autor descarta considerar el vocabulario
latino \emph{communicare} como el origen y se refiere al termino griego
\emph{koinóo}, \emph{``}más raramente citado pero igualmente
relevante'', que también significa hacer común, comunicar, compartir,
``pero también \emph{contaminar} o \emph{ensuciar}''.\footnote{Ibíd.,
  329-30.}

En ambos casos, desde dos enfoques muy diferentes, se enfrenta la tarea
de entender la comunicación y sus medios como un elemento central de la
sociedad actual y, tal vez por ello, Lucien Sfez afirma que ``una
convergencia estructural entre el uso sistemático de metáforas por la
ciencia de la comunicación y el nuevo hecho de que la comunicación
actual haya pasado a ser también una \emph{figura simbólica} de
importancia''.\footnote{Lucien Sfez, \emph{Crítica de la comunicación},
  trad. de Aníbal C. Leal (Buenos Aires: Amorrortu, 1995): 46.}

Las metáforas también han tenido un papel fundamental en la organización
de modelos alternativos de las teorías de la comunicación. Bajo el
sugerente título de ``El telégrafo y la orquesta'',\footnote{Ives
  Winkin, ``El telégrafo y la orquesta'', en \emph{La nueva
  comunicación}, de Gregory Bateson et al., trad. de Jorge Fibla
  (Barcelona: Kairos,1984): 11-25.} Ives Winkin opone dos formas de
explicar la comunicación.

Por un lado, la ``teoría matemática de la información'', de Claude
Shannon, que se postula como la transmisión exacta de un mensaje de un
punto a otro. La complejidad de la explicación matemática ha generado la
sensación de que el único elemento de la teoría de Shannon que ha podido
heredar ``a los legos en ingeniería sea la imagen del telégrafo que
impregna todavía el esquema original. Podríamos hablar así de \emph{un
modelo telegráfico de la comunicación}''.\footnote{Ibíd., 18.}

Por otro, un grupo interdisciplinario con diferentes sedes
universitarias compuesto por Gregory Bateson, Ray Birdwhistell, Edward
Hall, Erving Goffman, Don Jackson y Paul Watzlawick, entre otros,
animaban la idea de la ``comunicación como un todo integrado'', es
decir, como ``un proceso social permanente que integra múltiples modos
de comportamiento: la palabra, el gesto, la mirada, la mímica, el
espacio interindividual, etc.''.\footnote{Ibíd., 22-23.} Por ello,
Winkin sostiene que ``la analogía de la orquesta tiene la finalidad de
hacer comprender como puede decirse que cada individuo participa en la
comunicación, en vez de decir que constituye el origen o el fin de la
misma''.\footnote{Puede objetarse que un modelo no es una metáfora, pero
  en este tema cabe recordar el enfoque de Max Black cuando sostiene que
  ``toda metáfora es el aviso de un modelo sumergido''. Max Black citado
  en Andrés Rivadulla, ``Metáforas y modelos en ciencia y filosofía'',
  \emph{Revista de Filosofía} 31, n.º 2 (2006): 189-202.}

Winkin propone hablar de un \emph{modelo orquestal} de la comunicación
opuesto al lineal telegráfico. Lo cierto es que las telecomunicaciones y
la informática han transformado todo nuestro mundo, incluido el
imaginario \emph{de} y \emph{desde} el que se piensa la comunicación
humana.\footnote{Juan Luis Pintos, ``Comunicación, construcción de la
  realidad e imaginarios sociales'', \emph{Utopía y Praxis
  Latinoamericana} 10, n.º 29, (2005): 37-65.} Sin duda, el modelo
matemático de la información ha colonizado el universo metafórico con
una imagen/idea que lo ha inundado todo.\footnote{Véanse Pablo
  Rodríguez, \emph{Historia de la información. Del nacimiento de la
  estadística y la matemática moderna a los medios masivos y las
  comunidades virtuales} (Buenos Aires: Capital Intelectual, 2012);
  James Gleick, \emph{La información. Historia y realidad}, trad. de
  Juan Rabasseda y Teófilo de Lozoya (Barcelona: Crítica, 2011).} La
\emph{trasmisión} de un punto a otro, el \emph{transporte} de
información, un \emph{paquete} de datos, el \emph{conducto} y
\emph{canal}, el \emph{código}, la \emph{codificación} y la
\emph{decodificación}, la \emph{exactitud} matemática, entre otras, son
parte de un conjunto de metáforas con las que se comunican personas,
instituciones, empresas, gobiernos, y con las que reflexionan los
investigadores. Unos años después, James W. Carey propuso la célebre
distinción entre los modelos de \emph{transmisión} y el
\emph{ritual}\footnote{James W. Carey, ``A Cultural Approach to
  Communication'', en \emph{Communication as Culture: Essays on Media
  and Society} (Boston: Unwin Hyman, 1989): 13-36.} con algunas
coincidencias con lo comentado, pero con el objetivo de buscar una
teoría constitutiva de la comunicación.

Si nos centramos en las actuales tecnologías digitales, sobresalen otras
metáforas como \emph{viral}, \emph{nube}, \emph{red}. Lo \emph{viral}
proviene directamente del marketing diseñado para difundir información
muy rápidamente haciendo que sea altamente probable que se transmita de
persona a persona a través de medios electrónicos. El fenómeno de la
desinformación ha dejado al descubierto la ambivalencia del
\emph{virus}, destacando su papel de información manipulada
conscientemente para provocar estados de opinión o acciones en una
determinada dirección. Por su parte, la \emph{nube} tiene una rica
historia cultural en la literatura, la música y el arte que ``va más
allá de la representación de la sublimidad de la informática'' y cuyo
análisis ya ha sido desarrollado.\footnote{Vincent Mosco, \emph{La nube.
  Big Data en un mundo turbulento} (Barcelona: Intervención cultural /
  Biblioteca Buridán, 2014): 26. Véase John Durham Peters, \emph{The
  Marvelous Clouds. Toward a Philosophy of Elemental Media} (Chicago:
  The University of Chicago, 2015).} Se ha mencionado la metáfora de la
\emph{red}, quizás, la metáfora o ``palabra fetiche'', como afirma
Pierre Musso, más desarrollada desde diferentes puntos de vista como una
lógica social cultural y económica global.\footnote{Manuel Castells,
  \emph{La sociedad de la información.} Vol. 1, \emph{La sociedad red},
  trad. de Carmen Martínez Gimeno y Jesús Alborés (Madrid: Alianza,
  2000).}

La metáfora \emph{red} tiene capacidad fundacional del mundo en el que
vivimos. Toda metáfora requiere avanzar con cuidado, aunque en este caso
más aún porque ``las metáforas no son nada sin las posiciones políticas
y metafísicas que defienden''.\footnote{Ibíd., 47.} \emph{Red} tiene un
origen evidente en la fabricación de tejidos, la red como un conjunto de
hilos entrelazados, líneas y nudos. En ese sentido, hace referencia a
``la mitología del tejido''.\footnote{Musso, ``Génesis y crítica'', 203;
  véase Durand, \emph{Las estructuras}, 330-33.} Sin embargo, llega a la
actualidad a través de la concepción orgánica que desde Hipócrates,
pasando por Galeno y Harvey, concibe el cuerpo humano poseyendo una
\emph{red} oculta de nervios y venas con \emph{flujos} invisibles y
cerebro como \emph{malla}.\footnote{Véase Robert Jastrow, \emph{El telar
  mágico. El cerebro humano y el ordenador}, trad. de Domingo Santos
  (Barcelona: Salvat, 1993).} Una analogía que será utilizada de manera
recurrente por René Descartes, Denis Diderot, Henri Saint-Simon, Herbert
Spencer e, incluso, en la primera cibernética y los ideólogos
contemporáneos.\footnote{Musso, ``Génesis y crítica'', 205.} Sin
embargo, Musso considera que ``el concepto moderno de red se forma en la
filosofía de Saint-Simon. Él produce la teoría de esa nueva visión
lógica y biopolítica de la red''.\footnote{Ibíd., 211.}

El concepto actual de \emph{red} con canales, conductos, cables y ondas
permite la conexión múltiple, permanente y conmutable. Ya la revolución
industrial permitió inventar las redes mecánicas, como el telégrafo o el
ferrocarril, y las transformaciones de las técnicas informáticas hacen
posible las redes autoorganizadas.

La red no tiene comienzo ni fin y con toda su mezcla ``se impone como
una \emph{tecnología del espíritu}''.\footnote{Sfez, \emph{Crítica},
  379; destacado en el original.} Esto se hace evidente, por un lado, en
el uso constante de la analogía entre la mente y el ordenador y, por
otro, en la calificación de ``inteligente'' referido a los aparatos y
sus capacidades. La red es una figura intermedia entre el árbol,
demasiado lineal, y el caos, el desorden. La tecnología libera de la
linealidad y al mismo tiempo nos impide caer en el desorden. Hoy el
imaginario de la red se convierte en una llave maestra universal. Sin
embargo, por evidente que pueda parecer, la metáfora de la \emph{red}
entendida en la materialidad tecnológica y en su lógica de
funcionamiento no es el principal punto de acceso al imaginario textil
de la comunicación.

\hypertarget{lo-digital-la-escritura-y-lo-textil}{%
\section{Lo digital, la escritura y lo
textil}\label{lo-digital-la-escritura-y-lo-textil}}

Interpretar el imaginario textil de la comunicación requiere, en primer
lugar, pensar la relación de lo textil con la escritura para ver el
trato que Occidente ha dado al tejido como algo tradicional, no
evolucionado o en todo caso como representación de un pasado dentro de
la evolución de la comunicación humana. Incluso, en segundo lugar,
parece olvidarse que la revolución industrial implicó a los talleres
artesanales y a los telares mecánicos. Telares que, por otra parte, no
solo legaron la imaginación de lo binario con su tarjeta perforada sino
que modificaron la organización de los artesanos y su percepción de las
imágenes en competencia con otras tecnologías visuales como la
fotografía.

\hypertarget{escritura-y-tejido-linealidad-y-teleologa}{%
\subsection{Escritura y tejido: linealidad y
teleología}\label{escritura-y-tejido-linealidad-y-teleologa}}

La historia de la comunicación parece estructurarse desde una teleología
centrada en las tecnologías dominantes. Se la cuenta como la narración
de una serie de inventos y aparatos que culmina con las últimas
tecnologías del siglo XXI, en el contexto de los países del Norte y
cuyos agentes principales son las empresas, los gobiernos y, junto a
ellos, algunos individuos-genios y profesionales-visionarios. En esta
narración no hay referencias al Sur del planeta, a los actores y
movimientos sociales que solo aparecen como contenidos de historias
``extranjeras'', como testimonios de ``lo extraño'', del ``otro'' pobre
y primitivo al que hay que ayudar y darle la palabra. En todo caso, como
lugar para la transferencia de la tecnología, oportunidad de negocio,
incluso como ayuda ``para el desarrollo''.

Todo ello permea el imaginario según el cual la comunicación camina
evolutivamente de unos estadios primitivos a las tecnologías
contemporáneas. En este punto no cuestionado, la oralidad primaria (la
anterior a la existencia de la escritura en el sentido de Walter
Ong\footnote{Walter Ong, \emph{Oralidad y escritura. Tecnologías de la
  palabra}, trad. de Angélica Scherp (México: Fondo de Cultura
  Económica, 2011): 18.}) sería un estadio primitivo mitopoético
superado por la escritura, en especial la fonética ---la griega---. La
oralidad sería la comunicación de sociedades tradicionales, míticas,
repetitivas y cíclicas, mientras que la escritura alfabética sería la
comunicación de la filosofía, la ciencia, el progreso lineal.

No es que no se haya realizado una crítica de ese evolucionismo que
lleva de lo oral a lo escrito\footnote{Jacques Derrida, \emph{De la
  gramatología}, trad. de Oscar del Barco y Conrado Ceretti (México:
  Fondo de Cultura Económica, 2017).} y de las técnicas elementales a
las tecnologías digitales. El problema consiste en que sigue vigente en
los discursos periodísticos y en circuitos públicos formando parte de un
sentido común. Ello supone que la comunicación aparece como el producto
de una evolución de lo oral a la escritura alfabética, es decir, la
escritura ``completa'' en tanto representa la palabra hablada (con los
antecedentes de la escritura logosilábica y la silábica). Antes de ello
estaría la ``no escritura'' como las pinturas en las que los dibujos
primitivos expresan signos, pero no formas lingüísticas. Estas
concepciones se aplican aún hoy a las poblaciones americanas
descendientes de los pueblos precolombinos a las que se tacharía de
sociedades primitivas, ``ágrafas'', porque no tuvieron escritura
alfabética tal como la tenía la sociedad europea, pero como dice Luis
Ramiro Beltrán, la ``mayoría de las culturas nativas precolombinas no
eran ni primitivos ni ágrafas''.\footnote{Luis Ramiro Beltrán,
  introducción a \emph{La comunicación antes de Colón. Tipos y formas en
  Mesoamérica y los Andes}, de Luis R. Beltrán et al. (La Paz: CIBEC,
  2008): I-XII.}

La escritura alfabética se convirtió en el signo y símbolo de la
comunicación y el sentido, obviando ---o rebajando a débiles copia de
ella--- todas las otras prácticas culturales que expresan el sentido, lo
construyen y lo comunican. Así en el mundo precolombino se encuentran
distintos ``rasgos expresivos que estructuran una manifestación
simbólica sobre la base de la utilización y la organización de ciertos
elementos''\footnote{Luis Ramiro Beltrán et al., \emph{La comunicación},
  20.} como, por ejemplo, danza, música, poemas, cantos, himnos, plazas,
edificios, templos, vestimenta, orfebrería, dibujos, entre
otros.\footnote{Ibíd., 21.} La cultura es la producción, circulación y
consumo de significaciones y la comunicación humana es esencialmente
simbolización y relación/interacción de sentidos entre sujetos. Por
ello, comunicación y cultura no pueden estar separados como si fueran
``distintos'' ``objetos de estudio'' que no tuvieran
relación.\footnote{Héctor Schmucler, ``La investigación: un proyecto
  comunicación/cultura'', en \emph{Memoria de la Comunicación} (Buenos
  Aires: Biblos, 1997): 145-51.} La mirada a todas esas prácticas que
han enriquecido, por ejemplo, el legado de los pueblos
sudamericanos\footnote{Luis Ramiro Beltrán et al., \emph{La
  comunicación}, 47-73.} ---que se encuentran hoy en casi todos los
países latinoamericanos--- son similares a las que pueden hacerse de
muchos otros pueblos. Mirada que cuestiona la presentación histórica de
la comunicación\footnote{Como se hace, por ejemplo, en Francisco Sierra
  Caballero y Claudio Maldonado Rivera, coords., \emph{Comunicación,
  decolonialidad y Buen Vivir} (Quito: CIESPAL, 2016); Claudia
  Magallanes Blanco y José Manuel Ramos Rodríguez, coords.,
  \emph{Miradas propias. Pueblos indígenas, comunicación y medios en la
  sociedad global} (Quito: CIESPAL, 2016), o las intervenciones de
  Alejandro Barranquero y Juan Ramos Martín, ``Luis Ramiro Beltrán and
  Theorizing Horizontal and Decolonial Communication'', en \emph{The
  Handbook of Global Interventions in Communication Theory}, ed. de
  Yoshitaka Miike y Jing Yin (Nueva York: Routledge, 2022); Raúl Fuentes
  Navarro, ``Latin American Interventions to the Practice and Theory of
  Communication and Social Development: On the Legacy of Juan
  Díaz-Bordenave'', en \emph{The Handbook of Global Interventions in Communication Theory}, ed. de Yoshitaka Miike y Jing Yin (London: Routledge, 2022); Eva González
  Tanco y Carlos Arcila Calderón, ``Buen Vivir as a Critique of
  Communication for Development'', en \emph{The Handbook of Global Interventions in Communication Theory}, ed. de Yoshitaka Miike y Jing Yin (London: Routledge, 2022).}
y que no permite conceptualizarla de otro modo distinto a la centrada en
los medios técnicos. Entre todas esas prácticas tan diferentes de
comunicación/cultura se destaca la textil. La práctica textil como
práctica cultural tiene una dimensión alternativa a la comunicación
centrada en la escritura alfabética. Rescatar lo textil en el sustrato
de las explicaciones de la comunicación llevaría a reinterpretar el
fenómeno comunicacional desde ese otro lugar tan presente, por ejemplo,
en el textil andino\footnote{Luis Ramiro Beltrán et al., \emph{La
  comunicación,} 225-51.} donde lo técnico aparece de otro modo.

El imaginario comunicacional de la oralidad y la escritura en su
logocentrismo tuvo su afinidad electiva con el mediacentrismo. El
inconsciente de semejante relación se convirtió en la investigación
sobre la comunicación como consecuencia de sus instrumentos técnicos.
Fotografía, prensa, cine, radio, luego televisión, computadoras,
celulares e internet han permitido imaginar la centralidad de una
concepción más o menos evolutiva de la tecnología en la reflexión
medialógica, dando pasos a lecturas lineales deterministas y
teleológicas.

Más allá (o más acá) de la evolución de lo mítico a lo lógico o del
logocentrismo al mediacentrismo, puede que comunicación sea, también,
otra cosa y que descubrirlo pueda depender de la consideración de lo
textil como tramado de técnica y escritura, porque el tejido sigue
hablando de lo que es la comunicación. Aquí se intenta mostrar que lo
textil constituye un imaginario de la comunicación que ha permanecido
subyacente, pero vivo y actuante en el substrato cultural. Un
conocimiento naturalizado que se manifiesta en la atmósfera cultural
cuando se habla de comunicación.\footnote{Daniel H. Cabrera Altieri,
  ``Exploraciones sobre el significado de la técnica y la escritura'',
  en \emph{Cosas confusas. Comprender las tecnologías y la
  comunicación}, coord. de Daniel H. Cabrera Altieri (Valencia: Tirant
  lo Blanch, 2019): 13-29, y Daniel H. Cabrera Altieri, ``Lo textil como
  vía para repensar la comunicación/tecnología'', en Cabrera Altieri,
  \emph{Cosas confusas,} 35-48.}

\hypertarget{capitalismo-y-telares-mecnicos}{%
\subsection{Capitalismo y telares
mecánicos}\label{capitalismo-y-telares-mecnicos}}

La linealidad de la interpretación de los ``medios de comunicación''
suele dejar de lado una tecnología fundamental para el capitalismo
occidental como son los telares automatizados. Los telares mecánicos
fueron los instrumentos de la revolución industrial. Joseph-Marie
Jacquard (1752-1834) dio nombre al telar más famoso con el que el
capitalismo adquiere su perfil definitivo. Se empleaba ``para tejidos
con grandes dibujos, en los cuales todos o la mayoría de los hilos del
dibujo suben o bajan independientemente unos de otros; de esta manera
pueden fácilmente reproducirse en el tejido líneas y figuras de todas
clases''.\footnote{Thomas W. Fox, \emph{Maquinaria de tejidos}, trad. de
  Francisco Madurga (Barcelona: Bosch, 1919): 440.}

Contrariamente a lo que pueda sugerir su nombre la maquina automática de
tejer fue resultado de más de un siglo de diversos inventos y
construcciones.

Contemporáneo a las máquinas automáticas de tejer, Charles Babbage,
matemático inglés, diseñó una ``máquina diferencial'' para construir
tablas matemáticas y en 1834 propuso una ``máquina analítica'' para
realizar cálculos numéricos muy diversos. Sin embargo, señalaba
irónicamente que aquella máquina ``sería capaz de hacer cualquier cosa
menos componer piezas folclóricas''.\footnote{Huskey citado en Martin
  Davis, \emph{La computadora universal. De Leibniz a Turing}, trad. de
  Ricardo García Pérez (Barcelona: Debate, 2002): 165.} Aunque el
concepto de la máquina estaba aún alejado de una computadora de
propósito general como las actuales, planteaba la idea de un mecanismo
en relación con una lógica de funcionamiento. En este punto destaca Ada
Lovelace quien, como ayudante de Babbage y entusiasta de su máquina, fue
la que relacionó la máquina analítica con el telar de Jacquard:
``podemos decir sin temor a equivocarnos que la Máquina Analítica teje
modelos algebraicos exactamente igual que el telar de Jacquard teje
flores y hojas''.\footnote{Goldstine citado en Davis, \emph{La
  computadora}, 202.} Aunque también se sabe que el propio Babbage, que
tenía un telar de Jacquard, se propuso utilizar tarjetas perforadas como
las del telar automático en la máquina analítica que había diseñado.

Esas tarjetas usadas también por pianolas y máquinas censales de
comienzo del siglo XX se convirtieron en la primera memoria del mundo
digital. Las tarjetas perforadas se mostraron rápidamente como soportes
de un lenguaje binario que debía ser almacenado. Este primer disco duro
hacía del agujero ``la otra opción'' a la superficie y al espacio.

Aunque los telares mecánicos y su influencia son conocidos en la
historia de la computación, las historias de las técnicas
comunicacionales no los tienen en cuenta. Entre otros motivos, por la
incapacidad de ver el imaginario común que relaciona telares con medios
de comunicación, y telas con comunicación.

El funcionamiento técnico del telar y del hacer humano de los talleres
textiles permitieron una mayor productividad en la elaboración
industrial automatizada de tejidos, pero un importante problema fue la
separación entre el constructor de telares y el artesano de tejidos,
porque ``el constructor está falto de experiencia en el uso del telar y
el tejedor no posee conocimiento de mecánica suficiente para desarrollar
las ideas que durante la práctica se le ocurren''.\footnote{Fox,
  \emph{Maquinaria de tejidos}, 1.} En el caso de los tejedores de Lyon,
después de una etapa de rechazo, generaron nuevos modos de organización
en el uso de los telares que llevaron a aumentar la imaginación humana y
esforzarse por acrecentar la complejidad de las telas en su búsqueda de
hacer que los textiles compitieran con los medios dominantes de la época
(grabado, impresión, pintura) y con el nuevo medio de la fotografía. El
telar mecánico les permitió una nueva forma de imagen digital basada en
el textil.\footnote{Ganaele Langlois, ``Distributed Intelligence: Silk
  Weaving and the Jacquard Mechanism'', \emph{Canadian Journal of
  Communication} 44, n.º 4 (2019): 555--66,
  \url{https://doi.org/10.22230/cjc.2019v44n4a3723} . Véase Sadie Plant,
  \emph{Ceros + unos. Mujeres digitales + la nueva cultura}, trad. de
  Eduardo Urios (Barcelona: Destino, 1998).}

La comprensión de la relación entre el telar mecánico, es decir, del
aparato con su complejidad, la organización de los artesanos en su uso,
la producción de telas, incluso su comercialización e influencia en la
economía, plantea un problema común a la historia de las tecnologías y
la historia de los medios de comunicación. El primero es el determinismo
tecnológico que, en el caso de los medios, se aplica a la transformación
de la atención, la memoria, el conocimiento, la formación de la opinión
pública, los cambios de escala espacios-temporales. En estos
planteamientos, como los de Marshall McLuhan,\footnote{Marshall McLuhan,
  \emph{La galaxia Gutenberg. Génesis del ``Homo typographicus''}, trad.
  de Juan Novella. (Barcelona: Círculo de Lectores, 1993).} el problema
consiste en la teleología de los cambios que se concentra en los medios
y tecnologías que triunfan, subsumiendo la explicación a una linealidad
evolutiva, un progresismo que se centra en los medios poderosos y en las
relaciones de poder en las que participan. La historia de los medios de
comunicación aparece así integrada en un movimiento aparentemente
inconsciente que se mueve por la fuerza de su argumentación y de su
eficiencia y eficacia. El caso del telar de Jacquard como una tecnología
compleja de colaboración hombre-máquina es un ejemplo de un sistema de
medios digitales aún no dominante, que fomenta una inmensa creatividad.
Tal como lo ha analizado Ganaele Langlois, puede ser un ejemplo, que
lleva a pensar en las oportunidades para explorar el potencial de los
sistemas mediáticos no dominantes y que pasan desapercibidos por el
mediacentrismo.

\hypertarget{el-tejido-sus-metforas-e-imaginarios}{%
\section{El tejido, sus metáforas e
imaginarios}\label{el-tejido-sus-metforas-e-imaginarios}}

Adentrarse a las consideraciones de los tejidos implica constatar un
olvido de lo textil como comunicación, como producción de sentido. Una
especie de amnesia que tiene instrumentos de cura cuando se investiga su
omnipresencia en el vocabulario y en expresiones de la vida cotidiana
para, de esta manera, transparentar el antiguo parentesco entre el
tejido y la comunicación humana. Se trata de la posibilidad de centrar
la reflexión de la comunicación desde las prácticas subalternas, desde
las artesanías antes que en las tecnologías, desde las mujeres antes que
en los hombres, desde el Sur antes que en Occidente euroamericano, desde
la palabra y la comunidad antes que desde las tecnologías y las
empresas.

\hypertarget{amnesia-textil}{%
\subsection{¿Amnesia
textil?}\label{amnesia-textil}}

\begin{quote}
\ldots el despertar de recuerdos dormidos que causa la analogía parece
encontrarse tan próximo a la esencia de lo que significa ser humano que
es difícil imaginar cómo sería la vida mental en caso de no existir.

---Douglas Hofstadter y Emmanuel Sander, \emph{La analogía. El motor del
pensamiento}.
\end{quote}

Lo textil es una actividad y un imaginario que acompaña al ser humano
desde el Neolítico a la actualidad. Un pasado que se hace presente en
hallazgos arqueológicos constantes de instrumentos de tejido (husos,
telares) y, sobre todo, en el vocabulario de la vida cotidiana que
relaciona diversas acciones humanas con el mundo del hilo, la trama, la
urdimbre. El antropólogo Tim Ingold defiende la idea de que hilar,
trenzar y tejer están entre las artes humanas más arcaicas y afirma que
``la fabricación y uso de hilos puede ser un indicativo claro de la
aparición de las formas de vida característicamente
humanas''.\footnote{Tim Ingold, \emph{Líneas. Una breve historia}, trad.
  de Carlos García Simón (Barcelona: Gedisa, 2015): 69. Véase Tim
  Ingold, ``The textility of making'', \emph{Cambridge Journal of
  Economics} 34, n.º 1 (2010): 91-102, doi:10.1093/cje/bep042.}

A pesar de su importancia antropológica el tejido goza de un raro
privilegio, la invisibilidad de su presencia y la irrelevancia de su
consideración. Al punto que se ha ``pasado por alto el papel central de
los textiles en la historia de la tecnología, el comercio y la
civilización propiamente dicha''.\footnote{Virginia Postrel, \emph{El
  tejido de la civilización. Cómo los textiles dieron forma al mundo},
  trad. de Lorenzo Luengo (Madrid: Siruela, 2020): 12.} Por ello,
Virginia Postrel asegura que la sociedad actual padece una ``amnesia
textil''.\footnote{Ibíd., 286.} Dada la abundancia de textiles no hay
conciencia de lo entrelazada que está la vida humana, en sus diferentes
dimensiones, con el tejido.

\enlargethispage{-\baselineskip}

La producción textil está reflejada en los sistemas más antiguos de
escritura. Telas, husos y telares, entre otros retratos del mundo
textil, se representan en la escritura cuneiforme de Mesopotamia, en la
jeroglífica de Egipto y en la lineal del Egeo, haciendo referencia a un
universo conceptual cercano al actual.\footnote{Agnès García-Ventura,
  ``Imágenes del universo textil en las primeras escrituras'',
  \emph{Datatèxtil}, n.º 14 (2006): 20-31,
  \url{https://raco.cat/index.php/Datatextil/article/view/278625}.}

El textil es un fenómeno común a la humanidad, desde entrelazados y
nudos simples de hilos vegetales y animales a complejos trabajos en
múltiples culturas y lugares del mundo. La historia del tejido y su
evolución histórica se basa en tres hechos: el descubrimiento del
entrecruzamiento con materiales flexibles; su elaboración para conseguir
un hilo, y los instrumentos necesarios para sostener los hilos con la
tensión necesaria.\footnote{Rita Barendse y Antonio Lobera, \emph{Manual
  de artesanía textil} (Barcelona: Alta Fulla, 1987): 9.}

Los antiguos manuales industriales europeos definen tejer en relación a
la urdimbre y la trama como ``entrelazar una serie de hilos colocados en
el sentido de la longitud de la tela con otra serie de hilos colocados
transversalmente en el sentido del ancho''.\footnote{Fox,
  \emph{Maquinaria de tejidos,} 5.} Y dividen la industria textil en
tres partes: hilatura (``operaciones necesarias para obtener las
fibras...''), tejido (``fabricación de tejidos'') y el teñido y aprestos
(operaciones necesarias de hermoseado y acabado de los
tejidos'').\footnote{Max Gürtler y W. Kind, \emph{La industria textil},
  trad. de Ricardo Ferrer (Barcelona: Labor, 1947): 17.}

Desde el punto de vista artesanal andino, se ha definido tejido como
``entrecruzamiento de un sistema de hilos llamado urdimbre por un
sistema de hilos llamados tramas, cuyo rasgo constante es la formación
del paso o calada (espacio que se forma entre los hilos de urdimbre para
el pasaje de la trama). Los restantes rasgos pueden poseer siempre una
excepción''.\footnote{Clara M. Abal de Russo, \emph{Arte textil incaico
  en ofrendatorios de la alta cordillera andina}. \emph{Aconcagua,
  Llullaillaco, Chuscha} (Buenos Aires: CEPPA, 2010): 50.}

El entrecruzamiento y las etapas de la actividad textil provienen de
antiguo\footnote{Véase Anni Albers, \emph{On Weaving} (Nueva Jersey:
  Princeton University Press, 2003).} y permanecen vivas en diversos
lugares, entre los que destaca, de una manera especial la región andina
latinoamericana donde el tejido ocupa un lugar central en muchas
comunidades.\footnote{Eva Fischer, \emph{Urdiendo el tejido social.
  Sociedad y producción textil en los Andes bolivianos}, trad. de Eva
  Fischer (Berlín: Lit Verlag, 2008), 26.} La investigación de la región
andina obliga a reconsiderar la comunicación desde una antropología
precolombina aún viva en sus mestizajes e hibridaciones.

\hypertarget{el-lenguaje-del-tejido}{%
\subsection{El lenguaje del
tejido}\label{el-lenguaje-del-tejido}}

Las lenguas contemporáneas hacen referencia al antiguo mundo textil en
contextos muy diversos, aunque con especial dedicación a la expresión y
la comprensión de la comunicación humana. Las etimologías y la
fraseología brindan múltiples ejemplos.

Etimologías\footnote{Se han utilizado los siguientes recursos:
  \emph{Diccionario etimológico castellano en línea}, última
  modificación el 18 de mayo de 2022,
  \url{http://etimologias.dechile.net/}; \emph{Online Etymology
  Dictionary}, acceso el 20 de marzo de 2022,
  \url{https://www.etymonline.com/}; Julius Pokorny,
  \emph{Indogermanisches Etymologisches Wörterbuch}, base de datos,
  acceso el 20 de marzo de 2022,
  \url{https://indo-european.info/pokorny-etymological-dictionary/}\href{https://indo-european.info/pokorny-etymological-dictionary/;https://www.perseus.tufts.edu/hopper/}{;
  Perseus} Digital Library
  (\href{https://indo-european.info/pokorny-etymological-dictionary/;https://www.perseus.tufts.edu/hopper/}{Perseus
  Hopper)}, acceso el 20 de marzo de 2022,
  \url{https://www.perseus.tufts.edu/hopper/}, y Edward A. Roberts,
  \emph{Diccionario etimológico indoeuropeo de la lengua española}
  (Madrid: Alianza, 1996).} de las lenguas occidentales\footnote{Aunque
  las expresiones referidas corresponden a la lengua española existen
  equivalentes en otras lenguas occidentales contemporáneas.} recuerdan
la íntima relación entre ``textil'', ``texto'' y ``técnica''. Todas
derivadas de la raíz indoeuropea \emph{teks}- (significado de tejer,
fabricar, asemblar, carpintería) de donde deriva el griego \emph{tekton}
(significado de estructura, construcción, obra y, en castellano,
tectónico, arquitecto) y \emph{t\'echne} (significado de técnica, arte y,
en castellano, técnica, tecnócrata, tecnología) y a través del latín
\emph{tela} (llega: tela, telar, sutil); \emph{texere} (de donde tejer,
trenzar, entrelazar), \emph{textus} (tejido participio de \emph{texere},
de donde texto, pretexto, hipertexto).

Hay también un parentesco entre lino y línea. La raíz indoeuropea
\emph{*li-no-} a través del griego \emph{linon} y del latin \emph{linum}
ha dado en castellano, entre otras, lino, línea y \emph{online}. Y la presencia
de hilo desde la raíz indoeuropea \emph{*gwhi-} con significado hilo y
filamento y a través del latín \emph{filum} (filo, línea de un contorno,
hilo) pasó al castellano \emph{filo} (y filamento, filar, filete,
filigrana), \emph{fila} (y desfilar, enfilar), \emph{hilo} (hilar,
hilván), perfil y perfilar, vilo, entre otras.

Múltiples expresiones en español, con sus correlativos en lenguas
romances, se refieren a diversas dimensiones del textil, el tejido y el
tejer. Así se encuentran, por ejemplo:

\begin{itemize}
\item
  \emph{trama} como artificio o confabulación contra alguien; enredo de
  una obra dramática o comedia, ``la trama del relato''
\item
  \emph{urdir} como, por ejemplo, maquinar algo contra alguien
\item
  el uso de \emph{hilo} en expresiones como ``tirar del hilo'', ``el
  hilo de la cuestión'', ``no perder el hilo'', ``el hilo de la vida'',
  ``no dar puntada sin hilo'', además ``el hilo'', según se dice, se
  puede seguir, perder, cortar, retomar
\item
  dichos referidos al \emph{nudo} como ``nudo del problema'', ``nudo
  gordiano''
\item
  \emph{hilar} en ``hilar fino''
\item
  \emph{tejer} como ``entretejer'' temas, cuestiones, ideas, conceptos;
  discurrir, idear un plan
\item
  referencias a la \emph{textura} como disposición o estructura de una
  obra, un cuerpo
\item
  \emph{aguja}, ``encontrar la aguja en el pajar''
\item
  a la \emph{rueca,} como algo que se tuerce
\end{itemize}

Esta lista de expresiones y usos del lenguaje referidos al mundo textil
testimonia un imaginario del sentido de la vida, del hacer y hablar,
como también lo manifiesta la expresión inglesa \emph{spinning yarns}
como ``contar historias'' y antes, \emph{rapsodia}, como ``coser
canciones e historias''.

Una dimensión reciente de la investigación textil consiste en explorar
el papel de la tecnología textil en los universos mentales del pasado,
en el culto, los rituales, la mitología, las metáforas, la retórica
política, la poesía y el lenguaje de las ciencias.\footnote{Salvatore
  Gaspa, Cécile Michel y Marie-Louise Nosch, eds., \emph{Textile
  Terminologies from the Orient to the Mediterranean and Europe, 1000 BC
  to 1000 AD} (Nebraska: Zea Books Lincoln, 2017).} Las investigaciones
de este tipo concluyen que las expresiones textiles metafóricas y
figurativas no son meras herramientas estilísticas sino que están
arraigadas en realidades cognitivas, terminológicas y vivenciales del
pasado y que persisten hasta la actualidad en el lenguaje.

Oswald Panagl considera que una prueba de ello se encuentra en el
vocabulario inglés relacionado con \emph{weaving, spinning, net} que se
consolida en términos y expresiones técnicas del léxico de los medios
electrónicos como\emph{,} por ejemplo, \emph{web address, on the web,
web based, web browser, web designer, webcast, web forum, webhead,
webmaster, web page, web-site; spin doctor; network, internet, net
speak}.\footnote{Panagl Oswald, ``Der Text als Gewebe: Lexikalische
  Studien mi Sinnbezirk von Webstuhl und Kleid'', en \emph{Textile
  Terminologies from the Orient to the Mediterranean and Europe, 1000 BC
  to 1000 AD}, ed. de Salvatore Gaspa, Cécile Michel y Marie-Louise
  Nosch (Nebraska: Zea Books Lincoln, 2017): 419. Hay que señalar que
  desde el inglés pasó a las lenguas de un mundo ``globalizado''.} El
investigador analiza el campo semántico del tejido para sostener que no
se ha convertido en una metáfora muerta sino que ha seguido siendo
productiva desde la antigüedad hasta nuestros días.

Seguir las expresiones y palabras del vocabulario actual referidas al
mundo textil lleva a hacer visible una presencia silenciosa a la vez que
activa. Las metáforas muestran el camino hacia el imaginario de la
comunicación humana. Un imaginario que tiene como \emph{hummus} la
textilidad, el entrelazado creador de manos habilidosas, y que se
refiere a la vida y su cuidado junto al hablar, el cantar, el texto, el
escribir, el relacionarse con otros. Increíblemente esas palabras que
tomaron cuerpo y significado desde un conjunto de prácticas habituales
en el pasado permanecen vivas, pero tan escondidas y desconectadas de su
núcleo de sentido hasta el punto de no llamar la atención en su
capacidad explicativa ni en su fuerza heurística.

\hypertarget{el-caso-griego-imaginario-textil-entre-retrica-logos-y-mito}{%
\subsection{El caso griego: imaginario textil
entre retórica, logos y
mito}\label{el-caso-griego-imaginario-textil-entre-retrica-logos-y-mito}}

Los pueblos antiguos, como se ha señalado, ofrecen múltiples
experiencias de la práctica del tejido. Aquí mencionaremos, por su
influencia evidente en la cultura y las lenguas europeas, el vocabulario
y las expresiones textiles moldeadas en la antigua Grecia y que nos han
llegado a través de los mitos, la filosofía y el arte. El recurso a
Grecia conlleva la intención de llamar la atención sobre una historia y
una cultura explorada hasta, en algunos sentidos, volverlas sentido
común occidental. Reconsiderar este sentido ayuda a entender los
silencios y ausencias de los que está hecha la memoria subalterna,
popular y femenina.

Las referencias de la lengua de la Grecia clásica,\footnote{Cornelius
  Castoriadis, ``Notas sobre algunos medios de la poesía'', en
  \emph{Figuras de lo pensable} (Valencia: Cátedra, 1999): 36-61. El
  autor destaca la ``polisemia indivisible de las palabras y de los
  casos gramaticales'' (p. 36) de la lengua griega clásica por la que a
  veces resulta imposible traducir una palabra con un sentido único.} de
sus mitos y su filosofía partían de un contexto de la vida cotidiana y
la organización social donde lo textil ocupaba un lugar fundamental. La
labor,\footnote{Hannah Arendt, \emph{La condición humana}, trad. de
  Ramón Gil Novales (Barcelona: Paidós, 2009): 37-95.} como la define
Hannah Arendt, es una condición de posibilidad de la \emph{polis}, de la
``acción'', pero que permanece en la oscuridad del \emph{oikos}, lugar
de la desigualdad y de la necesidad gobernado por las mujeres. Dentro
del \emph{oikos, el gineceo} era un espacio arquitectónico y simbólico
del tejer y del tejido. Allí al ritmo de un telar vertical cuyas
urdimbres la asemejaban a una lira, se expresaba la habilidad
de la tejedora tanto con fuerza como con habilidad. Mientras se tejía se
contaban historias y mitos, y se aprendían traiciones y valores, que
fundaban la mentalidad de la sociedad griega.

En ese mundo, el tejido, su hacer, sus herramientas, sus insumos, eran
una referencia para pensar el sentido de la vida humana. El trenzar de
los hilos y los ritmos del telar creaban las telas con las que se hacían
las vestimentas y se concebían los sentidos de las historias y de la
vida humana que allí se abrigaba. Era ``coser y cantar'' según la
célebre expresión popular que coincide con la idea de que la lanzadera
es ``amiga de las canciones''.\footnote{Diana Segarra Crespo, ``Coser y
  cantar: a propósito del tejido y la palabra en la cultura clásica'',
  en \emph{Tejer y vestir. De la Antigüedad al Islam}, ed. de Manuela
  Marín (Madrid: CSIC, 2001): 200.} Circe, las Moiras y las Parcas
también hilaban y cantaban,\footnote{Ibíd., 201.} porque en la sociedad
clásica el modelo sagrado no solo era textil sino también lingüístico.
En ella las jóvenes narraban y fijaban en soporte textil, que era el
modo femenino.\footnote{Ibíd., 217.}

Se ha analizado el imaginario textil griego en relación con la
comunicación\footnote{Daniel H. Cabrera Altieri, ``El imaginario textil
  griego y la comunicación'', \emph{RAE-IC, Revista de la Asociación
  Española de Investigación de la Comunicación} 1, n.º 2 (2014): 65-73.}
destacando que la retórica era la teoría de la comunicación del
\emph{ágora}, pública y masculina, mientras que la actividad del tejido
propia del \emph{oikos}, doméstico y femenino, solo tiene referentes en
los mitos, son ``historias de mujeres''. Situación que culminará en la
época Clásica en la oposición entre \emph{logos} y \emph{mitos}, la
explicación y demostración racional frente a la narración, el relato,
las historias.\footnote{Hans Georg Gadamer, \emph{Mito y razón}, trad.
  de José Francisco Zúñiga García (Paidós: Barcelona, 1997): 25.}

Aristófanes en su comedia \emph{Lisístrata} (411 a.C.) y Platón en
\emph{El político} (367-361 a.C.) son dos casos donde lo femenino cruza
la frontera hacia la política. \emph{Lisístrata,} mujer, toma la palabra
para aconsejar cómo resolver los problemas de la falta de acuerdo de los
ciudadanos, ``los políticos'', y toda su recomendación consiste en una
aplicación de las tareas textiles

\begin{quote}
LISÍSTRATA.- Ante todo, como se hace con los vellones, habría que
desprender de la ciudad en un baño de agua toda la porquería que tiene
agarrada, quitar los nudos y eliminar a los malvados, vareándolos sobre
un lecho de tablas, y a los que aún se quedan pegados y se apretujan
para conseguir cargos arrancarlos con el cardador y cortarles la cabeza;
cardar después en un canastillo la buena voluntad común, mezclando a
todos los que la tienen sin excluir a los metecos y extranjeros que nos
quieren bien y mezclar también allí a los que tienen deudas con el
tesoro público y además, por Zeus, todas las ciudades que cuentan con
colonos salidos de esta tierra, comprendiendo que todas ellas son para
nosotros como mechones de lana esparcidos por el suelo cada cual por su
lado. Y luego, cogiendo de todos ellos un hilo, reunirlos y juntarlos
aquí y hacer con ellos un ovillo enorme y tejer de él un manto para el
pueblo.\footnote{Aristófanes, \emph{Lisístrata}, trad. de Luis M. Macía
  Aparicio (Madrid: Ediciones Clásicas, 1993), 575--85.}
\end{quote}

Limpiar, mezclar, reunir, juntar, tejer... crear lo nuevo, lo que
protege: ``si tuvieras una pizca de sentido común, según nuestras lanas
gobernaríais todo'' sin necesidad de guerras ni competencias estériles
entre las ciudades griegas.

Platón en \emph{El político} pone en boca de un extranjero una reflexión
sobre el proteger y el cuidado desde el campo semántico de
\emph{tekton,} \emph{t\'echne} y \emph{texere} que lo llevan a concluir
que ``a estas defensas y ropas que se confeccionan entretejiendo sus
mismos hilos les damos el nombre de vestidos; e igual que entonces
llamamos `política' al arte de cuidar la `polis'\,''. La política se
presenta como cuidado, protección, cubierta/cobertura,
arte/técnica.\footnote{Dimitri El Murr, ``La Symploke Politike: Le
  paradigma du tissage dans le Politique de Platon, ou les raisons d'un
  paradigma Arbitraire'', \emph{Kairos}, n.º 19 (2002): 49-95. Véase
  Cornelius Castoriadis, \emph{Sobre} El Político \emph{de Platón}
  (México: Fondo de Cultura Económica, 2002): 121-41.} En la Grecia
clásica el tejer podía postularse como un modelo de organización
política, pero en una comedia o en boca de un extranjero y, en ambos
casos, era una manera de sostener algo que parece impropio.

En la cultura europea la asociación del tejer con la mujer ha
permanecido hasta hoy aunque la realidad sea más compleja. Thomas
Blisniewski, hablando de las labores del tejido, tal como se representa
en el arte europeo y analizado desde el punto de vista
evolutivo,\footnote{Thomas Blisniewski, \emph{Las mujeres que no pierden
  el hilo. Retratos de mujeres que hilan, tejen y cosen de Rubens a
  Hopper} (Madrid: Maeva, 2009): 125-48.} constata el legado judío y el
grecorromano que, a través de la tradición cristiana, ha llevado a la
Europa occidental a considerar que una mujer virtuosa se relaciona
íntimamente con las labores textiles. La representación artística
occidental incide una y otra vez sobre la tradicional asociación de la
mujer con las artes de tejer, sugiriendo pasividad y dependencia
femenina. No obstante, también podría considerarse, como se ha hecho, el
tejer no solo como ``símbolo de sumisión doméstica'' sino, también,
``como una industria productiva'' y tal vez, a consecuencia de ello,
``signo de su virtud femenina''.\footnote{Postrel, \emph{El tejido}, 59.}

Resulta curioso que, históricamente, lo femenino parece encontrar
sujeción en lo textil, mientras lo textil no necesariamente se resuelve
en el mundo de la mujer. Más aún, la revolución industrial muestra cómo
la conversión en industria parece liberar lo textil del mundo femenino
para transformarse en tecnología patriarcal y economía productiva del
capitalismo. Lo que no sucede en Latinoamérica donde lo textil, corazón
de la cultura y del sentido, sigue participando de lo que pueda
entenderse por comunicación en su relación con los otros, con la tierra,
con la tradición y la religiosidad.\footnote{Véase Adalid Contreras
  Baspineiro, ``Aruskipasipxañanakasakipunirakispawa'', en Sierra
  Caballero y Maldonado Rivera, \emph{Comunicación}, 59-93.}

\hypertarget{para-una-teora-alternativa-de-la-comunicacin}{%
\section{Para una teoría alternativa de la
comunicación}\label{para-una-teora-alternativa-de-la-comunicacin}}

Lo dicho hasta aquí justifica postular una teoría alternativa del
fenómeno comunicativo centrada en la metáfora del textil y el imaginario
del tejido como alter/nativa al mediacentrismo. En su búsqueda se pueden
pensar tres aspectos a tener en cuenta para una conceptualización de la
comunicación: como entrecruzado textil creador, como cuidado de la vida
y como el tejer del tejido social.

\hypertarget{la-comunicacin-entrecruzado-textil-creador}{%
\subsection{La comunicación: entrecruzado
textil
creador}\label{la-comunicacin-entrecruzado-textil-creador}}

El antropólogo Cliffort Geertz afirma, siguiendo a Max Weber, que ``el
hombre es un animal inserto en tramas de significación que él mismo ha
tejido'' y continúa ``considero que la cultura es esa urdimbre y que el
análisis de la cultura ha de ser por lo tanto, no una ciencia
experimental en busca de leyes, sino una ciencia interpretativa en busca
de significaciones''.\footnote{Clifford Geertz, \emph{La interpretación
  de la cultura,} trad. de Alberto L. Bixio (Barcelona: Gedisa, 2003):
  20.}

Siguiendo estos lineamientos John B. Thompson en su estudio sobre la
Modernidad y los medios de comunicación afirma que ``los medios de
comunicación constituyen las ruecas del mundo moderno y, al utilizar
estos media, los seres humanos se convierten en fabricantes de tramas de
significado para consumo propio''.\footnote{John B. Thompson, \emph{Los
  media y la modernidad: una teoría de los medios de comunicación}
  (Barcelona: Paidós, 1998): 26.}

Estos autores utilizan la metáfora textil como un recurso narrativo,
pero puede usarse como instrumento heurístico. Vilém Flusser,
brevemente, hace esta interpretación subrayando el carácter artificial
de la comunicación cuyo objetivo es ``hacernos olvidar el contexto falto
de significación en el que nos hallamos por completo solos e
incomunicados'' y continua:

\begin{quote}
la comunicación humana teje un velo del mundo codificado, un velo de
arte y de ciencia, de filosofía y de religión en torno a nosotros y lo
teje cada vez más denso, para que nos olvidemos de nuestra propia
soledad y de nuestra muerte, y también de la muerte de aquellos a
quienes queremos... La teoría de la comunicación se ocupa del tejido
artificial que hace que nos olvidemos de la soledad.\footnote{Vilém
  Flusser, ``¿Qué es la comunicación?'', trad. de Victor Silva Echeto,
  en Cabrera Altieri, \emph{Cosas confusas}, 284.}
\end{quote}

En varias culturas, el tejido se metaforiza para comprender la vida y su
sentido. El laborioso trabajo de fabricar hilos, desde vegetales,
animales o insectos, para luego entrelazarlos en el telar para producir
tela. El ser humano sale del vientre unido por un cordón y, a falta de
pelos, plumas o gruesa piel, debe ser cobijado y abrigado. En América
del Sur se considera al tejido como una prolongación del vientre
materno.\footnote{Véase Fischer, \emph{Urdiendo el tejido social,} 250.}
La cultura con sus símbolos, mitos, narraciones, arte, imágenes, textos,
entre otros, se construye y funciona como el vientre materno creado por
el tejido de la comunicación.

A lo largo de la vida de los individuos y de la historia de la sociedad
se construyen redes de significaciones que dan sentido a las cuestiones
más diversas, comenzando por sus sensaciones, deseos, miedos,
inquietudes y esperanzas, así también como respuesta a amenazas,
oportunidades, competencias o para justificar acciones. La cultura es la
tela producida para cobijarse y protegerse ante la intemperie del
sin-sentido, mientras que la comunicación es la acción por la que se la
teje. La comunicación acción generativa, ``entrelazado
perpetuo'',\footnote{Roland Barthes, \emph{El placer del texto}, trad.
  de Nicolás Rosa (Buenos Aires: Siglo XXI, 1974): 81.} un movimiento
siempre presente, sea latente o visible. La cuestión central de la
comunicación es la producción de tramas de significación, conscientes o
inconscientes, implícitas o explícitas. Comunicar protege, da seguridad
ontológica, contacto, abrigo, piel, calor. ¿Y cómo lo hacen las
tejedoras? Ellas hablan del tejer como un acto meditativo en el que ``la
mente entra en un estado que puede describirse como `receptivo'. Uno
escucha mucho mejor cuando teje. No cuestionas, no peleas, ni impides.
Nada como escuchar música y tejer, o conversar y tejer o ver una
película y tejer''.\footnote{Annuska Angulo y Miriam Mabel Martinez,
  \emph{El mensaje está en el tejido} (México: Futura Textos, 2016): 17.}

Por ello defienden el tejer como un proceso de pensamiento y no solo
como una habilidad manual: el ``tejer desafía la mente''.\footnote{Postrel,
  \emph{El tejido}, 92.} La comunicación como escucha, recepción del
otro, reconocimiento y diálogo. La recepción como imaginario femenino,
apertura al otro, capacidad de dejarse fertilizar y de abarcar al otro.
Aquí ``recepción'' es otra cosa, es preparar mente y cuerpo para lo que
suceda. La comunicación como un preparativo para el acontecimiento
ordinario, para el transcurrir, para la contemplación de la vida. Sin
embargo, el tejer es también lo que se hace mientras se hacen otras
cosas. En algún sentido, el tejer distrae, ayuda a mirar de otro modo, a
concentrar la escucha en otros niveles de armonía y a un diálogo
distendido. El cuerpo entretenido a través de las manos produce la
quietud con la que comienza y se desarrolla otro nivel --- por usar la
metáfora de los videojuegos--- de la existencia.

\hypertarget{comunicacin-tejer-como-cuidado-de-la-vida}{%
\subsection{Comunicación: tejer como cuidado de
la
vida}\label{comunicacin-tejer-como-cuidado-de-la-vida}}

``El hilo de la vida'' es una expresión antigua y se encuentra presente
en culturas muy distintas. El hilo de las Moiras griegas y las Parcas
latinas, hilanderas de la vida, tiene presencia en otras culturas
indoeuropeas como la hitita, la antiguo-islandesa, la báltica, la eslava
y la albanesa, así como en la cultura india e iraní,\footnote{Miguel
  Ángel Andrés Toledo, \emph{El hilo de la vida y el lazo de la muerte
  en la tradición indoirania} (Valencia: Intitució Alfons el Magnànim,
  2010).} en la cultura inca\footnote{Shyntia Verónica Castañeda Yapura,
  Renato Cáceres Sáenz y David Peña Soria, \emph{Tejiendo la vida. Los
  textiles en Q'ero} (Lima: Ministerio de Cultura, 2018).} y en la
maya.\footnote{Manuel Alberto Morales Damián, ``La tejedora, la muerte y
  la vida. Simbolismo maya del trabajo textil en el Códice
  Tro-Cortesia'', \emph{Datatèxtil}, n.º 24 (2011): 76-83,
  \url{https://raco.cat/index.php/Datatextil/article/view/275363}.} Ese hilo
de la vida es el hilo del sentido, el hilo del sino. El hilo que se crea
retorciendo, se extiende, se sostiene y se corta. Como la suerte, como
el destino, la vida pende de un hilo.

En las comunidades andinas latinoamericanas se entiende que el textil es
como el cuerpo y como la vida, el telar como madre y el tejido como hijo
que, como un ser humano, va creciendo.\footnote{Denise Y Arnold, Juan de
  Dios Yapita y Elvira Espejo Ayca, \emph{Hilos sueltos: los Andes desde
  el textil} (La Paz, Bolivia: ILCA, 2016): 102-23; véase también Luis
  Ramiro Beltrán et al., \emph{La comunicación}, 225-51.} Se piensa que
el tejido trenza el hilo de la vida desde el centro, el ombligo,
haciendo circular el flujo de energía corporal y espiritual de un lugar
de fuerza a otro.\footnote{Arnold, Yapita y Espejo Ayca, \emph{Hilos
  sueltos}, 55.} Comunicar es proteger la vida, el cuerpo el tejido, y
la tela la piel a la que se refiere Marshall McLuhan\footnote{Marshall
  McLuhan, \emph{Comprender los medios de comunicación}, trad. de
  Patrick Ducher (Madrid: Paidós, 1996): 136ss.} y Andrea
Saltzman\footnote{Andrea Saltzman, \emph{La metáfora de la piel. Sobre
  el diseño de la vestimenta} (Buenos Aires: Paidós, 2019).} desde
ángulos muy distintos.

La \emph{cuestión de género} está muy presente. La realidad es que el
tejido no es una tarea solo de mujeres, ni en la historia, ni en la
actualidad, pero sí hay algo, como manifestó Jaques Lacan, por lo que
``la mujer es primordialmente una tejedora''.\footnote{Jaques Lacan,
  \emph{El Seminario.} Libro 10. \emph{La angustia}, trad. de Enric
  Berenguer (Buenos Aires: Paidós, 2018): 221.} En su importante libro
sobre el tejido, Postrel sostiene: ``la historia de los textiles no es
una historia masculina o femenina, ni una historia europea, africana,
asiática o americana. Es todo eso al mismo tiempo, algo acumulativo y
compartido: una historia humana, un tapiz tejido con incontables y
vivísimos hilos''.\footnote{Postrel, \emph{El tejido}, 287.}

La autora tiene una visión romántica y ``acumulativa'' de la historia.
Es cierto que en la antigüedad los hombres tejían, la revolución textil
industrial empleó hombres y mujeres, en guerra todos han tejido ropas,
por mencionar algunos. Sin embargo, eso no puede tapar lo que el tejer y
el tejido muestran. Todos los pueblos han tenido y tienen algún tipo de
actividad de entrelazado textil, sin embargo, la historia del tejido
también muestra las diferencias de género, de clases sociales, las
estrategias coloniales, y se encuentran testimonios, en la escritura y
en el arte, de divergencias, desajustes, discrepancias e incluso
oposiciones y simple dominación. El tapiz que compone la historia del
tejido tiene jirones, desgarros, hilos invisibilizados e hilos que tapan
y ocultan, hilos tensos y otros superficiales. La armonía y la belleza
del tejido de la humanidad también guarda injusticia, dolor y
sufrimiento, pero si los seres humanos seguimos tejiendo tal vez sea
porque seguimos buscando la belleza y la esperanza. La comunicación,
como el tejido, tiene una historia que corre paralela a la historia del
ser humano. Comunicación aparece así como algo más y distinto a la
coordinación y la interacción. Es, sobre todo, reconocimiento de la
alteridad y búsqueda de sentido. Y, como tal, es matriz vital y
entrelazado creador y protector de la vida. La comunicación humaniza,
hace humanos en un mundo humano.

El tejido tiene una relación con la vida y con el cuerpo femenino que va
mucho más allá del relevamiento de quién hace la actividad y dónde. El
tejido de lo común, el cuidado de los otros, proviene de un tejido que
``prolonga el cuerpo y también el cuidado, el vínculo y el
nexo''.\footnote{Mónica Nepote, prólogo ``Tejer las redes del cuidado''
  a \emph{El mensaje está en el tejido}, de Annuska Angulo y Miriam
  Mabel Martinez (México: Futura Textos, 2016): 7.} El que después de la
segunda guerra mundial las mujeres dejaran de tejer y coser en el
espacio doméstico, espacio esencialmente sin reconocimiento, hizo que la
sociedad se entregara al mercado de la moda de usar y tirar. Desde hace
unos años, mujeres de muchas partes del mundo han generado talleres de
tejido y de costura, de creación de tejido social en barrios pobres o
comunidades marginadas y de (auto)reparación de las identidades y de los
lazos sociales.\footnote{Hay múltiples referencias que se pueden
  encontrar con diversos nombres en los buscadores de la Internet. La
  red social de tejedoras más numerosa es
  \url{https://www.ravelry.com/}.}

\hypertarget{comunicacin-tejer-el-tejido-social}{%
\subsection{Comunicación: tejer el tejido
social}\label{comunicacin-tejer-el-tejido-social}}

La textilidad de la comunicación enfrenta la visión instrumental y
mediacéntrica con una estrategia ontológica de producción de vida en
relación con los otros ---antecesores, contemporáneos y
descendientes---, con la geografía ---la madre tierra y el conjunto de
los seres vivos--- y con las creencias y cultura de la comunidad. Ello
se hace patente en los activismos y luchas textiles que recorren las
ciudades y el campo de toda Meso y Sudamérica con innumerables
experiencias comunitarias y feministas. Solo por mencionar unos pocos
ejemplos. Las arpilleras chilenas desde los años setenta del siglo
pasado que plantaron cara a la dictadura de Augusto Pinochet. Mujeres
chilenas, madres, esposas, hijas y ellas mismas perseguidas políticas
por el régimen represivo y que encontraron y actualizaron una antigua
práctica mapuche como modo de existencia y resistencia.\footnote{Véanse
  Gaby Franger, \emph{Arpilleras: cuadros que hablan vida cotidiana y
  organización de mujeres} (Lima: Movimiento Manuela Ramos, 1988);
  Marjorie Agosín, \emph{Tapestries of Hope, Threads of Love-the
  Arpillera Movement in Chile 1974-1994} (Nuevo México: University of
  New Mexico Press, 1996).} En Perú, el Proyecto Quipu que con la ayuda
de tecnologías digitales se ha constituido como una organización que
lucha por la memoria y justicia contra la esterilización forzada
realizada en comunidades rurales e indígenas durante el gobierno de
Alberto Fujimori.\footnote{Proyecto Quipu, acceso el 20 de marzo de
  2022, \url{https://interactive.quipu-project.com/\#/es/quipu/intro}}
En Colombia el activismo textil se multiplica por todo el territorio
construyendo colectividad, para promover causas sociales y canalizar
denuncias o protestas.\footnote{Véase Eliana Sánchez-Aldana, Tania
  Pérez-Bustos y Alexandra Chocontá-Piraquive, ``¿Qué son los activismos
  textiles?: una mirada desde los estudios feministas a catorce casos
  bogotanos'', \emph{Athenea Digital} 19, n.º 3, (noviembre 2019),
  e2407, \url{https://doi.org/10.5565/rev/athenea.2407}.} Algo similar
sucede en Panamá y Guatemala, con la particularidad de una búsqueda de
protección legal \emph{sui generis} de propiedad intelectual
comunal.\footnote{Véase Gemma Celigueta Comerma y Mónica Martínez Mauri,
  ``¿Diseños mediáticos? Investigar sobre activismo indígena en Panamá,
  Guatemala y el espacio Web 2.0'', \emph{Revista Española de
  Antropología Americana} 50 (2020): 241-52.} Los casos son innumerables
y en todos los países, estos pocos ejemplos muestran la vivacidad de los
movimientos de activismo textil latinoamericano que se emparentan con
los que se desarrollan en las ciudades de Europa o Estados Unidos hace
tiempo, en el seno de movimientos feministas como la guerrilla o arte
callejero llamados \emph{yarn bombing, yarn storming}, guerrilla de
ganchillo, \emph{graffiti} \emph{crochet}, entre otros nombres para las
costuras subversivas y la rebeldía textil.\footnote{Véase Samantha
  Close, ``Knitting Activism, Knitting Gender, Knitting Race'',
  \emph{International Journal of Communication} 12 (2018): 867--89.}

Las distintas experiencias de militancias textiles manifiestan la
necesidad de reconsiderar qué es una práctica de la comunicación. Los
``modelos'' a los que se recurre en las comunidades para pensarlas y los
recursos antropológicos de interrelación significante no son los
``medios de comunicación'' ni sus tecnologías sino prácticas ancestrales
vivientes en las familias y en las comunidades. Entre ellas sobresale el
uso de nombres y prácticas ``textiles'' que articulan muchas luchas
sociales populares, indígenas y campesinas, con especial fuerza en
Latinoamérica, y con testimonios en algunas urbes del Norte Global. En
todas ellas, el uso del imaginario textil no acentúa los instrumentos de
comunicación para promocionar y publicitar los reclamos de justicia e
igualdad sino la acción comunal de trenzado de intereses y de lucha por
el reconocimiento mediante el sentarse juntos. Acentúa el trenzar, el
dialogar y el producir telas sin diferencias sociales, de género o de
raza. Militancias que generan un entramado protector, activo, vital que
se refuerza en el anudado colaborativo del compañerismo, la sororidad y
la comunidad.

La invisibilidad del textil reducido a una práctica subalterna,
mayoritariamente femenina, popular, pobre, ``propia'' de pueblos
``pobres'', manifiesta una dificultad de la investigación en
comunicación. La dificultad de enfocarse hacia fenómenos por fuera o al
margen de las tecnologías digitales empresarialmente dominantes. En este
sentido, hay posibilidades de una teoría alternativa, pero a condición
de descolonizar los estudios y la historia de la comunicación a fin de
discutir el tecnocentrismo dominante y su teleología. Tal vez así,
podría enfrentarse la ceguera comunicacional hacia lo textil y
reconstruir nuevas genealogías teniendo en cuenta las prácticas textiles
históricas y de diferentes comunidades en sus propios contextos. La
antropología y la historia del tejido ya han realizado parte del
trabajo, ahora toca escucharlo desde la comunicación popular, del Buen
Vivir, y decolonial del Sur.

\hypertarget{conclusin-la-centralidad-subterrnea-de-lo-textil}{%
\section{Conclusión: La centralidad subterránea de lo
textil}\label{conclusin-la-centralidad-subterrnea-de-lo-textil}}

Ante la pregunta de por qué lo textil como interpretación de la
comunicación, se podría responder provisionalmente y desde la teoría de
los imaginarios sociales:

\begin{itemize}
\item
  Porque las referencias a la comunicación y el sentido apoyados en el
  vocabulario y la metáfora del textil están omnipresentes en las
  lenguas occidentales. Hay algo en esta concurrencia ubicua, una
  evocación de un imaginario ancestral que se da por conocido, un
  sentido común del universo simbólico de la sociedad.
\item
  Porque la frecuencia del recurso cognitivo metafórico textil en las
  teorías de la comunicación también es múltiple, pero rara vez se
  explicita. Semejante a lo que sucede en el lenguaje de la vida
  cotidiana, se utiliza la metáfora como recurso narrativo sin más.
\item
  Porque al profundizar en el olvido y ausencia de lo textil puede
  rastrearse una negación que va unida a la ocultación de lo femenino,
  lo cotidiano, lo doméstico, lo subalterno, lo popular, lo artesanal.
\item
  Porque al estudiar el imaginario textil aparece una comprensión del
  fenómeno de la comunicación que abre nuevos horizontes de
  interrogación.
\end{itemize}

Y estas respuestas llevan a la necesidad de postular una teoría
alter/nativa de la comunicación desde un lugar distinto a la
comunicología tradicional mediacéntrica. Estudiar la comunicación
luchando contra las diversas formas de negación de la existencia de lo
diferente a la cultura occidental euroamericana.

Boaventura de Sousa Santos defiende que hay ``cinco formas sociales
principales de no existencia producidas o legitimadas por la razón
eurocéntrica dominante: lo ignorante, lo residual, lo inferior, lo local
o particular y lo improductivo''.\footnote{Boaventura de Sousa Santos,
  \emph{Descolonizar el saber, reinventar el poder} (Montevideo: Trilce
  / Universidad de la República, 2010): 22.} Lo \emph{ignorante}
producido por lógicas del saber que privilegian la ciencia positiva y la
estética de la alta cultura europea con sus cánones de verdad y belleza.
Lo \emph{atrasado} fruto de una lógica del tiempo lineal con sus ideas
de progreso, revolución, desarrollo y crecimiento. Lo \emph{inferior}
producto de una lógica de clasificación social, en particular racial y
sexual, y las diferencias y jerarquías que de allí brotan. Una lógica de
escala dominante que privilegia lo universal y lo global y deja de lado
lo \emph{particular} y lo \emph{local}. Y finalmente lo
\emph{improductivo} ---lo estéril o la pereza, por ejemplo---
consecuencia de una lógica productivista capitalista aplicada a la
naturaleza y al trabajo. La comprensión de la comunicación desde el
imaginario textil cumple con todas estas características y obliga a
enfrentar prácticas comunicacionales socialmente significantes y
alternativas a los ``medios de comunicación''.

Esta comprensión de la comunicación debe orientarse, entonces, hacia una
comunicología desde el Sur, intercultural, decolonial y subalterna, lo
que implica una consideración epistemológica de la comunicación y de su
estudio, la comunicología. El imaginario textil merece ser profundizado
para, también, repensar las apropiaciones de las tecnologías desde
prácticas y tácticas de la comunicación local, comunitaria y popular. De
esta manera lo analizado en este artículo (la linealidad evolutiva,
importancia de la oralidad, las referencias a las prácticas textiles
tradicionales, la relevancia de las actuales manifestaciones sociales y
políticas que toman como referencia lo textil) se encuentra con el
testimonio de la realidad de las comunidades latinoamericanas. Aquí es
donde el estudio de la comunicación puede encontrar nuevos caminos en los m\'argenes de la modernidad occidental.


\vspace*{3em}

\section{Bibliography}\label{bibliography}

\begin{hangparas}{.25in}{1} 



Abal de Russo, Clara M. \emph{Arte textil. Incaico en ofrendatorios de
la alta cordillera andina. Aconcagua, Llullaillaco, Chuscha}. Buenos
Aires: CEPPA, 2010.

Agosín, Marjorie. \emph{Tapestries of Hope, Threads of Love-the
Arpillera Movement in Chile 1974-1994}. Nuevo México: University of New
Mexico Press, 1996.

Albers, Anni. \emph{On Weaving}. Nueva Jersey: Princeton University
Press, 2003.

Andrés Toledo, Miguel Ángel. \emph{El hilo de la vida y el lazo de la
muerte en la tradición indoirania}. Valencia: Intitució Alfons el
Magnànim, 2010.

Angulo, Annuska y Miriam Mabel Martinez. \emph{El mensaje está en el
tejido}. México: Futura Textos, 2016.

Arendt, Hannah. \emph{La condición humana}. Trad. de Ramón Gil Novales.
Barcelona: Paidós, 2009.

Aristófanes. \emph{Lisístrata}. Trad. de Luis M. Macía Aparicio. Madrid:
Ediciones Clásicas, 1993.

Arnold, Denise Y, Juan de Dios Yapita y Elvira Espejo Ayca, \emph{Hilos
sueltos: los Andes desde el textil}. La Paz, Bolivia: ILCA, 2016.

Barendse, Rita y Antonio Lobera. \emph{Manual de artesanía textil}.
Barcelona: Alta Fulla, 1987.

Barranquero, Alejandro y Juan Ramos Martín. ``Luis Ramiro Beltrán and
Theorizing Horizontal and Decolonial Communication''. En Miike y Yin,
\emph{The Handbook}.

Barthes, Roland. \emph{El placer del texto}. Trad. de Nicolás Rosa.
Buenos Aires: Siglo XXI, 1974.

Bateson, Gregory. ``Una teoría del juego y de la fantasía''. En
\emph{Pasos hacia una ecología de la mente}. Trad. de Ramón Alcalde
(Buenos Aires: Editorial Lohlé-Lumen, 1991).

Beltrán, Luis Ramiro et al. \emph{La comunicación antes de Colón. Tipos
y formas en Mesoamérica y los Andes}. La Paz: CIBEC, 2008.

Blisniewski, Thomas. \emph{Las mujeres que no pierden el hilo. Retratos
de mujeres que hilan, tejen y cosen de Rubens a Hopper}. Madrid: Maeva,
2009.

Cabrera Altieri, Daniel H. coord. \emph{Cosas confusas. Comprender las
tecnologías y la comunicación}. Valencia, Tirant lo Blanch, 2019.

Cabrera Altieri, Daniel H. ``El imaginario textil griego y la
comunicación''. \emph{RAE-IC, Revista de la Asociación Española de
Investigación de la Comunicación} 1, n.° 2 (2014): 65-73.

Cabrera Altieri, Daniel H. ``Exploraciones sobre el significado de la
técnica y la escritura''. En Cabrera Altieri, \emph{Cosas confusas},
13-29.

Cabrera Altieri, Daniel H. \emph{Lo tecnológico y lo imaginario. Las
nuevas tecnologías como creencias y esperanzas colectivas}. Buenos
Aires: Biblos, 2006.

Cabrera Altieri, Daniel H. ``Lo textil como vía para repensar la
comunicación/tecnología''. En Cabrera Altieri, \emph{Cosas confusas},
35-48.

Cabrera Altieri, Daniel H. \emph{Tecnología como ensoñación. Ensayos
sobre el imaginario tecnocomunicacional}. Temuco: Ediciones Universidad
de la Frontera, 2022.
\url{http://bibliotecadigital.ufro.cl/?a=view\&item=1962}.

Carey, James W. ``A Cultural Approach to Communication'', en
\emph{Communication as Culture: Essays on Media and Society}. Boston:
Unwin Hyman, 1989: 13-36.

Castañeda Yapura, Shyntia Verónica, Renato Cáceres Sáenz y David Peña
Soria. \emph{Tejiendo la vida. Los textiles en Q'ero}. Lima: Ministerio
de Cultura, 2018.

Castells, Manuel. \emph{La sociedad de la información}. Vol. 1, \emph{La
sociedad red}. Trad. de Carmen Martínez Gimeno y Jesús Alborés. Madrid:
Alianza, 2000.

Castoriadis, Cornelius. \emph{La institución imaginaria de la sociedad}.
Trad. de Antoni Vicens y Marco Aurelio Galmarini. Barcelona: Tusquets,
1993.

Castoriadis, Cornelius. ``Notas sobre algunos medios de la poesía''. En
\emph{Figuras de lo pensable}. Valencia: Cátedra, 1999: 36-61.

Castoriadis, Cornelius. \emph{Sobre} El Político \emph{de Platón}.
México: Fondo de Cultura Económica, 2002.

Celigueta Comerma, Gemma y Mónica Martínez Mauri. ``¿Diseños mediáticos?
Investigar sobre activismo indígena en Panamá, Guatemala y el espacio
Web 2.0''. \emph{Revista Española de Antropología Americana}, n.° 50
(2020): 241-52.

Close, Samantha. ``Knitting Activism, Knitting Gender, Knitting Race''.
\emph{International Journal of Communication} 12 (2018): 867--89.

Contreras Baspineiro, Adalid. ``Aruskipasipxañanakasakipunirakispawa''.
En Sierra Caballero y Maldonado Rivera, \emph{Comunicación}, 59-93.

Craig, Robert. T. ``Communication Theory as a Field''.
\emph{Communication Theory} 9, n.° 2 (1999).

Dader, José Luis. ``La evolución de las investigaciones sobre la
influencia de los medios y su primera etapa: Teorías del impacto
directo''. En \emph{Opinión pública y comunicación política}, de Alonso
Muñoz et al. Madrid: Eudema, 1990.

Davis, Martin. \emph{La computadora universal. De Leibniz a Turing}.
Trad. de Ricardo García Pérez. Barcelona: Debate, 2002.

De Bustos, Eduardo. \emph{La metáfora. Ensayos transdisciplinares}.
Madrid: Fondo de Cultura Económica / UNED, 2000.

De Sousa Santos, Boaventura. \emph{Descolonizar el saber, reinventar el
poder}. Montevideo: Trilce / Universidad de la República, 2010.

Derrida, Jacques. \emph{De la gramatología}. Trad. de Oscar del Barco y
Conrado Ceretti. México: Fondo de Cultura Económica, 2017.

Durand, Gilbert. \emph{Las estructuras antropológicas del imaginario}.
Trad. de Víctor Goldstein. México: Fondo de Cultura Económica, 2004.

El Murr, Dimitri. ``La Symploke Politike: Le paradigma du tissage dans
le Politique de Platon, ou les raisons d'un paradigma Arbitraire''.
\emph{Kairos}, n.º 19 (2002): 49-95.

Fischer, Eva. \emph{Urdiendo el tejido social. Sociedad y producción
textil en los Andes bolivianos}. Trad. de Eva Fischer. Berlín: Lit
Verlag, 2008.

Flichy, Patrice. \emph{Lo imaginario de internet}. Trad. de Félix de la
Fuente y Mireia de la Fuente Rocafort. Madrid: Tecnos, 2003.

Flórez Miguel, Cirilo. ``Retórica, metáfora y concepto en Nietzsche''.
\emph{Estudios Nietzsche}, n.° 4 (2004): 51-67.

Flusser, Vilém. ``¿Qué es la comunicación?''. Trad. de Victor Silva
Echeto. En Cabrera Altieri, \emph{Cosas confusas}.

Fox, Thomas W. \emph{Maquinaria de tejidos}. Trad. de Francisco Madurga.
Barcelona: Bosch, 1919.

Franger, Gaby. \emph{Arpilleras: cuadros que hablan vida cotidiana y
organización de mujeres}. Lima: Movimiento Manuela Ramos, 1988.

Fuentes Navarro, Raúl. ``Cuatro décadas de internacionalización
académica en el campo de estudios de la comunicación en América
Latina''. \emph{Anuario Electrónico de Estudios en Comunicación Social,
Disertaciones} 9, n.° 2 (2016): 8-26.
\url{https://doi.org/10.12804/disertaciones.09.02.2016.01}

Fuentes Navarro, Raúl. ``Latin American Interventions to the Practice and Theory of
Communication and Social Development: On the Legacy of Juan
Díaz-Bordenave''. En Miike y Yin, \emph{The Handbook}.

Gadamer, Hans Georg. \emph{Mito y razón}. Trad. de José Francisco Zúñiga
García. Paidós: Barcelona, 1997.

García-Ventura, Agnès. ``Imágenes del universo textil en las primeras
escrituras''. \emph{Datatèxtil}, n.° 14 (2006): 20-31.
\url{https://raco.cat/index.php/Datatextil/article/view/278625}.

Gaspa, Salvatore, Cécile Michel y Marie-Louise Nosch, eds. \emph{Textile
Terminologies from the Orient to the Mediterranean and Europe, 1000 BC
to 1000 AD}. Nebraska: Zea Books Lincoln, 2017.

Geertz, Clifford. \emph{La interpretación de la cultura}. Trad. de
Alberto L. Bixio. Barcelona: Gedisa, 2003.

Gleick, James. \emph{La información. Historia y realidad}. Trad. de Juan
Rabasseda y Teófilo de Lozoya. Barcelona: Crítica, 2011.

González Tanco, Eva y Carlos Arcila Calderón. ``Buen Vivir as a Critique
of Communication for Development''. En Miike y Yin, \emph{The Handbook of Global Interventions in Communication Theory}, editado por Yoshitaka Miike y Jing Yin. London: Routledge, 2022..

Gürtler, Max y W. Kind. \emph{La industria textil}. Trad. de Ricardo
Ferrer. Barcelona: Labor, 1947.

Herkman, Juha. ``Current trends in media research''. \emph{Nordicom
Review} 29, n.° 1 (2008): 145--159.

Hofstadter, Douglas y Emmanuel Sander. \emph{La analogía. El motor del
pensamiento}. Trad. de Roberto Musa Giuliano. Barcelona: Tusquets, 2018.

Ingold, Tim. \emph{Líneas.} \emph{Una breve historia}. Trad. de Carlos
García Simón. Barcelona: Gedisa, 2015.

Ingold, Tim. ``The textility of making''. \emph{Cambridge Journal of
Economics} 34, n.° 1 (2010): 91-102. \url{https://doi.org/10.1093/cje/bep042}.

Innerarity, Daniel. ``La seducción del lenguaje. Nietzsche y la
metáfora''. \emph{Contrastes: revista internacional de filosofía}, n.º 3
(1998): 123-45.

Jäkel, Olaf, Martin Döring y Anke Beger. ``Science and metaphor: a truly
interdisciplinary perspective. The third international metaphorik.de
workshop'' en \emph{Metaphorik.de -- online journal on metaphor and
metonymy}, n.° 26 (2016).
\href{http://www.metaphorik.de/fr/journal/26/science-and-metaphor-truly-interdisciplinary-perspective-third-international-metaphorikde-workshop.html}{http://www.metaphorik.de/} \href{http://www.metaphorik.de/fr/journal/26/science-and-metaphor-truly-interdisciplinary-perspective-third-international-metaphorikde-workshop.html}{fr/journal/26/science-and-metaphor-truly-interdisciplinary-perspective-third-international-metaphorikde-workshop.html}.

Jastrow, Robert. \emph{El telar mágico. El cerebro humano y el
ordenador}. Trad. de Domingo Santos. Barcelona: Salvat, 1993.

Karam, Tanius. ``Tensiones para un giro decolonial en el pensamiento
comunicológico. Abriendo la discusión''. \emph{Chasqui. Revista
Latinoamericana de Comunicación}, n.º 133 (dic. 2016-mar. 2017):
247-264.

Katz, Elihu. ``The Two-Step Flow of Communication: An Up-To-Date Report
on an Hypothesis''. \emph{Political Opinion Quarterly} 21, n.° 1
(primavera 1957): 61-78. \url{https://doi.org/10.1086/266687}.

Krippendorff, Klaus. ``Principales metáforas de la comunicación y
algunas reflexiones constructivistas acerca de su utilización'' en
\emph{Construcciones de la experiencia humana II}, editado por Marcelo
Pakman. Barcelona: Gedisa, 1997: 107-46.

Lacan, Jaques. \emph{El Seminario.} Libro 10. \emph{La angustia}. Trad.
de Enric Berenguer. Buenos Aires: Paidós, 2018.

Lakoff, George y Mark Johnson. \emph{Metáforas de la vida cotidiana}.
Trad. de Carmen González Marín. Madrid: Cátedra, 2009.

Lancien, Thierry et al. ``La recherche en communication en France.
Tendences et carences''. \emph{Recherche \& communication}, dirigido por
Thierry Lancien. \emph{MEI} \emph{(Médiation et Information)}, n.° 14.
Saint-Denis: L'Harmattan, 2001.

Langlois, Ganaele. ``Distributed Intelligence: Silk Weaving and the
Jacquard Mechanism''. \emph{Canadian Journal of Communication} 44, n.° 4
(2019): 555--66. \url{https://doi.org/10.22230/cjc.2019v44n4a3723}.

Lazarsfeld, Paul F. y Elihu Katz. \emph{La influencia personal: el
individuo en el proceso de comunicación de masas}. Barcelona: Hispano
Europea, 1979.

Lizcano, Emmanuel. \emph{Metáforas que nos piensan}. Madrid: Ediciones
Bajo Cero / Traficantes de Sueños, 2014.

Magallanes Blanco, Claudia y José Manuel Ramos Rodríguez, coords.
\emph{Miradas propias. Pueblos indígenas, comunicación y medios en la
sociedad global}. Quito: CIESPAL, 2016.

Marques de Melo, José. \emph{Pensamiento comunicacional latinoamericano.
Entre el saber y el poder}. Sevilla: Comunicación Social, 2009.

Mattelart, Armand. \emph{La comunicación-mundo. Historia de las ideas y
de las estrategias}. Trad. de Gilles Multigner. México: Siglo XXI, 1997.

Mattelart, Armand. \emph{La invención de la comunicación}. Trad. de
Gilles Multigner. Barcelona: Bosch, 1995.

McCombs, Maxwell E. y Donald L. Shaw. ``The Agenda-Setting Function of
Mass Media''. \emph{Public Opinion Quarterly} 36, n.° 2 (verano 1972):
176-87.

McLuhan, Marshall. \emph{Comprender los medios de comunicación}. Trad.
de Patrick Ducher. Madrid: Paidós, 1996.

McLuhan, Marshall. \emph{La galaxia Gutenberg. Génesis del ``Homo
typographicus''}. Trad. de Juan Novella. Barcelona: Círculo de Lectores,
1993.

Meunier, Jean Pierre. ``Las metáforas de comunicación como metáforas que
cobran realidad''. \emph{Signo y Pensamiento} 16, n.° 30 (1997):
115--28.
\url{https://revistas.javeriana.edu.co/index.php/signoypensamiento/article/view/5537}.

Miike, Yoshitaka y Jing Yin, eds. \emph{The Handbook of Global
Interventions in Communication Theory}. Nueva York: Routledge, 2022.

Morales Damián, Manuel Alberto. ``La tejedora, la muerte y la vida.
Simbolismo maya del trabajo textil en el Códice Tro-Cortesia''.
\emph{Datatèxtil}, n.° 24 (2011): 76-83.
\url{https://raco.cat/index.php/Datatextil/article/view/275363}.

Mosco, Vincent. \emph{La nube. Big Data en un mundo turbulento}.
Barcelona: Intervención cultural / Biblioteca Buridán, 2014.

Muñoz-Torres, Juan Ramón. ``Abuso de la metáfora y laxitud conceptual en
comunicación''. \emph{Mediaciones Sociales. Revista de Ciencias Sociales
y de la Comunicación}, n.º 11 (2012): 3-26.
\url{http://dx.doi.org/10.5209/rev_MESO.2012.v11.41267}.

Musso, Pierre. ``Génesis y crítica de la noción de red'', trad. de Jorge
Márquez Valderrama, \emph{Ciencias Sociales y Educación} 2, n.º 3
(enero-junio 2013): 201-24.

Noelle-Neumann, Elizabeth. \emph{La espiral del silencio}. Trad. de
Javier Ruíz Calderón. Barcelona: Paidós, 2003.

Ong, Walter. \emph{Oralidad y escritura. Tecnologías de la palabra}.
Trad. de Angélica Scherp. México: Fondo de Cultura Económica, 2011.

Panagl, Oswald. ``Der Text als Gewebe: Lexikalische Studien mi
Sinnbezirk von Webstuhl und Kleid''. En \emph{Textile Terminologies from
the Orient to the Mediterranean and Europe, 1000 BC to 1000 AD}, editado
por Salvatore Gaspa, Cécile Michel y Marie-Louise Nosch. Nebraska: Zea
Books Lincoln, 2017.

Papalini, Vanina. ``La comunicación según las metáforas oceánicas''.
\emph{Razón y Palabra}, n.º 78 (noviembre 2011-enero 2012).
\url{http://www.razonypalabra.org.mx/varia/N78/1a\%20parte/02_Papalini_V78.pdf}.

Pérez Álvarez, Federico y Carmen Timoneda Gallart. ``El poder de la
metáfora en la comunicación humana: ¿Qué hay de cierto? La metáfora en
la teoría y la práctica perspectiva en neurociencia''.
\emph{International Journal of Developmental and Educational Psychology}
6, n.º 1 (2014): 493-500.
\url{https://doi.org/10.17060/ijodaep.2014.n1.v6.769}.

Peters, John Durham. \emph{Hablar al aire. Una historia de la idea de
comunicación}. Trad. de José María Ímaz. México: Fondo de Cultura
Económica, 2014.

Peters, John Durham. \emph{The Marvelous Clouds. Toward a Philosophy of
Elemental Media}. Chicago: The University of Chicago, 2015.

Pintos, Juan Luis. ``Comunicación, construcción de la realidad e
imaginarios sociales''. \emph{Utopía y Praxis Latinoamericana} 10, n.º
29 (2005): 37-65.

Plant, Sadie. \emph{Ceros + unos. Mujeres digitales + la nueva cultura}.
Trad. de Eduardo Urios. Barcelona: Destino, 1998.

Postrel, Virginia. \emph{El tejido de la civilización. Cómo los textiles
dieron forma al mundo}. Trad. de Lorenzo Luengo. Madrid: Siruela, 2020.

Reddy, Michael. ``The conduit metaphor: A case or Frame Conflict in our
language about language''. En \emph{Metaphor and Thought}, editado por
Eden A. Ortony. Cambridge: Cambridge University Press, 1993: 164-201.

\enlargethispage{\baselineskip}

Rivadulla, Andrés. ``Metáforas y modelos en ciencia y filosofía''.
\emph{Revista de Filosofía} 31, n.° 2 (2006): 189-202.

Roberts, Edward A. \emph{Diccionario etimológico indoeuropeo de la
lengua española}. Madrid: Alianza, 1996.

Rodríguez, Pablo. \emph{Historia de la información. Del nacimiento de la
estadística y la matemática moderna a los medios masivos y las
comunidades virtuales}. Buenos Aires: Capital Intelectual, 2012.

Saltzman, Andrea. \emph{La metáfora de la piel. Sobre el diseño de la
vestimenta}. Buenos Aires: Paidós, 2019.

Sánchez-Aldana, Eliana, Tania Pérez-Bustos y Alexandra
Chocontá-Piraquive. ``¿Qué son los activismos textiles?: una mirada
desde los estudios feministas a catorce casos bogotanos''. \emph{Athenea
Digital} 19, n.° 3 (noviembre 2019), e2407.
\url{https://doi.org/10.5565/rev/athenea.2407}.

Schmucler, Héctor. ``La investigación: un proyecto
comunicación/cultura'' en \emph{Memoria de la Comunicación} (Buenos
Aires: Biblos, 1997): 145-51.

Scolari, Carlos A. ``Ecología de los medios: de la metáfora a la teoría
(y más allá)''. En \emph{Ecología de los medios: entornos, evoluciones e
interpretaciones}. Gedisa: Barcelona, 2015: 15-42.

Segarra Crespo, Diana. ``Coser y cantar: a propósito del tejido y la
palabra en la cultura clásica'', en \emph{Tejer y vestir. De la
Antigüedad al Islam}, editado por Manuela Marín. Madrid: CSIC, 2001.

Sfez, Lucien. \emph{Crítica de la comunicación}. Trad. de Aníbal C.
Leal. Buenos Aires: Amorrortu, 1995: 46.

Sierra Caballero, Francisco y Claudio Maldonado Rivera, coords.
\emph{Comunicación, decolonialidad y Buen Vivir}. Quito: CIESPAL, 2016.

Taylor, Cynthia y Bryan M. Dewsbury . ``On the Problem and Promise of
Metaphor Use in Science and Science''. \emph{Journal of Microbiology \&
Biology Education} 19, n.º 1 (marzo 2018).
\url{https://doi.org/10.1128/jmbe.v19i1.1538}.

Thompson, John B. \emph{Los media y la modernidad: una teoría de los
medios de comunicación}. Barcelona: Paidós, 1998.

Torrico Villanueva, Erick R. ``Decolonizar la comunicación''. En Sierra
Caballero y Maldonado Rivera, \emph{Comunicación}, 95-112.

Torrico Villanueva, Erick R. \emph{La comunicación pensada desde América
Latina (1960-2009)}. Salamanca: Comunicación Social, 2016.

\pagebreak Winkin, Ives. ``El telégrafo y la orquesta''. En \emph{La nueva
comunicación}, de Gregory Bateson et al. Trad. de Jorge Fibla.
Barcelona: Kairós, 1984: 11-25.

Wunenburger, Jean-Jacques. \emph{La vida de las imágenes}. Trad. de Hugo
Francisco Bauzá. Buenos Aires: UNSAM, 2005.



\end{hangparas}


\end{document}