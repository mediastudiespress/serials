% see the original template for more detail about bibliography, tables, etc: https://www.overleaf.com/latex/templates/handout-design-inspired-by-edward-tufte/dtsbhhkvghzz

\documentclass{tufte-handout}

%\geometry{showframe}% for debugging purposes -- displays the margins

\usepackage{amsmath}

\usepackage{hyperref}

\usepackage{fancyhdr}

\usepackage{hanging}

\hypersetup{colorlinks=true,allcolors=[RGB]{97,15,11}}

\fancyfoot[L]{\emph{History of Media Studies}, vol. 2, 2022}


% Set up the images/graphics package
\usepackage{graphicx}
\setkeys{Gin}{width=\linewidth,totalheight=\textheight,keepaspectratio}
\graphicspath{{graphics/}}

\title[West Berlin's Critical Communication Studies and the Cold War]{West Berlin's Critical Communication Studies and the Cold War: A Study on Symbolic Power from 1948 to 1989} % longtitle shouldn't be necessary

% The following package makes prettier tables.  We're all about the bling!
\usepackage{booktabs}

% The units package provides nice, non-stacked fractions and better spacing
% for units.
\usepackage{units}

% The fancyvrb package lets us customize the formatting of verbatim
% environments.  We use a slightly smaller font.
\usepackage{fancyvrb}
\fvset{fontsize=\normalsize}

% Small sections of multiple columns
\usepackage{multicol}

% Provides paragraphs of dummy text
\usepackage{lipsum}

% These commands are used to pretty-print LaTeX commands
\newcommand{\doccmd}[1]{\texttt{\textbackslash#1}}% command name -- adds backslash automatically
\newcommand{\docopt}[1]{\ensuremath{\langle}\textrm{\textit{#1}}\ensuremath{\rangle}}% optional command argument
\newcommand{\docarg}[1]{\textrm{\textit{#1}}}% (required) command argument
\newenvironment{docspec}{\begin{quote}\noindent}{\end{quote}}% command specification environment
\newcommand{\docenv}[1]{\textsf{#1}}% environment name
\newcommand{\docpkg}[1]{\texttt{#1}}% package name
\newcommand{\doccls}[1]{\texttt{#1}}% document class name
\newcommand{\docclsopt}[1]{\texttt{#1}}% document class option name


\begin{document}

\begin{titlepage}

\begin{fullwidth}
\noindent\LARGE\emph{Exclusions in the History of Media Studies
} \hspace{25mm}\includegraphics[height=1cm]{logo3.png}\\
\noindent\hrulefill\\
\vspace*{1em}
\noindent{\Huge{West Berlin's Critical Communication\\\noindent Studies and the Cold War: A Study on\\\noindent Symbolic Power from 1948 to 1989\par}}

\vspace*{1.5em}

\noindent\LARGE{Maria Löblich}\par}\marginnote{\emph{Maria Löblich, Niklas Venema, and Elisa Pollack, ``West Berlin's Critical Communication Studies and the Cold War: A Study on Symbolic Power from 1948 to 1989,'' \emph{History of Media Studies} 2 (2022), \href{https://doi.org/10.32376/d895a0ea.d0db9590}{https://doi.org/ 10.32376/d895a0ea.d0db9590}.} \vspace*{0.75em}}
\vspace*{0.5em}
\noindent{{\large\emph{Freie Universität Berlin}, \href{mailto:maria.loeblich@fu-berlin.de}{maria.loeblich@fu-berlin.de}\par}} \marginnote{\href{https://creativecommons.org/licenses/by-nc/4.0/}{\includegraphics[height=0.5cm]{by-nc.png}}}

\vspace*{0.75em} 

\noindent{\LARGE{Niklas Venema} \href{https://orcid.org/0000-0001-9697-6208}{\includegraphics[height=0.5cm]{orcid.png}}\par}
\vspace*{0.5em}
\noindent{{\large\emph{Freie Universität Berlin}, \href{mailto:niklas.venema@fu-berlin.de}{niklas.venema@fu-berlin.de}\par}}

\vspace*{0.75em} % third author

\noindent{\LARGE{Elisa Pollack}\par}
\vspace*{0.5em}
\noindent{{\large\emph{Freie Universität Berlin}, \href{mailto:elisa.pollack@fu-berlin.de}{elisa.pollack@fu-berlin.de>}\par}}

\end{fullwidth}

\vspace*{1em}

\hypertarget{abstract}{%
\section{Abstract}\label{abstract}}

This paper examines how the West Berlin communication studies department, for over 40 years, was tied to or ``disciplined'' by the Cold War, leading to practices of exclusion and hegemony. Drawing on Pierre Bourdieu's concept of symbolic power, we analyze how anticommunism as a discourse formed the habitus, capital, and field logic of West Berlin communication studies. Sources are archival material from \emph{Freie Universität Berlin} (Free University of Berlin), minutes of \emph{Abgeordnetenhaus von Berlin} (parliament), press from East and West Germany, and autobiographical material and academic publications from West Berlin and West German communication scholars. The paper describes how the anticommunist discourse at first helped Emil Dovifat, professor and department director from 1928 on, to protect himself from attacks regarding his Nazi past and to reestablish his reputation after 1945. For his successor, the interim director Fritz Eberhard, the anticommunist discourse caused problems. Eberhard tried to consolidate the poorly reputed discipline at \emph{Freie Universität} 


\enlargethispage{2\baselineskip}

\vspace*{2em}

\noindent{\emph{History of Media Studies}, vol. 1, 2021}

\end{titlepage}

\noindent \emph{Berlin} during the 1960s; however, this effort was weakened because he had to defend himself against press attacks for being a socialist. Finally, the unique geopolitics of West Berlin, together with the anticommunist discourse, help to explain why the West Berlin department developed after 1968 into a (lonely) center of critical theory indebted to Marx and the Frankfurt School, and how these approaches were marginalized by the rest of the field and by the political system in the 1970s and 1980s.

\vspace*{2em}

\hypertarget{german-communication-studies-and-societal-critical-theories-during-the-cold-war}{%
\section{German Communication Studies and Societal Critical Theories
\\\noindent during the Cold
War}}\label{german-communication-studies-and-societal-critical-theories-during-the-cold-war}

\enlargethispage{1\baselineskip}

When looking at the denomination of chairs---as well as research and
teaching commitments---at the communications institute of the
\emph{Freie Universität Berlin} (Free University of Berlin) during the
second half of the twentieth century, it becomes clear that societal and
media critical theories in the traditions of Marx and the Frankfurt
School had no strong footing there. At first glance, this failure might
appear unsurprising. From 1948 to 1989, this institute was located in
the Western part of a divided city. The \emph{Institut für Publizistik}
(Institute for Media Studies) was established during the so-called
Berlin Blockade, a conflict of 1948--1949 during which U.S. authorities,
West German politicians, and local journalists invented the narrative of
West Berlin as an ``Outpost of Freedom.''\footnote{Stefanie Eisenhuth
  and Scott H. Krause, ``Inventing the `Outpost of Freedom':
  Transatlantic Narratives and the Historical Actors Crafting West
  Berlin's Postwar Political Culture,'' \emph{Zeithistorische
  Forschungen/Studies in Contemporary History} 11, no. 2 (2014): 195.}
The common story line turned the city's Western part ``into an
endangered fort in close proximity to the enemy'';\footnote{Eisenhuth
  and Krause, ``Inventing the `Outpost of Freedom,'\,'' 197.}
commentators considered West Berlin a \emph{Frontstadt} (frontier city).
The name of the new university in the Western part of the city of course
perfectly supported the symbolic system of a ``free world'' fighting
against the alleged threat posed by communism. This discursive creation
found nourishment in the Khrushchev ultimatum (1958) and the
construction of the Berlin Wall (1961), when tanks lined up on both
sides of the dividing line.\footnote{Geir Lundestad, \emph{East, West,
  North, South: International Relations since 1945} (Los Angeles: Sage,
  2014), 73.} It therefore does not seem presumptive to assume
that---especially in West Berlin and especially at the Freie
Universität, founded in 1948 with U.S. support---Cold War discourse
hindered the adoption of theories questioning the cornerstones of the
Western system. In the late 1970s, with the policy of détente relegated
to the past in Europe,\footnote{Lundestad, \emph{East, West, North,
  South}, 95.} the ``war of ideas and ideologies'' between East and West
re-emerged.\footnote{Linda Risso, ``Radio Wars: Broadcasting in the Cold
  War,'' \emph{Cold War History} 13, no. 2 (2013): 145.} Communication
studies throughout West Germany, not just in West Berlin, in fact
largely failed to institutionalize these approaches.\footnote{Andreas
  Scheu, \emph{Adornos Erben in der Kommunikationswissenschaft: Eine
  Verdrängungsgeschichte?} (Cologne: Halem, 2012); Andreas Scheu and
  Thomas Wiedemann, ``Kommunikationswissenschaft als
  Gesellschaftskritik: Die Ablehnung linker Theorien in der deutschen
  Kommunikationswissenschaft am Beispiel Horst Holzer,`` \emph{Medien \&
  Zeit} 23, no. 4 (2008): 9} Why, of all communications institutes in
the Federal Republic of Germany (FRG), should the one in West Berlin
have attempted to develop and institutionalize societal and media
critical ideas?

In his study on ``the inheritors of Adorno,'' the German communications
scholar Andreas Scheu wrote that, by 1970, the \emph{Institut für
Publizistik} in Berlin had become, at least temporarily, ``a clear and
rather lonely center of `critical communication studies'\,'' in the
Federal Republic.\footnote{Scheu, \emph{Adornos Erben}, 143.} The Free
University (FU) emerged as one of ``the hotbeds of ferment'' that helped
organize the movement of 1968.\footnote{Peter Simonson and David W.
  Park, ``On the History of Communication Study,'' in \emph{The
  International History of Communication Study}, eds. Peter Simonson and
  David W. Park (New York: Routledge, 2016), 15.} The majority of
so-called critical scholars (a term that applied to the Marxist and
Frankfurt School approaches) were academically socialized in Berlin or
at least somehow connected to the place. The paths of critical scholars
usually led to Berlin and, to a lesser extent, to the cities of Dortmund
and Hamburg.\footnote{Scheu, \emph{Adornos Erben}, 143.}

The present contribution elaborates on the thesis that the spread of
societal critical approaches and the failure of their
institutionalization in West Berlin resulted from the struggle for
supremacy between socialism and capitalism in the second half of the
twentieth century. We examine how critical communication studies were
both facilitated and restrained by the ideological confrontation between
East and West and within each bloc.\footnote{Risso, ``Radio Wars,'' 145.}
We draw on the sociology of Pierre Bourdieu, especially on his concept
of symbolic power,\footnote{Pierre Bourdieu, \emph{Language and Symbolic
  Power}, trans. Gino Raymond and Matthew Adamson (Cambridge: Polity
  Press, 1992).} to clarify how the Cold War structured the history of
the institute in West Berlin. Anticommunism emerged as a core element of
this war of words on the part of the West.\footnote{Shawn J.
  Parry-Giles, ``Propaganda, Effect, and the Cold War: Gauging the
  Status of America's `War of Words,'\,'' \emph{Political Communication}
  11, no. 2 (1994): 203.} The aim was to promote the superiority of
capitalist democracy, positioning the United States as the role
model.\footnote{Kenneth Cmiel, ``On Cynicism, Evil, and the Discovery of
  Communication in the 1940s,'' \emph{Journal of Communication} 46, no.
  3 (1996): 96.} Empirically, we analyzed a variety of sources such as
autobiographical and biographical material, academic publications,
material from the archive of the FU, press coverage from East and West
Germany, and minutes of meetings of West Berlin's parliament. We studied
the period from 1948, when the \emph{Institut für Publizistik} at the FU
was founded in the middle of the Berlin Blockade, to 1989, when the era
of socialism in the East ended and, roughly at the same time, the
institute underwent a restructuring.

Perhaps geographical distance carves out the space for the present
study. In his work, Scheu mentioned the Cold War as an external factor,
without considering it more deeply when writing his book in Munich, far
away from the earlier German-German border.\footnote{Scheu,
  \emph{Adornos Erben}, 274--75.} We, in contrast, work in Berlin, where
the traces of the Cold War are omnipresent. This location, some critics
may venture, could lead to an overemphasis on this global conflict. Will
our focus on the Cold War reduce the complex development of theory to a
simple product of just one abstract external force? Of course, other
external influences also shaped the West German field in general and the
fate of critical theories of societies in particular. Press
concentration and the introduction of television, for instance, which
led to economic conflicts and media policy debates. Politicians and
media practitioners expected communication studies to provide them with
numbers and data that they could wield in these debates of the 1960s and
1970s. The shift to descriptive empirical quantitative research (and not
to theories of society and the media) in the field was attended by a
boom of market research.\footnote{Maria Löblich, \emph{Die
  empirisch-sozialwissenschaftliche Wende in der Zeitungs- und
  Publizistikwissenschaft} (Cologne: Halem, 2010).} Furthermore, most
students taking up communication science expected to get trained as
journalists. Their strong practical orientation, even during the 1968
movement, might have contributed to dampening the popularity of theory
in the field.\footnote{Ulrich Neveling, ``Man musste das Vertrauen
  haben,'' in \emph{``Regierungszeit des Mittelbaus?'' Annäherungen an
  die Berliner Publizistikwissenschaft nach der Studentenbewegung}, eds.
  Maria Löblich and Niklas Venema (Cologne: Halem, 2020), 56.} Finally,
after 1945 the discipline wrestled with its Nazi past: in the middle of
a legitimacy crisis, closure threatened many of its
departments.\footnote{Löblich, \emph{Die
  empirisch-sozialwissenschaftliche Wende}.} We here take these other
influences into account and put them in relation to Cold War pressures.

One might also object that biographies and ideational trajectories were
responsible for the development of West Berlin's critical communication
studies. For one, according to a sociology-of-science approach, societal
developments influence the establishment or rejection of theories, and
they do so via institutions, ideas, or biographies.\footnote{Maria
  Löblich and Andreas Scheu, ``Writing the History of Communication
  Studies: A Sociology of Science Approach,'' \emph{Communication
  Theory} 21, no. 1 (2011): 1--22.} Second, the Cold War had an effect
on media content, media structures, and journalists.\footnote{Henrik G.
  Bastiansen, ``Norwegian Media and the Cold War, 1945--1991,''
  \emph{Nordicom Review} 35, special issue (2014): 155; Michael Meyen et
  al., ``Media and the Cold War: The East/West Conflict,'' in \emph{The
  Handbook of European Communication History}, eds. Klaus Arnold, Pascal
  Preston, and Susanne Kinnebrock (Hoboken, NJ: Wiley-Blackwell, 2020),
  205.} In Berlin, for instance, parallel broadcasts from the FRG and
the German Democratic Republic (GDR) targeted each other's
population.\textsuperscript{20} This effect on the research
objects of communication studies was not without repercussions on the
field itself.\textsuperscript{21} It therefore
makes sense to assume that the Western ``crusade against communism'' had
consequences for critical, ``left'' approaches. These approaches did not
seem to fit the idea that ``mass communication research'' could help
``build loyalty at home and stable, new, noncommunist nations around the
globe,''\textsuperscript{22} but they nevertheless\marginnote{\textsuperscript{20} Risso, ``Radio Wars.''} seemed\marginnote{\textsuperscript{21} Marko Ampuja, ``Four Moments of International
  Communication Research in the Cold War and Beyond,'' \emph{Javnost --
  The Public} 26, no. 4 (2019): 347--62; Michael Meyen, ``IAMCR on the
  East-West Battlefield: A Study on the GDR's Attempts to Use the
  Association for Diplomatic Purposes,'' \emph{International Journal of
  Communication} 8 (2014): 2071--89; Arvind Rajagopal, ``A View on the
  History of Media Theory from the Global South,'' \emph{Javnost -- The
  Public} 24, no. 4 (2019): 407--19; Christopher Simpson, \emph{Science
  of Coercion: Communication Research and Psychological Warfare,
  1945--1960} (New York: Oxford University Press, 1994); Thomas
  Wiedemann, Michael Meyen, and Maria Löblich, ``Communication Science
  at the Center of Cold War's Communication Battles: The Case of Walter
  Hagemann (1900-1964),'' in \emph{Communication @ The Center}, ed.
  Steve Jones (New York: Hampton Press, 2012), 107--20.} to have created a
distinctive period, at least in West Berlin's disciplinary history.

The case of West Berlin may contribute to the ``collective reflexivity''
regarding hegemony and exclusion in the history of communication
studies.\textsuperscript{23} First, the history of West Berlin's critical
communication studies demonstrates the ambivalence of exclusion in the
history of the field in the West. Even if the institutionalization of
critical approaches ultimately failed, representatives managed to
establish such research and teaching, at least partially and
temporarily. Moreover, although critical perspectives were largely
excluded from mainstream curricula, we cannot consider critical scholars
as marginalized actors in principle. They were, in fact, able to gain
acceptance by making concessions.\textsuperscript{24} Second, preventing the institutionalization of critical
approaches within the Western scientific community hindered openings for
other perspectives and ways of analyzing society and media. It therefore
helped maintain hegemony and delimitation, even in a global perspective.

The\marginnote{\textsuperscript{22} Cmiel, ``Cynicism, Evil, and the Discovery of
  Communication,'' 95.} second\marginnote{\textsuperscript{23} Simonson and Park, ``On the History of Communication
  Study,'' 2.} section\marginnote{\textsuperscript{24}\setcounter{footnote}{24} Scheu, \emph{Adornos Erben},
  123--32.} of the present article explains the linkage between
external factors and scientific development that Bourdieu provides to
avoid an overly simplistic view. In the third section, we describe the
sources of our study, followed by a fourth section on findings. There we
elaborate on three periods in the institute's history of critical
approaches during which the Cold War and institutional, as well as
partly personal, developments intersected.

\hypertarget{bourdieu-symbolic-power-and-the-field-of-science}{%
\section{Bourdieu: Symbolic Power and the Field of
Science}\label{bourdieu-symbolic-power-and-the-field-of-science}}

Bourdieu's sociology helps us avoid deterministic assumptions about the
Cold War:\footnote{Benno Nietzel, ``Propaganda, Psychological Warfare,
  and Communication Research in the USA and the Soviet Union during the
  Cold War,'' \emph{History of the Human Sciences} 29, no. 4--5 (2016):
  59--67.} his field-capital-habitus theory emphasizes the world of
science as a social world governed by its own rules.\footnote{Pierre
  Bourdieu, \emph{Vom Gebrauch der Wissenschaft: Für eine klinische
  Soziologie des wissenschaftlichen Feldes} (Constance: UVK, 1998).} If
we wanted to study how certain approaches developed under particular
political circumstances, we would need to unearth these rules at play in
the scientific field under study, as well as understand to what extent
members of the field managed to shape them. The latitude of a single
scholar and the degree to which he or she may influence the rules of the
field depend on their social position within the field. According to
Bourdieu,\footnote{Bourdieu, \emph{Vom Gebrauch der Wissenschaft}, 21.}
leading scholars or leading ``classic'' studies determine the universe
of significant research objects, issues, theories, and methods ``worth''
investigating for all community members. Each social position depends on
the available capital and on the way capital is distributed within the
field. Reputation emerges as the specific capital granted within the
scientific field. This capital, indicated for instance by publications
and awards, is convertible to material capital (for example, paid
positions or research grants).

The scientific field delineates the place of struggle for domination
between orthodoxy and heresy, which, despite dispute, may be linked by
solidarity and complicity.\footnote{David W. Park, ``Pierre Bourdieu und
  die Geschichte des kommunikationswissenschaftlichen Feldes: Auf dem
  Weg zu einem reflexiven und konfliktorientierten Verständnis der
  Fachentwicklung,'' in \emph{Pierre Bourdieu und die
  Kommunikationswissenschaft: Internationale Perspektiven}, eds. Thomas
  Wiedemann and Michael Meyen (Cologne: Halem, 2013), 132--33.} Bourdieu
assumes that societal demands (e.g., for specific problem solutions) do
not impart directly on knowledge production but are translated by the
field's own rules. This assumption applies to more or less autonomous
disciplines. Some disciplines, especially young ones and those without
much prestige, however, are less capable of breaking external
influence.\footnote{Bourdieu, \emph{Vom Gebrauch der Wissenschaft}, 19.}
This leads us to ask: How independent of politics were German
communication studies during the Cold War, particularly West Berlin's
\emph{Institut für Publizistik}?

How do we describe and explain the way critical perspectives in West
Berlin were involved in the Cold War? Symbolic power operates via a
system of words, signs, and discourses in society. This system of
symbolic structures expresses a particular ideology and serves
particular interests.\footnote{Bourdieu, \emph{Language and Symbolic
  Power}, 167.} It exercises power by defining reality, which leads to
the reproduction of objective structures. Social actors find symbolic
structures in a ``market'' that serves to steer discursive
production.\footnote{Pierre Bourdieu, \emph{Sozialer Sinn: Kritik der
  theoretischen Vernunft} (Frankfurt/Main: Suhrkamp, 1993), 54.} The
field of power, needing legitimacy, seeks to control this language
market. This market operates with a system of specific sanctions and
censorship processes.\footnote{Bourdieu, \emph{Language and Symbolic
  Power}, 138.} Symbolic power does not always operate in the modus of
politically motivated and publicly expressed propaganda; instead, it
might be embedded in institutions, subjective experiences, and
individual consciousness, as well as in everyday practices, which
reproduce structures of domination. Symbolic power may thus remain
tacit, accepted, and taken for granted. Heretical discourses, engaged in
struggles for truth, deploy alternative visions of the world to
transform structures of domination. Heretical discourse ``presupposes a
conjuncture of critical discourse and an objective crisis.''\footnote{Bourdieu,
  \emph{Language and Symbolic Power}, 128.}

Scholars specialize in symbolic production; they contribute to the
legitimation of dominant meaning systems and sometimes oppose them
within their specific ``symbolic universe.''\footnote{Bourdieu,
  \emph{Language and Symbolic Power}, 124.} Scholars in a field of
science may share the language of politics and prioritize overtly
normative, moral questions, perhaps motivated by their own political
ideas. Contrary to this practice, a ``bracketing of overtly
value-centered questions'' may be sustained by a field of science that
gives principal significance to ``theoretical questions'' and exhibits a
commitment to methods and procedures.\footnote{Jesse G. Delia,
  ``Communication Research: A History,'' in \emph{Handbook of
  Communication Science}, eds. Charles R. Berger and Steven H. Chaffee
  (Beverly Hills, CA: Sage, 1987), 59.} Depending on the discipline's
autonomy, social position, and habitus, scholars may either reproduce
the field's logic to deal with the symbolic power of a particular
ideology or challenge it.\footnote{Bourdieu, \emph{Sozialer Sinn}, 39.}
Different forms of discourse bear different opportunities for material
or symbolic profit, always according to the social position of the
``language producer.''\footnote{Bourdieu, \emph{Language and Symbolic
  Power}, 138.}

This leads to the question of the kinds of experiences West Berlin
scholars had with the United States and its institutions (as well as
with the East), and how they perceived anticommunist discourse. We also
need to ask which role the political field played in the ``struggle for
authority'' during the Cold War.\footnote{Bourdieu, \emph{Language and
  Symbolic Power}, 57.} The political field would include institutions,
such as the local parliament and government. What institutional
resources did the institute have, and to what extent did Berlin
scholars' habitus reproduce or attempt to alter both anticommunist
discourse and the rules of West German communication studies? Bourdieu
himself also raised the issue of how the field at large evaluated
critical communication studies in Berlin (for instance, in book
reviews).

In sum, we assume that the habitus of Berlin's communication scholars,
the capital distribution at the \emph{Institut für Publizistik}, and the
rules of communication studies all related to the symbolic power of the
Cold War. Politics and media constituted central fields in the
production and reproduction of anticommunist discourse. Habitus,
capital, and field provide insights into the autonomy, as well as the
entanglement, of a discipline---in this case of a particular theory
tradition---in the reproduction of structures of domination.

\hypertarget{research-categories-and-sources}{%
\section{Research Categories and
Sources}\label{research-categories-and-sources}}

We derived three main research categories from Bourdieu that helped us
choose and analyze sources.\footnote{Löblich and Scheu, ``Writing the
  History.''} We defined \emph{habitus} as the first category and
focused on key protagonists. We adapted Bourdieu's distinction of
\emph{modus operandi} and \emph{opus operatum}. Modus operandi helped us
study how protagonists dealt with symbolic power. It led us to the
interests of actors, their political and religious beliefs and values,
their understanding of communication science (goals of science, research
objects, theories, methods), and their academic (teaching, researching,
administration) and non-academic engagement (e.g., journalism, political
groups). Opus operatum (biographies, key experiences in life) helped us
understand why actors thought and acted the way they did.

We considered the scientific capital (reputation) and the social,
economic, and cultural capital available to the actors, the institute,
and the field of communication studies in West Germany (e.g., its
standing within the university) as the second category. This category
enabled us to comprehend why anticommunism was reproduced or challenged.
The scientific field of Berlin \emph{Publizistikwissenschaft} (media
studies), located within the FU and within the broader field of
communication studies in West Germany, served as our third main
category\emph{.} It allowed us to consider the institute's degree of
institutionalization and organization (e.g., appointment procedures).
Moreover, the field category included the academic rules of the field:
the common conception of communication studies in the FRG, and the role
of U.S. mass communication research, which influenced many countries in
western and northern Europe after the Second World War.\footnote{Peter
  Simonson and John Durham Peters, ``Communication and Media Studies,
  History to 1968,'' in \emph{The International Encyclopedia of
  Communication}, ed. Wolfgang Donsbach (Hoboken, NJ: Wiley, 2008),
  1--8.} By means of the field category we also researched anticommunist
discourse produced by the media and the political field, which affected
scholars' practices in West Berlin. These three main categories enabled
us to determine how far the discourses of anticommunism and the Cold War
influenced the institute's scientific autonomy.

The categories led to the selection of a variety of sources. First, our
material comprised autobiographical and biographical sources. A range of
biographical interviews with former professors, academic staff, and some
students at the \emph{Institut für Publizistik} formed part of this
material.\footnote{Maria Löblich and Niklas Venema, eds.,
  \emph{``Regierungszeit des Mittelbaus?'' Annäherungen an die Berliner
  Publizistikwissenschaft nach der Studentenbewegung} (Cologne: Halem,
  2020).} We analyzed scientific publications, a media policy newsletter
issued at the institute, articles authored by West Berlin communication
scholars in non-academic publications, and the course catalogue. These
sources helped us reconstruct the academic work of the members of the
institute. Additionally, we considered sources that provided insights
into the exercise of symbolic power in the scientific field, the media,
and politics: book reviews and other parts of the scholarly journal
\emph{Publizistik}, press coverage from East and West Germany, and
minutes of West Berlin's parliamentary debates. Besides these published
sources, we analyzed material from the FU's archive: reports on
appointment procedures, experts' reports, correspondence internal to the
institute, and that exchanged among the institute, the university's
president, management, and Berlin's Ministry of Science. Furthermore,
this archive provided curriculum development documents about student
enrollment numbers and internal job and career development, as well as
resources pertaining to the student movement, such as student journals,
brochures, and posters.

\hypertarget{critical-perspectives-facilitated-and-restrained-by-cold-war-symbolic-power}{%
\section{Critical Perspectives: Facilitated and Restrained by Cold
War\\\noindent Symbolic
Power}\label{critical-perspectives-facilitated-and-restrained-by-cold-war-symbolic-power}}

We chose two criteria to structure the Cold War history of critical
perspectives. One criterion was the turning points of the Cold War
(periods of intensive and diminishing symbolic power of
anticommunism),\footnote{Lundestad, \emph{East, West, North, South}.}
while the other was turning points in the institutional history of West
Berlin communication studies (which partly overlapped with personal
change).

\hypertarget{berlin-wall-and-the-spread-of-critical-perspectives}{%
\subsection{Berlin Wall and the Spread of
Critical
Perspectives}\label{berlin-wall-and-the-spread-of-critical-perspectives}}

Until the 1960s, anticommunism was one of the cornerstones of the
university's identity. Professors and students agreed on a rejection of
the GDR and the Soviet Union. The West Berlin press approved of the
university's positioning in the Cold War. Yet the student movement that
arose around 1965 seemed to end the anticommunist consensus. Much as
they did all around the globe, students in West Berlin began to
criticize Western politics, especially in view of the Vietnam War waged
by the United States.\footnote{Helmut Gollwitzer, ``Ein Go-Out der
  Professoren,'' in \emph{Rundfunkpolitische Kontroversen: Zum 80.
  Geburtstag von Fritz Eberhard}, eds. Manfred Kötterheinrich et al.
  (Frankfurt/Main: Europäische Verlagsanstalt, 1976), 481; Juliane
  Pfeiffer, ``Nicht-Wissen oder Nicht-Wissen-Wollen? Die
  Auseinandersetzung mit der NS-Vergangenheit Emil Dovifats am Berliner
  Institut für Publizistik in den langen Sechzigerjahren,'' in
  \emph{``Regierungszeit des Mittelbaus?'' Annäherungen an die Berliner
  Publizistikwissenschaft nach der Studentenbewegung}, eds. Maria
  Löblich and Niklas Venema (Cologne: Halem, 2020), 402; Ampuja, ``Four
  Moments,'' 351.} After the construction of the Berlin Wall, the
growing antiauthoritarian student movement reached the small
communication studies institute, and students started critically
discussing media, society, and their own discipline. The fresh critical
perspectives faced no great barriers because the institute's
anticommunist orthodoxist (Emil Dovifat) had retired, while a new
professor (Fritz Eberhard) was caught between sympathy and the Cold War
requirements in the service of his institute, weakened after the war
through an institutional and reputational crisis. Moreover, both
students and staff shared a concern about saving their institution.

\vspace*{0.75em} 

\begin{centering}

\noindent \emph{Interim Professor Fritz Eberhard}

\end{centering}


\vspace*{0.5em} 


\noindent The year Fritz Eberhard (1896-1982) started his work as the institute's
new director was the same year the Berlin Wall was erected. From 1961
onward, the number of students from the GDR enrolled at the FU
decreased. It did see an influx of students from the FRG, however, who
arrived with reservations about the West German
establishment.\footnote{Karol Kubicki and Siegward Lönnendonker,
  \emph{Die Freie Universität Berlin 1948--2007: Von der Gründung bis
  zum Exzellenzwettbewerb} (Göttingen: V\&R Unipress, 2008), 57.} As
students in West Berlin became more and more politicized, the works of
Marx and those coming out of the Frankfurt School became standard
literature among them. Although these works did not become part of the
literary canon under the media practitioner (rather than theorist)
Eberhard, whose career was interwoven with U.S. institutions after World
War II, he did not seek conflict with the new generation of critical
students. A fellow professor in the Department of Philosophy in fact
described Eberhard as a sympathizer of the student movement.\footnote{Gollwitzer,
  ``Ein Go-Out,'' 479.}

Eberhard, who had studied economics, did not come from the academic
world when he entered \emph{Publizistikwissenschaft}. He had written a
book about radio audiences while serving as director at a regional
public broadcasting organization in the southwest of the FRG
(Süddeutscher Rundfunk). A lack of qualified communication scholars
unburdened by the Nazi past led the appointment committee to consider
him a suitable candidate despite lacking scholarly achievements. The
committee decided based not on his book but on his journalistic
knowledge.\footnote{Peter Groos, ``Vision oder Zwangslage? Fritz
  Eberhards Positionierung in der akademischen Publizistik an der Freien
  Universität Berlin,'' in \emph{Fritz Eberhard. Rückblicke auf
  Biographie und Werk}, ed. Bernd Sösemann (Stuttgart: Franz Steiner,
  2001)} In the 1920s, Eberhard had worked as a journalist for the
socialist paper \emph{Der Funke}, switching to the BBC in 1937 when, as
a member of the antifascist resistance, he had to emigrate to London.
Moreover, the commission emphasized his symbolic capital as a politician
and public figure who had more than once proven his loyalty to Western
democracy and the United States.\footnote{Groos, ``Vision oder
  Zwangslage,'' 266.}

In his BBC broadcasts, Eberhard had raised questions about how to
restore peace in Europe. With the help of the U.S. High Command, he
returned to Germany in 1945. In Stuttgart, he reported to the American
Office of Strategic Services (the U.S. intelligence agency) and became
an adviser for the U.S. station Radio Stuttgart. In the following years,
Eberhard was active in the political field. He co-founded the Stuttgart
Social Democratic Party (SPD), was appointed state secretary in 1947 (in
the federal state of Württemberg-Baden), and chaired the Deutsches Büro
für Friedensfragen (German Office for Questions of Peace), which
operated under the influence of the Western Allies. He became a member
of the Parliamentary Council in Bonn, which developed the constitution
of the FRG. After vacating his political offices, he served, as
mentioned, as the director of a public broadcasting institution in the
south of the FRG from 1949 until 1958.\footnote{Dietrich Berwanger,
  ``Die Ankunft am Berliner Institut,'' in \emph{Fritz Eberhard:
  Rückblicke auf Biographie und Werk}, ed. Bernd Sösemann (Stuttgart:
  Franz Steiner, 2001); Hans Bohrmann, ``Fritz Eberhard als Förderer und
  Anreger der Kommunikationswissenschaft,'' in \emph{Fritz Eberhard:
  Rückblicke auf Biographie und Werk}, ed. Bernd Sösemann (Stuttgart:
  Franz Steiner, 2001); Stefan Graf Finck von Finckenstein, ``Vita in
  Stichworten,'' in \emph{Fritz Eberhard: Rückblicke auf Biographie und
  Werk}, ed. Bernd Sösemann (Stuttgart: Franz Steiner, 2001).}

Eberhard's appointment sparked media attacks from the conservative camp.
Several newspapers questioned his political integrity, not only the
radical right-wing \emph{Reichsruf} but also the Catholic, conservative
\emph{Rheinischer Merkur}, an influential weekly newspaper secretly
subsidized by Konrad Adenauer's government.\footnote{Stefan Beucke,
  Jochen Meiring, and Maximilian Russ, ``Konrad Adenauer,'' in
  \emph{Medienkanzler: Politische Kommunikation in der
  Kanzlerdemokratie}, ed. Thomas Birkner (Wiesbaden: VS, 2016), 57.}
Referring to his activities during his time in exile and denouncing him
as a traitor to his country, these media outlets declared Eberhard unfit
to lead the institute at the front line of the Cold War. The FU
investigated the accusations but eventually defended
Eberhard.\footnote{Bohrmann, ``Fritz Eberhard,'' 249; Juliane Pfeiffer,
  \emph{Die (Re-)Konstruktion der Vorgeschichte des Instituts für
  Publizistik (- und Kommunikationswissenschaft) an der Freien
  Universität Berlin (1948--1998): Unter besonderer Berücksichtigung der
  Darstellung der Rolle Emil Dovifats in der Zeit des
  Nationalsozialismus} (master's thesis, Freie Universität Berlin,
  2015), 57.} Also, due to his age of 65, Eberhard was seen as a
temporary solution. He was not appointed as a full professor, which
weakened his social position in the Department of Philosophy, which was
divided between an anticommunist camp and a camp willing to
reform.\footnote{Gollwitzer, ``Ein Go-Out,'' 482.}

When Eberhard arrived at the institute in the early 1960s, the field of
German \emph{Publizistikwissenschaft} found itself in crisis. It
suffered from lack of reputation, resources, and qualified talent. The
discipline had willingly served the Nazi dictatorship, had a poor
reputation in science policy, and did not meet with esteem in the
university. Starting in the 1960s, a generation of new professors,
career changers from the media or neighboring disciplines, tried to save
the discipline by attempting to develop it into an empirical social
science.\footnote{Maria Löblich, ``German \emph{Publizistikwissenschaft}
  and Its Shift from a Humanistic to an Empirical Social Scientific
  Discipline: Elisabeth Noelle-Neumann, Emil Dovifat, and the
  Publizistik Debate,'' \emph{European Journal of Communication} 22, no.
  1 (2007): 69--88.} Fritz Eberhard was part of this
generation.\textsuperscript{53}

\enlargethispage{1\baselineskip}

Under\marginnote{\textsuperscript{53}\setcounter{footnote}{53} Freddy Zeitz, ``Die Berufung von Harry Pross auf
  den Lehrstuhl für Publizistik,'' in \emph{``Regierungszeit des
  Mittelbaus?'' Annäherungen an die Berliner Publizistikwissenschaft
  nach der Studentenbewegung},'' eds. Maria Löblich and Niklas Venema
  (Cologne: Halem, 2020).} these conditions, Eberhard's main task was to ensure the survival
of \emph{Publizistikwissenschaft} by finding a permanent successor and
to improve the reputation of the discipline. Like almost all of the
seven professors in the field in West Germany around 1970, Eberhard
supported refocusing the discipline on U.S. mass communication research.
In his few scholarly publications, he adapted American narratives of
anticommunism and emphasized the importance of communication studies as
a weapon in the Cold War.\footnote{Fritz Eberhard, ``Thesen zur
  Publizistikwissenschaft,'' \emph{Publizistik} 6 (1961): 260--61.}
Hanno Hardt has ascribed the role model function that mass communication
fulfilled in West Germany to the search for political
rehabilitation.\footnote{Hanno Hardt, ``Am Vergessen scheitern: Essay
  zur historischen Identität der Publizistikwissenschaft, 1945--68,''
  \emph{Medien \& Zeit} 17, no. 2/3 (2002): 34.} Scholars confessed to a
research practice grown on ``democratic soil'' and tried to abandon the
discipline's Nazi past. Following Bourdieu's question of why symbolic
power is often legitimate power,\footnote{Bourdieu, \emph{Language and
  Symbolic Power}.} we understand that the need to leave the Nazi past
and the crisis behind explains the acceptance of mainstream American
research's supremacy. This refocusing led, among other things, to the
exclusion of (critical) social theory.\footnote{Ampuja, ``Four
  Moments''; Löblich and Scheu, ``Writing the History.''} In comparison
with other institutes, Berlin institutionalized this new focus late. A
chair for empirical methods was not established until the late 1980s, as
we will explain later.

Given the fragile situation of Berlin's \emph{Publizistikwissenschaft},
Eberhard's weak social position in the Department of Philosophy, and his
modest achievements in the field of communication research, he was not
in a position to challenge hegemonic discourses and practices, even if
he wanted to. Quite the contrary. Next to monetary and material
donations, exchange programs intended to facilitate the transfer of U.S.
research standards to Europe took shape.\footnote{Fritz von Bergmann,
  ``Die Hilfe der USA für die Freie Universität Berlin,'' in \emph{Freie
  Universität Berlin 1948--1973: Hochschule im Umbruch; Dokumentation
  III,} eds. Siegward Lönnendonker and Tilman Fichter (Berlin: Freie
  Universität Berlin, 1974), 192.} Eberhard made use of these
opportunities. He had wanted to travel to the United States himself, but
health issues made him send his right-hand woman instead.\footnote{Bohrmann,
  ``Fritz Eberhard,'' 250.} Elisabeth Löckenhoff had already worked as
an assistant under Eberhard's predecessor, Emil Dovifat, and had
supervised many of the department's organizational and administrative
tasks. In 1963, funded by the Ford Foundation, she visited the
communications departments at several U.S. universities to find eligible
candidates willing to work as research assistants in West
Berlin.\footnote{Hans Bohrmann, ``Elisabeth Löckenhoff im Institut für
  Publizistik der Freien Universität Berlin (1952--1985),'' in
  \emph{Publizistik und Journalismus in der DDR: Acht Beiträge zum
  Gedenken an Elisabeth Löckenhoff}, eds. Rolf Geserick and Arnulf
  Kutsch (Munich: K. G. Saur, 1988), 27; Hans Bohrmann, ``Habilitation
  von Dr. Elisabeth Löckenhoff,'' \emph{Publizistik} 17 (1972): 224.}

In his few academic texts, Eberhard mostly refrained from an explicit,
critical political and economic analysis of communication in Western
societies. He did acknowledge the problems of private media ownership
and endorsed public broadcasting.\footnote{Fritz Eberhard, ``Wie
  informiert das Fernsehen?,'' \emph{Die Zeit,} September 23, 1966: 77;
  Manfred Kötterheinrich et al., ``Vorwort,'' in \emph{Runfunkpolitische
  Kontroversen: Zum 80. Geburstag von Fritz Eberhard}, eds. Manfred
  Kötterheinrich, Ulrich Neveling, Ulrich Paetzold, and Hendrik Schmidt
  (Frankfurt/Main: Europäische Verlagsanstalt, 1976), 10.} He also
called for critical audiences and publicists.\textsuperscript{62} Yet his vocabulary remained cautious regarding the
``danger'' press concentration ``might involve,''\textsuperscript{63} a top issue of the student movement. Instead, in line with
prominent U.S. colleagues,\textsuperscript{64}
Eberhard highlighted the need for quantitative media effects research
using drastic metaphors.\textsuperscript{65} The
mass media, he\marginnote{\textsuperscript{62} Fritz Eberhard,
  ``\,`Kritik muß zersetzen': Auch Anpassung kann die Freiheit
  gefährden,'' \emph{Die Zeit}, June 30, 1967, 25; Fritz Eberhard,
  ``Gang durch den deutschen Blätterwald,'' \emph{Die Zeit}, October 10,
  1965, 30.} explained,\marginnote{\textsuperscript{63} Fritz
  Eberhard, ``Macht durch Massenmedien?'' \emph{Publizistik} 10 (1965):
  489.} were\marginnote{\textsuperscript{64} Nietzel, ``Propaganda,'' 66.} essential\marginnote{\textsuperscript{65}\setcounter{footnote}{65} Eberhard, ``Thesen,'' 261--66.} elements in the functioning of
modern democracies.\footnote{Eberhard, ``Blätterwald,'' 485.} Following
this conception of the media, and the typically politicized and
generalized use of the term \emph{media} in Cold War
discourse,\footnote{Rajagopal, ``History of Media Theory,'' 407.} he
held a reserved view of the Frankfurt School's pessimistic outlook on
mass media.

 \enlargethispage{-\baselineskip}
 
Rather, Eberhard introduced his students to books by Paul Lazarsfeld,
Harold Lasswell, Wilbur Schramm, and other leading scholars who had
built their careers on military and government funds.\footnote{Simpson,
  \emph{Science of Coercion}.} He invited guest lecturers from the other
side of the Atlantic and brought Elisabeth Noelle-Neumann, a renowned
public opinion researcher, to West Berlin, where she taught empirical
methods from 1961 to 1964.\footnote{Berwanger, ``Die Ankunft am Berliner
  Institut,'' 22; Pfeiffer, \emph{Die (Re-)Konstruktion der
  Vorgeschichte des Instituts für Publizistik}, 58.} Despite having to
play into the Cold War's symbolic power to save a weak institute,
Eberhard successfully created an intellectual climate that allowed young
scholars to develop critical views on mass media and society. In fact,
one group of student assis-
\newpage
\noindent tants critically analyzed West Berlin's press
coverage of the student movement.\footnote{Peter Schneider, Rolf Sülzer,
  and Wilbert Ubbens, ``Pressekonformität und studentischer Protest: Die
  West-Berliner Tagespresse analysiert anhand ihrer Berichterstattung
  über studentische Aktivitäten aus Anlass des Besuchs des Schah von
  Persien vor dem Hintergrund der allgemeinen Hochschulberichterstattung
  in den Monaten April-Juli 1967 im Vergleich mit ausgewählten
  Tageszeitungen ausserhalb Berlins: Eine statistisch vergleichende
  Aussagenanalyse'' (Manuscript, Institut für Publizistik der Freien
  Universität, 1969).}
  
  
  \vspace*{0.75em} 

\begin{centering}

\noindent \emph{The Cold Warrior Emil Dovifat}

\end{centering}


\vspace*{0.5em} 


\noindent Critical societal perspectives were not Emil Dovifat's concern either.
The conservative Catholic had been appointed professor in 1928, served
in that capacity during National Socialism with a short
interruption,\footnote{Pfeiffer, \emph{Die (Re-)Konstruktion der
  Vorgeschichte des Instituts für Publizistik}.} and became professor
again in 1948. After Eberhard's appointment, Dovifat (1890-1969)
continued teaching and researching until the mid-sixties. The former
director did not understand the critical new generation of students he
met in the late years of his career.\footnote{Klaus-Ulrich Benedikt,
  \emph{Emil Dovifat: Ein katholischer Hochschullehrer und Publizist}
  (Mainz: Matthias Grünwald, 1986), 194.} Not only did his political
orientation oppose antiauthoritarianism and critical social theory but
so did his personal experience of the Cold War. After the Second World
War, Dovifat had been one of the founders of the Christlich
Demokratische Union (Christian Democratic Union, CDU) and served as the
chief editor of the party organ, \emph{Neue Zeit}, in Berlin. This paper
appeared under the license of the Soviet administration. Dovifat lost
his position in the editorial office after only three months.\footnote{Benedikt,
  \emph{Emil Dovifat}.} The Soviet military administration controlled
the old Berlin university, where the \emph{Deutsche Institut für
Zeitungskunde}, as the institute used to be called, had had its place
before 1945.\footnote{James F. Tent, \emph{Freie Universität Berlin,
  1948--1988: Eine deutsche Hochschule im Zeitgeschehen} (Berlin:
  Colloquium, 1988), 33.} While the Soviet authorities accused Dovifat
of essentially continuing the career he had had in Nazi Germany, the
newly founded FU offered him the chance to retain his academic standing.
The Cold War continued to shape Dovifat's work after his appointment in
West Berlin. Ongoing criticism by East German politicians and
press,\footnote{Gunther Kuhnau, ``Lummers Moral,`` \emph{Berliner
  Zeitung}, February 8, 1961; Of, ``Jetzt auf Honorarbasis,''
  \emph{Berliner Zeitung}, May 26, 1961.} as well as by the field of
socialist journalism studies,\footnote{Hans-Joachim Raabe, ``Emil
  Dovifats Lehre von der Publizistik'' (PhD diss., Universität Leipzig,
  1962).} threatened his reputation in the FRG. Criticizing the GDR and
the Soviet Union thus became a main motive in Dovifat's teaching,
popular talks, and media engagement, and publications. He distinguished
``totalitarian'' and ``democratic'' types of media systems, equating the
GDR with Nazi Germany.\footnote{Emil Dovifat, ``Freiheit und Zwang in
  der politischen Willensbildung: Formen der demokratischen und der
  totalitären Meinungsführung,'' in \emph{Veritas, iustitia, libertas:
  Festschrift zur 200-Jahresfeier der Columbia University New York}, ed.
  Freie Universität Berlin (Berlin: Colloquium, 1954); Emil Dovifat,
  ``Publizistik als Wissenschaft: Herkunft -- Wesen -- Aufgabe,''
  \emph{Publizistik} 1, no. 1 (1956): 3--10; Emil Dovifat,
  \emph{Handbuch der Publizistik}, vol. 1, \emph{Allgemeine Publizistik}
  (Berlin: De Gruyter, 1968).} Despite reservations about the
commercialization of the press, Dovifat followed this simple dichotomy:
he advocated for the Western liberal model of a private press and public
broadcasting against that of state-controlled media in socialist
countries. As a member of the administrative board of the Northwest
German Broadcasting Corporation and subsequently of Radio Free Berlin,
he promoted public broadcasting as a counter-propaganda tool against the
GDR and the Soviet Union.\footnote{Emil Dovifat, \emph{Der NWDR in
  Berlin 1946--1954} (Berlin: Haude \& Spenersche Verlagsbuchhandlung,
  1970); Bernd Sösemann, ed., \emph{Emil Dovifat: Studien und Dokumente
  zu Leben und Werk} (Berlin: De Gruyter, 1998).}

\hypertarget{a-critical-habitus-among-students-of-communications-studies}{%
\section{A Critical Habitus among Students of Communications
Studies}\label{a-critical-habitus-among-students-of-communications-studies}}

Walled-in West Berlin had a highly concentrated press market in the
1960s. Students started to discuss the influence of press ownership on
opinion-building when they saw their activities portrayed in a biased
way. Particularly, the conservative, anticommunist newspapers of the
Axel Springer publishing house attacked the student movement and accused
it of paving the way for communism and the influence of the GDR in West
Berlin.\footnote{Kubicki and Lönnendonker, \emph{Die Freie Universität},
  74.} In 1967, Axel Springer owned about 70 percent of West Berlin's
print media.\footnote{Jürgen Wilke, ``Gewalt gegen die Presse: Episoden
  und Eskalationen in der deutschen Geschichte,'' in \emph{Unter Druck
  gesetzt: Vier Kapitel deutscher Pressegeschichte}, ed. Jürgen Wilke
  (Köln: Böhlau, 2002), 185.} From 1966 onward, Springer's headquarters
were located directly at the Wall. Students' growing concern with the
power of a concentrated commercial press merged with their demand for
more practical education, since most of them aimed to go into
journalism.\footnote{Ulrich Neveling, ``Man musste das Vertrauen
  haben,'' 56.} A range of self-organized activities fostered
politicization, including some regular courses, for instance, on the
issue of press concentration.\footnote{Andreas-Rudolf Wosnitza, ``Über
  Fritz Eberhard nachdenken, heißt, über sich selbst nachdenken,'' in
  \emph{Fritz Eberhard: Rückblicke auf Biographie und Werk}, ed. Bernd
  Sösemann (Stuttgart: Franz Steiner, 2001), 28.} At times, tension
existed between academic staff and students, but no major conflicts
erupted as they did in other parts of the university.\footnote{Hans
  Bohrmann, ``Ein politischer Habitus, den ich nicht vertreten habe,''
  in \emph{``Regierungszeit des Mittelbaus?'' Annäherungen an die
  Berliner Publizistikwissenschaft nach der Studentenbewegung}, eds.
  Maria Löblich and Niklas Venema (Cologne: Halem, 2020), 46.}
Löckenhoff, who later became a professor, was not particularly critical,
resulting in complaints from some students.\footnote{Bohrmann,
  ``Elisabeth Löckenhoff,'' 32.} Having lived and studied in the GDR
before coming to West Berlin in 1952, however, she dealt with the
socialist media system in her teaching in a more nuanced way than
Dovifat. Adopting a system-immanent approach to the GDR, she challenged
ruling narratives and most likely raised awareness of the
interdependence of politics and science.\footnote{Barbara Baerns, ``Eine
  Brücke schaffen zwischen Theorie und Praxis,'' in \emph{Ich habe
  dieses Fach erfunden}, eds. Michael Meyen and Maria Löblich (Cologne:
  Halem, 2007), 269.}

The fundamental ideas in the ``critical emancipatory'' habitus of the
politicized students were based on the reception of Marxist theory and
the Frankfurt School. They consisted of:
\begin{itemize}
\item
  A critical stance toward mainstream communication studies because of
  its lack of journalistic practice and for serving publishers'
  interests;
\item
  Criticism of a privately owned press, media manipulation, and the
  reproduction of domination; and
\item
  The aim of having communication studies contribute to human
  emancipation and to societal and media change.
\end{itemize}

\noindent Some of these ideas found formulation in student brochures, but mostly
they came into being in working groups, discussions, and at
congresses.\footnote{E.g., ASTA der Freien Universität Berlin,
  \emph{Kritische Universität: Sommer 68 -- Berichte und Programm}
  (Berlin, 1968).} Not until the next period of the Cold War and the
institute's history did societal critical perspectives start to gain a
foothold. Courses led to graduate theses on critical content analyses of
the press, journalism labor unions, critical media policy analyses,
press concentration, and so-called Third World issues. A curriculum was
developed, and critical students were hired to become research
assistants.

\hypertarget{the-policy-of-dtente-and-the-critical-center}{%
\subsection{The Policy of Détente and the
Critical
Center}\label{the-policy-of-dtente-and-the-critical-center}}

Starting at the end of the 1960s, the department took steps toward
institutionalizing critical perspectives. These steps occurred during
easing tensions in Europe in the context of the policy of
détente.\footnote{Lundestad, \emph{East, West, North}, \emph{South},
  74--75, 95.} The thrust of critical perspectives was also supported by
the politicized university, whose new rules were developed in the
slipstream of temporarily decreased anticommunism. Travel relief, the
result of the Four Powers Agreement on Berlin in 1971, constituted a
small building block. It enabled Westerners' purchase of the ``blue
volumes,'' the GDR edition of Marx and Engels, in East Berlin for little
money.\footnote{Pätzold, Ulrich, ``Ohne Berliner Modell wäre ich nie in
  Dortmund gelandet,'' in \emph{``Regierungszeit des Mittelbaus?''
  Annäherungen an die Berliner Publizistikwissenschaft nach der
  Studentenbewegung}, eds. Maria Löblich and Niklas Venema (Cologne:
  Halem, 2020), 76.} A university reform law, passed by the Social
Democratic West Berlin government in 1969, limited the power of
professors and provided parity among them, academic staff (assistants),
and students on the university's boards. Communication studies' new and
very left-leaning Department of Philosophy and Social Sciences even
included nonacademic staff in parity politics. Expanded codetermination
rights led to the formation of political groups that competed for
influence during appointment and employment procedures. For some years,
the Aktionsgemeinschaft von Demokraten und Sozialisten (Action Group of
Democrats and Socialists), closely tied to the Socialist Unity Party of
West Berlin, became the dominant group at the institute.

The sway of critical perspectives also came to bear on the institute's
leadership. After many years of searching, Eberhard had succeeded in
finding a candidate who was accepted by the appointment committee and
willing to take up this professorship. In 1968, the journalist Harry
Pross (1923-2010) was appointed full professor, a decision made during a
student strike at the institute. The students had two demands: the
appointment of the former director of Radio Bremen and the
democratization of appointment procedures. They welcomed Pross as a
practitioner and well-known book author.\footnote{Zeitz, ``Die Berufung
  von Harry Pross auf den Lehrstuhl für Publizistik.''} Pross, who had
studied social sciences and was active in the journalists' trade union,
had made a career of journalism after 1945. He had authored several
critical books about German history, politics, and mass media by the
time he switched to communication studies. After his appointment, Pross
continued to publish, yet neither in a Marxist or Frankfurt School
tradition nor in a terminology connectable to mainstream communication
studies.\footnote{Harry Pross, \emph{Memoiren eines Inländers:
  1923--1993} (Munich: Artemis \& Winkler, 1993).}

Pross's appointment ended the era of the ``one-man company.'' The
institute saw an expansion of paid jobs for professors, assistants, and
students. By 1980, the number of professors had grown from one to eight.
While professors did not contribute much to the institutionalization of
critical perspectives, particularly in the realms of research and
publication, students and young academic staff engaged in critical
perspectives.\footnote{Maria Löblich and Niklas Venema,
  ``\,`Regierungszeit des Mittelbaus'? Eine Einführung,'' in
  \emph{``Regierungszeit des Mittelbaus?'' Annäherungen an die Berliner
  Publizistikwissenschaft nach der Studentenbewegung},'' eds. Maria
  Löblich and Niklas Venema (Cologne: Halem, 2020).} The number of
students greatly increased starting at the end of the 1960s. While the
number of students majoring in \emph{Publizistik} had tripled by the
late 1970s,\footnote{Günter Barthenheier and Werner Hoffmann, \emph{IfP
  1978: Eine Dokumentation zum 30jährigen Bestehen des Instituts für
  Publizistik} (Berlin: Freie Universität Berlin, 1978), 13; Institut
  für Publizistik, Fachbereich Kommunikationswissenschaften, FU Berlin,
  \emph{Publizistik in Berlin 82} (Berlin: Freie Universität Berlin,
  1982), 15, 106.} the expansion of paid professorships remained
insufficient, with only the appointment of a second professorship in
1970 and a total of eight professorships by 1980. Teaching largely
rested on the shoulders of student tutors and mid-level academic
staff.\footnote{Löblich and Venema, ``\,`Regierungszeit des Mittelbaus'?
  Eine Einführung,'' 11, 19.}

\hypertarget{curriculum-the-berlin-model-of-journalism-education}{%
\section{Curriculum: The ``Berlin Model'' of Journalism
Education}\label{curriculum-the-berlin-model-of-journalism-education}}

Critical theory became an integral part of teaching at the West Berlin
institute in the 1970s.\footnote{Niklas Venema, ``Zwischen Marx und
  Medienpraxis: Das Berliner Modell der Journalistenausbildung,'' in
  \emph{``Regierungszeit des Mittelbaus?'' Annäherungen an die Berliner
  Publizistikwissenschaft nach der Studentenbewegung}, eds. Maria
  Löblich and Niklas Venema (Cologne: Halem, 2020).} With the support of
Pross, students and academic staff drafted a curriculum that integrated
practical orientation and critical approaches. The so-called Berlin
Model defined communication studies as part of ``critical-emancipatory
social sciences'' and aimed to educate practitioners for journalism,
public relations, and pedagogy.\footnote{Wissenschaftliche Einrichtung
  Publizistik, \emph{Studienplan für das Fach Publizistik und
  Dokumentationswissenschaft} (Berlin: Freie Universität Berlin, 1973),
  3.} Due to the interest of some of the teaching staff, a variety of
courses referred to critical theory. Apart from that, courses about
critical political economy formed a fixed part of the
curriculum.\footnote{Wissenschaftliche Einrichtung Publizistik,
  \emph{Studienplan für das Fach Publizistik und
  Dokumentationswissenschaft}, 14.} These courses were taught by
academic assistants. In the 1970s, especially the young sociologists
Volker Gransow and Burkhard Hoffmann tried to adapt Marxist theory to
communication studies and closely followed the classic works of Marx and
Engels. Furthermore, Gransow and Hoffmann were interested in the GDR.
Gransow's sociological dissertation dealt with that country's cultural
policy.\footnote{Volker Gransow, ``Zur kulturpolitischen Entwicklung in
  der Deutschen Demokratischen Republik bis 1973'' (PhD diss., Freie
  Universität Berlin, 1974).} Hoffmann considered literature from GDR in
his attempt to build a materialist communication theory.\textsuperscript{98}

Although\marginnote{\textsuperscript{98}\setcounter{footnote}{98} Burkhard
  Hoffmann, ``Zum Problem der Entwicklung einer materialistischen
  Kommunikationstheorie,'' in \emph{Gesellschaftliche Kommunikation und
  Information}, vol. 2, eds. Jörg Aufermann, Hans Bohrmann, and Rolf
  Sülzer, (Frankfurt/Main: Athenäum-Fischer, 1973); Burkhard Hoffmann,
  ``On the Development of a Materialist Theory of Mass Communications in
  West Germany,'' \emph{Media, Culture \& Society} 5, no. 1 (1983):
  7--24.} Pross, as the institute's new director, promoted the
establishment of the Berlin Model and supervised the work of scholars
such as Gransow and Hoffmann, he remained skeptical about the political
implications of educational efforts of groups such as the
Aktionsgemeinschaft von Demokraten und Sozialisten. Rejecting openness
toward the other German state,\footnote{Harry Pross, ``Uni mit Feuer,''
  \emph{Zeit-Magazin}, December 7, 1973.} he publicly warned of the
politicization of universities and criticized members of his institute
who aimed for the kind of ``cadre education'' practiced in the
GDR.\footnote{Harry Pross, ``Das Berliner Modell,'' in
  \emph{Journalistenausbildung: Modelle, Erfahrungen, Analysen}, ed.
  Walter Hömberg (Munich: Ölschläger, 1978), 150; Pross, \emph{Memoiren
  eines Inländers}, 307.} Pross successfully engaged in the appointment
of Ivan Bystřina (1924-2004) from Czechoslovakia, who left his country
due to the Prague Spring. Bystřina chaired the institute from 1970 until
1990, but he hardly left behind any traces of his presence.\textsuperscript{101}
Similar to the career of his much older predecessor, Pross's after 1945
led through U.S. institutions. From 1949 to 1952, he worked in the
propaganda division of the U.S. High Commissioner for Germany, John
McCloy, who also promoted financial support for the FU at the same time.
Pross\marginnote{\textsuperscript{101}\setcounter{footnote}{101} FU
  Berlin, UA, Professur-Akten, 5711/2-32, Akte-Bystrina/Hoffmann, Ordner
  349, Wiederbesetzung der Hochschullehrerstelle AH 5, 16.11.1972.} served as the editor of the propaganda journal
\emph{Ost-Probleme}. Retrospectively, he wrote that the Americans had
employed him because of his knowledge of Marxism. In 1952, Pross
traveled to the United States with a postgraduate research fellowship
from the private Commonwealth Fund. In his memoirs, he emphasized the
discovery of American propaganda research during that stay.\footnote{Pross,
  \emph{Memoiren eines Inländers}, 185--201.}

The Berlin Model became further institutionalized with the establishment
of a professorship dedicated to media practice. The appointment of
Alexander von Hoffmann (1924-2006) in 1974, however, did not lead to the
accumulation of scientific capital according to the rules of the broader
communication studies field in the West. The former editor of the
renowned news magazine \emph{Der Spiegel} certainly identified with the
critical approach of the Berlin Model, but he focused on teaching
instead of publishing.

The critical approach of communication studies met opposition within the
university and from the state government. The Academic Senate, the
highest board of the university, demanded that the curriculum be in line
with the ``free and democratic societal order'' of the Federal Republic
and offer ``pluralistic'' views.\footnote{Fachbereichsrat des
  Fachbereichs Philosophie und Sozialwissenschaften, ed., \emph{Akzente
  einer Studienreform: Dokumentation der Studienplanung im Fachbereich
  Philosophie und Sozialwissenschaften der Freien Universität Berlin}
  (Berlin: Freie Universität Berlin, 1975), 114.} The local government
of West Berlin never approved the critical curriculum. Among a long list
of complaints, the three Social Democratic ministers of science, who
were in office until 1981, questioned whether political economy was ``at
all relevant'' for the education of journalists.\footnote{Bernd Meyer,
  ``Das Berliner Modell: Ausbildung von Kommunikationspraktikern am
  Institut für Publizistik der FU Berlin; Darstellung und Entwicklung,
  kritische Bestandsaufnahme und Perspektiven dieses Konzeptes''
  (master's thesis, Freie Universität Berlin, 1979), attachment 5.}
Nevertheless, the Berlin Model remained the basis for teaching. The
significant number of students who received this education during this
time in West Berlin might prove a key to understanding why the institute
was perceived as a center of critical communication studies.

To a large extent, assistants carried out the institute's teaching.
These younger scholars also tried to make contacts with the East. In
November 1970, twenty-two students participated in a field trip to
Lomonosov University in Moscow. The journey aimed to start a dialogue
with scholars of the USSR and establish professional networks. During
their week there, however, the Berlin students had only one opportunity
to meet with members of the Moscow Faculty for Journalism. With their
orientation toward a practical education of journalists, the Moscow
students had a different understanding of communication studies than
those from West Berlin who were inspired to discuss Marx. Even though
the trip did not make for a success, it illustrated the West Berlin
students' interest in a Marxist-Leninist conception of communication
science and their willingness to overcome the Cold War in their
field.\footnote{FU Berlin, UA, IfP, Direktoriumsprotokolle,
  Abschlussbericht über die Moskau-Exkursion des Instituts für
  Publizistik, 1970.}

It was probably with the same intention that Burkhard
Hoffmann,\footnote{Concerning the ``Marxist-Leninist group,'' see
  Bohrmann, ``Ein politischer Habitus, den ich nicht vertreten habe,''
  48.} part of the self-declared ``Marxist-Leninist group'' at the
institute and working as a research assistant from 1970 to
1975,\footnote{Barthenheier and Hoffmann, \emph{IfP 1978}, 19.} and
Klaus Betz, a member of the Socialist Unity Party of West Berlin,
traveled to Leipzig. Hoffmann and Betz hoped to engage in theoretical
discussions, but they had to realize that Leipzig scholars did not have
any interest in theory.\footnote{Klaus-Dieter Betz, ``Ich bin ein Fan
  der Viertelparität geblieben,'' in \emph{``Regierungszeit des
  Mittelbaus?'' Annäherungen an die Berliner Publizistikwissenschaft
  nach der Studentenbewegung}, eds. Maria Löblich, and Niklas Venema
  (Cologne: Halem, 2020), 138.} It was Hoffmann, too, who suggested an
exchange program between Leipzig and West Berlin when he met Emil
Dusiska, then the director of Leipzig's journalism institute, at the
Association Internationale des Etudes et Recherches sur l'Information et
la communication (AIERI) conference in 1970. According to one of
Eberhard's former research assistants, Dusiska declined. He was
supposedly more interested in cooperation with scholars who held some
kind of power in West Germany, such as Noelle-Neumann, who did not
belong to the political opposition.\footnote{Bohrmann, ``Ein politischer
  Habitus, den ich nicht vertreten habe,'' 48.} According to Katharina
M. Mensing, head of the institute's library, selected GDR researchers
received the opportunity to study Western literature at the West Berlin
institute. Once a year, one scholar from Leipzig was granted access to
the institute's library for several weeks.\footnote{Katharina M.
  Mensing, ``Ich habe mich massiv engagiert,'' in \emph{``Regierungszeit
  des Mittelbaus?'' Annäherungen an die Berliner Publizistikwissenschaft
  nach der Studentenbewegung}, eds. Maria Löblich and Niklas Venema
  (Cologne: Halem, 2020), 108.} Our sources did not indicate how often
people from the East made use of this opportunity, but a general lack of
interest on the side of GDR scholars seems to have prevailed.\footnote{Alexander
  von Hoffmann, ``Aufbruch zur wissenschaftlichen
  Journalistenausbildung,'' in \emph{Kommunikationswissenschaft
  autobiographisch: Publizistik Sonderheft,} no. 1 (1997), eds. Arnulf
  Kutsch and Horst Pöttker, 174.} The theoretical approaches of the West
Berlin students did not resonate with the socialist model of practical
journalism education. An academic exchange and further discussion of
possible conceptions of the mass media did not happen.

\hypertarget{scientific-capital-research-and-publications}{%
\section{Scientific Capital: Research and
Publications}\label{scientific-capital-research-and-publications}}

The young scholars' engagement with \emph{kritische Publizistik}
(critical communication studies) did not lead to much scientific capital
in the form of publications.\footnote{ASTA der Freien Universität
  Berlin, \emph{Kritische Universität}, 75--76.} Only some managed or
were willing to mobilize other forms of capital for research and
publications. High student numbers, temporary employment, and a lack of
space contributed to this situation. Moreover, the rules of the
politicized scientific field required time spent in practical work:
media policy, trade union activism, or journalism. Some professors
retreated into private life due to aggressive political fights to obtain
majorities in committees (institutional capital). Those fights have led
some to remember the institute as a ``snake pit.''\footnote{Jürgen
  Prott, \emph{Aufstieg und Identität: Erinnerungen und soziologische
  Reflexionen,} vol. 2\emph{, Erwachsen in Hamburg} (Berlin:
  Autorenverlag K. M. Scheriau, 2018), 103.} Hanno Hardt from Iowa, who
replaced institute director Harry Pross in the late 1970s and later
declined an offered chair, remembered that many of the PhD theses
remained unfinished.\footnote{Hanno Hardt, ``Ein Gegenpol zum
  Mainstream,'' in \emph{Ich habe dieses Fach erfunden}, eds. Michael
  Meyen and Maria Löblich (Cologne: Halem, 2007), 109.} Dissertations
and graduate theses were nevertheless the first academic forms for young
scholars to reflect what \emph{kritische Publizistik} was all about and
to oppose the political situation.\textsuperscript{115}

Some\marginnote{\textsuperscript{115}\setcounter{footnote}{115} Manfred Knoche,
  ``Kommunikationsforschung und Verlegerpolitik: Zur Geschichte und
  Kritik der Publizistikwissenschaft in der BRD und West-Berlin in ihrem
  Verhältnis zur Kommunikationspolitik der Zeitungsverleger
  (1945--1967)'' (master's thesis, Freie Universität Berlin, 1973);
  Meyer, ``Das Berliner Modell''; Ulrich Pätzold, ``Der
  Springer-Arbeitskreis der Kritischen Universität 1967/68: Versuch
  einer publizistikwissenschaftlichen Einordnung'' (master's thesis,
  Freie Universität Berlin, 1970); Hendrik Schmidt, ``Aspekte der
  Diskussion über die Problematik privatwirtschaftlich organisierter
  Massenmedien sowie daraus folgende Konsequenzen für
  kommunikationspolitische Fragestellung und Forschung'' (master's
  thesis, Freie Universität Berlin, 1971).} young researchers had social capital and managed to mobilize
funding for research and publications. There was a productive milieu
with communication researchers and economic scholars at the FU who
published on the issue of political media economy and media
concentration.\footnote{Jörg Aufermann et al., eds.,
  \emph{Pressekonzentration: Eine kritische Materialsichtung und
  -systematisierung} (Munich-Pullach: Verlag Dokumentation, 1970); Jörg
  Aufermann, Bernd-Peter Lange, and Axel Zerdick, ``Pressekonzentration
  in der BRD: Untersuchungsprobleme, Ursachen und Erscheinungsformen,''
  in \emph{Gesellschaftliche Kommunikation und Information}, vol. 1,
  eds. Jörg Aufermann, Hans Bohrmann, and Rolf Sülzer (Frankfurt/Main:
  Fischer-Athenäum, 1973); Autorenkollektiv Presse, \emph{Wie links
  können Journalisten sein? Pressefreiheit und Profit} (Berlin
  {[}West{]}: Rowohlt, 1972); Klaus Kisker, Manfred Knoche, and Axel
  Zerdick, \emph{Wirtschaftskonjunktur und Pressekonzentration in der
  Bundesrepublik Deutschland} (Munich: Saur, 1979); Manfred Knoche,
  \emph{Einführung in die Pressekonzentrationsforschung: Theoretische
  und empirische Grundlagen, kommunikationspolitische Voraussetzungen}
  (Berlin: Spiess, 1978).} A group of young scholars, mainly women, was
interested in critical qualitative media content and media usage
research. Due to their journalistic contacts, they received funding from
public broadcasting. Several books, one award-winning, resulted from
their work.\textsuperscript{117} Research
assistants collaborated with journalists and journalists' trade unions,
filling edited volumes. These volumes contained essays on issues such as
media trade unions, media policy, and journalism education.\textsuperscript{118} Although Pross co-authored a textbook in which
his assistant introduced dialectic-materialist communication
research,\textsuperscript{119} and he
also supported some of the critical research initiatives, professors at
that time were either not involved in research and theory at all or did
not relate to Marxist perspectives.\textsuperscript{120}

The Berlin books by West German authors received mixed reviews in the
most important scholarly journal, \emph{Publizistik}, between 1968 and
1980: while some were descriptive, others openly rejected the material.
Most reviewers followed the standards of academic professionalism to
which the majority of the discipline subscribed, that is, those based on
the U.S. model: mid-range theory, empirical (quantitative) methods, and
value freedom. The distinct characteristics of alternative concepts of
science, such as materialism, were not acknowledged. A small number of
professors, repeatedly engaged in Berlin book reviews, criticized
``ideological\marginnote{\textsuperscript{117} Wolf Bauer, Elke Baur, and Bernd Kungel, \emph{Vier
  Wochen ohne Fernsehen: Eine Studie zum Fernsehkonsum} (Berlin: Spiess,
  1976); Elke Baur, \emph{Wenn Ernie mit der Maus in der Kiste rappelt:
  Vorschulerziehung im Fernsehen} (Frankfurt/Main: Fischer, 1975); Elke
  Baur and Bettina Brentano, ``Fernsehnachrichten: Falsche Themen, leere
  Worte,'' \emph{Psychologie heute} 6, no. 4 (1979): 54--61.} elements,''\marginnote{\textsuperscript{118} Jörg
  Aufermann and Ernst Elitz, eds., \emph{Ausbildungswege zum
  Journalismus: Bestandsaufnahmen, Kritik und Alternativen der
  Journalistenausbildung} (Opladen: Westdeutscher Verlag, 1975); Ulrich
  Pätzold and Hendrik Schmidt, eds., \emph{Solidarität gegen
  Abhängigkeit: Auf dem Weg zur Mediengewerkschaft} (Darmstadt:
  Luchterhand, 1973).} which they identified in the application of a
political economy approach or in references to Marx and
Engels.\textsuperscript{121} One warned: ``Keep away from the role of the
political decision-maker.''\textsuperscript{122}

\hypertarget{renewed-tensions-and-governmental-reorganization-of-the-institute}{%
\subsection{Renewed Tensions and Governmental
Reorganization of the
Institute}\label{renewed-tensions-and-governmental-reorganization-of-the-institute}}

The symbolic power of anticommunism contributed to the end of critical
perspectives. In the FRG, the \emph{Radikalenerlass} (employment ban for
left-wing extremists) of 1972 created a new political climate. One of
the consequences of this political decision, which targeted persons who
participated in activities denounced as anticonstitutional, was
political background checks before employment in public service (in
Berlin, this was applied until 1979). The threat translated into a
potential ban from the profession.\textsuperscript{123} In the second half of the 1970s,
codetermination rights were reduced, and professors had returned to them
the majority of votes in decision-making procedures.\textsuperscript{124} The politicized Department of
Philosophy and Social Sciences, to which the \emph{Institut für
Publizistik} belonged, was continuously involved in contentious\marginnote{\textsuperscript{119} Hanno Beth and Harry Pross, \emph{Einführung in die
  Kommunikationswissenschaft} (Stuttgart: Kohlhammer, 1976).} debate\marginnote{\textsuperscript{120} Löblich and Venema,
  ``\,`Regierungszeit des Mittelbaus'? Eine Einführung.''}
with\marginnote{\textsuperscript{121} For examples, see Winfried B. Lerg, Review of
  \emph{Kommunikation und Modernisierung: Meinungsführer und
  Gemeinschaftsempfang im Kommunikationsprozess}, by Jörg Aufermann,
  \emph{Publizistik} 18 (1973): 184; and Winfried B. Lerg, Review of
  \emph{Gesellschaftliche Kommunikation und Information}, eds., Jörg
  Aufermann, Hans Bohrmann, and Rolf Sülzer, \emph{Publizistik} 19/20
  (1974/75): 864.} the\marginnote{\textsuperscript{122} Franz Ronneberger, Review of
  \emph{Prognosen für Massenmedien als Grundlage der
  Kommunikationspolitik}, by Jan Tonnemacher, \emph{Publizistik} 25
  (1980): 413.} Berlin\marginnote{\textsuperscript{123} Tent, \emph{Freie
  Universität Berlin}, 407.} government\marginnote{\textsuperscript{124}\setcounter{footnote}{124} Christoph
  Nitz and Daniel Siegmund, ``\emph{`}Drittelparität': 1969 bis 1989,''
  in \emph{Geschichte der Freien Universität Berlin. Ereignisse -- Orte
  -- Personen}, eds. Jessica Hoffmann, Helena Seidel, and Nils Baratella
  (Berlin: Frank \& Timme, 2008), 73--85.}. Its letter of sympathy to Erich Honecker,
which criticized the political conditions in the Federal Republic,
became an object of dispute within the political field.\textsuperscript{125} Politically inspired hiring
freezes also targeted \emph{Publizistik}. The small institute, however,
only made for a sideshow, not only in terms of the development of
critical theory but also with regard to political fights. For the
political field, the symbolic function of the FU became an
issue.\textsuperscript{126}\marginnote{\textsuperscript{125} Abgeordnetenhaus
  von Berlin, minutes of July 6, 1978.}\marginnote{\textsuperscript{126} Maria Löblich, ``Eine `grundlegende Neugestaltung': Die
  Westberliner Wissenschaftspolitik und die Publizistikwissenschaft in
  den 1980er-Jahren,'' in \emph{``Regierungszeit des Mittelbaus?''
  Annäherungen an die Berliner Publizistikwissenschaft nach der
  Studentenbewegung}, eds. Maria Löblich and Niklas Venema (Cologne:
  Halem, 2020).}\marginnote{\textsuperscript{127} Abgeordnetenhaus von
  Berlin, minutes of June 24 1976, 1277.}\marginnote{\textsuperscript{128} Abgeordnetenhaus von Berlin, minutes of July 6,
  1978, 3936, 3940.}\marginnote{\textsuperscript{129} Abgeordnetenhaus von
  Berlin, minutes of July 6, 1978, 3937.}

Conservative politicians, in opposition to West Berlin's parliament in
the 1970s, employed anticommunist language. They fought what was, in
their view, the FU's ``communist infiltration.'' They accused the ruling
Social Democrats of colluding with ``communists,'' for instance, during
the university's presidential election.\textsuperscript{127} They blamed the Social
Democrats for the situation at the Philosophy and Social Science
Department, where they claimed ``socialist unity science'' dominated and
the Socialist Unity Party of West Berlin steered employment
decisions.\textsuperscript{128} They demanded the restoration of the ``freedom of
science,'' originally introduced and long been secured, they said, by
``our greatest protector, America.''\textsuperscript{129} In that spirit, a network of
conservative parliamentarians and professors campaigned and also sent
political reports about the FU to founding institutions in the United
States.\textsuperscript{130} When the FU was founded, donations of hot
meals and clothing had come from the United States, and the Radio~in
the American Sector (RIAS) of Berlin had promoted help for the new instit\marginnote{\textsuperscript{130}\setcounter{footnote}{130} Nikolai Wehrs, \emph{Protest der Professoren: Der
  ``Bund Freiheit der Wissenschaft'' in den 1970er Jahren} (Göttingen:
  Wallstein, 2014), 285.}ution's students.\footnote{Kubicki and Lönnendonker, \emph{Die
  Freie Universität}, 35.} The university had received millions of
Deutschmark subsidies, personally decided on by the U.S. High
Commissioner for Germany, John J. McCloy. In the early 1950s, the
subsidies exceeded the financial support received by all other
comparable institutions in the Federal Republic.\footnote{Tent,
  \emph{Freie Universität Berlin}, 214--19.} Around 1950, money from
private and public U.S. sources constituted the second most important
financial source after budgetary support from West Germany. The Ford
Foundation became a major funder. It financed the construction of large
buildings at the Dahlem Campus within the American occupational zone.
One of the most famous was named the Henry-Ford-Bau (Henry Ford
Building).\footnote{Tent, \emph{Freie Universität Berlin,} 222.} From
1948 to 1967, the United States provided the FU with a total of 79.5
million German Marks.\footnote{Bergmann, ``Die Hilfe der USA,'' 189.}

The local Social Democratic government defended itself by emphasizing
that the Social Democratic Party in West Berlin had pursued strict
anticommunist policies from 1945 forward. The governing mayor of West
Berlin replied to the conservatives in parliament: ``When we struggled
with communists in West Berlin and East Berlin, most of you did not even
know how to spell the word `freedom.'\,''\footnote{Abgeordnetenhaus von
  Berlin, minutes of June 24, 1976, 1282.} Peter Glotz, the SPD minister
of science in Berlin between 1977 and 1981, wrote in retrospect: ``The
Berlin Social Democrats feared that the conservatives would misrepresent
them as communist friendly. Leading civil servants in his ministry had
been trained as `communist eaters.'\,''\footnote{Peter Glotz, \emph{Von
  Heimat zu Heimat: Erinnerungen eines Grenzgängers} (Berlin: Econ,
  2005), 164.} Glotz emphasized in his memoirs the same argument applied
by parliamentary conservatives at that time: The communist infiltration
had become a ``severe harm'' for the image of West Berlin.\footnote{Glotz,
  \emph{Von Heimat zu Heimat}, 169; cf. Tent, \emph{Freie Universität
  Berlin}, 402--3.} Also, in view of the federal SPD government, which,
together with the federal states, had passed the \emph{Radikalenerlass},
we can understand why West Berlin's government had to admit ``defects''
at the FU, especially in the department to which communication studies
belonged.\footnote{Abgeordnetenhaus von Berlin, minutes if July 6, 1978,
  3938.}
  
\enlargethispage{-\baselineskip}

It was Glotz, the former communications scholar educated in Munich, who
dissolved this ``chemically pure left'' department.\footnote{Glotz,
  \emph{Von Heimat zu Heimat}, 169, 161.} Under his direction, and in
conversation with members of the institute, plans for a restructuring of
\emph{Publizistik} took shape. He also promised to create more
professorships. At the end of Glotz's time in office, he had managed to
fill only one of the seven promised professorships. Jobs for academic
staff even saw a reduction. Glotz followed a decision made by the FU's
board of trustees, over which he presided. The institution had
identified this academic status group, which advocated for critical
perspectives, as having caused the political problem.

According to Bourdieu, symbolic power may have consequences for
objective structures. Following this perspective, we can understand the
Social Democratic and subsequent conservative governmental interventions
into the Berlin institute as a consequence of anticommunist language. In 1981, the local West Berlin government changed from a social democratic to a conservative one; one year later the conservatives also won the federal election. The Christian Democrat minister of science in Berlin
personally took care of the ``fundamental reorganization'' of West
Berlin's \emph{Publizistik},\footnote{Senator für Wissenschaft und
  Kulturelle Angelegenheiten, Langschied. 11. Februar 1983. FUA, VP2 FB
  Komwi, Publizistik `Expertengremium' 5/1982 bis 6/83.} which he
announced in the local newspaper, \emph{Der Tagesspiegel}. The minister
described communication studies as a ``concrete problem.''\footnote{Wilhelm
  Kewenig, \emph{Reden und Aufsätze} (Berlin: Senator für Wissenschaft
  und Forschung, 1984), 62--63.} He installed an external experts'
committee to make new plans for the institute. Its members were all
conservatives, among them Noelle-Neumann, then a professor in Mainz and
a conservative party campaign advisor who had written her dissertation
under the supervision of Dovifat. The committee's paper suggested how
research and curriculum would achieve the ``level of development'' of
the discipline, ``also internationally.'' To ``reduce ideologization,''
it specified, ``empirical communication studies'' and the ``training in
methods'' should become a priority at the undergraduate level. Given the
geographic location of the institute, ``communist communication
systems'' could be a topic for research, though research on this issue
was often burdened by ``political . . . intentions,'' it claimed.
Moreover, the external assessment suggested remembering Dovifat's
achievements.\footnote{Gutachten des Beratungsgremiums für den
  Studiengang Publizistik an der Freien Universität Berlin, 24. Anlage
  zum Schreiben des Senators für Wissenschaft und Forschung an den
  Sprecher des Fachbereichs Kommunikationswissenschaften am 25. April
  1983. FUA, VP2 FB Komwi, Publizistik `Expertengremium' 5/1982 bis
  6/83.}

The minister of science followed these suggestions and appointed three
scholars as chairs, which became institutional cornerstones. The chairs
were dedicated to empirical research, communication history, and
journalism. The men taking up these positions had had nothing to do with
communication science until then, and nothing to do with critical
perspectives. One of them, a trained quantitative researcher, left the
University of Michigan for West Berlin after Noelle-Neumann had
approached him.\footnote{Lutz Erbring, ``Ausbildung ist eine Pflicht und
  keine Kür,'' in \emph{Ich habe dieses Fach erfunden}, eds. Michael
  Meyen and Maria Löblich, (Cologne: Halem, 2007), 255.} In the
following years, further professorships continued to enlarge these three
cornerstone areas.

The governmental intervention was met with a mixture of protest and
relief at the institute. While von Hoffmann, the media practice
professor, rejected the intervention in his farewell speech as a final
strike of ``persistent attempts from the right to smash this department
being one of the last places of refuge for left, critical
science,''\footnote{Alexander von Hoffmann, ``Schlussbemerkungen eines
  Spätaufklärers: Abschiedsrede am 12. Februar 1988,'' \emph{medium} 2
  (1988): 12.} other scholars welcomed it. A postdoc researcher at that
time said that only this intervention had brought ``normal academic
professionalism.''\footnote{Günter Bentele, ``Das war die Zeit, als der
  Mittelbau regiert hat,'' in \emph{``Regierungszeit des Mittelbaus?''
  Annäherungen an die Berliner Publizistikwissenschaft nach der
  Studentenbewegung}, eds. Maria Löblich and Niklas Venema (Cologne:
  Halem, 2020), 132.} Sources indicated that the Social Democratic
intervention in the 1970s had not already been rejected by all young
scholars, especially not by those who had prepared for a career in the
discipline. They had to consider the rules of communication studies in
the Federal Republic, rules we can also see as a product of the Cold
War. The field rejected the ideas of the student movement. Moreover,
around 1980, activism for societal critical approaches had diminished,
and the number of politicized students in West Berlin had shrunk.
Critical young scholars had left the institute because of a scarcity of
positions or unfinished dissertations. Few supporters of critical
perspectives remained in job positions. Against this background, the
revival of the Cold War warrior Emil Dovifat, suggested by the
conservative advisors of the minister and supported by some actors
within the institute, seem more comprehensible.\footnote{Schreiben des
  Geschäftsführenden Direktors, Institut für Semiotik und
  Kommunikationstheorie, an den Senator für Wissenschaft und Kulturelle
  Angelegenheiten Wilhelm Kewenig am 23. Februar 1983. FUA, VP2 FB
  Komwi, Publizistik `Expertengremium' 5/1982 bis 6/83.}

\hypertarget{conclusion}{%
\section{Conclusion}\label{conclusion}}

We examined the thesis that the spread of societal and media critical
approaches and the failure of their institutionalization at the West
Berlin institute from 1948 to 1989 were linked to the Cold War and,
particularly, to the anticommunist discourse of the West. Critical
communication studies in West Berlin for more than forty years were both
facilitated and restrained by the East-West conflict as this conflict
experienced periods of high and low intensity.

Three periods describe the history of this linkage. In the first period,
when the Berlin Wall was constructed, an impulse emerged for the
reception of theories such as Marxism and those of the Frankfurt School
among communication students engaged in the 1968 movement. The main
orthodox anticommunist at the institute retired at the time, replaced by
a new professor caught between sympathy for critical perspectives and
the discursive requirements of the Cold War. In the second period,
during the policy of détente, the Berlin institute took steps to
institutionalize societal critical approaches in the curriculum and in
research. A moderate increase in paid jobs and a strong increase in
student numbers drove this process. University reform likewise
facilitated these steps of institutionalization, resulting in the
university's politicization. At the time, students and mid-level
research staff obtained codetermination rights. Political camps fought
for institutional capital, for instance, for seats and majorities in
appointment committees where decisions over careers and, thus, theory
were made. Most professors, actors in positions with long-term contracts
and economic capital, did not engage in the accumulation of scientific
capital to consolidate (the reputation of) critical approaches. The era
of critical perspectives ended during the third period, when Cold War
tensions renewed and the state intervened at the institute. In that
period, the political field's anticommunist discourse produced the
dissolution of department structures, as decided on by the Social
Democratic government. A few years later, the conservative West Berlin
government restructured communication studies, leading---following the
advice of a conservative external experts' committee---to the
appointment of a group of scholars who had no interest in Marxist
perspectives.

Although U.S. institutions and money directly shaped biographies and the
foundation of the FU, we have made efforts to avoid oversimplification.
Bourdieu's analytical concepts (habitus, capital, field, and symbolic
power) helped us understand the complex and, in part, ambivalent link
between the Cold War and the history of critical perspectives. The
symbolic power of the Cold War became embedded in the thinking,
speaking, and writing of communications scholars; however, depending on
the individual opus operatum, it led to different modi operandi with
regard to critical perspectives. This insight also applies to actors of
the same generation such as Emil Dovifat and Fritz Eberhard. One felt
politically convinced by anticommunism and knew that it provided an
opportunity to regain a lost reputation. The other interpreted the
pro-Western discourse in a way that gave him the scope to sympathize
with the student movement. Beyond that, both of these men, who were
reaching the end of their careers, still wanted to earn their money.
Young scholars around 1970 opposed this discourse, including the U.S.
role model and media capitalism. Yet, over time, those opting for a
career in the discipline had to reconcile Marxist conformity with the
rules of the broader communications field in West Germany. The common
conception of this field, which had just started to recover from its
deep postwar crisis, implied an orientation toward U.S. mass
communication research and a clear distance from Marxist approaches.
Furthermore, from Bourdieu's perspective, linkages with other fields and
even other societal systems may have been considered. The symbolic power
of anticommunism shaped the Berlin institute through the political
field, which intervened via legal conditions, financing, and appointment
procedures. Yet politically motivated interventions into communication
studies, for instance via appointments, also occurred in other parts of
the Federal Republic beyond West Berlin.\footnote{Alexis Mirbach,
  ``Beate Schneider,'' in \emph{Biografisches Lexikon der
  Kommunikationswissenschaft,} eds\emph{.} Michael Meyen and Thomas
  Wiedemann (Cologne: Halem, 2014).} This heteronomy resulted from the
field's general problem accumulating scientific capital and gaining
legitimacy in the period under study, as well as from its entanglement
with Cold War politics. While we focused on the Western side of the Cold
War, the Eastern influence on the institute also warrants an in-depth
investigation. In that regard, our analysis suggested that scholars
engaged in GDR journalism studies tended to decline collaborative
initiatives proffered by young scholars from West Berlin.







\section{Bibliography}\label{bibliography}

\begin{hangparas}{.25in}{1} 



Abgeordnetenhaus von Berlin. Minutes, June 24, 1976.

Abgeordnetenhaus von Berlin. Minutes, July 6, 1978.

Ampuja, Marko. ``Four Moments of International Communication Research in
the Cold War and Beyond.'' \emph{Javnost -- The Public} 26, no. 4
(2019): 347--62.

ASTA der Freien Universität Berlin. \emph{Kritische Universität: Sommer
68 -- Berichte und Programm}. Berlin, 1968.

Aufermann, Jörg, and Ernst Elitz, eds. \emph{Ausbildungswege zum
Journalismus: Bestandsaufnahmen, Kritik und Alternativen der
Journalistenausbildung}. Opladen: Westdeutscher Verlag, 1975.

Aufermann, Jörg, Peter Heilmann, Hubertus Hüppauf, Wolfgang C. Müller,
Ulrich Neveling, and Gernot Wersig, eds. \emph{Pressekonzentration: Eine
kritische Materialsichtung und -systematisierung}. Munich-Pullach:
Verlag Dokumentation, 1970.

Aufermann, Jörg, Bernd-Peter Lange, and Axel Zerdick.
``Pressekonzentration in der BRD: Untersuchungsprobleme, Ursachen und
Erscheinungsformen.'' In \emph{Gesellschaftliche Kommunikation und
Information}, vol. 1, edited by Jörg Aufermann, Hans Bohrmann, and Rolf
Sülzer, 242--302. Frankfurt/Main: Fischer-Athenäum, 1973.

Autorenkollektiv Presse. \emph{Wie links können Journalisten sein?
Pressefreiheit und Profit.} Berlin (West): Rowohlt, 1972.

Baerns, Barbara. ``Eine Brücke schaffen zwischen Theorie und Praxis.''
In \emph{Ich habe dieses Fach erfunden}, edited by Michael Meyen and
Maria Löblich, 262--80. Cologne: Halem, 2007.

Barthenheier, Günther, and Werner Hoffmann. \emph{IfP 1978: Eine
Dokumentation zum 30jährigen Bestehen des Instituts für Publizistik}.
Berlin: Freie Universität Berlin, 1978.

Bastiansen, Henrik G. ``Norwegian Media and the Cold War, 1945--1991.''
\emph{Nordicom Review} 35 (2014): 155--69.

Bauer, Wolf, Elke Baur, and Bernd Kungel. \emph{Vier Wochen ohne
Fernsehen: Eine Studie zum Fernsehkonsum}. Berlin: Spiess, 1976.

Baur, Elke. \emph{Wenn Ernie mit der Maus in der Kiste rappelt:
Vorschulerziehung im Fernsehen}. Frankfurt/Main: Fischer, 1975.

Baur, Elke, and Bettina Brentano. ``Fernsehnachrichten: Falsche Themen,
leere Worte.'' \emph{Psychologie heute} 6, no. 4 (1979): 54--61.

Benedikt, Klaus-Ulrich. \emph{Emil Dovifat: Ein katholischer
Hochschullehrer und Publizist}. Mainz: Matthias Grünwald, 1986.

Bentele, Günter. ``Das war die Zeit, als der Mittelbau regiert hat.'' In
\emph{``Regierungszeit des Mittelbaus?'' Annäherungen an die Berliner
Publizistikwissenschaft nach der Studentenbewegung}, edited by Maria
Löblich and Niklas Venema, 121--34. Cologne: Halem, 2020.

Bergmann, Fritz von. ``Die Hilfe der USA für die Freie Universität
Berlin.'' In \emph{Freie Universität Berlin 1948--1973: Hochschule im
Umbruch; Dokumentation III}, edited by Siegward Lönnendonker and Tilman
Fichter, 189--93. Berlin: Freie Universität Berlin, 1974.

Berwanger, Dietrich. ``Die Ankunft am Berliner Institut.'' In
\emph{Fritz Eberhard: Rückblicke auf Biographie und Werk}, edited by
Bernd Sösemann, 19--23. Stuttgart: Franz Steiner, 2001.

Beth, Hanno, and Harry Pross. \emph{Einführung in die
Kommunikationswissenschaft}. Stuttgart: Kohlhammer, 1976.

Betz, Klaus-Dieter. ``Ich bin ein Fan der Viertelparität geblieben.'' In
\emph{``Regierungszeit des Mittelbaus?}'' \emph{Annäherungen an die
Berliner Publizistikwissenschaft nach der Studentenbewegung}, edited by
Maria Löblich and Niklas Venema, 135--48. Cologne: Halem, 2020.

Beucke, Stefan, Jochen Meiring, and Maximilian Russ. ``Konrad
Adenauer.'' In \emph{Medienkanzler: Politische Kommunikation in der
Kanzlerdemokratie}, edited by Thomas Birkner, 45--74. Wiesbaden: VS,
2016.

Bohrmann, Hans. ``Ein politischer Habitus, den ich nicht vertreten
habe.'' In \emph{``Regierungszeit des Mittelbaus?'' Annäherungen an die
Berliner Publizistikwissenschaft nach der Studentenbewegung}, edited by
Maria Löblich and Niklas Venema, 44--53. Cologne: Halem, 2020.

Bohrmann, Hans. ``Elisabeth Löckenhoff im Institut für Publizistik der
Freien Universität Berlin (1952--1985).'' In \emph{Publizistik und
Journalismus in der DDR: Acht Beiträge zum Gedenken an Elisabeth
Löckenhoff}, edited by Rolf Geserick and Arnulf Kutsch, 17--35. Munich:
K. G. Saur, 1988.

Bohrmann, Hans. ``Fritz Eberhard als Förderer und Anreger der
Kommunikationswissenschaft.'' \emph{Fritz Eberhard: Rückblicke auf
Biographie und Werk}, edited by Bernd Sösemann, 246--56. Stuttgart:
Franz Steiner, 2001.

Bohrmann, Hans. ``Habilitation von Dr. Elisabeth Löckenhoff.''
\emph{Publizistik} 17 (1972): 224.

Bourdieu, Pierre. \emph{Language and Symbolic Power}. Translated by Gino
Raymond and Matthew Adamson. Cambridge: Polity Press, 1992.

Bourdieu, Pierre. \emph{Sozialer Sinn: Kritik der theoretischen
Vernunft}. Frankfurt/Main: Suhrkamp, 1993.

Bourdieu, Pierre. \emph{Vom Gebrauch der Wissenschaft: Für eine
klinische Soziologie des wissenschaftlichen Feldes}. Constance: UVK,
1998.

Cmiel, Kenneth. ``On Cynicism, Evil, and the Discovery of Communication
in the 1940s.'' \emph{Journal of Communication} 46, no. 3 (1996):
88--107.

Delia, Jesse G. ``Communication Research: A History.'' In \emph{Handbook
of Communication Science}, edited by Charles R. Berger and Steven H.
Chaffee, 20--98. Beverly Hills, CA: Sage, 1987.

Dovifat, Emil. \emph{Der NWDR in Berlin 1946--1954.} Berlin: Haude \&
Spenersche Verlagsbuchhandlung, 1970.

Dovifat, Emil. ``Freiheit und Zwang in der politischen Willensbildung:
Formen der de-mokratischen und der totalitären Meinungsführung.`` In
\emph{Veritas, iustitia, libertas: Festschrift zur 200-Jahresfeier der
Columbia University New York}, edited by Freie Universität Berlin,
34--48. Berlin: Colloquium, 1954.

Dovifat, Emil. \emph{Handbuch der Publizistik}. Vol. 1, \emph{Allgemeine
Publizistik}. Berlin: De Gruyter, 1968.

Dovifat, Emil. ``Publizistik als Wissenschaft: Herkunft -- Wesen --
Aufgabe.'' \emph{Publizistik} 1 (1956): 3--10.

Eberhard, Fritz. ``Gang durch den deutschen Blätterwald.'' \emph{Die
Zeit}, October 10, 1965.

Eberhard, Fritz. ``\,'Kritik muß zersetzen': Auch Anpassung kann die
Freiheit gefährden.'' \emph{Die Zeit}, June 30, 1967.

Eberhard, Fritz. ``Macht durch Massenmedien?'' \emph{Publizistik} 10
(1965): 477--94.

Eberhard, Fritz. ``Thesen zur Publizistikwissenschaft.''
\emph{Publizistik} 6 (1961): 260--61.

Eberhard, Fritz. ``Wie informiert das Fernsehen?'' \emph{Die Zeit,}
September 23, 1966.

Eisenhuth, Stefanie, and Scott H. Krause, ``Inventing the `Outpost of
Freedom': Transatlantic Narratives and the Historical Actors Crafting
West Berlin's Postwar Political Culture.'' \emph{Zeithistorische
Forschungen/Studies in Contemporary History} 11, no. 2 (2014): 188--211.

Erbring, Lutz. ``Ausbildung ist eine Pflicht und keine Kür.'' In
\emph{Ich habe dieses Fach erfunden}, edited by Michael Meyen and Maria
Löblich, 246--61. Cologne: Halem, 2007.

Fachbereichsrat des Fachbereichs Philosophie und Sozialwissenschaften,
ed. \emph{Akzente einer Studienreform: Dokumentation der Studienplanung
im Fachbereich Philosophie und Sozialwissenschaften der Freien
Universität Berlin}. Berlin: Freie Universität Berlin, 1975.

Finckenstein, Stefan Graf Finck von. ``Vita in Stichworten.'' In
\emph{Fritz Eberhard: Rückblicke auf Biographie und Werk}, edited by
Bernd Sösemann, 73--82. Stuttgart: Franz Steiner, 2001.

Glotz, Peter. \emph{Von Heimat zu Heimat: Erinnerungen eines
Grenzgängers}. Berlin: Econ, 2005.

Gollwitzer, Helmut. ``Ein Go-Out der Professoren.'' In
\emph{Rundfunkpolitische Kontroversen: Zum 80. Geburtstag von Fritz
Eberhard}, edited by Manfred Kötterheinrich, Ulrich Neveling, Ulrich
Paetzold, and Hendrik Schmidt, 479--86. Frankfurt/Main: Europäische
Verlagsanstalt, 1976.

Gransow, Volker. ``Zur kulturpolitischen Entwicklung in der Deutschen
Demokratischen Republik bis 1973.'' PhD diss., Freie Universität Berlin,
1974.

Groos, Peter. ``Vision oder Zwangslage? Fritz Eberhards Positionierung
in der akademischen Publizistik an der Freien Universität Berlin.'' In
\emph{Fritz Eberhard: Rückblicke auf Biographie und Werk}, edited by
Bernd Sösemann, 257--71. Stuttgart: Franz Steiner, 2001.

Hardt, Hanno. ``Am Vergessen scheitern: Essay zur historischen Identität
der Publizistikwissenschaft, 1945--68.'' \emph{Medien \& Zeit} 17, no.
2/3 (2002): 34--39.

Hardt, Hanno. ``Ein Gegenpol zum Mainstream.'' In \emph{Ich habe dieses
Fach erfunden}, edited by Michael Meyen and Maria Löblich, 101--15.
Cologne: Halem, 2007.

Hoffmann, Alexander von. ``Aufbruch zur wissenschaftlichen
Journalistenausbildung.'' In \emph{Kommunikationswissenschaft
autobiographisch:} \emph{Publizistik Sonderheft}, no. 1 (1997), edited
by Arnulf Kutsch and Horst Pöttker, 161--83.

Hoffmann, Alexander von. ``Schlussbemerkungen eines Spätaufklärers:
Abschiedsrede am 12. Februar 1988.'' \emph{medium} 2 (1988): 9--17.

Hoffmann, Burkhard. ``On the Development of a Materialist Theory of Mass
Communications in West Germany.'' \emph{Media, Culture \& Society} 5,
no. 1 (1983): 7--24.

Hoffmann, Burkhard. ``Zum Problem der Entwicklung einer
materialistischen Kommunikationstheorie.'' In \emph{Gesellschaftliche
Kommunikation und Information}, vol. 2, edited by Jörg Aufermann, Hans
Bohrmann, and Rolf Sülzer, 191--206. Frankfurt/Main: Athenäum-Fischer,
1973.

Institut für Publizistik, Fachbereich Kommunikationswissenschaften, FU
Berlin. \emph{Publizistik in Berlin 82}. Berlin: Freie Universität
Berlin, 1982.

Kewenig, Wilhelm. \emph{Reden und Aufsätze}. Berlin: Senator für
Wissenschaft und Forschung, 1984.

Kisker, Klaus, Manfred Knoche, and Axel Zerdick.
\emph{Wirtschaftskonjunktur und Pressekonzentration in der
Bundesrepublik Deutschland}. Munich: Saur, 1979.

Knoche, Manfred. \emph{Einführung in die Pressekonzentrationsforschung:
Theoretische und empirische Grundlagen, kommunikationspolitische
Voraussetzungen}. Berlin: Spiess, 1978.

Knoche, Manfred. ``Kommunikationsforschung und Verlegerpolitik: Zur
Geschichte und Kritik der Publizistikwissenschaft in der BRD und
West-Berlin in ihrem Verhältnis zur Kommunikationspolitik der
Zeitungsverleger (1945--1967).'' Master's thesis, Freie Universität
Berlin, 1973.

Kötterheinrich, Manfred, Ulrich Neveling, Ulrich Paetzold, and Hendrik
Schmidt. ``Vorwort.'' In \emph{Runfunkpolitische Kontroversen: Zum 80.
Geburstag von Fritz Eberhard}, edited by Manfred Kötterheinrich, Ulrich
Neveling, Ulrich Paetzold, and Hendrik Schmidt, 9--17. Frankfurt/Main:
Europäische Verlagsanstalt, 1976.

Kubicki, Karol, and Siegward Lönnendonker. \emph{Die Freie Universität
Berlin 1948--2007: Von der Gründung bis zum Exzellenzwettbewerb.}
Göttingen: V\&R Unipress, 2008.

Kuhnau, Gunther. ``Lummers Moral.'' \emph{Berliner Zeitung,} February 8,
1961.

Lerg, Winfried B. Review of \emph{Gesellschaftliche Kommunikation und
Information}, edited by Jörg Aufermann, Hans Bohrmann, and Rolf Sülzer,
\emph{Publizistik} 19/20 (1974/75): 864.

Lerg, Winfried B. Review of \emph{Kommunikation und Modernisierung:
Meinungsführer und Gemeinschaftsempfang im Kommunikationsprozess}, by
Jörg Aufermann, \emph{Publizistik} 18 (1973): 184.

Löblich, Maria. \emph{Die empirisch-sozialwissenschaftliche Wende in der
Zeitungs- und Publizistikwissenschaft.} Cologne: Halem, 2010.

Löblich, Maria. ``Eine `grundlegende Neugestaltung': Die Westberliner
Wissenschaftspolitik und die Publizistikwissenschaft in den
1980er-Jahren.'' In \emph{``Regierungszeit des Mittelbaus?''
Annäherungen an die Berliner Publizistikwissenschaft nach der
Studentenbewegung}, edited by Maria Löblich and Niklas Venema, 490--544.
Cologne: Halem, 2020.

Löblich, Maria. ``German \emph{Publizistikwissenschaft} and Its Shift
from a Humanistic to an Empirical Social Scientific Discipline:
Elisabeth Noelle-Neumann, Emil Dovifat, and the \emph{Publizistik}
Debate.'' \emph{European Journal of Communication} 22, no. 1 (2007):
69-88.

Löblich, Maria, and Andreas Scheu. ``Writing the History of
Communication Studies: A Sociology of Science Approach.''
\emph{Communication Theory} 21, no. 1 (2011): 1--22.

Löblich, Maria, and Niklas Venema. ``\,`Regierungszeit des Mittelbaus'?
Eine Einführung.'' In \emph{``Regierungszeit des Mittelbaus?''
Annäherungen an die Berliner Publizistikwissenschaft nach der
Studentenbewegung}, edited by Maria Löblich and Niklas Venema, 9--42.
Cologne: Halem, 2020.

Löblich, Maria, and Niklas Venema, eds. \emph{``Regierungszeit des
Mittelbaus?'' Annäherungen an die Berliner Publizistikwissenschaft nach
der Studentenbewegung.} Cologne: Halem, 2020.

Lundestad, Geir. \emph{East, West, North, South: International Relations
since 1945}. Los Angeles: Sage, 2014.

Mensing, Katharina M. ``Ich habe mich massiv engagiert.'' In
\emph{``Regierungszeit des Mittelbaus?'' Annäherungen an die Berliner
Publizistikwissenschaft nach der Studentenbewegung}, edited by Maria
Löblich and Niklas Venema, 100--108. Cologne: Halem, 2020.

Meyen, Michael. ``IAMCR on the East-West Battlefield: A Study on the
GDR's Attempts to Use the Association for Diplomatic Purposes.''
\emph{International Journal of Communication} 8 (2014): 2071--89.

Meyen, Michael, Kaarle Nordenstreng, Carlos Barrera, and Walery Pisarek.
``Media and the Cold War: The East/West Conflict.'' In \emph{The
Handbook of European Communication History}, edited by Klaus Arnold,
Pascal Preston, and Susanne Kinnebrock, 205--20. Hoboken, NJ:
Wiley-Blackwell, 2020.

Meyer, Bernd. ``Das Berliner Modell: Ausbildung von
Kommunikationspraktikern am Institut für Publizistik der FU Berlin;
Darstellung und Entwicklung, kritische Bestandsaufnahme und Perspektiven
dieses Konzeptes.'' Master's thesis, Freie Universität Berlin, 1979.

Mirbach, Alexis. ``Beate Schneider.'' In \emph{Biographisches Lexikon
der Kommunikationswissenschaft}, edited by Michael Meyen and Thomas
Wiedemann. Cologne: Halem, 2020.
http://blexkom.halemverlag.de/beate-schneider/ (November 22, 2021)

Neveling, Ulrich. ``Man musste das Vertrauen haben.'' In
\emph{``Regierungszeit des Mittelbaus?'' Annäherungen an die Berliner
Publizistikwissenschaft nach der Studentenbewegung}, edited by Maria
Löblich and Niklas Venema, 54--65. Cologne: Halem, 2020.

Nietzel, Benno. ``Propaganda, Psychological Warfare, and Communication
Research in the USA and the Soviet Union during the Cold War.''
\emph{History of the Human Sciences} 29, no. 4--5 (2016): 59--67.

Nitz, Christoph, and Daniel Siegmund. ``\,`Drittelparität': 1969 bis
1989.'' In \emph{Geschichte der Freien Universität Berlin. Ereignisse --
Orte -- Personen}, edited byJessica Hoffmann, Helena Seidel, and Nils
Baratella, 73--85. Berlin: Frank \& Timme, 2008.

Of {[}\emph{sic}{]}. ``Jetzt auf Honorarbasis.'' \emph{Berliner
Zeitung}, May 26, 1961, 12.

Park, David W. ``Pierre Bourdieu und die Geschichte des
kommunikationswissenschaftlichen Feldes: Auf dem Weg zu einem reflexiven
und konfliktorientierten Verständnis der Fachentwicklung.'' In
\emph{Pierre Bourdieu und die Kommunikationswissenschaft: Internationale
Perspektiven}, edited by Thomas Wiedemann and Michael Meyen, 123--45.
Cologne: Halem, 2013.

Parry-Giles, Shawn J. ``Propaganda, Effect, and the Cold War: Gauging
the Status of America's `War of Words.'\,'' \emph{Political
Communication} 11, no. 2 (1994): 203­--13.

Pätzold, Ulrich. ``Der Springer-Arbeitskreis der Kritischen Universität
1967/68: Versuch einer publizistikwissenschaftlichen Einordnung.''
Master's thesis, Freie Universität Berlin, 1970.

Pätzold, Ulrich. ``Ohne Berliner Modell wäre ich nie in Dortmund
gelandet.'' In \emph{``Regierungszeit des Mittelbaus?'' Annäherungen an
die Berliner Publizistikwissenschaft nach der Studentenbewegung}, edited
by Maria Löblich and Niklas Venema, 66--78. Cologne: Halem, 2020.

Pätzold, Ulrich, and Hendrik Schmidt, eds. \emph{Solidarität gegen
Abhängigkeit: Auf dem Weg zur Mediengewerkschaft}. Darmstadt:
Luchterhand, 1973.

Pfeiffer, Juliane. ``Die (Re-)Konstruktion der Vorgeschichte des
Instituts für Publizistik (- und Kommunikationswissenschaft) an der
Freien Universität Berlin (1948--1998): Unter besonderer
Berücksichtigung der Darstellung der Rolle Emil Dovifats in der Zeit des
Nationalsozialismus.'' Master's thesis, Freie Universität Berlin, 2015.

Pfeiffer, Juliane. ``Nicht-Wissen oder Nicht-Wissen-Wollen? Die
Auseinandersetzung mit der NS-Vergangenheit Emil Dovifats am Berliner
Institut für Publizistik in den `langen Sechzigerjahren.'\,'' In
\emph{``Regierungszeit des Mittelbaus?'' Annäherungen an die Berliner
Publizistikwissenschaft nach der Studentenbewegung}, edited by Maria
Löblich and Niklas Venema, 396--432. Cologne: Halem, 2020.

Pross, Harry. ``Das Berliner Modell.'' In \emph{Journalistenausbildung:
Modelle, Erfahrungen, Analysen}, edited by Walter Hömberg, 149--58.
Munich: Ölschläger, 1978.

Pross, Harry. \emph{Memoiren eines Inländers: 1923--1993}. Munich:
Artemis \& Winkler, 1993.

Pross, Harry. ``Uni mit Feuer.'' \emph{Zeit-Magazin}, December 7, 1973.

Prott, Jürgen. \emph{Aufstieg und Identität: Erinnerungen und
soziologische Reflexionen}. Vol. 2\emph{, Erwachsen in Hamburg}. Berlin:
Autorenverlag K.M. Scheriau, 2018.

Raabe, Hans-Joachim. ``Emil Dovifats Lehre von der Publizistik.'' PhD
diss., Universität Leipzig, 1962.

Rajagopal, Arvind. ``A View on the History of Media Theory from the
Global South.'' \emph{Javnost -- The Public} 24, no. 4 (2019): 407--19.

Risso, Linda. ``Radio Wars: Broadcasting in the Cold War.'' \emph{Cold
War History} 13, no. 2 (2013): 145--52.

Ronneberger, Franz. Review of \emph{Prognosen für Massenmedien als
Grundlage der Kommunikationspolitik}, by Jan Tonnemacher.
\emph{Publizistik} 25 (1980): 413.

Scheu, Andreas. \emph{Adornos Erben in der Kommunikationswissenschaft:
Eine Verdrängungsgeschichte?} Cologne: Halem, 2012.

Scheu, Andreas, and Thomas Wiedemann. ``Kommunikationswissenschaft als
Gesellschaftskritik: Die Ablehnung linker Theorien in der deutschen
Kommunikationswissenschaft am Beispiel Horst Holzer.'' \emph{Medien \&
Zeit} 23, no. 4 (2008): 9--17.

Schmidt, Hendrik. ``Aspekte der Diskussion über die Problematik
privatwirtschaftlich organisierter Massenmedien sowie daraus folgende
Konsequenzen für kommunikationspolitische Fragestellung und Forschung.''
Master's thesis, Freie Universität Berlin, 1971.

Schneider, Peter, Rolf Sülzer, and Wilbert Ubbens. ``Pressekonformität
und studentischer Protest: Die West-Berliner Tagespresse analysiert
anhand ihrer Berichterstattung über studentische Aktivitäten aus Anlass
des Besuchs des Schah von Persien vor dem Hintergrund der allgemeinen
Hochschulberichterstattung in den Monaten April-Juli 1967 im Vergleich
mit ausgewählten Tageszeitungen ausserhalb Berlins: Eine statistisch
vergleichende Aussagenanalyse.`` Manuscript, Institut für Publizistik
der Freien Universität, 1969.

Simonson, Peter, and David W. Park. ``On the History of Communication
Study.'' In \emph{The International History of Communication Study},
edited by Peter Simonson and David W. Park, 1--22. New York: Routledge,
2016.

Simonson, Peter, and John Durham Peters. ``Communication and Media
Studies, History to 1968.'' In \emph{The International Encyclopedia of
Communication}, edited by Wolfgang Donsbach, 1--8. Hoboken, NJ: Wiley,
2008.

Simpson, Christopher. \emph{Science of Coercion: Communication Research
and Psychological Warfare, 1945--1960.} New York: Oxford University
Press, 1994.

Sösemann, Bernd, ed. \emph{Emil Dovifat: Studien und Dokumente zu Leben
und Werk}. Berlin: De Gruyter, 1998.

Tent, James F. \emph{Freie Universität Berlin, 1948--1988: Eine deutsche
Hochschule im Zeitgeschehen}. Berlin: Colloquium, 1988.

Venema, Niklas. ``Zwischen Marx und Medienpraxis: Das Berliner Modell
der Journalistenausbildung.'' In \emph{``Regierungszeit des
Mittelbaus?'' Annäherungen an die Berliner Publizistikwissenschaft nach
der Studentenbewegung}, edited by Maria Löblich and Niklas Venema,
337--77. Cologne: Halem, 2020.

Wehrs, Nikolai. \emph{Protest der Professoren: Der ``Bund Freiheit der
Wissenschaft'' in den 1970er Jahren}. Göttingen: Wallstein, 2014.

Wiedemann, Thomas, Michael Meyen, and Maria Löblich. ``Communication
Science at the Center of Cold War's Communication Battles: The Case of
Walter Hagemann (1900-1964).'' In \emph{Communication @ The Center},
edited by Steve Jones, 107--20. New York: Hampton Press, 2012.

Wilke, Jürgen. ``Gewalt gegen die Presse: Episoden und Eskalationen in
der deutschen Geschichte.'' In \emph{Unter Druck gesetzt: Vier Kapitel
deutscher Pressegeschichte}, edited by Jürgen Wilke, 129--98. Cologne:
Böhlau, 2002.

Wissenschaftliche Einrichtung Publizistik. \emph{Studienplan für das
Fach Publizistik und Dokumentationswissenschaft}. Berlin: Freie
Universität Berlin, 1973.

Wosnitza, Andreas-Rudolf. ``Über Fritz Eberhard nachdenken, heißt, über
sich selbst nachdenken.'' In \emph{Fritz Eberhard: Rückblicke auf
Biographie und Werk}, edited by Bernd Sösemann, 28--29. Stuttgart: Franz
Steiner, 2001.

Zeitz, Freddy. ``Die Berufung von Harry Pross auf den Lehrstuhl für
Publizistik.`` In \emph{``Regierungszeit des Mittelbaus?'' Annäherungen
an die Berliner Publizistikwissenschaft nach der Studentenbewegung},
edited by Maria Löblich and Niklas Venema, 304--35. Cologne: Halem,
2020.



\end{hangparas}


\end{document}