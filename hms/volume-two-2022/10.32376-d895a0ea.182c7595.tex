% see the original template for more detail about bibliography, tables, etc: https://www.overleaf.com/latex/templates/handout-design-inspired-by-edward-tufte/dtsbhhkvghzz

\documentclass{tufte-handout}

%\geometry{showframe}% for debugging purposes -- displays the margins

\usepackage{amsmath}

\usepackage{CJKutf8}

\usepackage{hyperref}

\usepackage{fancyhdr}

\usepackage{hanging}

\hypersetup{colorlinks=true,allcolors=[RGB]{97,15,11}}

\fancyfoot[L]{\emph{History of Media Studies}, vol. 2, 2022}


% Set up the images/graphics package
\usepackage{graphicx}
\setkeys{Gin}{width=\linewidth,totalheight=\textheight,keepaspectratio}
\graphicspath{{graphics/}}

\title[Journalism via Systems Cybernetics]{Journalism via Systems Cybernetics: The Birth of the Chinese Communication Discipline and Post-Mao Press Reforms} % longtitle shouldn't be necessary

% The following package makes prettier tables.  We're all about the bling!
\usepackage{booktabs}

% The units package provides nice, non-stacked fractions and better spacing
% for units.
\usepackage{units}

% The fancyvrb package lets us customize the formatting of verbatim
% environments.  We use a slightly smaller font.
\usepackage{fancyvrb}
\fvset{fontsize=\normalsize}

% Small sections of multiple columns
\usepackage{multicol}

% Provides paragraphs of dummy text
\usepackage{lipsum}

% These commands are used to pretty-print LaTeX commands
\newcommand{\doccmd}[1]{\texttt{\textbackslash#1}}% command name -- adds backslash automatically
\newcommand{\docopt}[1]{\ensuremath{\langle}\textrm{\textit{#1}}\ensuremath{\rangle}}% optional command argument
\newcommand{\docarg}[1]{\textrm{\textit{#1}}}% (required) command argument
\newenvironment{docspec}{\begin{quote}\noindent}{\end{quote}}% command specification environment
\newcommand{\docenv}[1]{\textsf{#1}}% environment name
\newcommand{\docpkg}[1]{\texttt{#1}}% package name
\newcommand{\doccls}[1]{\texttt{#1}}% document class name
\newcommand{\docclsopt}[1]{\texttt{#1}}% document class option name


\begin{document}

\begin{titlepage}

\begin{fullwidth}
\noindent\LARGE\emph{Exclusions in the History of Media Studies
} \hspace{25mm}\includegraphics[height=1cm]{logo3.png}\\
\noindent\hrulefill\\
\vspace*{1em}
\noindent{\Huge{Journalism via Systems Cybernetics:\\\noindent The Birth of the Chinese Communication\\\noindent Discipline and Post-Mao Press Reforms\par}}

\vspace*{1.5em}

\noindent\LARGE{Angela Xiao Wu}\par}\marginnote{\emph{Angela Xiao Wu, ``Journalism via Systems Cybernetics: The Birth of the Chinese Communication Discipline and Post-Mao Press Reforms,'' \emph{History of Media Studies} 2 (2022), \href{https://doi.org/10.32376/d895a0ea.182c7595}{https://doi.org/ 10.32376/d895a0ea.182c7595}.} \vspace*{0.75em}}
\vspace*{0.5em}
\noindent{{\large\emph{New York University}, \href{mailto:angelaxwu@nyu.edu}{angelaxwu@nyu.edu}\par}} \marginnote{\href{https://creativecommons.org/licenses/by-nc/4.0/}{\includegraphics[height=0.5cm]{by-nc.png}}}

% \vspace*{0.75em} % second author

% \noindent{\LARGE{<<author 2 name>>}\par}
% \vspace*{0.5em}
% \noindent{{\large\emph{<<author 2 affiliation>>}, \href{mailto:<<author 2 email>>}{<<author 2 email>>}\par}}

% \vspace*{0.75em} % third author

% \noindent{\LARGE{<<author 3 name>>}\par}
% \vspace*{0.5em}
% \noindent{{\large\emph{<<author 3 affiliation>>}, \href{mailto:<<author 3 email>>}{<<author 3 email>>}\par}}

\end{fullwidth}

\vspace*{1em}

\hypertarget{abstract}{%
\section{Abstract}\label{abstract}}


In China's reform era (1979--), to revitalize the newspaper for economic
modernization, journalism scholars turned to American mass communication
study after Wilbur Schramm's visit in 1982. This familiar story of the
birth of the Chinese communication discipline missed a critical thread.
Faced with constant political contractions, systems cybernetics crept
in: The supposed founder of communication, Wilbur Schramm, spoke in
Claude Shannon's name; China's ``King of Rocketry,'' Qian Xuesen,
pledged to augment Norbert Wiener; and Engels's writing on dialectical
materialism was seized upon as part of official Marxism. 1980s China is
known for the revival of humanism. However, in the tightly controlled
news sector, reformist scholars conjoined systems cybernetics and
Engels's Marxist philosophy, dismantling the human subject in the
process. This articulation of the epistemology and ethic of newsmaking,
which has persisted to this day in the disciplinary assumptions of
Chinese journalism-communication, allowed the regime to depart from
Maoism and legitimate a state-managed marketization. Excavating an
unknown episode in the global history of cybernetics, this revisionist
study sheds new light on the post-Mao trajectories of journalism and
media governance. It also dislodges the ossified binaries between
science and political ideology, and between socialist centralism and
capitalist liberalism, which are prevalent in (post-)Cold War
narratives.





\enlargethispage{2\baselineskip}

\vspace*{2em}

\noindent{\emph{History of Media Studies}, vol. 2, 2022}


\end{titlepage}

\begin{CJK*}{UTF8}{gbsn} 

\newthought{In the warm spring} of 1982, as universities scrambled to reinvigorate
themselves from the ruins of the Cultural Revolution, the
People's
Republic of China (PRC) welcomed the American
scholar-educator-administrator Wilbur Schramm (1907--1987). This was a
time when the press, journalism scholarship, and government propaganda
units, which later functioned more discretely, were still intimately
entwined. Overlapping groups of academic researchers, news workers, and
cultural cadres comprised the media sectors' ``establishment
intellectuals.''\footnote{Carol Hamrin and Timothy Cheek, \emph{China's
  Establishment Intellectuals} (Armonk, NY: M. E. Sharpe, 1986).} To
reinvent journalism and revitalize the newspaper, these media thinkers
turned to Western communication study, more precisely, the American mass
communication tradition represented by Schramm. Today, this turn is the
familiar story of the disciplinary formation of Chinese communication
study or \emph{chuanbo xue}.\footnote{To prevent confusion, the
  discipline of communication focuses on humans not machines (e.g.,
  ``telecommunications''). The Chinese term for the then-newly imported
  concept of ``communication'' is \emph{chuanbo}, a translation that
  evokes certain meanings while displacing others. \emph{Chuan} means
  ``transmission'' and \emph{bo} ``to propagate far and wide.''
  \emph{Chuanbo}'s closest English equivalent is perhaps
  ``dissemination.'' While the English word ``communication'' connotes
  exchange between actors not necessarily different in power,
  \emph{chuanbo} centers on the ability to distribute to a wider
  audience, assuming a powerful disseminator. \emph{Chuanbo} thus
  conveys a narrow focus on the operations of broadcast media or ``mass
  communication,'' a distinct though dominant subarea of communication
  study. John Durham Peters, ``Institutional Sources of Intellectual
  Poverty in Communication Research,'' \emph{Communication Research} 13,
  no. 4 (1986). This terminological equivalence effectively steered the
  Chinese field away from other communication phenomena. After all,
  unlike ``anthropology'' or ``sociology,'' ``communication'' designates
  the discipline's intellectual focus. Also notably, \emph{chuanbo}
  suggests a proximity to \emph{xuanchuan}, the signature phrase for
  doing socialist cultural work that is typically translated as
  ``propaganda.'' However, Chinese \emph{xuanchuan} is really a broad,
  non-pejorative concept that refers to authoritative entities making
  something known to and resonant with the people. In the socialist
  context, ``propagation'' is a closer phrase: to promote what one
  believes to be true, with a centuries-long tradition wherein
  scholar-officials ``propagated'' the dynasty's moral line in the
  orthodox language of the day. Timothy Cheek, \emph{Propaganda and
  Culture in Mao's China: Deng Tuo and the Intelligentsia} (London:
  Clarendon Press, 1997).} While textbooks portray this formation as a
moment of intense intellectual enlightenment, critical voices at the
margin---mostly from scholars outside of China---consider it an
unfortunate adoption of the positivist scholarship of liberal capitalism
propelled by the resolution to negate socialist legacies.\textsuperscript{3}

It is not an either-or story.

This article illuminates a key component that is missing from existing
accounts. During sporadic political contractions, systems cybernetics
crept in: The supposed founder of American communication, Wilbur
Schramm, spoke in Claude Shannon's name; China's ``King of Rocketry,''
Qian Xuesen, pledged to augment Norbert Wiener; and good old Engels's
writing on dialectical materialism was seized upon as part of official
Marxism. In the (often overlooked) absence of a countervailing social
science tradition, I demonstrate, the post-Mao discipline of
journalism-communication has been profoundly shaped by systems
cybernetics. By ``systems cybernetics'' I mean a highly infectious
mixture of ideas from cybernetic and information theories that was,
crucially, accompanied by gusto for system-level visions.

This revisionist history of the Chinese communication discipline adds an
important but neglected case to the global history of cybernetics and
systems theory. For its interdisciplinary propositions, this set of
knowledge acquired extraordinary political flexibility and appeal in the
hands of historical actors. It fostered cross-pollination between
knowledge fields in the postwar US,\textsuperscript{4} transcended the Cold War's divide as a ``neutral''
toolbox,\textsuperscript{5}
and helped inoculate Soviet sciences from ideological
attacks.\textsuperscript{6} In
China's ``post-Mao military-to-civilian conversation,'' systems
cybernetics also informed the one-child policy.\textsuperscript{7} Complementing this extensive literature, I delineate the Chinese
import of cybernetics through a kind of artful ``rediscovery'' of
Friedrich Engels, exemplified in Qian Xuesen's writings. This history
brings to the fore the resonance and overlapping genealogies\marginnote{\textsuperscript{3} E.g.,
  Chunfeng Lin and John Nerone, ``The `Great Uncle of Dissemination':
  Wilbur Schramm and Communication Study in China,'' in \emph{The
  International History of Communication Study}, ed. Peter Simonson and
  David W. Park (New York: Routledge, 2015); Hu Zhengrong, Ji Deqiang,
  and Zhang Lei, ``Building the Nation-State: Journalism and
  Communication Studies in China,'' in \emph{The International History
  of Communication Study}, ed. Peter Simonson and David W. Park (New
  York: Routledge, 2015; Yuezhi Zhao, ``批判研究与实证研究的对比分析''
  {[}Comparing critical and empirical approaches{]},\emph{国际新闻界}
  {[}Chinese journal of journalism and communication{]}, no. 11 (2006);
  Liu Hailong 刘海龙,
  ``\,`传播学'引进中的`失踪者':从1978年--1989年批判学派的引介看中国早期的传播学观念''
  {[}The `missing persons' in the import of communication{]}.
  \emph{新闻与传播研究} {[}Journalism and communication{]}, no. 4
  (2007).} between\marginnote{\textsuperscript{4} Peter Galison, ``The
  Ontology of the Enemy: Norbert Wiener and the Cybernetic Vision,''
  \emph{Critical Inquiry} 21, no. 1 (1994); Hunter Heyck, \emph{Age of
  System: Understanding the Development of Modern Social Science}
  (Baltimore, MD: Johns Hopkins University Press, 2015); Ronald R.
  Kline, \emph{The Cybernetics Moment: Or Why We Call Our Age the
  Information Age} (Baltimore, MD: Johns Hopkins University Press,
  2015).}
systems\marginnote{\textsuperscript{5} Eden Medina, \emph{Cybernetic Revolutionaries:
  Technology and Politics in Allende's Chile} (Cambridge: MIT Press,
  2014); Egle Rindzeviciute, \emph{The Power of Systems: How Policy
  Sciences Opened Up the Cold War World} (Ithaca, NY: Cornell University
  Press, 2016); Benjamin Peters, \emph{How Not to Network a Nation: The
  Uneasy History of the Soviet Internet} (Cambridge: MIT Press, 2016).} cybernetics\marginnote{\textsuperscript{6} Slava Gerovitch, \emph{From Newspeak to Cyberspeak: A
  History of Soviet Cybernetics} (Cambridge: MIT Press, 2004).} and\marginnote{\textsuperscript{7}\setcounter{footnote}{7} Susan
  Greenhalgh, ``Missile Science, Population Science: The Origins of
  China's One-Child Policy,'' \emph{China Quarterly}, no. 182 (2005):
  253.} dialectical materialism,\footnote{Richard
  Levins, ``Dialectics and Systems Theory,'' \emph{Science \& Society}
  62, no. 3 (1998).} which have been written off in the history of
science and technology because the Cold War lens views Marxist
philosophy as ``ideology'' and ``unscientific.''\footnote{Alexei
  Yurchak, ``Communist Proteins: Lenin's Skin, Astrobiology, and the
  Origin of Life'' \emph{Kritika: Explorations in Russian and Eurasian
  History} 20, no. 4 (2019). Also see Angus M. Taylor, ``Dialectics of
  Nature or Dialectics of Practice?'' \emph{Dialectical Anthropology} 4,
  no. 4 (1979).} With it, I highlight the structural and discursive
conditions conducive for contemporary Chinese social governance to
incorporate systems cybernetics, a broader historical process in need of
collective research efforts.

\end{CJK*}


This is also a sociohistorical study of how the moral and political
commitments of journalism, a normative interpretive institution, found
their articulations in languages of science and technology. What
unfolded as China transitioned to post-socialism, however, bears little
resemblance to the historical discovery of objectivity in American
journalism.\textsuperscript{10}
In the Mao era, it was understood that, despite its claim to represent
reality objectively, the capitalist nature of the Western press
inevitably led to thoughtless reproduction of fragmented appearances,
which was at best ``superficial truth'' (\emph{biaoxiang zhenshi}). The
Chinese socialist press, in contrast, was aimed at assisting class
struggle by delivering ``substantial truth'' (\emph{benzhi zhenshi}),
which exposed the fundamental state and relationships of society
according to official Marxist insights (i.e., ``the inherent logic of
reality''). After Mao, Chinese journalism theory appropriated systems
cybernetic ideas, but this appropriation never led to pronouncing any
form of ``scientific objectivity'' in reporting. Rather, it held onto
the cybernetic notion of ontology and, contingently, control.\textsuperscript{11} The
rise of these normative conceptions of news media in the 1980s has been
lost to the extant historiography on Chinese media reforms, which
centers on media marketization and advocacy for liberal professionalism
in the 1990s.

Indeed, this history troubles not only the binary of science and
political ideology, but also the binary of socialist centralism and
capitalist liberalism that characterizes many post-socialist narratives.
Known as China's ``New Enlightenment,'' the 1980s is defined by a
``humanism fever''---beginning with Marxist humanism and taken over by
liberal humanism.\textsuperscript{12} Even science's wide appeal in the 80s has been examined for
breeding humanist values amenable to democratic politics. Many
scientists, such as Fang Lizhi, became liberal dissidents and
spearheaded the 1989 pro-democratic movement.\textsuperscript{13} In the
meantime, critics on the left see the roots of China's neoliberalism:
``the abstract concept of human subjectivity and the concept of human
freedom and liberation, which played significant roles in the critique
of Mao's socialist experiment, lack vigor in the face of the social
crises encountered in the process of capitalist marketization and
modernization.''\textsuperscript{14} What happened to
the news sector, which the Party placed under its most intense,
unwavering scrutiny, deviates from this\marginnote{\textsuperscript{10} Michael Schudson, \emph{Discovering The News: A
  Social History Of American Newspapers} (New York: Basic Books, 1981).} understanding.\marginnote{\textsuperscript{11} See
  Galison, ``Ontology of the Enemy''; Andrew Pickering, \emph{The
  Cybernetic Brain: Sketches of Another Future} (Chicago: University of
  Chicago Press, 2010); Fred Turner, \emph{From Counterculture to
  Cyberculture: Stewart Brand, the Whole Earth Network, and the Rise of
  Digital Utopianism} (Chicago: University of Chicago Press, 2008).} To\marginnote{\textsuperscript{12} Gloria Davies, ``Discursive Heat: Humanism in
  1980s China,'' in \emph{A New Literary History of Modern China}, ed.
  David Der-Wei Wang (Cambridge: Belknap Press/Harvard University Press,
  2017).} battle\marginnote{\textsuperscript{13} Lyman H. Miller,
  \emph{Science and Dissent in Post-Mao China: The Politics of
  Knowledge} (Seattle: University of Washington Press, 1996).} Mao-era\marginnote{\textsuperscript{14}\setcounter{footnote}{14} Wang Hui, \emph{China's New Order: Society,
  Politics, and Economy in Transition} (Cambridge: Harvard University
  Press, 2003), 167; Lisa Rofel, \emph{Desiring China: Experiments in
  Neoliberalism, Sexuality, and Public Culture} (Durham: Duke University
  Press, 2007); Shu-Mei Shih, ``Is the Post in Postsocialism the Post in
  Posthumanism?'' \emph{Social Text} 30, no. 1 (2012).}
constrictions, reformist scholars wound up conjoining systems
cybernetics and Engels's Marxist philosophy, dismantling rather than
reviving the human subject in the process. This articulation of
newsmaking's epistemological and ethical grounds allowed the Chinese
Communist Party (CCP) to depart from Maoism and legitimate a
state-managed marketization, with a far-reaching legacy.

\hypertarget{knowledge-upgrade-for-party-journalism-communication-without-baggage}{%
\section{Knowledge Upgrade for Party Journalism: Communication
\\\noindent without
Baggage}\label{knowledge-upgrade-for-party-journalism-communication-without-baggage}}

\begin{CJK*}{UTF8}{gbsn} 

From late April to early May of 1982, Wilbur Schramm visited the PRC in
the company of his wife and Timothy Yu (1920--2012), translator and
enabler of the ``icebreaking'' trip. Yu was Schramm's former student at
Stanford and then-retired chair of Department of Journalism and
Communication at the Chinese University of Hong Kong.\footnote{Much has
  been written on this historic visit and the role of colonial Hong Kong
  as the nodal point mediating the import of communication scholarship.
  Timothy Yu et al., ``中国传播学研究破冰之旅的回顾:余也鲁教授访问记''
  {[}Chinese communication study's icebreaking trip in retrospect{]},
  \emph{新闻与传播研究} {[}Journalism and communication{]}, no. 4
  (2012); Lin and Nerone, ``\,`Great Uncle of Dissemination': Wilbur
  Schramm.''} In Beijing, thanks to a request put through by the Chinese
Academy of Social Science (CASS), Schramm met Bo Yibo, Vice Premier in
charge of science, technology, and finance.\footnote{Xiaoling 晓凌,
  ``他们精心治学`` {[}They are devoted to scholarships{]},
  \emph{新闻学会通讯} {[}Society of journalism letters{]}, no. 14
  (1982).} Having led multiple UNESCO endeavors in Asia and the Middle
East, Schramm's emphasis on communication for ``development''
understandably appealed to the CCP leaders.\footnote{A rarely noted
  detail was that, as a conservative veteran, Bo's unusual interest in
  the subject had been cultivated by his son, Bo Xilai, who happened to
  be among the first cohort of graduates studying at the CASS Journalism
  Institute, launched in 1979. See Wang Yihong 王怡红and Hu Yiqing
  胡翼青, eds.,\emph{中国传播学30年} {[}Chinese communication: thirty
  years{]} (Beijing: Chinese Encyclopedia Press 中国大百科全书出版社,
  2010), 613, 632. Three decades later, this princeling son, mayor of
  Chongqing Municipality and a CCP Politburo member, was sensationally
  purged by Xi's administration.} In the uncertain early-reform period,
the Vice Premier's explicit support made Schramm's sermon for
communication all the more phenomenal. Schramm delivered a speech to
hundreds of scholars and propaganda cadres that packed the top-floor
auditorium of the \emph{People's Daily}'s main building, then located on
Wangfujing Street, a central area of Beijing studded with aristocratic
residences from the Qing Dynasty. He also spoke at Renmin University and
CASS.

Schramm certainly knew how to strike a chord with his audience, which
mostly consisted of scholar-educators who were experienced news workers
in the Mao era and the first graduate student cohorts back to school,
eager to learn and innovate. Before diving into his topic, he typically
began by proudly retracing his past life as a reporter and an editor:
``The blood of journalism flows in my veins. In other words, I am a
newsman upside-down and inside-out.''\textsuperscript{18} Schramm
then characterized journalism studies, supposedly concerned with one
variety of human communication, as a subarea and ``precursor'' of
communication study.\textsuperscript{19} He demonstrated this through an
opportunistic telling of its American history: ``American journalism
departments turned to communication, integrated it into their journalism
education, and established communication research centers.''\textsuperscript{20}
  
\newpage

This\marginnote{\textsuperscript{18} Wilbur Schramm,
  ``美国"大众传播学"的四个奠基人'' {[}Four forefathers of American mass
  communication{]}, \emph{国际新闻界} {[}Chinese journal of journalism
  and communication{]}, no. 2 (1982): 2. Unless otherwise indicated,
  this and subsequent translations were made by the author.} was\marginnote{\textsuperscript{19} Zhou Zhi 周致,
  ``西方传播学的产生及其与新闻学的关系'' {[}Western communication's
  origin and relations with journalism study{]}, in \emph{传播学简介}
  {[}Introduction to communication study{]}, ed. 社科院新闻所世界室 CASS
  Journalism Institute (Beijing: 人民日报出报社 People's Daily Press,
  1983); Chen Chongshan 陈崇山, ``施拉姆的理论对我的指引'' {[}How
  Schramm's theory guided me{]}, \emph{新闻与传播研究} {[}Journalism and
  communication{]}, no. 4 (2012).} a\marginnote{\textsuperscript{20}\setcounter{footnote}{20} Xu
  Yaokui 徐耀魁,
  ``施拉姆对中国传播学研究的影响:纪念施拉姆来新闻研究所座谈30周年''
  {[}Schramm's impacts on Chinese communication research{]},
  \emph{新闻与传播研究} {[}Journalism and communication{]}, no. 4
  (2012): 10.} far cry from his own past and future English writings.
Schramm had always been explicit that it was the lavish resources from
governments, commerce, and the military that drove the social sciences
to splinter and form a separate foundation for communication as a
discipline.\footnote{Schramm, ``The Unique Perspective of Communication:
  A Retrospective View,'' \emph{Journal of Communication} 33, no. 3
  (1983): 7. Also see Peters, ``Institutional Sources of Intellectual
  Poverty.''} As ``skills departments,'' journalism coalesced with
communication to increase their academic status.\footnote{Schramm,
  ``Human Communication as a Field of Behavioral Science'' in
  \emph{Human Communication as a Field of Study}, ed. Sarah Sanderson
  King (Albany, NY: SUNY Press, 1989). Also see James W. Tankard Jr.,
  ``The Theorists,'' in \emph{Makers of the Media Mind: Journalism
  Educators and Their Ideas}, ed. William David Sloan (New York:
  Routledge, 1990); James W. Carey, ``Some Personal Notes on US
  Journalism Education,'' \emph{Journalism} 1, no. 1 (2000).} Many
practice-oriented faculty, however, continued to view their
research-focused colleagues as irrelevant for producing good
journalism---a perennial tension once popularly referred to as the
``green eyeshades vs. chi-squares'' division in journalism
departments.\footnote{The green eyeshade is the headgear that used to be
  worn by copy editors to cut the glare from the overhead lights. The
  chi-square comes from a statistical test. Jake Highton, ``\,`Green
  Eyeshade' Profs Still Live Uncomfortably with `Chi-Squares';
  Convention Reaffirms Things Haven't Changed Much,'' \emph{Journalism
  Educator} 44, no. 2 (1989).} Chi-squares failed to see much merit in
letting green eyeshades play their games either: ``The fact that a
single individual can teach courses in, say, magazine editing and
research techniques in social psychology {[}a dominant approach of mass
communication{]} is a tribute to human adaptability, not to a
well-conceived academic discipline.''\footnote{Jeremy Tunstall, ``The
  Trouble with U.S. Communication Research,'' \emph{Journal of
  Communication} 33, no. 3 (1983): 93.} This jab was literally pages
away from Schramm's own historical narrative in the \emph{Journal of
Communication}, both drafted within months of the latter's China visit.
In short, at American universities it was institutional politics that
coupled journalism and mass communication, and their internal division
persists. For the Chinese ears, however, Schramm changed this
organizational story to one about inexorable intellectual development.
The Chinese inquired keenly about whether turning to communication might
hinder journalism studies. ``There is neither conflict nor tension
between the two,'' assured Schramm, ``but only mutual enhancement. . . .
Communication is journalism's flesh and blood.''\footnote{Schramm,
  ``美国"大众传播学"的四个奠基人'' {[}Four forefathers{]}, 2.}
  


In fact, mass communication was indeed a godsend. In the early 1980s,
the main hurdle faced by reform-minded journalism scholars was the
incommensurability between ``Marxist proletarian (or socialist) theory
of the press'' and its ``bourgeois'' counterpart. The two were
considered incommensurable because the Maoist epistemology maintained
that the norms of the press, or the press theory of a society, were
determined by the given political economy. Therefore, to appropriate any
journalistic norms from the ``Western bourgeois society'' invited
ideological attacks. But thanks to its decontextualized abstraction, the
Schrammian strand of mass communication provided a fresh start. Its
circumscribed diagrams of the dissemination process invariably trace
symbolic content from point A to point B. Whereas critical scholars
across the Pacific increasingly recognized and critiqued this
obliviousness to fundamental questions about the relationship between
media and broader structures of power,\footnote{A string of such
  reflexive essays appeared in ``Ferment in the Field,'' the 1983
  landmark issue of the flagship \emph{Journal of Communication}.}
early-reform China found it particularly palatable.

What was remarkable about this affinity, if we can call it that, was
that it resulted from American communication's own disciplinary
formation during the Cold War. It was military and corporate patronage
that spawned its growth, directing the discipline's latent assumptions
about when and where the effects of mass media should be investigated.
Meanwhile, communication scholarship persistently veils the capitalist
and imperialist constitution of media operations, a kind of
``ideological work'' performed in service of power.\footnote{Peters,
  ``Institutional Sources of Intellectual Poverty''; Arvind Rajagopal,
  ``Communicationism: Cold War Humanism,'' \emph{Critical Inquiry} 46,
  no. 2 (2020).} In a sense, the relationship between American
communication and China was epitomized by Schramm's own life. During
World War II and the early years of the Cold War, he was deeply involved
in US security establishments. He co-wrote \emph{Four Theories of the
Press} (1956), which offered a typology of ideas that supposedly
encapsulate both government and the press of a given society. It
classified the Chinese press as a follower of the ``Soviet communist
theory,'' described as an expanded version of the ``Authoritarian
theory'' originating in feudal Europe.\footnote{Fred Siebert, Theodore
  Peterson, and Wilbur Schramm, \emph{Four Theories of the Press: The
  Authoritarian, Libertarian, Social Responsibility, and Soviet
  Communist Concepts of What the Press Should Be and Do} (Champaign, IL:
  University of Illinois Press, 1956).} In the late 1960s and 1970s, as
he moved to the ``development paradigm'' of communication, Schramm
became intrigued by China's economic growth and relied on his Taiwanese
and Hong Kong students for material.\footnote{Wilbur Schramm, Godwin C.
  Chu, and Frederick T. C. Yu, ``China's Experience with Development
  Communication: How Transferable Is That?'' in \emph{Communication and
  Development in China}, ed. Goodwin Chu et al. (Honolulu: East-West
  Center, 1975). Also see Lin and Nerone, ``\,`Great Uncle of
  Dissemination': Wilbur Schramm.''} During his time at CASS in 1982,
the Chinese brought up his \emph{Four Theories} book. Did his
classification change? As Timothy Yu recalls, ``To this Schramm
responded humorously, `it's been decades and I almost forgot about this
book. If I write it today, I wouldn't say China is authoritarian. I'd
think of another phrase. I'll tell you once I have it.'\,''\footnote{Yu
  et al., ``中国传播学研究破冰之旅的回顾:余也鲁教授访问记`` {[}Chinese
  communication study's icebreaking trip in retrospect{]}, 7.}



However, by opening itself up to communication in the 1980s, Chinese
journalism studies did not embark on behaviorist empirical research on
journalism phenomena because no such research paradigm was available.
Early American communication PhDs typically did coursework in sociology,
political science, and psychology departments, where social scientific
methods were inculcated. These empirical scholars then assumed
leadership roles in flourishing communication departments.\footnote{Everett
  M. Rogers, ``Wilbur Schramm and the Establishment of Communication
  Study,'' chap. 12 in \emph{History of Communication Study} (New York:
  Free Press, 1997).} But Schramm himself had little experience in
frontier research. His yearning, instead, was for the intellectual
integration of communication as a field.\footnote{Even in the eyes of
  his most ardent colleagues and protégés, Schramm's contribution to
  communication study was more organizational than intellectual. Rogers,
  \emph{History of Communication Study}; Emile G. McAnany, ``Wilbur
  Schramm, 1907--1987: Roots of the Past, Seeds of the Present,''
  \emph{Journal of Communication} 38, no. 4 (1988).} His preaching of
communication study by and large glossed over the nitty-gritty of
research design and execution. This omission of methodological issues
went unnoticed in China due to the impoverished state of its social
sciences.\footnote{Since the 1950s, the Mao regime had denounced Western
  social sciences such as sociology, as well as probabilistic
  statistics, as distorting social reality for capitalist domination.
  Xin Liu, \emph{The Mirage of China: Anti-Humanism, Narcissism, and
  Corporeality of the Contemporary World} (New York: Berghahn Books,
  2012); Arunabh Ghosh, \emph{Making It Count: Statistics and Statecraft
  in the Early People's Republic of China} (Princeton, NJ: Princeton
  University Press, 2020).} The ``chi-square'' had been missing in the
picture for decades; the Chinese socialist ``green eyeshades'' could not
educate themselves about it even if they wanted to.\textsuperscript{34}

As a result, it was Maoist press theoreticians and China's most
ideologically entrenched literary workers who took over
``communication'' as a resource for their own project: to envision and
justify media's imminent reforms. The chief challenge in this
appropriation has to do with values. Journalism is a socially grounded
institution reputed\marginnote{\textsuperscript{34}\setcounter{footnote}{34} Little
  wonder that Chinese communication considers its breakthrough in the
  ``quantitative research tradition'' a study that conjured up a
  mathematical formula for determining news values without even trying
  it out on actual news. Chen Lidan 陈力丹 and Wang Yigao 王亦高,
  ``我国新闻传播学量化研究的艰难起点'' {[}The fraught origins of
  quantitative research in Chinese journalism-communication{]},
  \emph{当代传播} {[}Contemporary communication{]}, no. 2 (2009); In
  parallel, commentary and philosophical critique had long been seen as
  ``qualitative research.'' Li Biao 李彪, ``新闻传播学研究方法的构造''
  {[}Research methods of journalism-communication{]}, \emph{国际新闻界}
  {[}Chinese journal of journalism and communication{]}, no. 1 (2008).} for its normative aspirations, whereas mass
communication as a behaviorist research program ``presupposes the
victory \ldots{} of means without ends.''\footnote{Allen Tate, quoted in
  Carey, ``US Journalism Education,'' 21. Communication majors usually
  join marketing, public relations, and audience analytics firms.
  Whereas Carey calls out the ``economic'' end of American
  communication, early-reform Chinese journalism education must look
  elsewhere.} This means Chinese reformists had to search amidst
whatever they took as communication for normative ``ends'' to supplant
existing journalism ideals. This search was eventually completed on
engineering terms, a turn seeded in Schramm's visit.

\end{CJK*}


\hypertarget{information-and-feedback-phasing-out-the-people}{%
\section{Information and Feedback: Phasing out the
People}\label{information-and-feedback-phasing-out-the-people}}

\begin{CJK*}{UTF8}{gbsn} 

To assert his vision of an all-encompassing communication study, Schramm
presented an eclectic mix of concepts and ideas, among which were
information theory and cybernetics. Absent sufficient context, the
Chinese were left with an impression that communication study had its
foundations in these areas. One and half years after his departure, in
the winter of 1983, triggered by a broader intellectual advocacy for
Marxist humanism, the CCP's conservative fraction struck out to curb the
rapid spread of Western-inspired liberal ideas. Communication study was
attacked for ``negating class struggle'' and being particularly
dangerous as a potential replacement of Chinese socialist press theory.
This ``Anti-Spiritual Pollution Campaign'' was short-lived but
admonitory for sensitive domains such as journalism. Thought experiments
in journalist periodicals started segueing along this impressionist link
to systems cybernetics, blowing it out of proportion. Specifically, the
narrow technical definitions of information and feedback were mobilized
to define ``news value'' as leveraging journalism from propaganda, a
feat indispensable for media reforms. This constitutes a significant
divergence from its American counterpart.

At CASS, Schramm declaimed that ``two physicists,'' Claude Shannon and
Warren Weaver, established information theory in order to ``study the
feedback phenomenon in the communication of human society'': ``Through
their hands, information theory has migrated from natural sciences to
social sciences. Their goal is to scientifically measure the circulation
of information among people.''\footnote{Schramm, ``传学与新闻及其它''
  {[}Communication study and journalism, and other issues{]},
  \emph{新闻学会通讯} {[}Society of journalism letters{]}, no. 14
  (1982): 20.} This, however, could only be a knowing
mischaracterization. Schramm edited the composite book \emph{The
Mathematical Theory of Communication} (1949) at the University of
Illinois Press, which contains a reprint of Shannon's 1948 Bell Labs
paper, preceded by Weaver's attempt to narrate the paper's intimidating
mathematics in lay terms.\footnote{Shannon and Weaver, \emph{The
  Mathematical Theory of Communication} (Champaign, IL: University of
  Illinois Press, 1949).} Notably, whereas Shannon insisted on limiting
the application of information theory to technical communication because
the theory excludes the ``semiotic aspects,'' Weaver optimistically
called to expand its reach to human communication.\footnote{Everett M.
  Rogers and Thomas W. Valente, ``A History of Information Theory in
  Communication Research,'' in \emph{Between Communication and
  Information}, ed. Brent D. Ruben and Jorge R. Schement (New York:
  Routledge, 2017); Kline, \emph{The Cybernetics Moment}, 126.} But
actual attempts to conceptualize human communication through information
theory began with Schramm,\footnote{Schramm, ``Information Theory and
  Mass Communication,'' \emph{Journalism Quarterly} 32, no. 2 (1955).}
and was extended in a few studies mainly produced by Schramm's own
students.\footnote{Rogers, \emph{History of Communication Study}, 434,
  440.}
  
The Shannon model of technical communication illustrates the
transmission of a message from the source to the destination, which
entails changing the message into the signal and converting it back upon
reception. Admittedly, this model resembles the vision of dissemination
in mass communication that features elements such as source, channel,
message, and receiver. But Shannon's mathematical theory focuses on the
channel capacity of technical systems such as the telephone system. It
is also for this purpose that Shannon defines information as something
that reduces uncertainty for mechanical functions (measured by its
ability to eliminate alternate options). American mass communication
scholarship, by contrast, is predominated by its focus on the effects of
communication, typically indicated by the audience's opinion or
behavioral change. This is not surprising, given that much of the field
grew not to (re)structure media systems (which characterizes Chinese
media reforms, as I will explain in greater detail below), but to aid
administrative tasks faced by media organizations, government agencies,
and corporations.

Feedback is a key term in cybernetic theory developed by Norbert Wiener
during wartime to improve anti-aircraft fire. To control a system
through a feedback mechanism means the system's future conduct is
constantly adjusted based on information of its past performance, so
that the system continues to approximate its goal. ``Feedback'' was made
popular as a construct in mass communication through the widely used
textbook \emph{The Process of Communication} by David Berlo, one of the
earliest communication doctorates under Schramm's
supervision.\footnote{Berlo, \emph{The Process of Communication: An
  Introduction to Theory and Practice} (New York: Holt, Rinehart and
  Winston, 1960).} Berlo modified Shannon's original model by inserting
``feedback'' (while also omitting ``noise''). Schramm further replicated
this hybrid model in his own teaching and monographs as the ``Shannon
model of communication'' (Fig. 1). This tinkering was, again, led by
communication's preoccupation with effects; having ``feedback'' in a
dissemination model represented progress from the one-way notion of mass
communication to account for media producers' consideration of audience
reactions when producing new content. But in reality, such consideration
is neither inevitable nor predetermined.

\begin{figure}
    \centering
\includegraphics{wu-figure-one.jpeg}
    \caption{The ``Shannon model'' with ``feedback'' but no ``noise'' from
\emph{传播学概论} {[}Introduction to Communication{]} (1984), 242, the
Chinese translation of Schramm's coauthored volume \emph{Men, Women,
Messages, and Media} (1982).}
    \label{fig:one}
\end{figure}


Simply put, besides the ubiquity of colloquial usage of these terms in
communication scholarship (and everyday life), the field of
communication developed largely independently from information
science,\footnote{Only toward the end of the 1980s did some
  boundary-crossing start to appear. Brent Ruben, \emph{Between
  Communication and Information} (Milton, United Kingdom: Routledge,
  1993). No evidence shows that Chinese scholars ever picked up this
  limited convergence in Anglophone academia.} and even more so from
cybernetics.\footnote{Rogers, \emph{History of Communication Study},
  chap. 10.} But the original engineering conceptions about information
and feedback---such as optimizing channel capacity and utilizing signal
transmission intrinsic to the system---quickly captured Chinese
journalism scholarship.

After several years of sparring, the debate over (socialist)
``substantial truth'' versus (capitalist) ``superficial truth'' proved
intellectually unproductive and politically precarious.\footnote{Exchanges
  in a high-profile symposium on journalistic truth appeared in
  \emph{新闻学刊} {[}Journalism{]}, no. 2 (1985), after which the
  attempts to reconceptualize journalistic truths were largely aborted.}
Journalism scholars concurred that issues about ``truth'' should be left
to propaganda/propagation work, which was tasked with spreading the
right thought. But what set journalism and propaganda apart? Or were
they indeed different? Scholars used the Cultural Revolution to
illustrate what happens when the two are treated as the same. Newspapers
were filled with long, plodding polemics full of stock phrases and
lengthy coverage of the party's guiding principles, government
achievements, and the heroic deeds of exemplary individuals. Being
``timely'' meant letting political consideration determine the ``right''
moment of publicization. Even the above content easily took months to
get in print. As the reform-era derision went, the press ended up
supplying nothing but ``political ravings,'' losing touch with both the
goings-on of the world and with the masses. This, theorized the
reformers, was journalism abiding by ``political value'' alone. While
both propagation and journalism should operate according to political
value, what defined journalism was ``news value'' (\emph{xinwen jiazhi}
or \emph{xinwenxing}).



But describing worthy news as immediate, crisp, and fresh sounded too
much like a submission to capitalist journalistic norms. A less
vulnerable reform proposition, news value was spelled out in terms of
``information''---both in Wiener's and in Shannon's senses. On one hand,
according to Wiener, information as feedback of the cybernetic system is
a signal automatically issued upon the system's recent performance; it
is hence treated as epiphenomenal of the motion of things. To consider
news as information so defined thus asserts that news can and should be
\emph{non-ideological}, and that words can be transparent in delivering
world happenings. On the other hand, following Shannon's definition, to
best negate the amount of uncertainty in a decision-making situation, it
is important to maximize the amount of information carried through
limited channel capacity. Chinese scholars came up with various
mathematical formulas to determine ``news information
content.''\footnote{Arguably the first such attempt, later designated as
  pioneering Jcomm's ``quantitative'' tradition, was Yu Guoming 喻国明,
  ``新闻作品信息含量问题初探'' {[}Exploring news information content{]},
  \emph{新闻学论集} {[}Annals of journalism studies{]} 8 (1984).} These
collective efforts, though crammed with jargon and technicalities, were
meant to drive slogans and bureaucratic trivialities out of newspaper
pages. The new consensus dictated: ``The most essential function of news
is to bring the latest occurrences (\emph{shishi}) to people in order to
reduce their uncertainty about objective reality.''\footnote{Zheng
  Xingdong 郑兴东, ``新闻价值相关论'' {[}Thesis on news value{]},
  \emph{新闻学论集} {[}Annals of journalism studies{]} 11 (1987).}
 

Articulating news value through the notion of information, which had
never appeared in the vocabulary of China's socialist cultural work,
marked a watershed.\footnote{Searches in the most comprehensive
  scholarly database, CNKI, support my observation, although it should
  be noted that my archives extend beyond digitized material. For
  example, \emph{Annals of Journalism Studies} and
  \emph{Journalism}---both key venues in the 1980s scholarly discourse
  and both shut down after 1989 Tiananmen crackdown---remain in print
  forms only.} While the aspiration for substantial truth hails the
proletarian class---or more broadly, the people---as a collective
subject, news value modelled after information is stripped of embodied
experience and concerns about positionality. In 1987, the State Science
and Technology Commission listed the media sector as ``information
industries.'' In 1988, a nationwide polling of news workers on press
reforms showed that more than two-thirds believed the primary function
of journalism to be ``disseminating information,'' and nearly as many
rejected the idea that all news should serve propagation
purposes.\footnote{Chen Chongshan 陈崇山 and Mi Xiuling 弭秀玲,
  eds.,\emph{中国传播效果透视} {[}Perspectives on media effects in
  China{]} (Shenyang: Shenyang Publishing 沈阳出版社, 1989): 184.} This
normative shift granted journalism some autonomy from the overarching
propaganda imperative as a distinct field of practice.\footnote{See also
  Timothy Cheek, ``Redefining Propaganda: Debates on the Role of
  Journalism in Post-Mao Mainland China,'' \emph{Issues \& Studies} 25,
  no. 2 (1989): 56.}

In parallel, Chinese scholars appropriated ``feedback''---which Schramm
claimed to be ``the most important social science concept'' that
engineering science had contributed---to legitimate the functioning of
the press beyond perimeters of the Party propaganda apparatus. At the
time, the CASS scholar Chen Chongshan was preparing for Beijing Audience
Survey, a pathbreaking exercise of representative sampling initially
framed as following the Party's mass line: ``from the masses, to the
masses.'' She explained this Maoist principle to Schramm and asked: ``Is
this what you call `feedback'?'' Schramm confirmed.\footnote{Chen,
  ``施拉姆的理论对我的指引'' {[}How Schramm's theory guided me{]},
  \emph{新闻与传播研究} {[}Journalism and communication{]}, no. 4
  (2012).} Thirty years later, in her reminiscence of the birth of the
survey project, Chen Chongshan described her thinking of Mao-era
journalism education and theory as

\begin{quote}
all about the objectives and techniques of news dissemination
(\emph{chuanbo}), ignoring the role of the audience, ignoring their
demand and opinion. This had led to the misalignment between the
disseminator and the recipient. I {[}thus{]} suggested journalism theory
research be turned ``upside down.'' First, we study the audience. Then,
based on the law of how the audience receives news information, we
determine the plan and techniques for dissemination, so that news
dissemination and reception align with one another, achieving the
optimal effects.\footnote{Wang and Hu, \emph{中国传播学30年} {[}Chinese
  communication: thirty years{]}, 582--83.}
\end{quote}

\noindent On the surface, this might seem a matter of recognizing the agency and
preferences of the audience (i.e., ``the feedback'') in a dissemination
model.\footnote{Also see Zhang Yong, ``From Masses to Audience: Changing
  Media Ideologies and Practices in Reform China,'' \emph{Journalism
  Studies} 1, no. 4 (2000).} But it is an anachronism to suppose so.
Moving from the Party's mass line to such a model took radical epistemic
changes. (Chen's wording represented the language of Chinese
journalism-communication that engineering grammars have since
infested.)\footnote{For a striking contrast, see An's contemporaneous
  account in ``研究我们的读者'' {[}Studying our readers{]}.} Contrary to
how the reform discourse later paints it, the Mao-era press was not
conceived in a top-down dissemination model. Indeed, it possessed the
``Party character'' (\emph{dangxing}) that guides and educates the
masses along party ideology. But simultaneously, it also assumed the
``people's character'' (\emph{renminxing})---that is, it was intended as
a site for the (proletarian) people to express their worldview and share
their experiences engaging in social change.\footnote{Admittedly, ``the
  people'' was always an abstract construct. Looking closely, one may
  find complex sociopolitical and technical processes mediating the
  ``people'' and what was in the newspaper. But it is crucial to
  recognize the investment of Maoist politics in keeping the people
  character alive, and that in real contestations this conception could
  be mobilized as a source of enormous power.} If represented in a
model, everything---the Party, the masses, the message---converges at
the same spot: Within the press; there is no place for directional
arrows. How is this possible? Because the epistemology of Maoism
recognizes ``no natural or pre-existing collective unity, ready-made and
handy.''\footnote{Liu, \emph{The Mirage of China}, 145.} Instead, it
requires continuous struggles, both institutional and discursive, to
bridge the singular and the collective of the people. This is the
essence of the Maoist political project, and integral to it is the
socialist press's constant striving for a simultaneous embodiment of the
Party and the people characters.

\end{CJK*}

Therefore, the composite dissemination model, and the efflorescence of
theoretical discussion that it ushered in, reflect Chinese journalism
scholarship's enduring discursive efforts to externalize the people from
the press as a distinct object for scrutiny. The Maoist charm, the
holistic imagery of ``the press as the people,'' quickly dissipated. In
its place was ``the press for the people,'' wherein the people ceased to
manifest in the press but came into their own in front of the
researcher, who then advised the press with their knowledge. As per
official language, this might still qualify as the press's ``people's
character,'' but we should note the underlying transformation in how the
people are being served. In short, what transpired in the early 1980s
was a (literally) ``scientific'' dissection that pulled much of the
Maoist entanglement out of the press, pinning the audience down next to
political directives; now press reforms would observe two distinct
parameters.

Doing so also effectively separated the people from the Party. The
priority of journalism shifted from embodying and uniting the collective
will through the party organ, as in Maoist representational politics, to
producing information that acts on and preempts an existing collective.
This collective, now external to the Party, awaited being ``studied.''
Representing that collective's beliefs and desires, however, would face
onerous political risks and implementation challenges. During the
``Anti-Spiritual Pollution Campaign,'' Chen Chongshan's surveys were
attacked after English media reported them as capitalist ``public
opinion polls.'' Further, in determining what aspects of the audience to
probe, researchers were participating in the very construction of that
collective subject. The dilemma was this: Audience surveys were premised
on the (newly) assumed misalignment between the press (organ of the
Party) and its readers (the people), and no perfect execution existed
when the mission was to flesh out such a misalignment.

Toward the end of the 1980s, the eventual rise of what I call ``systems
journalism'' bypassed this dilemma by phasing out the question of
audience subjectivity altogether. ``Feedback'' became associated with
abstracted spheres, specifically the economy, and the significance of
journalism was defined not in relation to human faces, but to the
interdependent systems that supposedly comprised society.

\hypertarget{systems-cybernetics-in-reform-china-qian-xuesen-and-friedrich-engels}{%
\section{Systems Cybernetics in Reform China: Qian Xuesen and\\\noindent Friedrich
Engels}\label{systems-cybernetics-in-reform-china-qian-xuesen-and-friedrich-engels}}

\begin{CJK*}{UTF8}{gbsn} 

If Wilbur Schramm served as a key ingredient and a catalyst for the
fusion of disparate ideas into ``systems journalism'' in China, a second
key ingredient can be traced to certain engineering expertise uniquely
preserved through the Mao era. In the 1960s and 1970s, the CCP promoted
``socialist science,'' which instituted participation from the
masses.\footnote{Sigrid Schmalzer, \emph{Red Revolution, Green
  Revolution} (Chicago: University of Chicago Press, 2016).} The only
group of highly specialized scientists shielded from these political
imperatives (as well as discipline and punishment) were those working on
military projects. Likewise, whereas cybernetics became a pervasive
ideological force in shaping Soviet sciences,\footnote{Gerovitch,
  \emph{From Newspeak to Cyberspeak.}} in China, Marxist theoreticians
and ideologues, albeit somewhat informed of the Soviet trends, never
managed to interfere with the work led by mostly US-trained
scientists.\footnote{Peng Yongdong 彭永东, \emph{控制论的发生与传播研究}
  {[}The origin and dissemination of cybernetics{]} (Taiyuan: Shanxi
  Education Press 山西教育出版社, 2012), 168--196.} After 1978, the
latter's expertise and more crucially their ethos of systems analysis,
mounted the front stage and animated broader intellectual debates,
popular science endeavors, and science fiction throughout the
1980s.\footnote{See Xiao Liu, \emph{Information Fantasies: Precarious
  Mediation in Postsocialist China} (Minneapolis: University of
  Minnesota Press, 2019).} Absent the social sciences, defense
scientists also cast strong influence over social policies for
modernization.\footnote{Greenhalgh, ``Missile Science, Population
  Science.''}

The tide was led by Qian Xuesen (1911-2009), who was likely oblivious to
Wilbur Schramm's visit and the fermentation in journalism scholarship.
Known as H. S. Tsien in his American life, Qian was an aerodynamicist
trained at MIT and an endowed professor at the California Institute of
Technology conducting missile research for the US Army. Qian was
persecuted under McCarthyism and managed to return to China in 1955.
After the 1960 Sino-Soviet fallout and departure of Soviet advisors,
Chinese military projects were entrusted to Qian and other returning
scientists. The successful construction, in a just few years, of the
carriers of atomic bombs, hydrogen bombs, and satellites earned him the
title of ``King of Rocketry.'' Qian also played a major role in bringing
cybernetics to China. In the early 1950s, after becoming a communist
suspect and shut out of military research in the US, he turned to the
design of mechanical and electronic systems using cybernetic theory,
culminating in his groundbreaking monograph \emph{Engineering
Cybernetics}.\footnote{Tsien, \emph{Engineering Cybernetics} (New York:
  McGraw-Hill, 1954).} Its Chinese edition appeared in 1958, serving as
a foundation for numerous research projects. As China reoriented itself
for economic development and promoted ``modern'' science as a ``force of
production,'' Qian's proven political fidelity and scientific prowess
made him the post-Mao ``model intellectual.'' The state honored him as
something of a national treasure.\footnote{Ning Wang, ``The Making of an
  Intellectual Hero: Chinese Narratives of Qian Xuesen,'' \emph{China
  Quarterly}, no. 206 (2011).} Systems cybernetic ideas began to
overspill.



In September 1978, Qian's landmark article ``The Techniques of
Management and Organization: Systems Engineering,'' ran pages in
\emph{Wenhui}, a major newspaper targeting educated readers.\footnote{Qian
  Xuesen 钱学森, Xu Guozhi 许国志, and Shouyun Wang 王寿云,
  ``组织管理的技术---系统工程`` {[}Techniques of organizational
  management: systems engineering{]}, \emph{文汇报} {[}Wenhui{]},
  September 27, 1978.} Building on their experience in directing massive
military projects, Qian and his coauthors promoted ``systems
engineering''---an array of techniques including systems analysis,
management science, and operational research---as a key approach to
develop the national economy.\footnote{Qian later confided that it was
  the second author, Xu Guozhi, who penned the article and invoked
  ``systems engineering'' as a category. Qian, ``我对系统学的认识历程''
  {[}How I came to systematics{]}, in \emph{系统科学进展}
  {[}Developments in systems science{]}, ed. Guo Lei郭雷(Beijing:
  Science Publishing 科学出版社, 2017). Xu was a mechanical engineer and
  mathematician whom Qian first met on the cruiser from America to
  China. Ye Yonglie 叶永烈, \emph{走进钱学森} {[}Qian Xuesen: A
  biography{]} (Shanghai: 上海交通大学出版社 Shanghai Jiao Tong
  University Press, 2016), 440. But history prefers to not remember this
  humble beginning; Qian Xuesen is seen as the brilliant mind behind it
  all.} Soon after, Qian began to talk about social, on top of material,
transformation, and especially addressed the relevance of cybernetics:

\begin{quote}
In 1948, Wiener considered the idea that {[}his{]} cybernetic theory may
have social efficacy ``false hopes,'' and that ``extend{[}ing{]} to the
fields of anthropology, of sociology, of economics, the methods of the
natural sciences, in the hope of achieving a like measure of success in
the social fields'' ``an excessive optimism.'' The modern development of
cybernetic theory has proved that Wiener's 1948 view was too
conservative. Applying engineering techniques to social domains is not
``an excessive optimism,'' but a reality.\footnote{Qian Xuesen 钱学森and
  Song Jian 宋健, \emph{工程控制论} {[}Engineering cybernetics{]}
  (Beijing: Science Publishing 科学出版社, 1980), xiv. Also see Norbert
  Wiener, \emph{Cybernetics or Control and Communication in the Animal}
  (Cambridge: MIT Press, 1965), 162.}
\end{quote}

\noindent These words come from the preface of the latest and substantially
expanded Chinese volumes of \emph{Engineering Cybernetics}. In contrast,
in the original edition of \emph{Engineering Cybernetics}, he referenced
the French origin of \emph{cybernétique} (``the science of civil
government'') and remarked: ``{[}this{]} grandiose scheme of political
sciences has not, and perhaps never will, come to fruition.''\footnote{Tsien,
  \emph{Engineering Cybernetics}, vii.} Apparently, decades in China
made him change his mind.

By that time, Song Jian, Qian's disciple, colleague, and the new
\emph{Engineering Cybernetics}'s coauthor, had used missile science to
project population growth---with severe oversights---and led the CCP
leadership onto the track of implementing the stringent one-child
policy.\footnote{Greenhalgh, ``Missile Science, Population Science.''}
This amounted to a concrete success of what Qian called ``social
cybernetics,'' ``a new science'' for accelerating China's socialist
modernization and the accompanying technological revolution.
``Scientific decision-making'' cast in systems cybernetic language came
to be framed as progressive governance. ``The scope of `engineering'
keeps expanding,'' Song wrote complacently in a 1984 encyclopedia entry,
``problems traditionally reserved for the social sciences can be solved
through engineering methods; in all cases {[}using these methods{]}
achieves more precise, prescient outcomes than leaving them to mere
bureaucratic judgments.''\footnote{Song Jian 宋健, ``工程控制论''
  {[}Engineering cybernetics{]}, \emph{系统工程理论与实践} {[}Systems
  engineering theory \& practice{]}, no. 2 (1985): 3.} Two years later,
Song became State Councilor, which ranks immediately below the Vice
Premiers.

Undergirding this ascension of systems cybernetics on the political
stage and official culture were some crucial discursive gymnastics that
also informed the formulation of systems journalism. Qian welded systems
cybernetics to the official doctrine. Remarkably, he did it in a way
that grafted ``modern'' technoscience, which the CCP now craved, onto
convictions from its revolutionary past, and thus boosted regime
legitimacy amid economic and political uncertainties. Qian's rootstock
here, when looked at closely, were the ideas of Friedrich Engels, not
Karl Marx, nor Mao Zedong.

To correct Norbert Wiener's above-quoted reservation, for example, Qian
resorted to Engels's \emph{Socialism: Utopian and Scientific}:

\begin{quote}
Engels once predicted, under the circumstances of socialism, ``anarchy
in social production is replaced by conscious organization on a planned
basis.'' The regulatory capacity of the socialist economy enables a
self-sustaining economic system, which is essentially an automatic
system. . . . The systemic dynamics that cybernetic theory tackles can
be found in high-level systems (i.e., social systems). Therefore it is
unreasonable to call the social efficacies of cybernetics ``false
hopes.'' Instead, it is an actual hope that has already dawned on
us.\footnote{Qian and Song, \emph{工程控制论} {[}Engineering
  cybernetics{]}, xiv.}
\end{quote}

\noindent Engels's passion for science and technology, and inspired by these
topics, his law-like prediction about history, were what made him so apt
for promoting a systems cybernetic conception of society. In the PRC and
the Soviet Union alike, Marx-Engels existed as a mystical joint identity
for orthodox enunciation. The history of global Marxisms, however, shows
that Marx's reception was usually mediated by Engels's ``defining
influence.''\footnote{Terrell Carver, Gerard Bekerman, and Cecil L.
  Eubanks, ``Marx and Engels. The Intellectual Relationship,''
  \emph{Studies in Soviet Thought} 31, no. 4 (1986).} Rather than
authoritative accounts of Marx's ideas, Engels's \emph{Dialectics of
Nature} and \emph{Anti-Dühring} (particularly its last section,
\emph{Socialism: Utopian and Scientific}) were immensely popular
simplifications and extrapolations of Marx's often dense writings.
Instead of giving voice to Marx posthumously, Engels was in fact
elaborating his own. And it was a voice that centered on the subject
that he, but not Marx, appeared genuinely interested in---natural
science. According to Engels (in Marx's name), both the human and the
natural are subject to the law of dialectics, and they make up a
totality where history is made through human mastery of nature, pushing
back its boundaries.\footnote{By contrast, Marx's arguments are largely
  confined within the capitalist mode of production. Paul Thomas,
  \emph{Marxism \& Scientific Socialism: From Engels to Althusser}
  (Milton Park, United Kingdom: Taylor \& Francis, 2008), 39--40. Also
  see Taylor, ``Dialectics of Nature or Dialectics of Practice?''} For
example, according to Engels, the socialist replacement of anarchy with
conscious planning (which Qian quoted) amounts to historical progress,
for ``man finally . . . leaves the conditions of animal
existence.''\footnote{Friedrich Engels, \emph{Herr Eugen Dühring's
  Revolution in Science (Anti-Dühring)}, ed. C. P. Dutt, trans. Emile
  Burns (New York: International Publishers, 1939), 318.} Inspired by
Darwin, Engels upholds the victory of socialism and communism as the
teleological logic of history.

The Engels portion of the Marx-Engels orthodoxy was invoked for local
needs. Neither the Russian nor Chinese Bolshevik Revolution followed
Marx's original arguments. But Engels's scientific socialism proved much
more assimilable in the ruling ideology of these Marxist
regimes.\footnote{Thomas, \emph{Marxism \& Scientific Socialism}.}
Stalin furthered this scientistic orientation by differentiating
historical materialism from dialectical materialism, positing the former
as the extension of the principles of the latter.\footnote{Joseph
  Stalin, \emph{Dialectical and Historical Materialism} (Moscow: Foreign
  Languages Publishing House, 1938).} This unequivocally implied the
inferiority of the studies of history to the studies of
science.\footnote{Taylor, ``Dialectics of Nature or Dialectics of
  Practice?''} In early 1950s China, fixated on the lost skull of Peking
Man and inspired by Engels's ``labor created humanity'' theory, the
newly-in-power CCP enlisted the scientific elite to propagate the idea
that the apes of China evolved into modern Chinese through hard labor, a
belief that brought together the entire nation.\footnote{Sigrid
  Schmalzer, \emph{The People's Peking Man: Popular Science and Human
  Identity in Twentieth-Century China} (Chicago: University of Chicago
  Press, 2009).} Although the USSR's augmentation of Engels had
profoundly influenced the doctrine of Chinese communism, Mao himself did
not care for all that talk about science and scientific laws. By
contrast, Mao awarded huge weight to human will and agency; for the
Maoist dialectics, history is unpredictable, always unfolding as the
willful human subject struggles against the power structure: There is no
end to the revolution.\footnote{Laikwan Pang, ``Dialectical
  Materialism,'' in \emph{Afterlives of Chinese Communism: Political
  Concepts from Mao to Xi}, ed. Christian Sorace, Ivan Franceschini, and
  Nicholas Loubere (London: Verso Books, 2019).}

In this sense, the reform rhetoric was the most efficient when official
Marxism displayed the face of Engels and hid that of Mao. By resorting
to Engels of the ``theory of Marx-Engels,'' epistemological consistency
was maintained. This was how Qian built on Xu Guozhi's ideas. In late
1979, he broadened Xu's initial notion of the ``system'' to include both
human practices and natural processes. In 1985, during the launch of the
Chinese Association for Systems Engineering, Qian outlined his vision
for ``systems science,'' a synergistic whole that leveraged the ``messy
assortment'' of related areas in the West to the plane of philosophy.
Qian's systems science has three interdependent layers, moving from the
most concrete techniques to the foundational \emph{basic science} that
Qian called ``systematics'' (\emph{xitong xue}), which he bridged to
Marxism (Fig. 2).\footnote{Qian, ``我对系统学的认识历程'' {[}How I came
  to systematics{]}.} By ironing out this chain of articulations and
using Engels as an interface, Qian demonstrated a dialectic relationship
between the development of science and technology and that of Marxist
philosophy. This insistence actually made Qian the conservative
stronghold in the prolonged debates about whether Marxism might serve as
the totalist structure for scientific knowledge.\footnote{Miller,
  \emph{Science and Dissent.}} On the opposite side were relatively
young scientists who advocated a non-instrumentalist view of science as
intrinsically valuable and autonomous from Marxism.

\begin{figure}
    \centering
    \includegraphics{wu-figure-two.png}
    \caption{Qian Xuesen's Systems Science}
    \label{fig:two}
\end{figure}

With his scheme of systems science, Qian undertook a grand journey to
appropriate and reinterpret bits and pieces from a variety of fields
including cognition, physiology, behavioral science, architecture,
aesthetics, education, and ecology. He insisted on thinking across
boundaries between machines and bodies, and nature and society,
promoting a fundamentally posthumanist outlook.\footnote{For Qian's take
  on \emph{qigong}, see Liu, n\emph{Information Fantasies}, chap. 1.}
Qian's enormous state support and cultural prestige helped popularize a
distinct way for conceptualizing the world: It framed all things in
relation to larger complex, hierarchical, nested systems that are
interconnected (co-variant) via flows of information (in a cybernetic
sense). In an incidental and ironic manner, viewing society in terms of
interdependent systems enabled an intellectual current free from the
yoke of Marxist economic determinism.\footnote{See He Guimei 贺桂梅,
  \emph{``新启蒙''知识档案------80年代中国文化研究} {[}Knowledge
  archives of the ``New Enlightenment''{]} (Beijing: Peking University
  Press 北京大学出版社, 2010): 309--33.}
  
\end{CJK*}


\hypertarget{systems-journalism}{%
\section{Systems Journalism}\label{systems-journalism}}

\begin{CJK*}{UTF8}{gbsn} 

In the aftermath of the 1983 Anti-Spiritual Pollution campaign and the
suppression of American references to mass communication, journalism
scholars similarly grafted systems cybernetics onto Engels's scientistic
Marxism. Engels's general evolutionary take on the development of
science and technology was leveraged to promote information technologies
and press infrastructures as proxies for historical progress. His claim
that motion in nature and human history, as well as the motion of
thought, all abide by one unitary set of dialectic laws became the final
seal to a view of news production, distribution, and consumption that
leaves no role for human subjectivity.

One of the first such pieces, entitled ``Information dissemination and
historical materialism,'' performed acrobatics like Qian's systematics
and in the process assimilated systems thinking in the
scholarship.\footnote{Fan Dongsheng 范东升, ``信息传播与历史唯物论``
  {[}Information communication and historical materialism{]},
  \emph{新闻学刊} {[}Journalism{]}, no. 3 (1985).} Journalism was ``the
subsystem of social information dissemination,'' the author specified,
and hence was interwoven with other subsystems such as politics,
economy, and ideology. He further opposed considering journalism part of
the superstructure or ideology, a bold conclusion reached with seemingly
Marxist analytics. Quoting Engels on the industrial revolution, he
argued that because the press of a given society inherited the same
relations and means of production, the relationship between the
resulting information system and society should be more precisely
described as ``isomorphic.'' This arc of arguments smacks of the
cybernetic flavor of information (i.e., news as signals generated by a
system in motion), glossing over the ideological negotiations by media
owners and producers from certain class positions. Importantly, while
the press as the subsystem processed output from the economic subsystem
(rather than audience feedback), the press itself functioned as the
feedback central to the evolution of the larger cybernetic system that
is society. The author's objective was to push for press reforms by
arguing that actively restructuring this crucial subsystem would effect
desirable changes of the suprasystem.

In response to student demonstrations demanding political---in addition
to economic--- reform from January to May 1987, the CCP wielded another
campaign to ``combat bourgeois liberalization.'' After this round of
political contraction, major journalism periodicals saw a burst of works
integrating systems cybernetics through Qian-styled ideological
acrobatics. In 1988, a renowned annual academic publication,
\emph{Annals of Journalism Studies}, dedicated the entire volume to
``analyzing journalistic phenomena using `systems science.'\,'' Many
even set out to reconstruct typical concepts and propositions from
American communication in terms of systems cybernetics. ``Evidently,''
wrote one, ``Schramm has underestimated the value of information theory,
cybernetics, and systems theory in building the theoretical frameworks
of {[}human{]} communication study.''\footnote{Wu Wenhu 吴文虎,
  ``传播学理论架构初探'' {[}Exploring communication's theoretical
  frameworks{]}, \emph{新闻学刊} {[}Journalism{]}, no. 5 (1986): 28.}

In early 1989, a monograph entitled \emph{Systems theory of journalism}
(\emph{xitonglilun xinwenxue}) came out, one of the first publications
with similar titles. Using the vocabulary of Engels's materialist
metaphysics, the author stated that news is the agentic reflection of
the objective reality in the mind of the reporter: ``When both `natural
information' and `cultural information' {[}of motion in nature and in
society, respectively{]} come into contact with the reporter's thinking
brain, news is the register left on his brain.''\footnote{Wang
  Yimin王益民, \emph{系统理论新闻学} {[}Systems theory of journalism{]}
  (Wuhan: Huazhong University of Science and Technology
  华中理工大学出版社, 1989), 54--56.} The book went on to spell out news
value using Engels's Darwinian-Marxist argumentation about humans'
continued transformation of nature: Occurrences in news (\emph{xinwen
shishi}), after coming to public attention, affect the public's inner
world and spur action to change the material world; if these changes are
positive, the news is deemed to have ``social effects,'' which was the
ultimate measure of news value.\footnote{Wang, 110--14.} News value was
no longer defined by how much information about world happenings it
contained, as was argued when Shannon's formula was the main influence.
It was now measured by the news's effects on the larger system.
Depending on the selection of actual occurrences in the news, its
dissemination could aid or hinder system functioning, by which the
author meant the state-led modernization project. Importantly, in this
conception, neither news production nor reception involves
meaning-making or more generally, ``the anthropological principle of
human interaction with the world.''\footnote{Taylor, ``Dialectics of
  Nature or Dialectics of Practice?'' 290.}

Yet the logic of systems journalism consolidated from a position of
structural power. In the first Hong Kong journalism conference joined by
mainland colleagues, local scholars were deeply impressed when the
deputy director of the CASS Journalism Institute started drawing the
plan for Chinese press reforms.\footnote{Leonard Chu 朱立,
  ``香港新闻传播学界的成名与想象 (1927--2006)'' {[}Fame and imagination
  in Hong Kong journalism and communication research{]},
  \emph{国际新闻界} {[}Chinese journal of journalism and
  communication{]}, no. 5 (2017): 85--108.} Commissioned by the central
propaganda department, CASS surveyed newspapers across the country and
mapped bureaucratic affiliations (all state-owned) and specialties of
the rapidly expanding press universe (CASS 1986). Through reform,
announced the team, this universe should develop into a ``party
press-anchored, multi-layered press system'' that efficiently
disseminates information to meet the variety of needs of economic
modernization. Importantly, with the anchoring layer of party organs in
place, the whole press system would allow localized management
experimentation.\footnote{Xu Yaokui 徐耀魁et al.,
  ``关于新闻体制改革的设想'' {[}Envisioning journalism reform{]},
  \emph{新闻学刊} {[}Journalism{]}, no. 5 (1986).} In line with this
general aspiration, other scholars pointed out that, as modernization
begot exponential information growth, the conventional mode of linear
media control hindered timely transmission and response to feedback.
Employing concepts such as noise and filter, they advocated that each
newspaper was itself a cybernetic subsystem that must have necessary
decision-making power to adapt to its changing environment, so that the
whole press system remains resilient and constantly improves information
delivery for the suprasystem.\footnote{Yu Guoming 喻国明,
  ``试论建立具有中国特色的社会主义新闻事业体制'' {[}Building Chinese
  journalism with socialist characteristics{]}, \emph{新闻学刊}
  {[}Journalism{]}, no. 1 (1986).} Put differently, the press needed
some autonomy for ``self-organization'' in order to better fulfill its
obligation as a state apparatus. At once strikingly totalistic and
excitingly open-ended, this vision can only emerge from the center of a
press universe where all the outlets are legible, responsive, and
subject to large-scale coordination. It is also a vision of technocracy,
held liable for the role of media in society and its ``subsystems.''

These conditions hardly exist in liberal capitalist societies. Chinese
journalism scholars had been enthralled by a bird's-eye perspective of
mass media even before the CCP's repetitive censures against
``liberal-capitalist'' influences (in 1983, 1987, and finally, 1989). In
1982 they were introduced to ``The Structure and Function of
Communication in Society'' by Harold Lasswell, a political scientist
that Schramm canonized as a ``forefather'' of communication study.
Inspired by postwar structural functionalism, this article discusses
mass media by ``viewing {[}it{]} as a whole in relation to the entire
social process.''\footnote{Lasswell, ``The Structure and Function of
  Communication in Society,'' in \emph{The Communication of Ideas}, ed.
  L. Bryson (New York: Harper and Row, 1948), 38.} This high-level
conceptualization of media as an (oddly holistic) institution found
resonance among Chinese scholars. In 1980s China, the dearth of such
structural analyses about ``the media system'' from American mass
communication\footnote{This differs from empirically studying mass
  communication in institutional and organizational settings, or
  studying audience behavior in its social location. See Josephine R.
  Holz and Charles R. Wright, ``Sociology of Mass Communications,''
  \emph{Annual Review of Sociology} 5 (1979).} was keenly
noticed.\footnote{It was, for example, lamented that after Lasswell no
  American communication scholar seemed to care about media-society
  relationships. Wu, ``传播学理论架构初探'' {[}Exploring communication's
  theoretical frameworks{]}, 27.} And it was Chinese journalism
scholarship's institutional position, rather than mere intellectual
curiosity, that explained its preoccupation with media-society
relationships. In a sense, it resorted to systems cybernetics to
displace a perceived vacuum in mass communication study. Moreover,
whereas functionalism provided little inspiration for initiating
structural transformation, systems journalism was premised on continued
evolution. Even better, it regarded the press as a locus to initiate
changes in other interdependent systems and eventually society's
material development, effectively granting agency to media reformers
(and later the propaganda leadership).

Weeks after the release of \emph{Systems theory of journalism},
protesters filled Tiananmen Square. Appropriating systems cybernetic
ideas, the 1980s journalism reformers painstakingly loosened
restrictions over organizational management, format, and genre---what we
may call newsmaking infrastructures. The limited amount of ``press
freedom'' thereby achieved was, it would not be far-fetched to say,
seized upon by the broader cultural currents---an explosive combination
of liberal and Confucian humanism---in a bid for political reforms.
Following the crackdown on June 4th, the CCP took a series of drastic
measures to ``tidy the house,'' and the media sector was hit hard.
Scholarly conversations in Chinese journalism again went through severe
rectification, particularly with regard to the field's ``blind''
adoption of Western communication study.

But the systems journalism strain remained intact. It was particularly
telling that, in the post-1989 climate, \emph{Systems theory of
journalism} kept releasing reprints into the new millennium.\footnote{A
  winner of several province-level awards, the book first appeared in
  March 1989, with its fourth reprint running in 1993, and new editions
  in 1996, 1999, and 2003.} With regard to news value, the functioning
system made further headway to displace ``objective reality.'' In a 1993
theory article entitled ``The ontology of news,'' the author builds on
Engels's and Qian's argumentation to contend that news is
``information'' rather than ``actual occurrences'' (\emph{shishi},
typically translated as ``facts''), as information represents human
praxis transcending the object-subject binary. News is worthy if people,
``as they set goals and make plans for themselves, can utilize the
information'' to ``constantly adjust the relationships between the
system and its environment and between various subsystems, in order to
optimize the overall benefit and steadily improve their
self-organization.''\footnote{Chen Jian 陈坚,
  ``新闻本体论------关于事实与信息的比较研究'' {[}The ontology of news:
  comparing facts and information{]}, \emph{新闻大学} {[}Journalism
  research{]}, no. 3 (1993): 16.} There was little mention of
``objective reality''; journalism was enclosed in the system. This was
never what Wilbur Schramm had in mind, despite his interest in
integrating information theory into communication study. As early as
1955, he acknowledged a chasm between Shannon's information, ``concerned
with the number of binary choices necessary to specify an event in a
system,'' and ``our concept of information in human communication,''
which was really ``concerned with the relation of a fact to outside
events.''\footnote{Schramm, ``Information Theory and Mass
  Communication,'' 144.}

The CASS Institute of Journalism became ``the Institute of Journalism
and Communication'' in 1992, a naming convention immediately followed by
numerous journalism departments across China. The juxtaposition
indicated by ``and'' quickly came off as redundant. Significantly, it
was also in 1992 that the Ministry of Education confirmed \emph{xinwen
chuanbo} or ``journalism communication study'' as a first-tier
discipline, a direct signal of academic stature and resource
allocation.\footnote{Hu, Ji, and Zhang, ``Building the Nation-State,''
  388--90.} In Chinese, this phrase literally means the ``study of news
dissemination.'' I hereafter use ``Jcomm'' to avoid its easy conflation
with ``communication'' in an Anglophone context.

\end{CJK*}


\hypertarget{legacies-of-systems-journalism}{%
\section{Legacies of Systems
Journalism}\label{legacies-of-systems-journalism}}

\begin{CJK*}{UTF8}{gbsn} 

As ``icebreaking'' as Schramm's visit was later portrayed to be, the
birth of Jcomm entailed much more than a takeover by---or, depending on
one's stand, a critical adoption of---American communication study. It
entailed reformist journalism scholars carrying forward normative
notions about journalism with a series of articulations relayed through
the ideas of Schramm, Qian Xuesen, and Engels.

Systems journalism could only have emerged in the broader cultural
currents of 1980s China. First was the dream of unified science, shared
by all three men. Schramm assured the Chinese that communication would
eventually integrate all other social sciences and announced it as part
of a broader convergence between social and natural sciences.\footnote{See
  Xu, ``施拉姆对中国传播学研究的影响:纪念施拉姆来新闻研究所座谈30周年''
  {[}Schramm's impacts on Chinese communication research{]}.} This
prediction tapped right into the particular zeitgeist, with Qian Xuesen
on his crusade to synthesize all the disciplines, albeit in his own
comfort zone of engineering and systems analysis. And given the
precarious political environment, all this cross-border knowledge
hybridization nonetheless gained momentum because Chinese Marxism
consisted of Marx's arguments about human history and Engels's
conjectures from the non-humanistic dimension of natural science.
Literally a ``worldview,'' this Marxist orthodoxy revered universal iron
laws (i.e., dialectics) on moral and ideological grounds. Further, the
Deng Xiaoping regime shared with Engel's natural dialectics the
pragmatism in human's unfolding interactions with technology. Finally,
the post-Mao regime's characteristic predilection (i.e., ``Cross the
river by feeling the stones'') bore some ``ontological affinity'' with
cybernetics---that is, leaving the ``black box'' closed but instead
preoccupied with continuous performative engagements between
heterogeneous agents.\footnote{For ``ontological affinity'' see
  Pickering, \emph{Cybernetic Brain}. Also, on the ontological stance of
  cybernetics, see Galison, ``Ontology of the Enemy.''} In Euro-America,
by contrast, this goes against the ``modern ontology'' presumed by
scientific and bureaucratic establishments, which explains cybernetics'
quick retreat into in ``nomad sciences'' and countercultures.\footnote{Pickering,
  \emph{Cybernetic Brain}; Turner, \emph{From Counterculture to
  Cyberculture.}}

The formulation of systems journalism also hinged on institutional
conditions. Qian Xuesen believed that ``social cybernetics'' would never
emerge in capitalist societies (``because the problem with capitalism is
that from the onset of production there is no conscious social
regulation'').\footnote{Qian and Song, \emph{工程控制论} {[}Engineering
  cybernetics{]}, xiv.} To be clear, however, in liberal democracies,
systems approaches had heavily influenced human sciences and state
bureaucratic management.\footnote{Heyck, \emph{Age of System}.} The
adoption of operational research in the US, for example, was entwined in
the rise of the military-industry-academic complex.\footnote{Erik P.
  Rau, ``The Adoption of Operations Research in the United States during
  World War II,'' in \emph{Systems, Experts, and Computers}, ed. Agatha
  C. Hughes and Thomas P. Hughes (Cambridge: MIT Press, 2011).} Social
administration by systems cybernetics, therefore, is less productively
explored as a feature of authoritarianism (or according to Qian,
socialism). Rather, it boils down to its interaction with the concrete
political, economic, and discursive contestations in a domain, and the
consequences of the exact engineering measures.\footnote{See also
  Peters, \emph{How Not to Network}.} In the case of Chinese population
control, for example, systems cybernetics triumphed because the state's
intervention in reproduction was deemed legitimate, the bureaucratic
capacity at its disposal extraordinary, and competing social science
expertise weak.

One finds similar circumstances in China's news sector, where Party
propagandists took it upon themselves to revamp the normative
injunctions about the press. But journalism also is distinct, for it
belongs to the realm of the symbolic. Systems cybernetics had to wiggle
its way through a congestive discursive terrain. On the one hand,
acknowledging the importance of relationality and societal betterment,
systems journalism rejected the image of individuals in isolation, the
purported flaw of Western communication. At the same time, by leaning on
the CCP's agenda to phase out the planned economy and push for
marketization, it refused to succumb to top-down propaganda imperatives.
Most remarkably, while systems journalism steered away from the
``Maoist'' view that valorized the press for providing a voice on behalf
of the people and for fostering the socialist revolution, it also
distanced itself from the ``capitalist'' view that evaluated news media
in the market of attention where all players act out of competing
interests. Finally, it had foreclosed on issues of social actors'
positionality in meaning-making.

In the ensuing decades, explicit references to systems cybernetics took
a backseat, but its legacies have prevailed in powerful
forms.\footnote{This means rather than tracing the ``citational
  network'' of specific theories, we may more productively examine these
  legacies as constitutive of a ``culture,'' in the form of shared
  parlance, tacit knowledge, and unspoken assumptions.} It is beyond
this article's scope to delineate these manifold developments, but I
offer a few observations as provocation. First, the ways in which
China's media reforms transpired in the 1990s bore much resemblance to
the ideas inspired by systems cybernetics. China's fast swelling media
sector underwent state-regulated marketization involving system-level
measures, such as administrative consolidation of party organs and
radical commercial experiments at the margins. These ``structurally
focused'' moves are often sidelined by discussions concentrating on
overt censorship ``aimed at politically dissenting
publications.''\footnote{Yuezhi Zhao, ``From Commercialization to
  Conglomeration: The Transformation of the Chinese Press Within the
  Orbit of the Party State,'' \emph{Journal of Communication} 50, no. 2
  (2000): 15; Angela Xiao Wu and Luzhou Li, ``Localism in Internet
  Governance: The Rise of China's Provincial Web,'' \emph{China
  Information}, 2021. As foreseen by Cheek: ``Control of order as much
  as suppression of free ideas are on the minds of {[}state
  propaganda{]} officials'' for press reform. Cheek, ``Redefining
  Propaganda.''} One also ponders the similarity between the liberating
vision of the socialist press comprised of numerous outlets as
self-organizing cybernetic systems and the so-called ``one head, many
mouths'' landscape in reality, where market-induced media
decentralization was wielded for state propaganda purposes.\footnote{Guoguang
  Wu, ``One Head, Many Mouths: Diversifying Press Structures in Reform
  China,'' in \emph{Power, Money, and Media: Communication Patterns and
  Bureaucratic Control in Cultural China} (Evanston, IL: Northwestern
  Univeristy Press, 2000); Daniela Stockmann, \emph{Media
  Commercialization and Authoritarian Rule in China} (Cambridge:
  Cambridge University Press, 2012).} As economic disparity increased,
media marketization led to the demise of outlets for underprivileged
groups.\footnote{Zhao, ``From Commercialization to Conglomeration."} But
another contributor may be the orientation of press reforms toward
information delivery for an economic (sub)system, unconcerned with
audience subjectivities and identities.

Systems journalism has sedimented in the now exceedingly resourceful
discipline and popular major of Jcomm, and the language of systems
cybernetics continues to percolate in its authoritative texts. China's
latest national textbook \emph{Introduction to journalism} continues to
define news as ``information about recent actual occurrences
(\emph{shishi}),'' in which information does not go by its colloquial
usage, but specifically means ``the thing that reduces the recipient's
uncertainty,'' ``as per its narrow definition by Shannon.''\footnote{Li
  Liangrong 李良荣, \emph{新闻学概论} {[}Introduction to journalism{]}
  (Shanghai: Fudan University Press 复旦大学出版社, 2018), 66--67.} The
reference to ``objective reality,'' which disappeared with the
prevailing systems journalism, remains absent from the current
definition. In a methodical content analysis of Chinese textbooks, one
Taiwanese researcher was deeply confused by the ubiquity of
``information'' in Chinese Jcomm textbooks, compared to its scarcity in
Taiwanese counterparts.\begin{CJK*}{UTF8}{bsmi}\footnote{Liu Shengji 劉勝驥,
  ``國科會專題研究:中國大陸高校教科書中政治思想教育之研究''
  {[}Political ideologies in mainland Chinese college textbooks{]},
  2004, 193--94.}\end{CJK*}\begin{CJK*}{UTF8}{gbsn} He also noted that the phrase ``social control''
(\emph{shehui kongzhi}) appeared hundreds of times and suspected it to
reflect Chinese authoritarianism,\footnote{Liu, 203.} without realizing
Jcomm's invocation of its cybernetic meaning---purposeful influence
toward a predetermined goal through information processing.\footnote{Ironically,
  it was American psychologists who applied cybernetics to develop
  techniques of ``coercive persuasion'' by experimenting with ``extreme
  environmental control.'' Rebecca Lemov, ``Running Amok in Labyrinthine
  Systems: The Cyber-Behaviorist Origins of Soft Torture,'' \emph{Limn}
  1, no. 1 (2011).} However, the ``goal'' here, or journalism's
principled priority of ``systematic benefit,'' is hard to pin down. In
textbooks, official documents, and authoritative commentaries, it shifts
between economic prosperity, social stability, and international
rivalry. Whereas this flexibility may echo the notion of adaptable,
learning systems, the assumed ``oneness'' is a far cry from Marxian
dialectics' attention to contradiction and historicity.\footnote{Levins,
  ``Dialectics and Systems Theory.''} It is especially so when the
system's shifting goals are invariably seen as hindered by
internal/domestic protests and grievances, which journalism is tasked to
assuage through dispelling people's ``uncertainty.''

Finally, the understanding of truth inherent in this conception of
newsmaking is about neither correspondence nor coherence.\footnote{See
  Jack Fuller, \emph{News Values: Ideas for an Information Age}
  (Chicago: University of Chicago Press, 1996).} First, it is not about
sorting out facts in pursuit of accounts that comprehensively correspond
to (an outside) reality. While the informational ideal of Western
journalism aspires to display disciplined impartiality,\footnote{Schudson,
  \emph{Discovering the News}.} news in China can be justifiably partial
as long as the selection of ``actual occurrences'' for dissemination
contributes to the vitality of the larger system. This understanding of
truth is also not about coherence. The imperative for coherence in
Chinese press history manifested most vividly in the notion of
``substantial truth,'' which had a high tolerance for fabricating
details as long as the produced account better reflects ``the inherent
logic of reality'' according to official ideology. But when viewed as
system-enhancing cybernetic information, news is not ideological, at
least not in the sense that ideology means ``seeking to efface
contradiction and produce an ostensibly coherent, readable account of
the world.''\footnote{Eva Cherniavsky, \emph{Neocitizenship: Political
  Culture after Democracy} (New York: NYU Press, 2017), 141.} In the
end, such newsmaking is not interested in the real. It is highly
malleable, essentially indifferent to the actual content, and has no
place for normative value judgments hinged on something external to the
system.

We may see such a conception of truth as \emph{performance}---as in
``performative ontology,'' which Andrew Pickering uses to name the
ontology staged by British cybernetics.\footnote{Pickering,
  \emph{Cybernetic Brain}.} Casting aside the need to comprehend and
articulate the working principles of the cybernetic project and its
surroundings alike---in our case journalism and the world it operates
in---actions can still be consequential, as the world changes through
continuing interlinked performances. Importantly, the resulting
performative relationship is not symmetric, as envisioned by
cybernetics-inspired theorists such as Haraway,\footnote{Donna J.
  Haraway, \emph{The Companion Species Manifesto: Dogs, People, and
  Significant Otherness} (Chicago: Prickly Paradigm Press, 2003).} but
``imperative,''\footnote{Galison, ``Ontology of the Enemy,'' 256.}
because norms and commitments of journalism are, foremost, furnished by
the overriding priorities of the regime suprasystem. While such an
ontological vision has manifested in many material and ideational forms,
Chinese news media may be a site to explore how it plays out in the
practice of the (re)production of reality and, in turn, of the public
culture that this reality animates.

\end{CJK*}





\section{Bibliography}\label{bibliography}

\begin{hangparas}{.25in}{1} 

\begin{CJK*}{UTF8}{gbsn}

An, Gang 安岗. ``研究我们的读者'' {[}Studying our readers{]}. In
\emph{China Journalism Yearbook}, 173, 177. Beijing: CASS, 1982.

Berlo, David K. \emph{The Process of Communication: An Introduction to
Theory and Practice}. New York: Holt, Rinehart and Winston, 1960.

Carey, James W. ``Some Personal Notes on US Journalism Education.''
\emph{Journalism} 1, no. 1 (2000): 12--23.

Carver, Terrell, Gerard Bekerman, and Cecil L. Eubanks. ``Marx and
Engels. The Intellectual Relationship.'' \emph{Studies in Soviet
Thought} 31, no. 4 (1986): 329--34.

Cheek, Timothy. \emph{Propaganda and Culture in Mao's China: Deng Tuo
and the Intelligentsia}. London: Clarendon Press, 1997.

Cheek, Timothy. ``Redefining Propaganda: Debates on the Role of
Journalism in Post-Mao Mainland China.'' \emph{Issues \& Studies} 25,
no. 2 (1989): 47--74.

Chen, Chongshan 陈崇山. ``施拉姆的理论对我的指引'' {[}How Schramm's
theory guided me{]}. \emph{新闻与传播研究} {[}Journalism and
communication{]}, no. 4 (2012): 14--18.

Chen, Chongshan 陈崇山, and Xiuling 弭秀玲 Mi, eds.
\emph{中国传播效果透视} {[}Perspectives on media effects in China{]}.
Shenyang: Shenyang Publishing 沈阳出版社, 1989.

Chen, Jian 陈坚. ``新闻本体论------关于事实与信息的比较研究'' {[}The
ontology of news: comparing facts and information{]}. \emph{新闻大学}
{[}Journalism research{]}, no. 3 (1993): 16--17, 48.

Chen, Lidan 陈力丹, and Yigao 王亦高 Wang.
``我国新闻传播学量化研究的艰难起点'' {[}The fraught origins of
quantitative research in Chinese journalism-communication{]}.
\emph{当代传播} {[}Contemporary communication{]}, no. 2 (2009): 8--10.

Cherniavsky, Eva. \emph{Neocitizenship: Political Culture after
Democracy}. New York: NYU Press, 2017.

Chu, Leonard 朱立. ``香港新闻传播学界的成名与想象(1927--2006)'' {[}Fame
and imagination in Hong Kong journalism and communication research{]}.
\emph{国际新闻界} {[}Chinese journal of journalism and communication{]},
no. 5 (2017): 85--108.

Davies, Gloria. ``Discursive Heat: Humanism in 1980s China.'' In \emph{A
New Literary History of Modern China}, edited by David Der-Wei Wang,
758--64. Cambridge: Belknap Press/Harvard University Press, 2017.

Engels, Friedrich. \emph{Herr Eugen Dühring's Revolution in Science
(Anti-Dühring)}. Edited by C. P. Dutt. Translated by Emile Burns. New
York: International Publishers, 1939.

Fan, Dongsheng 范东升. ``信息传播与历史唯物论`` {[}Information
communication and historical materialism{]}. \emph{新闻学刊}
{[}Journalism{]}, no. 3 (1985): 3--8.

Fuller, Jack. \emph{News Values: Ideas for an Information Age}. Chicago:
University of Chicago Press, 1996.

Galison, Peter. ``The Ontology of the Enemy: Norbert Wiener and the
Cybernetic Vision.'' \emph{Critical Inquiry} 21, no. 1 (1994): 228--66.

Gerovitch, Slava. \emph{From Newspeak to Cyberspeak: A History of Soviet
Cybernetics}. Cambridge: MIT Press, 2004.

Ghosh, Arunabh. \emph{Making It Count: Statistics and Statecraft in the
Early People's Republic of China}. Princeton, NJ: Princeton University
Press, 2020.

Greenhalgh, Susan. ``Missile Science, Population Science: The Origins of
China's One-Child Policy.'' \emph{China Quarterly}, no. 182 (2005):
253--76.

Hamrin, Carol, and Timothy Cheek. \emph{China's Establishment
Intellectuals}. Armonk, NY: M. E. Sharpe, 1986.

Haraway, Donna Jeanne. \emph{The Companion Species Manifesto: Dogs,
People, and Significant Otherness}. Chicago: Prickly Paradigm Press,
2003.

He, Guimei 贺桂梅. \emph{``新启蒙''知识档案------80年代中国文化研究
{[}Knowledge archives of the ``New Enlightenment''{]}}. Beijing: Peking
University Press北京大学出版社, 2010.

Heyck, Hunter. \emph{Age of System: Understanding the Development of
Modern Social Science}. Baltimore, MD: Johns Hopkins University Press,
2015.

Highton, Jake. ``\,`Green Eyeshade' Profs Still Live Uncomfortably with
`Chi-Squares'; Convention Reaffirms Things Haven't Changed Much.''\emph{
Journalism Educator} 44, no. 2 (1989): 59--61.

Holz, Josephine R., and Charles R. Wright. ``Sociology of Mass
Communications.'' \emph{Annual Review of Sociology} 5 (1979): 193--217.

Hu, Zhengrong, Ji Deqiang, and Zhang Lei. ``Building the Nation-State:
Journalism and Communication Studies in China.'' In \emph{The
International History of Communication Study}, edited by Peter Simonson
and David W. Park, 387--413. New York: Routledge, 2015.

Kline, Ronald R. \emph{The Cybernetics Moment: Or Why We Call Our Age
the Information Age}. Baltimore, MD: Johns Hopkins University Press,
2015.

Lasswell, Harold D. ``The Structure and Function of Communication in
Society.'' In \emph{The Communication of Ideas}, edited by L. Bryson,
37--51. New York: Harper and Row, 1948.

Lemov, Rebecca. ``Running Amok in Labyrinthine Systems: The
Cyber-Behaviorist Origins of Soft Torture.'' \emph{Limn} 1, no. 1
(2011). \url{https://escholarship.org/uc/item/2rw899hv}.

Levins, Richard. ``Dialectics and Systems Theory.'' \emph{Science \&
Society} 62, no. 3 (1998): 375--99.

Li, Biao 李彪. ``新闻传播学研究方法的构造'' {[}Research methods of
journalism-communication{]}. \emph{国际新闻界} {[}Chinese journal of
journalism and communication{]}, no. 1 (2008).

Li, Liangrong 李良荣. \emph{新闻学概论} {[}Introduction to
journalism{]}. Shanghai: Fudan University Press 复旦大学出版社, 2018.

Lin, Chunfeng, and John Nerone. ``The `Great Uncle of Dissemination':
Wilbur Schramm and Communication Study in China.'' In \emph{The
International History of Communication Study}, edited by Peter Simonson
and David W. Park, 414--33. Routledge, 2015.


Liu, Hailong 刘海龙.
"`传播学'引进中的`失踪者':从1978年---1989年批判学派的引介看中国早期的传播学观念"
{[}The `missing persons' in the import of communication{]}.
\emph{新闻与传播研究} {[}Journalism and communication{]}, no. 4 (2007):
29--35.

\end{CJK*}
\begin{CJK*}{UTF8}{bsmi}

Liu, Shengji 劉勝驥.
``國科會專題研究:中國大陸高校教科書中政治思想教育之研究'' {[}Political
ideologies in mainland Chinese college textbooks{]}, 2004.
http://nccur.lib.nccu.edu.tw/handle/140.119/5239.

\end{CJK*}

\begin{CJK*}{UTF8}{gbsn}

Liu, Xiao. \emph{Information Fantasies: Precarious Mediation in
Postsocialist China}. Minneapolis: University of Minnesota Press, 2019.

Liu, Xin. \emph{The Mirage of China: Anti-Humanism, Narcissism, and
Corporeality of the Contemporary World}. New York: Berghahn Books, 2012.

McAnany, Emile G. ``Wilbur Schramm, 1907--1987: Roots of the Past, Seeds
of the Present.'' \emph{Journal of Communication} 38, no. 4 (1988):
109--22.

Medina, Eden. \emph{Cybernetic Revolutionaries: Technology and Politics
in Allende's Chile}. Cambridge: MIT Press, 2014.

Miller, H. Lyman. \emph{Science and Dissent in Post-Mao China: The
Politics of Knowledge}. Seattle: University of Washington Press, 1996.

Pang, Laikwan. ``Dialectical Materialism.'' In \emph{Afterlives of
Chinese Communism: Political Concepts from Mao to Xi}, edited by
Christian Sorace, Ivan Franceschini, and Nicholas Loubere, 67--72.
London: Verso Books, 2019.

Peng, Yongdong 彭永东. \emph{控制论的发生与传播研究} {[}The origin and
dissemination of cybernetics{]}. Taiyuan: Shanxi Education Press
山西教育出版社, 2012.

Peters, Benjamin. \emph{How Not to Network a Nation: The Uneasy History
of the Soviet Internet}. Cambridge: MIT Press, 2016.

Peters, John Durham. ``Institutional Sources of Intellectual Poverty in
Communication Research.'' \emph{Communication Research} 13, no. 4
(1986): 527--59.

Pickering, Andrew. \emph{The Cybernetic Brain: Sketches of Another
Future}. Chicago: University of Chicago Press, 2010.

Qian, Xuesen 钱学森, Guozhi 许国志Xu, and Shouyun王寿云Wang.
"组织管理的技术---系统工程" {[}Techniques of organizational management:
systems engineering{]}. \emph{文汇报} {[}Wenhui{]}, September 27, 1978.

Qian, Xuesen 钱学森. ``我对系统学的认识历程'' {[}How I came to
systematics{]}. In \emph{系统科学进展} {[}Developments in systems
science{]}, edited by Guo Lei郭雷, 2--17. Beijing: Science Publishing
科学出版社, 2017.

Qian, Xuesen 钱学森, and Jian 宋健 Song. \emph{工程控制论}
{[}Engineering cybernetics{]}. Beijing: Science Publishing 科学出版社,
1980.

Rajagopal, Arvind. ``Communicationism: Cold War Humanism.''
\emph{Critical Inquiry} 46, no. 2 (2020): 353--80.

Rau, Erik P. ``The Adoption of Operations Research in the United States
during World War II.'' In \emph{Systems, Experts, and Computers}, edited
by Agatha C. Hughes and Thomas P. Hughes, 57--92. Cambridge: MIT Press,
2011.

Rindzeviciute, Egle. \emph{The Power of Systems: How Policy Sciences
Opened Up the Cold War World}. Ithaca, NY: Cornell University Press,
2016.

Rofel, Lisa. \emph{Desiring China : Experiments in Neoliberalism,
Sexuality, and Public Culture}. Durham: Duke University Press, 2007.

Rogers, Everett M. \emph{History of Communication Study}. New York: Free
Press, 1997.

Rogers, Everett M., and Thomas W. Valente. ``A History of Information
Theory in Communication Research.'' In \emph{Between Communication and
Information}, edited by Brent D. Ruben and Jorge R. Schement, 35--56.
New York: Routledge, 2017.

Ruben, Brent D. \emph{Between Communication and Information}. Milton,
United Kingdom: Routledge, 1993.

Schmalzer, Sigrid. \emph{Red Revolution, Green Revolution}. Chicago:
University of Chicago Press, 2016.

Schmalzer, Sigrid. \emph{The People's Peking Man: Popular Science and
Human Identity in Twentieth-Century China}. Chicago: University of
Chicago Press, 2009.

Schramm, Wilbur. ``Human Communication as a Field of Behavioral
Science.'' In \emph{Human Communication as a Field of Study}, edited by
Sarah Sanderson King, 13--26. Albany, NY: SUNY Press, 1989.

Schramm, Wilbur. ``Information Theory and Mass Communication.''
\emph{Journalism Quarterly} 32, no. 2 (1955): 131--46.

Schramm, Wilbur. ``The Unique Perspective of Communication: A
Retrospective View.'' \emph{Journal of Communication} 33, no. 3 (1983):
6--17.

Schramm, Wilbur. ``传学与新闻及其它'' {[}Communication study and
journalism, and other issues{]}. \emph{新闻学会通讯} {[}Society of
journalism letters{]}, no. 14 (1982): 19--22.

Schramm, Wilbur. ``美国"大众传播学"的四个奠基人'' {[}Four forefathers of
American mass communication{]}. \emph{国际新闻界} {[}Chinese journal of
journalism and communication{]}, no. 2 (1982): 2--4.

Schramm, Wilbur, Godwin C. Chu, and Frederick T. C. Yu. ``China's
Experience with Development Communication: How Transferable Is That?''
In \emph{Communication and Development in China}, edited by Goodwin Chu,
Fred Huang, Wilbur Schramm, Stephen Uhalley Jr, and Frederick T. C. Yu,
85--105. Honolulu: East-West Center, 1975.

Schudson, Michael. \emph{Discovering The News: A Social History Of
American Newspapers}. New York: Basic Books, 1981.

Shannon, Claude Elwood, and Warren Weaver. \emph{The Mathematical Theory
of Communication}. Champaign, IL: University of Illinois Press, 1949.

Shih, Shu-Mei. ``Is the Post in Postsocialism the Post in
Posthumanism?'' \emph{Social Text} 30, no. 1 (2012): 27--50.

Siebert, Fred, Theodore Peterson, and Wilbur Schramm. \emph{Four
Theories of the Press: The Authoritarian, Libertarian, Social
Responsibility, and Soviet Communist Concepts of What the Press Should
Be and Do}. Champaign, IL: University of Illinois Press, 1956.

Song, Jian 宋健. ``工程控制论`` {[}Engineering cybernetics{]}.
\emph{系统工程理论与实践} {[}Systems engineering theory \& practice{]},
no. 2 (1985): 2--4.

Stalin, Joseph. \emph{Dialectical and Historical Materialism}. Moscow:
Foreign Languages Publishing House, 1938.

Stockmann, Daniela. \emph{Media Commercialization and Authoritarian Rule
in China}. Cambridge: Cambridge University Press, 2012.

Tankard, James W., Jr. ``The Theorists.'' In \emph{Makers of the Media
Mind: Journalism Educators and Their Ideas}, edited by William David
Sloan, 227--86. New York: Routledge, 1990.

Taylor, Angus M. ``Dialectics of Nature or Dialectics of Practice?''
\emph{Dialectical Anthropology} 4, no. 4 (1979): 289--308.

Thomas, Paul. \emph{Marxism \& Scientific Socialism: From Engels to
Althusser}. Milton Park, United Kingdom: Taylor \& Francis, 2008.

Tsien, H. S. \emph{Engineering Cybernetics}. New York: McGraw-Hill,
1954.

Tunstall, Jeremy. ``The Trouble with U.S. Communication Research.''
\emph{Journal of Communication} 33, no. 3 (1983): 92--95.

Turner, Fred. \emph{From Counterculture to Cyberculture: Stewart Brand,
the Whole Earth Network, and the Rise of Digital Utopianism}. Chicago:
University of Chicago Press, 2008.

Wang, Hui. \emph{China's New Order: Society, Politics, and Economy in
Transition}. Cambridge: Harvard University Press, 2003.

Wang, Ning. ``The Making of an Intellectual Hero: Chinese Narratives of
Qian Xuesen.'' \emph{China Quarterly}, no. 206 (2011): 352--71.

Wang, Yihong 王怡红, and Yiqing 胡翼青 Hu, eds. \emph{中国传播学30年}
{[}Chinese communication: thirty years{]}. Beijing: Chinese Encyclopedia
Press 中国大百科全书出版社, 2010.

Wang, Yimin 王益民. \emph{系统理论新闻学} {[}Systems theory of
journalism{]}. Wuhan: Huazhong University of Science and Technology
华中理工大学出版社, 1989.

Wiener, Norbert. \emph{Cybernetics or Control and Communication in the
Animal}. Cambridge: MIT Press, 1965.

Wu, Angela Xiao, and Luzhou Li. ``Localism in Internet Governance: The
Rise of China's Provincial Web.'' \emph{China Information}, 2021.
\url{https://doi.org/10.1177/0920203X211055038}.

Wu, Guoguang. ``One Head, Many Mouths: Diversifying Press Structures in
Reform China.'' In \emph{Power, Money, and Media: Communication Patterns
and Bureaucratic Control in Cultural China}, 45­--67. Evanston, IL:
Northwestern University Press, 2000.

Wu, Wenhu 吴文虎. "传播学理论架构初探" {[}Exploring communication's
theoretical frameworks{]}. \emph{新闻学刊} {[}Journalism{]}, no. 5
(1986): 27--31.

Xiaoling 晓凌. ``他们精心治学`` {[}They are devoted to scholarships{]}.
\emph{新闻学会通讯} {[}Society of journalism letters{]}, no. 14 (1982):
22--23.

Xu, Yaokui 徐耀魁.
``施拉姆对中国传播学研究的影响:纪念施拉姆来新闻研究所座谈30周年''
{[}Schramm's impacts on Chinese communication research{]}.
\emph{新闻与传播研究} {[}Journalism and communication{]}, no. 4 (2012):
9--14.

Xu, Yaokui 徐耀魁, Peixu 孙培旭 Sun, Rugang 沈如纲 Shen, and Hongliang
曹宏亮 Cao. ``关于新闻体制改革的设想'' {[}Envisioning journalism
reform{]}. \emph{新闻学刊} {[}Journalism{]}, no. 5 (1986): 3--9.

Ye, Yonglie 叶永烈. \emph{走进钱学森} {[}Qian Xuesen: A biography{]}.
Shanghai: Shanghai Jiao Tong University Press上海交通大学出版社, 2016.

Yu, Guoming 喻国明. ``新闻作品信息含量问题初探`` {[}Exploring news
information content{]}. \emph{新闻学论集} {[}Annals of journalism
studies{]} 8 (1984): 80--99.

Yu, Guoming 喻国明. "试论建立具有中国特色的社会主义新闻事业体制"
{[}Building Chinese journalism with socialist characteristics{]}.
\emph{新闻学刊} {[}Journalism{]}, no. 1 (1986): 42--52.

Yu, Timothy 余也鲁, Qingbin 施清彬 Shi, Yufang 崔煜芳 Cui, and Qiqi
章琪琦 Zhang. ``中国传播学研究破冰之旅的回顾:余也鲁教授访问记``
{[}Chinese communication study's icebreaking trip in retrospect{]}.
\emph{新闻与传播研究} {[}Journalism and communication{]}, no. 4 (2012):
4--9.

Yurchak, Alexei. ``Communist Proteins: Lenin's Skin, Astrobiology, and
the Origin of Life.'' \emph{Kritika: Explorations in Russian and
Eurasian History} 20, no. 4 (2019): 683--715.

Zhang, Yong. ``From Masses to Audience: Changing Media Ideologies and
Practices in Reform China.'' \emph{Journalism Studies} 1, no. 4 (2000):
617--35.

Zhao, Yuezhi. ``From Commercialization to Conglomeration: The
Transformation of the Chinese Press Within the Orbit of the Party
State.'' \emph{Journal of Communication} 50, no. 2 (2000): 3--26.

Zhao, Yuezhi. ``批判研究与实证研究的对比分析'' {[}Comparing critical and
empirical approaches{]}. \emph{国际新闻界} {[}Chinese journal of
journalism and communication{]}, no. 11 (2006): 34--39.

Zheng, Xingdong 郑兴东. ``新闻价值相关论'' {[}Thesis on news value{]}.
\emph{新闻学论集} {[}Annals of journalism studies{]} 11 (1987): 88--99.

\enlargethispage{\baselineskip}

Zhou, Zhi 周致. ``西方传播学的产生及其与新闻学的关系'' {[}Western
communication's origin and relations with journalism study{]}. In
\emph{传播学简介} {[}Introduction to communication study{]}, edited by
社科院新闻所世界室 CASS Journalism Institute, 126--34. Beijing: People's
Daily Press 人民日报出报社, 1983.

\end{CJK*}

\end{hangparas}
\vspace{2em}

\hypertarget{acknowledgments}{%
\section{Acknowledgments}\label{acknowledgments}}

At various stages, this paper benefited from the insights of Lily
Chumley, Timothy Cheek, Finn Brunton, Sigrid Schmalzer, Joseph Man Chan,
Julie Yujie Chen, Fred Turner, He Bian, Yige Dong, and Elizabeth
Lenaghan. I am also grateful to audiences at the conferences
``Post(?)socialist Horizons'' and ``Exclusions in the History and
Historiography of Communication Studies,'' and the SHOT/HSS panel
``Systems Thinking in Cold War East Asia and Beyond,'' especially Alexei
Yurchak, Masha Salazkina, David Park, Jefferson Pooley, Peter Simonson,
and Egle Rindzeviciute for their comments. This research was supported
by the Henry Luce Foundation and the American Council of Learned
Societies, the Chiang Ching-kuo Foundation, and the American Association
of University Women.

\end{document}