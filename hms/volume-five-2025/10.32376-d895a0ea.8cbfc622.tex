% see the original template for more detail about bibliography, tables, etc: https://www.overleaf.com/latex/templates/handout-design-inspired-by-edward-tufte/dtsbhhkvghzz

\documentclass{tufte-handout}

%\geometry{showframe}% for debugging purposes -- displays the margins

\usepackage{amsmath}

\usepackage{hyperref}

\usepackage{fancyhdr}

\usepackage{hanging}

\hypersetup{colorlinks=true,allcolors=[RGB]{97,15,11}}

\fancyfoot[L]{\emph{History of Media Studies}, vol. 5, 2025}


% Set up the images/graphics package
\usepackage{graphicx}
\setkeys{Gin}{width=\linewidth,totalheight=\textheight,keepaspectratio}
\graphicspath{{graphics/}}

\title[A Note on the Film]{A Note on the Film} % longtitle shouldn't be necessary

% The following package makes prettier tables.  We're all about the bling!
\usepackage{booktabs}

% The units package provides nice, non-stacked fractions and better spacing
% for units.
\usepackage{units}

% The fancyvrb package lets us customize the formatting of verbatim
% environments.  We use a slightly smaller font.
\usepackage{fancyvrb}
\fvset{fontsize=\normalsize}

% Small sections of multiple columns
\usepackage{multicol}

% Provides paragraphs of dummy text
\usepackage{lipsum}

% These commands are used to pretty-print LaTeX commands
\newcommand{\doccmd}[1]{\texttt{\textbackslash#1}}% command name -- adds backslash automatically
\newcommand{\docopt}[1]{\ensuremath{\langle}\textrm{\textit{#1}}\ensuremath{\rangle}}% optional command argument
\newcommand{\docarg}[1]{\textrm{\textit{#1}}}% (required) command argument
\newenvironment{docspec}{\begin{quote}\noindent}{\end{quote}}% command specification environment
\newcommand{\docenv}[1]{\textsf{#1}}% environment name
\newcommand{\docpkg}[1]{\texttt{#1}}% package name
\newcommand{\doccls}[1]{\texttt{#1}}% document class name
\newcommand{\docclsopt}[1]{\texttt{#1}}% document class option name


\begin{document}

\begin{titlepage}

\begin{fullwidth}
\noindent\LARGE\emph{Up from the Stacks
} \hspace{73mm}\includegraphics[height=1cm]{logo3.png}\\
\noindent\hrulefill\\
\vspace*{1em}
\noindent{\Huge{A Note on the Film\par}}

\vspace*{1.5em}

\noindent\LARGE{Susanne K. Langer (1895–1985)}\par\marginnote{\emph{Susanne K. Langer, ``A Note on the Film,'' \emph{History of Media Studies} 5 (2025), original publication date: 1953, \href{https://doi.org/10.32376/d895a0ea.8cbfc622}{https://doi.org/ 10.32376/d895a0ea.8cbfc622}.} \vspace*{0.75em}}\marginnote{\href{https://creativecommons.org/licenses/by-nc/4.0/}{\includegraphics[height=0.5cm]{by-nc.png}}}
\vspace*{0.5em}

\end{fullwidth}

\par{\noindent\small{Republication of Susanne K. Langer,  "A Note on the Film," in \emph{Feeling and Form: A Theory of Art}, 411–5. New York: Charles Scribner’s Sons, 1953.}}


% \vspace*{0.75em} % second author

% \noindent{\LARGE{<<author 2 name>>}\par}
% \vspace*{0.5em}
% \noindent{{\large\emph{<<author 2 affiliation>>}, \href{mailto:<<author 2 email>>}{<<author 2 email>>}\par}}

% \vspace*{0.75em} % third author

% \noindent{\LARGE{<<author 3 name>>}\par}
% \vspace*{0.5em}
% \noindent{{\large\emph{<<author 3 affiliation>>}, \href{mailto:<<author 3 email>>}{<<author 3 email>>}\par}}



\vspace*{1em}
\enlargethispage{\baselineskip}

\newthought{Here is a new art}. For a few decades it seemed like noth­ing more than a
new technical device in the sphere of drama, a new way of preserving and
retailing dramatic performances. But today its devel­opment has already
belied this assumption. The screen is not a stage, and what is created
in the conception and realization of a film is not a play. It is too
early to systematize any theory of this new art, but even in its present
pristine state it exhibits---quite beyond any doubt, I think---not only
a new technique, but a new poetic mode.

Much of the material for the following reflections was collected by four
of my former seminar students,\footnote{Messrs. Joseph-Pattison, Louis
  Forsdale, William Hoth, and Mrs.~Virginia E. Allen. Mr.~Hoth is now
  Instructor in English at Cortland (New York) State Teachers College;
  the other three are members of the Columbia Teachers College staff.}
at Columbia Teachers College, who have kindly permitted me to use their
findings. I am likewise indebted to Mr.~Robert W. Sowers, who (also as a
member of that seminar) made a study of photography that provided at
least one valuable idea, namely that photographs, no matter how posed,
cut, or touched up, must \emph{seem factual}, or as he called it,
``authentic.'' I shall return later to that sug­gestion.

The significant points, for my purposes, that were demonstrated by the
four collaborating members were (1) that the structure of a motion
picture is not that of drama, and indeed lies closer to narrative than
to drama; and (2) that its artistic potentialities became evident only
when the moving camera was introduced.

The moving camera divorced the screen from the stage. The
straight­forward photographing of stage action, formerly viewed as the
only artistic possibility of the film, henceforth appeared as a special
technique. The screen actor is not governed by the stage, nor by the
conventions of the theater, he has his own realm and conventions;
indeed, there may be no ``actor'' at all. The documentary film is a
pregnant invention. The cartoon does not even involve persons merely
``behaving.''

\enlargethispage{2\baselineskip}

\vspace*{4em}

\noindent{\emph{History of Media Studies}, vol. 5, 2025}


 \end{titlepage}

% \vspace*{2em} | to use if abstract spills over

The fact that the moving picture could develop to a fairly high degree
as a silent art, in which speech had to be reduced and concen­trated into
brief, well-spaced captions, was another indication that it was not
simply drama. It used pantomime, and the first aestheticians of the film
considered it as essentially pantomime. But it is not pantomime; it
swallowed that ancient popular art as it swallowed the photograph.

The fact that the moving picture could develop to a fairly high degree
as a silent art, in which speech had to be reduced and concen­trated into
brief, well-spaced captions, was another indication that it was not
simply drama. It used pantomime, and the first aestheticians of the film
considered it as essentially pantomime. But it is not pantomime; it
swallowed that ancient popular art as it swallowed the photograph.

One of the most striking characteristics of this new art is that it
seems to be omnivorous, able to assimilate the most diverse materials
and turn them into elements of its own. With every new
invention---mon­tage, the sound track, Technicolor---its devotees have
raised a cry of fear that now its ``art'' must be lost. Since every such
novelty is, of course, promptly exploited before it is even technically
perfected, and flaunted in its rawest state, as a popular sensation, in
the flood of meaningless compositions that steadily supplies the show
business, there is usually a tidal wave of particularly bad rubbish in
association with every im­portant advance. But the art goes on. It
swallows everything: dancing, skating, drama, panorama, cartooning,
music (it almost always requires music).

Therewithal it remains a poetic art. But it is not any poetic art we
have known before; it makes the primary illusion---virtual history---in
its own mode.

This is, essentially, \emph{the dream mode}. I do not mean that it
copies dream, or puts one into a daydream. Not at all; no more than
literature invokes memory, or makes us believe that \emph{we} are
remembering. An art mode is \emph{a} \emph{mode of appearance}. Fiction
is ``like'' memory in that it is projected to compose a finished
experiential form, a ``past''---not the reader's past, nor the writer's,
though the latter may make a claim to it (that, as well as the use of
actual memory as a model, is a literary device). Drama is ``like''
action in being causal, creating a total immi­nent experience, a personal
``future'' or Destiny. Cinema is ``like'' dream in the mode of its
presentation: it creates a virtual present, an order of direct
apparition. That is the mode of dream.

The most noteworthy formal characteristic of dream is that the dreamer
is always at the center of it. Places shift, persons act and speak, or
change or fade---facts emerge, situations grow, objects come into view
with strange importance, ordinary things infinitely valuable or
horrible, and they may be superseded by others that are related to them
essen­tially by feeling, not by natural proximity. But the dreamer is
always ``there,'' his relation is, so to speak, equidistant from all
events. Things may occur around him or unroll before his eyes; he may
act or want to act, or suffer or contemplate; but the \emph{immediacy}
of everything in a dream is the same for him.

This aesthetic peculiarity, this relation to things perceived,
char­acterizes the \emph{dream mode}: it is this that the moving picture
takes over, and whereby it creates a virtual present. In its relation to
the images, actions, events, that constitute the story, the camera is in
the place of the dreamer.

But the camera \emph{is} not a dreamer. We are usually agents in a
dream. The camera (and its complement, the sound track) is not itself in
the picture. It is the mind's eye and nothing more. Neither is the
picture (if it is art) likely to be dreamlike in its structure. It is a
poetic com­position, coherent, organic, governed by a definitely
conceived feeling, not dictated by actual emotional pressures.

The basic abstraction whereby virtual history is created in the dream
mode is immediacy of experience, ``givenness,'' or as Mr.~Sowers calls
it, ``authenticity.'' This is what the art of the film abstracts from
actu­ality, from our actual dreaming.

The percipient of a moving picture sees with the camera; his stand­point
moves with it, his mind is pervasively present. The camera is his eye
(as the microphone is his ear---and there is no reason why a mind's eye
and a mind's ear must always stay together). \emph{He takes the place of
the dreamer}, but in a perfectly objectified dream---that is, he is not
in the story. The work is the appearance of a dream, a unified,
continu­ously passing, significant \emph{apparition}.

Conceived in this way, a good moving picture is a work of art by all the
standards that apply to art as such. Sergei Eisenstein speaks of good
and bad films as, respectively, ``vital'' and ``lifeless''\footnote{\emph{The
  Film Sense}, p.~17.}; speaks of photographic shots as
``elements,''\footnote{\emph{Ibid.}, p.~4.} which combine into
``images,'' which are ``objectively unpresentable'' (I would call them
poetic impressions), but are greater elements compounded of
``representations,'' whether by mon­tage or symbolic acting or any other
means.\footnote{\emph{Ibid.}, p.~8.} The whole is governed by the
``initial general image which originally hovered before the creative
artist''\footnote{\emph{Ibid.}, p 31.}---the matrix, the commanding
form; and it is this (not, be it remarked, the artist's emotion) that is
to be evoked in the mind of the spectator.

Yet Eisenstein believed that the beholder of a film was somewhat
specially called on to use his imagination, to create his own experience
of the story.\footnote{\emph{Ibid.}, p.~33: ``... the spectator is drawn
  into a creative act in which his individuality is not subordinated to
  the author's individuality, but is opened} Here we have, I think, an
indication of the powerful illu­sion the film makes not of things going
on, but of the dimension in which they go on---a \emph{virtual} creative
imagination; for it \emph{seems} one's own\marginnote{up throughout the process of
  fusion with the author's intention, just as the individuality of a
  great actor is fused with the individuality of a great playwright in
  the creation of a classic scenic image. ln fact, every spectator
  \ldots{} creates an image in accordance with the representational
  guidance, suggested by the author, leading him to understanding and
  experience of the author's theme. This is the same image that was
  planned and created by the author, but this image is at the same time
  created also by the spectator himself.''} crea­tion, direct visionary
experience, a ``dreamt reality.'' Like most artists, he took the virtual
experience for the most obvious fact.\footnote{Compare the statement in
  Ernest Lindgren's \emph{The Art of the Film}, p.~92, apropos of the
  moving camera: ``It is the spectator's own mind that moves.''}

The fact that a motion picture is not a plastic work but a poetic
presentation accounts for its power to assimilate the most diverse
ma­terials, and transform them into non-pictorial elements. Like dream,
it enthralls and commingles all senses; its basic abstraction---direct
appari­tion---is made not only by visual means, though these are
paramount, but by words, which punctuate vision, and music that supports
the unity of its shifting ``world.'' It needs many, often convergent,
means to create the continuity of emotion which holds it together while
its visions roam through space and time.

It is noteworthy that Eisenstein draws his materials for discussion from
epic rather than dramatic poetry; from Pushkin rather than Chekhov,
Milton rather than Shakespeare. That brings us back to the point noted
by my seminar students, that the novel lends itself more readily to
screen dramatization than the drama. The fact is, I think, that a story
narrated does not require as much ``breaking down'' to become screen
apparition, because it has no framework itself of fixed \emph{space}, as
the stage has; and one of the aesthetic peculiarities of dream, which
the moving picture takes over, is the nature of its space. Dream events
are spatial---often intensely concerned with space-intervals, endless
roads, bottomless canyons, things too high, too near, too far---but they
are not oriented in any total space. The same is true of the moving
picture, and distinguishes it---despite its visual character---from
plastic art: \emph{its space comes and goes}. It is always a secondary
illusion.

The fact that the film is somehow related to dream, and is in fact in a
similar mode, has been remarked by several people, sometimes for reasons
artistic, sometimes non-artistic. R. E. Jones noted its freedom not only
from spatial restriction, but from temporal as well. ``Motion
pic­tures,'' he said, ``are our thoughts made visible and audible. They
flow in a swift succession of images, precisely as our thoughts do, and
their speed, with their flashbacks---like sudden uprushes of
memory---and their abrupt transition from one subject to another,
approximates very closely the speed of our thinking. They have the
rhythm of the thought-stream and the same uncanny ability to move
forward or backward in space or time\ldots. They project pure thought,
pure dream, pure inner life.''\footnote{\emph{The Dramatic Imagination},
  pp.~17--18.}

The ``dreamed reality'' on the screen can move forward and backward
because it is really an eternal and ubiquitous virtual present. The
action of drama goes inexorably forward because it creates a future, a
Destiny; the dream mode is an endless Now.




\section{Bibliography}\label{bibliography}

\begin{hangparas}{.25in}{1} 



EISENSTEIN, SERGEI M., \emph{The Film Sense}. Translated and edited by
Jay Leyda. New York: Harcourt, Brace \& Co., 1942.

JONES, R. E., \emph{The Dramatic Imagination: Reflections and
Speculations on the Art of the Theatre}. New York: Duell, Sloan \&
Pearce, 1941.

LINDGREN, ERNEST, \emph{The Art of the Film}. London: Allen \& Unwin,
1948.



\end{hangparas}


\end{document}