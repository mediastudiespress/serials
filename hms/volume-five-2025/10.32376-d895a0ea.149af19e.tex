% see the original template for more detail about bibliography, tables, etc: https://www.overleaf.com/latex/templates/handout-design-inspired-by-edward-tufte/dtsbhhkvghzz

\documentclass{tufte-handout}

%\geometry{showframe}% for debugging purposes -- displays the margins

\usepackage{amsmath}

\usepackage{hyperref}

\usepackage{fancyhdr}

\usepackage{hanging}

\hypersetup{colorlinks=true,allcolors=[RGB]{97,15,11}}

\fancyfoot[L]{\emph{History of Media Studies}, vol. 5, 2025}


% Set up the images/graphics package
\usepackage{graphicx}
\setkeys{Gin}{width=\linewidth,totalheight=\textheight,keepaspectratio}
\graphicspath{{graphics/}}

\title[Susanne Langer's Film Theory]{Susanne Langer's Film Theory: Elaboration and Implications} % longtitle shouldn't be necessary

% The following package makes prettier tables.  We're all about the bling!
\usepackage{booktabs}

% The units package provides nice, non-stacked fractions and better spacing
% for units.
\usepackage{units}

% The fancyvrb package lets us customize the formatting of verbatim
% environments.  We use a slightly smaller font.
\usepackage{fancyvrb}
\fvset{fontsize=\normalsize}

% Small sections of multiple columns
\usepackage{multicol}

% Provides paragraphs of dummy text
\usepackage{lipsum}

% These commands are used to pretty-print LaTeX commands
\newcommand{\doccmd}[1]{\texttt{\textbackslash#1}}% command name -- adds backslash automatically
\newcommand{\docopt}[1]{\ensuremath{\langle}\textrm{\textit{#1}}\ensuremath{\rangle}}% optional command argument
\newcommand{\docarg}[1]{\textrm{\textit{#1}}}% (required) command argument
\newenvironment{docspec}{\begin{quote}\noindent}{\end{quote}}% command specification environment
\newcommand{\docenv}[1]{\textsf{#1}}% environment name
\newcommand{\docpkg}[1]{\texttt{#1}}% package name
\newcommand{\doccls}[1]{\texttt{#1}}% document class name
\newcommand{\docclsopt}[1]{\texttt{#1}}% document class option name


\begin{document}

\begin{titlepage}

\begin{fullwidth}
\noindent\LARGE\emph{Up from the Stacks
} \hspace{73mm}\includegraphics[height=1cm]{logo3.png}\\
\noindent\hrulefill\\
\vspace*{1em}
\noindent{\Huge{Susanne Langer's Film Theory: Elaboration and Implications\par}}

\vspace*{1.5em}

\noindent\LARGE{Courtenay Wyche Beinhorn}\par\marginnote{\emph{Courtenay Wyche Beinhorn, ``Susanne Langer's Film Theory: Elaboration and Implications,'' \emph{History of Media Studies} 5 (2025), original publication date: 1974, \href{https://doi.org/10.32376/d895a0ea.149af19e}{https://doi.org/ 10.32376/d895a0ea.149af19e}.} \vspace*{0.75em}}
\vspace*{0.5em}\marginnote{\href{https://creativecommons.org/licenses/by-nc/4.0/}{\includegraphics[height=0.5cm]{by-nc.png}}}

% \vspace*{0.75em} % second author

% \noindent{\LARGE{<<author 2 name>>}\par}
% \vspace*{0.5em}
% \noindent{{\large\emph{<<author 2 affiliation>>}, \href{mailto:<<author 2 email>>}{<<author 2 email>>}\par}}

% \vspace*{0.75em} % third author

% \noindent{\LARGE{<<author 3 name>>}\par}
% \vspace*{0.5em}
% \noindent{{\large\emph{<<author 3 affiliation>>}, \href{mailto:<<author 3 email>>}{<<author 3 email>>}\par}}

\end{fullwidth}



\small\noindent{Republication of Courtenay Wyche Beinhorn, ``Susanne
Langer's Film Theory: Elaboration and Implications,''
\emph{Cinema Journal} 13, no. 2
(1974): 41--54. \url{https://doi.org/10.2307/1225250.}
Republished with permission.}



\vspace*{1em}


\newthought{In her essay,} ``A Note On The Film,'' Susanne K. Langer develops a
thought-provoking theory of film as art which, perhaps because of its
brevity, has received too little attention from film scholars and
theorists.\footnote{Susanne K. Langer, \emph{Feeling and Form} (New
  York: Charles Scribners Sons, 1953), p. 411.} Draw­ing upon the analogy
of film to dream, Langer extracts from dream the quality of ``the
on-going present'' and proceeds to show how this quality is transformed
into the ``virtual present'' of filmic art. Langer's theory of film is
particularly significant when considered in relation to her general
philos­ophy of art: together they provide not only a framework within
which film may be compared with the other arts, but also a set of
standards by which one may evaluate the merits of individual films.

Although she never explicitly states that still photography is not an
art, the bias is implicit in the occasional comparison she makes between
the photograph and other art forms. In a painting, for example, the
artist exer­cises his creative ability to make visible all the entities
which in real life are known by touch, movement, and memory, rather than
through sight alone.\footnote{\emph{Ibid.,} p. 73}

\begin{quote}
The visual substitutes for the non-visible ingredients in space
experience make the great difference between photographic rendering and
creative rendering; the latter is necessarily a departure from direct
imitation, because it is a construction of color alone ... by all sorts
of devices in order to present at once, with complete authority, the
primary illusion of a perfectly visible and perfectly intelligible total
space.\footnote{\emph{Ibid.}}
\end{quote}

\noindent Langer subscribes to Hildebrand's idea that pictorial art is created
through an ``architectonic'' process---that is, space is ``built up'' by
the application of paint to canvas.\footnote{\emph{Ibid.,} pp. 72--73.}
This gives the artist a great deal of creative flexibility. A painted
apple, for example, need look nothing like a real apple; moreover, its
relation to an orange may be completely imaginary. Presumably, since the
camera records an image directly onto film, this creative flexibility is
minimal. An apple will look like a real apple, and its relation to an
orange is a duplication of reality.

\enlargethispage{2\baselineskip}

\vspace*{2em}

\noindent{\emph{History of Media Studies}, vol. 5, 2025}


 \end{titlepage}

% \vspace*{2em} | to use if abstract spills over

If we accept the idea that visual art must be ``architectonic,'' then
Lan­ger's apparent bias against photography seems to be justified. It is
interest­ing, however, that her broad definition of art can be used to
defend the medium. We could argue, with some justification, that a
photograph is an illusion, and therefore an abstract or symbolic form. A
photograph of a house is not an actual house; it presents the semblance
of a house.

Furthermore, we could argue that photographic space is not real space,
but ``virtual space.'' For instance, Brett Weston's ``Garapata Beach''
pre­sents the semblance of planes of vision, or virtual space.\footnote{See
  \emph{The Print} (New York: Time-Life Books, 1970), p. 121.} The
interplay of light and shadow creates a series of visual planes which
extends from the jagged rocks in the foreground to the mountains and sky
in the back­ ground. This is not the real space in which we live and act,
but virtual space.

If a photograph can, in fact, be an abstraction or a symbolic form, then
how can it express human feeling? ``Tide-Battered Tree Trunk,'' by Minor
White, is not a simple record of an uprooted tree on a beach.\footnote{See
  \emph{mirrors messages manifestations minor white}, An Aperture
  Monograph, 1969, p. 188.} The trunk rushes downward with the force of
the waves which beat upon and smoothed it. What it expresses is the
almost indescribable feeling of being swept up in a power greater than
oneself. Isn't this a pattern of sentience? Even a strictly documentary
photograph like ``Migrant Mother,'' by Doro­thea Lange, has a timeless
quality that is both particular and universal.\footnote{See Edward
  Steichen, ed., \emph{The Bitter Years: 1935--1941} (New York: Museum
  of Modem Art, 1962).} In the gaunt face and squinted eyes, there is
concern, bewilderment, weari­ness, strength, and an infinite number of
more indefinable feelings. Jerry Uelsmann employs a technique (which
might be called architectonic) to create gestalt-like images which evoke
inexplicable primordial associations.\footnote{See \emph{Jerry N.
  Uelsmann}, An Aperture Monograph, 1970, passim.} Langer herself says
that one of the distinguishing qualities of art is its ability to
express the total range of emotive life in ways that are not al­ways
verbally expressible.

The ability of the photograph to transcend direct imitation has been
ex­plained by Minor White, whose photographs, although firmly grounded in
reality, have been compared to ``inner landscapes.''

\begin{quote}
Camera is always pointed at a subject, always. Occasionally one of the
camera's photographs points away from the subject toward the mind, or
the imagination.\footnote{\emph{Mirrors,} p. 108.}
\end{quote}

\noindent If, however, as Langer seems to imply, photography is direct imitation,
then how can cinema, which uses the photographic image, be an art? The
answer, in Langer's view, is that cinema \emph{moves}.\footnote{Langer,
  \emph{Feeling and Form,} pp. 411--12.} The space in a still pho­tograph
is fixed, and generally the spatial relationships between the ob­jects
therein cannot be altered. In cinema, however, the camera can pan, tilt,
dolly, zoom, focus-through, and move freely when attached to a crane. In
this sense, the moving camera can manipulate the spatial relationships
between the objects in its field of view. Cinema, then, transcends
direct imitation, because the moving camera allows the artist to
exercise his crea­tive flexibility and imagination.

Langer notes that some of the earliest films directly recorded stage
plays. At this point, she feels, film had not yet become an art.

\begin{quote}
For a few decades it seemed like nothing more than a new technical
device in the sphere of drama, a new way of preserving and retailing
dramatic performances ... The moving camera divorced the screen from the
stage. The straightforward photographing of stage action, formerly
viewed as the only artistic possibility of the film, henceforth appeared
as a special tech­nique.\footnote{\emph{Ibid.}, p. 411.}
\end{quote}

\noindent We may conclude from this statement that, for Langer, recording the move­
ment of figures is just another form of imitation. The camera had to be
freed from its fixed position before film could become an art.

By implication, we can expand Langer's concept of movement to include
the movement created in the editing process. Space can also be
manipulated through the juxtaposition of close-ups with medium and long
shots. The spatial relationship between eyes, nose, and mouth in an
extreme close-up is radically altered when the subject next appears in a
long shot which shows his relationship to his total environment. This is
supported by Pudov­kin's explanation of filmic space.

\begin{quote}
By the conjunction of the separate shots, the director builds a filmic
space entirely his own. He unites and compresses the separate elements
that have recorded different points of real, actual space into one
filmic space.\footnote{Lewis Jacobs, \emph{The Movies as Medium} (New
  York: Farrar, Strauss and Giroux, 1970), pp. 131--132.}
\end{quote}

\noindent This suggests an ``architectonic'' process which resembles Langer's
concept of the creation of pictorial art.

Furthermore, the editing process also manipulates space through the
juxtaposition of different locations. Whether connected by a dissolve,
fade, or simple cut, the movement from one location to another creates
an en­tirely new set of spatial relationships. In \emph{Intolerance,} for
example, Griffith cross-cuts between parallel action in four different
parts of the world at four different times. The manipulation of space
created by the juxtaposi­tion of separate spatial areas heightens the
dramatic impact of the film.

Although Langer never specifically mentions editorial movement, it is
significant that she has included Andre Malraux's essay, ``A Sketch for
the Psychology of the Moving Pictures,'' in her book \emph{Reflections
on Art.} Since she says that she agrees with the artistic concepts
explored in all the essays included therein,\footnote{Susanne K. Langer,
  ed., \emph{Reflections on Art} (Baltimore: Johns Hopkins Press, 1958),
  p. xi.} we may conclude that Malraux's theory of spatial ma­nipulation
in cinema reflects and amplifies Langer's own.

\begin{quote}
The birth of cinema as a means of expression (not reproduction) dates
from the abolition of that defined space; from the time when the cutter
thought of dividing his continuity into `planes' (close-up,
intermediate, remote, etc.) ... when the director took to bringing
forward the camera ... and moving it back; and above all, to replacing
the theatre set by an open field of vision corresponding to the area of
the screen ... The means of reproduction in cinema is the moving
photograph, but its means of expression is a sequence of
\emph{planes}.\footnote{\emph{Ibid.,} p. 320.}
\end{quote}

\noindent The manipulation of space afforded by editorial and camera movement
frees cinema to become a medium of expression.

Langer, however, does not place film in the same category as the plastic
arts. The reason, she says, is that, unlike painting or sculpture, the
images of actors and objects on film are not oriented in any
\emph{total} space.\footnote{Langer, \emph{Feeling and Form,} p. 415.}
Further­more, in Langer's view, virtual space is not the primary illusion
of cinema, because it is not a strictly visual medium. It incorporates
music, words, and dramatic action as well as visual images to create an
entirely different sort of illusion. Film is defined by Langer as a
\emph{poetic} art, and like poetry, its primary illusion is ``virtual
experience.''\footnote{\emph{Ibid.,} p. 412.}

As explained in \emph{Feeling and Form,} the primary illusion of poesis
is the creation of virtual history or experience: ``the semblance of
events lived and felt ... a piece of virtual life.''\footnote{\emph{Ibid.,}
  p. 212.} Cinema, Langer says, creates virtual history in its own
special mode---that of the dream.

\begin{quote}
Cinema is like dream in the mode of its presentation: it creates a
virtual present, an order of direct apparition. That is the mode of the
dream.\footnote{\emph{Ibid.,} p. 412.}
\end{quote}

\noindent Unlike the novel, which creates the illusion of a virtual past, and
drama which sets up a virtual future, film presents the semblance of a
virtual on­ going present. Thus, because its primary illusion is closer
to virtual experi­ence than virtual space, Langer defines film as a
poetic art.

In passing, it is interesting to note that there are other correlations
be­ tween film and poetry. In her analysis of Blake's ``The Tyger,''
Langer men­tions that the first words of a poem effect a ``break'' with
the reader's actual environment. This allows him to enter fully into the
poetic experience.\footnote{\emph{Ibid.,} p. 214.} This ``break'' is
comparable to the viewer's entrance into the movie theater. When the
lights dim and the film begins, the real world outside the theater is
forgotten. The images on the screen, which create the illusion of life,
are for a few hours more vivid or ``real'' than the
viewer\textquotesingle s actual life. Further­ more, cinema partakes of
one quality of lyric poetry. Because it is written in the present tense,
lyric poetry creates the illusion of subjective experi­ence. Film also
presents the illusion of an on-going present.

How do film and dream work to create the illusion of a virtual present?
In dream, the dreamer stands at the vortex of an ever-changing series of
images and events. Because he is the central percipient, all the
elements of the dream partake of the sense of immediate
experience.\footnote{\emph{Ibid.,} p. 413.}

\begin{quote}
Places shift, persons act and speak or change or fade---facts emerge,
situa­tions grow, objects come into view with strange importance,
ordinary things infinitely valuable or horrible, and they may be
superseded by others that are related to them essentially by feeling,
not natural proximity. But the dreamer is always `there', this relation
is, so to speak, equidistant from all events ... the immediacy of
everything in the dream is the same for him.\footnote{\emph{Ibid.}}
\end{quote}

\noindent Depending on our perception of our own dreams, we may quarrel with
Langer's assertion that the dreamer is at the center. Sometimes it seems
that we stand on the periphery of the dream and watch the unfolding of
the images and events therein. What Langer seems to mean, however, is
that the dream is perceived through the dreamer's own eyes, or the
``eye'' of his mind. There is no objective separate point of view. Nor
is there a sense of past or future in the dream. All the events seem to
be happening\emph{ ``}now.''

According to Langer, film abstracts from the dream this sense of
im­mediacy.\footnote{\emph{Ibid.}} There is no real past or future in the
film. It creates the illusion of an on-going present that ends only when
the house lights come up. In \emph{Hiroshima, Mon Amour,} for example,
the images of the actress' childhood are, for the viewer, as immediate
as the moments she spends with her lover in the hotel room.

Insofar as immediacy is concerned, Langer notes a parallel between
thought-time and film-time.\footnote{\emph{Ibid}., p. 415.} In thought,
as in film, our focus may shift from the present to the past and back
again; we may try to imagine the future as well. The difference between
thought and film time is that some memories and daydreams may be quite
vivid, while others may be only dimly remembered. In film, however, all
the images are equally immediate. As Langer notes in her reference to R.
E. Jones, film is like thought in that both may ``cut'' back and forth
in time, and from one image to another with equal ease.\footnote{\emph{Ibid.}}
The similarity lies in the ability to make quick transitions from one
subject to another.

The significant difference between film and dream is that dreams are
dictated by the dreamer's own emotional pressure, while film images are
not created by the viewer. They are created by the film maker, and the
viewer's visual and aural vantage point moves with the camera and the
manipulation of sound. The camera is his eye, the microphone is his ear.
Thus, as Langer says, film is an ``objectified'' dream, and the viewer
is, perhaps, a virtual dreamer.

Langer takes Eisenstein to task on this very point; she says that he con­
fuses the virtual experience of the spectator with actual
experience.\footnote{\emph{Ibid.}, p. 414.} In \emph{The Film Sense,}
Eisenstein states that the viewer's individuality is fused with the
author's intention, so that he may participate in the creative pro­cess.

\begin{quote}
In fact, every spectator ... creates an image in accordance with the
repre­sentational guidance suggested by the author, leading him to
understanding and experience of the author\textquotesingle s theme. This
is the same image that was plan­ned and created by the author, but this
image is at the same time created also by the spectator
himself.\footnote{Sergei Eisenstein, \emph{The Film Sense,} trans, and
  ed. by Jay Leyda (New York: Harcourt, Brace and Co., 1942), p. 33.}
\end{quote}

\noindent The creation of an ``image'' is the most basic principle of montage. It
stems, Eisenstein says, from the fact that ``two film pieces of any
kind, placed to­gether, inevitably combine into a new concept, a new
quality, arising out of that juxtaposition.''\footnote{\emph{Ibid.}, p.
  4.} Of the Kuleshov experiment, Eisenstein would say that with each
new juxtaposition of shots, the viewer created a new con­cept. Langer
seems to argue that because film images and the order of their
apparition do not spring from the viewer's own mind, the creation of a
concept is a virtual, rather than an actual, experience.

\begin{quote}
Here we have, I think, an indication of the powerful illusion the film
makes not of things going on, but of the dimension in which they go
on---a virtual creative imagination; for it \emph{seems} one's own
creation, direct visionary experi­ence, a dreamt reality.\footnote{Langer,
  \emph{Feeling and Form}, p. 414.}
\end{quote}

\noindent Langer's objection is not entirely justified. Even though the apparition
of images may lie in the realm of virtual imagination and the concepts
with which the spectator emerges are intended by the film maker, the
viewer \emph{does} make actual mental associations which, on a very
basic level, may be called creative.

\hypertarget{2}{%
\section{\textbf{2}}\label{2}}

In \emph{Sex, Psyche, Etcetera in the Film,} Parker Tyler attacks both
Langer and Kracauer on the grounds that they say that film should
duplicate, re­spectively, external reality and the dream world.

\begin{quote}
The point of our awkward, Janus-face pair of theories is that
\emph{both} assume that film's dominant function is reportorial: one
reports the physical aspect of life, the other reports a special mental
aspect---or if you will, seeks to duplicate its `mode'.\footnote{Parker
  Tyler, \emph{Sex, Psyche, Etcetera in the Film} (Horizon Press, 1969),
  p. 121.}

Her dream-mode film also implies the `open end' and `flow of life' as
neces­sary traits of a disorderly, uncontrolled world without true
climax, sustained rhythm, or film spatial orientation.\footnote{\emph{Ibid.},
  p. 122.}
\end{quote}

\noindent Tyler's objection stems from what is apparently a cursory reading of
Langer\textquotesingle s film essay and an unfamiliarity with her
principles of art. Langer explicitly states that film does not
\emph{copy} dream, as elsewhere she states that art is not
imitation.\footnote{Langer, \emph{Feeling and Form}, p. 412.}
Furthermore, a dream can no more be a work of art than a doodle on a
piece of paper, or a crying child, because it lacks the quality of
abstraction.

\begin{quote}
A dream is not a work of art ... because it is improvised for purely
self­ expressive ends, or for romantic satisfaction, and has to meet no
standards of coherence, organic form, or more than personal
interest.\footnote{\emph{Ibid.}, p. 168.}
\end{quote}

\noindent Film, then, does not duplicate the seemingly jumbled, impressionistic
qual­ity of the dream. Like the poet, the film maker organizes the images
into a coherent composition which is governed by a definite idea or
``matrix,'' not the actual emotional pressure which dictates a
dream.\footnote{\emph{Ibid.}, p. 168.} Film is in the dream \emph{mode}
because both present the semblance or primary illusion of the on-going
present. As she explains, an art mode is a mode of \emph{appearance.}
Film is \emph{like} dream because its images appear in the manner of
direct appari­tion.\footnote{\emph{Ibid.}, p. 412.}

The reference to Kracauer, however, is interesting. Langer and Kracauer
differ radically in their approach to film as a photographic medium.
Basi­cally, Kracauer believes that film is an \emph{outgrowth} of still
photography.

\begin{quote}
It \emph{{[}Theory of Film{]}} rests upon the assumption that film is
essentially an ex­ tension of photography and therefore shares with this
medium a marked affinity for the visible world around us. Films come
into their own when they record and reveal reality.\footnote{Siegfried
  Kracauer, \emph{Theory of Film} (New York: Oxford University Press,
  1960), p. ix.}
\end{quote}

\noindent This approach is diametrically opposed to Langer's view that film as art
cannot function as a simple recording device. As mentioned earlier,
Langer implies that still photography cannot be an art because, unlike
other visual media, such as painting, it merely records what is seen
through the lens of the camera. In her view, the artist's ability to
transform physical reality is minimal. According to Langer, camera and
editorial movement ``redeem'' the photographic nature of cinema, and
free it to become a medium of ex­pression.

If we probe more deeply, however, it becomes apparent that Langer and
Kracauer converge on one essential point. Langer's concept of film as an
expression of human feeling is not incompatible with Kracauer's
inference that cinema may evoke an inner reality beyond material
reality. As Kra­cauer states:

\begin{quote}
They {[}cinematic films{]} point beyond the physical world to the extent
that the shots or combinations of shots from which they are built carry
multiple meanings. Due to the continuous influx of psychophysical
correspondences thus aroused, they suggest a reality which may fittingly
be called `life' ... The concept, `flow of life', then, covers the
stream of material situations and happenings with all that they intimate
in terms of emotions, values, thoughts.\footnote{\emph{Ibid.}, p. 81.}
\end{quote}

\noindent Furthermore, Langer's view that film as art must have a ``matrix'' or
``com­manding form'' is comparable to Kracauer's view that the film
maker's ``formative energies'' must intervene ``in all the dimensions
which the med­ium has to cover''---i.e., the different aspects of
physical reality.

\begin{quote}
It is evident that the cinematic approach materializes in ... all films
which follow the realist tendency. This implies that even films almost
devoid of creative aspirations, such as newsreels, scientific or
educational films, artless documentaries, etc. are tenable propositions
from an aesthetic point of view ... \emph{But as with photographic
reportage, newsreels and the like meet only the minimum requirement} ...
As in photography, everything depends on the `right' balance between the
realist tendency \emph{and the formative tendency}.\footnote{\emph{Ibid.},
  pp. 38--39.} {[}Italics mine.{]}
\end{quote}

\noindent It is important to note, however, that Kracauer believes that
``formative energies'' may enter into the creation of a still
photograph, while Langer sees still photography as a recording device.
Then too, Kracauer believes that the formative tendency should ``follow
the lead'' of the realist tenden­cy, while Langer implies that the matrix
determines both the structure and the content of a film.\footnote{\emph{Ibid.},
  p. 39.}

The essential difference between the two is in the definition of a
cine­matic film. Kracauer's theory deals mainly with the \emph{content}
of a film. Due to his emphasis on physical reality, he favors films
which depict ``moments of everyday life.''\footnote{\emph{Ibid}., p.
  302.} Langer, on the other hand, approaches film, as she does the
other arts, as an expression of forms symbolic of human feeling. Be­cause
she is not as concerned with content as she is with feeling and form, it
is doubtful that she would arbitrarily exclude dramatic, narrative,
historical, and fantasy films from the category of cinematic film. Her
concept of life extends beyond the physical reality to a deeper life of
feeling. The superficial content matter of the film may not be as
important as what it expresses.\footnote{See Section 3, below.}

One of the peculiar qualities of film, Langer states, is its ability to
as­similate diverse materials, such as picture, words, and music, and
transform them into a completely filmic vision.

\begin{quote}
One of the most striking characteristics of this new art is that it
seems to be omnivorous, able to assimilate the most diverse elements and
turn them into elements of its own. With every new invention---montage,
the soundtrack, Technicolor---its devotees have raised the cry of fear
that now its art must be lost ... But the art goes on. It swallows
everything: dancing, skating, drama, panorama, cartooning, music
\ldots{}\footnote{Langer, \emph{Feeling and Form}, p. 412.}
\end{quote}

\noindent It is the fact that film is a poetic art that seems to account, in
Langer's view, for its power of assimilation. The creation of a virtual
present necessarily involves more than the visual element. Thus film may
also include word, music, and sound effects. However, she overstates the
case when she says that film ``enthralls and commingles all the
senses.''\footnote{\emph{Ibid.}, p. 414.} In a dream we may be dimly
aware of the sense of touch, and possibly of smell and taste. Film,
however, is limited at present to sight and hearing.

Generally, the primary illusions of the other arts constitute secondary
illusions in film. Virtual space and time, for instance, are
incorporated in the primary illusion of the virtual present. Although
the picture remains paramount, the semblance of space is a transient
illusion; similarly, words and music are supportive elements. Langer
notes that the film ``needs many, often convergent, means to create the
continuity of emotion which holds it together while its visions roam
through space and time.''\footnote{\emph{Ibid.}} All of the sec­ondary
illusions, then, enrich the substance of film and contribute to the
primary illusion of the virtual present.

It is interesting to compare Langer\textquotesingle s theory of
assimilation with that of Arnheim. In \emph{Film as Art,} Arnheim states
that the separate arts are discrete and complete structural forms; they
may, in some instances, combine suc­cessfully, but the ``personality of
the two partners remains intact, neverthe­less.''\footnote{Rudolf
  Arnheim, \emph{Film as Art} (Berkeley and Los Angeles: University of
  Cali­fornia Press, 1966), pp. 207-8.} The talking film, he feels, is a
hybrid form. The basic problem is that dialogue introduces an alien
element into what is essentially a visual medium. Words, in his view,
serve only to distract from the import of the moving image.\footnote{\emph{Ibid.},
  p. 211.} Langer, on the other hand, says that one art can totally
assimilate and transform the elements of another. Despite the fact that
a song, for instance, contains both words and music, it is not a hybrid
form. The words combine with music to create the illusion of virtual
time. ``When words and music come together in song, music swallows
words; not only mere words and literal sentences, but even literary word
structures, poet­ry.''\footnote{Langer, \emph{Feeling and Form}, p. 152.}
By implication, the inclusion of dialogue in a film supplements the
primary illusion---the creation of a virtual present.

The essential difference between the two is that Arnheim regards film as
a visual medium (although he does admit that music may complement and
reinforce visual imagery), while Langer defines it as a poetic art. Or,
if you will, Arnheim emphasizes the material elements of the film
(sound, picture), while Langer sees it in terms of the illusion that the
materials, when imaginatively used, create. Arnheim explicitly states
that the addi­tion of dialogue to a visual medium cannot be justified by
the fact that in daily life, ``visual and auditory elements are
intimately connected and, in fact, inseparably fused. There must be
artistic reasons for such a combina­tion; it must serve to express
something that could not be said by one of the media alone.''\footnote{Arnheim,
  \emph{Film as Art,} p. 215.} While Langer would agree that film should
not imitate daily life, she would argue that the visual image alone
cannot create the semblance of the virtual present. Words support and
``punctuate'' the total illusion.\footnote{Langer, \emph{Feeling and
  Form,} p. 414.}

At the end of her film essay, Langer mentions that she believes the
novel may be translated into film more easily than drama.\footnote{\emph{Ibid.},
  p. 415.} Spatial considerations are paramount. Drama unfolds within
the fixed space of the stage. Apart from the physical movement of the
actors and a possible change of scenery, the spatial relationships
between the actors and objects therein cannot be significantly altered.
In the novel, however, there is no framework of fixed space. Our focus
shifts from minute detail to vast panorama; locations change; characters
enter and leave at the novelist\textquotesingle s discretion.

Langer notes that the space in both the novel and film is very much like
dream space.

\begin{quote}
Dream events are spatial---often intensely concerned with
space---intervals, endless roads, bottomless canyons, things too high,
too near, too far---but they are not oriented in any total space. The
same is true of the moving picture ... \emph{its space comes and goes}
...\footnote{\emph{Ibid.}}
\end{quote}

\noindent Movable space, then, is the factor which is common to both the film and
the novel.

\hypertarget{3}{%
\section{\textbf{3}}\label{3}}

In the introduction to \emph{Feeling and Form,} Langer states that she
does not offer standards for judging masterpieces against lesser works
of art.\footnote{\emph{Ibid.}, p. vii.} Ac­cordingly, ``A Note on the
Film'' does not establish guidelines for deter­mining if a particular
film is ``good'' or ``bad.'' However, there are tentative implications
which may be drawn from the essay that may be used as cri­teria in
evaluating specific films.

First, if film creates the semblance of virtual present, then the
\emph{complete,} or fully developed, film will incorporate picture,
sound, and, perhaps, color. This conclusion does not negate films that
do not fulfill all three re­quirements; it means simply that they are
incomplete forms of the art. Second, if a work of art, by definition,
expresses forms of human feeling, then those feelings must be organized
in a coherent, harmonious manner and, further, must reflect patterns of
sentience with accuracy. I emphasize that these conclusions are somewhat
speculative.

Furthermore, Langer's theory of assimilation seems to imply that film is
a growing, changing art. The silent, and some black-and-white films,
then, may be viewed as stages in the development of the art, which, in
present perspective, are incomplete. This assumption is also borne out
when Langer says that film developed to a ``fairly high degree'' as a
``silent art.''\footnote{\emph{Ibid.}, p. 412.} The fact that she places
conditions upon the merit of the silent film as a viable art form seems
to indicate that it is one stage in the development of the medium.

Films which function primarily as records of life or experience are, in
the most basic sense, incomplete. These include the early films which
used the still camera to photograph action in a straightforward manner.
The action may have been spontaneous, as in Lumiere's \emph{Le Repas de
bébé,} or staged, as in Edison's \emph{John Rice-May Irwin Kiss,} or
\emph{The Beheading of Mary, Queen of Scots,} which was filmed in one
setup. Into the same category fall those films which record dramatic
plays, such as \emph{Queen Elizabeth,} which was a still-camera,
one-shot affair. Presumably we could also include War­hol's
\emph{Empire,} and his other one-shot films, in that they are basically
photo­graphs on film. Lack of the moving camera, which in Langer's view
freed film to become an art, is the characteristic of all such films.
Although Lan­ger remarks that film has assimilated color, which, by
implication, brought it to a higher stage of development, it is not
clear if she would insist that the complete film incorporate color.
Black-and-white cinematography may be justified where it seems
appropriate for the subject matter, and where it complements the import
of the film.

Langer's definition of film as a poetic art supports the validity of the
narrative as a viable film form. Most poems, plays, novels, and other
pieces of prose fiction tell a story of one sort or another. The
narrative line may be dominant, or it may be submerged in mood,
atmosphere, or introspec­tive reflection. However, it is usually possible
to detect an action line or sequence of events in literary works. The
fact that Langer feels that the novel may be most easily translated into
film may also support this con­clusion, although her bias is grounded on
spatial rather than narrative con­siderations. Presumably all genres are
permissible in that, no matter what the subject, whether it be gangster,
western, horror, musical, science fic­tion, etc., film creates the
illusion of the virtual present.

The question may be raised whether futuristic films or historical
recon­structions can convey the illusion of the virtual present. Although
Langer never discusses either type, it would seem that the ability to
communicate virtual experience depends heavily, as in all other films,
upon the skillful blending of sensitive directing, acting, decor, and
atmosphere with the other elements of the films.

In her essay on film, Langer says that documentary film is a
``pregnant'' invention.\footnote{\emph{Ibid.}} This seems to have
something to do with the fact that there are no ``actors'' as such in
the documentary. It is not clear if she means that it is an incomplete
form of the art.

I would argue that the documentary is a viable film form in Langer's
theory if it is unified by the matrix or import of the film. This may be
as simple as showing how people live, as in Flaherty's \emph{Man of
Aran.} Or it may be more complex: the film maker may attempt to persuade
the audience of his convictions, as Wiseman does in \emph{High School.}
As long as the docu­mentary is not an apparently random flow of images
without any guiding purpose, it would seem to be a valid film form.

If film creates a virtual experience, then we may argue that
representational images are some part of our perception of experience.
We do not nor­mally see in terms of the lines, dots, and amorphous images
of which the films of John Whitney and Jordan Belson are composed. We
might con­clude that the abstract film is not a viable form; it may be a
hybrid form, part film and part ``pictorial'' art. Still it must be
admitted that sometimes these films do communicate a sense of the rhythm
of life. In terms of Langer's theory, however, it is fairly certain that
these films do not create a sense of the virtual present as easily or as
effectively as those which em­ ploy representational images.

Surrealist and Dada films are a troublesome category. \emph{Le Chien
Andalou,} for instance, is composed of identifiable images---quite
realistically shot, in terms of photographic style---that do not attempt
to make sense in conven­tional terms. In one sense, these films resemble
our actual dreams, in that they appear to be a jumble of disconnected or
inexplicably juxtaposed images. The question, however, may be one of
intent and perception. What the author intends may not be perceived by
the viewer, or the intent may be purposely obscure, or the intent may be
``meaninglessness'' itself. The fact remains that these films intrigue
us, and sometimes evoke strong but verbally inexpressible feelings. No
absolute conclusions may be drawn from Langer\textquotesingle s theory
as to the general validity of this form.

At least we may say that films which resemble dreams are not necessarily
``good'' films. To reiterate, Langer explicitly states that film does
not \emph{copy} dream; it is \emph{like} dream in that it creates the
illusion of a virtual present. This raises the question of the validity
of films like \emph{L'Etoile de Mer,} the German Expressionist films,
and the more recent sub-genre of drug films, which employ distorted
images of an inner reality. Although Langer never deals with the issue,
it would seem that the effectiveness of these films in creating a
virtual present would depend, at least in part, upon the ability of the
audience to relate intellectually or experientially to the distorted
images.

\hypertarget{4}{%
\section{\textbf{4}}\label{4}}

The concept of significance or vital import runs throughout Langer's
philosophy of art. It is the matrix, ``commanding form,'' or unifying
idea or feeling that the artist conceives of before he begins to create
a work of art, and which dictates the structure of that work, as well as
all the detail in­cluded therein. Langer likens the matrix to
Eisenstein's concept of ``the initial general image which originally
hovered before the creative artist.''\footnote{\emph{Ibid.}, p. 414.} It
is the abstraction of the forms of human feeling.

This standard is particularly useful in determining ``what the film is
about,'' not in terms of subject matter but in terms of the author's
intent. Did he carry through with his apparent intent? If we are unsure
of the vital import of the film, it may be that the artist was unsure of
his intent. The standard is also helpful in determining if elements of
style, structure, or detail are wholly in accordance with the author's
intent. Sometimes we feel that something is ``wrong'' or out of place in
a film. By referring to the matrix, we may determine if this is in fact
the case, and if so, why it doesn't work.

In discussing music, Langer remarks that the function of the arts is not
to stimulate the senses, but to express human feeling. We may infer that
a film whose sole function is entertainment of one sort or another is
not a viable form of the art. Into this category fall hard-core
pornography, films which use the sex-and-violence formula for no other
purpose than to give the viewer a vicarious thrill, and the myriad
productions starring Debbie Reynolds, Doris Day, Annette Funicello, and
others, whose sole function is to keep the viewer amused for a couple of
hours. This does not mean that a film which is entertaining or which
arouses the emotions cannot be a work of art. It simply means that we
must look beyond entertainment value and sensory stimulation to the
vital import of the film. Indeed, a ``good'' film may combine all three
elements.

Langer's theory of musical assimilation suggests that in the ``good''
film, picture, words, and music will be blended harmoniously to create
the illu­sion of virtual experience. In a sense, this is similar to
Panovsky's principle of co-expressibility. However, it does not preclude
the possibility, for ex­ample, of using music as counterpoint to the
visual images. It means only that no element of the film should be so
jarring or distracting or obvious that it calls extreme attention to
itself, unless, perhaps, it is part of the film maker's intent. In this
case, significance would take precedence over ideal assimilation. It may
also be implied that the author\textquotesingle s intent should not
dominate the film. In other words, it should \emph{evolve} out of the
combined elements of the film, rather than being blatantly obvious.

If art is the abstract expression of forms of human feeling, I would
argue that those feelings which constitute the matrix of a particular
work of art must be ``true to life.'' That is, the import of the film
must accurately re­flect those patterns of sentience that the film is
about.

In discussing expressiveness, Langer says that in the early stages of
the growth of a work of art, the ``envisagement'' of the symbol may be
inter­fered with by unformed or unrecognized emotions that may distort
the artist's imagination of subjective experience.

\begin{quote}
Art which is thus distorted at its very source by a lack of candor is
bad art, and it is bad art because it is not true to \emph{what a candid
envisagement would have been.} Candor is the standard; `seeing
straight,' the vernacular calls it.\footnote{\emph{Ibid.}, pp. 380--81.}
\end{quote}

\noindent She concludes vehemently that bad art is corrupt art, and should be
de­stroyed.\footnote{\emph{Ibid.}, p. 381.}

If truth is an essential ingredient in a good work of art, then all
those films whose matrix has been distorted by deceit, dishonesty,
greed, or con­fusion must be called bad films. Often, films tell us more
about their makers than what the intent was supposed to be. As Parker
Tyler has noted in his essay on \emph{The Portrait of Dorian Grey,} the
``real'' truth has a nasty way of sneaking into the film.

The value of Langer's theory is that it enables one to examine film from
the perspective of all the arts and discover what it has in common with,
and how it differs from, the others. Although theories which attempt to
narrow the field of art to a few special filmic modes, like those of
Arnheim and Kracauer, may provide specific insights into the nature of
film, still they seem limited in comparison.

Langer's theory also gives us a clue to those special qualities which
dis­tinguish a masterpiece from lesser forms of the art. A ``good'' film
will en­thrall us again and again. With each new viewing, we may learn
more about the patterns of feeling that form its matrix, and, possibly,
more about our­ selves and mankind. This, perhaps, is what the art is all
about.


\end{document}