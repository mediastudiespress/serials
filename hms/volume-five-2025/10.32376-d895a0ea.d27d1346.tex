% see the original template for more detail about bibliography, tables, etc: https://www.overleaf.com/latex/templates/handout-design-inspired-by-edward-tufte/dtsbhhkvghzz

\documentclass{tufte-handout}

%\geometry{showframe}% for debugging purposes -- displays the margins

\usepackage{amsmath}

\usepackage{hyperref}

\usepackage{fancyhdr}

\usepackage{hanging}

\hypersetup{colorlinks=true,allcolors=[RGB]{97,15,11}}

\fancyfoot[L]{\emph{History of Media Studies}, vol. 5, 2025}


% Set up the images/graphics package
\usepackage{graphicx}
\setkeys{Gin}{width=\linewidth,totalheight=\textheight,keepaspectratio}
\graphicspath{{graphics/}}

\title[A Note on Langer]{A Note on Langer: An Introduction to the Susanne Langer on Film Special Section} % longtitle shouldn't be necessary

% The following package makes prettier tables.  We're all about the bling!
\usepackage{booktabs}

% The units package provides nice, non-stacked fractions and better spacing
% for units.
\usepackage{units}

% The fancyvrb package lets us customize the formatting of verbatim
% environments.  We use a slightly smaller font.
\usepackage{fancyvrb}
\fvset{fontsize=\normalsize}

% Small sections of multiple columns
\usepackage{multicol}

% Provides paragraphs of dummy text
\usepackage{lipsum}

% These commands are used to pretty-print LaTeX commands
\newcommand{\doccmd}[1]{\texttt{\textbackslash#1}}% command name -- adds backslash automatically
\newcommand{\docopt}[1]{\ensuremath{\langle}\textrm{\textit{#1}}\ensuremath{\rangle}}% optional command argument
\newcommand{\docarg}[1]{\textrm{\textit{#1}}}% (required) command argument
\newenvironment{docspec}{\begin{quote}\noindent}{\end{quote}}% command specification environment
\newcommand{\docenv}[1]{\textsf{#1}}% environment name
\newcommand{\docpkg}[1]{\texttt{#1}}% package name
\newcommand{\doccls}[1]{\texttt{#1}}% document class name
\newcommand{\docclsopt}[1]{\texttt{#1}}% document class option name


\begin{document}

\begin{titlepage}

\begin{fullwidth}
\noindent\LARGE\emph{Article
} \hspace{98mm}\includegraphics[height=1cm]{logo3.png}\\
\noindent\hrulefill\\
\vspace*{1em}
\noindent{\Huge{A Note on Langer: An Introduction to the Susanne Langer on Film Special Section\par}}

\vspace*{1.5em}

\noindent\LARGE{Jefferson Pooley} \href{https://orcid.org/0000-0002-3674-1930}{\includegraphics[height=0.5cm]{orcid.png}}\par\marginnote{\emph{Jefferson Pooley and Sue Curry Jansen, ``A Note on Langer: An Introduction to the Susanne Langer on Film Special Section,'' \emph{History of Media Studies} 5 (2025), \href{https://doi.org/10.32376/d895a0ea.d27d1346}{https://doi.org/ 10.32376/d895a0ea.d27d1346}.} \vspace*{0.75em}}
\vspace*{0.5em}
\noindent{{\large\emph{University of Pennsylvania}, \href{mailto:jeff.pooley@asc.upenn.edu}{jeff.pooley@asc.upenn.edu}\par}} \marginnote{\href{https://creativecommons.org/licenses/by-nc/4.0/}{\includegraphics[height=0.5cm]{by-nc.png}}}

\vspace*{0.75em} 

\noindent{\LARGE{Sue Curry Jansen}\par}
\vspace*{0.5em}
\noindent{{\large\emph{Muhlenberg College}, \href{mailto:jansen@muhlenberg.edu}{jansen@muhlenberg.edu}\par}}

% \vspace*{0.75em} % third author

% \noindent{\LARGE{<<author 3 name>>}\par}
\vspace*{0.5em}
% \noindent{{\large\emph{<<author 3 affiliation>>}, \href{mailto:<<author 3 email>>}{<<author 3 email>>}\par}}

\end{fullwidth}

\vspace*{1em}


\newthought{The American philosopher} Susanne K. Langer (1895--1985) came of age when
mass media---film and radio---were still in their formative stages.
World War I would establish their strategic significance and prioritize
their technological development. Sustained scholarly analysis of the
popular culture that they produced, did not, however, emerge in the U.S.
until the 1950s. While much of Langer's work focused on aesthetics, the
``popular artists of the screen, the jukebox, the shop-window, and the
picture magazine'' did not attract her fierce, penetrating
attention.\footnote{Susanne K. Langer, ``The Cultural Importance of
  Art,'' in \emph{Philosophical Sketches} (Baltimore: Johns Hopkins
  University Press, 1962), 93.} She made an exception just once, for
film, which she heralded as a ``new poetic mode''---but in a mere five
pages, as an appendix to a 1953 book on art.\footnote{Susanne K. Langer,
  ``A Note on the Film,'' in \emph{Feeling and Form: A Theory of
  Art} (New York: Charles Scribner's Sons, 1953), 411.}

We reprint ``A Note on the Film'' here, together with a pair of
essay-commentaries on Langer's fragment, one from 1974 and the other
2020.\footnote{Courtenay Wyche Beinhorn, ``Susanne Langer's Film Theory:
  Elaboration and Implications,'' \emph{Cinema Journal} 13, no. 2 (1974)
  {[}reprinted with permission{]}; Tereza Hadravová, ``Film as a Dream
  of the Modern Man: Interpretation of Susanne Langer's Note on the
  Film,'\,'' \emph{Eidos. A Journal for Philosophy of Culture} 4, no. 1
  (2020) {[}reprinted with light revisions, under the original
  publication's Creative Commons Attribution-NonCommercial-NoDerivs 3.0
  Unported License{]}. The only other substantial treatment is an
  insightful 1978 dissertation, available online: Trisha Curran, ``A New
  Note on the Film: A Theory of Film Criticism Derived from Susanne K.
  Langer's Philosophy of Art'' (PhD diss., Columbus,
  Ohio State University, 1978).} The book to which it was appended,
\emph{Feeling and Form} (1953), is a masterwork---a philosophically
confident, richly informed, and nuanced tour, medium by medium, through
the arts: sculpture, painting, poetry, music, dance, and drama. That
book, in turn, was a sequel to Langer's surprise 1941 bestseller
\emph{Philosophy in a New Key}, which developed its core argument
through the example of music.\textsuperscript{4}

The new key that Langer hoped would reinvigorate philosophy was the
study of \emph{symbolism.} She distinguished between two types:
\emph{discursive} and \emph{presentational}. Philosophers, she said,
have tended to fixate on the first---language and its extensions in math
and formal

\enlargethispage{2\baselineskip}

\vspace*{2em}

\noindent{\emph{History of Media Studies}, vol. 5, 2025}


 \end{titlepage}

% \vspace*{2em} | to use if abstract spills over



\noindent logic.\textsuperscript{4}\setcounter{footnote}{4}\marginnote{Langer, \emph{Feeling and Form: A
  Theory of Art} (New York: Charles Scribner's Sons, 1953); Langer,
  \emph{Philosophy in a New Key: A Study in the Symbolism of Reason,
  Rite, and Art} (Cambridge, MA: Harvard University Press, 1942).} They have largely ignored the vast sea of knowledge
surrounding that ``tiny, grammar-bound island.''\footnote{Langer,
  \emph{Philosophy in a New Key}, 82.}

Langer's life project was to make sense of this elusive, ``ineffable''
class of symbolism. She insisted that presentational symbols are
rational in their own way---that they bear vital meanings that science
and discursive language cannot express. She developed the argument in
luminous, metaphoric prose that, in formal terms, helped exemplify her
claims. Her analysis ranged over pre-linguistic modes like magic,
ritual, and myth, as well as the ``riotous'' symbolism of dreams. But
her main object of attention was art\emph{---}all kinds of art, with
music, literature, drama, and architecture included.

In \emph{Feeling and Form} Langer explained how the arts make their
meanings. Each major form is a distinctive way of knowing, with its own
(non-discursive) articulateness. Dance, for example, ``speaks'' through
gesture; a dancer's movements are symbols for the living experience of
power and agency. Music, for its part, represents time---it ``spreads
out time for our direct and complete apprehension, by letting our
hearing mobilize it.''\footnote{Langer, \emph{Feeling and Form}, 110.}
Each of the arts is distinguishable, even autonomous, from the others,
with a symbolic mode---a way of knowing---all its own.

\emph{Feeling and Form}'s formalist treatment of the arts, chapter by
chapter, has some obvious parallels with the ``medium theory'' tradition
associated with Marshall McLuhan and his medium-is-the-message formula.
McLuhan himself read and annotated Langer's \emph{Philosophy in a New
Key}, yet he apparently never cited her.\footnote{The only reference to
  Langer in McLuhan's published work is as translator of German
  philosopher Ernst Cassirer's \emph{Language and Myth} (New York:
  Harper, 1946). See McLuhan, \emph{The Gutenberg Galaxy: The Making of
  Typographic Man} (Toronto: University of Toronto Press, 1962), 282. On
  McLuhan's ``lightly annotated'' copy of \emph{Philosophy in a New
  Key}, see Marshall McLuhan Library Collection, Thomas Fisher Rare Book
  Library, University of Toronto Libraries,
  \url{https://fisher.library.utoronto.ca/mcluhan-library}, 245.} And
subsequent scholars invested in medium specificity have rarely engaged
with Langer's work.\footnote{The eclectic media ecology movement has
  recognized her relevance to their project, but few treatments have
  gone beyond listing her among the movement's key figures. See, for
  example, Casey Man Kong Lum, \RL{``}Notes Toward an Intellectual
  History of Media Ecology,'' in \emph{Perspectives on Culture,
  Technology, and Communication: The Media Ecology Tradition}, ed. Casey
  Man Kong Lum (New York: Hampton Press, 2006), 22, 28. Media ecologist
  Christine Nystrom's 2000 essay on Langer and Benjamin Lee Whorf is a
  smart excep-}

We have cited one likely reason for this neglect: Langer's relative
indifference to commercial media. Another explanation has to do with her
home discipline. For a variety of reasons, some of them of her own
making, Langer did not command much influence among fellow philosophers
during her lifetime, with the notable exception of her contributions to
aesthetics. That subfield was coded as less central to the philosophic
project, more intuitive, and more feminine; and much of Langer's
marginal status otherwise can be traced to the discipline's
well-documented hostility to women in the mid-twentieth century. Because
she insisted that the knowledge conveyed by the arts is logical in the
strictest sense, Langer seems to have attracted a particularly vehement
version of that hostility.\textsuperscript{9} As a marginal
member of the guild, her ultimate ambition---to reinvent philosophy
itself and with it a new theory of society---would have also invited
rejection.

The relative neglect of her work is beginning to lift. The Langer
Circle, a scholarly society launched in 2020, sponsors conferences and
conversation\marginnote{tion, as is John Power's 2006 chapter on Langer's
  philosophy of mind. Nystrom, ``Symbols, Thought, and Reality: The
  Contributions of Benjamin Lee Whorf and Susanne K. Langer to Media
  Ecology,'' \emph{New Jersey Journal of Communication} 8, no. 1 (2000);
  and John H. Powers, ``Susanne Langer's Philosophy of Mind:
  Some Implications for Media Ecology,'' in \emph{Perspectives on
  Culture, Technology, and Communication: The Media Ecology Tradition},
  ed. Casey Man Kong Lum (New York: Hampton Press, 2006).} among\marginnote{\textsuperscript{9} See Randall E. Auxier,
  ``Foreword,'' in \emph{The Bloomsbury Handbook of Susanne K.
  Lange}r, ed. Lona Gaikis (London: Bloomsbury, 2024), xii--xiv; Arthur
  C. Danto, ``Three Careers,'' in \emph{The Visionary Academy of
  Ocular Mentality: Atlas of the Iconic Turn}, ed. Luca Del Baldo
  (Berlin: de Gruyter, 2020), 126; and Adrienne Dengerink Chaplin,
  \emph{The Philosophy of Susanne Langer: Embodied Meaning in Logic, Art
  and Feeling} (London: Bloomsbury, 2020), chap. 3.}\setcounter{footnote}{9} the growing ranks of the Langer-curious. Full-length
monographs---including Adrienne Dengerink Chaplin's superb
\emph{Philosophy of Susanne Langer} (2019)---have been published
alongside collections like \emph{The Bloomsbury Handbook of Susanne K.
Langer} (2024). Researchers across a range of fields are discovering, or
re-discovering, Langer.\footnote{Susanne K. Langer Circle,
  \url{https://langercircle.sites.uu.nl}; Chaplin, \emph{The Philosophy of
  Susanne Langer}; and Lona Gaikis, ed., \emph{The Bloomsbury Handbook
  of Susanne K. Langer} (London: Bloomsbury, 2024).}

To date, however, media scholars have contributed little to the Langer
renaissance now underway. This is unfortunate, since the richness and
relevance of her work has much to contribute to a deeper understanding
of media and communication---on multiple registers. This modest
collection---``A Note on the Film'' and the two exegeses re-published
here---is an invitation to probe Langer's full, rousingly generative
writings. We think of the fragment on film as something like a
save-the-date notice, an advertisement for the media-relevant thinking
that permeates \emph{Philosophy in a New Key}, \emph{Feeling and Form},
and her final, monumental project, the three-volume \emph{Mind} (1967,
1972, and 1982).\footnote{Susanne K. Langer, \emph{Mind: An Essay on
  Human Feeling,} 3 vols. (Baltimore: Johns Hopkins University Press,
  1967, 1972, 1982).}

\hypertarget{a-note-on-the-note}{%
\section{A Note on `The Note'}\label{a-note-on-the-note}}

Cinema, Langer declared in \emph{Feeling and Form}, is a new art. ``Our
own age,'' she wrote, ``has seen the birth of the motion picture, which
is not only in a new medium, but is a new mode.''\footnote{Langer,
  \emph{Feeling and Form}, xii.} She later admitted that the spare,
five-page ``A Note on the Film'' was ``only an impressionistic sketch''
for a simple reason: She wasn't much of a movie-goer. The ``number of
films I've seen in my life I could count on my fingers and toes,'' she
said. ``If I had seen more, I would have written an extra chapter like I
did for the other arts.''\footnote{Langer, ``A Note on the Film,'' 411.
  The Langer quotes are from a 1978 interview conducted by Trisha
  Curran, ``A New Note on the Film,'' 1.}

Those other arts, recall, each have formal properties that, when
realized in works, make for distinctive meanings. Film's key feature,
she argued, is the \emph{moving camera}, its roaming through space and
time. Drama is fixed to a stage and an audience. The movie viewer, by
contrast, ``sees with the camera; his standpoint moves with it, his mind
is pervasively present.''\footnote{Langer, ``A Note on the Film,''
  413.} A film is indeed a \emph{motion} picture, a space- and
time-machine that lends its eyes to the audience.

Langer's stress on the moving camera is interesting enough, even if it
is a point not infrequently made in the film-theory canon---the bundle
of texts identified and extended by the then-new film studies field in
the decades after ``A Note on the Film.'' Art historian Erwin Panofsky's
1937 essay on cinema, for example, also centers on film's departure from
the fixed stage---its ``dynamization of space'' and ``spatialization of
time.''\footnote{Erwin Panofsky, ``Style and Medium in the Moving
  Pictures,'' \emph{Transition} 26 (1937).} Langer did not cite
Panofsky, and she only sparingly references other writers on the medium.
A notable exception is Sergei Eisenstein, the Soviet filmmaker and
theorist, whose 1943 \emph{The Film Sense} Langer uses to sharpen her
crucial but counter-intuitive point about the ``virtual'' experience of
the film-goer.\footnote{Langer, \RL{``}A Note on the Film,'' 413--15;
  Sergei Eisenstein, \emph{The Film Sense} (London: Faber and Faber,
  1943). See Tereza Hadravová, ``Film as a Dream of the Modern
  Man,'' reprinted here, for a rich exploration of Langer's differences
  with, and debts to, Eisenstein.}

The term ``virtual,'' alas, now refers to expensive VR headsets. So it
takes effort to occupy the meaning that Langer had in mind. The key to
understanding Langer on film---and the other arts, for that matter---is
the idea of \emph{formal} resemblance. A film, or a painting, doesn't
directly represent experience. Instead, the film (or painting or poem)
mimics the \emph{shape} of human activity. Art bears a likeness to human
life, in other words, but not in the usual sense of ``looks like'' or
``sounds like'' or ``has come to refer to.'' The relationship of art to
life is about similarity of \emph{form}.

Consider a piece of instrumental music, a defining example for Langer
(an accomplished cellist herself). In its pitch, cadence, tone,
orchestral arrangement---its morphology, so to speak---music may
register aspects of life (``vital impulse, balance, conflict, the ways
of living and dying and feeling'') without describing or depicting
them.\footnote{Langer, \emph{Philosophy in a New Key}, 207.} All of the
arts have this indirect, but still logical, relation to experience---a
certain distance from lived reality, even for those modes (like film and
photography) that are hard to pry loose from the scenes they ostensibly
``capture.''

Here, then, is the sense of ``virtual'' that Langer intends: alike, but
distinct from, life\emph{.} The arts, including film, present virtual
overlays of lived experience. They provide insights about human
existence, as revealed by their formal---their virtual---re-enactments.
The arts, film among them, provide vital knowledge---important, yes, but
vital in the sense of life and living, the organic patterns and rhythms
that are the condition of human existence. We cannot get to these
insights otherwise; they escape everyday reflection and the discursive
mode of reasoning that we are all steeped in.

The distinctive sense that film creates---its ``primary illusion''---is
``virtual history.''\footnote{Langer, ``A Note on the Film,'' 412.}
What Langer means can be drawn out by comparison with a sibling art,
poetry.\footnote{Indeed, Langer classifies film, poetry proper,
  narrative, and drama as ``poetic arts,'' each with a distinctive
  relationship to the experience of time. Langer, \emph{Feeling and
  Form}, 266.} Despite its arrangement of words, poetry for Langer is
non-discursive. Through its rhythms, tensions, and balances, a poem
symbolizes experience itself. It's not about an \emph{actual}
experience, nor is it an invitation to feel something. Instead, Langer
writes, the poet's ``business is to create the appearance of
`experiences,' the semblance of events lived and felt, to organize them
so they constitute a purely and completely experienced reality, a piece
of \emph{virtual life}.'' In contrast to the jumble of everyday
experience, a poem pulls out---distills, in a sense---experience itself:
``The \emph{illusion of life} is the primary illusion of all poetic
art.''\footnote{Langer, \emph{Feeling and Form}, 212, 213.}

Virtual life, virtual history---these are the primary ``illusions'' of
poetry and film, respectively. They are illusions in that they aren't
life or history, but instead \emph{symbols} of life, \emph{symbols} of
history. Poetry and film both symbolize through resemblance, but their
similarity to the symbolized is formal---the stuff of shape, pattern,
and morphology.

By ``history'' Langer means the experience of \emph{immediacy}---the
sense of the ongoing moment, or what she calls the ``endless Now.'' It
is a virtual present, of course, given in form. Film's symphonic,
omnivorous character---the way it assimilates other arts as well as the
sensory manifold---contributes to this feeling of immediacy. Film,
Langer writes, ``swallows everything: dancing, skating, drama, panorama,
cartooning, music,'' even as it ``enthralls and commingles all senses.''
At the same time, the camera lends the viewer its own roving eye, in a
space-shifting gambol through an ``eternal and ubiquitous virtual
present.'' If the medium has a message, then, it is an abstracted sense
of now-ness.\footnote{Langer, ``A Note on the Film,'' 415, 412,
  414, 415.}

In developing the point, Langer cites the dream-like quality of
spectating. The analogy to dreaming is, of course, as old as film
itself; the metaphor played midwife, among other things, to the\\\noindent 1970s
psychoanalytic turn in cinema studies.\footnote{Laura Rascaroli,
  \RL{``}Oneiric Metaphor in Film Theory,'' \emph{Kinema: A Journal for
  Film and Audiovisual Media} (Fall 2002).} So Langer's reflections on
the ``dream mode'' may come off, on first read, as yet another nod to a
worn theme. But here again it is formal resemblance she has in
mind---not the thick claim that, say, darkened-theater movie-goers
regress into a dream state. What is similar about watching a film and
dreaming is that both involve rapid and cutting shifts of location, a
tumbling immediacy in space that ``comes and goes.''\footnote{Langer,
  \RL{``}A Note on the Film,'' 412--15.} Film watching is also
\emph{unlike} dreaming---the spectator isn't in the film, in contrast to
the dreamer-protagonist. It is, instead, the space-hopping immediacy,
the roving eye-camera, that connects film to the dream mode.

\hypertarget{media-theory-in-a-new-key}{%
\section{Media Theory in a New Key}\label{media-theory-in-a-new-key}}

As Courtenay Wyche Beinhorn observes in her 1974 essay, reprinted here,
``The value of Langer's theory is that it enables one to examine film
from the perspective of all the arts and discover what it has in common
with, and how it differs from, the others.''\footnote{Beinhorn,
  \RL{``}Susanne Langer\RL{'}s Film Theory,'' 54.} Beinhorn is right,
and we republish ``A Note on the Film'' with that aim in mind---to share
a mostly forgotten fragment of film theory, for its original and
intelligent meditation on the medium.

We have an additional motive. Over the last two years, we have read
through Langer's works, with gathering excitement. We started with
\emph{Philosophy in a New Key}, after a podcast re-kindled our
curiosity.\footnote{``Susanne Langer on Our Symbol-Making Nature,''
  \emph{The Partially Examined Life}, March 28, 2022,
  \url{https://partiallyexaminedlife.com/2022/03/28/ep290-1-langer-symbolism/}.}
The book's astonishing perspicacity, its (uncredited) prescience\\\noindent---even
the way its sentences re-enact their claims in metaphor---kept us
reading, through \emph{Feeling and Form,} then a pair of brilliant essay
collections, and on to Langer's three-volume opus \emph{Mind}, published
over 15 years and cut short by Langer's failing eyesight. We recalled
(in the first person, for one of us) \emph{Philosophy in a} \emph{New
Key}'s ubiquity in the 1950s paperback revolution. And we registered the
testimony of friends and colleagues, who remembered the book as an
important, if inchoate, influence.

We knew from the book's first pages what the rest of her works
confirmed: Susanne Langer was a media theorist, set in a different, more
expansive key. For Langer, it was symbolization all the way down, and
all the way back. She held that most of our symbolizing, for most of
human history, has been non-discursive, a vast ocean surrounding (once
it surfaced, rather late) a tiny, grammar-bound island. She made that
point in brilliant, self-exemplifying, iridescent prose.

Who now reads Langer? Plenty of people, as we learned when we stumbled
upon the Langer revival now underway. But not media scholars---not yet.
Thus this small collection, centered on a five-page film-theory
fragment, is a tease, an invitation, and a promissory note:

\begin{quote}
The modern mind is an incredible complex of impressions and
transformations; and its product is a fabric of meanings that would make
the most elaborate dream of the most ambitious tapestry-weaver look like
a mat. The warp of that fabric consists of what we call `data,' the
signs to which experience has conditioned us to attend, and upon which
we act often without any conscious ideation. The woof is symbolism. Out
of signs and symbols we weave our tissue of `reality.'\footnote{Langer,
  \emph{Philosophy in a New Key}, 235--36.}
\end{quote}


\newpage\section{Bibliography}\label{bibliography}

\begin{hangparas}{.25in}{1} 



Auxier, Randall E. ``Foreword.'' In \emph{The Bloomsbury Handbook of
Susanne K. Langer,} edited by Lona Gaikis, xii--xiv. London: Bloomsbury,
2024.

Beinhorn, Courtenay Wyche. ``Susanne Langer's Film Theory: Elaboration
and Implications.'' \emph{Cinema Journal} 13, no. 2 (1974): 41--54.
\url{https://doi.org/10.2307/1225250}.

Cassirer, Ernst. \emph{Language and Myth}, translated by Susanne K.
Langer. New York: Harper, 1946

Chaplin, Adrienne Dengerink. \emph{The Philosophy of Susanne Langer:
Embodied Meaning in Logic, Art and Feeling}. London: Bloomsbury, 2020.

Curran, Trisha. ``A New Note on the Film: A Theory of Film Criticism
Derived from Susanne K. Langer's Philosophy of Art.'' PhD diss., Ohio
State University, 1978.
\url{https://etd.ohiolink.edu/acprod/odb_etd/ws/send_file/send?accession=osu148707946402129\&disposition=inline}.

Danto, Arthur C. ``Three Careers.'' In \emph{The Visionary Academy of
Ocular Mentality: Atlas of the Iconic Turn}, edited by Luca Del Baldo,
125--30. Berlin: de Gruyter, 2020.

Eisenstein, Sergei. \emph{The Film Sense}. London: Faber and Faber,
1943.

Gaikis, Lona. \emph{The Bloomsbury Handbook of Susanne K. Langer}.
London: Bloomsbury, 2024.

Hadravová, Tereza. ``Film as a Dream of the Modern Man: Interpretation
of Susanne Langer's `Note on the Film.'\,'' \emph{Eidos. A Journal for
Philosophy of Culture} 4, no. 1 (2020): 38--48.
\url{https://doi.org/10.14394/eidos.jpc.2020.0004}.

Langer, Susanne K. \emph{Philosophy in a New Key: A Study in the
Symbolism of Reason, Rite, and Art}. Cambridge, MA: Harvard University
Press, 1942.

Langer, Susanne K. ``A Note on the Film.'' In \emph{Feeling and Form: A
Theory of Art}, 411--5. New York: Charles Scribner's Sons, 1953.

Langer, Susanne K. \emph{Feeling and Form: A Theory of Art}. New York:
Charles Scribner's Sons, 1953.

Langer, Susanne K. \emph{Mind: An Essay on Human Feeling}. 3 vols.
Baltimore: Johns Hopkins University Press, 1967, 1972, 1982.

Langer, Susanne K. ``The Cultural Importance of Art.'' In
\emph{Philosophical Sketches}, 83--94. Baltimore: Johns Hopkins
University Press, 1962.

Lum, Casey Man Kong. ``Notes Toward an Intellectual History of Media
Ecology.'' In \emph{Perspectives on Culture, Technology, and
Communication: The Media Ecology Tradition}, edited by Casey Man Kong
Lum, 1--60. New York: Hampton Press, 2006.

\enlargethispage{\baselineskip}

Marshall McLuhan Library Collection, Thomas Fisher Rare Book Library,
University of Toronto Libraries,
\url{https://fisher.library.utoronto.ca/mcluhan-library}.

McLuhan, Marshall. \emph{The Gutenberg Galaxy: The Making of Typographic
Man}. Toronto: University of Toronto Press, 1962.

Nystrom, Christine L. "Symbols, thought, and reality: The contributions
of Benjamin Lee Whorf and Susanne K. Langer to media ecology.''
\emph{New Jersey Journal of Communication} 8, no. 1 (2000): 8--33.

Panofsky, Erwin. ``Style and Medium in the Moving Pictures.''
\emph{Transition} 26 (1937): 121--33.

Powers, John H. ``Susanne Langer's Philosophy of Mind: Some Implications
for Media Ecology.'' In \emph{Perspectives on Culture, Technology, and
Communication: The Media Ecology Tradition}, edited by Casey Man Kong
Lum, 303--34. New York: Hampton Press, 2006.

``Susanne Langer on Our Symbol-Making Nature.'' \emph{The Partially
Examined Life} {[}podcast episode{]}, March 28, 2022.\\
\hspace{.25in}\href{https://partiallyexaminedlife.com/2022/03/28/ep290-1-langer-symbolism}{https://partiallyexaminedlife.com/2022/03/28/ep290-1-langer-}\\ \hspace{.25in}\href{https://partiallyexaminedlife.com/2022/03/28/ep290-1-langer-symbolism}{symbolism}.

Rascaroli, Laura. ``Oneiric Metaphor in Film Theory.'' \emph{Kinema: A
Journal for Film and Audiovisual Media} (Fall 2002).
\url{https://doi.org/10.15353/kinema.vi.982}.



\end{hangparas}


\end{document}