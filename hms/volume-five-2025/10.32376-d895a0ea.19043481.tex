% see the original template for more detail about bibliography, tables, etc: https://www.overleaf.com/latex/templates/handout-design-inspired-by-edward-tufte/dtsbhhkvghzz

\documentclass{tufte-handout}

%\geometry{showframe}% for debugging purposes -- displays the margins

\usepackage{amsmath}

\usepackage{hyperref}

\usepackage{fancyhdr}

\usepackage{hanging}

\usepackage{longtable}

\hypersetup{colorlinks=true,allcolors=[RGB]{97,15,11}}

\fancyfoot[L]{\emph{History of Media Studies}, vol. 5, 2025}


% Set up the images/graphics package
\usepackage{graphicx}
\setkeys{Gin}{width=\linewidth,totalheight=\textheight,keepaspectratio}
\graphicspath{{graphics/}}

\title[Los estudios sobre medios indígenas en Argentina]{Perspectivas en torno a la configuración histórica de los estudios sobre medios indígenas en Argentina} % longtitle shouldn't be necessary

% The following package makes prettier tables.  We're all about the bling!
\usepackage{booktabs}

% The units package provides nice, non-stacked fractions and better spacing
% for units.
\usepackage{units}

% The fancyvrb package lets us customize the formatting of verbatim
% environments.  We use a slightly smaller font.
\usepackage{fancyvrb}
\fvset{fontsize=\normalsize}

% Small sections of multiple columns
\usepackage{multicol}

% Provides paragraphs of dummy text
\usepackage{lipsum}

% These commands are used to pretty-print LaTeX commands
\newcommand{\doccmd}[1]{\texttt{\textbackslash#1}}% command name -- adds backslash automatically
\newcommand{\docopt}[1]{\ensuremath{\langle}\textrm{\textit{#1}}\ensuremath{\rangle}}% optional command argument
\newcommand{\docarg}[1]{\textrm{\textit{#1}}}% (required) command argument
\newenvironment{docspec}{\begin{quote}\noindent}{\end{quote}}% command specification environment
\newcommand{\docenv}[1]{\textsf{#1}}% environment name
\newcommand{\docpkg}[1]{\texttt{#1}}% package name
\newcommand{\doccls}[1]{\texttt{#1}}% document class name
\newcommand{\docclsopt}[1]{\texttt{#1}}% document class option name


\begin{document}

\begin{titlepage}

\begin{fullwidth}
\noindent\LARGE\emph{Article
} \hspace{99mm}\includegraphics[height=1cm]{logo3.png}\\
\noindent\hrulefill\\
\vspace*{1em}
\noindent{\Huge{Perspectivas en torno a la configuración\\\noindent histórica de los estudios sobre medios\\\noindent indígenas en Argentina\par}}

\vspace*{1.5em}

\noindent\LARGE{Mariana de los Ángeles Ortega}\par\marginnote{\emph{Mariana de los Ángeles Ortega, ``Perspectivas en torno a la configuración histórica de los estudios sobre medios indígenas en Argentina,'' \emph{History of Media Studies} 5 (2025), \href{https://doi.org/10.32376/d895a0ea.19043481}{https://doi.org/ 10.32376/d895a0ea.19043481}.} \vspace*{0.75em}}
\vspace*{0.5em}
\noindent{{\large\emph{Universidad Nacional de Salta}, \href{mailto:marortega.012@gmail.com}{marortega.012@gmail.com}\par}} \marginnote{\href{https://creativecommons.org/licenses/by-nc/4.0/}{\includegraphics[height=0.5cm]{by-nc.png}}}

% \vspace*{0.75em} % second author

% \noindent{\LARGE{<<author 2 name>>}\par}
% \vspace*{0.5em}
% \noindent{{\large\emph{<<author 2 affiliation>>}, \href{mailto:<<author 2 email>>}{<<author 2 email>>}\par}}

% \vspace*{0.75em} % third author

% \noindent{\LARGE{<<author 3 name>>}\par}
% \vspace*{0.5em}
% \noindent{{\large\emph{<<author 3 affiliation>>}, \href{mailto:<<author 3 email>>}{<<author 3 email>>}\par}}

\end{fullwidth}

\vspace*{1em}


\hypertarget{resumen}{%
\section{Resumen}\label{resumen}}

El objetivo de este artículo es analizar el curso histórico de los
estudios sobre comunicación indígena en Argentina dentro del contexto
amplio de Latinoamérica. Para ello, indaga en los conceptos,
instituciones y biografías de los actores involucrados en el proceso de
emergencia e incipiente institucionalización de un campo analítico
propio desde principios de los 2000 hasta la actualidad. Busca
caracterizar las dimensiones analíticas y destacar la incidencia del
contexto socio-político en el nacimiento del interés por investigar este
fenómeno. Por último, señala el carácter interdisciplinario, emergente y
liminal de esta arena epistémica. El texto se apoya en el relevamiento y
análisis de la producción bibliográfica generada y en la consulta a
investigadores pioneros.

\hypertarget{abstract}{%
\section{Abstract}\label{abstract}}

The aim of this article is to analyze the historical course of
Indigenous communication studies in Argentina within the broader context
of Latin America. To this end, it explores the concepts, institutions
and biographies of the actors involved in the process of emergence and
incipient institutionalization of an analytical field of its own from


\vspace*{3em}

\noindent{\emph{History of Media Studies}, vol. 5, 2025}

\pagebreak\noindent the early 2000s to the present day. It seeks to characterize the
analytical dimensions and to highlight the incidence of the
socio-political context in the birth of interest in researching this
phenomenon. Finally, it points out the interdisciplinary, emergent and
liminal character of this incipient body of studies. The text is based
on the survey and analysis of the bibliographical production generated
and consultation with pioneering researchers.

\enlargethispage{2\baselineskip}


\vspace{.1in}

\hypertarget{introduccin}{%
\section{Introducción}\label{introduccin}}

Este texto se propone identificar qué perspectivas analíticas se
visualizan en torno a la configuración histórica de los estudios sobre
comunicación indígena en Argentina.\footnote{Liliana Lizondo y Magdalena
  Doyle, ed. \emph{Pueblos Indígenas y territorios mediáticos. Estudios
  sobre comunicación indígena en Argentina}. (Bogotá: FES Comunicación,
  2023).} Para ello, se basa en el modelo de «constelación de
disciplinas y campos sociales relacionados»\footnote{Maria Löblich y
  Andreas Matthias Scheu, «Writing the history of communication studies:
  A sociology of science approach», \emph{Communication Theory}, n. º 21
  (2011), 6.} propuesto para el análisis del curso histórico de los
estudios de comunicación. El artículo se organiza del siguiente modo: en
primer lugar, aborda los contextos, enfoques y diferentes perspectivas
disciplinares que marcaron la constitución del campo de análisis sobre
comunicación indígena en Latinoamérica, prestando especial atención a
los conceptos elaborados por los propios pueblos originarios\footnote{En
  este texto consideramos a la palabra \textless pueblo\textgreater{}
  como sustantivo colectivo, seguida de los adjetivos
  \textless indígena\textgreater{} u \textless originario\textgreater.
  Nos basamos en el glosario de la guía de \textless{} Recomendaciones
  para el tratamiento mediático sobre pueblos indígenas\textgreater{}
  elaborada por el Instituto Nacional de Asuntos Indígenas y la
  Defensoría del Público de Argentina. Si bien se reconocen diferencias
  entre ambas denominaciones, se sugiere el uso de las dos categorías.
  Asimismo, empleamos las minúsculas para referir a ambos conceptos,
  apoyándonos en las reglas de escritura de los sustantivos colectivos
  en español.}. En segundo lugar, presenta un recorrido temporal sobre
los momentos que determinaron el surgimiento de las primeras
investigaciones sobre expresiones mediáticas de pueblos indígenas en
Argentina. A continuación, analiza las ideas/conceptos, métodos,
instituciones y actuaciones profesionales de los actores que
intervinieron en la formación de un campo de interés común sobre el
fenómeno de los medios indígenas, hasta llegar al planteo de
institucionalización. Por último, recupera las discusiones y
anudamientos conceptuales que prevalecen en el estudio de este fenómeno.


\hypertarget{metodologa}{%
\section{Metodología}\label{metodologa}}

Como hemos dicho, para el desarrollo de esta investigación nos apoyamos
en el modelo de «constelación de disciplinas y campos sociales
relacionados», el cual sugiere un formato de análisis aplicable al curso
histórico de los estudios de comunicación. Esta propuesta es definida
como «un marco analítico y heurístico que consiste en un sistema de
categorías»,\footnote{Löblich y Scheu, «Writing the history of
  communication studies», 7.} entre ellas las ideas o conceptos,
instituciones y biografías de los investigadores. Dichas categorías «no
solo ayudan a estructurar el tema de importancia, sino que también
orientan todo el proceso de investigación al igual que la producción y
elección de fuentes»\footnote{Íbid.}. La revisión de cada una de estas
dimensiones permite una aproximación al entendimiento de la complejidad
de las


 \end{titlepage}

% \vspace*{2em} | to use if abstract spills over



\noindent  interacciones y debates que van modelando la formación de campos
de conocimiento, en el sentido bourdieano\footnote{Pierre Bourdieu, «El
  campo científico», \emph{Redes: revista de estudios sociales de la
  ciencia} 1, n.º 2 (1994).} de lugares de lucha y despliegue de
estrategias, posiciones y relaciones de fuerzas entre diferentes agentes
en torno a la conservación o subversión de la autoridad científica.


Para la puesta en práctica de este modelo nos basamos en la consulta y
examinación de fuentes bibliográficas (artículos publicados en revistas
académicas, ponencias presentadas a congresos, artículos de libros,
tesis de posgrado y tesinas de grado) generadas por los docentes e
investigadores que fueron aportando al abordaje del tema en cuestión,
como así también en la realización de entrevistas a investigadores de
las universidades nacionales de Salta y Jujuy, quienes se dedican al
estudio del fenómeno de los medios de comunicación indígenas en la
Argentina. Estas comunicaciones fueron hechas de modo virtual y
presencial, entre finales de 2024 y los primeros meses de 2025. Con las
entrevistas quisimos conocer la mirada de actores universitarios
pioneros en la aproximación a experiencias mediáticas indígenas, a fin
de identificar la participación de las universidades, y también de
comprender desde cuándo y cómo dichas experiencias se convirtieron en
objetos de interés para el estudio académico.

Con este artículo no pretendemos ofrecer una lectura cerrada o acabada
sobre la historia de los estudios de comunicación indígena en Argentina;
simplemente queremos poner de relieve algunas consideraciones sobre los
procesos y aportes que confluyen en la definición de un naciente campo
académico propio.

\hypertarget{caracterizacin-de-la-comunicacin-indgena-en-amrica-latina-y-perspectivas-analticas}{%
\section{Caracterización de la comunicación indígena en América
Latina\\\noindent y perspectivas
analíticas}\label{caracterizacin-de-la-comunicacin-indgena-en-amrica-latina-y-perspectivas-analticas}}

Tal como señalan Arcila-Calderón, Barranquero y González
Tanco,\footnote{Carlos Arcila Calderón, Alejandro Barranquero y Eva
  González Tanco, ``From media to buen vivir: Latin American approaches
  to Indigenous Communication'', \emph{Communication Theory}, n.º 28
  (2018), 181.} para abordar la temática «comunicación indígena» en
Latinoamérica es necesario advertir que se trata de una categoría «aún
en construcción» que se presenta como «resbaladiza y
homogeneizadora»,\footnote{Alejandro Barranquero-Carretero y Eva
  González-Tanco, «Editorial», \emph{Anuario Electrónico de Estudios en
  Comunicación Social Disertaciones} 11, n.º 2 (2018).} rasgos que
dificultan la posibilidad de ofrecer certezas, si es que ello fuera
posible, sobre un tópico de interés que todavía no alcanzamos a
dimensionar. Pese a ello, dentro de la Escuela Latinoamericana de
Comunicación\footnote{José Marques de Melo, \emph{Pensamiento
  comunicacional latinoamericano. Entre el saber y el poder} (Sevilla:
  Comunicación Social Ediciones y Publicaciones, 2009).} se han generado
esfuerzos por abordar el fenómeno de la comunicación de los pueblos
indígenas a partir de esquemas de análisis que dilucidan aspectos poco
conocidos, como los marcos históricos de surgimiento de este tipo de
experiencias, las matrices epistemológicas predominantes en la
generación de teoría sobre estas referencias empíricas, y contribuciones
que permiten visualizar un emergente campo de estudio.

\pagebreak La dificultad para categorizar un fenómeno tan complejo responde a la
diversidad de atravesamientos culturales, étnicos e históricos que se
emplazan en las diferentes formas de comunicación que ponen en prácticas
los sujetos políticos «pueblos indígenas» en Latinoamérica. Dos factores
resultan sobresalientes: la trama de continuidades y discontinuidades
socio-culturales, territoriales y económicas que acarreó la conquista y
colonización de los territorios que actualmente se inscriben en
diferentes estados nacionales o plurinacionales; y la incorporación de
nuevas tecnologías, entre ellas, los medios de comunicación modernos
---tales como la radio, la televisión, los vídeos y, más recientemente,
el fenómeno de internet--- en las existencias cotidianas de los
innumerables espacios que habitan los pueblos indígenas de esta parte
del continente americano.

Por lo tanto, cuando se quiere abordar la comunicación de los pueblos
indígenas es imposible soslayar su larga historia, tanto para
referenciar al tiempo pre-hispánico, es decir, a la comunicación antes
de Colón tal como la definen Beltrán-Salmón et al.;\footnote{Luis Ramiro
  Beltrán-Salmón et al., \emph{La comunicación antes de Colón. Tipos y
  formas en Mesoamérica y Los Andes} (La Paz: Centro Interdisciplinario
  Boliviano de Estudios de la Comunicación, 2008).} como para dar cuenta
de las imbricaciones, desplazamientos y nuevos modos de relacionamiento
generados a partir de la adopción de tecnologías derivadas de Occidente.
Todo esto se traduce en que aquello a lo que llamamos comunicación
indígena aluda a un conjunto heterogéneo de prácticas, sensibilidades,
posibilidades de conexión sensorial y espiritual entre humanos y no
humanos que trasciende a la tecnología moderna como esquema de
definición. En consecuencia, es preciso realizar una distinción entre
comunicación indígena mediática o aquella que está atravesada por alguna
forma de mediación tecnológica moderna (radio, televisión, plataformas
digitales, entre otras) y comunicación indígena integral, que incluye
aspectos relativos a la acción de comunicarse entre grupos humanos y
también a la puesta en relación con agencias espirituales, que no
necesariamente están mediadas por un dispositivo tecnológico.

En América Latina, la participación de integrantes de pueblos
originarios en la gestión de medios de comunicación se remonta al
nacimiento de la oleada de experiencias de comunicación alternativa,
popular, educativa, comunitaria que data de finales de 1940\footnote{Doyle
  (2015) revela que las primeras experiencias de participación indígena
  en medios de comunicación datan de 1950, algunas de las primeras
  experiencias son: radio San Gabriel de Bolivia (1955); radio Onda Azul
  de Perú (1958), radio Pío XII de Bolivia (1959), entre otras.}. Si
bien la participación indígena en estos procesos está documentada, por
ejemplo, en el caso de las radios mineras sindicales de Bolivia durante
el período 1940--1950,\footnote{Karina Herrera Miller, \emph{¿Del grito
  pionero...al silencio? Las radios sindicales mineras en la Bolivia de
  hoy} (La Paz: Plural Editores, 2006).} en aquel momento las
identificaciones étnicas quedaban al margen o eran eclipsadas por el
predominio de otros factores, como los de clase o las reivindicaciones
de sectores rurales-campesinos.\footnote{Algunas de las radios pioneras
  en Latinoamérica son: en Bolivia, radio Mallku Quiririyá (1990,
  Potosí); Ondas del Titicaca (1993); en Chile, la agencia de noticias
  Prensa Indígena (1994), el boletín de noticias Mapuexpress y el portal
  de internet Colectivo Lientur de Contrainformación (2000) que luego se
  transformó en el periódico Azkintuwe.}

Este solapamiento comenzó a cambiar recién durante las décadas de 1980 y
1990 con la aparición de las primeras experiencias de medios
auto-reconocidas como indígenas. En el plano regional, esta
transformación no fue fortuita, más bien fue resultado del impacto del
proceso definido como de «emergencia étnica»,\footnote{José Bengoa,
  \emph{La emergencia indígena en América Latina} (Santiago de Chile:
  Fondo de Cultura Económica, 2000).} que alude al camino progresivo por
el cual los pueblos indígenas alcanzaron nuevas condiciones para el
reconocimiento de derechos específicos en los sistemas nacionales e
internacionales desde 1980 en adelante, desplazamiento que también se
tradujo en la adquisición de herramientas para el ejercicio de
modalidades de ciudadanía en clave étnica. Este momento,
paradójicamente, coincidió con la consolidación del neoliberalismo como
sistema político-económico dominante en los estados
latinoamericanos.\footnote{Fabiola Escárzaga, «La emergencia indígena
  contra el neoliberalismo», \emph{Política y Cultura}, n.º 22 (2004).}

En torno a este nuevo escenario, durante la primera década de los 2000,
en el contexto de Abya Yala, los pueblos indígenas articularon sus
reclamos colectivos de larga data con la demanda por el reconocimiento
de una comunicación propia. Fueron representantes directos de los
pueblos originarios quienes elaboraron los primeros conceptos sobre
comunicación indígena, entre ellos, podemos mencionar al manifiesto
impulsado en 2003 por el Caucus Indígena del Sur, Centro y México, el
cual expresa:

\begin{quote}
Los Pueblos Indígenas reafirmamos nuestras propias formas e instrumentos
de comunicación tradicional, como válidos para una comunicación humana
que enriquezca la vida social, con equilibrio y respeto entre los seres
humanos y la Madre Naturaleza. La comunicación es una práctica social
cotidiana y milenaria de los pueblos Indígenas que es fundamental para
la convivencia armónica entre los seres humanos y la naturaleza. Para
los pueblos Indígenas la comunicación es integral pues parte de una
cosmovisión en la cual todos los elementos de la vida y la naturaleza se
hallan permanente relacionados e influidos entre sí. Por esta razón la
comunicación tiene como fundamento una ética y una espiritualidad en el
que los contenidos, los sentimientos y los valores son esenciales en la
comunicación.\footnote{Caucus Indígena del Sur, Centro y México,
  «Declaración de los Pueblos Indígenas ante la Cumbre Mundial de la
  Sociedad de la Información». \emph{América Latina en movimiento}, 12
  de diciembre de 2003.}
\end{quote}

\noindent Con el mismo ánimo, la comunicación indígena también ha sido
conceptualizada como una práctica:

\begin{quote}
Constitutiva del sujeto social indígena y de sus expresiones simbólicas,
tejido nervioso de la trama o tejido que le da identidad sociocultural
{[}incluyendo{]} innumerables expresiones como el habla, la vestimenta,
los tejidos, los rituales que requieren ser más estudiados de manera
integral y no sólo a partir del enfoque mediático o tecnológico
moderno.\footnote{Jorge Agurto y Jahve Mescco, «La comunicación indígena
  como dinamizadora de la comunicación para el cambio social». En
  \emph{Actas del XI Congreso Latinoamericano de Investigadores en
  Comunicación} (Montevideo: ALAIC, 2012), 4.}
\end{quote}

\noindent Considerando la importancia de proclamar una noción integral de
comunicación que no remita solo a los medios de comunicación de origen
occidental, los pueblos indígenas de Colombia han distinguido dos
conceptos: una comunicación propia y una comunicación apropiada.
Mientras que la primera refiere a los:

\begin{quote}
Principios consuetudinarios que están basados en la Ley de Origen o
Derecho Mayor de los Pueblos Indígenas, y a prácticas tradicionalmente
orales que involucran no solamente la relación entre seres humanos sino
también entre estos y todos los demás seres y elementos de la naturaleza
(materiales y espirituales).\footnote{Comisión Nacional de Comunicación
  de los Pueblos Indígena, \emph{Política pública de comunicación de y
  para los pueblos indígenas} (Bogotá: CONCIP, 2018), 8.}
\end{quote}

\noindent La comunicación apropiada alude a los «dispositivos tecnológicos y
lenguajes provenientes de las sociedades no indígenas como la radio, la
televisión, el cine, la internet, la telefonía, la fotografía, la
prensa, entre otros medios».\footnote{Comisión Nacional de Comunicación
  de los Pueblos Indígenas, \emph{Política pública de comunicación de y
  para los pueblos indígenas,} 8\emph{.}} En estrecha conexión con este
contraste, Agurto y Messco destacan la importancia de considerar a la
comunicación de los pueblos indígenas no solo como experiencias
participativas o no lucrativas. Por esta razón, proponen comprender a la
comunicación indígena como expresión de:

\begin{quote}
La visión, demandas y propuestas de los pueblos indígenas, cuyo
protagonismo social y político se ha incrementado en los últimos veinte
años. Hablamos de una comunicación que responde a situaciones de
exclusión, discriminación y hasta exterminio del sujeto histórico
«pueblo indígena» que lucha por sobrevivir y conservar sus identidades
en un mundo dominado por la globalización.\footnote{Agurto y Mescco,
  \emph{La comunicación indígena como dinamizadora,} 1.}
\end{quote}

\noindent Habiendo presentado algunos acuerdos en cuanto al entendimiento de la
comunicación indígena como una práctica integral, ancestral,
extra-mediática o extra-tecnologías de la información modernas, en este
artículo nos centraremos exclusivamente en las expresiones de
comunicación mediática de los pueblos originarios o en los medios de
comunicación indígenas. Dicho esto, en el contexto latinoamericano, los
procesos de vinculación y adopción de instancias de acceso y uso de
medios de comunicación tradicionales o digitales por parte de pueblos
originarios han sido analizados desde distintas disciplinas, entre las
que se destacan la comunicación y la antropología.

En primer lugar, en cuanto a los marcos teóricos utilizados dentro del
campo disciplinar de la comunicación, sobresalen herramientas analíticas
derivadas de los estudios de comunicación alternativa, popular y
comunitaria. Estos últimos han funcionado como andamiaje para el
abordaje de medios indígenas que se definen también como comunitarios,
permitiendo su comprensión como experiencias que persiguen fines de
transformación y/o cambio social. Algunos análisis destacan la
innovación o dinamismo que incorporan los medios indígenas dentro de la
tradición de comunicación alternativa en Latinoamérica, subrayando, por
ejemplo, cómo ciertos aportes, entre ellos el concepto del «buen
vivir»,\footnote{Adalid Contreras-Baspineiro,
  «Aruskipasipxañanakasakipunirakispawa: We must communicate each other,
  despite the differences, and solve them through communication»,
  \emph{Razón y Palabra}, n. º 93 (2016).} alientan nuevos enfoques con
que mirar y poner en práctica la comunicación indígena y la comunicación
para el cambio social.\footnote{Alejandro Barranquero-Carretero y Chiara
  Sáez-Baeza, «Comunicación y buen vivir. La crítica descolonial y
  ecológica a la comunicación para el desarrollo y el cambio social»,
  \emph{Palabra Clave 18}, n.º 1 (2015).}

Una investigación que revisa la emergencia de los estudios sobre
comunicación indígena dentro el campo latinoamericano de la comunicación
caracteriza a este grupo de abordajes como un ``emergente tópico de
debates''\footnote{Arcila Calderón, Barranquero y González Tanco,
  \emph{From media to buen vivir}, 181.} que se afianza al ritmo del
desarrollo de los estudios indígenas en general. La formulación de este
marco de análisis académico se fue tejiendo entre la diversidad propia
de la riqueza cultural de cada país y los puntos en común que delimitan
un mapa analítico regional en Latinoamérica. Según revela este texto, el
nacimiento de los primeros estudios sobre comunicación indígena se
produjo durante las primeras décadas del siglo XXI y su estructura gira
en torno a los siguientes cuatro debates: el fenómeno de la apropiación
de tecnologías de la información por parte de pueblos originarios; la
regulación y elaboración de políticas de comunicación; la relación con
la comunicación alternativa; y la incorporación de temáticas como el
«buen vivir». Agregamos a este ordenamiento la creciente colocación de
los estudios decoloniales en los corpus teóricos de investigaciones
sobre comunicación indígena en la academia latinoamericana.\footnote{Eliana
  Herrera Huérfano, Francisco Sierra Caballero y Carlos Del Valle Rojas,
  «Hacia una epistemología del sur. Decolonialidad del saber/poder
  informativo y nueva comunicología latinoamericana. Una lectura crítica
  de la mediación desde las culturas indígenas», \emph{Chasqui. Revista
  Latinoamericana de Comunicación}, n.º 131 (2016).}

Con un interés similar, Doyle\footnote{Magdalena Doyle, «Los medios
  masivos de comunicación en las luchas de los pueblos indígenas.
  Abordaje desde los estudios sobre comunicación en América Latina»
  (tesis de maestría, Universidad Nacional de Córdoba, 2013); Magdalena
  Doyle, «Matrices y vertientes de pensamiento sobre los medios
  indígenas en América Latina», \emph{History of Media Studies} 2
  (2022).} precisa que recién desde de la primera década de los 2000 se
observa la enunciación de estudios sistemáticos sobre comunicación
indígena en Latinoamérica. Esta autora propone tres etapas para
clasificar y organizar las trayectorias de investigación que dieron
forma a un campo de tematización académica sobre comunicación indígena:
el primero, que comprende el período entre 1970 hasta mediados de 1980,
atravesado por las tensiones con los marcos analíticos de la
comunicación alternativa, popular y comunitaria, y por el contrapunto
entre las teorías de la dependencia y el modelo desarrollista. El
segundo, a partir de 1985 y la primera década de los 90, momento bisagra
para la definición posterior, cuando se visualizan las primeras
investigaciones enfocadas en reflexionar sobre las particularidades
culturales y étnicas de los incipientes medios de comunicación
indígenas. Y, finalmente, una tercera etapa entre finales de los 90
hasta 2022 aproximadamente, en la que se observan matrices definidas de
estudios focalizados en, por un lado, los usos de las tecnologías de la
comunicación modernas por parte de pueblos y comunidades indígenas, y,
por otro, en la indagación de la relación entre comunicación e
identidades étnicas en clave de negociación, construcción y
performatividad dentro de contextos de mediatización. En este último
período también se introdujeron pautas metodológicas centrales para el
análisis académico, como la pertinencia de la perspectiva de los actores
y la importancia de considerar las narrativas producidas directamente
por los sujetos que integran las experiencias mediáticas. Por ejemplo,
un estudio cuyo alcance se extiende a los medios indígenas de
Latinoamérica, da cuenta del rol mediador estratégico de las tecnologías
de la comunicación modernas en la producción y afianzamiento de los
discursos de auto-determinación política y reconocimiento
étnico-cultural entre el último cuarto del siglo XX y el primer decenio
del XXI.\footnote{Juan Francisco Salazar, «Activismo indígena en América
  Latina: estrategias para una construcción cultural de las tecnologías
  de información y comunicación», \emph{Journal of Iberian and Latin
  American Studies} 8, n.º 2 (2002).}

En cuanto a las perspectivas teóricas preponderantes a lo largo de la
historia de los estudios sobre comunicación indígena en América Latina,
Doyle, Siares y Belloti\footnote{Magdalena Doyle, Emilse Siares y
  Francesca Belloti, «Los medios de pueblos originarios en América
  Latina: historia, aproximaciones y desafíos». En \emph{Hacia un
  periodismo indígena}, comp. por Damián Andrada (Buenos Aires:
  Universidad del Salvador, 2019), 203-218.} destacan tres dimensiones:
el determinismo tecnológico ---el más frecuente--- de acuerdo al cual el
acceso de los pueblos indígenas a las tecnologías de la información y
comunicación supone condiciones de empoderamiento que impactan
directamente en el mejoramiento de sus condiciones de vida; el
instrumentalismo tecnológico y eventualmente difusionismo, que concibe
al medio en función de su utilidad y eficacia; y, finalmente, en
coincidencia con los trabajos precursores que marcaron la formación de
una arena temática específica, se observa una perspectiva académica
dispuesta a indagar en «los procesos de configuración de identidades en
el espacio público mediatizado».\footnote{Doyle, Siares y Belloti,
  \emph{Los medios de pueblos originarios en América Latina}, 211.}

Hasta ahora, hemos mostrado que, desde la segunda mitad de los 90 hasta
la actualidad, los estudios de comunicación indígena han disputado y
ganado lugar en el corpus de investigaciones del campo latinoamericano
de la comunicación. Movimiento que se hizo en paralelo a los
levantamientos y grandes luchas encabezadas por los pueblos indígenas de
los diferentes países de Latinoamérica, y en el escenario de
consolidación de virajes trascendentales para el campo de la
comunicación, como lo fueron, entre las décadas de los 80 y 90, los
desplazamientos hacia el reconocimiento de la centralidad de la
mediación de la cultura en los procesos comunicacionales. En parte,
gracias a esta herencia, es que la comunicación se perfiló tempranamente
como una de las vías analíticas para explicar el fenómeno del uso de
medios de comunicación en pueblos indígenas.

En segundo lugar, otra de las disciplinas que se interesa por el estudio
de los medios indígenas es la antropología. Bajo el nombre de
``antropología de medios'' se constituyó un sub-campo enfocado en el
abordaje de las relaciones sociales y experiencias que se generan
alrededor de los medios de comunicación. Dentro de este grupo de
investigaciones, la etnografía se posicionó como una instancia
metodológica privilegiada para ampliar el foco de comprensión de los
medios de comunicación masivos, permitiendo identificar y observar:

\begin{quote}
No solo cómo los medios están impregnados en la vida cotidiana de las
sociedades sino también cómo consumidores y productores están imbricados
en universos discursivos, situaciones políticas, circunstancias
económicas, marcos nacionales, momentos históricos y flujos
transnacionales.\footnote{Faye D. Ginsburg, Lila Abu-Lughod y Brian
  Larkin, \emph{Media worlds: Anthropology on new terrain} (Berkeley:
  University of California Press, 2002), 2.}
\end{quote}

\noindent Aunque la «antropología de medios» sea en la actualidad una arena
analítica con mayor grado de delimitación que en la primera década del
siglo XXI, algunos debates generados sobre el lugar que esta disciplina
ocupa en el marco de un escenario atravesado y compartido por las
ciencias de la comunicación son oportunos para poner de relieve la trama
plural y sinérgica que conforma el incipiente conjunto de estudios sobre
medios indígenas que analizamos en este artículo. Esto es trabajado por
la genealogía crítica de Askew,\footnote{Kelly Askew, «Introduction». En
  \emph{The anthropology of media: a reader}, ed. por Kelly Askew y
  Richard Wilk (Malden: Blackwell Publishers, 2002), 1-13.} quien
identifica cuatro formas de entender a los medios desde la antropología:
como textos mediáticos; tecnologías; contextos; o mediadores culturales.
A su vez, estas categorías se corresponden con tradiciones epistémicas
diferentes. Así, mientras los abordajes iniciales concebían a las
tecnologías, en especial a las visuales, como textos, o sea como volumen
de significados, desde las perspectivas posteriores, influenciadas por
el giro interpretativista, los medios pasaron a ser analizados bajo el
enfoque de la mediación cultural, por lo que las audiencias y públicos
fueron visualizadas como agentes productores de sentidos situados en
flujos de relaciones de poder. Finalmente, de acuerdo a esta mirada, la
pertinencia de la antropología para tomar a los medios como objeto de
estudio radica en el entendimiento de éstos como aspectos de la vida
cotidiana,\footnote{Lila Abu-Lughod, «Interpretando la(s) cultura(s)
  después de la televisión: sobre el método», \emph{Íconos}, n.º 24
  (2006).} no demasiado diferentes a la ley, la economía o la religión,
lo que en sus palabras se traduce en el rechazo a la «tendencia común de
tratar al medio como separado de la vida social».\footnote{Askew,
  «Introduction», 10.}

\enlargethispage{\baselineskip}

A raíz de la proximidad disciplinar que la antropología mantiene con
poblaciones no occidentales, uno de los dominios más extendidos de este
subcampo ha sido el de los medios indígenas o, más precisamente, el
análisis de los procesos de apropiación e impacto de artefactos como
radios, cámaras fotográficas, videos e internet en las dinámicas
socio-culturales de numerosas poblaciones indígenas a lo largo del
mundo. Pese a esta cercanía, el estudio de las mediaciones tecnológicas
en las sociedades indígenas es aún considerado «un lugar secundario en
las etnografías y en los estudios antropológicos sobre estos
grupos».\footnote{Gemma Orobitg, «Antropología de los medios de
  comunicación indígenas y afro en América Latina: una presentación»,
  \emph{Disparidades. Revista de Antropología} 76, n.º 2 (2021): 2.}
Trabajos como los de Ginsburg,\footnote{Faye D. Ginsburg, «Screen
  memories: Resignifying the traditional in Indigenous Media». En
  \emph{Media worlds. Anthropology on new terrain}, ed. por Faye D.
  Ginsburg, Lila Abu-Lughod y Brian Larkin (Berkeley: University of
  California Press, 2002), 39-57.} Turner\footnote{Terence Turner,
  «Representation, politics, and cultural imagination in Indigenous
  video. General points and Kayapo examples». En \emph{Media worlds.
  Anthropology on new terrain}, ed. por Faye D. Ginsburg, Lila
  Abu-Lughod y Brian Larkin (Berkeley: University of California Press,
  2002), 75-89.} y Prins\footnote{Harald E. L. Prins, «Visual
  anthropology». En \emph{A companion to the anthropology of American
  Indians}, ed. por Thomas Biolsi (Malden: Blackwell Publishing, 2004),
  506-525.} analizan los procesos de mediación cultural que se
despliegan a partir de la incorporación y uso de tecnologías visuales,
como el video, en sociedades indígenas. Tales indagaciones dan cuenta de
la multiplicidad de agencias que se movilizan en torno a la apropiación
de estos artefactos, mostrando, al igual que vimos en los estudios de
comunicación, la importancia de centrar los abordajes en los procesos de
mediación en que ocurren los usos de las tecnologías, desplazando así la
centralidad asignada al término «medio». Estos análisis, además de
introducir pautas metodológicas, muestran cómo, a través de la mediación
cultural, los sujetos indígenas generan y recrean activamente instancias
de representación, re-significación de memorias, resistencias y
problematización de sus realidades e identidades en la cotidianeidad.

Más cercano a nuestro interés, en Latinoamérica, el campo de estudios
sobre comunicación indígena también recibió la influencia de
contribuciones provenientes de la antropología. Sin pretender dar una
mirada acabada, un conjunto de estudios se enfocó en el trabajo de campo
con experiencias comunicacionales mediáticas lideradas por pueblos o
comunidades indígenas,\footnote{Oscar Grillo, «Políticas de identidad en
  internet. Mapuexpress: imaginario activista y procesos de
  hibridación», \emph{Razón y Palabra}, n.º 54 (2006); Diego Mauricio
  Cortés, «Radio indígenas y Estado en Colombia ¿Herramientas
  ``políticas'' o instrumentos ``policivos''?», \emph{Chasqui. Revista
  Latinoamericana de Comunicación}, n.º 140 (2019); Gemma Orobitg,
  «Etnografía de los medios de comunicación indígenas y afroamericanos:
  propuestas metodológicas», \emph{Revista Española de Antropología
  Americana}, n.º 50 (2020); Gemma Orobitg y Roger Canals, «Hipertexto,
  multivocalidad y multimodalidad para una etnografía sobre los medios
  de comunicación: la web MEDIOS INDÍGENAS», \emph{Revista Española de
  Antropología Americana}, n.º 50 (2020); Francisco Gil-García, Beatriz
  Pérez Galán y Pedro Pitarch, «Fragmentos para una etnografía de las
  radios comunitarias en América Latina», \emph{Disparidades. Revista de
  Antropología} 76, n.º 2 (2021).} dando cabida a la emergencia de un
agrupamiento conocido como «etnografía de medios indígenas».\footnote{Gemma
  Orobitg et al., «Los medios indígenas en América Latina: Usos,
  sentidos y cartografías de una experiencia plural», \emph{Revista de
  Historia}, n.º 83 (2021); Orobitg, «Antropología de los medios de
  comunicación indígenas».} En ellos sobresalen los aportes que la
perspectiva etnográfica añade al universo de investigaciones que abordan
fenómenos mediáticos de pueblos indígenas, especialmente, impulsando una
crítica al énfasis puesto en el mensaje al momento de analizar estas
instancias de mediatización. Precisamente, la etnografía viene a
proponer un enfoque centrado en la acción de los sujetos sobre las
tecnologías de la información modernas, buscando identificar qué
relaciones, conexiones, usos y prácticas son habilitadas por este nexo
en las formas de vida de cada población.

A modo de resumen, presentamos las facetas o dimensiones que delinean de
manera general los estudios de comunicación indígena en Latinoamérica.
Primero, buscamos evidenciar las disciplinas que alimentan esta arena
académica, destacando a la antropología y la comunicación. No alcanzamos
a dar cuenta de la diversidad de trabajos que enriquecen este tema pues
semejante tarea excede las finalidades de este artículo, simplemente
quisimos esbozar los puntos de partida que cimentaron la aparición de
investigaciones afines a este fenómeno, y su impacto en la modelación de
un campo de estudios propio en esta región del continente americano. A
su vez, tratamos de dar cuenta de los principales anudamientos teóricos
que se visualizan en el estudio de la comunicación indígena.

\hypertarget{breve-repaso-histrico-por-los-plazos-de-la-comunicacin-indgena-en-argentina}{%
\section{Breve repaso histórico por los plazos de la comunicación\\\noindent indígena en
Argentina}\label{breve-repaso-histrico-por-los-plazos-de-la-comunicacin-indgena-en-argentina}}

Para ubicar el contexto de surgimiento de medios de comunicación de
carácter indígena en la Argentina, ofrecemos algunos datos sobre el
panorama étnico de este país. De acuerdo al último censo nacional
realizado en el año 2022, un 2,9 \% de población (equivalente a
1.306.730 personas) se reconoce como indígena o descendiente de pueblos
originarios, sobre un total de 45.892.285 personas, correspondiente a la
población total;\footnote{Instituto Nacional de Estadística y Censos,
  Censo nacional de población, hogares y viviendas 2022\emph{.
  Resultados definitivos. Población indígena o descendiente de pueblos
  indígenas u originarios} (Buenos Aires: Instituto Nacional de
  Estadística y Censos, 2024), 9.} lo que refleja un aumento con
respecto al censo del año 2010, en el que la población indígena
representaba un 2,4 \%.\footnote{Instituto Nacional de Estadística y
  Censos, Censo Nacional de Población, Hogares y Viviendas 2010. Censo
  del Bicentenario\emph{. Pueblos originarios} (Ciudad Autónoma de
  Buenos Aires: Instituto Nacional de Estadística y Censos, 2015), 8.}
El grupo étnico con mayor cantidad de población es el Mapuche, con
145.783 personas, seguido por el pueblo Guaraní con 135.232 integrantes.
En tercer y cuarto lugar se encuentran los pueblos Diaguita y Qom/Toba
con 86.022 y 80.124 habitantes respectivamente. Las provincias con mayor
porcentaje de pueblos indígenas son Jujuy (10,1 \%), Salta (10 \%),
Chubut (7,9 \%), Formosa (7,8 \%), Neuquén (7,7 \%) y Río Negro (6,4
\%). Esto indica que los extremos norte y sur de la Argentina son las
regiones más densamente habitadas por pueblos originarios a nivel
nacional.\footnote{Instituto Nacional de Estadística y Censos,
  \emph{Censo 2022}, 11.} Según el mismo censo, se reconoce la presencia
de 58 pueblos indígenas a lo largo del país.

La actual taxonomía étnica de la Argentina es resultado de los procesos
geográficos e históricos que cimentaron el nacimiento del Estado. Entre
mediados del siglo XIX y principios del XX, el proyecto de modelación
del Estado nacional argentino, fundado sobre pilares de blanquitud y
herencia europea, visualizó a los indígenas desde matrices ideológicas
como la oposición «civilización vs. barbarie», y la apelación al
desierto para referir a sus territorios, ambas estrategias discursivas
para justificar su exterminio.\footnote{Diana Lenton, «Política
  indigenista argentina: una construcción inconclusa», \emph{Anuário
  Antropológico}, n.º 35 (2010).} Hasta la actualidad, a través de
numerosas dinámicas, los sujetos étnicos de la Argentina han sido
alcanzados por construcciones de aboriginalidad diversas y situadas, es
decir, por instancias de alterización y auto-marcación identitaria en
cada jurisdicción, en torno a las cuales afirmaron su posición en el
presente.\footnote{Claudia Briones, «Construcciones de aboriginalidad en
  Argentina», \emph{Société suisse des Américanistes}, n.º 68 (2004).}
Gordillo y Hirsch\footnote{Gastón Gordillo y Silvia Hirsch, «La
  presencia ausente: invisibilizaciones, políticas estatales y
  emergencias indígenas en la Argentina». En \emph{Movilizaciones
  indígenas e identidades en disputa en la Argentina}, comp. por Gastón
  Gordillo y Silvia Hirsch (Buenos Aires: La Crujía, 2010), 15-38.}
reflexionan sobre el carácter aparentemente «ausente» de los pueblos
originarios en la Argentina, a causa de la invisibilización hegemónica a
la que fueron sometidos. Estos autores observan que los sujetos
originarios representan una «presencia-ausente», esto es una «presencia
no-visible, latente y culturalmente constitutiva de formas hegemónicas
de nacionalidad».\footnote{Gordillo y Hirsch, \emph{La presencia
  ausente}, 16.} Estos intentos de borramiento intencionalmente
provocados vienen siendo cuestionados y, en algún punto, revertidos
desde mediados del siglo XX, aunque con mayor fuerza desde 1980 en
adelante.

Para encontrar las primeras experiencias de participación de integrantes
de pueblos originarios en iniciativas mediáticas comunitarias o de
promoción educativa, debemos remitirnos a la década de 1970. Los
precedentes de la comunicación indígena mediática son participaciones
pioneras de integrantes de pueblos originarios en proyectos impulsados
por organismos como el Instituto de Cultura Popular y el Servicio
Pastoral para la Comunicación del Obispado de Neuquén, en el norte y sur
de la Argentina.\footnote{Magdalena Doyle, Mariana Ortega y Liliana
  Lizondo, «Los tiempos largos de la comunicación indígena en argentina.
  Trayectorias nacionales y análisis de un caso», \emph{Contracorrente},
  n.º 17 (2021): 27-52.}

En términos generales, el surgimiento de experiencias mediáticas de
pueblos indígenas en este país tuvo lugar en el contexto internacional
de transición entre los ciclos de emergencia y re-emergencia
étnica,\footnote{José Bengoa, «¿Una segunda etapa de la Emergencia
  Indígena en América Latina?», \emph{Cuadernos de Antropología Social},
  n.º 29 (2009): 7-22.} desde finales de los 90 y la primera década de
los 2000. De forma más específica, en el plano nacional, los pueblos
originarios también fueron alcanzados por el impulso de la oleada
democrática que significó el final de la dictadura cívico-militar en el
año 1983. Estas secuencias gestaron una serie de corrimientos
fundamentales en materia de derechos humanos, como, por ejemplo: la
incorporación dentro de los marcos normativos nacionales de herramientas
de carácter internacional como el convenio 169 sobre Derechos de los
Pueblos Indígenas y Tribales de la Organización Internacional del
Trabajo (OIT), y la inclusión del artículo 75, inciso 17, que reconoce
la preexistencia étnica de los pueblos indígenas a la formación del
Estado argentino en la reforma constitucional de 1994.

En el seno de este contexto, los pueblos indígenas desplegaron
diferentes modalidades de organización para encausar sus procesos
colectivos de lucha, siendo la comunicación uno de los ejes de mayor
relevancia. Entre las primeras iniciativas de medios gestionados por
pueblos originarios en este país se encuentran la FM Comunitaria La Voz
Indígena (2002), y las radios del pueblo Mapuche NewenHueche (2004),
Aletwy Wiñilfe (2005), Wajzugun (2006) y Petu Mongelein (2008).

En sintonía con este proceso, observamos que la primera década de los
2000 marcó el comienzo de un nuevo ciclo en los tiempos largos de la
comunicación indígena en Argentina. Precisamente, fue en el año 2009
---cuando finalmente se sancionó la Ley de Servicios de Comunicación
Audiovisual (en adelante LSCA) tras un intenso camino previo de debates
y movilizaciones por el planteo general de lograr democratizar la
comunicación--- que se generaron por primera vez condiciones favorables
para que pueblos, comunidades y organizaciones indígenas puedan acceder
a medios de comunicación propios. A partir de este momento, que, entre
otros aspectos, destrabó obstáculos legales para allanar el camino hacia
la tenencia y gestión de medios para pueblos originarios, se fue
estructurando progresivamente un conjunto de estudios motivados por la
vocación de análisis de este fenómeno.

La aprobación de la LSCA puede considerarse un momento bisagra en la
historia del acceso de los pueblos indígenas a medios de comunicación.
Durante el transcurso de la segunda década de los 2000 no solo se
multiplicaron los casos de medios de comunicación en manos de pueblos y
comunidades indígenas de todo el país,\footnote{RICCAP, \emph{Informe de
  relevamiento de los servicios de comunicación audiovisual
  comunitarios, populares, alternativos, cooperativos y de pueblos
  originarios en Argentina} (Buenos Aires, RICCAP: 2019).} sino que
también aumentaron notablemente las investigaciones sobre medios
indígenas.

En este escenario, distintas organizaciones y activistas de pueblos
originarios se congregaron en el espacio denominado Coordinadora de
Comunicación Audiovisual Indígena de Argentina (CCAIA) para elaborar un
programa conceptual y modelo de lineamientos orientado a garantizar el
cumplimiento del derecho a la comunicación indígena. Este colectivo
propuso la categoría «Comunicación Con Identidad» (en adelante, CCI)
para nombrar a sus prácticas de comunicación con base en los siguientes
principios:

\begin{quote}
La Comunicación con Identidad se enmarca en un contexto de
revalorización de la identidad originaria, con el uso de las nuevas
tecnologías de la comunicación y la información, proceso que se viene
desarrollando de formas muy diversas en nuestro país y todo el
continente, en un contexto amplio donde la preparación de hombres y
mujeres en materia de la comunicación desde una perspectiva de
desarrollo integral con derecho, contribuye con la mayor pertinencia y
realidad en lo referente a argumentaciones y auto-representaciones
sólidas y legítimas de los propios Pueblos Indígenas.\footnote{Vanina
  Baraldini et al., C\emph{omunicación con identidad: Aportes para la
  construcción del modelo de comunicación Indígena en Argentina} (Buenos
  Aires: INAI, 2011), 9.}
\end{quote}

\noindent Además de brindar precisión conceptual, este modelo recogió la demanda
de articular aspectos definitorios de una «comunicación propia» que
pueda ser distinguida de los demás modos occidentales de hacer
comunicación. En parte, ese es uno de los motivos por el que este
concepto gravita en torno a la identidad étnica, dado que esta categoría
es el componente medular estratégico que habilita la diferenciación y
proyección de fronteras para forjar un sentido de lo propio, y a su vez,
inviste de significados y proyecta los bordes de aquellas singularidades
que engloba la comunicación indígena: «cosmovisiones diferentes a la
occidental {[}\ldots{]} que son horizontales, no lucrativas,
democráticas y explícitamente colectivas {[}\ldots{]} lo que la hace
única e irrepetible».\footnote{Vanina Baraldini et al., «Comunicación
  con identidad». En \emph{Aportes para la construcción del modelo de
  comunicación indígena en Argentina} (Buenos Aires: INAI, 2011), 5.}

Los pueblos indígenas desempeñaron roles activos en la fundamentación de
por qué era necesaria una ley que diera lugar a la pluralización de
voces en los sistemas mediáticos, además de constituir uno de los
sectores que interpeló el proyecto de ley inicial al plantear la
incorporación de los pueblos originarios como sujetos plenos de derecho.
La participación indígena finalmente logró que se debata e incorpore el
reconocimiento de la atribución de los de los pueblos originarios a
tener acceso a sus propios medios de comunicación al margen de los demás
tipos de prestadores (públicos, privados con y sin fines de lucro). De
este modo, la ley reconoce el estatus constitucional de los pueblos
indígenas en tanto sujetos pre-existentes al Estado argentino y los pone
ante el desafío de configurar una tipología específica de comunicación
que no es ni comercial, ni pública, ni comunitaria.

\enlargethispage{\baselineskip}

En general, nos interesa destacar que el interés por investigar el
fenómeno de los medios indígenas en Argentina está directamente
relacionado con la intención de comprender las luchas y demandas en
torno al reconocimiento del derecho a la comunicación. Buscando ordenar
los tiempos en que se fueron generando aportes académicos a lo que hoy
denominamos «estudios sobre comunicación indígena», proponemos dos
períodos con en el base en el relevamiento de textos: el primer decenio
del siglo XXI (2000--2010), y la segunda década de los 2000 en adelante
(2010--2023).

Para el primer período (2000--2010) identificamos un conjunto de
investigaciones enfocadas en el análisis exploratorio de proyectos
comunicacionales pioneros, como es el caso de La Voz Indígena en la
provincia de Salta,\footnote{Liliana Lizondo y Ariel Sandoval, «La voz
  del pueblo indígena». En \emph{Actas del VII Seminario Internacional
  Aprendizaje y Servicio Solidario} (Buenos Aires: Ministerio de
  Educación, ciencia y tecnología de la nación, 2004), 58-60; Leda
  Kantor et al., «La voz indígena». \emph{Margen. Revista de Trabajo
  Social}, n.º 42. (2006); Liliana Lizondo, «¿Extensión o
  aprendizaje-servicio?». En \emph{Aprendizaje-servicio en la educación
  superior. Una mirada analítica desde los protagonistas}, comp. por
  Alba González y Rosalía Montes (Buenos Aires: EUDEBA, 2008), 103-108;
  Fernando Bustamante, «La constitución del sujeto indígena en el Chaco
  Salteño. Disputas simbólicas y estrategias de comunicación en torno al
  desarrollo». En \emph{Luchas y transformaciones sociales en Salta},
  ed. por Víctor Arancibia y Alejandra Cebrelli (Salta: Centro
  Promocional de Investigaciones en Historia y Antropología, 2011),
  127-153.} de una experiencia radiofónica comunitaria con integrantes
del pueblo Wichí en la localidad El Potrillo de la provincia de
Formosa,\footnote{Jorge Huergo, Kevin Morawicki y Lourdes Ferreira, «Los
  medios, las identidades y el espacio de comunicación. Una experiencia
  de radio comunitaria con aborígenes wichí», \emph{Comunicar. Revista
  científica de comunicación y educación}, n.º 26 (2005).} y también de
la Red de Comunicación Indígena de la provincia de Chaco.\footnote{Gabriela
  Sosa y Liliana Lizondo, «Pensando en común II. Pueblos originarios:
  construir la visibilidad». En \emph{Construyendo comunidades:
  reflexiones actuales sobre comunicación comunitaria,} ed. por Área de
  Comunicación Comunitaria de la Universidad Nacional de Entre Ríos
  (Buenos Aires: La Crujía, 2009), 123-126.} El abordaje de estas
experiencias, generadas a partir del diálogo entre investigadores,
docentes y estudiantes de universidades nacionales interesados en
brindar capacitaciones en herramientas de comunicación radiofónica,
anticipa el rol que desempeñaron las universidades en la animación de
estos procesos. Dichos abordajes se concentraron en el rol de la
comunicación en tanto práctica promotora de la expresión de las
necesidades, problemáticas y expectativas de las comunidades indígenas
involucradas en los proyectos comunicacionales; y, al mismo tiempo,
caracterizaron a la radio o a los talleres formativos como modalidades
de incipiente participación política para estos actores. A pesar de
primar el uso de categorías inscriptas en la tradición de comunicación
popular, se observa la formulación de herramientas analíticas iniciales
para pensar la singularidad de estos casos o se advierte la
insuficiencia del marco popular para describir las experiencias
indígenas de gestión de medios de comunicación.

Durante el segundo período (2010--2023) en adelante, que comprende la
aprobación de la LSCA y su impacto posterior, se produjo una importante
expansión de los estudios interesados en las expresiones mediáticas de
los pueblos originarios. En el transcurso de este momento tendrá lugar
la elaboración de las principales categorías y debates que marcarán el
desarrollo del pensamiento sobre el fenómeno de la comunicación indígena
en la Argentina. Nos referimos al concepto de Comunicación con
Identidad, derecho a la comunicación indígena y diferentes reflexiones
sobre los entrecruzamientos y articulaciones generados desde y en torno
a las iniciativas mediáticas.

Recapitulando, hemos presentado dos etapas fundacionales del incipiente
campo de estudios sobre comunicación indígena en la Argentina. El debate
y posterior aprobación de la ley N° 26.522 constituyó un acontecimiento
marcador, ya sea por la participación de los pueblos indígenas en las
movilizaciones que la impulsaron o por la fuerte emergencia de medios
que tuvo lugar con la normativa ya sancionada. De una u otra forma, este
hecho ocupa un lugar central en el eslabón de episodios que terminaron
desencadenando el nacimiento de un campo de análisis específico.
Precisamente, consideramos que es recién a partir de 2009 en adelante
que es posible visualizar un interés de investigación común sobre el
fenómeno de los medios indígenas, por lo que nuestro análisis se
centrará en este período, buscando dar cuenta de lo que entendemos son
las primeras bases del pensamiento sobre manifestaciones mediáticas
indígenas en Argentina.

\hypertarget{modelacin-histrica-de-los-estudios-de-comunicacin-indgena-en-el-caso-argentino-conceptos-instituciones-y-biografas}{%
\section{Modelación histórica de los estudios de comunicación
indígena\\\noindent en el caso argentino: conceptos, instituciones y
biografías}\label{modelacin-histrica-de-los-estudios-de-comunicacin-indgena-en-el-caso-argentino-conceptos-instituciones-y-biografas}}

En este apartado buscaremos poner en práctica el modelo de «constelación
de disciplinas y campos sociales relacionados»\footnote{Maria Löblich y
  Andreas Matthias Scheu, «Writing the history of communication studies:
  A sociology of science approach», \emph{Communication Theory}, n. º 21
  (2011), 6.} aplicado a los estudios de comunicación indígena.
Comenzaremos por dar cuenta de las ideas.

El estudio del fenómeno de los medios de comunicación gestionados por
pueblos indígenas se inscribió, como dijimos previamente, en un contexto
de particular efervescencia política, marcado por el acontecimiento de
movilización y posterior sanción de la LSCA en 2009. Muchas de las ideas
originadas en este momento gravitan en torno al interés por identificar
las implicancias y alcances del derecho a la comunicación indígena.

En este punto, se observa el predominio de dos debates fundamentales:
por un lado, perspectivas que problematizan el status de sujeto público
no estatal alcanzado por los pueblos originarios en el cuerpo de la ley
de comunicación, y su impacto en la habilitación de una serie de
garantías, entre ellas, el derecho a acceder de forma directa a
autorizaciones de tenencia y gestión de medios;\footnote{Anabel
  Manasanch, «Pueblos Originarios: comunicación con identidad»,
  \emph{Revista Tram{[}p{]}as de la comunicación y la cultura}, n.º 69
  (2010); Claudia Villamayor, «La ley de SCA y la visibilización de los
  pueblos originarios». En \emph{Ley 26.522. Hacia un nuevo paradigma en
  comunicación audiovisual}, coord. por Mariana Baranchuk y Javier
  Rodríguez Usé (Buenos Aires: AFSCA-UNLZ, 2011), 131-142; Magdalena
  Doyle, «Debates y demandas indígenas sobre derechos a la comunicación
  en América Latina», \emph{Temas Antropológicos Revista Científica de
  Investigaciones Regionales} 37, n.º 2 (2015); Magdalena Doyle, «El
  derecho a la comunicación de los pueblos originarios. Límites y
  posibilidades de las reivindicaciones indígenas en relación al sistema
  de medios de comunicación en Argentina» (tesis doctoral, Universidad
  de Buenos Aires, 2016); Magdalena Doyle, «El derecho a la comunicación
  con identidad. Aportes de los debates indígenas en Argentina para
  pensar la noción de derecho a la comunicación», \emph{Mediaciones},
  n.º 18 (2017); Magdalena Doyle, «Acceso y participación de los pueblos
  indígenas} y, por otro,
exploraciones que se centran en desentrañar las peculiaridades de la
CCI, esto es, en reconocer los modos de organización, gestión,
sostenibilidad e incidencia socio-política de las experiencias
mediáticas indígenas.\textsuperscript{56}

Dentro del primer grupo, buena parte de las discusiones se concentra en
la puesta en diálogo de dos conceptos: ciudadanía y comunicación, a
efectos de comprender cómo y mediante qué formas las prácticas
comunicacionales ---generadoras de sentidos--- posibilitaron que sujetos
como los pueblos indígenas adquirieran nuevos caminos para encausar sus
activismos y movilizaciones amplias. En otras palabras, el concepto de
ciudadanía aplicado a las interacciones comunicacionales, incluyendo a
las mediaciones, explica la magnitud que presentan las prácticas
públicas de comunicación en los escenarios modernos, y en añadidura,
revela que la construcción de ciudadanía puede pensarse más allá de los
límites asignados al derecho. En definitiva, la articulación entre
comunicación y derecho resultó en la exploración de las diversas
estrategias y prácticas abiertas por los proyectos de gestión y tenencia
de medios para la acción política de los pueblos indígenas en sus
diferentes territorios.

\pagebreak Quizás\marginnote{en el sistema de medios de Argentina», \emph{Anuario
  Electrónico De Estudios En Comunicación Social Disertaciones}
  11\emph{,} n.º 2 (2018); Francesca Belloti, «Los medios de
  comunicación de los pueblos originarios frente a la Ley de Servicios
  de Comunicación Audiovisual: Experiencias, contradicciones, desafíos».
  \emph{Informe de investigación (2016---2017)} (Buenos Aires:
  Universidad Nacional de Quilmes, 2018); María Cecilia Hang, «Pueblos
  originarios y Ley de Servicios de Comunicación Audiovisual. El caso
  Wall Kintun TV», \emph{Tram{[}p{]}as de la comunicación y la cultura},
  n.º 84 (2019); Mariana Ortega, «En primera persona. Ciudadanías
  comunicacionales en la experiencia de la FM Comunitaria La Voz
  Indígena», \emph{Question/Cuestión} 3, n.º 70 (2021).} el concepto\marginnote{\textsuperscript{56}\setcounter{footnote}{56} Florencia Yaniello, «Descolonizando la
  palabra. Los medios de comunicación del Pueblo Mapuche en Puelmapu
  (Argentina)» (tesis de licenciatura, Universidad de Buenos Aires,
  2012); Liliana Lizondo y Mariana Ortega, «Comunicación con identidad,
  entre la Ley de Servicios de Comunicación Audiovisual y la
  comunicación popular». En \emph{Actas del VI Encuentro Panamericano de
  Comunicación Industrias culturales, medios y públicos: de la recepción
  a la apropiación en los contextos socio-políticos contemporáneos}
  (Córdoba: Universidad Nacional de Córdoba, 2013); Paula Cecchi,
  «Estrategias de lucha y comunalización de los pueblos indígenas en
  torno al proyecto de Comunicación con Identidad» (tesis de
  licenciatura, Universidad de Buenos Aires, 2014); Liliana Lizondo,
  «Comunicación con identidad o comunicación comunitaria. El caso de la
  FM ``La voz indigena''» (tesis de maestría, Universidad Nacional de La
  Plata, 2015); Liliana Lizondo, «La comunicación con identidad.
  Regulaciones y un estudio de caso», \emph{Anuario Electrónico De
  Estudios En Comunicación Social Disertaciones} 11, n.º 3 (2018);
  Liliana Lizondo, «Coincidencias y oposiciones entre la Comunicación
  con Identidad y la comunitaria». En \emph{Prácticas y saberes de
  comunicación alternativa}, comp. por Mary Esther Gardella (Tucumán:
  Manuales Humánitas, 2018), 289-305; Emilse Siares y Francesca
  Bielotti, «Los medios indígenas en Argentina: caracterización y
  desafíos a partir de la experiencia de dos radios kollas», Anuario
  Electrónico de Estudios en Comunicación Social Disertaciones 11, n.º 2
  (2018); Magda-} de mayor relevancia generado hasta el momento sea el
de CCI, elaborado por investigadores y activistas de pueblos indígenas
con la finalidad de fundamentar su propuesta política de comunicación.
Esta categoría define los valores, horizontes emancipatorios y anclajes
culturales que convergen en una manera singular de concebir y gestionar
sus medios, cuyo centro es justamente la identidad étnica. Las líneas de
tematización que buscaron profundizar en este concepto destacan aspectos
como los siguientes: que el reconocimiento y ejecución del derecho de
los pueblos originarios a gestionar sus propios medios de comunicación
es una vía para fortalecer y revitalizar luchas de larga data (ligadas a
reclamos territoriales, por ejemplo), que a su vez configura una
modalidad estratégica para establecer autonomía con respecto a la figura
del estado-nación, y que en ese sentido la CCI es una instancia de
ejercicio del derecho a la auto-determinación.

En este aspecto, otro anudamiento central surge del interés por
identificar las similitudes y diferencias entre la CCI y las expresiones
de comunicación alternativa, popular, comunitaria. Bajo esta
perspectiva, se buscó comprender las singularidades de la comunicación
de los pueblos originarios, especialmente a causa de su explícito
rechazo a que sea considerada como un apellido más de la comunicación
sin fines de lucro. Si bien se destacan semejanzas y se trazan
genealogías que dan cuenta de una raíz común, y por ende de la
existencia de cierto emparentamiento, la mayor parte de los estudios
sobre esta dimensión resaltan el componente autónomo de los medios
indígenas, fundado en la diferencia cultural y la pre-existencia étnica.

Con el propósito de ahondar en el señalamiento hecho por diferentes
representantes indígenas, pero buscando ir más allá de la identidad como
marco de referencia, se presentan indagaciones que examinan los modos en
que efectivamente tienen lugar las prácticas de comunicación indígena
mediática. Recurriendo al análisis de la anatomía de estas experiencias
comunicacionales ---esto es, mirando sus lógicas organizativas, de
sostenibilidad, los actores con los que dialogan (políticas estatales,
por ejemplo), y sus componentes discursivos--- se visualiza que estos
medios se enmarcan en iniciativas de organización de mayor alcance, en
torno a las cuales el medio proyecta relaciones de poder, de disputas y
acuerdos, dentro de cada escenario puntual.\textsuperscript{57} Por lo tanto, no se trata de experiencias
ideales o cristalizadas que respondan a una escenificación estática de
cultura, ni que estén guiadas por un conjunto de parámetros que indiquen
un «deber ser», más bien se observa que son iniciativas que crean,
dinamizan y resignifican los cursos socio-políticos de cada proyecto. Un
dato relevante es que los trabajos de Oscar Grillo se destacan por ser
de los primeros abordajes etnográficos ---y pocos en el caso de
experiencias indígenas\marginnote{lena Doyle y Emilse Siares, «Indigenous Peoples' right to
  communication with identity in Argentina, 2009--2017», \emph{Latin
  American Perspectives} 45, n.º 3 (2018); Paula Milana y Emilia
  Villagra, «Comunicación indígena en el noroeste argentino: el caso de
  la radio FM Ocan (Salta, Argentina)»\emph{, Anuario Electrónico de
  Estudios en Comunicación Social Disertaciones} 11, n.º 2 (2018);
  Emilia Villagra, «Los procesos político-comunicacionales de una
  organización indígena en Salta, Argentina», \emph{Comunicación y
  Medios} 45 (2021).}--- sobre\marginnote{\textsuperscript{57}\setcounter{footnote}{57} Oscar Grillo,
  «Políticas de identidad en internet»; Oscar Grillo, \emph{Aproximación
  etnográfica al activismo Mapuche a partir de internet y tres viajes de
  trabajo de campo} (Buenos Aires: Al margen, 2013); Emilia Villagra y
  Ramón Burgos, «Radios comunitarias desde una perspectiva indígena. La
  experiencia de la Organización de Comunidades Aborígenes de Nazareno,
  Salta-Argentina», \emph{Logos} 1, n.º 24 (2017); Francisco Gil-García,
  «Definir el medio. Radios comunitarias e indígenas en la Quebrada de
  Humahuaca y la Puna de Jujuy, noroeste argentino». En \emph{Medios
  indígenas: teorías y experiencias de la comunicación indígena en
  América Latina}, coord. por Gemma Orobitg (Madrid: Iberoamericana
  Verbuert, 2020), 149-178.} procesos de mediatización en el campo
digital, especialmente en internet.

Por último, de la indagación de las agencias de los sujetos que
protagonizan estas experiencias, se identifica la diversidad de rumbos y
objetivos que delimitan su radio de acción. Aparece aquí la idea de que
la comunicación configura el trasfondo de manifestaciones heterogéneas,
como, por ejemplo, de expresiones de liderazgo y luchas sexo-genéricas,
y se establecen conexiones con otros ejes de conocimiento como la
educación y los estudios de la memoria. De acuerdo a esta lógica, la
comunicación resulta un nuevo «territorio» donde se despliegan redes de
articulación entre diferentes interlocutores y procesos, lo que da
cuenta del peso que tienen estos medios para las comunidades y grupos
que los gestionan, incluso trascendiendo la función informativa y de
formación de opinión pública tradicionalmente atribuida a las
tecnologías comunicacionales modernas.\footnote{Ileana, Mamaní, «La
  estética radiofónica en el discurso de la FM Comunitaria La Voz
  Indígena» (tesis de licenciatura, Universidad Nacional de Salta-Sede
  Regional Tartagal, 2017); Magdalena Doyle, «Las luchas por territorios
  ancestrales en los medios indígenas. El caso de FM La Voz Indígena»,
  \emph{Comunicación y medios}, n.º 38 (2018), 177-189; Soledad Amaya,
  «Las estrategias políticas que despliegan las mujeres Wichí en FM
  Comunitaria La Voz Indígena» (tesis de licenciatura, Universidad
  Nacional de Salta Sede Regional Tartagal, 2018); Delfina Acosta,
  «Comunicación para el buen vivir desde la FM comunitaria La Voz
  Indígena» (tesis de licenciatura, Universidad Nacional de Salta-Sede
  Regional Tartagal, 2018); Damián Andrada, comp., \emph{Hacia un
  periodismo indígena} (Buenos Aires: Ediciones Universidad del
  Salvador, 2019); Karen Machado y Silvana Karuchek. «La radio como
  estrategia pedagógica en el CAJ de la Escuela EMETA II de Yacuy»
  (tesis de licenciatura, Universidad Nacional de Salta-Sede Regional
  Tartagal, 2018); Ana}
Algunas de estas investigaciones son receptoras de perspectivas que se
hicieron fuertes en las últimas décadas, como la de género y el enfoque
ambiental. La noción de territorio empleada aquí conlleva una
significación metafórica que asocia el movimiento de lucha por el
territorio con la proyección y efectiva ocupación de nuevos espacios de
reproducción de la vida y configuración de sentidos, probablemente en el
plano de la cultura.

Los métodos utilizados son prioritariamente cualitativos, se destacan la
etnografía y el análisis del discurso y de contenido. En cuanto a las
técnicas implementadas, en gran parte se utilizan entrevistas en
profundidad, relevamiento y escucha de producciones radiales, digitales
o audiovisuales, consulta e indagación de noticias y programaciones. En
ese marco, las investigaciones etnográficas, además de considerar los
discursos y narrativas públicas de los actores, incorporan las historias
de vida y observaciones de las conductas de los sujetos que hacen parte
de los proyectos comunicacionales. Mientras que otros estudios, como
aquellos basados en análisis discursivos o de contenido, optan por
triangular los mensajes emitidos con las dimensiones organizativas del
medio. En resumen, las metodologías se ubican dentro de los enfoques
interpretativistas de las ciencias sociales, y en particular, buscan dar
cuenta del análisis del fenómeno de la producción de sentidos como
problema central de las ciencias de la comunicación.

Los hechos científicos en este conjunto de estudios son construidos a
partir de recursos metodológicos como la perspectiva de los actores, en
el caso de las etnografías, o la escucha e identificación de categorías
del orden del discurso para las investigaciones basadas en análisis de
narrativas. Concretamente, los hechos refieren a las prácticas de los
actores, a cómo llevan adelante sus medios, a cómo significan los
procesos\marginnote{María Siuffi, «Planificación de una radio escolar
  en la comunidad Chorote y Wichí Lapacho II» (tesis de licenciatura,
  Universidad Nacional de Salta-Sede Regional Tartagal, 2017); Liliana
  Lizondo, «Para una perspectiva del debate naturaleza-cultura desde los
  medios de comunicación. La cobertura de inundaciones del río Pilcomayo
  en 2018 según la composición de los mundos» (tesis de doctorado,
  Universidad Nacional de La Plata, 2021); Virginia Collivadino, «Los
  procesos comunicacionales de la FM La Voz de la Quebrada: una mirada a
  los sentidos sobre lo comunitario y la gestión de la emisora» (tesis
  de licenciatura, Universidad Nacional de Salta, 2022); Gisella Murillo
  y Evelyn Zimmermann, «La comunicación entre la comunidad Wichi de
  Santa Rosa y el Sistema de Salud Pública. Elaboración de un protocolo
  que contemple los sentidos de la maternidad de esa comunidad desde una
  perspectiva intercultural, comunicativa y participativa. 2019-2020»
  (tesis de licenciatura, Universidad Nacional de Salta, 2022);
  Magdalena Doyle, «Recapitulando\ldots{} sobre la comunicación
  indígena, los espacios públicos y los medios». En \emph{Pueblos
  indígenas y territorios mediáticos. Estudios sobre comunicación
  indígena en Argentina}, ed. por Liliana Lizondo y Magdalena Doyle
  (Bogotá: Fundación Friedrich Ebert, 2023), 223-228; José Sajama y
  Emilse Siares, «FM Pachakuti: comunicación, territorio y comunidad
  kollas». En \emph{Pueblos indígenas y territorios mediáticos. Estudios
  sobre comunicación indígena en Argentina}, ed. por Liliana Lizondo y
  Magdalena Doyle (Bogotá: Fundación Friedrich Ebert, 2023), 113-128;
  Ana Müller, «Antenas entre cerros: contribuciones de las radios
  rurales populares e indígenas al campo de la comunicación en el norte
  argentino. Salta, periodo 2012-2022» (tesis de maestría, Universidad
  Nacional de Córdoba, 2023); Cristina Cabral, «Radios comunitarias y la
  disputa en la construcción de memorias en pueblos originarios»,
  \emph{Dar a Leer Revista de Educación Literaria} 6, n.º 12 (2024).} de ingreso a las arenas mediáticas de sus regiones. Es decir,
que un hecho para los investigadores de este campo remite a las acciones
de creación e intercambio de sentido de los agentes indígenas; lo que no
significa que se construya una visión transparente e ingenua del
lenguaje, sino que se incorpora dentro de esa construcción a las redes
socio-técnicas y actores que disputan y negocian poder con los sujetos
protagonistas de las iniciativas mediáticas.

Por otro lado, no se encuentran investigaciones sobre las audiencias o
estudios de recepción en casos de medios indígenas. Se destaca el
predominio de análisis centrados en las experiencias mediáticas, ya sea
sobre los procesos de organización y gestión, como en sus expresiones
discursivas, políticas y estéticas. Las labores de investigación,
incluyendo la pertenencia disciplinar y el desarrollo de conceptos,
provienen mayormente de las ciencias de la comunicación, y en menor
medida de la antropología. Aunque se utilicen métodos como la etnografía
---particularmente en el caso de las etnografías de radios--- y se tomen
en cuenta discusiones dadas en otras disciplinas ---como, por ejemplo,
los aportes antropológicos al pensamiento sobre identidad y cultura---
los conceptos son ubicados y capitalizados dentro del campo de los
estudios de comunicación en general.

Las instituciones son definidas por Löblich y Scheu como «formas de
organización social a través de las cuales ---a largo plazo--- las ideas
son establecidas».\footnote{Löblich y Scheu, \emph{Writing the history
  of communication studies}, 8.} En este caso, las instituciones
involucradas en la generación de conocimientos sobre el fenómeno de los
medios indígenas de comunicación son mayormente universidades
nacionales, con el apoyo eventual de organismos de financiamiento de
proyectos de investigación como el Consejo Nacional de Investigaciones
Científicas y Técnicas de Argentina\footnote{El Consejo Nacional de
  Investigaciones Científicas y Técnicas es un organismo estatal
  dedicado a la promoción de la ciencia y la tecnología en la Argentina.
  Es la principal institución científica de este país. Funciona en}
(CONICET). Así, son las universidades nacionales de Salta
(principalmente la Sede Regional Tartagal), Jujuy, Tucumán, Córdoba,
Buenos Aires, La Plata, Comahue y Río Negro los espacios de radicación
de la mayor parte de las investigaciones que alimentan este campo. Por
otra parte, también se destaca la presencia de estudios realizados por
investigadores de universidades extranjeras, como el caso de la
Universidad Complutense de Madrid y la Universidad de Barcelona.

Si miramos al interior de estas instituciones, preguntándonos en qué
consiste realmente su participación, veremos que, en el caso de las
universidades nacionales, se trata de casas de altos estudios ubicadas
en lugares habitados por un importante número de población indígena
(especialmente Salta, Jujuy, Comahue y Río Negro) o de universidades que
reciben estudiantes de todo el país, precedidas por largas trayectorias
históricas como la Universidad de Buenos Aires y la Universidad Nacional
de Córdoba. Por lo general, las indagaciones académicas devienen de
incursiones particulares de docentes que por sus trayectorias\marginnote{todas
  las regiones de la Argentina, a través del sistema de formación de
  investigadores y becarios. Actualmente desarrolla actividades en
  cuatro áreas de conocimiento: ciencias agrarias, de ingeniería y de
  materiales, ciencias biológicas y de la salud, ciencias exactas y
  naturales y ciencias sociales y humanidades. Fue creado en 1958 bajo
  la gestión de Bernardo Houssay, Premio Nobel de Medicina en 1947.} activistas
llegaron a conocer y formar parte de experiencias mediáticas de pueblos
indígenas, y a partir de allí desarrollaron proyectos de extensión e
investigación. Fueron estos mismos individuos quienes guiaron y animaron
a estudiantes en la realización de investigaciones de grado y posgrado
sobre esta temática. De ese modo, se fue expandiendo el interés por
conocer más sobre los medios indígenas de comunicación, lo que
progresivamente se plasmó en la formulación de proyectos de
investigación financiados por las propias universidades y también por
organismos nacionales autónomos como el CONICET. A su vez, en el seno de
las universidades, se observa que se trata de iniciativas de
agrupaciones específicas de docentes y estudiantes, que no
necesariamente reflejan el compromiso ni la posición política oficial.

Otras instituciones estatales que fueron parte de la constitución de
este campo de estudios son la Secretaría de Agricultura Familiar,
Campesina e Indígena (SAFCI) y el Instituto Nacional de Tecnología
Agropecuaria (INTA), ambas dedicadas a la intervención en el ámbito
rural. Su incidencia se puede apreciar en el financiamiento y apoyo
técnico a iniciativas mediáticas de pueblos indígenas, y también en la
realización de actividades de extensión universitaria que dieron cabida
a discusiones académicas sobre la comunicación indígena. Asimismo,
muchos de los trabajadores de las universidades nacionales se
desempeñaban al mismo tiempo como técnicos de estos organismos.

El crecimiento y estabilización de ciertas discusiones, como aquellas
relacionadas con el derecho a la comunicación de los medios
alternativos, entre ellos, los medios indígenas, y con el monitoreo de
los avances e incumplimientos de la LSCA, han sido impulsadas desde
organizaciones sin fines de lucro de investigadores y activistas del
campo de la comunicación, como la Mesa de Comunicación Popular de Salta
y Jujuy y la Red Interuniversitaria de Investigadores en Comunicación
Comunitaria, Alternativa y Popular (RICCAP). Esta última todavía
desarrolla iniciativas colectivas de investigación sobre los medios de
comunicación indígenas en Argentina, dentro del mapa amplio de
experiencias alternativas, populares y comunitarias; y reúne a
investigadores de las universidades de Avellaneda (UNDAV), Buenos Aires
(UBA), Quilmes (UNQ), Chilecito de la provincia de La Rioja (UNDeC), del
Comahue en las provincias de Neuquén y Río Negro (UNCOMA), Córdoba
(UNC), Entre Ríos (UNER), La Plata (UNLP), Río Negro (UNRN), Salta
(UNSa), Jujuy (UNju) y Tucumán (UNT).

En general, la incipiente institucionalización de los estudios sobre
medios indígenas en la Argentina responde a los desempeños activistas y
profesionales de diferentes docentes e investigadores. Sin intentar
mencionar a la totalidad, las trayectorias de Oscar Grillo, Liliana
Lizondo, Ramón Burgos y Magdalena Doyle, se caracterizan por haber dado
los primeros pasos en el diálogo y participación dentro de iniciativas
de comunicación indígena, por el desarrollo de proyectos de
investigación con la finalidad de interpretar estas prácticas
emergentes, y también por la dirección de trabajos de tesis de grado y
posgrado. Asimismo, han buscado conseguir y fortalecer iniciativas para
brindar apoyo técnico a los proyectos comunicacionales, a la par de
interesarse por promover prácticas de investigación. Un dato clave de
estas participaciones indica que el interés académico no se presenta
separado de la intervención, por el contrario, la investigación es una
consecuencia del activismo e involucramiento sociopolítico. Para
precisar la dimensión generacional de los autores y su formación
académica, brindamos algunos datos al respecto:

\vspace{.2in}

\begin{fullwidth}
    

\begin{table}[ht]
\caption{}\label{n2zyn6pw6v2}
\begin{tabular}{@{}p{0.3\textwidth}p{0.26\textwidth}p{0.358\textwidth}p{0.2\textwidth}@{}}
\emph{Nombre y apellido} & \emph{Año de nacimiento} & \emph{Título de posgrado} & \emph{Año de egreso} \\ [.2in]
\midrule
\addlinespace[.2in]
Oscar Grillo & 1950 & Doctorado sobre la Sociedad de la Información y el Conocimiento & 2009 \\ 
\addlinespace[.1in]
Liliana Lizondo & 1963 & Doctorado en Comunicación Social & 2020 \\
\addlinespace[.1in]
Ramón Burgos & 1974 & Doctorado en Comunicación Social & 2014 \\
\addlinespace[.1in]
Magdalena Doyle & 1983 & Doctora en Antropología & 2016 \\
\end{tabular}
\end{table}

\end{fullwidth}

\vspace{.3in}


Los primeros indicios de institucionalización se pueden apreciar también
en la aprobación de proyectos específicos sobre comunicación indígena
por parte de los organismos de investigación de las universidades
nacionales. Esto responde al esfuerzo de diferentes docentes e
investigadores por llegar a instalar el tema en las agendas de sus
instituciones, sometiendo a sus proyectos a instancias evaluativas
rigurosas dentro de sus comunidades académicas. De este modo, el tema va
ganando notoriedad, aunque todavía se trate de un tópico poco frecuente.

Desde una perspectiva de comparación con otras temáticas del campo de la
comunicación social, el interés por analizar las experiencias mediáticas
indígenas resulta escaso, es decir que no son muchas las investigaciones
sobre comunicación indígena, aunque han aumentado considerablemente
luego de la sanción de la LSCA. Si llevamos esta observación al campo
amplio de las ciencias sociales, incluyendo aquí a la antropología
socio-cultural y la ciencia política, veremos que la estrechez se
acentúa aún más. Asimismo, otro rasgo a destacar es que las
contribuciones académicas provienen mayormente de universidades ubicadas
en lugares periféricos de la Argentina, de áreas geográficas alejadas de
los grandes centros urbanos y políticos del país.

En cuanto a la relación entre las historias de vida de los autores y las
instituciones, notamos la existencia de indicios de articulación, ya que
se trata de docentes y estudiantes que, en el marco de sus trayectorias
y desempeños dentro de los contextos universitarios, fueron tejiendo
encuentros y nexos con experiencias mediáticas indígenas. En torno a
esos intercambios fue posible poner en marcha iniciativas de militancia
y apoyo orientadas a promover el ingreso de las voces y demandas de los
pueblos originarios a los escenarios académicos, mediante acciones como
la participación en congresos y la realización de jornadas de
visibilización sobre las luchas de los pueblos indígenas por el derecho
a la comunicación. Dicho de otro modo, consideramos que las
universidades fueron herramientas con las que contaron los docentes e
investigadores para construir conocimiento al mismo tiempo que para
impulsar intervenciones en clave política.

En este aspecto cabe destacar la participación de organismos dedicados a
la regulación, control o cuidado del cumplimiento de determinadas
prácticas de protección de derechos de la ciudadanía en el ámbito de la
comunicación e información, como los casos de la Autoridad Federal de
Servicios de Comunicación Audiovisual (AFSCA), la Defensoría del Público
y el Consejo Federal de Comunicación Audiovisual, entidades creadas por
la LSCA. Si bien no se trata de instituciones inicialmente abocadas a la
investigación, su desempeño de hecho terminó por impactar indirectamente
en la animación o apoyo a la generación de conocimientos sobre la
temática, a través de acciones de auspicio a la realización de eventos
académicos sobre los medios indígenas, y también en la recepción de
investigadores y activistas pioneros en el estudio del fenómeno de la
comunicación indígena, como el caso de Liliana Lizondo y Matías Melillán
(pueblo Mapuche), integrantes del Consejo Federal en representación de
las universidades y pueblos originarios.

\hypertarget{a-modo-de-cierre}{%
\section{A modo de cierre}\label{a-modo-de-cierre}}

Nuestro interés buscó poner de relieve las trayectorias de investigación
que hicieron posible el nacimiento de un conjunto de estudios focalizado
en la cuestión de las experiencias mediáticas indígenas. Resaltamos que
el acceso a la institucionalización de este campo emergente se comenzó a
gestar en la segunda década de los 2000, en coincidencia con la
materialización del reclamo indígena por una comunicación con identidad
y en el fragor de las luchas por lograr un sistema democrático de
comunicación.

Podemos decir, con las evidencias presentadas, que hoy es posible
referir a la existencia de un esquema de textos, categorías y
discusiones que estructuran y guían modos de entrar y salir a la
temática de los medios de comunicación de pueblos originarios, algo que,
hasta hace menos de 20 años, era imposible sin recurrir a las
herramientas conceptuales de la comunicación alternativa, popular,
comunitaria. Es decir, el camino de institucionalización se perfila
también como un gesto de autonomización de aquel campo.

En lo relativo a la institucionalidad, es visible que el interés por
investigar este fenómeno ha sido moldeado por la acción de diferentes
instituciones. Desde el contexto singular en el que se movilizaron las
demandas indígenas por la gestión de sus propios medios, pasando por la
disposición de recursos estatales para dicho fin (concursos para el
acceso a fondos a través de políticas públicas), incluyendo la
realización de capacitaciones y asesorías, hasta el involucramiento de
docentes e investigadores universitarios en la colaboración dentro de
experiencias específicas y el desarrollo de tareas de investigación, se
visualiza la formación de un campo modelado por la actuación de agentes
heterogéneos o con inscripciones institucionales variadas.

\enlargethispage{\baselineskip}

Por otro lado, es notable que se trata de un naciente agrupamiento de
estudios caracterizado por su interdisciplinariedad y por el estrecho
nexo entre investigación y activismo. La mayoría de los investigadores
son a su vez activistas y colaboradores con las experiencias mediáticas
que son la base empírica de sus trabajos, muchos son incluso impulsores
de proyectos de comunicación dentro de diferentes colectivos indígenas.
Del mismo modo también observamos que se trata de estudios
particularmente feminizados, es decir, la mayoría de los aportes son
realizados por investigadoras mujeres o por diversidades sexo-genéricas
femeninas.

Retomando el sentido de campo académico entendido como lugar dinámico de
disputas, consideramos que este campo se caracteriza por su liminalidad
y juventud. Se trata de una arena epistémica construida desde los
márgenes o bordes de los sectores que concentran los capitales
académicos de este país, con aportes generados desde universidades
pequeñas, ubicadas en regiones alejadas de las grandes urbes, como
Salta, Jujuy, Rio Negro y Neuquén, e investigadores que fueron
acumulando y generando los primeros capitales científicos. Por esto
último la conservación o modificación de la autoridad científica aparece
ligada a los esfuerzos por colocar en agenda a este fenómeno en el
ámbito de las ciencias sociales y humanas, antes que por disputar el
control o predominio de la legitimidad sobre el tema.

Sin embargo, observamos la existencia de dinámicas de disputa
relacionadas con la diferenciación de otros agrupamientos de estudios,
como la comunicación alternativa. Desde sus primeros pasos, pero con
mayor fuerza desde la segunda década de los 2000, muchos de los
investigadores que actualmente hacen parte de los estudios sobre
comunicación indígena se posicionan y procuran el análisis particular de
este fenómeno. Esto ha significado el desprendimiento de otros marcos de
referencia, y la exposición de tensiones y señalamientos críticos sobre
tradiciones de reconocida trayectoria como los estudios de comunicación
comunitaria.

En este punto es destacable el papel de distintas instituciones, no solo
las universidades públicas sino también organismos creados por la LSCA y
otras dependencias estatales en especial en el período 2009--2015, para
la formación y recepción de intereses relacionados con la posibilidad de
acceder a recursos para favorecer el crecimiento de los proyectos de
comunicación, y a la vez dotar de legitimidad a través de la
investigación a esos procesos de democratización de la comunicación.

Consideramos que el conjunto de investigaciones reunidas hasta la fecha
sobre comunicación indígena no puede pensarse al margen de los procesos
socio-políticos vivenciados en la Argentina a lo largo del siglo XXI. La
invitación a reflexionar sobre la creciente ampliación de derechos hacia
determinados sectores considerados como minoritarios o históricamente
postergados, entre ellos, los pueblos originarios, es parte de la
construcción del interés por investigar a los medios indígenas. A modo
de síntesis, podemos decir que la emergencia de estos estudios deviene
de la intención de dar cuenta de este movimiento de irrupción social, de
colocación inusual de los pueblos originarios en lugares pocas veces
ocupados.

Por último, resulta necesario hacer referencia a la situación actual de
la República Argentina. Desde finales de 2023, el gobierno de Javier
Milei ha tomado decisiones que afectan directamente a los actores que
integran este campo. Algunas de las medidas más concretas se observan en
la modificación de la LSCA con el objetivo de suprimir las limitaciones
a la tenencia múltiple de licencias audiovisuales, retomando cambios que
ya había introducido la gestión de Mauricio Macri en diciembre de 2015.
En consonancia con el retorno a un modelo de comunicación guiado por el
mercado, también se eliminaron las pocas políticas públicas destinadas
al fomento del derecho a la comunicación en medios alternativos,
incluyendo aquí a los indígenas. En cuanto a las instituciones
educativas y de investigación, el gobierno nacional sostiene un proceso
de desfinanciamiento sobre las universidades públicas y el CONICET, que
también alcanza a organismos como el INTA. El caso de la SAFCI es
particular, ya que fue directamente cerrada. Del mismo modo, el gobierno
quitó el financiamiento a entidades de pueblos originarios, como el
Instituto Nacional de Asuntos Indígenas (INAI), y derogó leyes e
instrumentos creados en función de los estándares constitucionales e
internacionales de derecho indígena. En resumidas cuentas, las
condiciones actuales para la generación de conocimientos en esta
temática no son las más favorables, no solo por lo anteriormente
mencionado, sino principalmente porque desde la estructura estatal se
contradice y deslegitima a la investigación en ciencias sociales y
humanas. En este escenario, cabe preguntarse por las posibilidades de
sostenibilidad de este naciente conjunto de estudios, al menos en el
plano institucional.




\section{Bibliography}\label{bibliography}

\begin{hangparas}{.25in}{1} 



Abu-Lughod, Lila. «Interpretando la(s) cultura(s) después de la
televisión: sobre el método», \emph{Íconos}, n.º 24 (2006): 119-141.

Acosta, Delfina. «Comunicación para el buen vivir desde la FM
Comunitaria La Voz Indígena». Tesis de licenciatura, Universidad
Nacional de Salta-Sede Regional Tartagal, 2018.

Agurto, Jorge y Jahve Mescco. «La comunicación indígena como
dinamizadora de la comunicación para el cambio social». En \emph{Actas
del XI Congreso Latinoamericano de Investigadores en Comunicación:
1--26}. Montevideo: ALAIC, 2012.
\url{https://www.servindi.org/pdf/ALAIC_comunicaci\%C3\%B3nindigena2012.pdf}

Amaya, Soledad. «Las estrategias políticas que despliegan las mujeres
Wichí en FM Comunitaria La Voz Indígena». Tesis de licenciatura,
Universidad Nacional de Salta Sede Regional Tartagal, 2018.

Andrada, Damián, comp. \emph{Hacia un periodismo indígena}. Buenos
Aires: Universidad del Salvador, 2019.

Arcila Calderón, Carlos, Alejandro Barranquero y Eva González Tanco.
«From media to buen vivir: Latin American approaches to Indigenous
communication», \emph{Communication Theory}, n. º 28 (2018): 180--201.
doi:10.1093/ct/qty004

Askew, Kelly. «Introduction». En \emph{Anthropology of media: A reader},
editado por Kelly Askew y Richard Wilk, 1--13. Malden: Blackwell
Publishers, 2002.

Baraldini, Vanina, Maiten Cañicul, Chefi Cian, Juan Chico y Matías
Melillán. «Comunicación con identidad». En \emph{Aportes para la
construcción del modelo de comunicación indígena en Argentina}. Buenos
Aires: INAI, 2011.

Barranquero-Carretero, Alejandro y Eva González-Tanco. «Editorial»,
\emph{Anuario Electrónico de Estudios en Comunicación Social
Disertaciones} 11, n.º 2 (2018): 5--11.
\url{https://revistas.urosario.edu.co/index.php/disertaciones/article/view/6838}.

Barranquero-Carretero, Alejandro y Chiara, Sáez-Baeza. «Comunicación y
buen vivir. La crítica descolonial y ecológica a la comunicación para el
desarrollo y el cambio social», \emph{Palabra Clave} 18, n.º 1 (2015):
41--82. doi: 10.5294/pacla.2015.18.1.3.

Belloti, Francesca. \emph{Los medios de comunicación de los pueblos
originarios frente a la Ley de Servicios de Comunicación Audiovisual:
Experiencias, contradicciones, desafíos. Informe de investigación
(2016--2017)}. Buenos Aires: Universidad Nacional de Quilmes, 2018.

Beltrán-Salmón, Luis Ramiro, Erick Torrico-Villanueva, Karina
Herrera-Miller y Esperanza Pinto-Sardón. \emph{La comunicación antes de
Colón. Tipos y formas en Mesoamérica y Los Andes}. La Paz: Centro
Interdisciplinario Boliviano de Estudios de la Comunicación, 2008.

Bengoa, José. \emph{La emergencia indígena en América Latina}. Santiago
de Chile: Fondo de Cultura Económica, 2000.

Bengoa, José. «¿Una segunda etapa de la Emergencia Indígena en América
Latina?», \emph{Cuadernos de Antropología Social}, n.º 29 (2009): 7--22.
\url{http://www.redalyc.org/articulo.oa?id=180913914001}.

Bourdieu, Pierre. «El campo científico», \emph{Redes: revista de
estudios sociales de la ciencia} 1, n.º 2 (1994): 129--160.
\url{https://ridaa.unq.edu.ar/bitstream/handle/20.500.11807/317/07R1994v1n2.pdf?sequence=1\&isAllowed=y}

Briones, Claudia. «Construcciones de aboriginalidad en Argentina»,
\emph{Société Suisse des Américanistes,} n.º 68 (2004): 73--90.

Bustamante, Fernando. «La constitución del sujeto indígena en el Chaco
Salteño. Disputas simbólicas y estrategias de comunicación en torno al
desarrollo». En \emph{Luchas y transformaciones sociales en Salta},
editado por Víctor Arancibia y Alejandra Cebrelli, 127--153. Salta:
Centro Promocional de Investigaciones en Historia y Antropología, 2011.

Cabral, Cristina. «Radios comunitarias y la disputa en la construcción
de memorias en pueblos originarios», \emph{Dar a Leer Revista de
Educación Literaria} 6, n.º 12 (2024): 30--42.
\url{https://revele.uncoma.edu.ar/index.php/daraleer/article/view/5610/62501}.

Caucus Indígena del Sur, Centro y México. «Declaración de los Pueblos
Indígenas ante la Cumbre Mundial de la Sociedad de la Información».
\emph{América Latina en movimiento}, 12 de diciembre de 2003.
\url{https://www.alainet.org/es/articulo/108953}.

Cecchi, Paula. «Estrategias de lucha y comunalización de los pueblos
indígenas en torno al proyecto de Comunicación con Identidad». Tesis de
licenciatura, Universidad de Buenos Aires, 2014.

Collivadino, Virginia. «Los procesos comunicacionales de la FM La Voz de
la Quebrada: una mirada a los sentidos sobre lo comunitario y la gestión
de la emisora». Tesis de licenciatura, Universidad Nacional de Salta,
2022.

Comisión Nacional de Comunicación de los Pueblos Indígenas\emph{.
Política pública de comunicación de y para los pueblos indígenas}.
Bogotá: CONCIP, 2018.

Contreras-Baspineiro, Adalid. «Aruskipasipxañanakasakipunirakispawa: We
must communicate each other, despite the differences, and solve them
through communication», \emph{Razón y Palabra}, n. º 93 (2016).
\url{https://www.revistarazonypalabra.org/index.php/ryp/article/viewFile/3/pdf}.

Cortés, Diego Mauricio. «Radio indígenas y Estado en Colombia:
¿Herramientas ``políticas'' o instrumentos ``policivos''?»,
\emph{Chasqui. Revista Latinoamericana de Comunicación}, n\emph{.}º 140
(2019): 47--62. http://hdl.handle.net/10469/18225.

Defensoría del Público e Instituto Nacional de Asuntos Indígenas.
\emph{Recomendaciones para el tratamiento mediático sobre los pueblos
indígenas}. Buenos Aires: Defensoría del Público, 2023.

Doyle, Magdalena. «Los medios masivos de comunicación en las luchas de
los pueblos indígenas. Abordaje desde los estudios sobre comunicación en
América Latina». Tesis de maestría, Universidad Nacional de Córdoba,
2013.

Doyle, Magdalena. «Debates y demandas indígenas sobre derechos a la
comunicación en América Latina», \emph{Temas Antropológicos Revista
Científica de Investigaciones Regionales} 37, n.º 2 (2015): 89--118.
\url{https://www.redalyc.org/pdf/4558/455844901004.pdf}.

Doyle, Magdalena. «El derecho a la comunicación de los pueblos
originarios. Límites y posibilidades de las reivindicaciones indígenas
en relación al sistema de medios de comunicación en Argentina». Tesis de
doctorado, Universidad de Buenos Aires, 2016.

Doyle, Magdalena. «El derecho a la comunicación con identidad. Aportes
de los debates indígenas en Argentina para pensar la noción de derecho a
la comunicación», \emph{Mediaciones}, n.º 18 (2017): 40--56. doi:
10.26620/uniminuto.mediaciones.13.18.2017.40-56.

Doyle, Magdalena. «Las luchas por territorios ancestrales en los medios
indígenas. El caso de FM La Voz Indígena», \emph{Comunicación y medios},
n.º 38 (2018): 177--189. doi: 10.5354/0719-1529.2018.50650.

Doyle, Magdalena. «Acceso y participación de los pueblos indígenas en el
sistema de medios de Argentina», \emph{Anuario Electrónico de Estudios
en Comunicación Social ``Disertaciones''}, n.º 11(2018), 30-49. doi:
10.12804/revistas.urosario.edu.co/disertaciones/a.5479.

Doyle Magdalena \& Emilse Siares. «Indigenous Peoples' Right to
Communication with Identity in Argentina, 2009--2017», \emph{Latin
American Perspectives} 45, n.º 3 (2018): 55-67.
doi: 10.1177/\\\hspace{.267in}0094582X18766909.

Doyle, Magdalena. «Matrices y vertientes de pensamiento sobre los medios
indígenas en América Latina». \emph{History of Media Studies}, n.º 2
(2022): 2--25. doi:10.32376/d895a0ea.292b1261.

Doyle, Magdalena. «Recapitulando\ldots{} sobre la comunicación indígena,
los espacios públicos y los medios». En \emph{Pueblos indígenas y
territorios mediáticos. Estudios sobre comunicación indígena en
Argentina}, editado por Liliana Lizondo y Magdalena Doyle, 223--228.
Bogotá: Fundación Friedrich Ebert, 2023.

Doyle, Magdalena, Mariana Ortega y Liliana Lizondo. «Los tiempos largos
de la comunicación indígena en Argentina. Trayectorias nacionales y
análisis de un caso», \emph{Contracorrente}, n.º 17 (2021): 27--52. doi:
10.59666/cc-ppgich.v0i17.

Doyle, Magdalena, Emilse Siares y Francesca Belloti. «Los medios de
pueblos originarios en América Latina: historia, aproximaciones y
desafíos». En \emph{Hacia un periodismo indígena}, compilado por Damián
Andrada, 203--218. Buenos Aires: Universidad del Salvador, 2019.

Escárzaga, Fabiola. «La emergencia indígena contra el neoliberalismo»,
\emph{Política y Cultura}, n.º 22 (2004): 101--121.
\\\hspace{.267in}\href{http://www.scielo.org.mx/scielo.php?script=sci_arttext\&pid=S0188-77422004000200006}{http://www.scielo.org.mx/scielo.php?script=sci_arttext}.

Gil-García, Francisco. «Definir el medio. Radios comunitarias e
indígenas en la Quebrada de Humahuaca y la Puna de Jujuy, noroeste
argentino». En \emph{Medios indígenas: teorías y experiencias de la
comunicación indígena en América Latina}, coordinado por Gemma Orobitg,
149--178. Madrid: Iberoamericana Verbuert, 2020.

Gil-García, Francisco, Beatriz Pérez Galán y Pedro Pitarch. «Fragmentos
para una etnografía de las radios comunitarias en América Latina»,
\emph{Disparidades. Revista de Antropología} 76, n.º 2 (2021): 2--9.
doi: 10.3989/dra.2021.015c.

Ginsburg, Faye D. «Screen memories. Resignifying the traditional in
Indigenous media». En \emph{Media worlds: Anthropology on new terrain},
editado por Faye D. Ginsburg, Lila Abu-Lughod y Brian Larkin, 39--57.
Berkeley: University of California Press, 2002.

Ginsburg, Faye D., Lila Abu-Lughod y Brian Larkin. «Introduction». En
\emph{Media worlds. Anthropology of new terrain}, editado por Faye D.
Ginsburg, Lila Abu-Lughod y Brian Larkin, 2--36. Berkeley: University of
California Press, 2002.

Gordillo, Gastón y Silvia Hirsch. «La presencia ausente:
invisibilizaciones, políticas estatales y emergencias indígenas en la
Argentina». En \emph{Movilizaciones indígenas e identidades en disputa
en la Argentina}, compilado por Gastón Gordillo y Silvia Hirsch, 15--38.
Buenos Aires: La Crujía, 2010.

Grillo, Óscar. «Políticas de identidad en internet. Mapuexpress:
imaginario activista y procesos de hibridación», \emph{Razón y Palabra},
n.º 54 (2006): 2--19.
\url{https://www.redalyc.org/articulo.oa?id=199520736010}.

Grillo, Óscar. \emph{Aproximación etnográfica al activismo Mapuche a
partir de internet y tres viajes de trabajo de campo}. Buenos Aires: Al
margen, 2013.

Hang, María Cecilia. «Pueblos originarios y Ley de Servicios de
Comunicación Audiovisual. El caso Wall Kintun TV», \emph{Tram{[}p{]}as
de la comunicación y la cultura}, n.º 84 (2019): 2--18. doi:
10.24215/231427\\\hspace{.267in}4xe036.

Herrera-Huérfano, Eliana, Francisco Sierra Caballero y Carlos Del Valle
Rojas. «Hacia una epistemología del sur. Decolonialidad del saber/poder
informativo y nueva comunicología latinoamericana. Una lectura crítica
de la mediación desde las culturas indígenas»,'' \emph{Chasqui. Revista
Latinoamericana de Comunicación}, n.º 131 (2016): 77--105.
\url{https://dialnet.unirioja.es/servlet/articulo?codigo=5792037}.

Herrera-Miller, Karina. \emph{¿Del grito pionero...al silencio? Las
radios sindicales mineras en la Bolivia de hoy}. La Paz: Plural
Editores, 2006.

Huergo, Jorge, Kevin Morawicki y Lourdes Ferreira. «Los medios, las
identidades y el espacio de comunicación. Una experiencia de radio
comunitaria con aborígenes wichí», Comunicar\emph{. Revista científica
de comunicación y educación}, n.º 26 (2005): 103--110. doi:
\href{https://doi.org/10.3916/C26-2006-16}{10.3916/C26-2006-16}.

Instituto Nacional de Estadística y Censos. \emph{Censo nacional de
población, hogares y viviendas 2010. Censo del Bicentenario. Pueblos
originarios.} Ciudad Autónoma de Buenos Aires: Instituto Nacional de
Estadística y Censos, 2015.

Instituto Nacional de Estadística y Censos. \emph{Censo nacional de
población, hogares y viviendas 2022. Resultados definitivos. Población
indígena o descendiente de pueblos indígenas u originarios}. Buenos
Aires: Instituto Nacional de Estadística y Censos, 2024.

Kantor, Leda, David Ruiz, Liliana Lizondo y Clarisa Pleguezuelos. «La
voz Indígena», \emph{Margen. Revista de Trabajo Social}, n.º 42. (2006):
1--3. \url{http://www.margen.org/suscri/numero42.html}.

Lenton, Diana. «Política indigenista argentina: una construcción
inconclusa», \emph{Anuário Antropológico}, n.º 35 (2010): 57--97.

Lizondo, Liliana. «¿Extensión o aprendizaje-servicio?». En
\emph{Aprendizaje-servicio en la Educación Superior. Una mirada
analítica desde los protagonistas}, compilado por Alba González y
Rosalía Montes, 103--108. Buenos Aires: EUDEBA, 2008.

Lizondo Liliana y Mariana Ortega. «Comunicación con identidad, entre la
Ley de Servicios de Comunicación Audiovisual y la comunicación popular».
En \emph{Actas del VI Encuentro Panamericano de Comunicación Industrias
culturales, medios y públicos: de la recepción a la apropiación en los
contextos socio-políticos contemporáneos,} 1-12. Córdoba: Universidad
Nacional de Córdoba, 2013.

Lizondo, Liliana. «Comunicación con identidad o comunicación
comunitaria. El caso de la FM ``La voz indígena''». Tesis de maestría,
Universidad Nacional de La Plata, 2015.

Lizondo, Liliana. «Coincidencias y oposiciones entre la Comunicación con
Identidad y la comunitaria». En \emph{Prácticas y saberes de
comunicación alternativa,} compilado por Mary Esther Gardella, 289--305.
Tucumán: Manuales Humánitas, 2018.

Lizondo, Liliana. «La comunicación con identidad. Regulaciones y un
estudio de caso», \emph{Anuario Electrónico de Estudios en Comunicación
Social Disertaciones} 11, n.º 2 (2018): 50--65. doi:
10.12804/revistas.urosario.edu.co/disertaciones/a.5745.

Lizondo, Liliana. «Para una perspectiva del debate naturaleza-cultura
desde los medios de comunicación. La cobertura de inundaciones del río
Pilcomayo en 2018 según la composición de los mundos». Tesis de
doctorado, Universidad Nacional de La Plata, 2021.

Lizondo, Liliana y Magdalena Doyle, ed. \emph{Pueblos Indígenas y
territorios mediáticos. Estudios sobre comunicación indígena en
Argentina}. Bogotá: FES Comunicación, 2023.

Lizondo, Liliana y Ariel Sandoval. «La voz del pueblo indígena». En
\emph{Actas del VII Seminario Internacional Aprendizaje y Servicio
Solidario,} 58--60\emph{.} Buenos Aires: Ministerio de Educación,
Ciencia y Tecnología de la Nación, 2004.
\url{http://www.bnm.me.gov.ar/giga1/documentos/EL001174.pdf}.

Löblich, Maria y Andreas Matthias Scheu. «Writing the history of
communication studies: A sociology of science approach»,
\emph{Communication Theory}, n.º 21 (2011): 1--22. doi:
10.1111/j.1468-2885.2010.01373.x.

Machado, Karen y Silvana Karuchek. «La radio como estrategia pedagógica
en el CAJ de la Escuela EMETA II de Yacuy». Tesis de licenciatura,
Universidad Nacional de Salta Sede Regional Tartagal, 2018.

Mamaní, Ileana. «La estética radiofónica en el discurso de la FM
Comunitaria La Voz Indígena». Tesis de licenciatura, Universidad
Nacional de Salta-Sede Regional Tartagal, 2017.

Manasanch, Anabel. «Pueblos Originarios: comunicación con identidad»,
\emph{Revista Tram{[}p{]}as de la comunicación y la cultura}, n.º 69
(2010): 76-81. \url{http://sedici.unlp.edu.ar/handle/10915/35582}

Marques de Melo, José. \emph{Pensamiento comunicacional latinoamericano.
Entre el saber y el poder}. Sevilla: Comunicación Social Ediciones y
Publicaciones, 2009.

Milana, Paula y Emilia Villagra. «Comunicación indígena en el noroeste
argentino: el caso de la radio FM Ocan (Salta, Argentina)»,
\emph{Anuario Electrónico de Estudios en Comunicación Social
Disertaciones} 11, n.º 2 (2018): 128--142. doi:
10.12804/revistas.urosario.edu.co/disertaciones/a.5722.

Müller, Ana. «Antenas entre cerros: contribuciones de las radios rurales
populares e indígenas al campo de la comunicación en el norte argentino.
Salta, periodo 2012--2022». Tesis de maestría, Universidad Nacional de
Córdoba, 2023.

Murillo, Gisella y Evelyn Zimmermann. «La comunicación entre la
comunidad Wichi de Santa Rosa y el Sistema de Salud Pública. Elaboración
de un protocolo que contemple los sentidos de la maternidad de esa
comunidad desde una perspectiva intercultural, comunicativa y
participativa. 2019--2020». Tesis de licenciatura, Universidad Nacional
de Salta, 2022.

Orobitg, Gemma. «Etnografía de los medios de comunicación indígenas y
afroamericanos: propuestas metodológicas», \emph{Revista Española de
Antropología Americana}, n.º 50 (2020): 211--214. doi:
10.5209/reaa.71751.

Orobitg, Gemma. «Antropología de los medios de comunicación indígenas y
afro en América Latina: una presentación», \emph{Disparidades. Revista
de Antropología} 76, n.º 2 (2021): 1--5. doi: 10.3989/\\\hspace{.267in}dra.2021.015a.

Orobitg, Gemma y Roger Canals. «Hipertexto, multivocalidad y
multimodalidad para una etnografía sobre los medios de comunicación: la
web MEDIOS INDÍGENAS», \emph{Revista Española de Antropología
Americana}, n.º 50 (2020): 215--227. doi: 10.5209/reaa.\\\hspace{.267in}70398.

Orobitg Gemma, Mònica Martínez-Mauri, Roger Canals, Gemma Celigueta,
Francisco M. Gil García, Sebastián Gómez Ruiz, Gabriel Izard, Julián
López García, Óscar Muñoz Morán, Beatriz Pérez- Galán y Pedro Pitarchk.
«Los medios indígenas en América Latina: Usos, sentidos y cartografías
de una experiencia plural». \emph{Revista de Historia}, n.º 83 (2021):
132--164. doi: 10.15359/rh.83.6.

Ortega, Mariana. «En primera persona. Ciudadanías comunicacionales en la
experiencia de la FM Comunitaria La Voz Indígena»,
\emph{Question/Cuestión} 3, n.º 70 (2021): 1--27.

Prins, Harald E. «Visual anthropology». En \emph{A companion to the
anthropology of American Indians}, editado por Thomas Biolsi, 506--525.
Malden: Blackwell Publishing, 2004.

RICCAP. \emph{Informe de relevamiento de los servicios de comunicación
audiovisual comunitarios, populares, alternativos, cooperativos y de
pueblos originarios en Argentina.} Buenos Aires, RICCAP: 2019.

Sajama, José y Emilse Siares. «FM Pachakuti: comunicación, territorio y
comunidad kollas». En \emph{Pueblos indígenas y territorios mediáticos.
Estudios sobre comunicación indígena en Argentina}, editado por Liliana
Lizondo y Magdalena Doyle, 113--128. Bogotá: Fundación Friedrich Ebert,
2023.

Salazar, Juan. «Activismo indígena en América Latina: estrategias para
una construcción cultural de las tecnologías de información y
comunicación», \emph{Journal of Iberian and Latin American Studies} 8,
n.º 2 (2002), 62--80. doi: 10.1080/13260219.2002.10431783

Siares, Emilse y Francesca Bielotti. «Los medios indígenas en Argentina:
caracterización y desafíos a partir de la experiencia de dos radios
kollas», \emph{Anuario Electrónico de Estudios en Comunicación Social
Disertaciones} 11, n.º 2 (2018): 86--103. doi:
10.12804/revistas.urosario.edu.co/.

Siuffi, Ana María. «Planificación de una radio escolar en la comunidad
Chorote y Wichí Lapacho II». Tesis de licenciatura, Universidad Nacional
de Salta-Sede Regional Tartagal, 2017.

Sosa, Gabriela y Liliana Lizondo. «Pensando en común II. Pueblos
originarios: construir la visibilidad». En \emph{Construyendo
comunidades: reflexiones actuales sobre comunicación comunitaria},
editado por Área de Comunicación Comunitaria de la Universidad Nacional
de Entre Ríos, 123--126. Buenos Aires: La Crujía, 2009.

Turner, Terence. «Representation, politics, and cultural imagination in
Indigenous video. General points and Kayapo examples». En \emph{Media
worlds. Anthropology on new terrain}, editado por Faye D. Ginsburg, Lila
Abu-lughod y Brian Larkin, 75--89. Berkeley: University of California
Press, 2002.

Villagra, Emilia. «Los procesos político-comunicacionales de una
organización indígena en Salta, Argentina», \emph{Comunicación y Medios}
45 (2021): 115--126. doi: 10.5354/0719-1529.2021.58759.

Villagra, Emilia y Ramón Burgos. «Radios comunitarias desde una
perspectiva indígena. La experiencia de la Organización de Comunidades
Aborígenes de Nazareno, Salta-Argentina», \emph{Logos} 1, n.º 24 (2017),
93--105.
\url{https://www.e-publicacoes.uerj.br/logos/article/view/28512/21246}.

Villamayor, Claudia. «La ley de SCA y la visibilización de los pueblos
originarios». En \emph{Ley 26.522. Hacia un nuevo paradigma en
comunicación audiovisual}, coordinado por Mariana Baranchuk y Javier
Rodríguez Usé, 131--142. Buenos Aires: AFSCA-UNLZ, 2011.

Yaniello, Florencia. «Descolonizando la palabra. Los medios de
comunicación del Pueblo Mapuche en Puelmapu (Argentina)». Tesis de
licenciatura, Universidad de Buenos Aires, 2012.



\end{hangparas}


\end{document}