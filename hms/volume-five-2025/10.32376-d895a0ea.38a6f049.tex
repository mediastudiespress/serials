% see the original template for more detail about bibliography, tables, etc: https://www.overleaf.com/latex/templates/handout-design-inspired-by-edward-tufte/dtsbhhkvghzz

\documentclass{tufte-handout}

%\geometry{showframe}% for debugging purposes -- displays the margins

\usepackage{amsmath}

\usepackage{hyperref}

\usepackage{fancyhdr}

\usepackage{hanging}

\hypersetup{colorlinks=true,allcolors=[RGB]{97,15,11}}

\fancyfoot[L]{\emph{History of Media Studies}, vol. 5, 2025}


% Set up the images/graphics package
\usepackage{graphicx}
\setkeys{Gin}{width=\linewidth,totalheight=\textheight,keepaspectratio}
\graphicspath{{graphics/}}

\title[Film as a Dream of the Modern Man]{Film as a Dream of the Modern Man: Interpretation of Susanne Langer's ``Note on the Film''} % longtitle shouldn't be necessary

% The following package makes prettier tables.  We're all about the bling!
\usepackage{booktabs}

% The units package provides nice, non-stacked fractions and better spacing
% for units.
\usepackage{units}

% The fancyvrb package lets us customize the formatting of verbatim
% environments.  We use a slightly smaller font.
\usepackage{fancyvrb}
\fvset{fontsize=\normalsize}

% Small sections of multiple columns
\usepackage{multicol}

% Provides paragraphs of dummy text
\usepackage{lipsum}

% These commands are used to pretty-print LaTeX commands
\newcommand{\doccmd}[1]{\texttt{\textbackslash#1}}% command name -- adds backslash automatically
\newcommand{\docopt}[1]{\ensuremath{\langle}\textrm{\textit{#1}}\ensuremath{\rangle}}% optional command argument
\newcommand{\docarg}[1]{\textrm{\textit{#1}}}% (required) command argument
\newenvironment{docspec}{\begin{quote}\noindent}{\end{quote}}% command specification environment
\newcommand{\docenv}[1]{\textsf{#1}}% environment name
\newcommand{\docpkg}[1]{\texttt{#1}}% package name
\newcommand{\doccls}[1]{\texttt{#1}}% document class name
\newcommand{\docclsopt}[1]{\texttt{#1}}% document class option name


\begin{document}

\begin{titlepage}

\begin{fullwidth}
\noindent\LARGE\emph{Article
} \hspace{98mm}\includegraphics[height=1cm]{logo3.png}\\
\noindent\hrulefill\\
\vspace*{1em}
\noindent{\Huge{Film as a Dream of the Modern Man:\\\noindent Interpretation of Susanne Langer's\\\noindent``Note on the Film''\par}}

\vspace*{1.5em}

\noindent\LARGE{Tereza Hadravová} \href{https://orcid.org/0000-0002-5305-5457}{\includegraphics[height=0.5cm]{orcid.png}}\par\marginnote{\emph{Tereza Hadravová, ``Film as a Dream of the Modern Man: Interpretation of Susanne Langer's `Note on the Film','' \emph{History of Media Studies} 5 (2025), \href{https://doi.org/10.32376/d895a0ea.38a6f049}{https://doi.org/ 10.32376/d895a0ea.38a6f049}.} \vspace*{0.75em}}
\vspace*{0.5em}
\noindent{{\large\emph{Charles University}, \href{mailto:Tereza.Hadravova@ff.cuni.cz}{Tereza.Hadravova@ff.cuni.cz}\par}} \marginnote{\href{https://creativecommons.org/licenses/by-nc-nd/3.0/deed.en}{\includegraphics[height=0.5cm]{graphics/by-nc-nd.png}}}

% \vspace*{0.75em} % second author

% \noindent{\LARGE{<<author 2 name>>}\par}
% \vspace*{0.5em}
% \noindent{{\large\emph{<<author 2 affiliation>>}, \href{mailto:<<author 2 email>>}{<<author 2 email>>}\par}}

% \vspace*{0.75em} % third author

% \noindent{\LARGE{<<author 3 name>>}\par}
% \vspace*{0.5em}
% \noindent{{\large\emph{<<author 3 affiliation>>}, \href{mailto:<<author 3 email>>}{<<author 3 email>>}\par}}

\end{fullwidth}

\vspace*{2em}


\small\noindent{Republication of Tereza Hadravová, ``Film as a Dream of
the Modern Man: Interpretation of Susanne Langer's `Note on the
Film,'\,''  \emph{Eidos. A Journal for Philosophy of
Culture} 4, no. 1 (2020): 38--48,
\url{https://doi.org/10.14394/eidos.jpc.2020.0004.}
The author has used the opportunity of its republication to polish its
style and structure; however, no significant additions have been made.
{[}\href{https://creativecommons.org/licenses/by-nc-nd/3.0/deed.en}{CC
BY-NC-ND 3.0}{]}}

\vspace*{2em}

\hypertarget{abstract}{%
\section{Abstract}\label{abstract}}

The paper concerns a ``Note on the Film,'' a short appendix to
\emph{Feeling and Form} by Susanne K. Langer. The interpretation
interweaves the Note into a larger context of Langer's philosophical
work---primarily in terms of her understanding of the dream as a lower
symbolic form, to which the film is compared---as well as in terms of
her account of literary arts among which, she suggests, cinema belongs.
Langer's references to Sergei Eisenstein are discussed, and their
respective concepts of cinema are compared. An implicit political
dimension of Langer's writing on film is emphasized by relating her
critique of modern civilization, as sketched in the last chapter of
\emph{Philosophy in a New Key}, to her film aesthetics. At the end of
the paper, I compare my interpretation of the Note with the one that was
offered by Trisha Curran.





\enlargethispage{2\baselineskip}

\vspace*{10em}

\noindent{\emph{History of Media Studies}, vol. 5, 2025}


 \end{titlepage}

% \vspace*{2em} | to use if abstract spills over

\newthought{In the short} ``Note on the Film,'' published as an Appendix to her book
\emph{Feeling and Form},\footnote{Langer, \emph{Feeling and Form},
  Appendix.} Susanne Langer extensively refers to the writings of Sergei
Eisenstein, who was three years her junior. In
Eisenstein\textquotesingle s ideas, she recognizes a similarity to her
own understanding of the symbolic nature of art and, specifically, to
what she calls a dream-like character of motion pictures. The detected
affinity---strengthened by an unusual lack of distance and direct
criticism from Langer\footnote{In her books, Langer often extensively
  quotes and interprets sources coming from various fields, including
  logic, philosophy, biology, and neuroscience; her writing style is
  dialogic. Rather than to corroborate her own view, she, however,
  usually looks for inadequacies and inaccuracies in the ideas of her
  dialogic partners.}---may seem surprising since, at first sight, the
differences prevail. While Eisenstein famously promoted montage as the
key element of film language, Langer's interest in the art of film
concentrates on the work of camera. Moreover, Eisenstein's film work is
hardly separable from its political context, whereas for Langer
film\\\noindent---as every art---is an autonomous sphere. The aim of this essay is
to find out how close relation between Langer and Eisenstein one can
sustain and where do their respective lines of thought potentially
intersect.

The idea that a human being is essentially a symbolic creature, which
Langer articulates throughout her earlier book \emph{Philosophy in a New
Key},\footnote{Langer, \emph{Philosophy in a New Key}, originally
  published in 1942. Hereafter referred to as \emph{PNK} with page
  number(s).} has far-reaching consequences, one of them being a claim
that the arts---as an intentional and self-conscious symbol-making
activity---possess the power to deeply change the world one lives in.
The idea that humans can be transformed through the arts also deeply
resonates with Sergei Eisenstein's artistic aspirations and commitments.

Although Langer's philosophical work is, if not completely apolitical,
then far less so than Eisenstein's famous revolutionary
manifestos\\\noindent---such as \emph{Strike} (1925) or \emph{Battleship Potemkin}
(1926); one can register a tone of uneasiness at the modern society in
her writing. She considers society dangerously out of balance due to
what represents its greatest achievement---the highly sophisticated use
of symbols. The childhood of mankind, which she locates in myth, rite,
and dream, is gone, but the victory of reason, however successful in
terms of power and knowledge, is likely to be short-lived, as humanity,
bereft of dreams, would eventually lose the ability to think creatively.
I will argue that the worry about the state of contemporary society that
Susanne Langer outlined in the final chapter of \emph{Philosophy in a
New Key} is, in a way, revisited in the appendix to \emph{Feeling and
Form}, where a remedy to the modern malaise is found.

In what follows, I first introduce some of the themes Langer develops in
\emph{Philosophy in a New Key}; especially those related to dreaming as
a lower symbolic form. I introduce the distinction between
sign-perception and symbol-perception and outline Langer's critique of
modern civilization as lacking resources for symbol-perception. In the
second part of the paper, I try to unpack the, rather inchoate, ``Note
on the Film''---a short essay of barely twenty paragraphs---in which
Langer alludes to similarities between the experience of film and that
of dreaming. Similarly to Langer I then, in the third part, use
Eisenstein's writings as a mirror of her own approach. However, unlike
her, I also look at some aspects of Eisenstein's thought that she has
left unaddressed; I am interested in similarities as well as differences
between them. In the last part of the paper, I get back to her critique
of modernity and pose film as one of the means of making the modern
humans dream again. Finally, I compare my interpretation of Langer's
Note with the first (and only) systematic study of Langer's film
aesthetics by Trisha Curran.

\hypertarget{i}{%
\section{I.}\label{i}}

A significant part of the \emph{Philosophy in a New Key} is devoted to
vindicating the idea of the primitivity of symbolization. The human mind
has a ``profoundly symbolific character'' (\emph{PNK}, 127) and,
unavoidably, resorts to a ``perceiving as'' mode as its basis. Humans in
Langer's understanding must relate things, abstract forms, and perceive
connections. There is no primitive, animal-like stage; reality gives
itself as a layered network of connections from the very beginning even
prior to language, in either a phylogenetic or ontogenetic sense. There
are no ``sense-data,'' no primary materials out of which the edification
of human knowledge can be built and raised. At the beginning of the
human mind there are, rather, ``pure'' symbols---synthetic and integral
repositories of an open set of potential meanings. Rather than referring
to a concrete entity in the world, their role consists in marking
certain configurations as meaningful.

One of the simplest possible, thoroughly private materializations of
what Langer calls a ``symbolic instinct'' is found in dreaming. This is
how she describes the experience of dreaming:

\begin{quote}
In our most primitive presentations---the metaphorical imagery of
dreams---it is the symbols, not its meaning, that seems to command our
emotions. We do not know it as a symbol. In dream-experience we very
often find some fairly commonplace object---a tree, a fish, a pointed
head, a staircase---fraught with intense value or inspiring the greatest
terror. We cannot tell what makes the thing so important. It simply
seems to be so in the dream. (\emph{PNK}, 131)
\end{quote}

\noindent The perceptual experience encountered in dreams is thus purely symbolic;
the objects perceived in dreams are embodied values. This is what makes
dreams so closely related to ritualistic and superstitious relationships
with the world; they all belong to what Langer calls ``the lower forms
of symbolistic thinking'' (\emph{PNK}, 114) for which it is
characteristic that the things one encounters are, so to speak, enhanced
with a radiating meaningfulness. While superstition and ritual are ways
of perceiving the world typically associated with ``primitive
societies,'' dreaming seems to be the last available form of lower
symbolistic thinking for people living in a modern, technologically
advanced, disenchanted civilization, where the only faith that endures
is ``blind faith in the conquest of nature through science.''\footnote{This
  is how Langer and her co-author characterize an intellectual
  atmosphere of the United States before the Second World War: ``This
  unreflecting mood of what one can only call `worldly faith'---blind
  faith in the conquest of nature through science---was even more marked
  in America than Europe ... partly because we had no established
  philosophical tradition to hold the balance against so much practical
  activity.'' And they continue: ``The blind faith in science has ended
  in disillusion and no faith at all. ... {[}H{]}owever great
  {[}nation's{]} scientific achievements, the greatest are always for
  purposes of destruction.'' Susanne Langer and Eugene T. Gadol, ``The
  Deepening Mind: A Half-Century of American Philosophy,''
  \emph{American Quarterly} 2, no. 2 (Summer 1950): 118--32. Quoted from
  Chaplin, \emph{The Philosophy of Susanne Langer}, 22.}

In the real, waking life of modern humans, the perceptual experience
known from dreams occurs less frequently, although, as Langer notes,
there are certain objects that do commonly induce it, such as a cross or
a ship (\emph{PNK}, 239--40). While dream-perception is replete with
symbols, ordinary perceptual experience is merged with what Langer calls
sign-perception. ``Signs,'' as defined by Langer, refer to signals or
stimuli and, in the context of perception, can be understood as reaction
prompts.\footnote{I find the idea that perception is predominantly
  sign-perception strikingly similar to J. J. Gibson's concept of
  perception in terms of what he calls ``affordances.'' See Gibson,
  ``The Theory of Affordances.'' For more on difference of signs and
  symbols, see Chaplin, ``Langer's Logic of Signs and Symbols.''} This
is even more emphasized in the perception of modern humans, whose
thinking is almost purely instrumental.

With some approximation, we can thus draw a continuum between
``sign-perception'' and ``symbol-perception.'' Pure sign-perception is
considered by Langer mainly in relation to animals, which, in her views,
are capable of thinking in these fairly limited ways; that is, within a
short horizon of immediate affordances and dangers. Symbol-perception,
on the other hand, is a uniquely human way of understanding the world.
Dreams represent symbol-perception at its extreme (or, more precisely,
one form of it), while everyday practically oriented perception lies
closer to the sign pole of the continuum, though it is not equivalent to
it.\footnote{As Langer says: ``If we have a literal conception of a
  house, we cannot merely think of a house, but \emph{know one when we
  see it}; for a sensory sign stimulating practical action also answers
  to the image with which we think'' (PNK, 225). See Chaplin,
  \emph{Philosophy of Susanne Langer}, 179.}

Before proceeding, one more conceptual clarification is necessary. In
Langer's terminology, there are two main types of symbolism that
encompass all forms of human mental activity (from dreaming and
believing, to perceiving, thinking, and creating art): presentational
and discursive symbolism. This is one of the key conceptual distinctions
she introduces; however, for reasons of space, it cannot be properly
discussed in this paper.\footnote{An excellent discussion of discursive
  and non-discursive symbolisms can be found in \emph{Ibid.}, 167--73.}
Suffice it to say that while discursive symbolism, with its discrete
elements and composite structure, is tailored ``to describe the
external, physical and empirically observable world;'' presentational
symbolism, with its synthetical nature, participates in formulation of
experience and thus contributes to its development.\footnote{In this
  sense, presentational symbolism is ``foundational to the development
  of the mind.'' \emph{Ibid.}, 170. More on formulative role of symbols
  in \emph{ibid.}, 173--79.} The most refined form of discursive
symbolism is found in the sciences, while the arts, according to Langer,
embody presentational symbolism at its best.

For Langer, it is of utmost importance to argue that both types of
symbolism, that is discursive and presentational, are equal in their
epistemological status, and a good society acknowledges that. She claims
that the arts are on the same footing with the sciences when it comes to
what should be considered knowledge. However, equality in importance
does not mean that they do not, in practice, compete for their status in
society. An ideal society, so to speak, would hold both symbolisms as
equal and help them to flourish with equal care. The society Langer
lived in, however, was far from this ideal; it was marked by the acute
dominance of discursive symbolism over presentational.\footnote{The
  opposite disequilibrium, seen in societies that prioritize
  presentational symbolism at the expense of discursive symbolism, is
  characteristic of ``primitive societies.'' Langer is careful not to
  look down at these societies; in her analysis, hers is similarly
  flawed, though suffering from the opposite problem (\emph{PNK}, 131).}

Both discursive and presentational symbolisms are primarily composed of
symbols, not signs. Thus science, which is the most elaborated form of
discursive symbolism, would not have been possible if humanity's
interest in the world had been limited to its practical dimension,
searching in nature only for ``signs for behavior.'' Science is not,
according to Langer, born from a practical need but rather from ``the
restless desire of an ever-imaginative mind to exploit the possibilities
of the factual world as a field for constructive thought'' (\emph{PNK},
229). It emerged from symbol-perception, rather than sign-perception.

However, once science matured, the symbolic character and purely
intellectual orientation of it were corrupted by the technical progress
it had, almost coincidentally, brought about. This observation is
elaborated, and its consequences are pursued, in the final chapter of
\emph{Philosophy in a New Key} titled ``The Fabric of Meaning.'' In the
modern age, the symbols of science are increasingly treated as signs,
albeit highly complex ones; only their practical dimension is sought
after. This conception of science, along with its dominance over
presentational symbolism, has infiltrated all forms of human mental
engagement with the world. The ever-growing web of practically oriented
knowledge and the supremacy of discursive symbolism have created a
dangerous imbalance in the mental life of society---with severe
consequences. To repeat, it is not simply the growth of scientific
knowledge per se that Langer regards with suspicion but rather a certain
attitude toward knowledge; a specific, highly instrumentalized
conception of knowledge.

The original source of all knowledge, that is the ability to perceive
connections and understand the reality as charged with meanings, has
dried out. The modern human, ``that mighty and rather terrible figure''
(\emph{PNK}, 232), becomes an unimportant part of their own system of
facts and truths. ``All old symbols are gone, and thousands of average
lives offer no new materials to a creative imagination'' (\emph{PNK},
245), Langer complains.

\begin{quote}
Most men never see the goods they produce, but stand by a traveling band
and turn a million identical passing screws or close a million identical
passing wrappers in a succession of hours, days, years. This sort of
activity is too poor, too empty, for even the most ingenious mind to
invest it with symbolic content. ... Most people have no home that is a
symbol of their childhood. ... Many no longer know the language that was
once their mother-tongue. (\emph{PNK}, 245)
\end{quote}

\noindent Since its reality has been stripped of symbols, such a society, Langer
argues, becomes liable to yield to cheap mysticism, nationalism, and
propaganda. She suggestively describes the loss of orientation in a
world of neutral facts and the anxiety as its natural follow-up. The
\emph{loose, half-baked ideas} that modern humans often find so
seductive are a poor replacement for a world replete with symbols, which
would offer them a safe repository for their experience. The modern man,
so to speak, is a human without dreams.

\hypertarget{ii}{%
\section{II.}\label{ii}}

Film, as Susanne Langer states in the appendix to \emph{Feeling and
Form} titled ``A Note on the Film,'' is derived from a dream. This is
not the first time she connects dreaming with the creation of expressive
symbolic forms. In the previously published \emph{Philosophy in a New
Key}, the dream was characterized as providing material from which two
distinct traditions emerged. One tradition resulting in supernatural
narratives, such as fairy-tales or ghost stories, fills the public
imagination with fragments that can be used to satisfy private fantasies
of a wide scope. The other, which can be called myth, transforms private
material into a realm that is ultimately inaccessible from a solely
private perspective. Both myth and fairy-tale thus use the same symbolic
material---the primitive symbolism of human dreams---yet they build
distinct worlds. On the one hand, there is the world built from private,
though commonly shared, fantasies; on the other, there is the world that
transcends the individual and situates one within the perspective of a
non-human subject (\emph{PNK}, Chapter 7).

The relationship between cinema and dreaming that Langer highlights in
the Note is, however, different. Dreaming does not provide cinema with
material in the sense of content; it is not a repository of images and
structures of relationships available to filmmakers. What makes film
close to a dream is its mode of presentation, which is dream-like. This
claim needs to be understood against the backdrop of the theory of art
developed in \emph{Feeling and Form}.

According to Langer, film is one of the poetic arts. In the book to
which the ``Note on the Film'' is appended, she distinguishes two basic
poetic arts: literature and drama. As poetic arts, they share what she
calls the primary illusion---a virtual life---but, as different types of
poetic arts, they approach virtual life from different perspectives.
Whereas narrative (that is, fiction or literature) is essentially
related to the past---the experience has a past character---drama
presents its events in a future-oriented way; everything happening or
occurring on stage is directed toward the future, carrying an air of the
destiny. Drama and literature may thus share the same characters,
setting, and plot, i.e., have essentially same narrative content;
however, the content takes on quite different features in each. In the
former case, the narrated events have a past character; even if set in
the narrative present, they are understood as consequences of the past,
and the past itself is what the narration is about, so to speak. In
drama, on the other hand, the mode of presentation gives the events a
prescient character; they contain signs of the events yet to come; the
focal point of drama thus resides in the future.

When Langer says that the film has a dream-like character, she refers to
the sense of the present encountered in a dream. There is no sense of
the past, no sense of the future; one is entirely in the here and now.
Recall her description of the experience of dreaming mentioned earlier:
one is immersed in a world, which is thoroughly meaningful, yet it does
not refer to any specific meaning. Commonplace objects are, as she says,
``fraught with intense value or inspiring the greatest terror,'' but
these emotions are not related to any past or future (un)happiness or a
threat---we do not fear these objects because of what they have caused
or might cause. ``It is the symbol, not its meaning, that seems to
command our emotions'' (\emph{PNK}, 131). In a film, one may encounter
the same content as in a piece of drama or literature, but the film art
form makes it seen anew: without a relation to the past, without a
relation to the future. The focal point of the film, what the film
\emph{is about}, is the present.

As a consequence, film presents objects, as Langer puts it, as
``equidistant'' from the eye. This is somewhat difficult to understand
since, first, the camera, can of course present things in detail or in
full frame as being literally more or less proximate to the viewer.
Moreover, there are necessarily internal narrative emphases in the story
of a film---some characters have leading roles while others function as
supporting actors; some objects serve as significant props, while others
take up only a circumstantial part of the \emph{mise-en-scène}. Thus,
both literally and metaphorically, some elements of a film are closer to
the eye, while others are further away, located at the periphery of
vision or on the horizon. How then can the claim about the equidistant
character of the objects presented in film be taken seriously?

In Langer's understanding of cinema, however, no roles are minor, and no
part of the diegetic world is accidental. By saying that film presents
its objects as equidistant from the eye, Langer means that each and
every part of the work is, potentially, meaningful---there are no
insignificant elements. Everything is interconnected, necessarily and
meaningfully present, and open to analysis. Films, like dreams, are thus
replete with symbols.

The film experience is similar to the dream experience; in both,
commonplace objects are ``fraught with intense value and inspiring the
greatest terror'' (\emph{PNK}, 131). The film offers modern humans a
tool to carry out the symbolic function that the world they live in no
longer sustains. In the context of modern society, the ability to enrich
the lived present by a virtual dimension might be an especially valuable
contribution of the cinematic work of art.

\hypertarget{iii}{%
\section{III.}\label{iii}}

In the middle of the Note, Langer turns to Sergei Eisenstein. The common
ground they share lies in their understanding of the nature of the film
image. Although based on pictorial representations, film is essentially
pictureless, as Eisenstein puts it (and Langer affirms); the film image
is ``objectively unrepresentable---a new idea, a new conception, a new
image.''\footnote{Eisenstein, \emph{The Film Sense}, 8. In a later
  essay, while speaking about hieroglyphs, which Eisenstein compares to
  the basic structural principle of the film image, he says: ``By the
  combination of two `depictables' is achieved the representation of
  something that is graphically undepictable.'' Eisenstein, \emph{Film
  Form}, 30.} It has the power to ``assimilate the most diverse
materials, and transform them into non-pictorial elements.''\footnote{Langer,
  \emph{Feeling and Form,} 414. Interestingly, just several paragraphs
  before the quoted sentence, she uses similar wording with slight
  variations, saying that cinema ``seems to be omnivorous, able to
  assimilate the most diverse materials and turn them into elements of
  its own''; see \emph{ibid.}, 412.}

This idea is, once again, difficult to grasp since it seems obvious that
films are composed of individual frames and these, literally, are
depictions, that is, images. However, what Eisenstein makes clear in his
writing is that what he, as a director, attempts at creating---via
individual frames and their juxtapositions---is an image without
picture; an impossible image, so to speak: an image of thought. Such an
image, as he says, is a ``psychological representation,''\footnote{Eisenstein,
  \emph{Film Form}, 32.} existing only in the mind of the spectator. And
that is what the film as an art form is built of; its basic structural
element is, in Langer's vocabulary, ``virtual.'' It transcends its
original material; it ``hovers,'' as she says with Eisenstein, ``in the
mind of the spectator.''\footnote{Langer, \emph{Feeling and Form,} 414.}

Eisenstein would thus also agree with Langer's categorization of film
under the heading of the poetic arts. In spite of the fact that the
literal material of film is an image, and the primary sense modality
used in its reception is sight, film does not belong to the visual arts.
The art form that lies closest to film, according to Eisenstein, is
poetry. The structure of cinema, in its compressed form, is well
exemplified by haiku---which Eisenstein even claims is more cinematic
than the Japanese cinema of his time.

All the mentioned similarities in their thinking about cinema
notwithstanding, Langer and Eisenstein ultimately have different foci.
Langer is not interested in particular films but rather in the deep
structure of film as an art form---in its virtual dimension. Her claim
that the world represented by film is dream-like does not pertain to
individual films and cannot be used to evaluate them as more or less
cinematic (as Eisenstein did when he, ironically, glossed over Japanese
cinema as less cinematic than Japanese haiku). Cinema, in its virtual
form, represents an egocentric materialization of the symbolic instinct,
which, in actual films, can be developed in various ways.

What is even more striking is that Langer does not mention montage,
despite it being both the main subject of Eisenstein's essay she quotes
and the key cinematographic element in his view. Instead, as she
declares at the outset of the Note, it is camera movement that she
regards as the fundamental cinematographic principle. How could we
approach montage in a Langerian way?

In both his films and writing, Eisenstein pursues a specific development
of the dream-like material Langer identifies in cinema. I would like to
argue that the elaboration Eisenstein offers in his films encompasses
both forms mentioned earlier in the context of Langer's description of a
dream as material for another symbolic form. Dream, Langer suggests in
the seventh chapter of \emph{Philosophy in a New Key}, can evolve into
supernatural narratives, such as fairy-tales, which aim at satisfying
private fantasies, as well as into myths, which are saturated with
impersonal or even cosmological values. In Eisenstein's film work,
particularly in his use of montage, one can encounter both types of
dream development.

In the former case, Eisenstein uses film montage to materialize
\emph{private} fantasies. The best example can be found in his first
film fragment, \emph{Glumov's Diary} (screened as a part of his theatre
production of Ostrovskij's play \emph{Enough Stupidity in Every Wise
Man}). In this work, montage is used to represent---literally
objectify---the secret objects of one's dreams. The power of film, which
Eisenstein satirically comments on in the fragment, is to give one what
he or she desires.

\vspace{.3in}

\begin{figure}
    \centering
    \includegraphics{graphics/glumov1.jpeg}
    \label{fig:one}
\end{figure}

\begin{figure}
    \centering
    \includegraphics{graphics/glumov2.jpeg}
    \label{fig:two}
\end{figure}

\begin{figure}
    \centering
    \includegraphics{graphics/glumov3.jpeg}
    \label{fig:three}
\end{figure}

\begin{figure}
    \centering
    \includegraphics{graphics/glumov4.jpeg}
    \label{fig:four}
    \vspace{.2in}
    \noindent\small{Stills from \emph{Glumov's Diary} (Wikimedia Commons)}
\end{figure}




\newpage In most of his films, however, Eisenstein uses montage in the opposite
way; its main power consists in precisely \emph{not} giving one what
they want. The spectator is instead prompted to fill in the missing
part. This, Eisenstein believes, can help viewers to move beyond their
personal perspective and focus on the meaning instead---entering the
world of ideas. Using Langer's terminology, the kind of dreaming
Eisenstein orchestrates in most of his films aims to construct a myth
rather than to offer a satisfying image of a fantasy we happen to share.

\hypertarget{iv}{%
\section{IV.}\label{iv}}

The main thread of this paper went from the final chapter of
\emph{Philosophy in a New Key} to the appendix of \emph{Feeling and
Form}. The crisis of modernity that Langer described almost eighty years
ago was, in her analysis, characterized by the dominance of one symbolic
activity over the other---discursive symbolism over presentational
symbolism. I argued that, for Langer, the arts---and the art of film in
particular---may be understood as playing a crucial role in revitalizing
non-discursive thinking and thereby restoring balance to the distorted
mental life of society.

In my reading of the ``Note on the Film,'' I emphasized two related
ideas: first, Langer's claim that cinema is like a dream, and second,
her categorization of film under the heading of poetic arts. Looking
back at her description of the dream experience in \emph{Philosophy in a
New Key}, I argued that the mode of experience familiar from dreaming is
related to the lower symbolic forms that human thought assumed in
``primitive'' societies. I called this mode ``symbol-perception'' and
contrasted it with ``sign-perception,'' a purely instrumental
``survival'' mode that Langer associates with animal behavior.

Film, Langer further claims in the Note, is one of the poetic arts. This
assertion makes film an important addition to the two poetic arts Langer
explores in greater depth in her book: the art of literature, which is
characteristic by past-oriented narration, and the art of drama, whose
narration is future-oriented. Film is, so to speak, the missing third,
as its characteristic dream-like mode of presentation roots one in the
present; ``{[}T{]}he dream mode in an endless now.''\footnote{\emph{Ibid}.\emph{,}
  415.} I suggested that symbol-perception (as opposed to
sign-perception) is similarly present-oriented in its understanding the
experienced world as meaningful without assigning it any concrete
meaning.

In the final chapter of \emph{Philosophy in a New Key}, Langer describes
her contemporary society in terms of crisis. This crisis can be analyzed
in two steps: First, science, as the most refined form of discursive
symbolism, has not only become the predominant epistemic relationship to
the world but has also absorbed an instrumental orientation focused on
effective, powerful, and profitable engagement with the world. Second,
this conception of knowledge as instrumentalized science, materialized
in technology, has eventually infiltrated all mental attitudes that
humans adopt toward the world and toward themselves. As a consequence,
the opportunities for symbol-perception, as well as the very capacity
for it, have been weakened.

I argued that Langer's analysis of moving pictures as a ``dreamed
reality'' on the screen, which can ``move forward and backward because
it is really an eternal and ubiquitous virtual present,''\footnote{\emph{Ibid.}}
makes film a particularly fitting tool for an initiation and
preservation of symbol-perception in modern humans. It is important to
note that this is not Langer's explicit claim; she does not revisit her
critique of modern civilization, outlined in 1942, when discussing what
she, somewhat surprisingly, called ``a new art'' eleven years later. Her
only suggestion in this context lies in her alluding to Sergei
Eisenstein, who famously used his film to modify and invigorate the
thoughts of spectators. I argued, nevertheless, that there are more
differences than similarities between their respective views of cinema.

Finally, let me compare my interpretation of the Note to the only work
on Langer's inchoate conception of cinema known to me: \emph{A New Note
on the Film} by Trisha Curran. Although my claim that film is an apt
means of making modern humans dream again can be criticized as too
far-fetched and ultimately unverifiable, it seems to have one clear
advantage: my interpretation remains more faithful to Langer's ``Note on
the Film'' than Curran's thorough reconstruction of Langerian film
aesthetics.

In the book, Curran argues against Langer's claim that cinematic mode of
presentation is the same as the one known from dreams:

\begin{quote}
Film images bear as little relation to dream images as stream of
consciousness writing bears to stream of consciousness mental activity.
Our dreams are no more Citizen Kane than our idle thought processes are
Mrs. Dalloway. But our dreams are ours. We possess them, and we analyze
them. The films we view are separate from us. We go to a film. Our
subconscious is not in control, thus we view things we would never
dream; our inner censor is free to concede to the Motion Picture
Association.\footnote{Curran, \emph{A New Note on the Film}, 31.}
\end{quote}

\noindent Although seemingly convincing, this critique, in my view, is based on a
false presupposition. Curran conceives of the dream as a private affair,
whose material is subject to psychoanalysis. For Langer, however, the
dream is a lower symbolic form, characteristic of a specific perceptual,
emotional, and cognitive engagement with the world. I emphasized that
the analogy between dreams and films Langer suggested pertains not to
the content of dreams but to their form.

``She had seen far too few films,'' Curran explains in her \emph{New
Note}: ``Thus her observation that `the immediacy of everything in a
dream is the same for film'.'' And, Curran continues, ``nor did
{[}Langer{]} comprehend the nature of filmic space.''\footnote{\emph{Ibid.}}
Finally, Curran argues that film is not a poetic art at all, as Langer
suggested, but rather a cinematic art. In essence, nearly all of
Langer's claims about film are dismissed one by one. Curran's aim was to
write a \emph{new} note on film and for that, I suppose, she felt it
necessary first to discard the old one.

Contrary to Trisha Curran, my intention in this paper is not to write a
new note on film to replace the old one but rather to flesh it out and
reinvigorate it using the resources available within Langer's
philosophical work.



\hypertarget{acknowledgments}{%
\section{Acknowledgments}\label{acknowledgments}}

The first version of the paper was written for the occasion of the
International Philosophy of Culture Week, which took place in Warsaw in
June 2019. I would like to thank the organizers---Randall Auxier,
Przemysław Bursztyka, Eli Kramer and Marcin Rychter---for organizing
this inspiring encounter of scholars, and one of the speakers at the
event, Adrienne Dengerink Chaplin, whose deep acquaintance with Susanne
Langer's philosophy I was happy to benefit from during personal
conversations. Her newly published book \emph{The Philosophy of Susanne
Langer. Embodied Meaning in Logic, Art and Feeling} (London: Bloomsbury
Academic, 2020) would have been much more present in this essay, if I
had chance to read it earlier than in the last stage of writing this
paper. I am also grateful to the anonymous referees for \emph{Eidos. A
Journal for Philosophy of Culture} for their valuable comments and
suggestions. My work on revising the paper for its republication was
supported by the Czech Science Foundation (GAČR) under the project
``Susanne K. Langer: Cognitive Aesthetics'' (GA25-17273S).


\section{Bibliography}\label{bibliography}

\begin{hangparas}{.25in}{1} 



Chaplin, Adrienne Dengerink. \emph{The Philosophy of Susanne Langer:
Embodied Meaning in Logic, Art and Feeling}. London: Bloomsbury
Academic, 2020. \url{https://doi.org/10.5040/9781350030565}.

Chaplin, Adrienne Dengerink. ``Langer's Logic of Signs and Symbols: Its
Sources and Applications.'' \emph{Eidos. A Journal for Philosophy of
Culture} 3, no. 4 (2019): 44--54.
\url{https://doi.org/10.14394/eidos.jpc.2019.0041}.

Curran, Trisha. \emph{A New Note on the Film: A Theory of Film Criticism
Derived from Susanne K. Langer's Philosophy of Art.} New York: Arno
Press, 1980.

Eisenstein, Sergei. \emph{Film Form: Essays in Film Theory}. Edited and
translated by Jay Leyda. New York and London: Harcourt Brace Jovanovich,
1949.

Eisenstein, Sergei. \emph{The Film Sense.} Edited and translated by Jay
Leyda. New York: Meridian Books, 1957.

Gibson, James Jerome. ``The Theory of Affordances.'' In
\emph{Perceiving, Acting, and Knowing: Toward an Ecological Psychology},
edited by Robert Shaw and John Bransford, 67--82. Hillsdale, NJ:
Lawrence Erlbaum.

Langer, Susanne K. \emph{Feeling and Form.} New York: Charles Scribner's
Sons, 1953.

Langer, Susanne K. \emph{Philosophy in a New Key: A Study in Symbolism
of Reason, Art, and Rite}. New York and Toronto: New American Library,
1954.



\end{hangparas}


\end{document}