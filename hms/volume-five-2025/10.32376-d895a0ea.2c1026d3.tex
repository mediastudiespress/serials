% see the original template for more detail about bibliography, tables, etc: https://www.overleaf.com/latex/templates/handout-design-inspired-by-edward-tufte/dtsbhhkvghzz

\documentclass{tufte-handout}

%\geometry{showframe}% for debugging purposes -- displays the margins

\usepackage{amsmath}

\usepackage{hyperref}

\usepackage{fancyhdr}

\usepackage{hanging}

\hypersetup{colorlinks=true,allcolors=[RGB]{97,15,11}}

\fancyfoot[L]{\emph{History of Media Studies}, vol. 5, 2025}


% Set up the images/graphics package
\usepackage{graphicx}
\setkeys{Gin}{width=\linewidth,totalheight=\textheight,keepaspectratio}
\graphicspath{{graphics/}}

\title[Stories Are Weapons]{Stories are Weapons: Psychological Warfare and the American Mind} % longtitle shouldn't be necessary

% The following package makes prettier tables.  We're all about the bling!
\usepackage{booktabs}

% The units package provides nice, non-stacked fractions and better spacing
% for units.
\usepackage{units}

% The fancyvrb package lets us customize the formatting of verbatim
% environments.  We use a slightly smaller font.
\usepackage{fancyvrb}
\fvset{fontsize=\normalsize}

% Small sections of multiple columns
\usepackage{multicol}

% Provides paragraphs of dummy text
\usepackage{lipsum}

% These commands are used to pretty-print LaTeX commands
\newcommand{\doccmd}[1]{\texttt{\textbackslash#1}}% command name -- adds backslash automatically
\newcommand{\docopt}[1]{\ensuremath{\langle}\textrm{\textit{#1}}\ensuremath{\rangle}}% optional command argument
\newcommand{\docarg}[1]{\textrm{\textit{#1}}}% (required) command argument
\newenvironment{docspec}{\begin{quote}\noindent}{\end{quote}}% command specification environment
\newcommand{\docenv}[1]{\textsf{#1}}% environment name
\newcommand{\docpkg}[1]{\texttt{#1}}% package name
\newcommand{\doccls}[1]{\texttt{#1}}% document class name
\newcommand{\docclsopt}[1]{\texttt{#1}}% document class option name


\begin{document}

\begin{titlepage}

\begin{fullwidth}
\noindent\LARGE\emph{Book review
} \hspace{88mm}\includegraphics[height=1cm]{logo3.png}\\
\noindent\hrulefill\\
\vspace*{1em}
\noindent{\Huge{\emph{Stories are Weapons: Psychological Warfare\\\noindent and the American Mind}\par}}

\vspace*{1.5em}

\noindent\LARGE{A.J. Bauer}\par\marginnote{\emph{Stories are Weapons: Psychological Warfare and the American Mind}, reviewed by A.J. Bauer, \emph{History of Media Studies} 5 (2025), \href{https://doi.org/10.32376/d895a0ea.2c1026d3}{https://doi.org/ 10.32376/d895a0ea.2c1026d3}.} \vspace*{0.75em}
\vspace*{0.5em}
\noindent{{\large\emph{University of Alabama}, \href{mailto:ajbauer2@ua.edu}{ajbauer2@ua.edu} \href{https://orcid.org/0000-0001-8458-2586}{\includegraphics[height=0.5cm]{orcid.png}}\par}} 

% \vspace*{0.75em} % second author

% \noindent{\LARGE{<<author 2 name>>}\par}
% \vspace*{0.5em}
% \noindent{{\large\emph{<<author 2 affiliation>>}, \href{mailto:<<author 2 email>>}{<<author 2 email>>}\par}}

% \vspace*{0.75em} % third author

% \noindent{\LARGE{<<author 3 name>>}\par}
% \vspace*{0.5em}
% \noindent{{\large\emph{<<author 3 affiliation>>}, \href{mailto:<<author 3 email>>}{<<author 3 email>>}\par}}

\end{fullwidth}

\vspace*{1em}


\noindent Annalee\marginnote{\href{https://creativecommons.org/licenses/by-nc/4.0/}{\includegraphics[height=0.5cm]{by-nc.png}}} Newitz. \emph{Stories are
Weapons: Psychological Warfare and the American Mind}.
272 pp. W. W. Norton, 2024. \$18.99 (paper).

\vspace{0.2in}

\newthought{In the acknowledgments} to \emph{Stories Are Weapons}, Annalee Newitz
notes that their book started as a ``manifesto'' but transformed along
the way into an intensive work of historical research. As a weapon, the
story Newitz tells has some political utility. It smartly and
provocatively weaves the histories of US settler colonialism, eugenics,
racial segregation, misogyny, and homophobia into contemporary debates
over mis- and disinformation in electoral politics. It can best be read
as a popular front text---an instrument for guiding ``resistance''
liberals toward more radical (and accurate) interpretations of US
history.

Perhaps the book's most intriguing throughline is its focus on Paul
Linebarger---a US Army officer during World War II who helped build the
Office of War Information and wrote the canonical text on military
propaganda, \emph{Psychological Warfare} (1948). Linebarger, a prolific
writer of science fiction, was better known by the pen name Cordwainer
Smith. Newitz draws fruitful parallels between the worldbuilding
strategies of science fiction authors and successful psychological
operations (``psyops''), noting that both science fiction and miliary
propaganda involve mixing fact with fiction to create psychologically
compelling narratives.

As a fellow science fiction writer, Newitz clearly sees in Linebarger a
sort of kindred spirit. \emph{Psychological Warfare} becomes a
diagnostic manual---cited like gospel through the book, used to draw
parallels 

\enlargethispage{2\baselineskip}

\vspace*{2em}

\noindent{\emph{History of Media Studies}, vol. 5, 2025}


 \end{titlepage}

% \vspace*{2em} | to use if abstract spills over

\noindent between domestic US ideological projects and modes of
political communication, especially on the right, and military
``psyops.'' Linebarger's idea of ``psychological disarmament'' becomes
the foundation for Newitz's own prescriptions for mitigating the harmful
effects of right-wing propaganda.

``Weapons intended for use in combat zones are now being deployed in the
American suburbs,'' Newitz writes (xvi--xvii), somewhat troublingly
excusing the use of psychological manipulation in warfare. ``When we use
psyops in our cultural conflicts, we tear down the wall between what's
appropriate in domestic disagreements among Americans and what's
acceptable in combat against a foreign enemy.''

But stories don't function like kinetic weapons---not really. Contrary
to fanciful vernacular understandings of media effects, words aren't
``magic bullets.'' The tension between that stubborn fact and this
book's framing limits its contribution to media historiography. It also
undercuts the book's central aim---opposing the imposition and
maintenance of unjust social, cultural, and political hierarchies.

If this book were titled \emph{We Have Always Been Post-Truth}, and
framed accordingly, I would have little to critique. In my reading, that
is this book's latent (and compelling) argument. As Newitz shows, from
the ``Indian Wars'' to the Cold War to contemporary Culture Wars, US
history is replete with instances of ideological struggle---conflicting
attempts to make sense of the world and to cultivate a shared sense of
reality. Too often, narratives designed to perpetuate or impose
hierarchies based upon gender, race, sexuality, and class have achieved
hegemony, distorting reality in favor of white heteropatriarchy.

But Newitz does not narrate it in this way. Instead, they read US
political and social history through the lens of ``psychological
warfare.'' In their telling, everything becomes a ``psyop,'' or
``weaponized storytelling''---from Edward Bernays's involvement in
toppling Jacobo Árbenz in Guatemala, to the concept of ``Manifest
Destiny,'' to Jim Crow laws, to anti-LGBTQ campaigns. But Newitz's
definition of ``psyop'' aligns too neatly with narratives that they find
morally and ethically harmful. The result is an unhelpful blurring of
the lines between the book's normative and empirical claims.

For example, in a chapter titled ``A Fake Frontier,'' Newitz unpacks
settler discourse in the US colonial and early republic eras, including
concepts like ``manifest destiny'' and the ``myth of the vanishing
Indian,'' to show how it distorted the reality of European genocide
against Indigenous peoples in order to soothe settlers' guilty
consciences.

\newpage ``It was a very seductive psyop, perpetuated by the US government,
military, and pop culture,'' Newitz writes (39). ``European settlement
was recast as an inevitable process of population replacement, in which
Indigenous nations were naturally erased by the spreading borders of the
United States of America.''

There is no disputing that last sentence. But what does calling settler
colonial ideological formation a ``psyop'' do, exactly? By blaming
``propaganda aimed primarily at white settlers'' (38), Newitz
unwittingly frames European settlers as hapless dupes, rather than as
active participants in the mythologization of the ``frontier.''

Later, Newitz quotes a tweet by science fiction writer N. K. Jemisin,
arguing that ``white supremacy is a psyop'' (104). This makes it seem as
though white people need to be tricked into believing longstanding and
deeply entrenched fictions that justify their elevated status in racial
hierarchy.

In both instances, and throughout this book, the term ``psyop'' is used
more as a pejorative to indicate normative disapproval than as an
adequate description of how supremacist ideologies actually work.

Contra Newitz, ideology is rarely reducible to mere top-down conveyance
or psychological manipulation---and never within the contexts of liberal
democracy. It is more often produced dialectically between elites and
regular people, whose interests, values, and sentiments often align in
unjust and violent constellations. This mutual construction of meaning
occurs in the context of ideological conflict, where competing
narratives of reality vie for social and cultural predominance.

To their credit, Newitz is by no means the first intelligent and
politically right-minded person to believe that propaganda is a cause,
rather than an effect, of that ideological conflict. In the late 1930s,
a group of journalists and scholars launched the Institute for
Propaganda Analysis (IPA) to combat anti-democratic propaganda through
media literacy. They believed that by training Americans to identify
common rhetorical strategies used by propagandists, they could inoculate
the public against agreeing with reactionary messaging.

Among the IPA's critics was none other than Edward Bernays, a villain in
Newitz's tale. But Bernays agreed with the IPA's premise that democracy
was ``in danger from the conflicting ideologies that are competing with
it for the interest and support of the public.'' He differed in method:
``Might it not even be a sound procedure to carry on a campaign to
establish a greater validity for the symbols of democracy with the
public directly?''\footnote{A.J. Bauer, ``Glittering Generalities:
  Reconsidering the Institute for Propaganda Analysis,''
  \emph{International Journal of Communication} 18 (2024): 1983.}

\newpage That is, while the IPA sought to promote democracy by mitigating the
effects of anti-democratic propaganda, Bernays argued in favor of openly
propagandizing in favor of democracy.

Like the IPA's ``propaganda analysis,'' Newitz's ``psychological
warfare'' heuristic misdiagnoses the problem at hand, resulting in
ineffective prescriptions. The final third of \emph{Stories Are Weapons}
advocates for ``psychological disarmament,'' pointing to the importance
of archives, libraries, and social media reforms designed to mitigate
for the salience of right-wing propaganda campaigns.

Despite earlier waxing poetic about worldbuilding in science fiction,
and noting its capacity for effective use in psyops, at the end Newitz
finds themself yearning for a world \emph{without} propaganda.
``\emph{We must end this war},'' Newitz writes (203). ``That means
trusting one another enough to put down our psychological weapons.''

As we face another Trump administration, I can sympathize with Newitz's
desire for an end to politics. But this conclusion is diametrically
opposed to the argument throughout the book---one that attributes
considerable historical agency to psychologically compelling stories.

If we want to live in a more egalitarian world---one where genocide,
white supremacy, misogyny, and homophobia are no longer thinkable, let
alone actionable---we need more compelling narratives of what that
reality would look like. We need the communication tools and educational
scale to make those stories widely salient. If stories are weapons---and
let's be clear, they are not---we don't need disarmament. We need to
\emph{scale up} armament on the left.

The only way out is through.




\section{Bibliography}\label{bibliography}

\begin{hangparas}{.25in}{1} 



Bauer, A.J. ``Glittering Generalities: Reconsidering the Institute for
Propaganda Analysis.'' \emph{International Journal of Communication} 18
(2024): 1976--1994.
\url{https://ijoc.org/index.php/ijoc/article/viewFile/21783/4549}.

Newitz, Annalee. \emph{Stories Are Weapons: Psychological Warfare and
the American Mind}. W. W. Norton, 2024.



\end{hangparas}


\end{document}