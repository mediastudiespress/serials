% see the original template for more detail about bibliography, tables, etc: https://www.overleaf.com/latex/templates/handout-design-inspired-by-edward-tufte/dtsbhhkvghzz

\documentclass{tufte-handout}

%\geometry{showframe}% for debugging purposes -- displays the margins

\usepackage{amsmath}

\usepackage{hyperref}

\usepackage{fancyhdr}

\usepackage{hanging}

\hypersetup{colorlinks=true,allcolors=[RGB]{97,15,11}}

\fancyfoot[L]{\emph{History of Media Studies}, vol. 5, 2025}


% Set up the images/graphics package
\usepackage{graphicx}
\setkeys{Gin}{width=\linewidth,totalheight=\textheight,keepaspectratio}
\graphicspath{{graphics/}}

\title[The Nowness of Chairing an English Department]{The Nowness of Chairing an English Department: A Review Essay of John Guillory's \emph{Professing Criticism} and Michael J. Sproule's \emph{Democratic Vernaculars}} % longtitle shouldn't be necessary

% The following package makes prettier tables.  We're all about the bling!
\usepackage{booktabs}

% The units package provides nice, non-stacked fractions and better spacing
% for units.
\usepackage{units}

% The fancyvrb package lets us customize the formatting of verbatim
% environments.  We use a slightly smaller font.
\usepackage{fancyvrb}
\fvset{fontsize=\normalsize}

% Small sections of multiple columns
\usepackage{multicol}

% Provides paragraphs of dummy text
\usepackage{lipsum}

% These commands are used to pretty-print LaTeX commands
\newcommand{\doccmd}[1]{\texttt{\textbackslash#1}}% command name -- adds backslash automatically
\newcommand{\docopt}[1]{\ensuremath{\langle}\textrm{\textit{#1}}\ensuremath{\rangle}}% optional command argument
\newcommand{\docarg}[1]{\textrm{\textit{#1}}}% (required) command argument
\newenvironment{docspec}{\begin{quote}\noindent}{\end{quote}}% command specification environment
\newcommand{\docenv}[1]{\textsf{#1}}% environment name
\newcommand{\docpkg}[1]{\texttt{#1}}% package name
\newcommand{\doccls}[1]{\texttt{#1}}% document class name
\newcommand{\docclsopt}[1]{\texttt{#1}}% document class option name


\begin{document}

\begin{titlepage}

\begin{fullwidth}
\noindent\LARGE\emph{Review essay
} \hspace{86mm}\includegraphics[height=1cm]{logo3.png}\\
\noindent\hrulefill\\
\vspace*{1em}
\noindent{\Huge{The Nowness of Chairing an English \\\noindent Department: A Review Essay of John \\\noindent Guillory's \emph{Professing Criticism} and Michael J. Sproule's \emph{Democratic Vernaculars}\par}}

\vspace*{1.5em}

\noindent\LARGE{Kate J. Ryan}\par\marginnote{\emph{Kate J. Ryan, ``The Nowness of Chairing an English Department: A Review Essay of John Guillory's \emph{Professing Criticism} and Michael J. Sproule's \emph{Democratic Vernaculars},'' \emph{History of Media Studies} 5 (2025), \href{https://doi.org/10.32376/d895a0ea.ad5891ed}{https://doi.org/ 10.32376/d895a0ea.ad5891ed}.} \vspace*{0.75em}}
\vspace*{0.5em}
\noindent{{\large\emph{Montana State University}, \href{mailto:kathleen.ryan3@montana.edu}{kathleen.ryan3@montana.edu}\par}} \marginnote{\href{https://creativecommons.org/licenses/by-nc/4.0/}{\includegraphics[height=0.5cm]{by-nc.png}}}

% \vspace*{0.75em} % second author

% \noindent{\LARGE{<<author 2 name>>}\par}
% \vspace*{0.5em}
% \noindent{{\large\emph{<<author 2 affiliation>>}, \href{mailto:<<author 2 email>>}{<<author 2 email>>}\par}}

% \vspace*{0.75em} % third author

% \noindent{\LARGE{<<author 3 name>>}\par}
% \vspace*{0.5em}
% \noindent{{\large\emph{<<author 3 affiliation>>}, \href{mailto:<<author 3 email>>}{<<author 3 email>>}\par}}

\end{fullwidth}

\vspace*{1.3em}


\noindent John Guillory. \emph{Professing
Criticism: Essays on the Organization of Literary
Study}. 456 pp. University of Chicago Press, 2022.
\$29 (paper).

\vspace*{0.15 in}

\noindent J. Michael Sproule.\emph{Democratic
Vernaculars: Rhetorics of Reading, Writing, Speaking, and Criticism
since the Enlightenment}. 382 pp. Routledge Press,
2020. \$42.49 (paper).

\vspace*{0.2 in}

\newthought{In Kathleen Blake Yancey's} ``Made Not Only in Words: Composition in a
New Key,'' delivered March 22, 2004, as the ``Chair's Address'' to the
Conference on College Composition and Communication and later published
in \emph{College Composition and Communication}, Yancey draws attention
to the ways historical practices of reading and contemporary public
writing outside of school can inform what we do now as English
departments. She recommends faculty in rhetoric and composition take
advantage of the ``nowness,'' or deicity, of increased public screen
writing to design new curricula, update writing-across-the-curriculum
efforts, and develop majors in composition and rhetoric. Yancey's
address kairotically invokes the deictic quality of a moment in time to
envision a new future for writing instruction.



\enlargethispage{2\baselineskip}

\vspace*{5em}

\noindent{\emph{History of Media Studies}, vol. 5, 2025}


 \end{titlepage}

% \vspace*{2em} | to use if abstract spills over


I put my own nowness into play to frame this review of John Guillory's
\emph{Professing Criticism: Essays on the Organization of Literary
Study} and J. Michael Sproule's \emph{Democratic Vernaculars: Rhetorics
of Reading, Writing, Speaking, and Criticism since the Enlightenment}.
As Collin Brooke writes in ``Weblogs as Deictic Systems: Centripetal,
Centrifugal, and Small-World Blogging,'' ``there is an immediacy to
deixis that functions rhetorically as an invitation to shared
experience: we are here, in this place, and now, at this time, and we
are connected, however briefly, through the shorthand of
deixis.''\footnote{Collin Brooke, ``Weblogs as Deictic Systems:
  Centripetal, Centrifugal, and Small-World Blogging,'' \emph{Computers
  and Composition Online} (Fall 2025).} And so, I invite readers to
share this moment of time with me, when I am a rhetoric and writing
professor reading these texts in the immediacy of my role chairing an
English department at a rural research university in the American West.
My tenure as a department chair has often been marked by the immediacy
of the crisis rhetoric that haunts the humanities; I'm regularly pulled
into concrete manifestations of this discourse because of decreased
enrollment in our three majors (Literature, Writing, and English
Education), concerns about student retention, and disciplinary frictions
that arise among faculty when the stakes seem high and the rewards
minimal. These themes are surely familiar for readers across the United
States and abroad.

Yancey's address also helps me put these seemingly disparate books---one
an essay collection on the profession of literary studies by a
well-regarded professor of literature and the other a monograph about
the rhetorical history and pedagogy of the English vernacular since the
Enlightenment by an esteemed professor emeritus in communication
studies---into conversation with each other and my current moment. They
invite me to think about the fields of literary studies and writing and
rhetoric studies to consider how these histories reflect current faculty
values and conversations and how they can inform future possibilities.
Guillory and Yancey offer divergent takes on this moment from their
different disciplinary commitments. Sproule's study of the rhetorics and
pedagogies of reading, writing, and speaking in school, home, and
business contexts is a helpful precursor to Yancey's interest in
teaching public screen literacies, following Elizabeth Daley's argument
that ``the screen is the language of the vernacular, that if we do not
include it in the school curriculum, we will become as irrelevant as
faculty professing in Latin.''\footnote{Kathleen Blake Yancey, ``Made
  Not Only in Words: Composition in a New Key,'' \emph{College
  Composition and Communication} 56, no. 2 (2004): 305.} According to
Yancey, Daley thinks screen literacies should be taught in media studies
programs to ``get in step with life practices,''\footnote{Yancey, 305.}
and Yancey, of course, wants to see writing studies take on this role,
too. In doing so, she updates Sproule's historicized observation that
rhetoric is central to the modern vernacular revolution for the
twenty-first century (2). And both Sproule and Yancey provide counter
histories and narratives to Guillory's belief that rhetoric died with
the rise of the vernacular and his equation of writing faculty with only
first-year writing, whom, he opines, lack a ``fully disciplinary
status'' (313). He envisions, in fact, doing away with first year
composition courses altogether in favor of a suite of extracurricular
means to elevate writing instruction out of writing studies altogether.

John Guillory's dense and ambitious four-hundred-page essay collection
offers a historical and sociological analysis and commentary on literary
studies as a profession. The collection, whose title is a twist on
Gerald Graff's \emph{Professing Literature}, takes as its primary
question ``What does it mean to `profess criticism?'\,'' (ix). By way of
answer, Guillory offers his sociological history of the rise of academia
as a profession, and he explores the historical arc of ``professing
criticism'' as the guiding force of literary studies. He maps the
emergence of the concept of professions and the rise of academic
specializations, and considers the history of criticism as the
formalization of the discipline from the nineteenth century public,
journalistic critic in conflict with the academic critic who emerged in
the 1920s to the rise of the scholar critic's primacy centered largely
on period specialization and close reading (56) and later followed by
the emergence of political topicality in the 1960s (74). He follows this
with a critique of how social criticism has served as a justification
and inspiration for the literature professoriate. He argues that
``literature needs to be \emph{recentered} by the literary professoriate
in order to reestablish its public claim to expertise'' (80), and he
wants this recentering to focus on the human record as the object of
study.

Guillory also establishes his overarching claim that the literary
professoriate is marked by its historical and ongoing formation and
deformation. In other words, Guillory understands the shaping of a
profession as a way of seeing the world that always also entails the
limits or biases on this specialized way of seeing. A key argument that
he revisits over the course of the book is that elevating literary
critique to social critique is an ``overestimation of aim,'' one that
``deforms the discipline'' (81). Guillory wants literary studies to
focus on records or documents as objects of study rather than using
texts for social critique. This is a provocative claim, particularly for
literature colleagues like mine who see social critique as a
foundational justification and method of doing literary criticism---of
moving discussions students encounter in novels, short stories, drama,
and poetry into broader social interpretation and critique. His critique
is also interesting considering the recommendation that our department's
program reviewers made last year that we are ``behind'' in having
sufficient DEI-inflected curricula to address low enrollment by
capturing more students' interest. This recommendation reflects a common
understanding within the discipline and the university that broader
social concerns are a legitimate part of literary criticism, and yet
outside of the university DEI commitments are currently under
significant attack by US Republicans at the state and federal levels.

Guillory's focus leads him to take up different facets and histories of
this dynamic relationship between disciplinary formation and
deformation, which allows him to concretize his own understanding of
literary studies. Guillory identifies and expands on Erwin Panofsky's
terms ``monument'' and ``document'' to argue that the humanities should
study ``records left by man,'' located in the dialectics of
documents/documentality and monuments/monumentality (118). He claims for
the humanities, including literary studies and presumably writing
studies and communications, what he sees as a more ``coherent''
description of what the humanities study rather than arguing about why
we should study them, which is how he frames social critique (110). He
directs readers towards the idea that the humanities center on ``the
study of a particular kind of object'' that ``calls to us across the
long time of human existence, exceeding by far the duration of any one
human life'' (123). In other words, Guillory is particularly oriented
towards archival and historical research methods. I'm not persuaded that
the ``what'' and ``why'' are so easily separated, and I'm also aware of
the limits of this claim for humanities scholarship that isn't centered
on written, historical texts, like communication scholar Natasha
Seegert's work at the intersections between humans and more-than-humans,
where the object of study might be coyote tracks located in human
habitats. Or the ways scholars in the humanities conduct quantitative
and qualitative research. The remaining chapters in this section take up
``dead'' or failed predecessors to literary studies, including rhetoric,
\emph{belles lettres}, and philology; historical locations of literature
relative to the sciences and the rise of the vernacular curriculum as
formative to the discipline; and a reflection on the decolonization of
global literature.

Guillory also focuses on four disciplinary problems that he sees
weighing down literary studies: graduate education, evaluation of
scholarship, first-year composition, and reading practices. Guillory is
less interested in solving problems than he is in raising questions and
observations that might inform future planning in English departments.
Guillory returns to the argument that the literature profession ought to
focus on human records---novels, drama, and poetry---as the objects of
study with their attendant reading practices, which for him entails
looking at the relationship between lay reading and professional reading
practices. He raises interesting questions about the reading practices
taught in literature classrooms, including whether the literature
classroom is a place for reading for pleasure or if professional reading
is meant to be work.

With its five rationales for literary study, the conclusion of
\emph{Professing Criticism} is meant to inspire hope for readers. What I
find most interesting is how Guillory's discussion affirms his belief
that literary study is ``a combination of positive knowledge and
cognitive training'' (348). That is, literary studies is ``at once a
kind of learning and the cultivation of an \emph{art},'' specifically
the arts of reading and writing. He argues that these arts were
``arguably the first versions of media studies in Western education''
and ``the deepest foundation for the future development of literary
study'' (355). This closing essay loops back effectively to the opening
discussion of the rise of professions, kinds of knowledge valued in this
shift, and their relationship to craft knowledge, or techne. I find this
discussion particularly interesting given Guillory's argument that the
``death'' of rhetoric as techne occurs in favor of the rise of
scientific disciplinary information as evidence. As a scholar of
feminist rhetorics, I find his claim of the death of rhetoric premature,
and narrowly conceived, even as I'm fascinated by his discussion of the
opposition between techne and information, since I understand rhetorical
studies as centering on knowledge acquisition and production. I can't
help but get the feeling he wants only the literature professoriate to
take on teaching the arts of writing and reading.

Ultimately, Guillory expresses his ``hope that this framework of
analysis will help to explain the perennial churn in literary study''
(ix), and the collection does this well, including underscoring how
disciplinary perspective and preferences play out in this ``churn.'' As
I read, I was regularly aware that Guillory is a privileged white male
and successful scholar of literary history; it's much easier to dismiss
rhetoric as dead, composition studies as a misguided literacy project,
and social literary critique as overreach from his position of relative
institutional power. I grant that the social aims of my literature
faculty have not elevated our department's cultural capital in or beyond
the university, but our department is an important place on campus where
students can engage different beliefs, experiences, and perspectives
from across the Anglophone world. Nonetheless, I encourage faculty,
graduate students, and professionals to read Guillory's collection
alongside academic books like \emph{Permanent Crisis: The Humanities in
a Disenchanted Age}\footnote{Paul Reitter and Chad Wellmon,
  \emph{Permanent Crisis: The Humanities in a Disenchanted Age}
  (University of Chicago Press, 2021).}, the numerous public articles
and essays proclaiming or challenging the demise of English majors
and/or the humanities, and the 2024 World Humanities Report,\footnote{https://worldhumanitiesreport.org/.}
but also to read other disciplinary perspectives on reading, writing,
speaking, and listening as well.

J. Michael Sproule's \emph{Democratic Vernaculars: Rhetorics of Reading,
Writing, Speaking and Criticism since the Enlightenment}, a 350-page,
impressive historical study of the evolution of Anglophone vernaculars
from the seventeenth century to the early decades of the twentieth
century, represents the kind of humanities scholarship Guillory values.
Sproule's meticulous historical scholarship centers closely on human
records, even as his argument and evidence effectively counter
Guillory's claim in ``The Postrhetorical Condition'' that the ``end of
rhetoric is concurrent with the extension of literacy to the populace as
a whole'' (129). Guillory goes on to write, ``Vernacularization is a
condition and a cause of the demise of rhetoric, a force undermining the
`dead languages' of antiquity that could not be resisted forever''
(131). Sproule's extensive study proves otherwise. Sproule argues that
``the modern theory, pedagogy, practice, and criticism of rhetoric
proves central to the vernacular revolution---and in no way represents
an epiphenomenon vis-à-vis grammar and literature'' (2). I recommend
reading Sproule's study in dialogue with Romeo Garcia and Damian Baca's
\emph{Rhetorics Elsewhere and Otherwise: Contested Modernities,
Decolonial Visions}, Jeffrey Ringer's \emph{Vernacular Christian
Rhetoric and Civil Discourse: The Religious Creativity of Evangelical
Student Writers}, and Carmen Kynard's \emph{Vernacular Insurrections:
Race, Black Protest, and the New Century in Composition-Literacies
Studies} to expand on the scope of theorizing and inclusivity of
vernacular studies.\footnote{Romeo Garcia and Damian Baca, eds.,
  \emph{Rhetorics Elsewhere and Otherwise: Contested Modernities,
  Decolonial Visions} (NCTE, 2019); Jeffrey M. Ringer, \emph{Vernacular
  Christian Rhetoric and Civil Discourse: The Religious Creativity of
  Evangelical Student Writers} (Routledge Press, 2018); Carmen Kynard,
  \emph{Vernacular Insurrections: Race, Black Protest, and the New
  Century in Composition-Literacies Studies} (State University of New
  York Press, 2013).} Nonetheless, Sproule's examination of 750 texts
related to speaking, writing, and reading in school, home, and business
is a robust and compelling study.

Sproule's introduction is critical in offering his framework and
argument; four principles he shares in the introduction form his
operating definition of rhetoric, namely that rhetoric is a process of
negotiating idea-communication and expression (aesthetics and
correctness) for different audiences for different purposes. His
plotting of the dimensions of his study along an x and y axis graph is a
useful frame. He describes the x-axis as ``denoting the span of
conceptual themes serving functional requisites---from simple spelling
and penmanship books to systematic and multifaceted treatises,'' while
the y-axis ``adds refinement by recognizing rhetoric's multilayered
applications in primary grades, upper grammar school, academy or high
school, college, home study, group or club, business-professional
office, and high-level scholarship'' (4). This graph effectively
illustrates the depth and breadth of Sproule's study and supports his
argument that Anglophones ``built vernacular competence over the course
of their studies by assimilating genres'' rhetorically (4). The third
key principle that establishes rhetoric's centrality to the vernacular
is found in the dialectical relationship he traces between
idea-communication and expression across a number of texts, a concept he
attributes to Locke and traces across other influential figures (2).
Sproule writes, ``From grammar rhetoric derives its expression criterion
of \emph{correctness}, and from literature rhetoric absorbs values
respecting \emph{beauty} of expression'' (3). Finally, Sproule reminds
readers that ``Theorists and teachers of the post-Lockean vernacular
redeployed classical-era concepts, chiefly from Cicero and Quintilian,
to such an extent that modern rhetoric represents a continuation rather
than a full revolution'' (13). In other words, the clean break from
classical rhetoric Guillory envisions is neither clean nor complete.

The fourth of Sproule's ten operating principles identifies gaps in
rhetoric's ``familiar archive of texts,'' including ``the working class,
white women, and diverse ethnicities'' (6). While Sproule briefly
addresses feminist rhetorical studies across his book, much more study
needs to be done in the areas Sproule mentions, but also in those he
doesn't, including sites like agriculture, domestic work, and by second
language users, to name a few. While I sometimes lost sight of Sproule's
larger arguments and principles within the nitty gritty of
chapters---it's hard to see the forest for all the trees---and in the
ways some of the relations between sections are glossed (likely due
to the scope of the study), it's clear that vernaculars---as dynamic and
adaptable forms of language---have played a critical role in shaping
modern rhetoric in practical and intellectual forms. The remaining
principles Sproule offers address themes of incompletion and
underappreciation to underscore areas needing additional scholarly
attention, all derived as ways of more fully understanding rhetoric's
historical relevance, whether that means identifying the influences of
Cicero and Quintilian or that of postmodern rhetorical studies.

Sproule locates his ten principles in different Anglophone texts and
contexts across the seventeenth to early twentieth centuries. He
establishes John Locke's ``central{[}ity{]} to the
communicative-rhetorical tangent of English-language vernaculars'' (24),
as well as the influences of other figures on the modern vernacular.
This he follows with a textual study of how ``Lockean idea-communication
intersected with classically inflected \emph{belles lettres} to create a
framework for new vernaculars in contexts scholarly, pedagogical,
professional, and social'' (85). Sproule then shifts the domain of study
beyond ``formal school'' to public contexts like debating societies, the
home, and business, where one can ``hone skills of conversation,
letter-writing, grammar, and speaking'' (163). For example, he surveys
the uses of advice books, professional handbooks for secretaries, debate
manuals, and manuals for extemporaneous speaking to trace language use
via idea-communication and expression. Sproule then surveys the
relationship of communication and expression across the genres of
composition-rhetoric textbooks, composition handbooks for writing (e.g.,
the popular \emph{The Elements of Style}) and oratory (e.g., \emph{The
Art of Oratorical Composition}), handbooks for newspaper reporting to
complement on-the-job learning, and criticism, providing an additional
discussion of the professional versus academic critic. From here,
Sproule more narrowly centers on elocution. One interesting tension
Sproule points to at the turn of the century is between the disciplines
of speech and English, ``the expressive culture of elocution'' and ``an
ever more powerful English-studies establishment for whom the vernacular
apex amounted to critical appreciation of great works silently read''
(271). Sproule's discussion of literary criticism alongside public
speaking explores an interesting disciplinary relationship to English
departments, as Guillory doesn't significantly consider the discipline
of communication studies. Sproule looks across the eighteenth and
nineteenth centuries at the place of eloquence in communication, largely
emphasizing it as a site of ``vernacular quality and language efficacy''
in oratory (314) displaced by rhetorical, literary, and cultural
criticism. These chapters are densely detailed and descriptive, although
the book ends with a more sweeping gesture towards the importance of
postmodern perspectives on vernaculars.

\enlargethispage{\baselineskip}

While Guillory proclaims the death of rhetoric, and the attendant
failures he perceives in first-year composition (he hesitates to call it
first-year writing), Sproule reminds us that we can also look to
communication studies as a profession for studying the humanities and
how the arts of writing, reading, and speaking circulate in contemporary
vernaculars. These texts are both valuable for what they do accomplish
as well as what they don't accomplish, pushing readers to reflect on
disciplinary formations and deformations to inspire our reflection on
our own departments, our own professions, and our own moments, our
nowness. From Guillory I better understand my literature faculty and
their desires to elevate social critique and distinguish between the
reading they want from students versus lay reading practices. I also see
reflected in his book the ways they devalue first-year writing and the
writing major, imagining these programs as largely instrumental or
workplace-oriented rather than related to the arts they profess.
Sproule's monograph is in many ways a comfort because it points to the
duration, functionality, and aesthetic dimension of rhetorical pedagogy
and practice in diverse academic and public sites. Rereading Yancey's
essay now, I'm struck by how exciting it was for me back then, as a new
assistant professor of rhetoric and writing, listening as Kathy Yancey
identified a moment for promoting exciting professional changes in
college writing instruction, including the establishment of writing
majors. While I'm more jaded now, over twenty years later as a
department chair and a faculty member in a writing major, I'd still like
us to see how we might jointly contribute to teaching the arts of
reading, writing, and speaking---creating coalitions within and across
the humanities---rather than trying to shore up our siloed departments
or professions.




\section{Bibliography}\label{bibliography}

\begin{hangparas}{.25in}{1} 



Brooke, Collin. ``Weblogs as Deictic Systems: Centripetal, Centrifugal,
and Small-World Blogging.'' \emph{Computers and Composition Online}
(Fall 2005). \url{http://cconlinejournal.org/brooke/index.html}.

Garcia, Romeo, and Damian Baca, eds. \emph{Rhetorics Elsewhere and
Otherwise: Contested Modernities, Decolonial Visions}. NCTE, 2019.

Guillory, John. \emph{Professing Criticism: Essays on the Organization
of Literary Study.} University of Chicago Press, 2022.

Kynard, Carmen. \emph{Vernacular Insurrections: Race, Black Protest, and
the New Century in Composition-Literacies Studies}. State University of
New York Press, 2013.

Reitter, Paul, and Chad Wellmon. \emph{Permanent Crisis: The Humanities
in a Disenchanted Age}. University of Chicago Press, 2021.

Ringer, Jeffrey M. \emph{Vernacular Christian Rhetoric and Civil
Discourse: The Religious Creativity of Evangelical Student Writers}.
Routledge Press, 2018.

Sproule, J. Michael. \emph{Democratic Vernaculars: Rhetorics of Reading,
Writing, Speaking, and Criticism since the Enlightenment}. Routledge
Press, 2020.

Yancey, Kathleen Blake. ``Made Not Only in Words: Composition in a New
Key.'' \emph{College Composition and Communication} 56, no. 2 (2004):
297--328.



\end{hangparas}


\end{document}