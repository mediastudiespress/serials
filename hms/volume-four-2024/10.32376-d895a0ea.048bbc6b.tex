% see the original template for more detail about bibliography, tables, etc: https://www.overleaf.com/latex/templates/handout-design-inspired-by-edward-tufte/dtsbhhkvghzz

\documentclass{tufte-handout}

%\geometry{showframe}% for debugging purposes -- displays the margins

\usepackage{amsmath}

\usepackage{hyperref}

\usepackage{fancyhdr}

\usepackage{hanging}

\hypersetup{colorlinks=true,allcolors=[RGB]{97,15,11}}

\fancyfoot[L]{\emph{History of Media Studies}, vol. 4, 2024}


% Set up the images/graphics package
\usepackage{graphicx}
\setkeys{Gin}{width=\linewidth,totalheight=\textheight,keepaspectratio}
\graphicspath{{graphics/}}

\title[Media Imperialism]{Media, Intellectual, and Cultural Imperialism Today} % longtitle shouldn't be necessary

% The following package makes prettier tables.  We're all about the bling!
\usepackage{booktabs}

% The units package provides nice, non-stacked fractions and better spacing
% for units.
\usepackage{units}

% The fancyvrb package lets us customize the formatting of verbatim
% environments.  We use a slightly smaller font.
\usepackage{fancyvrb}
\fvset{fontsize=\normalsize}

% Small sections of multiple columns
\usepackage{multicol}

% Provides paragraphs of dummy text
\usepackage{lipsum}

% These commands are used to pretty-print LaTeX commands
\newcommand{\doccmd}[1]{\texttt{\textbackslash#1}}% command name -- adds backslash automatically
\newcommand{\docopt}[1]{\ensuremath{\langle}\textrm{\textit{#1}}\ensuremath{\rangle}}% optional command argument
\newcommand{\docarg}[1]{\textrm{\textit{#1}}}% (required) command argument
\newenvironment{docspec}{\begin{quote}\noindent}{\end{quote}}% command specification environment
\newcommand{\docenv}[1]{\textsf{#1}}% environment name
\newcommand{\docpkg}[1]{\texttt{#1}}% package name
\newcommand{\doccls}[1]{\texttt{#1}}% document class name
\newcommand{\docclsopt}[1]{\texttt{#1}}% document class option name


\begin{document}

\begin{titlepage}

\begin{fullwidth}
\noindent\Large\emph{History of Communication Studies across the Americas
} \hspace{18mm}\includegraphics[height=1cm]{logo3.png}\\
\noindent\hrulefill\\
\vspace*{1em}
\noindent{\Huge{Media, Intellectual, and Cultural\\\noindent Imperialism Today\par}}

\vspace*{1.5em}

\noindent\LARGE{Afonso de Albuquerque} \href{https://orcid.org/0000-0002-2608-7605}{\includegraphics[height=0.5cm]{orcid.png}}\par\marginnote{\emph{Afonso de Albuquerque, ``Media, Intellectual, and Cultural Imperialism Today,'' \emph{History of Media Studies} 4 (2024), \href{https://doi.org/10.32376/d895a0ea.048bbc6b}{https://doi.org/ 10.32376/d895a0ea.048bbc6b}.} \vspace*{0.75em}}
\vspace*{0.5em}
\noindent{{\large\emph{Universidade Federal Fluminense}, \href{mailto:afonsoalbuquerque@id.uff.br}{afonsoalbuquerque@id.uff.br}\par}} \marginnote{\href{https://creativecommons.org/licenses/by-nc/4.0/}{\includegraphics[height=0.5cm]{by-nc.png}}}

% \vspace*{0.75em} % second author

% \noindent{\LARGE{<<author 2 name>>}\par}
% \vspace*{0.5em}
% \noindent{{\large\emph{<<author 2 affiliation>>}, \href{mailto:<<author 2 email>>}{<<author 2 email>>}\par}}

% \vspace*{0.75em} % third author

% \noindent{\LARGE{<<author 3 name>>}\par}
% \vspace*{0.5em}
% \noindent{{\large\emph{<<author 3 affiliation>>}, \href{mailto:<<author 3 email>>}{<<author 3 email>>}\par}}

\end{fullwidth}

\vspace*{1em}


\hypertarget{abstract}{%
\section{Abstract}\label{abstract}}

Cultural imperialism was once the subject of a vibrant debate in
international scholarship. Yet, the debate on cultural imperialism has
lost much of its previous influence and centrality. This does not mean
that cultural imperialism has lost its relevance. On the contrary,~in
the wake of the neoliberal globalization process, cultural imperialism
is now stronger than ever. This article argues that cultural imperialism
comprises two dimensions: media imperialism and intellectual
imperialism, and it is important to understand how they interact. To
illustrate how their interplay works and what consequences follow, the
article examines how US academic institutions educated and organized
Brazilian media elites who helped to legitimate Lava Jato, a politically
driven judicial operation that led to the downgrade of Brazilian
democracy, and ultimately paved the way to the rise of Jair Bolsonaro to
the presidency.

\hypertarget{resumen}{%
\section{Resumen}\label{resumen}}

En un tiempo el imperialismo cultural fue tema de debate acalorado en
los estudios internacionales. Sin embargo, dicho debate ha perdido gran
parte de su anterior influencia~y centralidad, lo que no quiere decir
que el imperialismo cultural carezca hoy de relevancia; al contrario,
como secuela del proceso neoliberal global se ha fortalecido más que
nunca. En este artículo se argumenta que el imperial-

\vspace*{2em}

\noindent{\emph{History of Media Studies}, vol. 4, 2024}

\noindent ismo cultural abarca
dos dimensiones---el imperialismo mediático y el imperialismo
intelectual---y que es importante entender cómo interactúan. Para
ilustrar dicha interacción y sus consecuencias, se analiza cómo las
instituciones académicas estadounidenses formaron y organizaron a las
élites mediáticas brasileñas que ayudaron a legitimar Lava Jato, una
operación jurídica con fines políticos que terminó degradando la
democracia brasileña y allanando el camino para que Jair Bolsonaro
llegara a la presidencia.


\enlargethispage{2\baselineskip}

\vspace*{2em}




 \end{titlepage}

% \vspace*{2em} | to use if abstract spills over

\newthought{From the 1960s} to the 1980s, cultural imperialism was the subject of a
vibrant debate in international scholarship. At that time, intellectuals
and activists around the world raised concerns about the threat posed by
the US media to other cultures. In response, UNESCO published a report
in 1980 proposing a New World Information and Communication Order.
However, attention to this topic has declined sharply since then. To be
sure, cultural imperialism is still a topic of interest in the
intellectual milieu, especially with respect to media
imperialism\footnote{Oliver Boyd-Barrett, ``Cultural Imperialism and
  Communication,'' \emph{Oxford Research Encyclopedia of Communication},
  June 2018; Oliver Boyd-Barrett and Tanner Mirrlees, eds., \emph{Media
  Imperialism: Continuity and Change} (Lanham, MA: Rowman \&
  Littlefield, 2019); and Kaarle Nordenstreng, ``How the New World Order
  and Imperialism Challenge Media Studies,'' \emph{tripleC:
  Communication, Capitalism \& Critique} 11, no. 2 (2013).} and, more
recently, platform imperialism.\footnote{Stuart Davis, ``What is Netflix
  Imperialism? Interrogating the Monopoly Aspirations of the `World's
  Largest Television Network,'\,'' \emph{Information, Communication, and
  Society} 26, no. 6 (2023); and Dal Y. Jin, \emph{Digital Platforms,
  Imperialism, and Political Culture} (London: Routledge, 2015).} Yet,
the debate on cultural imperialism has lost much of its previous
influence and centrality. A new batch of emerging concepts, such as
``Americanization,''\footnote{Jeremy Tunstall, \emph{The Media are
  American: Anglo-American Media in the World} (London: Constable,
  1977).} ``globalization,''\footnote{John Tomlinson, \emph{Cultural
  Imperialism: A Critical Introduction} (London: Continuum, 1991).} and
``asymmetric interdependence''\footnote{Joseph D. Straubhaar, ``Beyond Media
  Imperialism: Asymmetrical Interdependence and Cultural Proximity,''
  \emph{Critical Studies in Mass Communication} 8, no. 1
  (1991).}
came to replace cultural imperialism as an analytical tool for
describing the cultural influence exerted by powerful countries over
others. Furthermore, the ties between cultural imperialism scholarship
and anti-imperialism activism are much weaker now than in the past.
Finally, present-day Anglophone scholars (in many cases working from the
same countries described as cultural imperialists) exert a leading role
in the international scholarship on cultural imperialism.\footnote{Manuel
  B. Aalbers, ``Creative Destruction through the Anglo-American
  Hegemony: A Non-Anglo-American View on Publications, Referees, and
  Language,'' \emph{Area} 36, no. 3 (2004); and Afonso de Albuquerque,
  ``The Institutional Basis of Anglophone Western Centrality,''
  \emph{Media, Culture \& Society} 43, no. 1 (2021).} There is some
irony in that.

This article proposes that this decline in the scholarly interest in
cultural imperialism did not happen because cultural imperialism itself
lost relevance in the real world and ceased to exist. Instead, I argue
just the opposite: that at present, in the wake of the neoliberal
globalization process,\footnote{Efe C. Gürkan, \emph{Imperialism after
  the Neoliberal Turn} (London: Routledge, 2022); and David Harvey,
  \emph{The New Imperialism} (Oxford: Oxford University Press, 2003).}
cultural imperialism is stronger than ever. Since the 1990s, the logic
of cultural imperialism has penetrated deeply into international
scholarship, allowing western (especially US) institutions to set the
standards for global scholarship. These circumstances contributed
significantly to discouraging research on cultural imperialism. I aim
here to discuss how this has happened and provide evidence of some
contemporary features of cultural imperialism.

To fully understand this statement, it is necessary to precisely define
cultural imperialism. Here, the first step is to note that two scholarly
traditions have used the term to describe different phenomena.
\emph{Intellectual} imperialism refers to the means that certain
countries employ to exert intellectual dominance over others. This
effort is primarily directed at the elites of less powerful countries.
Intellectual imperialists try to convince these elites about the
intellectual (and moral) superiority of their own societies, which are
presented as role models for others.\footnote{Syed Farid Alatas,
  ``Intellectual Imperialism: Definition, Traits, and Problems,''
  \emph{Southeast Asian Journal of Social Science} 28, no. 1 (2000); and
  Afonso de Albuquerque, ``Transitions to Nowhere: Western Teleology and
  Regime-Type Classification,'' \emph{International Communication
  Gazette} 85, no. 6 (2023).} This can be accomplished via high culture,
the academic apparatus, and think tanks. The other phenomenon is
\emph{media} imperialism, which employs the media as a resource to
conquer the hearts and minds of the public in general and influence
public opinion. Although scholars sometimes refer to both as ``cultural
imperialism,'' concretely the dialogue between them is quite limited.

This article proposes a new approach to cultural imperialism that
considers the interplay between intellectual and media imperialisms.
Based on this, I identify an emerging dynamic of media imperialism
operating nowadays. Here, media imperialism does not primarily have to
do with exporting media content from imperialistic countries to others.
Rather, this model of cultural imperialism works through education and
network-building. Its key purpose is to prepare the local media elites
and professionals to act as local agents, echoing the cultural
imperialistic agenda. This article pays special attention to how the
Knight Center for Journalism in the Americas has promoted educational
initiatives and social networks aimed at modeling Brazilian journalistic
practices and worldviews in accordance with US interests. The case of
Lava Jato gives us a concrete example of how US interference in
Brazilian politics engaged the cultural imperialism networks.

\hypertarget{cultural-imperialism-media-imperialism-and-intellectual-imperialism}{%
\section{Cultural Imperialism, Media Imperialism, and Intellectual\\\noindent Imperialism}\label{cultural-imperialism-media-imperialism-and-intellectual-imperialism}}

What do cultural imperialism, media imperialism, and intellectual
imperialism have in common? What distinguishes them? The definitions of
these terms are mostly imprecise. Cultural imperialism has served as a
synonym for both intellectual imperialism\footnote{Pierre Bourdieu and
  Loic Wacquant, ``On the Cunning of the Imperialist Reason,''
  \emph{Theory, Culture \& Society} 16, no. 1 (1999).} and media
imperialism.\footnote{Oliver Boyd-Barrett, \emph{Media Imperialism} (Los
  Angeles: SAGE, 2015).} However, very few studies take intellectual
imperialism and media imperialism together as different aspects of a
broader cultural imperialism phenomenon. For this reason, this section
offers a conceptual framework for these concepts.

The most important aspect to consider is that they are all particular
features of a more general phenomenon: imperialism. It is worth noting
that much of the research on media imperialism has ignored this basic
lesson. There are different ways to approach imperialism. Lenin's
definition of it as ``the highest stage of capitalism'' is very popular
among many scholars. However, it is not particularly useful for our
purposes, as it focuses mainly on the economic aspects of imperialist
relations. By contrast, this article emphasizes the political aspects of
imperialism, more related to the power exerted by some societies over
others. Both state and non-state agents perform important roles in this
respect.

First of all, some conceptual clarification is needed. Recently,
international scholarship has given a lot of attention to the need to
decolonize media and communication research.\footnote{Antje Gluck,
  ``De-Westernization and Decolonization in Media Studies,''
  \emph{Oxford Research Encyclopedia of Communication}, December 2018;
  and Last Moyo, \emph{The Decolonial Turn in Media Studies in Africa
  and the Global South} (New York: Springer, 2020).} Colonialism and
imperialism focus on the same problem, but from different viewpoints.
Literature on colonialism often focuses on the societies that fell
victim to colonization. In most cases, it is assumed that colonialism is
an event that took place in the past but still exerts a negative
influence on the societies that experienced it. In particular, the
literature on decolonization primarily associates colonialism with
Western European nations.\footnote{W. D. Mignolo, \emph{The Darker Side
  of Western Modernity: Global Futures, Decolonial}

  \emph{Options} (Durham, NC: Duke University Press, 2011); and Edward
  Said, \emph{Orientalism} (London: Penguin Books, 1978).} In other
cases, literature on imperialism focuses on the colonizers. Imperialism
derivates from ``empire'' but looks at it from a negative perspective.
Very often, it refers to societies that act as empires, although they
pretend not to be so.\footnote{Krishan Kumar, ``Colony and Empire,
  Colonialism and Imperialism: A Meaningful Distinction?''
  \emph{Comparative Studies in History and Society} 63, no. 2 (2021).}
Since the end of World War II, most references to imperialism have been
directed to the United States.\footnote{Ariel Dorfman and Armand
  Mattelart, \emph{Para leer al Pato Donald: Comunicación de Masa y
  Colonialismo} (Buenos Aires: Siglo Veinteuno, 1971); and Herbert I.
  Schiller, \emph{Mass Communications and the American Empire} (New
  York: A. M. Kelley, 1969).} It is worth noting that, recently, the
decolonial rhetoric has been appropriated by agents associated with the
US industrial-military apparatus as a tool for legitimizing US interests
in the international arena. A recent example comes from the article
``Decolonize Russia,'' written by Casey Michel and published in
\emph{The Atlantic}, a vehicle closely related to NATO. Michel argues
that Russia obtained its huge territory by colonizing other people,
which is true. He proposes that, to contain its aggression (as recently
demonstrated in Ukraine), it would be necessary to fragment Russia into
many independent countries.\footnote{Casey Michel, ``Decolonize
  Russia,'' \emph{The Atlantic}, May 27, 2022,
  \href{https://www.theatlantic.com/ideas/archive/2022/05/russia-putin-colonization-ukraine-chechnya/639428/.}{https://www.theatlantic.com/ideas/\\\noindent archive/2022/05/russia-putin-colonization-ukraine-chechnya/639428/}.}
The ``decolonize Russia'' agenda has been given voice in several other
sources, including the UK's BBC.\footnote{``Decolonising Russia,''
  \emph{Sounds}, BBC Radio 4, February 27, 2024, radio broadcast, 28:00,
  \url{https://www.bbc.co.uk/programmes/m001wq4c}.} However, the very same
argument applies to the United States as well. The US also expanded its
territory at the expense of native people, and its foreign policy is
indeed much more aggressive than Russia's. This suggests that the
``decolonial'' rhetoric has been appropriated by the US
military-industrial complex and has thus become a tool for the US
imperialistic agenda.

Cultural imperialism refers to efforts aimed at providing cultural
legitimacy to imperialism. It aims to make the population of the society
subjected to imperialism arrangements---or at least some fractions of
it---accept the imperialism domination as unavoidable or even desirable.
They do this by presenting the culture of the imperialist countries as
being essentially superior to the culture of the countries submitted to
them. Cultural imperialism employs different approaches to reaching this
objective.

Intellectual imperialism refers to efforts targeting the intellectual
elites. The international scholarship system is one of the most
important grounds on which intellectual imperialism operates.
Universities are an important part of this system, but other
institutions, such as academic ranking systems, the ``quality journals''
system, and funding institutions, among others, are very important,
too.\footnote{Albuquerque, ``The Institutional Basis''; and Simon
  Marginson and Marjikvan der Wende, ``To Rank or to Be Ranked: The
  Impact of Global Rankings in Higher Education,'' \emph{Journal of
  Studies in International Education} 11, no. 3/4 (2007).} In a general
manner, this system pre-determines who is able to formulate
``world-class'' theories and who is not.

\newpage Media imperialism works differently. It aims to conquer the hearts and
minds of common people through the diffusion of media content from some
countries to others. In this case, cultural influence is exerted through
subtle means. Media products do not frequently convey explicit political
messages. Rather, they work by deeming a universal status for
culture-specific values and aesthetic forms.

\hypertarget{intellectual-imperialism}{%
\section{Intellectual Imperialism}\label{intellectual-imperialism}}

Syed Hussein Alatas defines intellectual imperialism as ``the domination
of one people by another in their world of thinking.''\footnote{Alatas,
  ``Intellectual Imperialism.''} He contends that intellectual
imperialism has six main characteristics:

\begin{enumerate}
\item
  \textbf{Exploitation}: peripheral societies provide raw data for the
  central ones, and these process and manufacture them in the form of
  books and articles.
\item
  \textbf{Tutelage}: the knowledge produced and consumed in the
  peripheries is assumed to be dependent on the knowledge produced by
  central societies.
\item
  \textbf{Conformity}: people working in the peripheries are compelled
  to adhere to theories produced by imperialist societies.
\item
  \textbf{Uneven division of intellectual work}: When they collaborate
  on academic projects, scholars from peripheral societies work in
  secondary roles. For instance, they provide raw data from their
  countries for comparative studies.
\item
  \textbf{Rationalization}: Imperialism justifies itself as a civilizing
  mission.
\item
  \textbf{Mediocrity}: Most often, scholars from the central societies
  who work in the colonies are not among the central societies' most
  brilliant.
\end{enumerate}

But how does intellectual imperialism work? A vast literature has
described how, after the end of World War II, the United States exerted
intellectual imperialism worldwide.\footnote{Robert F. Arnove, ed.,
  \emph{Philanthropy and Cultural Imperialism: The Foundations at Home
  and Abroad} (Boston: G.K. Hall, 1980); Teresa Hayler, \emph{Aid as
  Imperialism} (Middlesex, UK: Penguin Books, 1971); and Inderjeet
  Parmar, ``The Big 3 Foundations and American Global Power,''
  \emph{American Journal of Economics and Sociology} 74, no. 4 (2015).}
The United States took advantage of the poor conditions in Western
Europe to become the new intellectual hegemon. Not only did the United
States replace the Western European colonial powers as the main
intellectual reference for the majority world,\footnote{The concept of
  majority world was first presented by Shahidul Alam, as a manner to
  refer to the societies existing beyond the Western World. Contrary} but it exerted growing influence on
Western Europe, too. Naturally, this process did not happen overnight.

``Philanthropic foundations'' exerted a central role in this process.
Since the beginning of the twentieth century, they have been key actors
in US intellectual development.\textsuperscript{21} After World War II, they expanded the scope of their
actions worldwide. They distribute large amounts of\marginnote{to
  the perception implied by western-centered approaches and
  terminologies, these societies comprise the majority of humankind. By
  calling attention to this demographic reality, Alam's term highlights
  absurdity of a status quo in which a very small fraction of the human
  population presents itself as representing the highest values of the
  humankind as a whole. See Shahidul Alam, ``Majority World: Challenging
  the West\textquotesingle s Rhetoric of Democracy,'' \emph{Amerasia
  Journal} 34, no. 1: 87--98.} money\marginnote{\textsuperscript{21}\setcounter{footnote}{21} Parmar, ``The Big 3
  Foundations.''} to scholars
and intellectuals to stimulate research on certain topics, using certain
theoretical perspectives and methodologies.\textsuperscript{22} By doing this, they exert a tremendous influence on
setting the agenda of the intellectual debate.

The academic infrastructure providing the United States with an enormous
advantage in setting the agenda of the global intellectual debate has
grown in sophistication since the 1990s. After the demise of the Soviet
Union and the Central and Eastern European regimes associated with it,
the United States became the uncontested leader of a unipolar global
order. Acting together with international financial institutions such as
the World Bank and the International Monetary Fund, the United States
promoted\marginnote{\textsuperscript{22}\setcounter{footnote}{22} Bruce Cumings,
  ``Boundary Displacement: Area Studies and International Studies during
  and after the Cold War,'' \emph{Bulletin of Concerned Asian Scholars}
  29, no. 1 (1997).} the setting of a new global order based on neoliberal
principles.\footnote{John Williamson, ``A Short History of the
  Washington Consensus,'' in \emph{The Washington Consensus
  Reconsidered: Towards a New Global Governance}, ed. Narcis Serra and
  Joseph E. Stiglitz (Oxford: Oxford University Press,
  2008).}

A concrete consequence of this move was the restructuring of the entire
logic of international scholarship around the principles of academic
capitalism.\footnote{Sheila Slaughter and Gary Rhoades, \emph{Academic
  Capitalism and the New Economy: Markets, State and Higher Education}
  (Baltimore: Johns Hopkins University Press, 2004).} According to these
principles, academic institutions must compete for resources to sustain
teaching and research based on their efficiency. A global academic
ranking system provides the basis for measuring that efficiency. The
United States controls most of the ranking institutions, and
unsurprisingly, its academic institutions and scholars occupy the most
important positions in these rankings. Other western Anglophone
countries also benefit from this system.\footnote{Aalbers, ``Creative
  Destruction''; and Albuquerque, ``The Institutional Basis.''} Beyond
this there is a series of unspoken rules that minimize the chances of
scholars working from the Global South being published in the ``world
class'' journals and having their work cited, among other
aspects.\footnote{Paula Chakravartty et al., ``\#CommunicationSoWhite,''
  \emph{Journal of Communication} 68, no. 2 (2016); and Marton Demeter,
  \emph{Academic Knowledge Production and the Global South} (London:
  Palgrave-Macmillan, 2020).}

This structural advantage provides the United States with the power to
``universalize particularisms linked to its singular historical
tradition.''\footnote{Bourdieu and Wacquant, ``On the Cunning of the
  Imperialist Reason.''} This has allowed the United States to exert
huge normative influence over the rest of the world. In this scenario,
the US institutions become the model to follow. Very often, the ruling
elites of other countries are educated and socialized in US
universities.\footnote{Marion Fourcade, ``The Construction of a Global
  Profession: The Transnationalization of Economics,'' \emph{American
  Journal of Sociology} 112, no. 1 (2006).} In the process, they learn
to look at the world and their own societies through US
lenses.\footnote{Yves Dezalay and Bryant G. Garth, \emph{The
  Internationalization of Palace Wars: Lawyers, Economists and the
  Contest to Transform Latin American States} (Chicago: University of
  Chicago Press, 2002).}

\hypertarget{media-imperialism}{%
\section{Media Imperialism}\label{media-imperialism}}

The debate on media imperialism originated in the 1960s in Latin
America. There are concrete reasons for this to happen. Being considered
the US's backyard, Latin America was the first region of the world to
fully experience the power of US media imperialism. During World War II,
the United States began to employ its media power as a resource for
exerting diplomatic influence on Latin American countries. For Latin
Americans, cultural imperialism was a matter of activism rather than
mere intellectual interest. They reacted to what they perceived as a
threat to their native cultures posed by a foreign agent.\footnote{Dorfman
  and Mattelart, \emph{Para leer al Pato Donald}; and Luis Ramiro
  Beltrán, ``TV Etchings in the Minds of Latin Americans: Conservative,
  Materialism, and Conformism,'' \emph{International Communication
  Gazette} 24, no. 1 (1976).}

The initial debate on media imperialism drew on dependency theory.
Dependency theory emerged as a critical response to the US-promoted
modernization theory, a theoretical standpoint that privileged the
United States as a model for the development of societies belonging to
the majority world.\footnote{Peter Simonson, Jefferson Pooley, and David
  W. Park, ``The History of Communication Studies Across the Americas: A
  View from the United States,'' \emph{MATRIZes} 17, no. 3 (2023).} Many
of these societies had until recently been colonies of European
countries. According to modernization theory, these societies (rebranded
as underdeveloped or developing societies) should follow the script
prescribed by the United States to develop.\footnote{Zaheer Baber,
  ``Modernization Theory and the Cold War,'' \emph{Journal of
  Contemporary Asia} 31, no. 1 (2001).} Otherwise, for dependency
theory, the main consequence of the policies proposed by modernization
theory is an increasing dependence of majority world societies on the
West. According to Latin American critics, the massive export of US
media content to the majority world was a tool for producing cultural
dependency.

The early debate on media imperialism was not limited to the academic
milieu. Indeed, the activism in which Latin American intellectuals
engaged, together with colleagues from the entire world, had practical
results.\footnote{Florencia Enghel and Martín Becerra, ``Here and There:
  (Re)Situating Latin America in International Communication Theory,''
  \emph{Communication Theory} 28, no. 2 (2018).} Ultimately, it led
UNESCO's MacBride Commission to publish a document proposing a New World
Information and Communication Order (NWICO). The US government and media
reacted angrily and left UNESCO in protest against the NWICO
proposals.\footnote{Abu Bhuyan, \emph{Internet Governance and the Global
  South: A Demand for a New Framework} (London: Palgrave MacMillan,
  2014).} In the following decades, the debate on media imperialism lost
much of the visibility it had achieved. Several factors contributed to
this decline. First, the consolidation of the new system of
international scholarship---described above---contributed to erasing
Latin American voices in favor of US and Anglophone ones.\footnote{Albuquerque,
  ``The Institutional Basis.''} This means the authority to speak about
media imperialism was transferred to scholars working in the very same
countries that the Latin American intellectuals accused of perpetrating
media imperialism. In other words, the emerging neoliberal system of
intellectual imperialism had a decisive impact on the debate around
media imperialism.

Although the initial focus of Anglophone studies on media imperialism
was closer to that of its Latin American counterparts---this is the case
of the work of Herbert Schiller,\footnote{Schiller, \emph{Mass
  Communications and the American Empire}.} for instance---it soon
followed its own path. Progressively, studies on media imperialism began
to emphasize the inequality in the production of media content in the
international arena,\footnote{\hypertarget{straubhaar-beyond-media-imperialism.}{%
  \section{Straubhaar, ``Beyond Media
  Imperialism.''}\label{straubhaar-beyond-media-imperialism.}}} to the
detriment of the idea that media imperialism is a particular aspect of a
more general phenomenon.

Moreover, as we have seen, the United States at that time was rapidly
becoming the unipolar hegemon of the world. In other words, US
imperialism reached its peak as it became able to force other countries
to adopt neoliberal reforms. These reforms in turn shaped different
aspects of their societies: economy,\footnote{Fourcade, ``Construction
  of a Global Profession.''} law,\footnote{Alvaro Santos, ``The World
  Bank's Uses of the `Rule of Law' Promise in Economic Development,'' in
  \emph{New Law and Economic Development: A Critical Appraisal,} ed.
  David. M. Trubek and Alvaro Santos (Cambridge: Cambridge University
  Press, 2006).} political organization,\footnote{Michael Christensen,
  ``Interpreting the Organizational Practices of North American
  Democracy Assistance,'' \emph{International Political Sociology} 11,
  no. 2 (2017).} and no less importantly, the academic
milieu.\footnote{Slaughter and Rhoades, \emph{Academic Capitalism}.}

Therefore, it is no surprise that, at that time, speaking about
imperialism was not a popular topic. Instead, new intellectual models
emerged to provide intellectual justification for the US-centered global
order. One of the most influential ideologues of this new order was
Joseph Nye, Jr. In a book affirming that the United States was ``bound
to lead'' the global order, he coined the concept of ``asymmetrical
interdependence'' to describe the new global order.\footnote{Straubhaar, ``Beyond Media
  Imperialism.''} This
logic also made its mark on studies of media imperialism. In fact,
Joseph Straubhaar borrowed from Nye the concept of ``asymmetrical
interdependence'' to justify why ``media imperialism'' was not a useful
concept anymore.\footnote{Joseph S. Nye, Jr., \emph{Bound to Lead: The
  Changing Nature of American Power} (New York: Basic Books, 1990).} His
core argument is that, in the last few decades, numerous countries have
begun to export their media products, too. A core example refers to
Brazil's Rede Globo de Televisão (Globo Television Network), which
exported its telenovelas to other countries with considerable success.

What this argument loses sight of is that, contrary to what happened to
the United States, the export of media products was not an expression of
pre-existing patterns of imperialistic dominance. There is a rich
literature on how Hollywood has systematically echoed the US political
agenda in their depictions of the military, for instance.\footnote{Stacy Takacs, ``The US Military as Cold War
  Programmer,'' \emph{Journal of Popular Culture} 50, no. 3
  (2017).}
Added to this, the United States has often used economic and political
pressure to force other countries to open their markets for US-produced
media content.\footnote{Paul Moody, ``Embassy Cinema: What WikiLeaks
  Reveals about US State Support for Hollywood,'' \emph{Media, Culture
  \& Society} 39, no. 7 (2017).} The manner in which the United States
dealt with the NWICO document provides vivid evidence in this respect.
To be sure, nothing similar happened in other countries.

Recently, there has been a new wave of attention regarding media
imperialism, specifically with reference to ``platform
imperialism.''\footnote{Jin, \emph{Digital Platforms.}} It is worth
pointing out that platforms and media are not exactly the same thing,
and that digital media (originally described as ``the internet'')
originated not only as a US project but as a US military project. The
entire infrastructure of global digital media is centered on the United
States. This strategic advantage allowed US platforms to
quasi-monopolize the exchange of messages online.

\vspace{2em}

\hypertarget{intellectual-imperialism-meets-media-imperialism}{%
\section{Intellectual Imperialism Meets Media
Imperialism}\label{intellectual-imperialism-meets-media-imperialism}}

As we have seen, debates on intellectual imperialism and media
imperialism have mostly run in parallel. Each one explores the problem
of cultural imperialism from a different angle. I argue that, to
understand the current dynamics of cultural imperialism, we must
consider both at the same time. There are many examples of the interplay
between intellectual imperialism and media imperialism. Here, I intend
to explore a specific case: the use of the academic apparatus as a
resource for socializing media personnel of other countries into the
values and professional culture of the imperialist culture.

Recently, a professor working at the Center for Sustainable Democracy at
the University of South Florida published a post on LinkedIn
commemorating that his center received a \$150,000 grant for training
Brazilian journalists.\footnote{\href{https://www.linkedin.com/posts/jscacco\_democracy-technology-futureofnews-activity-7122591167690457088-RRfZ}{https://www.linkedin.com/posts/jscacco\\\noindent\_democracy-technology-futureofnews-\\\noindent activity-7122591167690457088-RRfZ}.}
The sponsor of that grant was the US Department of State, through the US
Embassy and Consulates in Brazil. The obvious question is: why should a
country provide professional training for native journalists from other
countries? The most remarkable aspect of this initiative is that it is
not exceptional. Indeed, several similar initiatives have occurred in
Brazil and many other countries.

In the last four decades, exporting democracy and offering media
assistance have been common motifs in the US's strategy to exert
influence on other countries. The landmark of this initiative was the
launch of the National Endowment for Democracy (NED) in 1983. NED has
founded ``democracy promotion'' initiatives around the world. It must be
noted that NED defines ``democracy promotion'' in a very broad
manner.\footnote{Christensen, ``Interpreting the Organizational
  Practices.''} In Brazil, NED has provided resources for groups
championing progressive political values but also for ultraconservative
groups that supported the former far-right president Jair
Bolsonaro.\footnote{Camila Felix Vidal and Jahde Lopez,
  ``(Re)pensando a dependência latino-americana: Atlas Network e
  institutos parceiros no governo Bolsonaro,'' \emph{Revista Brasileira
  de Ciência Política} 38
  (2022).}
This indicates that, more than supporting specific political causes, NED
attempts to control the entire scope of Brazilian politics.

Recently, some initiatives have blended elements of intellectual
imperialism and media imperialism. They profit from the structural
advantages that the global scholarship schema offers to the United
States as an instrument to shape acquiescent media elites in other
countries. The Knight Center for Journalism in the Americas, located at
the University of Texas at Austin, provides an excellent example in this
regard. The University of Texas has a reputation as a center of
excellence in Latin American studies. The reasons behind this reputation
go beyond the quality of the researchers working at the university. The
massive investments that US universities receive make it impossible for
universities located in other countries to compete with them.\footnote{Afonso
  de Albuquerque, ``Towards a Multipolar Communication International
  Scholarship?'' \emph{World of Media: Journal of Russian Media and
  Journalism Studies} 1, no. 2 (2023).} The University of Texas, for
instance, can afford the exorbitant prices of the books published by
commercial publishers (such as Routledge, Taylor and Francis, and
others), while the universities located in the majority world cannot. A
recent study found that the University of Texas counted ninety-two
editorial board members in communication journals included on the
Journal of Citation Reports sample. This is more than double the number
of editorial board members from all Latin American and Caribbean
countries together. Concretely, this gives the University of Texas the
power to speak about Latin America with more authority than Latin
American scholars themselves.\footnote{Afonso de Albuquerque et al.,
  ``Structural Limits to the De-Westernization of the Communication
  Field: The Editorial Board in Clarivate's JCR System,''
  \emph{Communication, Culture \& Critique} 13, no. 2 (2020).}

In particular, the scholars affiliated with the Knight Center for
Journalism in the Americas write extensively about Latin American
journalism, but they rarely cite authors working in Latin America. This
geographic exclusion persists even though most of these scholars are
proficient in Spanish and Portuguese and therefore have access to the
scholarly literature published by scholars working in Latin America.
Still more important, a growing number of scholars working in Latin
America have published in ``prestige'' international journals.
Nonetheless, the practice of disregarding items published by authors
working outside Anglophone (and other Western) universities is very
common across communication studies and in other fields as
well.\footnote{See, for instance, Ana C. Suzina, ``English as Lingua
  Franca: On the Sterilization of Scientific Work,'' \emph{Media,
  Culture \& Society} 43, no. 1 (2021); and Mohan Dutta et al.,
  ``Decolonizing Open Science: Southern Interventions,'' \emph{Journal
  of Communication} 71, no. 5 (2021).}

What makes the Knight Center's case particularly interesting is that the
tactic of silencing local scholars' voices is coupled with the intention
of exerting technical (and political) influence over the countries where
these scholars live. After all, the core purpose of the Knight Center is
to provide journalism education for societies that already have their
own journalism courses. The first logical step to justify this intention
is to deny importance to the education and research institutions
existing in those societies. Offering courses for journalists and
students living in Latin America is just one of the means that the
Knight Center employs to achieve its ends. The play of imperial
influence is furthered by strategic networking, which refers to
systematic initiatives aiming to prepare local media elites to be
attuned to the values and viewpoints of the United States. One way it
has done this is by sponsoring professional associations designed to
operate in other countries.

The Brazilian investigative journalism association Associação Brasileira
de Jornalismo Investigativo (ABRAJI) provides a good illustration of
this principle. It was created in 2002 under the auspices of the Knight
Center. Since then, it has acquired growing influence as a journalistic
(and political) agent in Brazil. For instance, in the mid-2010s, ABRAJI
actively promoted the idea that journalism was a key factor in fighting
corruption. At that time, the Brazilian media was engaged in a political
campaign against the Workers' Party (Partido dos Trabalhadores,
hereafter PT). Politically biased accusations of corruption were a
central part of this campaign. Indirectly, at least, ABRAJI provided
support for this strategy. ABRAJI has also worked as a representative
association for Brazilian journalists, in addition to FENAJ (Federação
Nacional dos Jornalistas), the Brazilian journalists' union.

Since 2018, ABRAJI has played a pivotal role as an organizer of
multi-stakeholder initiatives aiming to fight disinformation. The most
known example in this respect is the Comprova Project. The model for
Comprova was the US First Draft project, which integrated universities,
news media, civil society organizations, and platforms in a media
coalition. ABRAJI was one of the founders of Comprova, with the
financial support of the Facebook Journalistic Project (now Meta) and
Google News Initiative, several news media outlets (both legacy and
native digital vehicles), and fact-checking agencies.\footnote{Salvador
  Strano, ``Projeto Comprova reúne 24 veículos contra a fake news,''
  \emph{Meio \& Mensagem}, June 19, 2018.} Comprova has served as a
truth-certifying system, which legitimizes some actors to the detriment
of others. Essentially, ABRAJI works as an intermediary between US
interests and the Brazilian journalistic community. There are numerous
examples in this respect. For instance, ABRAJI sponsored a course on
investigative journalism to be offered to Brazilian journalists in
association with the US Embassy.\footnote{US Mission Brazil, ``U.S.
  Embassy and Abraji Open Enrollment for Investigative Journalism
  Course,'' US Embassy \& Consulates in Brazil, July 20, 2023,
  https://br.usembassy.gov/pt/embaixada-dos-eua-e-abraji-abrem-inscricoes-para-curso-de-jornalismo-investigativo/.}

\hypertarget{cultural-imperialism-in-practice-the-lava-jato-operation}{%
\section{Cultural Imperialism in Practice: The Lava Jato
Operation}\label{cultural-imperialism-in-practice-the-lava-jato-operation}}

Started in 2014, the Lava Jato Judicial Operation provides a dramatic
example of US interference in Brazilian politics,\footnote{Natália
  Viana, Andrew Fishman, and Maryam Saleh, ``Como a Lava Jato escondeu
  do governo federal visita do FBI e procuradores americanos,''
  \emph{Agência Pública/Intercept Brasil}, March 12, 2020.} with
disastrous consequences for Brazil. The PT and its political allies in
general, and former president Luiz Inácio Lula da Silva in particular,
were the main targets of Lava Jato investigations. Judge Sergio Moro and
Prosecutor Deltan Dallagnol led the Lava Jato investigations. Both had
solid ties with US agencies such as the Department of State and the
FBI.\footnote{Viana, Fishman, and Saleh, ``Como a Lava Jato.''}

The legacy media fully supported Lava Jato.\footnote{Mads Damgaard,
  ``Cascading Corruption News: Explaining the Bias of Media Attention to
  Brazil's Political Scandals,'' \emph{Opinião Pública} 24, no. 1
  (2018); Liziane Guazina, Helder Prior, and Bruno Araújo, ``Framing of
  a Brazilian Crisis: Dilma Rousseff's Impeachment in National and
  International Editorials,'' \emph{Journalism Practice} 13, no. 5
  (2019); Francisco P. J. Marques, Camila Mont'Alverne, and Isabele
  Mitozo, ``Editorial Journalism and Political Interests: Comparing the
  Coverage of Dilma Rousseff's Impeachment in Brazilian Newspa­pers,''
  \emph{Journalism} 22, no. 11 (2021); and Teun A. van Dijk, ``How Globo
  Media Manipulated the Impeachment of Brazilian President Dilma
  Rousseff,'' \emph{Discourse \& Communication} 11, no. 2 (2017).} They
provided massive coverage for Lava Jato and teamed with Judge Moro and
the prosecutors to leak information provided by them with the purpose of
damaging PT's and Lula's public image, making it easier for the
prosecutors to convict him. Lava Jato succeeded in destabilizing the
PT-led government. In 2016, President Dilma Rousseff left the presidency
after an impeachment process, and in 2017, Judge Moro convicted Lula,
sentencing him to a twelve-year term in prison under corruption
allegations. A news piece published by the newspaper \emph{O Globo} was
the main evidence for convicting Lula. This news item prominently
featured the allegation that Lula privileged the Odebrecht construction
company in exchange for renovations to a three-story apartment he had
bought. Later, it was proved that Lula never acquired such an
apartment.\footnote{Emilio P. N. Meyer, ``Judges and Courts
  Destabilizing Constitutionalism: The Brazil­ian Judiciary Branch's
  Political and Authoritarian Character,'' \emph{German Law Journal} 19,
  no. 4 (2018).}

Despite the fragility of the evidence against him, Lula was sent to jail
in 2018 and prevented from running in the presidential election that
year. ABRAJI, undeterred by this faulty judicial process, provided full
support to the legacy media coverage of Lava Jato, presenting it as
quality investigative journalism. In fact, the association's vice
president, Vladimir Netto, published a book praising Judge Moro as a
national hero. His book served as the basis for a Netflix
series---\emph{The Mechanism}---released in 2018, the year of the
Brazilian presidential election. Lava Jato and the legacy media coverage
of it fostered a climate of generalized suspicion regarding
institutional politics.\footnote{Nahuel Ribke, ``Netflix and Over the
  Top Politics? The \emph{Mechanism} TV Series and the Dynamics of
  Entertainment Intervention,'' \emph{Critical Studies on Television}
  16, no. 1 (2021).} Ultimately, this created the conditions that
allowed the far-right candidate Jair Bolsonaro to become the Brazilian
president after a surprising victory in 2018.

But how does the cultural imperialism framework proposed in this article
help us understand how foreign powers (the United States, especially)
meddled in Brazilian judicial and political affairs? Intellectual
imperialism was a key element of Lava Jato, as it ultimately provided
the rationale for it. The intellectual roots of Lava Jato lie in the
literature that, from a neoliberal perspective, describes corruption as
a major threat to democracy. The World Bank and Transparency
International have been the main intellectual sources of scholarly
discourse on corruption, understood as ``the private abuse of a public
office.''\footnote{Williamson, ``A Short
  History.''} This agenda came to
exert tremendous influence in political science and law studies. In
Brazil, this literature helped to consolidate a view that emphasized the
so-called accountability institutions as being central to a ``good
democracy'' at the expense of representative politics.\footnote{See, for
  instance, Timothy J. Power and Matthew M. Taylor, eds.,
  \emph{Corruption and Democracy in Brazil} (Notre Dame, IN: University
  of Notre Dame Press, 2011).} This view was significant in that it
provided legitimacy to the idea that Lava Jato would purge the sins of
representative politics---especially those associated with the PT
government.

Still more important, US institutions had played a central role in
training Brazilian law officials on the anticorruption
agenda.\footnote{See Eduardo M. Menuzzi and Fabiano Engelmann, ``Elites
  jurídicas e relações internacionais: Wilson Center e agenda
  anticorrupção no judiciário brasileiro,'' \emph{Conjuntura Astral} 11,
  no. 54 (2020).} In fact, their interference in Brazilian politics went
far beyond that. US universities and scholars actively promoted Lava
Jato as a revolutionary event in the history of Latin American justice:
the most significant anticorruption operation in Latin American
history.\footnote{See, for instance, Luciano Da Ros and Matthew M.
  Taylor, \emph{Brazilian Politics on Trial: Corruption and Reform under
  Democracy} (Boulder, CO: Lynne Rienner Publishers, 2022);

  Paul Lagunes and Jan Svejnar, eds., \emph{Corruption and the Lava Jato
  Scandal in Latin America} (New York: Routledge, 2021).} Numerous
academic works promoted this idea, and US scholars guided Moro on visits
to US universities.\footnote{Fausto Macedo, ``Moro e Carmen falam de
  Lava Jato e Corrupção na Universidade Colúmbia,'' \emph{Blog do Fausto
  Macedo}, \emph{Estado de São Paulo}, February 4, 2017.} Notre Dame
University provided Moro with the Notre Dame Award in 2017 and an
honorary PhD title in the following year.\footnote{JusBrasil (website),
  ``Juiz Federal Sérgio Moro recebe Título de Doutor Honoris Causa da
  University of Notre Dame,'' \emph{JusBrasil}, May 21, 2018.}

Intellectual imperialism was not the only dimension of cultural
imperialism influencing Lava Jato. In fact, it merged with media
imperialism in the promotion of the anticorruption agenda. ABRAJI was
instrumental in this regard, as it helped to legitimize a political
campaign against Lula and PT as quality journalism. This is recognized
by association itself. In fact, they contend that ABRAJI was part of a
Latin American network organized around Lava Jato.\footnote{Catalina
  Lobo-Guerrero, ``How Lava Jato Brought Together Latin America's
  Investigative Journalists,'' \emph{Global Investigative Journalism
  Network}, August 14, 2019.} Vladimir Netto, who served as vice
president of ABRAJI from 2016 to 2017 was the author of a book that
presented Judge Sergio Moro in very favorable terms.\footnote{Vladimi
  Netto, \emph{Lava Jato: o juiz Sergio Moro e os bastidores da operação
  que abalou o Brasil} (Rio de Janeiro: Sextante, 2016).} In 2020, a
news series published by \emph{Intercept Brasil} showed evidence that
Netto and other journalists colluded with Lava Jato prosecutors to
release public information favorable to Lava Jato.

\hypertarget{conclusion}{%
\section{Conclusion}\label{conclusion}}

This article sheds new light on the debate about cultural imperialism.
Historically, scholars have employed this term to describe two different
phenomena: intellectual imperialism and media imperialism. They have
been explored by two academic traditions, which rarely find dialogue
with each other. Taking a different approach, this article suggests that
intellectual and media imperialism are complementary aspects of a more
general phenomenon. By taking these two aspects together, it identifies
the original dynamics of media imperialism, which differ significantly
from those identified by the classical approach. At present, scholars
mostly associate media imperialism with the mechanisms allowing
imperialistic countries to inundate other countries with their media
content. In this article, we explore a different strategy, which is one
that takes advantage of intellectual imperialism. It works by educating
the local media elite and journalists in accordance with the worldview
and interests of the imperialistic countries. Added to this, it employs
networks that consolidate and diffuse foreign practices and models among
local journalists.

To illustrate how this logic works, this article analyzes the interplay
between intellectual imperialism and media imperialism. In particular,
it discusses the role that the Knight Center for Journalism in the
Americas, an institution located at the University of Texas, has played
as an organizer of Brazilian journalists. For instance, the Knight
Center was one of the sponsors of the creation of ABRAJI, the Brazilian
investigative journalism association. The article takes the Lava Jato
episode as a concrete example of how intellectual imperialism and media
imperialism impacted Brazilian politics. As the Brazilian case
demonstrates, scholarly research and education frequently operate as
tools at the service of imperialistic interests.




\section{Bibliography}\label{bibliography}

\begin{hangparas}{.25in}{1} 



Aalbers, Manuel B. ``Creative Destruction through the Anglo-American
Hegemony: A Non-Anglo-American View on Publications, Referees, and
Language.'' \emph{Area} 36, no. 3 (2004): 319--22.

Alam, Shahidul. ``Majority World: Challenging the West\textquotesingle s
Rhetoric of Democracy.'' \emph{Amerasia Journal} 34, no. 1: 87--98.

Alatas, Syed Farid. ``Intellectual Imperialism: Definition, Traits, and
Problems.'' \emph{Southeast Asian Journal of Social Science} 28, no. 1
(2000): 23--45.

Albuquerque, Afonso de. ``The Institutional Basis of Anglophone Western
Centrality.'' \emph{Media, Culture \& Society} 43, no. 1 (2021):
180--88.

Albuquerque, Afonso de. ``Transitions to Nowhere: Western Teleology and
Regime-Type Classification.'' \emph{International Communication Gazette}
85, no. 6 (2023): 479--97.

Albuquerque, Afonso de. ``Towards a Multipolar Communication
International Scholarship?'' \emph{World of Media: Journal of Russian
Media and Journalism Studies} 1, no. 2 (2023): 5--18.

Albuquerque, Afonso de, Thaiane M. Oliveira, Marcelo A. Santos Junior,
and Sofia O. F. Albuquerque. ``Structural Limits to the
De-Westernization of the Communication Field: The Editorial Board in
Clarivate's JCR System.'' \emph{Communication, Culture \& Critique} 13,
no. 2 (2020): 185--203.

Arnove, Robert F., ed. \emph{Philanthropy and Cultural Imperialism: The
Foundations at Home and Abroad}. Boston: G.K. Hall, 1980.

Baber, Zaheer. ``Modernization Theory and the Cold War.'' \emph{Journal
of Contemporary Asia} 31, no. 1 (2001): 71--85.

Bhuyan, Abu. \emph{Internet Governance and the Global South: A Demand
for a New Framework}. London: Palgrave MacMillan, 2014.

Bourdieu, Pierre, and Loic Wacquant. ``On the Cunning of the Imperialist
Reason.'' \emph{Theory, Culture \& Society} 16, no. 1 (1999): 41--58.

Boyd-Barrett, Oliver. ``Cultural Imperialism and Communication.''
\emph{Oxford Research Encyclopedia of Communication}, June 2018.
\url{https://doi.org/10.1093/acrefore/9780190228613.013.678}.

Boyd-Barrett, Oliver. \emph{Media Imperialism}. Los Angeles: SAGE, 2015.

Boyd-Barrett, Oliver, and Tanner Mirrlees, eds. \emph{Media Imperialism:
Continuity and Change}. Lanham, MA: Rowman \& Littlefield, 2019.

Chakravartty, Paula, Rachel Kuo, Victoria Grubbs, and Charlton Mcllwain.
``\#CommunicationSoWhite.'' \emph{Journal of Communication} 68, no. 2
(2016): 254--66.

Christensen, Michael. ``Interpreting the Organizational Practices of
North American Democracy Assistance.'' \emph{International Political
Sociology} 11, no. 2 (2017): 148--65.

Cumings, Bruce. ``Boundary Displacement: Area Studies and International
Studies during and after the Cold War.'' \emph{Bulletin of Concerned
Asian Scholars} 29, no. 1 (1997): 6--26.

Damgaard, Mads. ``Cascading Corruption News: Explaining the Bias of
Media Attention to Brazil's Political Scandals.'' \emph{Opinião Pública}
24, no. 1 (2018): 114--43.

Da Ros, Luciano, and Matthew M. Taylor. \emph{Brazilian Politics on
Trial: Corruption and Reform under Democracy}. Boulder, CO: Lynne
Rienner Publishers, 2022.

Davis, Stuart. ``What is Netflix Imperialism? Interrogating the Monopoly
Aspirations of the `World's Largest Television Network.'\,''
\emph{Information, Communication, and Society} 26, no. 6 (2023):
1143--58. \url{https://doi.org/10.1080/1369118X.2021.1993955.}

Demeter, Marton. \emph{Academic Knowledge Production and the Global
South}. London: Palgrave-Macmillan, 2020.

Dezalay, Yves, and Bryant G. Garth. \emph{The Internationalization of
Palace Wars: Lawyers, Economists and the Contest to Transform Latin
American States}. Chicago: University of Chicago Press, 2002.

Dorfman, Ariel, and Armand Mattelart. \emph{Para leer al Pato Donald:
Comunicación de Masa y Colonialismo}. Buenos Aires: Siglo Veinteuno,
1971.

Dutta, Mohan, Srividya Ramasubramanian, Mereana Barrett, Christine
Elers, Devina Sarwatay, Preeti Raghunath, Satveer Kaur, et al.
``Decolonizing Open Science: Southern Interventions.'' \emph{Journal of
Communication} 71, no. 5 (2021): 803--26.

Enghel, Florencia, and Martin Becerra. ``Here and There: (Re)Situating
Latin America in International Communication Theory.''
\emph{Communication Theory} 28, no. 2 (2018): 111--30.
\url{https://doi.org/10.1093/ct/qty005}.

Fourcade, Marion. ``The Construction of a Global Profession: The
Transnationalization of Economics.'' \emph{American Journal of
Sociology} 112, no. 1 (2006): 145--94.

Fuentes Navarro, Raúl. ``Institutionalization and Internationalization
of the Field of Communication Studies in Mexico and Latin America.'' In
\emph{The International History of Communication Study}, edited by Peter
Simonson and David W. Park, 325--45. New York: Routledge, 2016.

Gluck, Antje. ``De-Westernization and Decolonization in Media Studies.''
\emph{Oxford Research Encyclopedia of Communication}, December 2018.
\url{https://doi.org/10.1093/acrefore/9780190228613.013.898}.

Guazina, Liziane, Helder Prior, and Bruno Araújo. ``Framing of a
Brazilian Crisis: Dilma Rousseff's Impeachment in National and
International Editorials.'' \emph{Journalism Practice} 13, no. 5 (2019):
620--37. \url{https://doi.org/10.1080/17512786.2018.1541422}.

Gürkan, Efe C. \emph{Imperialism after the Neoliberal Turn}. London:
Routledge, 2022.

Harvey, David. \emph{The New Imperialism}. Oxford: Oxford University
Press, 2003.

Hayler, Teresa. \emph{Aid as Imperialism}. Middlesex, UK: Penguin Books,
1971.

Jin, Dal Y. \emph{Digital Platforms, Imperialism, and Political
Culture}. London: Routledge, 2015.

JusBrasil (website). ``Juiz Federal Sérgio Moro recebe Título de Doutor
Honoris Causa da University of Notre Dame.'' \emph{JusBrasil}, May 21,
2018.
\href{https://www.jusbrasil.com.br/noticias/juiz-federal-sergio-moro-recebe-titulo-de-doutor-honoris-causa-da-university-of-notre-dame/580906390}{https://www.jusbrasil.com.br/noticias/juiz-federal-sergio-moro-recebe-titulo-de-doutor-honoris-causa-da-university-of-notre-dame/580906390}.

Kumar, Krishan. ``Colony and Empire, Colonialism and Imperialism: A
Meaningful Distinction?'' \emph{Comparative Studies in History and
Society} 63, no. 2 (2021): 280--309.

Lagunes, Paul, and Jan Svejnar, eds. \emph{Corruption and the Lava Jato
Scandal in Latin America}. New York: Routledge, 2021.

Lobo-Guerrero, Catalina. ``How Lava Jato Brought Together Latin
America's Investigative Journalists.'' \emph{Global Investigative
Journalism Network}, August 14, 2019.
\href{https://gijn.org/stories/how-lava-jato-brought-together-latin-americas-investigative-journalists/}{https://gijn.org/stories/how-lava-jato-brought-together-latin-americas-investigative-journalists/}.

Macedo, Fausto. ``Moro e Carmen falam de Lava Jato e Corrupção na
Universidade Colúmbia.'' \emph{Blog do Fausto Macedo. Estadão de São
Paulo}, February 4, 2017. \href{https://politica.estadao.com.br/blogs/fausto-macedo/moro-e-carmen-falam-de-lava-jato-e-corrupcao-na-universidade-columbia/}{https://politica.estadao.com.br/blogs/fausto-macedo/moro-e-carmen-falam-de-lava-jato-e-corrupcao-na-universidade-columbia/}.

Manning, Jennifer. ``Decolonial Feminist Theory: Embracing the Gendered
Colonial Difference in Management and Organizational Studies.''
\emph{Gender, Work \& Organization} 28, no. 4 (2021): 1203--19.

Menuzzi, Eduardo M., and Fabiano Engelmann. ``Elites jurídicas e
relações internacionais: Wilson Center e agenda anticorrupção no
judiciário brasileiro.'' \emph{Conjuntura Astral} 11, no. 54 (2020):
105--22.

Marginson, Simon, and Marjik van der Wende. ``To Rank or to Be Ranked:
The Impact of Global Rankings in Higher Education.'' \emph{Journal of
Studies in International Education} 11, no. 3/4 (2007): 306--29.

Marques, Francisco P. J., Camila Mont'Alverne, and Isabele Mitozo.
``Editorial Journalism and Political Interests: Comparing the Coverage
of Dilma Rousseff's Impeachment in Brazilian Newspa­pers.''
\emph{Journalism} 22, no. 11 (2021): 2816--35.
\url{https://doi.org/10.1177/1464884919894126}.

Meyer, Emilio P. N. ``Judges and Courts Destabilizing Constitutionalism:
The Brazil­ian Judiciary Branch's Political and Authoritarian
Character.'' \emph{German Law Journal} 19, no. 4 (2018): 727--68.

Mignolo, W. D. \emph{The Darker Side of Western Modernity: Global
Futures, Decolonial Options}. Durham, NC: Duke University Press, 2011.

Moody, Paul. ``Embassy Cinema: What WikiLeaks Reveals about US State
Support for Hollywood.'' \emph{Media, Culture \& Society} 39, no. 7
(2017): 1063--77.

Moyo, Last. \emph{The Decolonial Turn in Media Studies in Africa and the
Global South}. New York: Springer, 2020.

Netto, Vladimir. \emph{Lava Jato: o juiz Sergio Moro e os bastidores da
operação que abalou o Brasil}. Rio de Janeiro: Sextante, 2016.

Nordenstreng, Kaarle. ``How the New World Order and Imperialism
Challenge Media Studies.'' \emph{tripleC: Communication, Capitalism \&
Critique} 11, no. 2 (2013): 348--58.

Nye, Joseph S., Jr. \emph{Bound to Lead: The Changing Nature of American
Power}. New York: Basic Books, 1990.

Oliveira, Thaiane, Alves Marcelo Evangelista Simone, and Rodrigo Quinan.
``\,`Those on the Right Take Chloroquine': The Illiberal
Instrumentalization of Scientific Debates during the Covid‐19 Pandemic
in Brasil.''~\emph{Javnost---The Public}~28, no. 2 (2021): 165--84.

Parmar, Inderjeet. ``The Big 3 Foundations and American Global Power.''
\emph{American Journal of Economics and Sociology} 74, no. 4 (2015):
676--703.

Ramiro Beltrán, Luis. ``TV Etchings in the Minds of Latin Americans:
Conservative, Materialism, and Conformism.'' \emph{International
Communication Gazette} 24, no. 1 (1976): 61--85.

Recuero, Raquel, Felipe B. Soares, Otavio I. Vinhas, Taiane Volcane,
Luis Ricardo Huttner, and Victoria Silva. ``Bolsonaro and the Far Right:
How Disinformation about COVID-19 Circulates on Facebook in Brazil.''
\emph{International Journal of Communication} 18 (2022): 148--71.

Revista Fórum. ``Nova Vaza Jato: Jornalistas combinaram matérias e
submeteram textos a procuradores da Lava Jato.'' \emph{Revista Fórum,}
December 12, 2019.
\href{https://revistaforum.com.br/politica/2019/12/20/nova-vaza-jato-jornalistas-combinaram-materias-submeteram-textos-procuradores-da-lava-jato-66235.html}{https://revistaforum.com.br/politica/2019/ 12/20/nova-vaza-jato-jornalistas-combinaram-materias-submet eram-textos-procuradores-da-lava-jato-66235.html}.

Ribke, Nahuel. ``Netflix and Over the Top Politics? The \emph{Mechanism}
TV Series and the Dynamics of Entertainment Intervention.''
\emph{Critical Studies on Television} 16, no. 1 (2021): 47--61.
\url{https://doi.org/10.1177/1749602020980139}.

Said, Edward. \emph{Orientalism}. London: Penguin Books, 1978.

Santos, Alvaro. ``The World Bank's Uses of the `Rule of Law' Promise in
Economic Development.'' In \emph{The New Law and Economic Development: A
Critical Appraisal}, edited by David M. Trubek and Alvaro Santos,
253--300. Cambridge: Cambridge University Press, 2006.

Schiller, Herbert I. \emph{Mass Communications and the American Empire}.
New York: A. M. Kelley, 1969.

Shome, Raka. ``When Postcolonial Studies Meets Media Studies.''
\emph{Critical Studies in Media Communication} 33, no. 3 (2016):
245--63. \url{https://doi.org/10.1080/15295036.2016.1183801}.

Simonson, Peter, David W. Park, and Jefferson Pooley.
``Exclusions/ Exclusiones: The Role for History in the Field's
Reckoning.'' \emph{History of Media Studies} 2 (2022).
\url{https://doi.org/10.32376/d895a0ea.ed348e03}.

Simonson, Peter, Jefferson Pooley, and David W. Park. ``The History of
Communication Studies Across the Americas: A View from the United
States.'' \emph{MATRIZes} 17, no. 3 (2023): 189--216.

Slaughter, Sheila, and Gary Rhoades. \emph{Academic Capitalism and the
New Economy: Markets, State and Higher Education}. Baltimore: Johns
Hopkins University Press, 2004.

Strano, Salvador. ``Projeto Comprova reúne 24 veículos contra a fake
news.'' \emph{Meio \& Mensagem}, June 19, 2018.

Straubhaar, Joseph D. ``Beyond Media Imperialism: Asymmetrical
Interdependence and Cultural Proximity.'' \emph{Critical Studies in Mass
Communication} 8, no. 1 (1991): 39--59.

Suzina, Ana C. ``English as Lingua Franca: On the Sterilization of
Scientific Work.'' \emph{Media, Culture \& Society} 43, no. 1 (2021):
171--79.

Takacs, Stacy. ``The US Military as Cold War Programmer.'' \emph{Journal
of Popular Culture} 50, no. 3 (2017): 540--60.

Tomlinson, John. \emph{Cultural Imperialism}. London: Continuum, 1991.

Tunstall, Jeremy. \emph{The Media are American: Anglo-American Media in
the World}. London: Constable, 1977.

Williamson, John. ``A Short History of the Washington Consensus.'' In
\emph{The Washington Consensus Reconsidered: Towards a New Global
Governance}, edited by Nancy Serra and Joseph E. Stiglitz, 14--30.
Oxford: Oxford University Press, 2008.

van Dijk, Teun A. ``How Globo Media Manipulated the Impeachment of
Brazilian President Dilma Rousseff.'' \emph{Discourse \& Communication}
11, no. 2 (2017): 199--229.

Viana, Natália, Andrew Fishman, and Maryam Saleh. ``Como a Lava Jato
escondeu do governo federal visita do FBI e procuradores americanos.''
\emph{Agência Pública/Intercept Brasil}, March 12, 2020.
\href{https://apublica.org/2020/03/como-a-lava-jato-escondeu-do-governo-federal-visita-do-fbi-e-procuradores-americanos/}{https://apublica.org/2020/03/como-a-lava-jato-escondeu-do-governo-federal-visita-do-fbi-e-procuradores-americanos/}.

Vidal, Camila F., and Jahde Lopez. ``(Re) pensando a dependência
latino-americana: Atlas Network e institutos parceiros no governo
Bolsonaro.'' \emph{Revista Brasileira de Ciência Política} 38 (2022):
1--40.



\end{hangparas}


\end{document}