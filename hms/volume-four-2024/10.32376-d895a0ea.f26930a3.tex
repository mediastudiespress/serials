% see the original template for more detail about bibliography, tables, etc: https://www.overleaf.com/latex/templates/handout-design-inspired-by-edward-tufte/dtsbhhkvghzz

\documentclass{tufte-handout}

%\geometry{showframe}% for debugging purposes -- displays the margins

\usepackage{amsmath}

\usepackage{hyperref}

\usepackage{fancyhdr}

\usepackage{hanging}

\hypersetup{colorlinks=true,allcolors=[RGB]{97,15,11}}

\fancyfoot[L]{\emph{History of Media Studies}, vol. 4, 2024}


% Set up the images/graphics package
\usepackage{graphicx}
\setkeys{Gin}{width=\linewidth,totalheight=\textheight,keepaspectratio}
\graphicspath{{graphics/}}

\title[When Communication Became a Discipline]{When Communication Became a Discipline} % longtitle shouldn't be necessary

% The following package makes prettier tables.  We're all about the bling!
\usepackage{booktabs}

% The units package provides nice, non-stacked fractions and better spacing
% for units.
\usepackage{units}

% The fancyvrb package lets us customize the formatting of verbatim
% environments.  We use a slightly smaller font.
\usepackage{fancyvrb}
\fvset{fontsize=\normalsize}

% Small sections of multiple columns
\usepackage{multicol}

% Provides paragraphs of dummy text
\usepackage{lipsum}

% These commands are used to pretty-print LaTeX commands
\newcommand{\doccmd}[1]{\texttt{\textbackslash#1}}% command name -- adds backslash automatically
\newcommand{\docopt}[1]{\ensuremath{\langle}\textrm{\textit{#1}}\ensuremath{\rangle}}% optional command argument
\newcommand{\docarg}[1]{\textrm{\textit{#1}}}% (required) command argument
\newenvironment{docspec}{\begin{quote}\noindent}{\end{quote}}% command specification environment
\newcommand{\docenv}[1]{\textsf{#1}}% environment name
\newcommand{\docpkg}[1]{\texttt{#1}}% package name
\newcommand{\doccls}[1]{\texttt{#1}}% document class name
\newcommand{\docclsopt}[1]{\texttt{#1}}% document class option name


\begin{document}

\begin{titlepage}

\begin{fullwidth}
\noindent\LARGE\emph{Book review
} \hspace{88mm}\includegraphics[height=1cm]{logo3.png}\\
\noindent\hrulefill\\
\vspace*{1em}
\noindent{\Huge{\emph{When Communication Became a Discipline}\par}}

\vspace*{1.5em}

\noindent\LARGE{Joshua Gunn}\par\marginnote{\emph{When Communication Became a Discipline}, reviewed by Joshua Gunn \emph{History of Media Studies} 4 (2024), \href{https://doi.org/10.32376/d895a0ea.f26930a3}{https://doi.org/10.32376/d895a0ea.f26930a3}.} \vspace*{0.75em}
\vspace*{0.5em}
\noindent{{\large\emph{University of Texas at Austin}, \href{mailto:josh_gunn@austin.utexas.edu}{josh_gunn@austin.utexas.edu}\par}} 

% \vspace*{0.75em} % second author

% \noindent{\LARGE{<<author 2 name>>}\par}
% \vspace*{0.5em}
% \noindent{{\large\emph{<<author 2 affiliation>>}, \href{mailto:<<author 2 email>>}{<<author 2 email>>}\par}}

% \vspace*{0.75em} % third author

% \noindent{\LARGE{<<author 3 name>>}\par}
% \vspace*{0.5em}
% \noindent{{\large\emph{<<author 3 affiliation>>}, \href{mailto:<<author 3 email>>}{<<author 3 email>>}\par}}

\end{fullwidth}

\vspace*{1em}


\noindent William\marginnote{\href{https://creativecommons.org/licenses/by-nc/4.0/}{\includegraphics[height=0.5cm]{by-nc.png}}} F. Eadie. \emph{When
Communication Became a Discipline}. 184 pp., 1 fig.,
index. Lanham, MD: Lexington Books, 2022. \$100 (cloth).

\vspace{0.2in}

\newthought{Last January a PhD student} provoked heated discussion in a post they
made to the ``Communication Scholars for Transformation'' Facebook
interest group.\footnote{The discussion began with a query by Bethany
  Keely-Jonker about what to retitle an ``Oral Interpretation'' course
  in their department. Bethany Keely-Jonker, ``Colleagues, my department
  has an old course called `Oral Interpretation' we\textquotesingle d
  like to bring back,'' Facebook, January 23, 2024,
  \url{https://www.facebook.com/share/p/nGsLyvHUwqhn9otc/}.} Insisting that
the true understanding of the discipline of performance studies
originated only at their MA alma mater, New York University, the junior
scholar asserted that the tradition of performance studies that arose
from the oral interpretation tradition in Speech (Communication) was not
performance studies: properly understood, performance concerns the
``analysis of culture''---that is, analysis from the point of
reception---rather than the actual production of performances (e.g.,
folks that self-describe as ``performance practitioners'').\footnote{Joan
  Joda, ``Speech and Oration. It's not performance studies,'' Facebook,
  January 23, 2024, comment on Keely-Jonker, ``Colleagues,''
  \url{https://www.facebook.com/share/p/nGsLyvHUwqhn9otc/}.} Numerous
luminaries of performance studies housed in what were formerly Speech
departments---Tracy Stephenson Shaffer, Jonny Gray, and Trish Suchy, to
name a few---challenged the trolling gatekeeper, but to no avail: the
student deleted their offending posts but still stood their ground. NYU
is the origin of performance studies, and this is a hill that they are
willing to die on.

Because arguments about what constitutes a field or discipline---its
proper object and method of study---constitute disciplinarity as such,
dying on (as opposed to running up or down) this hill is foolish. Of
course, there are many forms of performance studies with different
origin narratives, and certainly a lot of overlap and awareness between
them: one tradition is anchored in theater departments, which overlaps
with the speech tradition rooted in public speaking and oral
interpretation (e.g., often associated with textbooks like Ronald J.
Pelias and Tracy Stephenson Shaffer's \emph{Performance Studies: The
Interpretation of Texts}), which in turn overlaps with the NYU tradition


\enlargethispage{2\baselineskip}

\vspace*{2em}

\noindent{\emph{History of Media Studies}, vol. 4, 2024}


 \end{titlepage}

% \vspace*{2em} | to use if abstract spills over

\noindent that draws upon Victor Turner's conception of ``liminality'' and related
anthropological perspectives (and now often associated with Richard
Schechner's textbook \emph{Performance Studies: An Introduction}). The
defiant, quasi-public demand for the real performance studies to stand
(and apparently not deliver) is one of the comforts of conceptual
parsimony and linear origin narratives, which retroactively reframe
ambiguity and repackage uncertainty into more tolerable luggage.

We confront a greater disciplinary confusion with the fields of
``communication'' or ``communication studies.'' It's common, for
example, for undergraduate communication studies students at the
University of Texas at Austin to identify themselves as ``communications
majors'' when the ``s'' coda usually denotes communication technologies
of one sort or the other (e.g., mass comm, telecom, and so on). Graduate
students entering into the study of communication(s) seem even more
bewildered about the horizon of the field(s), insofar as there are
arguably at \emph{least} four North American traditions that claim the
term: (1) the speech tradition that arose in the late nineteenth century
as ``oral English,'' becoming the field of speech, then
speech-communication in the late 1960s, then communication studies in
the 1990s; (2) the tradition often associated with Paul Lazarsfeld's
studies of propaganda and popular opinion, eventually becoming ``mass
communication'' and associated with journalism departments; (3) the
field that originated with Wilbur Schramm's establishment of The
Institute of Communications Research at the University of Illinois,
which was eventually helmed by James W. Carey and later associated with
Larry Grossberg's brand of cultural studies (which in turn was
influenced by the British study of ``communication'' pioneered by
Richard Hoggart and Stuart Hall); and (4) the Canadian communication
tradition, often described in the US as concerning ``media ecology'' and
associated with figures such as Marshall McLuhan (who had a profound
influence on number three). Dispersed under these admittedly
oversimplified descriptions of traditions are studies of film,
television, radio, and contemporary studies of social media, which may
associate institutionally with any of these (for example, radio,
television, and film programs have historically been housed with speech
and journalism as well as theater and dance or became independent
departments). In short, when someone claims to be a scholar, student, or
teacher of communication, this has long indexed little more than a
floating signifier requiring further contextualization in common
conversation (certainly with deans and provosts, but also with folks
whom one meets on an airplane).

As John Durham Peters might suggest, what links all of these
communication traditions together conceptually is an impossible ``dream
of communication'' that bodies---alien, animal, artificial---can be
transcended, which is a harmful fantasy to be sure (e.g., the problems
associated with the afeared ``metaphysics of presence'').\footnote{John
  Durham Peters, \emph{Speaking into the Air: A History of the Idea of
  Communication} (Chicago: Chicago University Press, 1999.)} Because the
objects and methods of any field are constantly debated, understanding
the fields of communication(s) is perhaps better grasped by plotting an
\emph{institutional} history. Of course, chronologies of institutional
formation often play into Hegelian dreams of linear progress, but
without some sort of temporal device, conceptual genealogies are---at
least initially---disorienting, which inspires the self-styled sentries
I bemoan.

I've rehearsed the tricky pickle of disciplinary identification at
length to better nest William F. Eadie's \emph{When Communication Became
a Discipline} (\emph{WCBD}), as readers of this journal unquestionably
hail from different communication homes. \emph{WCBD} joins a number of
earlier histories of communication rooted in the speech tradition that
hew closely to institutional evolution (notably, Eadie worked as an
associate director of the National Communication Association for seven
years). Largely covering the period between the early 1960s to the early
1980s in the United States, Eadie's argument is presented mostly as an
\emph{intellectual} history that picks up where previous histories of
the speech tradition leave off, such as Pat Gehrke's \emph{The Ethics
and Politics of Speech} or William Keith's \emph{Democracy as
Discussion}.\footnote{Pat J. Gehrke, \emph{The Ethics and Politics of
  Speech} (Carbondale: Southern Illinois University Press, 2009);
  William M. Keith, \emph{Democracy as Discussion: Civic Education and
  the American Forum Movement} (New York: Lexington Books, 2007). One of
  the most definitive early histories here is Herman Cohen, \emph{The
  History of Speech Communication: The Emergence of a Discipline,
  1914--1945} (Annandale, VA: Speech Communication Association, 1994).}
As a complement to these earlier studies, Eadie's book traces a
trajectory that is careful to detail the interrelationship between
\emph{journalism} and the speech tradition, largely in respect to the
professional and credentialing organizations of the National
Communication Association (NCA; formerly the Speech Communication
Association), the International Communication Association (ICA), and the
Association of Education in Journalism and Mass Communication
(AEJMC).\footnote{I should also mention that Eadie edited a massive,
  two-volume reference book for communication studies writ large:
  William F. Eadie, \emph{21\textsuperscript{st} Century Communication:
  A Reference Handbook} (Thousand Oaks, CA: Sage, 2009).} Other
communication traditions are mentioned in passing, but it is the speech
circuit that helps to circumscribe that would otherwise be an even more
sprawling story.

In the first three chapters, Eadie moves to describe what he means by
``discipline,'' which is narrower than the conception of ``field.''
Eadie asserts that ``disciplines . . . are communities of inquiry
organized around a particular topic'' and can be somewhat crudely
discerned in terms of department names or the titles of degrees that
``we'' award students (4). (Who ``we'' are is a vexed notion that Eadie
also takes up in the introduction.) Although many of us use the terms
``field'' and ``discipline'' interchangeably, Eadie's understanding of
discipline seems to suggest it refers to academic recognition in local
departments as well as through national professional organizations.
Eadie says that the story he will tell ``takes more the form of argument
than it does history, though I do support that argument with historical
evidence'' (5). His argument is that communication achieved ``respect in
the academy'' in the early 1980s (141), when ``journalism and speech
professors succeeded in becoming communication professors'' (5). Eadie
says that ``by 1982 communication was only poised to become a
discipline---it would take several years more before its disciplinary
status would be recognized {[}by whom is never quite clear{]}'' (5). To
this end, Eadie isolates five ``strands'' of communicative inquiry in
subsequent chapters respectively, each of which contributes toward a
growing coherence in regard to the object of communication and methods
of study in the twentieth century.

Eadie proceeds by recursively collapsing succeeding chapters into
different conceptions of object and method: public opinion and media
studies (chapter four); rhetorical and language studies (chapter five);
``information transmission'' or mass media studies as they intersect
with communication systems (chapter six); interpersonal communication
(largely social science--based, chapter seven); critical/cultural
studies (chapter eight); and a concluding chapter that glosses field
developments since 1982. Like a palimpsest, the different methods and
approaches to communication are written on top of one another, roving
back and forth over the twentieth century. Such an organizational scheme
is smart, for while there is a rough chronology to Eadie's story, the
recursive character of different vantage points helps to diffuse the
assumption that the discipline(s) or field(s) of communication developed
in a straight line. For example, Eadie devotes much attention throughout
the study to a ``sub-group'' of the Speech Association of America (a
precursor to NCA), the National Society for the Study of Communication,
and its journal, \emph{The Journal of Communication}, which were
formally established in 1950 (28). After tracing its split from NCA into
the International Communication Association in 1970, discussion of ICA
will appear periodically throughout the study, such as in the seventh
chapter in which a focus on public speech begins to push toward the
study of ``person-to-person communication as a context for study'' and
the creation of a much respected annual journal, the \emph{Communication
Yearbook} (107). After establishing the ur-plot of the field, each
chapter moves back and forth in time, pulling through themes to the
present, then returning again to the past to pull up again from a
different angle.

\emph{When Communication Became a Discipline} is a valuable historical
argument for how speech communication---and to a lesser extent
journalism---achieved disciplinary ``status'' through various
permutations of study.\footnote{Also see Pat J. Gehrke and William M.
  Keith, eds., \emph{A Century of Communication Studies: The Unfinished
  Conversation} (New York: Routledge, 2015).} Earlier chronicles of
speech departments address a number of struggles and skirmishes in the
1920s among speech instructors about how best to study the object of
speech: social science or humanities? The decision by the 1930s was
``both,'' which is belatedly reflected in the name change to the Speech
Communication Association in the late 1960s (``speech'' referenced the
humanities and ``communication'' the social sciences; notably, the
``behaviorists'' left with ICA despite the change anyway). This tension
over methodology has never left the field, but the focus of disciplinary
history has largely been on the formative years before the postwar
period. Eadie picks up the thread after the Second World War pulling it
into the 1980s, filling in some gaps of understanding (especially in the
1970s) and weaving together very different methodologies and approaches
to the study of ``communication'' into a more or less messy but coherent
narrative. The book is chock-full of nuggets of information and detail,
and numerous luminaries inside and outside ``the field'' appear to be
given due justice. It is no small feat to address developments in social
science and the humanities simultaneously, which is a tribute to Eadie's
pastoral grasp of---and attention to---detail.

At times \emph{WCBD} is a tedious read; this is particularly true when
Eadie outlines the peculiarities of communication theory and
experimental design. In chapter seven, for example, Eadie is at pains to
describe the influence of a book titled \emph{Pragmatics of Human
Communication} in the development of so-called relational communication
(vis-à-vis interpersonal communication), which inspires M.R. Parks to
advance a number of ``axioms'' and ``theorems'' in 1977, which Eadie in
turn lists seriatim over the course of three pages. Such detail in lieu
of paraphrasing will not be of interest to the general reader in the
wider studies of communication, although I recognize the importance of
describing the operationalization of concepts in many influential
studies ``for the record.'' However, one doesn't need to get into the
weeds with Eadie to appreciate the years-long research and citational
depth of his account of communication as a ``discipline.''

One of the most perplexing dimensions of Eadie's sport is the insistence
that communication is a discipline and not merely a field. I will hold
off on detailing the morass of distinguishing field from discipline---a
popular topic of higher education studies, particularly in the wake of
the resource failures of ``interdisciplinarity''---other than to say
that what appears to be at stake here is a longed-for recognition: when
did communication scholars and teachers become \emph{respected} as
genuine scholars and teachers in the wider academy? Without appeals to
his authority as an experienced officer of the NCA---which he could have
easily done---Eadie makes his case based on the historical record and
the citational gravitas of important books, articles, and figures over
time. Few readers of this journal will fail to recognize the legacy of
disrespect toward communication teaching and scholarship that is
routinely rehearsed in popular culture: a degree in communication or
communications is often lampooned as a vacuous specialization. From
\emph{The Simpsons} to late night talk show digs---most recently on
\emph{Real Time with Bill Maher}---communication is derided as an ``easy
major.'' Although we should stop far short of saying that the speech
tradition represents \emph{the} discipline of communication, as an
expression of the whole, Eadie makes a strong case for the rigorous
character of communication research and the insights produced by the
field, if only by example.




\section{Bibliography}\label{bibliography}

\begin{hangparas}{.25in}{1} 



Cohen, Herman. \emph{The History of Speech Communication: The Emergence
of a Discipline, 1914--1945.} Annandale, VA: Speech Communication
Association, 1994.

Eadie, William F. \emph{21}\textsuperscript{\emph{st}}\emph{ Century
Communication: A Reference Handbook}. Thousand Oaks, CA: Sage, 2009.

Gehrke, Pat J. \emph{The Ethics and Politics of Speech}. Carbondale:
Southern Illinois University Press, 2009.

Gehrke, Pat J., and William M. Keith, eds. \emph{A Century of
Communication Studies: The Unfinished Conversation}. New York:
Routledge, 2015.

Keith, William M. \emph{Democracy as Discussion: Civic Education and the
American Forum Movement}. New York: Lexington Books, 2007.

Peters, John Durham. \emph{Speaking into the Air: A History of the Idea
of Communication}. Chicago: Chicago University Press, 1999.



\end{hangparas}


\end{document}