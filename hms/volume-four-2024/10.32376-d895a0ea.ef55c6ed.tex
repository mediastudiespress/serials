% see the original template for more detail about bibliography, tables, etc: https://www.overleaf.com/latex/templates/handout-design-inspired-by-edward-tufte/dtsbhhkvghzz

\documentclass{tufte-handout}

%\geometry{showframe}% for debugging purposes -- displays the margins

\usepackage{amsmath}

\usepackage{hyperref}

\usepackage{fancyhdr}

\usepackage{hanging}

\hypersetup{colorlinks=true,allcolors=[RGB]{97,15,11}}

\fancyfoot[L]{\emph{History of Media Studies}, vol. 4, 2024}


% Set up the images/graphics package
\usepackage{graphicx}
\setkeys{Gin}{width=\linewidth,totalheight=\textheight,keepaspectratio}
\graphicspath{{graphics/}}

\title[Dead Men's Propaganda]{\emph{Propaganda: Ideology and Utopia in Comparative Communications Studies}} % longtitle shouldn't be necessary

% The following package makes prettier tables.  We're all about the bling!
\usepackage{booktabs}

% The units package provides nice, non-stacked fractions and better spacing
% for units.
\usepackage{units}

% The fancyvrb package lets us customize the formatting of verbatim
% environments.  We use a slightly smaller font.
\usepackage{fancyvrb}
\fvset{fontsize=\normalsize}

% Small sections of multiple columns
\usepackage{multicol}

% Provides paragraphs of dummy text
\usepackage{lipsum}

% These commands are used to pretty-print LaTeX commands
\newcommand{\doccmd}[1]{\texttt{\textbackslash#1}}% command name -- adds backslash automatically
\newcommand{\docopt}[1]{\ensuremath{\langle}\textrm{\textit{#1}}\ensuremath{\rangle}}% optional command argument
\newcommand{\docarg}[1]{\textrm{\textit{#1}}}% (required) command argument
\newenvironment{docspec}{\begin{quote}\noindent}{\end{quote}}% command specification environment
\newcommand{\docenv}[1]{\textsf{#1}}% environment name
\newcommand{\docpkg}[1]{\texttt{#1}}% package name
\newcommand{\doccls}[1]{\texttt{#1}}% document class name
\newcommand{\docclsopt}[1]{\texttt{#1}}% document class option name


\begin{document}

\begin{titlepage}

\begin{fullwidth}
\noindent\LARGE\emph{Book review
} \hspace{88mm}\includegraphics[height=1cm]{logo3.png}\\
\noindent\hrulefill\\
\vspace*{1em}
\noindent{\Huge{\emph{Dead Men's Propaganda: Ideology and Utopia in Comparative
Communications Studies}\par}}

\vspace*{1.5em}

\noindent\LARGE{Sue Curry Jansen}\par\marginnote{\emph{Dead Men's Propaganda: Ideology and Utopia in Comparative
Communications Studies}, reviewed by Sue Curry Jansen, \emph{History of Media Studies} 4 (2024), \href{https://doi.org/10.32376/d895a0ea.ce4ea1b0}{https://doi.org/10.32376/d895a0ea.ef55c6ed}.} \vspace*{0.75em}
\vspace*{0.5em}
\noindent{{\large\emph{Muhlenberg College}, \href{mailto:jansen@muhlenberg.edu}{jansen@muhlenberg.edu}\par}}

% \vspace*{0.75em} % second author

% \noindent{\LARGE{<<author 2 name>>}\par}
% \vspace*{0.5em}
% \noindent{{\large\emph{<<author 2 affiliation>>}, \href{mailto:<<author 2 email>>}{<<author 2 email>>}\par}}

% \vspace*{0.75em} % third author

% \noindent{\LARGE{<<author 3 name>>}\par}
% \vspace*{0.5em}
% \noindent{{\large\emph{<<author 3 affiliation>>}, \href{mailto:<<author 3 email>>}{<<author 3 email>>}\par}}

\end{fullwidth}

\vspace*{1em}


\noindent Terhi\marginnote{\includegraphics[height=0.5cm]{by-nc.png}} Rantanen. \emph{Dead Men's
Propaganda: Ideology and Utopia in Comparative Communications
Studies}. 349 pp. London: LSE
Press, 2024. \$35 (paperback).

\vspace{0.2in}

\newthought{No, Napoleon, Lenin,} Mussolini, Hitler are not feature players in this
provocatively titled volume. Propaganda is primarily, but not
exclusively, a second order concept in Rantanen's \emph{Dead Men's
Propaganda}. She focuses on the work of Harold Lasswell, Kent Cooper,
Fred Siebert, Theodore Peterson, and Wilbur Schramm, as well as
frequently overlooked figures like Paul Kecskemeti and Nathan Leites and
their contemporaries: the ``dead men'' who made contributions to the
comparative study of communication from the 1920s until mass
communication emerged as an independent academic discipline in the
United States in the 1950s. They were sociologists, political
scientists, journalists, policy scientists. Some of them were émigré
European scholars displaced by the political upheavals of the century of
total war. They were variously affiliated with universities, government,
private foundations, and the RAND Corporation: most were involved in
propaganda analysis. Rantanen refers to them as ``the forefront
generation'' (ix). However, she points out that they ``did not only
research propaganda, they were also propagandists'' (x): their
propaganda served the Allied cause during World War II and Western
interests during the Cold War. That service, she notes, raises questions
about their ethical and critical independence: questions that she
addresses. And, she stresses, they were men. In the introduction and at
other junctures in the volume, Rantanen flags

\enlargethispage{2\baselineskip}

\vspace*{3em}

\noindent{\emph{History of Media Studies}, vol. 4, 2024}


 \end{titlepage}

% \vspace*{2em} | to use if abstract spills over



\noindent the exclusion and/or
invisibility of women in the incubation of the communication discipline. Although she identifies as a media and communication scholar, Rantanen
describes the intended audience for her book as readers outside of her
discipline. She hopes to reach interdisciplinary thinkers interested in
international affairs, comparative analysis, and policy: people
wrestling with big global issues, much like members of the forefront
generation. She maintains that our world, like theirs, is increasingly
polarized, crisis-ridden, and experiencing a rise in authoritarianism
and populism. This ascendancy is being fueled by both domestic and
international misinformation and disinformation campaigns, which
undermine the legitimacy of democratic institutions and established
canons of knowledge.

Rantanen suggests that the discipline of communication has lost the
daring global vision that animated the comparative communicative studies
of the founding generation. She acknowledges that the discipline's
sub-field, ``international communication,'' does comparative work;
however, she contends that it lacks the scope, policy relevance, and
resonance of the earlier work. And, I would add, generous wartime,
government, and foundation funding. Rantanen argues that the current
global crises require mobilization of responses as audacious as those
that her subjects undertook. She believes that examining the life
histories of the forefront generation---their missteps as well as
achievements---can constructively inform and inspire development of such
an initiative.

She looks to sociology to provide the theoretical groundings for her
analysis of the founders' life histories, drawing most extensively on
the contributions of Hungarian émigré scholar Karl Mannheim and American
Robert K. Merton. Mannheim's historical sociology of knowledge and
Merton's critique of Mannheim provide Rantanen's point of departure.
Both sociologists agreed that the conditions for the production of
knowledge are important and under-studied. However, they had
significantly different life histories and intellectual perspectives.
Mannheim was a scholar of Jewish descent, whose career was twice
disrupted by historical events in Hungary and Germany, before he found
an uneasy refuge in England in 1933; his sociology examined the
existential crisis of Western culture in response to the rise of Nazism
and fascism. Merton, an assimilated American of Jewish descent from a
working-class background, was a Harvard trained sociologist who spent
most of his career at Columbia University where he advocated for
``middle-range'' theories anchored in empirical social science research.

Rantanen uses Merton's distinction between ``outsiders'' and
``insiders'' to describe how the different life histories of her
subjects influenced their scholarly and social perspectives, and
collectively enriched their contributions to comparative communicative
research. For example, she contends that the perspectives of outsiders,
whether by birth, circumstance, or choice, challenged and expanded the
insularity of insiders; conversely, insiders enlarged the horizons of
outsiders by sharing access to knowledge not usually accessible to them.
Further, she notes, sometimes outsiders became insiders, and insiders
lost their privileged status.

She applies Mannheim's dialectic of ``ideology'' and ``utopia'' to her
analysis as well as his concept of ``generation.'' Within Mannheim's
schema, symbols, ideas, dreams, and fantasies ``are `ideological' if
they serve the purpose of glossing power or stabilizing the existing
social reality; `utopian' if they inspire collective activity which aims
to conform with their goals, which transcend reality'' (23; quoted from
Mannheim, \emph{Ideology and Utopia}). The two concepts work in tandem.
However, the goals of scholars, activists, or practitioners can change
in the course of their life histories. They may become disillusioned,
frustrated, or defeated in their efforts to realize their utopian
objectives, or converted or co-opted by the prevailing ideology.
Conversely, servants of the dominant ideology (insiders) may also change
course, embrace utopian goals, and become critics, whistleblowers, or
rebels. Rantanen cites the example of Lasswell's ``shift from a young
man influenced by the League of Nations to an old man who had not only
left behind his idealistic view of international understanding but even
changed his own research interests to focus on law and order'' (247).
She also notes that ``forgetting'' can play a significant role in both
life histories and historical scholarship. Individuals may downplay or
repress their youthful enthusiasms, and future historians may ascribe
more or less significance to public figures or events than
contemporaries.

Mannheim is generally credited with pioneering ``generation'' studies.
He contends that young people, who experience key historical events
during their formative years, may develop a common
\emph{weltanschauung}: a distinctive worldview and value orientation.
However, Mannheim does not regard this as inevitable, strictly based on
birthdate, or as all-inclusive of an entire cohort. In claiming
generational consciousness for the subjects of her life histories,
Rantanen conforms to Mannheim's definitional cautions. While most
members of the forefront generation were born in the early twentieth
century and were too young to serve in the military during the First
World War and too old to serve in the second, she also includes Cooper,
who was born in 1880. He qualifies because his transformation of the
Associated Press took place during the 1920--1950 activist period of the
forefront generation and was driven by challenges posed by the world
wars and Russian Revolution.

To compose the life histories, Rantanen undertook extensive original
archival research as well as an expansive review of secondary materials.
Her opus includes seventy-five pages of supporting notes and references
and a generous array of photos. She dedicates full chapters to Lasswell
and to Cooper. Schramm also receives extensive coverage as an
institutionally savvy administrator and advocate for the academic study
of mass communication. He was also a co-author of \emph{Four Theories of
the Press}, which Rantanen describes as ``the `bible' of comparative
communication studies'' (256). Despite the fact that the text has
received extensive negative criticism, she contends that Four Theories
``became a landmark to which everyone had to refer'' in framing their
own critical dialectics (256). To wit, she points out that the 1956 US
volume has had a significant second life in China beginning in the
1980s, and a third in Russia where it became ``the foundation text for
media and journalism theory'' when it was translated in 1998 (230).

Rantanen's ambitious effort not only documents the life histories of the
forefront generation, but also the creative synergy produced by
collaborations involving people of diverse experiences and perspectives.
One can quibble about who is included and excluded. Why Cooper? Why not
Walter Lippmann? Almost a decade younger than Cooper, Lippmann was the
consummate cosmopolitan, journalist with an international audience,
policy wonk, author of communication classics \emph{Liberty and the
News} (1920) and \emph{Public Opinion} (1922), army propagandist during
World War I, and propaganda critic who co-authored ``A Test of the
News,'' now generally regarded as the prototype of modern content
analysis, the method later adopted by Lasswell. But quibbling aside, the
scholars and practitioners Rantanen does include have earned their
places in the annals of comparative communication study. In addition to
her advocacy for comparative study, she also adds to the recent
reassessment of the rich theoretical legacy of Karl Mannheim's sociology
of knowledge.

Whether any scholarly treatise can reach and mobilize Rantanen's ideal
audience of cosmopolitan movers and shakers in the present age of
digital distraction is open to question. Yet, she concludes \emph{Dead
Men's Propaganda} by finding a measure of hope in the fact that, despite
our differences, ``many academics and men {[}and women{]} of practice
share a utopian view that international communication plays a role in
peace and understanding of nations'' (270).



\end{document}