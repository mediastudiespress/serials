% see the original template for more detail about bibliography, tables, etc: https://www.overleaf.com/latex/templates/handout-design-inspired-by-edward-tufte/dtsbhhkvghzz

\documentclass{tufte-handout}

%\geometry{showframe}% for debugging purposes -- displays the margins

\usepackage{amsmath}

\usepackage{hyperref}

\usepackage{fancyhdr}

\usepackage{hanging}

\hypersetup{colorlinks=true,allcolors=[RGB]{97,15,11}}

\fancyfoot[L]{\emph{History of Media Studies}, vol. 4, 2024}


% Set up the images/graphics package
\usepackage{graphicx}
\setkeys{Gin}{width=\linewidth,totalheight=\textheight,keepaspectratio}
\graphicspath{{graphics/}}

\title[Coloniality and Resistance]{Coloniality and Resistance: The Revolutionary Moment in Communication Study in the Anglophone Caribbean} % longtitle shouldn't be necessary

% The following package makes prettier tables.  We're all about the bling!
\usepackage{booktabs}

% The units package provides nice, non-stacked fractions and better spacing
% for units.
\usepackage{units}

% The fancyvrb package lets us customize the formatting of verbatim
% environments.  We use a slightly smaller font.
\usepackage{fancyvrb}
\fvset{fontsize=\normalsize}

% Small sections of multiple columns
\usepackage{multicol}

% Provides paragraphs of dummy text
\usepackage{lipsum}

% These commands are used to pretty-print LaTeX commands
\newcommand{\doccmd}[1]{\texttt{\textbackslash#1}}% command name -- adds backslash automatically
\newcommand{\docopt}[1]{\ensuremath{\langle}\textrm{\textit{#1}}\ensuremath{\rangle}}% optional command argument
\newcommand{\docarg}[1]{\textrm{\textit{#1}}}% (required) command argument
\newenvironment{docspec}{\begin{quote}\noindent}{\end{quote}}% command specification environment
\newcommand{\docenv}[1]{\textsf{#1}}% environment name
\newcommand{\docpkg}[1]{\texttt{#1}}% package name
\newcommand{\doccls}[1]{\texttt{#1}}% document class name
\newcommand{\docclsopt}[1]{\texttt{#1}}% document class option name


\begin{document}

\begin{titlepage}

\begin{fullwidth}
\noindent\Large\emph{History of Communication Studies across the Americas
} \hspace{18mm}\includegraphics[height=1cm]{logo3.png}\\
\noindent\hrulefill\\
\vspace*{1em}
\noindent{\Huge{Coloniality and Resistance: The\\\noindent Revolutionary Moment in Communication Study in the Anglophone Caribbean\par}}

\vspace*{1.5em}

\noindent\LARGE{Nova Gordon-Bell}\par\marginnote{\emph{Nova Gordon-Bell, ``Coloniality and Resistance: The Revolutionary Moment in Communication Study in the Anglophone Caribbean,'' \emph{History of Media Studies} 4 (2024), \href{https://doi.org/10.32376/d895a0ea.bd98a921}{https://doi.org/ 10.32376/d895a0ea.bd98a921}.} \vspace*{0.75em}}
\vspace*{0.5em}
\noindent{{\large\emph{University of the West Indies}, \href{mailto:nova.gordonbell02@uwimona.edu.jm}{nova.gordonbell02@uwimona.edu.jm}\par}} \marginnote{\href{https://creativecommons.org/licenses/by-nc/4.0/}{\includegraphics[height=0.5cm]{by-nc.png}}}

% \vspace*{0.75em} % second author

% \noindent{\LARGE{<<author 2 name>>}\par}
% \vspace*{0.5em}
% \noindent{{\large\emph{<<author 2 affiliation>>}, \href{mailto:<<author 2 email>>}{<<author 2 email>>}\par}}

% \vspace*{0.75em} % third author

% \noindent{\LARGE{<<author 3 name>>}\par}
% \vspace*{0.5em}
% \noindent{{\large\emph{<<author 3 affiliation>>}, \href{mailto:<<author 3 email>>}{<<author 3 email>>}\par}}

\end{fullwidth}

\vspace*{2em}

\hypertarget{abstract}{%
\section{Abstract}\label{abstract}}

This essay discusses the development of communication and media Studies in the Anglophone Caribbean. Despite the prolific work of Anglophone researchers and scholars and over half a century of institutional engagement in media and communication education and training, a body of knowledge identifiable as the English-speaking Caribbean’s contribution to media and communication study and theory remains unidentifiable. North American, British, and European paradigms and theories constitute the hegemon of Anglophone Caribbean scholarship in media and communication study. This is consistent with the ideological role of education in the historical context of a British colonial legacy, which valorizes knowledge transfer over indigenous knowledge generation and typically elides the mastery of the English language with communicative competence. In response to the demands of the local and regional media industry, training in practical skills for journalism and media production is consistently prioritized over theory and scholarship regarding human communication. The paper points to the rich contributions of individual scholars, researchers, and teachers, and the possibilities for a distinctive Caribbean contribution to the study of communication, which could inform production and training for industry and expand theoretical and analytical approaches to research and scholarship.

\vspace*{2em}

\noindent{\emph{History of Media Studies}, vol. 4, 2024}

\hypertarget{resumen}{%
\section{Resumen}\label{resumen}}

En este ensayo se rastrea el desarrollo de los estudios de comunicación y medios en el Caribe anglófono. Pese a la nutrida obra de investigadores y estudiosos anglófonos y al más de medio siglo de compromiso institucional con la formación y capacitación en comunicación y medios, no existe todavía un corpus de conocimiento identificable como la aportación del Caribe anglófono al estudio y teoría de comunicación y medios. Los paradigmas y teorías norteamericanos, británicos y europeos ejercen la hegemonía sobre la investigación de medios y comunicación en el Caribe anglófono. Esto no sorprende dado el papel ideológico de la educación en el contexto histórico de un legado colonial británico, que privilegia la transferencia de conocimientos por encima de la generación de conocimientos indígenas, y donde suele identificarse la competencia comunicativa con el dominio del idioma inglés. Ante las demandas de la industria mediática local y regional, se prioriza la capacitación en habilidades prácticas de periodismo y de producción mediática, dejando en un segundo plano la teoría e investigación de la comunicación humana. En este trabajo se señalan las ricas aportaciones de estudiosos, investigadores y maestros individuales, así como las posibilidades de una aportación caribeña propia al estudio de la comunicación, lo que podría incidir en la producción y la capacitación para la industria, y, en última instancia, ampliar los enfoques teóricos y analíticos del campo académico y la investigación.

\enlargethispage{2\baselineskip}


 \end{titlepage}

% \vspace*{2em} | to use if abstract spills over

\newthought{The formal body} of literature, research, and inquiry which took
institutional shape after World War II in Europe and North America
``when a family of fields concerned with communication and media
institutionalized themselves around the world'' inevitably found its way
into the colleges and universities in the Anglophone Caribbean as the
canon of knowledge for media and journalism training.\footnote{Peter
  Simonson and David Park, ``Introduction,'' in \emph{The International
  History of Communication Studies}, ed. Peter Simonson and David Park
  (New York: Routledge, 2016), 3.} Institutions generating that
literature post--World War II had undoubtedly been in the business of
research and academic inquiry long before the British Act of
Emancipation in 1834 legally recognized Black British West Indians as
human beings. Fifty years after communication and media studies entered
the higher education landscape in the former British West Indies, the
Anglophone Caribbean, no body of literature is yet distinguishable as a
distinctive body of knowledge representing Anglophone Caribbean
communication and media studies.

The institutionalization of knowledge is central to the emergence and
recognition of a discipline and a field of study. Institutions
dialectically manage society as products of a society's histories,
cultures, and social norms, shaping understandings and perceptions and
structuring opportunities and limitations.\footnote{Clarence Ayres,
  \emph{Toward a Reasonable Society} (Austin: University of Texas Press,
  1961); Ayres, \emph{The Theory of Economic Progress} (New York:
  Schocken Books, 1962); John Commons, ``Institutional Economics,''
  \emph{American Economic Review} 21, no. 4 (1931); Walter Neale,
  ``Institutions,'' \emph{Journal of Economic Issues} 21, no. 3 (1987).}
Institutions serve an ideological function. Hall, having experienced
British Colonial culture and education as a child growing up in Jamaica,
proposed that inquiry into former British colonies in the West Indies
must consider how institutional cultures emerged from the peculiarities
of the British West Indian modes of production.\footnote{Stuart Hall,
  ``Pluralism, Race and Class in Caribbean Society,'' in \emph{Selected
  Writings on Race and Difference by Stuart Hall}, ed. Paul Gilroy and
  Ruth Wilson Gilmore (Durham, NC: Duke University Press, 1977).} Hall
used a theoretical frame from the region---the plantation
model\footnote{Kari Levitt and Lloyd Best, ``Character of Caribbean
  Economy,'' in \emph{Caribbean Economy: Dependence and Backwardness},
  ed. George L. Beckford (Mona, Jamaica: University of the West Indies,
  1975), 34--37.}---to critique institutions in the Anglophone
Caribbean. Central to this model is the premise that a mode of
production designed to produce and extract wealth from the colony for
the economic interests of the mother country became the mold for social,
political, and economic development in the colony.

This essay establishes educational institutions' ideological role as a
critical part of the infrastructure of domination and control, fostering
a culture of academic dependency and knowledge transfer over knowledge
generation. Secondly, we explore how engagement in that body of
literature in communication and media studies that emerged post--World
War II from the US, coupled with the British Cultural Studies movement,
would provide a kind of Trojan Horse--like cover for local scholars to
seize a ``revolutionary moment'' for critical engagement with themes and
issues related to Caribbean media and communication. This was realized
primarily through oral engagement and discussions in the ``classroom''
rather than a distinctive outflow of literature. Yet again, for a region
whose people have a strong ancient oral tradition, these contributions
would not\newpage\noindent have the same presence among the institutionalized traditions
of ``writing'' and publishing from the North.\footnote{Wimal
  Dissayanake, ``The Production of Asian Theories of Communication,'' in
  \emph{De-westernizing Communication Research: Altering Questions and
  Changing Frameworks,} ed. Georgette Wang (New York: Routledge, 2011).}

The oral tradition notwithstanding, Caribbean scholars have made
significant contributions to the literature of various international
academic fields and disciplines.\footnote{Marjan DeBruin, ``IAMCR and
  the Caribbean Region: Rethinking Our Thinking; Understanding the
  Epistemic Effects of Colonialism in Higher Education,'' in
  \emph{Reflections on the International Association for Media and
  Communication Research: Many Voices, One Forum,} ed. Jörge Becker and
  Robin Mansell (Cham, Switzerland: Palgrave McMillan, 2023).} The
Anglophone Caribbean, however, is unrepresented in the documented
histories of Communication Studies authored by scholars from North
America, Europe, or even Latin America.\footnote{Livingston White,
  ``Charting the Course of Communication Studies in the English-Speaking
  Caribbean: Disciplines, Developments and Future Directions,''
  \emph{The Journal of Human Communication Studies in the Caribbean} 1,
  no. 1 (2015).} The Anglophone Caribbean is not represented in the
documented contribution of the Americas to communication
research.\footnote{Robert T. Craig, ``Communication as a Field and
  Discipline,'' in \emph{The International Encyclopedia of
  Communication}, ed. Wolfgang Donsbach (Oxford, UK: Wiley-Blackwell,
  2008), 675­--88.} This is not a matter of exclusion or oversight. A
coherent body of Anglophone Caribbean communication research and theory
is conspicuously missing.\footnote{DeBruin, ``IAMCR and the Caribbean.''}
This essay proposes that the dominant institutional perspectives in the
region saw communication and media as limited to a context of training
for the burgeoning local and global media and journalism industries.

We highlight how the current local institutional landscape prioritizes
knowledge transfer as central to development. Local scholars faced with
the perennial need for sponsorship and funding tend to focus less on
theorizing and knowledge generation and more on applying existing
theories and approaches to resolving the plethora of social challenges
that require communication and media intervention.

\hypertarget{the-ideology-of-education-in-the-british-west-indies}{%
\section{The Ideology of Education in the British West
Indies}\label{the-ideology-of-education-in-the-british-west-indies}}

Britain enacted the Negro Education Grant in the same year that the
Emancipation Act freed enslaved Africans in their West Indian
possessions. The Education Grant sought to regulate ``the condition of
the Negroes as may combine their welfare with the interests of the
proprietors {[}of the slave sugar plantations{]} to carry on the
production of sugar.''\footnote{Shirley C. Gordon, ``The Negro Education
  Grant 1835--1845: Its Application in Jamaica,'' \emph{British Journal
  of Educational Studies} 6, no. 2 (1958): 140.} Such education as was
to be provided was concerned only with the basics required for
functioning on the sugar plantation. Education became institutionalized
as a utilitarian function. This would continue post-independence,
prioritizing the transfer of knowledge and technology to the local labor
force that services the assets of regional and global elites.

Education in the British West Indies was never concerned with knowledge
generation because knowledge was understood to reside in the mother
country. While Spanish and French colonies in the region had
universities from the early eighteenth century and mid-nineteenth
century, respectively,\footnote{DeBruin, ``IAMCR and the Caribbean,''
  364.} the British West Indies would not have a university until 1948.
Ironically, the University College of the West Indies was located on the
site of a former plantation in Jamaica. Established by royal charter as
a college of the University of London, the UCWI began as a faculty of
medicine. In 1962, the same year Jamaica became independent, the UCWI
became the University of the West Indies, UWI, a degree-granting
institution in its own right.

Queen Elizabeth II's aunt, Princess Alice, was the
university\textquotesingle s first chancellor, and her spotted lion
emblem still forms part of the university's coat of arms today. Until
2019, Queen Elizabeth II held the role of Visitor, which gave her the
right to occasionally inspect the institution and mediate disputes among
or between staff and students where all other remedies had failed.
Contributing territories of the UWI are still members of the British
Commonwealth of Nations, and some, including Jamaica, still retain the
British monarch as their Head of State. Cultures of coloniality make
maintaining these vestiges of colonialism in the region's premier
educational institution seem unproblematic. The legacy of our colonial
education systems reinforces a culture of academic and thought
dependence on Britain.\footnote{Patrick Bryan, \emph{The Jamaican
  People: 1880--1902} (London: MacMillan Caribbean, 1991).}

After five centuries of colonial rule, our institutions continue to
validate and privilege imported knowledge. Post-independence, the
education system remained critical to applying and implementing
prescriptions from the industrialized world. These institutions were
already predisposed to imitate and import knowledge uncritically. The
emergence of nationalist leaders and movements for self-government and,
ultimately, independence did not significantly dismantle the
institutional systems and culture of coloniality. A litany of literature
from regional scholars describes how the emergence of a nationalist
Black and mixed-race educated class maintained systems of coloniality in
the various sectors of the Anglophone Caribbean, consolidating their
power through control of state institutions and as mediators and agents
for British and eventually American and European interests.\footnote{Anthony
  Bogues, \emph{``}Politics, Nation and Postcolony: Caribbean
  Inflections,'' \emph{Small Axe} 6, no. 1 (2002); Aggrey Brown,
  \emph{Color, Class and Politics in Jamaica} (New Brunswick, NJ:
  Transaction, 1978); Hall, ``Pluralism, Race and Class in Caribbean
  Society''; Holger Henke, ``Ariel\textquotesingle s Ethos: On the Moral
  Economy of Caribbean Existence,'' \emph{Cultural Critique} 56 (2004):
  33--63; Percy Hintzen, ``Reproducing Domination, Identity and
  Legitimacy Constructs in the West Indies,'' \emph{Social Identities}
  3, no. 1 (1997); Aaron Kamugisha, ``The Coloniality of Citizenship in
  the Contemporary Anglophone Caribbean,'' \emph{Institute of Race
  Relations} 49 (2007); Rupert Lewis, ``Reconsidering the Role of the
  Middle Class in Caribbean Politics,'' in \emph{New Caribbean Thought:
  A Reader}, ed. Brian Meeks and Folke Lindahl (Kingston: University of
  the West Indies Press, 2001).} Coloniality, distinct from
colonialism,\textsuperscript{14} defines the locally
crafted systems of domination that apply Western Eurocentric cultural
systems in the interest of the local elite and their political and
cultural capital. The leaders, themselves products of British and
colonial education systems and cultures, were almost destined to
recreate systems of inequity and power that had for centuries been the
British West Indian experience.\textsuperscript{15}

Hall likens the transition from feudal Europe to a capitalist economy to
the transition from colonial rule to independence in the British West
Indies.\textsuperscript{16} In much the same way that the bourgeoisie in Europe
formed alliances with the feudal lords as they exchanged power, the
nationalist leaders in the Anglophone region made clear interest-based
alliances with the outgoing colonial masters.\textsuperscript{17} They
also declared support for the emerging Anglophone superpower in the
region, the USA.\textsuperscript{18} The
institutions in post-colonial territories were generally unaltered in
their culture of academic dependency.

The\marginnote{\textsuperscript{14} Ramon Grosfoguel, ``The Epistemic Decolonial
  Turn,'' \emph{Cultural Studies} 21, no. 2 (2007).} British\marginnote{\textsuperscript{15} Cyril L. R. James, \emph{Party
  Politics in the West Indies} (San Juan: Vedic Enterprises, 1962).} mercantile\marginnote{\textsuperscript{16} Hall, ``Pluralism, Race and Class in Caribbean
  Society,'' 156.} system\marginnote{\textsuperscript{17} Hall, 156.} had\marginnote{\textsuperscript{18}\setcounter{footnote}{18} Hopeton Dunn, ``Creative Resilience and
  Globalization from within Evolving Constructs for Analysing Culture,
  Innovation, and Enterprise in the Global South,'' \emph{Annals of the
  International Communication Association} 44, no. 1 (2020).} left the West Indian societies with
agricultural monocultures and an undiversified economy, with export
crops rapidly losing value on the global market.\footnote{Dawn Richards
  Elliott and Ransford Palmer, ``Institutions and Caribbean Economic
  Performance: Insights from Jamaica,'' \emph{Studies in Comparative
  International Development} 43, no. 2 (2008); Michael Witter,
  ``Caribbean Economic Thought: Advances, Retreat, Current Challenges,''
  \emph{Agrarian South: Journal of Political Economy} 10, no. 3 (2021).}
Large territories like Jamaica that were extensively used for plantation
sugar production faced extreme poverty, unemployment, and social
decay.\footnote{Norman Girvan, ``Reinterpreting the Caribbean'' in
  \emph{New Caribbean Thought: A Reader}, ed. Brian Meeks and Folke
  Lindahl (Kingston: University of the West Indies Press, 2001);
  Richards Elliott and Palmer, ``Institutions and Caribbean Economic
  Performance''; Witter, ``Caribbean Economic Thought.''} Hardly were
they through the gates of political independence, but they would have to
comply with economic directives and prescriptions for development from
agents of their old and new overlords through organizations like the WTO
and the IMF. New industries, such as bauxite and tourism, primarily
owned by foreign interests, needed skilled labor. The local media and
communication institutions, dependent on advertising and the promotional
needs of local and overseas business interests, were more concerned with
acquiring relevant skills than with research and knowledge generation.
Dominant wisdom in the region considered knowledge transference through
education as the most efficient mechanism for development.

\enlargethispage{\baselineskip}

The University of the West Indies (UWI) evolved as a symbolic part of
the educational institutional structure and systems that still imitate
and reflect British traditions.\footnote{Alan Cobley, ``The Historical
  Development of Higher Education in the Anglophone Caribbean,'' in
  \emph{Higher Education in the Caribbean: Past, Present, and Future
  Directions,} ed. Glenford D. Howe (Kingston: University of the West
  Indies Press, 2000); Bevis Peters, ``Tertiary Education Development in
  Small States: Constraints and Prospects,'' \emph{Caribbean Quarterly}
  47, no. 2--3 (2001).} To understand this is to appreciate how the
education system in the region produced an educated class of mixed-race
and Black citizens who would never be hostile to colonial values and the
British legacy long after the Union Jack was lowered and national flags
were raised. Institutions like UWI became important in validating status
in the class system post-independence. The nationalist ``independence''
moment was far from a ``revolutionary'' moment regarding transformation
from dominant elements of coloniality. UWI is still ``fed'' by a
secondary education system organized around systems of ``selection,''
which critics recognize as largely privileging the already privileged
even as it does provide some opportunities for marginalized
groups.\footnote{Collette Applewhaite, ``Is the Common Entrance
  Examination in Barbados Valid in the 21st Century in Light of Issues
  of Male Underachievement?,'' \emph{Journal of Education \& Development
  in the Caribbean} 17, no. 1 (2018); Jerome De Lisle, ``Secondary
  School Entrance Examinations in the Caribbean: Legacy, Policy, and
  Evidence within an Era of Seamless Education,'' \emph{Caribbean
  Curriculum} 19 (2012).} (A UWI education is still perceived as the
path to the middle-stratum and middle-strata professions through
traditional university courses and program offerings. UWI, for many in
the society, is still validated by its colonial association with the
monarchy and externally validated bodies of knowledge.

While many locals were, and are still, able to achieve some measure of
social and economic mobility through the education systems, the
regulation of persons to specific jobs and social strata through the
processes of selection managed by educational institutions is still
evident.\footnote{De Lisle, ``Secondary School Entrance Examinations'';
  Erica Donna Gordon, ``Problems, Pressures and Policies Affecting the
  Progress of the Caribbean Examinations Council Examinations: A
  Postcolonial Response to Secondary Education in Jamaica'' (PhD diss.,
  University College London, 2019).} One of the main criteria for social
mobility was cultural, and the command of English was central to social
status and professional advancement. If knowledge transfer is to be
effective, the citizenry would need more than a basic command of
English. The study of Communication as advanced instruction in English
serves this ideological end.

What is widely known as Communication Studies in the region is offered
as part of the Caribbean Advanced Proficiency Examinations (CAPE),
administered by the Caribbean Examinations Council (CXC) since 1994. The
University of the West Indies greatly assisted in developing the CXC and
the Anglophone Caribbean's regional school leaving exams, which replaced
the overseas examinations administered by the University of Cambridge.
The Communication Studies syllabus has three modules: Gathering and
Processing Information, Language and Community, and Speaking and
Writing. The first general objective for all three modules requires
students to ``. . . use the structures of \emph{Caribbean Standard
English} correctly and appropriately and with a degree of elegance.''
The first specific objective of all three modules requires that students
be able ``to speak and write with \emph{effective} control of the
grammar, vocabulary, mechanics and conventions of \emph{Caribbean
Standard English usage}.''

As of April 2024, on the CXC website, we read the following:

\begin{quote}
Communication Studies builds students' awareness of the centrality of
language to the normal functioning of human beings and facilitates their
ability to operate in the Caribbean linguistic environment and beyond. .
. . \emph{It focuses primarily on developing advanced competencies in
Standard English, mainly Caribbean Standard English . . .}
\end{quote}

\noindent Outside of the institutional provisions of CXC, we find compulsory
communication courses in several Jamaican post-secondary and tertiary
institutions that focus extensively on developmental, continued, or
advanced instruction in English. The ``problem'' of Creole interference
in students' use of English makes continued systematic instruction in
English necessary past high school.\footnote{Joseph Farquharson, ``The
  Black Man's Burden? Language and Political Economy in a Diglossic
  State and Beyond\emph{,}'' \emph{Zeitschrift fur Anglistik und
  Amerikanistik} 63 (2015): 158.} Communication Studies or Communication
then becomes a euphemism for continued instruction in English.
Post-colonial culture in the Anglophone Caribbean sees mastery of
English as central to and inextricable from effective communication and
presents English language competence as central to all that is
considered communication.

\hypertarget{communication-studies-the-revolutionary-moment}{%
\section{Communication Studies: The Revolutionary
Moment}\label{communication-studies-the-revolutionary-moment}}

What World War II was to the development of the body of literature and
the field we now recognize as Communication Studies in North America and
Europe, the Non-Aligned Movement and the emergent call for a New World
Information Order would become, for the development of media and
communication study in the Anglophone Caribbean. American hegemony in
both media as a global industry and Anglo-American and European
communication media research and theory created an opening for
Communication Studies to be recognized as a legitimate field within the
University of the West Indies' traditional offerings. Media training and
mass media development were part of a dominant Western paradigm for
developing post-colonial societies.\footnote{Thomas McPhail,
  \emph{Electronic Colonialism: The Future of International Broadcasting
  and Communication} (Beverley Hills, CA: Sage, 1981).} The ferment in
the United Nations and the role of international development agencies
who actively prescribed mass media development and training as critical
to Third World Development significantly influenced our higher education
systems.

Anglophone nationalist leaders bought into this narrative, swiftly
setting up national radio and television systems. After all, the
colonial powers had set up media systems, primarily radio, to expedite
empire management and control, especially during the war
years.\footnote{Hopeton S. Dunn, ``Imperial Foundations of 20th-Century
  Media Systems in the Caribbean,''

  \emph{Critical Arts: A South-North Journal of Cultural \& Media
  Studies} 28, no. 6 (2014).} Brown suggests that nationalist leaders
launched into setting up national media systems without fully
appreciating the implications of the technologies, especially
television, which would rely heavily on imported content.\footnote{Aggrey
  Brown, ``The Mass Media of Communications and Socialist Change in the
  Caribbean: A Case Study of Jamaica,'' \emph{Caribbean Quarterly} 22,
  no. 4 (1976): 47--48.} By the time the UWI started the first program
offerings---a diploma and then a degree in Mass Communication in the
1970s---Jamaica and the region were already inundated with foreign media
content and an established British tradition in journalism and
broadcasting. Newspaper journalism and broadcasting companies prepared
workers through apprenticeship and skill development in the newsroom.
Broadcasters were sent to the BBC for training and mentorship, and that
was considered a privilege and an essential part of the ranking or
status of journalists who were so honored.

North America's emphasis on professionalizing media and journalism
through college and university education would gradually influence media
and communication education in Jamaica and the wider Anglophone
Caribbean. The cumulative external stimulus from the North made the
Anglophone regional university receptive to a modified status quo in its
otherwise traditional offerings. Another factor that stirred the
university to make room for communication and media training was the
need for growing local and regional media organizations.

The political mood in the region in the 1970s also primed the
university's otherwise traditional culture to be receptive to new
offerings. CARIMAC, the Caribbean Institute of Mass Communication (now
the Caribbean School of Media and Communication), was set up to serve
the Anglophone Caribbean through the collaborative effort of UNESCO and
the Friedrich Ebert Foundation. Its terms of reference were primarily
formal training in media and communication. Regional and local scholars
saw the opportunity to use the rationale of media training as a platform
for germinating a ``Caribbean consciousness'' among media workers and a
critical perspective regarding the study of Communication and Media. The
establishment of CARIMAC on UWI's Jamaica campus coincided with the
Democratic Socialist ``moment'' in Jamaican politics.

Jamaica's Prime Minister, Michael Manley, became a significant voice in
the Non-Aligned Movement. He was undergirded by a constituency of
regional academics and scholars who ranged from moderate to left of
center and far left. The University of the West Indies became the hotbed
for much of this ferment.\footnote{Witter, ``Caribbean Economic
  Thought,'' 475.} Witter notes the impact of this moment on stimulating
Caribbean contributions to economic theory and the opportunities it
provided for those thought leaders to ``reinterpret conventional and
radical theories for the analysis of Caribbean reality'' and revise
university curricula.\footnote{Witter, 468.} While Michael
Manley\textquotesingle s administration was repatriating assets and
levying multinational bauxite companies, Caribbean scholars in
economics, sociology, and other disciplines were reflecting new critical
perspectives in journals and books.

\enlargethispage{\baselineskip}

However, communication and media studies in the region seemed locked
into a paradigm of training for media production and journalism. The
dominant perspective was that capitalist economies taking shape in the
region needed media services for their promotional
operations.\footnote{Brown, ``The Mass Media of Communications and
  Socialist Change,'' 47.} In the interest of funding and local and
external support, journalism, media production skills, and improved
practice had to be seen to be prioritized. Our scholars record the
emergence of Communication and Media Studies in the Anglophone Caribbean
as the rollout of primarily undergraduate degree programs, rather than a
body of distinctly Anglo-Caribbean original literature in Communication
and Media.\footnote{Korah Belgrave, ``The Development of Communication
  Studies at the University of the West Indies, Cave Hill Campus:
  Lessons from the Trenches,'' \emph{Journal of Human Communication
  Studies} 1, no. 1 (2015); Godfrey Steele, ``The Human Communication
  Discipline: Pathways in the Anglophone Caribbean,'' \emph{Journal of
  Human Communication Studies} 1, no. 1 (2015); White, ``Charting the
  Course''; DeBruin, ``IAMCR and the Caribbean.''} DeBruin notes the
absence of regional or local texts and reading material as resources for
journalism and media programs in the region.\footnote{DeBruin, ``IAMCR
  and the Caribbean,'' 367.} Institutionally, this would allow program
development as it was generally accepted that appropriate and relevant
knowledge is easily obtained from North American and British texts and
readers. Institutional cultures are central to maintaining a culture of
coloniality that institutionalizes the ``uncritical adoption of
externally generated theoretical constructs.''\footnote{Hopeton Dunn et
  al., ``Re-Imagining Communication in Africa and the Caribbean:
  Releasing the Psychic

  Inheritance,'' in \emph{Re-Imagining Communication in Africa and the
  Caribbean: Global South Issues in Media, Culture and Technology}, ed.
  Hopeton Dunn et al. (Basingstoke, UK: Palgrave Mcmillan, 2021), 6.}

Despite attempts to resolve the textbook issue,\footnote{Hopeton Dunn,
  ``Caribbean,'' in \emph{Inventory of Textbooks in Communication
  Studies Around the World: Final Report of the Project}, ed. Kaarle
  Nordenstreng (Tampere, Finland: University of Tampere, 1998).} North
American texts still outstrip the generation of local or regional
resources. Media and communication scholars continue to discuss the
production of texts and readers for use in Caribbean classrooms;
however, with universities ranking textbooks and teaching resources low
on the scale of points required for promotion, motivation seems low.
Programs in communication and media continue to proliferate in national
and private institutions in the Anglophone Caribbean without any
significant emergence of regional literature on issues in Caribbean
Communication Studies or analytical texts on Caribbean media. Jamaica
never paid considerable attention to the development of Spanish as a
second language beyond occasional utterances by political leaders, so
students enrolled in media and communication programs had limited access
to emerging Latin American literature in the field.

A few core courses concerned developing Caribbean thought in media and
communication. The late Professor Emeritus W. Aggrey Brown (1941--2011),
the longest-serving director of CARIMAC from 1979 to 2002, was a
significant figure in this moment. Brown earned his MA and PhD in
Political Science from Princeton University. On his return to Jamaica,
Brown immersed himself in local media as a newspaper columnist, TV
videographer, radio announcer, news analyst, and moderator of a
widespread and often controversial radio call-in program. A culturally
and historically contextual response to studying communication in the
Anglophone Caribbean came in an almost Trojan Horse--like context
embedded in a compulsory course for first-year students. Communication
and Society, which evolved into Communication Culture and Caribbean
Society, was quickly dubbed ``Aggrey's course.''

Elements of the institutionalized family of fields which Simonson and
Park reference found their way into those discussions alongside Latin
American contributions to development communication.\footnote{Simonson
  and Park, ``Introduction.''} Freire's \emph{Pedagogy of the Oppressed}
was compulsory reading, as was \emph{The Media Are American} and Mass
\emph{Communication and Society}.\footnote{Paulo Freire, \emph{Pedagogy
  of the Oppressed} (Harmondsworth, UK: Penguin, 1972); Jeremy Tunstall,
  \emph{The Media Are American} (New York: Columbia University Press,
  1977); James Curran, Michael Gurevitch, and Janet Woollacott, eds.,
  \emph{Mass Communication and Society} (London: Hodder Arnold, 1979).}
Not to be left out was the copious documentation of the MacBride Report,
\emph{Many Voices, One World: Towards a New, More Just, and More
Efficient World Information and Communication Order},\footnote{International
  Commission for the Study of Communication Problems, \emph{Many Voices,
  One World: Towards a New, More Just, and More Efficient World
  Information and Communication Order} (New York: UNESCO, 1980).} and
yellowing photocopies of Chief Seattle's speech, which too was required
reading.

``Aggrey's course'' was a course of questions and redefinitions. The
seed of potential Caribbean communication thought presented more as a
classroom experience for students than a formal published body of
literature. Questions about media ownership and control, the role of
journalists, and the evolution of new forms of media and occupations
provoked fiery debates between perspectives left of center and
perspectives that reminded us of the ``real world'' of the Caribbean,
where powerful local private sector--owned enterprises and transnational
corporations ---including advertising agencies---were or would soon be
the students' employers. Brown determined that the students' primary
task was to re-signify based on our context and culture. True
independence and development would only come from naming our world and
setting our priorities.\footnote{Aggrey Brown, ``The Dialectics of
  Mediated Communication in Development as Historical Process,'' in
  ``Culture and Development vs. Cultural Development,'' ed. Helen Gould
  and Kees Epskamp, special issue, \emph{Culturelink Review} (2000).} As
descendants of people brought to the plantations as ``resources,'' we in
the South needed to be wary of the new neoliberal orthodoxies of
structural adjustment, deregulation, liberalization, and privatization
that felt the need to signify us as ``resources'' albeit
``human.''\footnote{Brown, 172.}

Western paradigms that framed development as material conquest of the
environment needed to be redefined into ``the reciprocal action between
people and their environment that leads to the actualization of human
potential in all its dimensions and to the conservation and continuity
of the environment as people create society and
{[}H{]}istory.''\footnote{Brown, 162.} History with the uppercase
\emph{H} is ``the total of the dialectic of people and their
environment.''\footnote{Brown, 162.} The lower case \emph{h} is
transferred to ``the recorded abstraction from that totality, frozen in
time,'' which is subject to the ``scribe or griot's perspective and the
purpose to be served by such abstraction.''\footnote{Brown, 159.}

Brown's sentiments harmonized with the advocates of development
communication who directed attention away from Western paradigms that
universalized their historicity and contexts with their inherent
emphases on materialism and empiricism.\footnote{Jan Servaes,
  \emph{Communication for Development: One World, Multiple Cultures}
  (New York: Hampton Press, 1999).} Media was a tool that could be used
for different agendas. Nationalist leaders in the Anglophone Caribbean,
not understanding ``why and how the media influence the behaviour of
people'' were misusing the media and subsequently undermining the
national transformation needed for true independence.\footnote{Aggrey
  Brown, ``The Dialectics of Mass Communication in National
  Transformation,'' \emph{Caribbean Quarterly} 27, no. 2--3 (1981): 40.}
Brown observes:

\begin{quote}
The mass media of communications are a vital industry within the
consumer economy, used to advertise goods and services within the
profit-making milieu. Disseminating news and information and providing
entertainment are ancillary functions to this primary
purpose.\footnote{Brown, ``Mass Media of Communications,'' 47.}
\end{quote}

\noindent In his classes, Brown argued that revolutionaries must inevitably
challenge abstractions. Brown references Freire and Fanon as
revolutionaries who challenged the abstractions of {[}h{]}istory, which
the hegemon typically treats as absolute and universal. For genuinely
independent young nations to experience transformation, we must
understand that all human action and interaction is \emph{mediated}
through tools of signs, symbols, language, and media hardware. However,
more importantly, ``people are required to perform tasks they never
performed before and may never have thought themselves capable. From
passive observers of the process, they are impelled to become active
participants. From being mere objects of {[}h{]}istory, they become
subjects creating {[}H{]}istory.''\footnote{Brown, ``The Dialectics of
  Mediated Communication,'' 160.}

Much of what was shared in the classes then was oral, sporadically
documented and largely informal. As the relevant citations indicate,
Brown did go on to publish much of his insight. His PhD dissertation
\emph{Color, Class and Politics in} \emph{Jamaica} provided a
significant platform for redefinitions and resignifications in our
understanding of class, which has always been a central concept in
considering Caribbean societies.\footnote{Brown, \emph{Color, Class and
  Politics}.} Brown's work demonstrated what Hall himself identified as
the impact of the peculiarities of the plantation on West Indian culture
and its institutions through the reification of race and class in local
political and power systems.\footnote{Hall, "Pluralism, Race and Class
  in Caribbean Society"; Stuart Hall, ``Negotiating Caribbean
  Identities'' in \emph{New Caribbean Thought: A Reader}, ed. Brian
  Meeks and Folke Lindahl (Mona, Jamaica: University of the West Indies
  Press, 2001).} Brown redefined class from the limitations of Marxist
and neo-Marxist articulations, which Hall suggested could not adequately
explain how ``thought'' and idea could be definitively tied to
socio-economic class.\textsuperscript{49} Brown proposed\marginnote{\textsuperscript{49}\setcounter{footnote}{49} Stuart Hall, ``The Problem of Ideology:
  Marxism without Guarantees,'' in \emph{Stuart Hall: Critical
  Dialogues,} ed. David Morley and Kuan-Hsing~Chen (London: Routledge,
  1996).} that ``class'' be understood based on the
direction of human energy regarding society. In every society,
therefore, only three classes could be found to exist: 1) a conservative
class that exerted its energies (performed tasks) aimed at maintaining
the status quo, 2) a destructive class that simply destroyed the status
quo without any idea of alternatives, and 3) a transformative class that
pursued dialogue and a collaborative search for justice.\footnote{Brown,
  \emph{Color, Class and Politics}, 8, 9.} The relevance of Brown's
ideas for a critical Caribbean perspective on communication and media
theorizing is evident. One of the many questions is why it did not come
to drive and shape the distinctive body of literature and
institutionalize the field in the region beyond media and journalism
training.

In 1994, CARIMAC launched the first graduate program, a taught master's
in Communication Studies. In 1998, the graduate program offering added
an MPhil/PhD program in Communication Studies. Hopeton Dunn was the
chief architect of that program, working closely with Aggrey Brown and
Marjan DeBruin. Professor of Communications Policy and Digital Media,
Dunn was Director of CARIMAC from 2012 to 2018 and is one of the most
extensively published researchers on media and ICT in the Anglophone
Caribbean. A former head of UWI's Mona ICT Policy Centre (MICT), Dunn
extended Communication Studies to incorporate telecommunications
developments through research into technology policy reforms, new media
in the Global South, telecommunications for development, and
broadcasting regulation. Dunn's contribution was particularly
significant in its response to infrastructural relics of the colonial
period, e.g., Cable and Wireless Jamaica, which held a monopoly over
Jamaican telecoms for centuries until the twenty-first century. Having
earned his PhD at City University London in 1991, his dissertation was
entitled ``Telecommunications and Underdevelopment: A Policy Analysis of
the Historical Role of Cable and Wireless in the Caribbean.'' Courses in
policy and governance were added to the curriculum.

As an advocate for critical thought, Dunn also advocated the need for
new insights and new approaches, that perspectives for Caribbean
scholarship ``must come from all parts of the world and not just from
the privileged North.''\footnote{Dunn, ``Reimagining,'' 6.} His
scholarly collaboration with African scholars infused Caribbean
Communication Studies with fresh thought and new deliberations on the
de-westernization and indigenization of theory from the Global South.

The first cohort of graduate students in the MA in Communication Studies
were PR practitioners, a trade unionist turned HR director (who would
later go on to become a member of Parliament and a minister of
government), a university lecturer (a trained journalist who had also
worked in advertising and public information), a newspaper columnist who
described himself in his by-line as a communication consultant even
though his formal academic credentials were in the natural sciences, and
one who described himself as a ``working journalist.'' Most of the
cohort had formal journalism/mass communication training locally or
overseas. Unlike the diploma and bachelor's degree, the MA in
Communication Studies curriculum taught no production skills. It
involved interdisciplinary incursion into political science, sociology,
policy, and research with a significant influence from the literature of
the British Cultural Studies movement and mainstream material from North
America and Europe. An overview of the courses in the program evinces
the intent of the course designers:

\begin{quote}
Caribbean Media Communication \& Society; Socio-Cultural Issues in
Caribbean Communication: Race, Class, Ethnicity and Gender;
Communications Policy \& Technology in the Caribbean, Communication
Theories, Design \& Methods in Communication Research; Communications \&
Media Management; Research Design and Fieldwork.
\end{quote}

\noindent Much of the deliberations, themes and issues germinating in the
undergraduate program returned in 1994. However, much had changed in the
global environment, and the internet was well on its way to redefining
what we formerly knew as mass media. The Thatcher and Reagan years had
also redefined the discourse and Democratic Socialism, and Manley had
retired from the Jamaican landscape. New political leaders overtly
affirmed their commitment to the capitalist economy, and it was not
insignificant that the new Jamaican prime minister, Edward Seaga, who
replaced Manley, was the first foreign leader to be invited to meet with
the newly elected President Ronald Reagan.

\hypertarget{working-with-the-hegemon-knowledge-transfer-over-knowledge-generation}{%
\section{Working with the Hegemon: Knowledge Transfer over Knowledge
Generation}\label{working-with-the-hegemon-knowledge-transfer-over-knowledge-generation}}

By 1994, the revolutionary moment was behind us. The neoliberal
orthodoxy was by then the hegemon, and much of what had emerged as
``critical'' thought had gone into retreat. Disciplines like economics
and government that had a ``rich tradition of critical thought'' in
Caribbean scholarship experienced retreats and recessions ``in the face
of the economic power of the IMF and the World Bank, the prestige of the
Washington Consensus among political leaders, and the generation of
economists that succeeded the New World thinkers, and the military might
of the United States.''\footnote{Witter, ``Caribbean Economic Thought,''
  473.} Witter describes how UWI's departments replaced academic
literature in the curriculum with resources that reflected the new
economic hegemon:

\begin{quote}
Sectoral studies, particularly monetary studies that sought to
contribute to the management of the economy defined conventionally
balanced budgets, stable prices, low debt. The rationale was that these
conditions would somehow lead to economic growth, the benefits of which
would trickle down to the masses of the population.\footnote{Witter,
  473.}
\end{quote}

It would not be accurate to say that critical thought also receded in
Communication Studies as that definable body of Caribbean theory
comparable to Caribbean thought and scholarship in economics and other
disciplines had never materialized. Most of the graduate students who
passed through the Communication Studies program would return to jobs in
the corporate world, the public sector, and non-governmental
organizations to function as managers of communication, information, and
PR departments. Few would go on to focus on theory and research, and not
many could be absorbed by these departments to pursue the kind of
writing and study that generates literature that will define a
``school'' of thought. Fewer than fifteen PhD graduates have come out of
UWI's PhD in Communication Studies program since it started in the
1990s. Our output has been modest and has primarily been the application
of existing theories to local and or regional contexts. Graduates of
higher degrees typically pursue consultancies. The application of
existing mainstream theories to local and regional contexts generally
attracts more funding and corporate support.

The economic role of mass media in our emerging consumer economies
obliged the UWI to consider its role in the regional economy and its
responsibility in the job market. The leading employers of labor, large
private-sector enterprises, the public sector, and other training and
educational institutions recognized UWI as necessary for the job market.
UWI also had to ensure that their certification was respected in the
North American job markets, owing to high levels of migration and
international mobility of the Jamaican middle class and lower middle
class.\footnote{Carl Stone, ``Public Opinion Perspectives on the
  University of the West Indies,'' \emph{Caribbean Quarterly} 29, no.
  3--4 (1983): 22.} While some reports laud the university as the space
for resistant thought and counter-hegemonic thinking,\footnote{DeBruin,
  ``IAMCR and the Caribbean''; Verene Shepherd, ``Obstacles to the
  Creation of Afrocentric Societies in the Commonwealth Caribbean''
  (presentation at the 10th session of the Working Group of Experts on
  People of African Descent, Geneva, Switzerland, March--April, 2011).}
which was reflected as much in media commentaries as it was in lecture
and tutorial rooms, UWI was also the space for much tension between
resistant and counter-hegemonic thinkers. Stone articulates this tension
in his review of public perspectives on the UWI.

\begin{quote}
All societies reproduce their political institutions and structures of
power partly through the impact of dominant social, economic and
political ideologies and belief systems which reinforce and consolidate
the status quo. Additionally, societies experience the challenge to
change partly induced (among other things) by the impact of questioning
ideologies, which become a basis for critical evaluation of the status
quo. Universities generate and reproduce both types of ideological
currents and therefore will be seen by subsectors of national public
opinion as either strengthening or weakening the existing structure of
power and institutionalized authority in the society.\footnote{Stone,
  ``Public Opinion,'' 23.}
\end{quote}

\noindent CARIMAC's diversification of program offerings in response to market and
industry needs may have inadvertently relocated it within the dominant
ideological current, in which communication and media studies are about
skills for industry. Graduate offerings now include an MA in
Communication for Social and Behaviour Change, an MA in Integrated
Marketing Communication, and an MSc in Media Management. Student inflows
to the more industry-related and ``practice'' oriented programs have
resulted in the cessation (we hope temporarily) of the MA in
Communication Studies offering. Of course, the PhD in Communication
Studies remains a generic program for any candidate who wishes to
conduct research related to media or communication.

Two other campuses of the UWI are now offering degrees in Communication
that are presented as distinctive from CARIMAC's since CARIMAC is still
widely perceived as a school for media training. Professor Godfrey
Steele, the crafter of the undergraduate degree in Communication on
UWI's St. Augustine campus in Trinidad, describes that institution's
offering of Human Communication Studies as ``an amalgamation of many
things, but probably most notably, it seeks to understand how people
create, exchange and share messages and meaning in every context and how
the study of human communication relates to our humanity.''\footnote{Godfrey
  Steele, ``Human Communication Studies: What, Why and How from a
  Caribbean Perspective,'' inaugural lecture, University of the West
  Indies at St. Augustine, Trinidad and Tobago, January 27, 2022,
  YouTube video, 1:56:06.} UWI's Cave Hill campus in Barbados offers an
undergraduate minor seeking to develop student skills in rhetoric and
rhetorical criticism.\footnote{Belgrave, ``Development of Communication
  Studies at the University of the West Indies.''}

The dominant perspective of the study of language, which is the study of
communication, still thrives on the Mona campus in Jamaica. The
Linguistics department offers a BA in Language, Communication, and
Society, which:

\begin{quote}
focuses on the social and communicative value of language and includes
courses that develop a deeper understanding of the organising principles
behind language structure. Some signature courses in this major include
Language, Gender \& Sex, The Language of Negotiation, and Language
Planning, Phonology, Syntax, Structure of the English Language, and the
Sociology of Language.
\end{quote}

\noindent The understanding is that CARIMAC's communication offerings are to be
understood as ``mediated communication'' or, more precisely, media
production and journalism training and related research.

Communication as a field still presents significantly as taught courses,
even at the graduate level. Several efforts at collaboration across
institutions locally and regionally have occurred with limited success.
Institutions delivering media and communication programs seemed more
focused on competing for the student market than developing the field.

Graduate students in research programs in communication study and
graduates of these degrees have published and are publishing using the
theories, methods, and approaches acquired from mainstream schools. The
institutionalization of teaching and training for the job market and
instruction in skills outpaces reflection, knowledge creation, and
critical inquiry. Our scholars respond to pressing research needs and
opportunities and the research needs of our graduate students with
existing theoretical tools as that more often facilitates approval
internally and externally. A systematic review and reflection of what
our scholars and graduate students have been researching and publishing
would be an essential first step towards at least being able to describe
what we have been exploring and what themes and issues are present in
our existing literature.

Surinamese writer Sankatsing defines the Caribbean experience of thought
dependence as ``envelopment and not true development'':

\begin{quote}
Achievements of the West, separated from their specific historicity,
were transferred to other latitudes as universal, context-free
yardsticks for the future of all geographic destinies and landscapes.
Three continents, including ours, were reduced to ``trailer societies''
without the engine and heartbeat to shape their history.\footnote{Glenn
  Sankatsing, ``The Caribbean between Envelopment and Development''
  (lecture, University of Quintana Roo, Mexico, May 15, 2003).}
\end{quote}

\noindent In almost fifty years of institutional operation as a facility for the
study of media and communication, we have in the Anglophone Caribbean a
rich tradition of insights and potential theoretical frameworks of
thought that can shape an Anglophone Caribbean contribution to
Communication Studies. Ideological and cultural stimuli driving the
factors defining Communication Studies in the region compel an emphasis
on practical skills for industry function to meet the needs of a third
wave of globalization for a capitalist economy. Historically, we have
yet to be acculturated to reflect. We have been primed to accept what is
taken for granted and build with the tools others give us.

White suggests that change can only begin when Caribbean researchers
start to reflect on their work and the rationale for their modes of
inquiry.\footnote{White, ``Charting the Course,'' 15.} Some worry that
over the decades, we have become mired in reflection, redefinitions,
discussion, and informal debate, but White is not alone in the call for
reflection. Sankatsing suggests that

\begin{quote}
Caribbean scholars can . . . bring about a fundamental reflection, not
to satisfy supreme academic interests, but to ensure the very survival
of our societies and cultures in the Caribbean Basin; a crucial
reflection to provide us with a viable plan for the society in an
encounter of our present and future, starting with the genesis of the
countries of the Caribbean.\footnote{Sankatsing, ``The Caribbean between
  Envelopment and Development,'' 1.}
\end{quote}

\noindent In a media-dominated world and economy, a Caribbean response to
knowledge generation and theorizing in Communication and Media Studies
would be a strategic point of departure.




\section{Bibliography}\label{bibliography}

\begin{hangparas}{.25in}{1} 



Applewhaite, Collette M. ``Is the Common Entrance Examination in
Barbados Valid in the 21st Century in Light of Issues of Male
Underachievement?'' \emph{Journal of Education \& Development in the
Caribbean} 17, no. 1 (2018): 194--216.

Ayres, Clarence. \emph{Toward a Reasonable Society.} Austin: University
of Texas Press, 1961.

Ayres, Clarence. \emph{The Theory of Economic Progress.} New York:
Schocken Books, 1962.

Belgrave, Korah. ``The Development of Communication Studies at the
University of the West Indies, Cave Hill Campus: Lessons from the
Trenches.'' \emph{Journal of Human Communication Studies} 1, no. 1
(2015).

Bogues, Anthony. ``Politics, Nation and Postcolony: Caribbean
Inflections.'' \emph{Small Axe} 6, no. 1 (2002): 1--30.

Brown, Aggrey. \emph{Color, Class and Politics in Jamaica.} New
Brunswick, NJ: Transaction, 1979.

Brown, Aggrey. ``The Dialectics of Mass Communication in National
Transformation.'' \emph{Caribbean Quarterly} 27, no. 2--3 (1981):
40--46.

Brown, Aggrey. ``The Dialectics of Mediated Communication in Development
as Historical Process.'' In ``Culture and Development vs. Cultural
Development,'' edited by Helen Gould and Kees Epskamp. Special issue,
\emph{Culturelink Review} (2000): 159--72.

Brown, Aggrey. ``The Mass Media of Communications and Socialist Change
in the Caribbean: A Case Study of Jamaica.'' \emph{Caribbean Quarterly}
22, no. 4 (1976): 43--49.

Bryan, Patrick. \emph{The Jamaican People: 1880--1902.} London:
MacMillan Caribbean, 1991.

Cobley, Alan. ``The Historical Development of Higher Education in the
Anglophone Caribbean.'' In \emph{Higher Education in the Caribbean:
Past, Present, and Future Directions}, edited by Glenford D. Howe,
1--23. Kingston: University of the West Indies Press, 2000.

Commons, John. ``Institutional Economics.'' \emph{American Economic
Review} 21, no. 4 (1931): 648--57.

Craig, Robert T. ``Communication as a Field and Discipline.'' In
\emph{The International Encyclopedia of Communication}, edited by
Wolfgang Donsbach, 675--88. Oxford, UK: Wiley-Blackwell, 2008.

Curran, James, Michael Gurevitch, and Janet Woollacott. \emph{Mass
Communication and Society.} London: Hodder Arnold, 1979.

DeBruin, Marjan. ``IAMCR and the Caribbean Region: Rethinking Our
Thinking; Understanding the Epistemic Effects of Colonialism in Higher
Education.'' In \emph{Reflections on the International Association for
Media and Communication Research: Many Voices, One Forum,} edited by
Jörge Becker and Robin Mansell, 361--71. Cham, Switzerland: Palgrave
McMillan, 2023.

De Lisle, Jerome. ``Secondary School Entrance Examinations in the
Caribbean: Legacy, Policy, and Evidence within an Era of Seamless
Education.'' \emph{Caribbean Curriculum} 19 (2012): 109--43.

Dissayanke, Wimal. ``The Production of Asian Theories of
Communication.'' In \emph{De-westernizing Communication Research:
Altering Questions and Changing Frameworks,} edited by Georgette Wang,
222--37. New York: Routledge, 2011.

Dunn, Hopeton. ``Caribbean.'' In \emph{Inventory of Textbooks in
Communication Studies around the World: Final Report of the Project,}
edited by Kaarle Nordenstreng. Tampere, Finland: University of Tampere,
1998. Accessed March 24, 2019. \url{http://www.uta.fi/cmt/textbooks/}.

Dunn, Hopeton. ``Creative Resilience and Globalization from within
Evolving Constructs for Analysing Culture, Innovation, and Enterprise in
the Global South.'' \emph{Annals of the International Communication
Association} 44, no. 1 (2020): 4--18.

Dunn, Hopeton. ``Imperial Foundations of 20th-Century Media Systems in
the Caribbean\emph{.}'' \emph{Critical Arts: A South-North Journal of Cultural \& Media Studies}
28, no. 6 (2014): 938--57.

Dunn, Hopeton S., Dumisani Moyo, William O. Lesitaokana, and Shanade
Bianca Barnabas. ``Re-Imagining Communication in Africa and the
Caribbean: Releasing the Psychic Inheritance.'' In \emph{Re-Imagining
Communication in Africa and the Caribbean: Global South Issues in Media,
Culture and Technology}, edited by Hopeton S. Dunn, Dumisani Moyo,
William O. Lesitaokana, and Shanade Bianca Barnabas, 1--15. Cham,
Switzerland: Palgrave Macmillan, 2021.

Farquharson, Joseph T. "The Black Man\textquotesingle s Burden? Language
and Political Economy in a Diglossic State and
Beyond."~\emph{Zeitschrift Für Anglistik Und Amerikanistik}~63, no. 2
(2015): 157--77.

Freire, Paulo. \emph{Pedagogy of the Oppressed}. Harmondsworth, UK:
Penguin, 1972.

Girvan, Norman. ``Reinterpreting the Caribbean.'' In \emph{New Caribbean
Thought: A Reader}, edited by Brian Meeks and Folke Lindahl, 3--23.
Kingston: University of the West Indies Press, 2001.

Gordon, Erica D.~``Problems, Pressures and Policies Affecting the
Progress of the Caribbean Examinations Council Examinations: A
Postcolonial Response to Secondary Education in Jamaica.''~PhD diss.,
University College London, 2019.

Gordon, Shirley C. ``The Negro Education Grant 1835--1845: Its
Application in Jamaica.''~\emph{British Journal of Educational
Studies}~6, no. 2 (1958): 140--50.

Grosfoguel, Ramon. ``The Epistemic Decolonial Turn.'' \emph{Cultural
Studies} 21, no. 2 (2007): 211--23.

Hall, Stuart. ``Negotiating Caribbean Identities.'' In \emph{New
Caribbean Thought: A Reade}r, edited by Brian Meeks and Folke Lindahl,
24--39. Kingston: The University of the West Indies Press, 2001.

Hall, Stuart. "Pluralism, Race and Class in Caribbean Society
{[}1977{]}." In \emph{Selected Writings on Race and Difference}, edited
by Paul Gilroy and Ruth Wilson Gilmore, 136­--60. Durham, NC: Duke
University Press, 2021.


Hall, Stuart. ``The Problem of Ideology: Marxism without Guarantees.''
In \emph{Critical Dialogues,} edited by David Morley and Kuan-Hsing
Chen, 25--46. London: Routledge, 1996.

Henke, Holger. ``Ariel\textquotesingle s Ethos: On the Moral Economy of
Caribbean Existence.'' \emph{Cultural Critique} 56 (2004): 33--63.

Hintzen, Percy. ``Reproducing Domination, Identity and Legitimacy
Constructs in the West Indies.'' \emph{Social Identities} 3, no. 1
(1997): 47--76.

International Commission for the Study of Communication
Problems.~\emph{Many Voices, One World: Towards a New, More Just and
More Efficient World Information and Communication Order}. New York:
UNESCO, 1980.

James, Cyril L.R. \emph{Party Politics in the West Indies}. San Juan:
Vedic Enterprises, 1962.

Kamugisha, Aaron. ``The Coloniality of Citizenship in the Contemporary
Anglophone Caribbean.'' \emph{Institute of Race Relations} 49, no. 2
(2007): 20--40.

Levitt, Kari Polanyi, and Lloyd Best. ``Character of Caribbean
Economy.'' In \emph{Caribbean Economy: Dependence and} Backwardness,
edited by George L. Beckford, 34--60. Mona, Jamaica: Institute of Social
and Economic Research, University of the West Indies, 1975.

Lewis, Rupert. ``Reconsidering the Role of the Middle Class in Caribbean
Politics.'' In \emph{New Caribbean Thought: A Reader,} edited by Brian
Meeks and Folke Lindahl, 127--43. Kingston: University of the West
Indies Press, 2001.

McPhail, Thomas. \emph{Electronic Colonialism: The Future of
International Broadcasting and Communication}. Beverley Hills, CA: Sage,
1981.

Neale, Walter C. ``Institutions.'' \emph{Journal of Economic Issues} 21,
no. 3 (1987): 177--206.

Nordenstreng, Kaarle. \emph{Inventory of Textbooks in Communication
Studies around the World: Final Report of the Project}. Tampere,
Finland: University of Tampere, 1998.

Peters, Bevis. ``Tertiary Education Development in Small States:
Constraints and Prospects.'' \emph{Caribbean Quarterly} 47, no. 2--3
(2001): 44--57.

Richards Elliott, Dawn, and Ransford W. Palmer. ``Institutions and
Caribbean Economic Performance: Insights from Jamaica.'' \emph{Studies
in Comparative International Development} 43, no. 2 (2008): 181--205.

Sankatsing, Glenn. ``The Caribbean between Envelopment and
Development.'' Lecture, University of Quintana Roo, Mexico, May 15,
2003. \href{https://coleccion.aw/show/?BNA-DIG-ARTIKEL-SANKATSING-2016}{https://coleccion.aw/show/?BNA-DIG-ARTIKEL-SANKATSING-2016}.

Servaes, Jan. \emph{Communication for Development: One World, Multiple
Cultures.} New York: Hampton Press, 1999.

Shepherd, Verene. ``Obstacles to the Creation of Afrocentric Societies
in the Commonwealth Caribbean.'' Presentation at the 10th session of the
Working Group of Experts on People of African Descent, Geneva,
Switzerland, March--April, 2011.

Simonson, Peter and David Park. ``Introduction.'' In \emph{The
International History of Communication Studies}, edited by Peter
Simonson and David Park, 1--3. New York: Routledge, 2016.

Steele, Godfrey. ``Human Communication Studies: What, Why and How from a
Caribbean Perspective.'' Inaugural lecture. University of the West
Indies at St. Augustine, Trinidad and Tobago, January 27, 2022. YouTube
video, 1:56:06. \url{https://www.youtube.com/watch?v=N0Tksimo5DY}.

Stone, Carl. ``Public Opinion Perspectives on the University of the West
Indies.'' \emph{Caribbean Quarterly} 2, no. 3--4 (1983): 21--39.

Tunstall, Jeremy. \emph{The Media Are American.} New York: Columbia
University Press, 1977.

White, Livingston A. ``Charting the Course of Communication Studies in
the English-Speaking Caribbean: Disciplines, Developments and Future
Directions.'' \emph{The Journal of Human Communication Studies in the
Caribbean} 1, no.1 (2015): 7--17.

Witter, Michael. ``Caribbean Economic Thought: Advances, Retreat,
Current Challenges.'' \emph{Agrarian South: Journal of Political
Economy} 10, no. 3 (2021): 463--91.



\end{hangparas}


\end{document}