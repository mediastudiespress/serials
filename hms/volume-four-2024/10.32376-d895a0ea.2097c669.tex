% see the original template for more detail about bibliography, tables, etc: https://www.overleaf.com/latex/templates/handout-design-inspired-by-edward-tufte/dtsbhhkvghzz

\documentclass{tufte-handout}

%\geometry{showframe}% for debugging purposes -- displays the margins

\usepackage{amsmath}

\usepackage{hyperref}

\usepackage{fancyhdr}

\usepackage{hanging}

\usepackage{tabu}

\usepackage{longtable}[=v4.13]

\hypersetup{colorlinks=true,allcolors=[RGB]{97,15,11}}

\fancyfoot[L]{\emph{History of Media Studies}, vol. 4, 2024}


% Set up the images/graphics package
\usepackage{graphicx}
\setkeys{Gin}{width=\linewidth,totalheight=\textheight,keepaspectratio}
\graphicspath{{graphics/}}

\title[`Western Communication']{`Western Communication': Eurocentrism and Modernity: Marks of the Predominant Theories in the Field} % longtitle shouldn't be necessary

% The following package makes prettier tables.  We're all about the bling!
\usepackage{booktabs}

% The units package provides nice, non-stacked fractions and better spacing
% for units.
\usepackage{units}

% The fancyvrb package lets us customize the formatting of verbatim
% environments.  We use a slightly smaller font.
\usepackage{fancyvrb}
\fvset{fontsize=\normalsize}

% Small sections of multiple columns
\usepackage{multicol}

% Provides paragraphs of dummy text
\usepackage{lipsum}

% These commands are used to pretty-print LaTeX commands
\newcommand{\doccmd}[1]{\texttt{\textbackslash#1}}% command name -- adds backslash automatically
\newcommand{\docopt}[1]{\ensuremath{\langle}\textrm{\textit{#1}}\ensuremath{\rangle}}% optional command argument
\newcommand{\docarg}[1]{\textrm{\textit{#1}}}% (required) command argument
\newenvironment{docspec}{\begin{quote}\noindent}{\end{quote}}% command specification environment
\newcommand{\docenv}[1]{\textsf{#1}}% environment name
\newcommand{\docpkg}[1]{\texttt{#1}}% package name
\newcommand{\doccls}[1]{\texttt{#1}}% document class name
\newcommand{\docclsopt}[1]{\texttt{#1}}% document class option name


\begin{document}

\begin{titlepage}

\begin{fullwidth}
\noindent\Large\emph{History of Communication Studies across the Americas
} \hspace{18mm}\includegraphics[height=1cm]{logo3.png}\\
\noindent\hrulefill\\
\vspace*{1em}
\noindent{\Huge{`Western Communication': Eurocentrism\\\noindent and Modernity: Marks of the Predominant Theories in the Field\textsuperscript{1}\setcounter{footnote}{1}\par}}

\vspace*{1.5em}

\noindent\LARGE{Erick R. Torrico Villanueva} \href{https://orcid.org0000-0003-1237-9241}{\includegraphics[height=0.5cm]{orcid.png}}\par\marginnote{\emph{Erick R. Torrico Villanueva, ```Western Communication': Eurocentrism and Modernity: Marks of the Predominant Theories in the Field,'' \emph{History of Media Studies} 4 (2024), \href{https://doi.org/10.32376/d895a0ea.2097c669}{https://doi.org/ 10.32376/d895a0ea.2097c669}.} \vspace*{0.75em}}
\vspace*{0.5em}
\noindent{{\large\emph{Universidad Andina Simón Bolívar}, \href{mailto:etorrico@uasb.edu.bo}{etorrico@uasb.edu.bo}\par}} \marginnote{\href{https://creativecommons.org/licenses/by-nc/4.0/}{\includegraphics[height=0.5cm]{by-nc.png}}}

% \vspace*{0.75em} % second author

% \noindent{\LARGE{<<author 2 name>>}\par}
% \vspace*{0.5em}
% \noindent{{\large\emph{<<author 2 affiliation>>}, \href{mailto:<<author 2 email>>}{<<author 2 email>>}\par}}

% \vspace*{0.75em} % third author

% \noindent{\LARGE{<<author 3 name>>}\par}
% \vspace*{0.5em}
% \noindent{{\large\emph{<<author 3 affiliation>>}, \href{mailto:<<author 3 email>>}{<<author 3 email>>}\par}}

\end{fullwidth}

\vspace*{1em}




\noindent\emph{Translation by William Quinn}

\hypertarget{abstract}{%
\section{Abstract}\label{abstract}}

The\marginnote{\textsuperscript{1} This article is
  a translation of Erick R. Torrico Villanueva, ``La `comunicación
  occidental': Eurocentrismo y Modernidad: Marcas de las teorías
  predominantes en el campo,'' \emph{Journal de Comunicación Social} 3,
  no. 3 (2015). The original article carries a
  \href{https://jcomsoc.ucb.edu.bo/a/openaccess}{Creative Commons
  BY-NC-SA license}.} main communication theories in use have been developed primarily by
American and European authors, reflect the characteristics of the
industrialized societies of the North, and are framed within the
parameters of scientificity as established by modernity. They focus on
mass communication, its technological means and its effects. Thus, with
a framework constituted above all by positivist epistemology,
empirical-quantitative research strategies, and functionalist
sociological theory, communication, as a field of knowledge, has
structured its profile of scientificity to conform to modern procedural
requirements as well as the expansionist objectives of the
civilizational model in which it was born. Such theories have reached,
in practice, a ``universal'' scope, a ``canonical'' recognition and
level, and are reproduced in the most diverse latitudes in both training
processes and professional practice, as well as in mainstream discourse.
Faced with the predominance of this ``Western'' communication, Latin
American critical communicational thought is challenged to seek a new
understanding of the phenomenon of communication and its study in the
perspective of its de-Westernization. 

\vspace*{6em}

\noindent{\emph{History of Media Studies}, vol. 4, 2024}

\hypertarget{resumen}{%
\section{Resumen}\label{resumen}}

Las principales teorías de la Comunicación en uso han sido
fundamentalmente desarrolladas por autores estadounidenses y europeos,
responden a las características de las sociedades industrializadas del
Norte y se enmarcan en los parámetros de cientificidad establecidos por
la Modernidad. Están focalizadas en la comunicación masiva, sus medios
tecnológicos y sus efectos. De esa forma, con una armazón constituida
ante todo por la epistemología positivista, las estrategias
investigativas empírico-cuantitativas y la teoría sociológica
funcionalista, la Comunicación, en cuanto campo de saber, estructuró su
perfil de cientificidad a la medida de las exigencias procedimentales
modernas como también de los objetivos de expansión del modelo
civilizatorio en que vio la luz. Tales teorías han alcanzado, en la
práctica, un alcance ``universal'', un reconocimiento y un nivel
``canónicos'' y son reproducidas en las más diversas latitudes tanto en
los procesos de formación como en el ejercicio profesional, lo mismo que
en el sentido común. Ante el predominio de esa ``Comunicación
`occidental'\,'', el pensamiento comunicacional crítico latinoamericano
está desafiado a buscar un nuevo entendimiento del fenómeno de la
comunicación y de su estudio en la perspectiva de su
desoccidentalización.


\enlargethispage{2\baselineskip}




 \end{titlepage}

% \vspace*{2em} | to use if abstract spills over

\newthought{Like practically all} fields of knowledge, communication is dominated by
the assumptions, purposes, and scientificity criteria of the ``modern''
and ``Western'' world, i.e., those established within the framework of
racialized hierarchies and dualistic reasoning\footnote{Racialization
  involves the adoption of the idea of \emph{race} to differentiate
  biologically and culturally ``superior'' and ``inferior'' populations.
  In turn, \emph{dualistic reason} is what operates with such binary
  classifications and is also linked to the emergence of the so-called
  ``two cultures,'' i.e., the separation between the search for the true
  (science) and the good (philosophy).} that became the norm when Europe
became the economic and political center of the planet after controlling
the Atlantic at the end of the fifteenth century and conquering the
``New World,"\footnote{Also known as the ``West Indies,'' this
  geographical space found by Christopher Columbus's expedition in 1492
  was called America from 1507 onwards.} which it then proceeded to
colonize.

Although there is no explicit agreement among scholars in the field
regarding the existence of a canon in the field of communication
theory,\footnote{The controversy in this regard is raised in the
  introduction to the book by Elihu Katz et al., eds., \emph{Canonic
  Texts in Media Research: Are There Any? Should There Be? How About
  These?} (New York: Polity Press, 2008), 1--8, which rightly asks
  whether there are any canonical texts as much as whether there should
  be.} the idea is implicitly manifested through the prevalence of a set
of ideas and assertions from a handful of American and European authors
which are widely seen as canonical. The repeated reference, both in
texts for a general audience and in university programs on different
continents, to a few notions and theories, as well as to a limited
number of thinkers responsible for their elaboration, has ended up
sedimenting an accepted way of thinking about communication and of
characterizing its analysis, which, almost without discussion, is
considered to be of universal scope and validity.

The tangible result of this predominance is a Euro-American conception
of communication---basically understood as the transmission of mass
messages, through technology, to exert political, business, or religious
influence---with an outsized presence and use not only in academia but
also in the professional practices of the field and even in non-experts'
shared discourse. This ``dominant paradigm'' sees communication above
all as an instrumental resource, supporting the interests of power (of
the emitters and/or their financial backers), which is why it confers on
its research an immediately practical utility rather than a capacity to
generate autonomous social knowledge that can be scientifically and
socially relevant.

In addition, the identification of four ``initiators'' or ``founding
fathers'' of communication research and theorizing---Kurt Lewin
(Prussian, psychologist), Carl Hovland (American, psychologist), Harold
Lasswell (American, political scientist) and Paul Lazarsfeld (Austrian,
sociologist)---by the American writer Wilbur Schramm not only was and
continues to be taken as the word of authority, but also almost rules
out the possibility of recognizing any other pre--World War II origin
for these matters, or at least of assuming that there was some other
contemporary source.\footnote{Wilbur Schramm, ed., \emph{La ciencia de
  la comunicación humana} (Quito: CIESPAL, 1965).} In this sense, even
precedents of research and reflection on the press that were found in
the sociological works of seminal European thinkers such as Gabriel
Tarde, Karl Marx, Émile Durkheim, or Max Weber\footnote{See, for
  example, Blanca Muñoz, \emph{Cultura y comunicación: Introducción a
  las teorías contemporáneas} (Barcelona: Barcanova, 1989); Éric
  Maigret, \emph{Sociología de la comunicación y los medios} (Bogotá:
  Fondo de Cultura Económica, 2005); and Paulo Serra, \emph{Manual de
  teoria da comunicação} (Convilhã, Portugal: Universidade da Beira
  Interior, 2007).} were shunted aside, not to mention those that can be
found in Latin American thought from the nineteenth century
onwards.\textsuperscript{7}

Consequently,\marginnote{\textsuperscript{7}\setcounter{footnote}{7} Cf. Luis Ramiro Beltrán, ``Estado y perspectivas de la
  investigación en comunicación en América Latina,'' \emph{SIDCOM}, no.
  2 (1982); and José Marques de Melo, ``Difusão dos paradigmas da escola
  latino-americana de comunicação nas universidades brasileiras,''
  \emph{Comunicação e Sociedade}, no. 25 (1996).} as a field dedicated to the examination of processes of
meaningful (inter)relation, communication emerged in the West during the
first third of the twentieth century with the stamp of empirical,
measurable, and applicable knowledge, linked from its beginnings to the
political and economic interests of capitalism, since its first
developments took place within the liberal framework of research
initiatives undertaken by the government, business foundations, and
certain private corporations in the United States of America.\footnote{See
  Jefferson Pooley, ``The New History of Mass Communication Research,''
  in \emph{The History of Media and Communication Research: Contested
  Memories}, ed. David W. Park and Jefferson Pooley (New York: Peter
  Lang, 2008).}

A process of formulating a canon was thus generated, in the sense that
the field of communication found itself ensconced, albeit indirectly, in
a predominant mode of intellectual organization that conditioned both
the apprehension of the phenomena of interest and the constitution of
the main currents of theoretical production and dissemination on the
subject.

\hypertarget{the-wests-knowledge}{%
\section{The West's Knowledge}\label{the-wests-knowledge}}

The West, in addition to referring geographically to one of the cardinal
points, is a historical metaphor that, in terms of knowledge,
prioritizes the founding condition and the supposedly superior
capacities of imperial Europe and its North American extension in the
``New World,'' and is therefore also the ideological metaphor of
cultures (European and Europeanized) that define themselves as a
universal civilizing pattern marked by the ideals of individual freedom,
economic accumulation, and endless material progress. This model, apart
from being intrinsically connected to technology and its logic of
permanent obsolescence, is also directly linked to the rationalist and
empiricist concept of science that was perfected within the second
modernity from the seventeenth century onwards\footnote{Enrique Dussel,
  \emph{1492: El encubrimiento del Otro; Hacia el origen del ``Mito de
  la modernidad''} (La Paz, Bolivia: Biblioteca Indígena, 2008). For
  this author---and his criterion is shared here---the first modernity
  began in 1492 when it became possible, with the incorporation of
  America into universal geography, for history to be unified on a
  universal scale as well.} and that gave rise to what Boaventura de
Sousa Santos calls ``abysmal thinking.''\footnote{Boaventura de Sousa
  Santos, \emph{Para descolonizar Occidente: Más allá del pensamiento
  abismal} (Buenos Aires: CLACSO, 2010).}

This way of thinking granted the monopoly of true knowledge to positive
science, disqualified alternative ways of knowing represented by
philosophy or theology, and established a \emph{line of the abyss}
beyond which there are only ``beliefs, opinions, magic, idolatry,
intuitive or subjective understandings'' and never ``real
knowledge''\footnote{de Sousa Santos, 14.}; it also established two key
epistemological premises: the symmetry between present, past, and future
(which comes from Isaac Newton), and the body-soul dualism (which comes
from René Descartes).

De Sousa himself summarizes elsewhere the fundamental ideas that make up
this positivist approach:

\begin{quote}
. . . the distinction between subject and object and between nature and
society or culture; the reduction of the complexity of the world to
simple laws, susceptible of being formulated mathematically; a
conception of reality dominated by a deterministic mechanism and of
truth as a transparent representation of reality; a strict distinction
between scientific knowledge---considered the only rigorous and valid
kind---and other forms of knowledge, such as common sense or the
knowledge considered in the humanities; priority given to functional
causality, and hostility to the investigation of ``ultimate causes''
considered metaphysical and focused on the manipulation and
transformation of the reality studied by science.\footnote{José de Souza
  Silva, ``Desobediencia epistémica desde ABYA YALA (América Latina):
  Tiempos de descolonización y reconstrucción en el pensamiento social
  latinoamericano'' (paper presented at the Primer Congreso Internaciona
  Pensamiento Social Latinoamericano, Cuenca, Ecuador, June 2008),
  41--42.}
\end{quote}

\noindent To these features was added the belief in the self-constitution of
modern science as a \emph{zero point} of observation, i.e., as a
platform from which it is possible to observe the real without being the
object of observation---in other words, as a privileged, neutral, and
absolute standpoint that would therefore enable the observer to grasp
universal truths without distortion or bias.\footnote{Cf. Santiago
  Castro-Gómez and Ramón Grosfoguel, eds., \emph{El giro decolonial:
  Reflexiones para una diversidad epistémica más allá del capitalismo
  global} (Bogotá: IESCO-Pensar, 2007), in particular 79--85; and
  Santiago Castro-Gómez, \emph{La hybris del punto cero: Ciencia, raza e
  ilustración en la Nueva Granada (1750--1816)} (Bogotá: Pontificia
  Universidad Javeriana, 2010).}

Consequently, all the knowledge developed in and by the West adopted
these epistemological assumptions of modern science and was inscribed
within the limits of its perspective, i.e., of its linear
self-referential gaze centered on the profiles, developments, problems,
and teleology of societies with capitalist and industrial development,
which led them to treat other peoples and their ways of conceiving,
knowing, and interpreting the world with a subordinating and even
contemptuous air.

Even though communication was a latecomer to the field of scientific
knowledge,\footnote{This emergence took place in the late 1920s, with
  Lasswell's first analyses of war propaganda (cf. John Durham Peters
  and Peter Simonson, eds., \emph{Mass Communication and American Social
  Thought: Key Texts 1919--1968} {[}Lanham, MD: Rowman \& Littlefield,
  2004{]}, 47--50), but its relative consolidation began twenty years
  later with Lasswell himself and the other three ``initiators'' already
  mentioned.} it could not escape this context, and very soon the
authors who established this new field developed arguments in favor of
empiricism, objectivist evidence, and the instrumental usefulness of
knowledge. In a famous article published in 1949, Lasswell not only
corrected himself regarding his inaugural qualitative analysis of war
propaganda in 1927, but also defended the importance of quantitative
procedures to control the uncertainty of data.\footnote{Harold D.
  Lasswell, ``Why Be Quantitative?,'' in \emph{Language and Politics:
  Studies in Quantitative Semantics}, ed. Lasswell and Nathan Leites
  (Cambridge, MA: MIT Press, 1949).} And it was from that same decade
until the 1960s that the still current main line of communication
studies was deployed, oriented towards the verification and presumed
measurement of the effects produced by the mass dissemination of
messages, opting for statistical and even experimental methods to verify
said effects.\footnote{See Schramm, \emph{La ciencia de la comunicación
  human}; and Abraham Nosnik, \emph{El desarrollo de la comunicación
  social: Un enfoque metodológico} (Mexico City: Trillas, 1991).}

Thus, with a framework consisting primarily of positivist epistemology,
empirical-quantitative research strategies, and functionalist
sociological theory, communication structured its profile of
scientificity to meet modern procedural requirements as well as the
expansion objectives of the civilizational model in which it was born.
This is the origin of its ``Westernism,'' i.e., its adherence to the
nature, characteristics, and purposes of Western science but at the same
time to the purported supremacy of ``Western culture'' and its global
capitalist designs.\footnote{Cf. Walter Mignolo, \emph{Historias
  locales/diseños globales: Colonialidad, conocimientos subalternos y
  pensamiento fronterizo} (Madrid: Akal, 2003).} Moreover, this is also
the origin of its \emph{Europhony},\textsuperscript{18}
i.e., its\marginnote{\textsuperscript{18}\setcounter{footnote}{18} Cf. Ousmane Kane,
  \emph{África y la producción intelectual no eurófona: Introducción al
  conocimiento islámico al sur del Sáhara} (Madrid: Oozebap, 2011).} condition as a terrain of expression of the Western
epistemological order in one or more of the European languages that have
dominated scientific production, publication, and debate for centuries:
English, French, and German (in that order of hierarchy), followed far
behind by Spanish.

However, it should be added that, despite all this epistemological,
theoretical, and methodological subjection, communication has not yet
been fully admitted into the privileged circle of consecrated
disciplines due to, among other things, the unresolved discussion
regarding its object of study\footnote{For several decades, disagreement
  has persisted about this question, during which time the preferences
  of authors have shifted the identification of the object of
  communicational study from the fidelity of technical transmission, the
  peculiarities and habits of the audiences, and the manifest and latent
  meanings of the messages, to the competencies of the receivers,
  cultural mediations, and the alleged democratizing benefits of
  technologies.} or its theoretical shortcomings.\footnote{The latter
  issue is discussed in Luiz Martino, ed., \emph{Teorias da comunicação:
  Muitas ou poucas?} (São Paulo: Ateliê, 2007).} The elements of
relativization introduced since the mid-1980s by the postmodernist and
\emph{cultural studies} currents, although they contributed to opening
other fronts of analysis such as transdisciplinarization or
postdisciplinarity,\footnote{Cf. Immanuel Wallerstein, ``El
  eurocentrismo y sus avatares: los dilemas de las ciencias sociales''
  (paper presented at the Future of Sociology in East Asia, Seoul,
  November 1996); and Eduardo Restrepo et al., eds., \emph{Sin
  garantías: Trayectorias y problemáticas en estudios culturales}
  (Popayán, Colombia: Envión, 2010).} did not provide a concrete
solution to the problem of the secondary position of the field of
communication in the spectrum of established academic knowledge, or to
the question of its scientific status.

\hypertarget{locus-features-and-actors-of-western-communication}{%
\section{Locus, Features, and Actors of ``Western''
Communication}\label{locus-features-and-actors-of-western-communication}}

The basic ideas about communication that prevail at the international
level, as well as the contours attributed to the field of communication,
are above all the product of Euro-American elaborations, as has already
been pointed out, that obviously respond to the nature and needs of the
social orders within which they emerged.

As early as 1976, in a seminal article, Luis Ramiro Beltrán referred to
this fact, concluding that communication (communicology, for him) was
born of this same process:

\begin{quote}
Understandably and legitimately, the United States designed and
constructed, in philosophy, object and method, the type of social
sciences that corresponded to its particular structural circumstances
(cultural, economic and political). They were, eminently, sciences for
adjustment, oriented fundamentally to studying conformity with the
prevailing needs, goals, values and norms of the established social
order, in such a way as to help the ruling system achieve ``normalcy''
and avoid ``deviant'' behaviors.\footnote{Quoted in Miquel de Moragas,
  ed., \emph{Sociología de la comunicación de masas} (Barcelona: Gustavo
  Gili, 1982), 103.}
\end{quote}

\noindent And in 1978 Jesús Martín-Barbero argued that Latin Americans' dependence
in this field should not be seen only in the practice of repeating
imported theories but also ``in the very conception of science, of
scientific work and its function in society'' that prevailed in the
region; he also added that ``The `science' of communication is born
controlled and oriented toward perpetuating the `North American style of
democracy.'\,''\footnote{Jesús Martín-Barbero, \emph{Procesos de
  comunicación y matrices de cultura: Itinerario para salir de la razón
  dualista} (Mexico City: Gustavo Gili, 1987), 20, 21.}

Thus, the initial \emph{locus} of enunciation of communicational
knowledge was specifically marked by a geographical location, a
historical situation, an epistemological conception, a notion of
science, a methodological criterion, and a linguistic-cultural device,
as well as by historical-social interests and purposes aligned with the
Eurocentric civilizational designs.

Within this framework, thinkers and analysts of the capitalist North,
imbued with the facts and aspirations of modernity, granted prerogatives
of scientificity to mediatized communication and devoted most of their
explicatory efforts to it, leaving aside the basic fact of
meaning-bestowing human (inter)relation itself. These factors, in the
end, shaped the ``Western'' intellectual tradition that continues to
prevail as the main reference in the field.

\hypertarget{towards-a-brief-profile-of-the-western-viewpoint}{%
\section{Towards a Brief Profile of the ``Western''
Viewpoint}\label{towards-a-brief-profile-of-the-western-viewpoint}}

But what are the distinctive aspects of this predominant communicational
vision? In order to answer this question, below is a brief approximation
to the proposals of eleven theorists from Europe and the United States
of America whose texts were and continue to be widely used in university
training and research processes in Ibero-America.\textsuperscript{24} The
books and authors selected for this purpose are the following:
\begin{fullwidth}

\tabulinesep=1.9mm
{\begin{longtabu} to 1.55\textwidth { X[l] X[c] X[c] X[c]} 
\large{\emph{Title} & \emph{Author(s)} & \emph{Nationality} & \emph{Year}}\\
\endfirsthead
\large{\emph{Title} & \emph{Author(s)} & \emph{Nationality} & \emph{Year}}\\
\endhead
\emph{Theories of Mass Communication} & Melvin L. DeFleur & American &
1966 \\
\emph{Mass Communication Theory: An Introduction} & Denis McQuail &
English & 1983 \\
\emph{Teorie delle comunicazioni di massa} & Mauro Wolf & Italian &
1985 \\
\emph{Cultura y comunicación: Introducción a las teorías
contemporáneas} & Blanca Muñoz & Spanish & 1989 \\
\emph{La pensée communicationnelle} & Bernard Miège & French & 1995 \\
\emph{La science de la communication} & Judith Lazar & French &
1996 \\
\emph{Historia de las teorías de la comunicación} & Armand and Michèle
Mattelart & Belgian & 1997 \\
\emph{Teorías de la comunicación: Ámbitos, métodos y perspectivas} &
Miquel R. Alsina & Spanish & 2001 \\
\emph{Sociologie de la communication et des médias} & Éric Maigret &
French & 2003 \\
\emph{Teoría de la comunicación: La comunicación, la vida y la
sociedad} & Manuel Martín Serrano & Spanish & 2007
\end{longtabu}}



\hspace{.08in}\textsuperscript{* The original titles and the years of appearance of the
first editions are cited here.}

\end{fullwidth}

\vspace{1em}

A\marginnote{\textsuperscript{24}\setcounter{footnote}{24} The
  bibliographic and state-of-the-art accounts of communication theories
  in Latin America list each of these texts at least once among the
  theoretical reference materials considered to be fundamental. See
  Jesús Galindo et al., \emph{Cien libros hacia una comunicología
  posible} (Mexico City: Universidad Autónoma de la Ciudad de México,
  2005); Luiz Martino, ``Teorias da comunicação: O estado da arte no
  Universo de Língua Espanhola'' (paper presented at the XXIX Encuentro
  de los Núcleos de Investigación de la INTERCOM, Brasilia, August 29,
  2007); and Raúl Fuentes Navarro, ``Bibliografías, biblionomías,
  bibliometrías: Los libros fundamentales en el estudio de la
  comunicación,'' \emph{Comunicación y Sociedad}, no. 10 (2008).} look at the aspects of each of these books that relate to the notions
of scientific knowledge, science, theory, and communication; and at the
theoretical sources used by their authors, at the lines of specialized
thought they identify or else the type or types of communication they
favor, and at the geographical areas they consider relevant in the
origin of the ideas or theories may allow us to outline the defining
features of what we have come to call ``Western'' communication. See now
briefly each of the cases based on this initial evaluation scheme.

\hypertarget{book-by-book}{%
\section{Book by Book}\label{book-by-book}}

The title of Melvin DeFleur's book already gives two valuable clues for
what we are interested in examining here: he speaks of theories, in the
plural, and of mass communication, which quickly establishes the
orientation adopted by this author, who in this book summarizes and
explains a small group of psychological and sociological
theories\footnote{These are theories of individual differences, social
  categories, social relations, and cultural norms.} of communication
based on the use of mass media, all of them related to the idea of media
influence on individual behaviors. Thus, DeFleur finds only a reduced
range of scarcely elaborated theoretical approaches to the mass
communication phenomenon, which is why he demands a ``theoretical
integration'' (which, however, he does not consider very feasible),
while he also proposes the need for a logical and empirical
strengthening of the existing theories. In this respect, he says that
these theories are in fact ``pre-theories'' due to their degree of
simplicity, the vagueness of their formulations and their lack of both
systematicity and supporting evidence.\footnote{Melvin DeFleur,
  \emph{Teorías de la comunicación masiva} (Buenos Aires: Paidós, 1976),
  223--28.}

DeFleur distinguishes human communication from that of other non-human
organisms, to which he attributes a lack of conscious processes,
learning, and culture, while affirming that the key to human
communication systems lies in the achievement of ``isomorphism of
meanings'' among those who participate in the ``symbolic
interaction.''\footnote{DeFleur, 121, 137--38.} Although he speaks of
this ``communicative exchange'' between individuals, the descriptive
model that he introduces shows a one-way relationship that goes from a
``source'' to a ``destination'' mediated by a technical ``transmitter''
and a technical ``receiver.''\footnote{DeFleur, 140.}

Among the authors on which DeFleur bases his ideas are Paul Lazarsfeld,
Bernard Berelson, Joseph T. Klapper, Charles Osgood, Charles Wright,
Robert Merton, Carl Hovland, Elihu Katz, Harold Lasswell, and Wilbur
Schramm.

Denis McQuail's book assumes, in the title itself, that it is feasible
to speak in the singular of ``the theory of mass communication,''
although in its content (first contradiction) he provides a rather long
list of different theoretical elaborations.\footnote{In general terms,
  these refer to the relationships between mass communication, society,
  and social change, as well as to the objectives, functions, roles,
  contents, audiences, and effects of mass media.} His second
contradiction has to do with the fact that he doubts the existence of
the theory he claims to affirm because ``it tends to be imprecisely
formulated'' and has ``made little progress in the constitution of a
`science of mass communication,' in the sense of a set of firm theses
that can be used to improve the effectiveness of communication
media.''\footnote{Denis McQuail, \emph{Introducción a la teoría de la
  comunicación de masas} (Barcelona: Paidós, 1985), 267, 268.}

The other element that should be highlighted in McQuail's text is that
it focuses exclusively on the \emph{mass media} form of communication,
from which we can derive a practical equivalence between the phenomenon
itself (which the author does not define in any precise sense) and the
technological media in which this phenomenon manifests itself: press,
book, radio, television, cinema, and recorded music.\footnote{McQuail,
  17--37.}

The authors McQuail most frequently cites in his bibliographical
references are Bernard Berelson, Jay Blumler, George Gerbner, James
Halloran, Elihu Katz, Paul Lazarsfeld, Everett Rogers, Karl Rosengren,
Gaye Tuchman, Jeremy Tunstall, and Charles Wright, who all belong to a
similar ideological register, although he also mentions Theodor Adorno,
Raymond Williams, Louis Althusser, Antonio Gramsci, Armand Mattelart,
and Stuart Hall, occasionally used to account for what he calls
``alternative approaches to mass communication.''

In his book, Mauro Wolf offers a chronological overview of the currents
and theories that have marked the trajectory of mass communication
research, leading to what he describes as ``new trends'' of this
``dominant paradigm.''

As in the preceding case, while the original title speaks of theory in
the singular, the content of the text presents at least nine different
traditional theories and another three later ones,\footnote{The author
  details them: ``the hypodermic theory, the theory linked to
  empirical-experimental visions, the theory derived from empirical
  field research, the theory of the structural-} all
related to the use of mass media to communicate, and the author admits
not only that it is not always accurate to use the term theory to
designate\marginnote{functionalist approach,
  the critical theory of the media, the culturological theory,
  \emph{cultural studies}, the communicative theories.'' Mauro Wolf,
  \emph{La investigación de la comunicación de masas} (Barcelona:
  Paidós, 1987), 22. In the second part (155ff.), he adds the theories
  of long-term effects and those of the sociology of emitters.} the explanatory tables in use---since, he says, they are
sometimes far from being sets of coherent and verified propositions and
hypotheses---but also that the object to which they refer varies: it is
sometimes identified as the ``mass media'' and sometimes as ``mass
culture.''\footnote{Wolf, 22}

The main authors considered by Wolf are Theodor Adorno, Raymond Bauer,
Umberto Eco, Carl Hovland, Elihu Katz, Paul Lazarsfeld, Denis McQuail,
Claude Shannon, Gaye Tuchman, and Charles Wright.

For Blanca Muñoz it is possible to articulate a ``theory of
communication'' on the basis of concepts derived from theories that came
out of social philosophy, as well as the sociology of knowledge and
empirical sociology, but such a theory concerns an area that in her
opinion is delimited by the relationship between communication and mass
media and by an effective equivalence between ``communicative system''
and ``cultural system.''\footnote{Muñoz, \emph{Cultura y comunicación,}
  2, 1.}

This author assumes the validity of the Lasswellian model as a
``methodological perspective'' (idem) and recognizes ``two major lines
of research'' in the field of communication: ``the North American
paradigm,'' concerned with effects, and ``the European paradigm,''
centered on ideology.\footnote{Muñoz, 3.} Consequently, her focus is on
mass communication as a phenomenon of contemporary society and her
central proposition is that the two lines that nourish the ``theory of
mass communication'' converge, in the end, in the ``analysis of the
formation of symbolic processes in post-industrial
societies.''\footnote{Muñoz, 419}

Therefore, Muñoz finds theoretical unity possible and argues that,
consequently, communication theory becomes a kind of superior synthesis
of the descriptive (sociological) and interpretative (philosophical)
approaches to social reality and that it also serves as a ``bridge''
between them.

Theodor Adorno, Louis Althusser, Roland Barthes, Jean Baudrillard,
Daniel Bell, Walter Benjamin, Umberto Eco, Michel Foucault, Jürgen
Habermas, Max Horkheimer, Claude Lévy-Strauss, Herbert Marcuse, and Max
Weber are the authors most frequently mentioned in the book's
bibliographical references. The only Latin American who appears in this
bibliography is Eliseo Verón.

The multiplicity and disconnection of specialized scientific production
are probably the two main features that Bernard Miège looks at as he
examines the stages of development of ``communicational
thinking.''\footnote{``Even if it has reached a certain level of
  elaboration, which allows it from now on to understand the complexity
  of the phenomena it is trying to explain, this thought is not unified,
  and is not ready to present itself as such.'' Wolf, \emph{La
  investigación de la comunicación de masas}, 114.} This is clearly
reflected in the very organization of his book, in which it is very
difficult to find a consistent organizing axis.\textsuperscript{38} Thus, when he refers to the fact that ``there are three
founding currents,'' it must be deduced that these are ``the cybernetic
model,'' ``the empirical-functionalist approach to the media,'' and
``the structural method\marginnote{\textsuperscript{38}\setcounter{footnote}{38} The book is
  structured in three parts: ``The Founding Currents (1950s and
  1960s),'' ``The Expansion of the Issues (1970s and 1980s),'' and
  ``Current Questions.'' It is clear that this delimitation does not
  follow a substantial homogeneous criterion. Bernard Miège, \emph{El
  pensamiento comunicacional} (Mexico City: Universidad Iberoamericana,
  1996).} and its linguistic applications,'' even though
he then proceeds, without making any distinction as to the foundational
or derivative character he attributes to them, to expound on his
appreciations of the ``sociology of mass culture,'' ``critical
thinking,'' ``psychology,'' and ``McLuhanian thought.''\footnote{Miège,
  13--43.}

Apart from this, however, what is interesting is the
author\textquotesingle s use of the notion of ``communicational
thinking'' to refer to the set of existing ideas and theories on
communication, which, while understanding it as the fruit of modernity,
he defines as a ``requisite for facilitating the modernization of social
structures.''\footnote{Miège, 10.}

Miège states that this thinking ``participates at the same time in
speculative reflection and scientific production,'' has diverse
disciplinary origins, and is concerned with observing a variety of
phenomena ranging from state policies to the practices of social actors;
he also notes that it assumes a more practical orientation in the United
States of America and initially a more critical one in Western
Europe.\footnote{Miège, 9.} Finally, he points out that as a product of
theoreticians, practitioners, and social activity itself, this thinking
has partially become an ideology in the sense that it produces and
circulates myths in contemporary society.\footnote{Miège, 115--16.}

Miége\textquotesingle s bibliography includes Theodor Adorno, Max
Horkheimer, Bernard Berelson, Regis Debray, Patrice Flichy, Jürgen
Habermas, Elihu Katz, Harold Lasswell, Paul Lazarsfeld, Claude
Lévy-Strauss, Marshall McLuhan, Armand Mattelart, Edgar Morin, Herbert
Schiller, Wilbur Schramm, Lucien Sfez, Claude Shannon, Warren Weaver,
Paul Watzlawick, and Raymond Williams.

Judith Lazar, for her part, is categorical in affirming the existence of
a ``science of communication,'' the nature of which would be based on
the fact that communication ``has become a well-established, rigorous
discipline with university departments, doctoral programs, research
methods, publications and scientific organizations.''\footnote{Judith
  Lazar, \emph{La ciencia de la comunicación} (Mexico City:
  Publicaciones Cruz O., 1995), 6.} To this she adds that this
``science'' encompasses the individual, interpersonal, intergroup,
organizational, and mass levels, the last ranking highest because it is
related ``to the totality of social life.''\footnote{Lazar, 7.}

As for its object of study, citing Steven Chafee and Charles Berger, she
says that it would be ``the production, processing and effects of
symbols and sign systems,'' which can be observed using quantitative and
qualitative procedures and which ties in with the ``universe of social
science research.''\footnote{Lazar, 6--7.}

In terms of what interests us here, the book offers in its first two
chapters a brief overview of the history of communication research
starting with the ``Chicago School'' and the ``founding fathers,'' then\newpage\noindent
describes the ``diverse orientations"\footnote{Among these
  ``orientations,'' she cites, for example, without resorting to any
  criterion of order: political economy, cultural imperialism, the use
  of media, the diffusion of innovations, dependency, technological
  determinism, agenda setting, and the spiral of silence (26--32).} and
presents four ``general approaches": ``cybernetics,'' ``anthropology,''
``psychology,'' and ``semiology and structuralism.''\footnote{Lazar,
  33--46.}

The main authors referred to in her bibliography are Carl Hovland, Elihu
Katz, Harold Lasswell, Paul Lazarsfeld, Kurt Lewin, Marshall McLuhan,
Robert Merton, Wilbur Schramm, Roland Barthes, Gregory Bateson, Ludwig
Bertalanffy, Umberto Eco, and Erving Goffman.

In their survey of the history of communication theories, Armand and
Michèle Mattelart assume from the outset a polysemic vision of this
notion, and therefore a multiplicity of issues and of analytical
approaches to the study of the phenomenon of communication.

The overview they present follows the ``order of appearance'' of the
``schools, currents or trends'' that have dealt with issues related to
communication---from the development of communication systems (means of
transportation and channels of communication) to the economic,
political, and subjective implications of mass media processes and
technologized networks---thus providing a complex panorama of the
complementary or opposing ideas generated in different fields and from
the perspective of different specializations.\footnote{Armand Mattelart
  and Michèle Mattelart, \emph{Historia de las teorías de la
  comunicación} (Barcelona: Paidós, 1997), 14. These ideas are organized
  in the book starting with those related to the ``social organism'' as
  a network of information exchange and then moving on to those
  contributed by ``New World empiricism,'' reviewing ``information
  theory,'' ``critical theory,'' ``\emph{Cultural Studies,''}
  ``political economy'' with its variants and studies on the actor and
  reception, and ending with the currents that examine ``postmodernity''
  in the world of networks.} Hence, the Mattelarts speak of ``the
plurality and fragmentation of this field of scientific observation''
and emphasize the impossibility of positing a linear and chronological
history of communicational theories.\footnote{Mattelart and Mattelart,
  9.} At the same time, they note that there has been a generalization
of ``administrative research'' that ``goes hand in hand with the
liberalization of the mode of communication,'' which has impregnated
with pragmatism ``even the ways of saying communication'' and feeds an
instrumentalism that hinders the attainment of ``true legitimacy'' for
the communicational field.\footnote{Mattelart and Mattelart, 126.}

The bibliography used by the Mattelarts is one of the most extensive; in
addition to classic Western authors such as Louis Althusser, Bernard
Berelson, Umberto Eco, Theodor Adorno, Jürgen Habermas, Stuart Hall,
Elihu Katz, Harold Lasswell, Paul Lazarsfeld, Marshall McLuhan, Herbert
Schiller, Wilbur Schramm, and Raymond Williams, it includes references
to important Latin American thinkers: Luis Ramiro Beltrán, Juan Díaz
Bordenave, Oswaldo Capriles, Paulo Freire, Néstor García Canclini,
Humberto Maturana, Francisco Varela, Antonio Pasquali, Héctor Schmucler,
Jesús Martín-Barbero, and Eliseo Verón.

Miquel Rodrigo Alsina, who accepts the existence of ``communication
sciences,'' maintains that ``the object of study of communication
theories is human communication in its manifestations in everyday
life,'' although he indicates that ``mass communication'' constitutes
``a preferential object of study'' for these sciences.\footnote{Miquel
  Rodrigo Alsina, \emph{Teorías de la comunicación: Ámbitos, métodos y}
  perspectivas (Barcelona: Aldea Global, 2001), 14, 44.} He then
identifies three ``perspectives'' of analysis---the ``interpretative,''
the ``functionalist,'' and the ``critical''---which, in turn, imply
``currents,'' i.e., the ``sources'' that ``feed theories of
communication with theoretical content.''\footnote{Rodrigo Alsina, 161.
  This author says that a ``perspective'' implies ``a similar approach
  to a similar object of study and a similar conception of communication
  within society'' (163). The ``currents'' he identifies in the
  ``interpretative perspective'' are the ``Palo Alto School,''
  ``symbolic interactionism,'' ``Erving Goffman,'' ``constructionism,''
  and ``ethnomethodology"; in the ``functionalist perspective,'' on the
  other hand, he gives a brief historical review of functionalism and
  presents its principles, describes the functions and dysfunctions of
  mass communication, and summarizes the criticisms of functionalism;
  finally, in the ``critical perspective,'' he returns to the
  ``currents": ``the Frankfurt school,'' ``political economy,'' and
  ``cultural studies.'' (163--207).}

In this case, among the authors with the most bibliographical references
are the Spaniards Luis Badía, Jordi Berrío, Manuel Castells, Josep
Gifreu, Daniel Jones, Miquel de Moragas, Manuel Parés i Maicas, Miquel
Rodrigo himself, Enric Saperas, Felicísimo Valbuena, and Manuel Vázquez
Montalbán, as well as Jay Blumler, Pierre Bourdieu, Umberto Eco, Anthony
Giddens, Erving Goffman, William Gudykunst, Michel Mafessoli, Edgar
Morin, Herbert Schiller, Wilbur Schramm, Paul Watzlawick, and Mauro
Wolf. Also on the list are the Latin Americans Néstor García Canclini,
José Carlos Lozano, Jesús Martín-Barbero, Humberto Maturana, Guillermo
Orozco, and Antonio Pasquali.

Éric Maigret focuses his sociological examination on the field of
``communication and media'' (up to the internet) and expresses that
``any theory of communication proposes a set of momentarily indivisible
elements: a model of the functional exchange between people, a point of
view regarding their relations of power and culture, a vision of the
political order that unites them''; it can be said that these three
elements give rise to the three levels of the communication phenomenon
as he understands them: ``natural or functional,'' ``social or
cultural,'' and ``creativity,'' which ``correspond to the levels of
people\textquotesingle s involvement in the universe of objects, of
interindividual relations and of sociopolitical orders.''\footnote{Éric
  Maigret, \emph{Sociología de la comunicación y los medios} (Bogotá:
  Fondo de Cultura Económica, 2005), 15--16.}

``Communication is primarily a cultural and political fact and not a
technical one'' and any theory consists of ``scientific presuppositions
and ideological, ethical and political points of view,'' he says,
assertions that undoubtedly translate into his proposal of ``applying
the gaze of the social sciences to the media,'' with which he conducts
an overview that goes from nineteenth-century European studies of the
press to the ``new technologies'' and ``electronic
democracy.''\footnote{Maigret, 17, 16, 49. The critical review provided
  by this author involves, among other theoretical approaches, studies
  of effects, the Frankfurt School, the theory of mass culture, the
  mathematical model of information, the anthropology of communication,
  technological determinism, semiology, \emph{Cultural Studies,} and
  theories of public opinion and public space (85--463).}


A singular element of Maigret\textquotesingle s book is the
incorporation of ``European founding fathers'' into the most widely
known theoretical tradition: Karl Marx, Émile Durkheim, Alexis de
Tocqueville, Georg Simmel, Ferdinand Tönnies, and Max Weber. However,
his criticism of them suggests why for the most part they did not
generate research traditions in the field:

\begin{quote}
On the question of the media, most of the European Founding Fathers were
not close-lipped but short-sighted. They looked closely---they kept
their distance from the common people with respect to the harmful
effects of the media or proposed practical study programs---but\end{quote}\newpage \begin{quote}they
misunderstood the place of communication in modernity, they
underestimated its social importance.\footnote{Maigret, 67.}
\end{quote}

\noindent The authors repeatedly cited in this work are Roland Barthes, Pierre
Bourdieu, Michel de Certeau, John Dewey, Émile Durkheim, Umberto Eco,
Anthony Giddens, Erving Goffman, Jürgen Habermas, Stuart Hall, Richard
Hoggart, Max Horkheimer, Elihu Katz, Harold Lasswell, Paul Lazarsfeld,
Walter Lippmann, Sonia Livingstone, Karl Marx, Armand Mattelart,
Marshall McLuhan, David Morley, Érik Neveu, Dominique Pasquier,
Jean-Claude Passeron, Ferdinand de Saussure, Philip Schlesinger, Roger
Silverstone, Georg Simmel, Gabriel Tarde, Alexis de Tocqueville, Alain
Touraine, Gaye Tuchman, Jeremy Tunstall, Max Weber, Norbert Wiener, and
Dominique Wolton. The only Latin Americans included in the references
are Jesús Martín-Barbero and Eliseo Verón.

Finally, Manuel Martín Serrano proposes his own original explanation of
the development and nature of communication, as well as the construction
of theory about it and the place of this theorization among current
knowledge.

This author questions the ``communicative anthropocentrism'' that, in
his opinion, confines communication theories to a ``pre-scientific''
state, and defends the need to develop theory from an evolutionary
conception and not just a cultural one.\footnote{Manuel Martín Serrano,
  \emph{Teoría de la comunicación: La comunicación, la vida y la
  sociedad} (Madrid: McGraw Hill, 2007), xiv, xxi, xx.} He views
communication as ``a type of interaction that is initially at the
service of biological needs and that follows zoological patterns,'' and
which, therefore, ``becomes a support for culture, but does not start
with it.''\footnote{Serrano, xviii.} This means that before ``human
communication'' there were ``pre-communicative uses of information'' in
other living beings---which is why, he emphasizes, it is necessary to
understand the general processes of evolution in order to know and
theorize communication.\footnote{Serrano, 23.}

Consequently, he considers that there is a notable gap between current
theorizing about communication and scientific knowledge in other areas
and states that ``the study of communication has to start when there was
neither culture nor society, nor values.\footnote{This has to do with
  what Martín Serrano calls the ``paleontology of communication''
  (51--66).} And it only concludes when it becomes clear how
communication has participated in the characteristics of humans, of
their societies: in the existence of an abstract and axiological
universe.''\footnote{Serrano, xviii, xix.}

Since this is not an account of existing theories, the only authors
traditionally linked to the field of communication mentioned in the
bibliography are Theodor Adorno, Roland Barthes, Émile Durkheim, Ludwig
von Bertalanffy, Max Horkheimer, Karl Marx, Abraham Moles, Claude
Shannon, Paul Watzlawick, and Norbert Wiener, to which are added others
coming from philosophy or language studies, epistemology, zoology,
biological evolution, or culture in general, such as Ernst Cassirer,
Charles Darwin, Friedrich Engels, Georg Hegel, Edmund Husserl, André
Leroi-Gourhan, Konrad Lorenz, Jean Piaget, Jean J. Rousseau, Ian
Tattersall, Nikolaas Tinbergen, and Ludwig Wittgenstein.

\hypertarget{elements-of-western-ideas-of-communication-and-communication}{%
\section{Elements of ``Western'' Ideas of Communication}\label{elements-of-western-ideas-of-communication-and-communication}}

From the brief review made in the previous section, it is possible to
tease out some noteworthy common elements among the books examined:

The primary focus of the theories in use is on ``mass communication.``

\begin{itemize}
\item
  Communication is conceived first and foremost as a
  \emph{mass-mediated} process and, therefore, as a process of
  transmitting content to specific audiences.
\item
  This communication is related to the power of the media or power over
  the media, thus emphasizing its instrumental aspect or use.
\item
  It is recognized that the study of communication has a plural origin
  in a number of disciplinary fields.
\item
  Theories identify repeatedly two basic blocs: pragmatic and critical.
\item
  The few references made to the necessary scientificity refer to
  positivist guidelines: provision of evidence, quantification, and even
  assimilation to the natural sciences.
\item
  The main source authors used, all of them American and European, are
  recurrent, while Latin American authors, cited at best exceptionally,
  are not given the same prominence.
\item
  The epistemological concern for the definition of theory, object, and
  method is circumstantial, low-priority, and lacking in consistency.
\item
  Except in the case of Martín Serrano, there is no evident ontological
  concern about communication as a phenomenon.
\item
  There are also no explicit, rigorous criteria for classifying the
  theories presented and their corresponding components.
\item
  And it is generally assumed that existing theories, apart from being
  weak and even unsystematic, are distinguished by their multiplicity,
  disconnection, and fragmentation.
\end{itemize}

As for the differences between the visions of the eleven authors whose
books have been presented so far, the following can be pointed out:

\begin{itemize}
\item
  The mention of several theories prevails over the eventual allusion to
  the possibility of developing a unified theory.
\item
  Some distinguish ``human communication'' from ``animal
  communication.'' Most converge on ``mass communication."
\item
  Some speak of a ``science of communication,'' some of ``communication
  sciences'' and others claim that communication cannot be a science.
\item
  The foundations for the study of communication are found in different
  sources: in communication itself, in the social sciences, in the
  humanities, in the cognitive sciences, and even in the natural
  sciences.
\item
  The greatest disagreement seems to be in the naming and classification
  of the theoretical frameworks that---without any visible conceptual
  analysis---are called theories, currents, perspectives, paradigms,
  lines, or schools, even within the proposal of a single author.
\end{itemize}

From all these elements it is possible to infer that in the ``Western''
conception, communication is a technologically mediated process that
generates effects and in which the unilateral action of the transmitters
has preeminence over the receivers, even as receivers are recognized as
having competences of re-signification; at the same time, communication
is seen as an area of knowledge that lacks a defined scientific status,
deals with multiple objects and shows a theoretical weakness that might
nonetheless be manageable when specific research aims at influencing
certain practices.

As was bound to happen, this ``Western-centric''\footnote{``Westcentric''
  in the original version of Shelton Gunaratne, ``De-Westernizing
  Communication/Social Science Research: Opportunities and
  Limitations,'' \emph{Media, Culture \& Society} 32, no. 3 (2010): 475.}
view of the communicational phenomenon and its study is inscribed in the
epistemological space of modernity, and both its scope (what it allows
us to think) and its conditioning factors (the way it suggests what is
thinkable) are those installed in the general social theories or
theoretical matrices that serve as its frame of reference.\footnote{On
  the characteristics of these matrices, see Erick Torrico,
  \emph{Comunicación: De las matrices a los enfoques} (Quito: CIESPAL,
  2010), 25--59.} And this is precisely where its link with coloniality
comes from,\footnote{This concept, initially developed by Aníbal
  Quijano, refers to the hierarchies of power inherited from the
  colonial era, which after independence have been traditionally
  reproduced by the institutional devices and control structures of
  non-imperial nations and internalized by their subalternized
  populations. Cf. Eduardo Restrepo and Axel Rojas, \emph{Inflexión
  decolonial: Fuentes, conceptos y cuestionamientos} (Popayán, Colombia:
  Universidad del Cauca, 2010), 15ff.} since this view, both in its
conservative wing and in the one that claims to be progressive,
effectively imposes and legitimizes the Eurocentric civilizing model
whose age-old universal pretension prevents it from realizing that, in
reality, it is nothing more than a provincialism projected from the
colonial core of the fifteenth century which persists in denying and
disqualifying worldviews and histories that it sees as ``other.''

Consequently, due to the frequency with which Latin American
academia---as well as Asian and African academia---has been imitating or
uncritically repeating certain imported conceptions for decades, as well
as adopting certain intellectual fashions, ``Western'' communication was
successfully implanted in quasi-canonical terms in the region and beyond
without any significant resistance, with the main effect of
marginalizing or simply inhibiting the regions' own reflections.

\hypertarget{de-westernizing-communication}{%
\section{De-Westernizing
Communication?}\label{de-westernizing-communication}}

Notwithstanding this long ``Western'' predominance and the custom of
epistemic, theoretical, and methodological ``borrowing,'' Latin America
generated from the 1960s onwards a critical-utopian vision in
communication that, while not homogeneous, has been marking an
alternative analytical route and today faces the challenge of
de-Westernization and, consequently, of its own emancipation.

This option is new given that for the first time, within the framework
of the ``decolonial turn'' initiated at the end of the 1990s, discussion
has occurred about the possibility of reinterpreting world history and
thereby dismantling the Eurocentric logic with which this notion had
been constructed.\footnote{Castro-Gómez and Grosfoguel, \emph{El giro
  decolonial}. This academic dependence was critically portrayed in the
  excellent article cited above by Luis Ramiro Beltrán on the foreign
  character of the premises, objects, and methods present in Latin
  American communicational research, a text which for this very reason,
  since its publication in 1976, has never ceased to be relevant. See
  Restrepo and Rojas, 94--119.} This implies, at the same time, the
opportunity to overcome, via ``epistemic disobedience,'' the constraints
of the paradigms established by the West to guarantee for itself the
oligopolistic advantage of knowledge.\footnote{Souza Silva,
  ``Desobediencia epistémica.''}

It is worth remembering that the previous Latin American critique was
almost always derived from intellectual and political positions---from
historical materialism to postmodernism---born in the heart of modernity
and that at no time proposed going beyond the borders of that project
forged in unison with the reign of capital, resulting from the
incorporation of America into the planetary map.

This is why the challenge of the present is different and bigger, and
meeting it still calls for a great deal of preparation. In any case, it
is not a matter of throwing overboard all the knowledge already
accumulated or of chasing dreamy autochthonisms, but rather of
critically reevaluating what is already known and channeling another
understanding of communication---more human, social, communitarian,
inclusive, humanizing, and democratizing than that of the ``dominant
paradigm''---as well as instituting a space of quali-quantitative
knowledge around a theoretical core that gives priority to consensus as
its purpose and to interrelation over technical mediations.

De-Westernizing, therefore, means ceasing to see communication and its
field through the eyes of technocracy, the market, blind faith, and
political control, and in so doing, to recover the liberating content of
its meaning and praxis.




\section{Bibliography}\label{bibliography}

\begin{hangparas}{.25in}{1} 


Bibliography}{ Bibliography}}\label{bibliography}}

Beltrán, Luis Ramiro. ``Estado y perspectivas de la investigación en
comunicación en América Latina.'' \emph{SIDCOM}, no. 2 (1982): 41--49.

Castro-Gómez, Santiago. \emph{La hybris del punto cero: Ciencia, raza e
ilustración en la Nueva Granada (1750--1816).} Bogotá: Pontificia
Universidad Javeriana, 2010.

Castro-Gómez, Santiago, and Ramón Grosfoguel, eds. \emph{El giro
decolonial: Reflexiones para una diversidad epistémica más allá del
capitalismo global.} Bogotá: IESCO-Pensar, 2007.

DeFleur, Melvin. \emph{Teorías de la comunicación masiva}. Buenos Aires:
Paidós, 1976.

Dussel, Enrique. \emph{1492: El encubrimiento del Otro; Hacia el origen
del ``Mito de la modernidad.''} La Paz, Bolivia: Biblioteca Indígena,
2008.

Fuentes Navarro, Raúl. ``El campo académico de la comunicación: 25 años
de fermentación.'' \emph{Anuario ININCO} 21, no. 1 (2009): 25--42.

Fuentes Navarro, Raúl. ``Bibliografías, biblionomías, bibliometrías: Los
libros fundamentales en el estudio de la comunicación.''
\emph{Comunicación y Sociedad}, no. 10 (2008): 15--53.

Galindo, Jesús, et al. \emph{Cien libros hacia una comunicología
posible}. Mexico City: Universidad Autónoma de la Ciudad de México,
2005.

Gunaratne, Shelton. ``De-Westernizing Communication/Social Science
Research: Opportunities and Limitations.'' \emph{Media, Culture \&
Society} 32, no. 3 (2010): 473--500.

Kane, Ousmane. \emph{África y la producción intelectual no eurófona:
Introducción al conocimiento islámico al sur del Sáhara}. Madrid:
Oozebap, 2011.

Katz, Elihu, John Durham Peters, Tamar Liebes, and Avril Orloff, eds.
\emph{Canonic Texts in Media Research: Are There Any? Should There Be?
How About These?} New York: Polity Press, 2008.

Lasswell, Harold D. ``Why Be Quantitative?'' In \emph{Language and
Politics: Studies in Quantitative Semantics}, edited by Harold D.
Lasswell and Nathan Leites, 40--52. Cambridge, MA: MIT Press, 1949.

Lazar, Judith. \emph{La ciencia de la comunicación.} Mexico City:
Publicaciones Cruz O., 1995.

Maigret, Éric. \emph{Sociología de la comunicación y los medios}.
Bogotá: Fondo de Cultura Económica, 2005.

Marques de Melo, José. ``Difusão dos paradigmas da escola
latino-americana de comunicação nas universidades brasileiras.''
\emph{Comunicação e Sociedade}, no. 25 (1996): 9--20.

Martín-Barbero, Jesús. \emph{Procesos de comunicación y matrices de
cultura: Itinerario para salir de la razón dualista}. Mexico City:
Gustavo Gili, 1987.

Martín Serrano, Manuel. \emph{Teoría de la comunicación: La
comunicación, la vida y la sociedad.} Madrid: McGraw Hill, 2007.

Martino, Luiz, ed. \emph{Teorias da comunicação: Muitas ou poucas?} São
Paulo: Ateliê, 2007.

Martino, Luiz. ``Teorias da comunicação: O estado da arte no Universo de
Língua Espanhola.'' Paper presented at the XXIX Encuentro de los Núcleos
de Investigación de la INTERCOM, Brasilia, August 29, 2007.

Mattelart, Armand, and Michèle Mattelart. \emph{Historia de las teorías
de la comunicación}. Barcelona: Paidós, 1997.

McQuail, Denis. \emph{Introducción a la teoría de la comunicación de
masas.} Barcelona: Paidós, 1985.

Miège, Bernard. \emph{El pensamiento comunicacional}. Mexico City:
Universidad Iberoamericana, 1996.

Mignolo, Walter. \emph{Historias locales/diseños globales: Colonialidad,
conocimientos subalternos y pensamiento fronterizo.} Madrid: Akal, 2003.

Moragas, Miquel de, ed. \emph{Sociología de la comunicación de masas}.
Barcelona: Gustavo Gili, 1982.

Muñoz, Blanca. \emph{Cultura y comunicación: Introducción a las teorías
contemporáneas}. Barcelona: Barcanova, 1989.

Nosnik, Abraham. \emph{El desarrollo de la comunicación social: Un
enfoque metodológico.} Mexico City: Trillas, 1991.

Peters, John Durham, and Peter Simonson, eds. \emph{Mass Communication
and American Social Thought: Key Texts 1919--1968.} Lanham, MD: Rowman
\& Littlefield, 2004.

Pooley, Jefferson. ``The New History of Mass Communication Research.''
In \emph{The History of Media and Communication Research: Contested
Memories}, edited by David W. Park and Jefferson Pooley, 43--69. New
York: Peter Lang, 2008.

Restrepo, Eduardo, et al., eds. \emph{Sin garantías: Trayectorias y
problemáticas en estudios culturales.} Popayán, Colombia: Envión, 2010.

Restrepo, Eduardo, and Axel Rojas. \emph{Inflexión decolonial: Fuentes,
conceptos y cuestionamientos.} Popayán, Colombia: Universidad del Cauca,
2010.

Rodrigo Alsina, Miquel. \emph{Teorías de la comunicación: Ámbitos,
métodos y perspectivas.} Barcelona: Aldea Global, 2001.

Schramm, Wilbur, ed. \emph{La ciencia de la comunicación humana}. Quito:
CIESPAL, 1965.

Serra, Paulo. \emph{Manual de teoria da comunicação}. Convilhã:
Universidade da Beira Interior, 2007.

Sousa Santos, Boaventura de. \emph{Para descolonizar Occidente: Más allá
del pensamiento abismal.} Buenos Aires: CLACSO, 2010.

Souza Silva, José de. ``Desobediencia epistémica desde ABYA YALA
(América Latina): Tiempos de descolonización y reconstrucción en el
pensamiento social latinoamericano.'' Paper presented at the Primer
Congreso Internaciona Pensamiento Social Latinoamericano, Cuenca,
Ecuador, June 2008.

Torrico Villanueva, Erick R. \emph{Comunicación: De las matrices a los
enfoques.} Quito: CIESPAL, 2010.

Torrico Villanueva, Erick R. ``La `comunicación occidental':
Eurocentrismo y Modernidad: Marcas de las teorías predominantes en el
campo.'' \emph{Journal de Comunicación Social} 3, no. 3 (2015): 41--64.
\url{https://doi.org/10.35319/jcomsoc.201531062}.

Wallerstein, Immanuel. \emph{Impensar las ciencias sociales: Límites de
los paradigmas decimonónicos.} Mexico City: Siglo XXI, 1998.

Wallerstein, Immanuel. ``El eurocentrismo y sus avatares: los dilemas de
las ciencias sociales.'' Paper presented at the Future of Sociology in
East Asia, Seoul, November 1996.

Wolf, Mauro. \emph{La investigación de la comunicación de masas}.
Barcelona: Paidós, 1987.



\end{hangparas}


\end{document}