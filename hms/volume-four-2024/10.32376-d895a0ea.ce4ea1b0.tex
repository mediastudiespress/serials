% see the original template for more detail about bibliography, tables, etc: https://www.overleaf.com/latex/templates/handout-design-inspired-by-edward-tufte/dtsbhhkvghzz

\documentclass{tufte-handout}

%\geometry{showframe}% for debugging purposes -- displays the margins

\usepackage{amsmath}

\usepackage{hyperref}

\usepackage{fancyhdr}

\usepackage{hanging}

\hypersetup{colorlinks=true,allcolors=[RGB]{97,15,11}}

\fancyfoot[L]{\emph{History of Media Studies}, vol. 4, 2024}


% Set up the images/graphics package
\usepackage{graphicx}
\setkeys{Gin}{width=\linewidth,totalheight=\textheight,keepaspectratio}
\graphicspath{{graphics/}}

\title[Thinking with Sound]{\emph{Thinking with Sound: A New Program in the Sciences and Humanities around 1900}} % longtitle shouldn't be necessary

% The following package makes prettier tables.  We're all about the bling!
\usepackage{booktabs}

% The units package provides nice, non-stacked fractions and better spacing
% for units.
\usepackage{units}

% The fancyvrb package lets us customize the formatting of verbatim
% environments.  We use a slightly smaller font.
\usepackage{fancyvrb}
\fvset{fontsize=\normalsize}

% Small sections of multiple columns
\usepackage{multicol}

% Provides paragraphs of dummy text
\usepackage{lipsum}

% These commands are used to pretty-print LaTeX commands
\newcommand{\doccmd}[1]{\texttt{\textbackslash#1}}% command name -- adds backslash automatically
\newcommand{\docopt}[1]{\ensuremath{\langle}\textrm{\textit{#1}}\ensuremath{\rangle}}% optional command argument
\newcommand{\docarg}[1]{\textrm{\textit{#1}}}% (required) command argument
\newenvironment{docspec}{\begin{quote}\noindent}{\end{quote}}% command specification environment
\newcommand{\docenv}[1]{\textsf{#1}}% environment name
\newcommand{\docpkg}[1]{\texttt{#1}}% package name
\newcommand{\doccls}[1]{\texttt{#1}}% document class name
\newcommand{\docclsopt}[1]{\texttt{#1}}% document class option name


\begin{document}

\begin{titlepage}

\begin{fullwidth}
\noindent\LARGE\emph{Book review
} \hspace{88mm}\includegraphics[height=1cm]{logo3.png}\\
\noindent\hrulefill\\
\vspace*{1em}
\noindent{\Huge{\emph{Thinking with Sound: A New Program in\\\noindent the Sciences and Humanities around 1900}\par}}

\vspace*{1.5em}

\noindent\LARGE{Brent Malin}\par\marginnote{\emph{Thinking with Sound: A New Program in the Sciences and Humanities around 1900}, reviewed by Brent Malin, \emph{History of Media Studies} 4 (2024), \href{https://doi.org/10.32376/d895a0ea.ce4ea1b0}{https://doi.org/ 10.32376/d895a0ea.ce4ea1b0}.} \vspace*{0.75em}
\vspace*{0.5em}
\noindent{{\large\emph{University of Pittsburgh}, \href{mailto:bmalin@pitt.edu}{bmalin@pitt.edu}\par}} \marginnote{\href{https://creativecommons.org/licenses/by-nc/4.0/}}

% \vspace*{0.75em} % second author

% \noindent{\LARGE{<<author 2 name>>}\par}
% \vspace*{0.5em}
% \noindent{{\large\emph{<<author 2 affiliation>>}, \href{mailto:<<author 2 email>>}{<<author 2 email>>}\par}}

% \vspace*{0.75em} % third author

% \noindent{\LARGE{<<author 3 name>>}\par}
% \vspace*{0.5em}
% \noindent{{\large\emph{<<author 3 affiliation>>}, \href{mailto:<<author 3 email>>}{<<author 3 email>>}\par}}

\end{fullwidth}

\vspace*{1em}


\noindent Viktoria\marginnote{\includegraphics[height=0.5cm]{by-nc.png}} Tkaczyk. \emph{Thinking with
Sound: A New Program in the Sciences and Humanities around
1900}. 304 pp., index. Chicago: University of Chicago
Press, 2023. \$55 (cloth).

\vspace{0.2in}

\newthought{Viktoria Tkaczyk's} \emph{Thinking with Sound: A New Program in the
Science and Humanities around 1900} is a fascinating, well-written
intellectual history of the many ways that various early twentieth
century European thinkers took up sound as an area of study. Tkaczyk
devotes substantial time to discussions of such well-known thinkers as
Sigmund Freud, Albert Einstein, Ferdinand de Saussure, Henri Bergson,
Hermann von Helmholtz, Ernst Mach, and Carl Stumpf, as well as some
lesser-known figures who also spent time thinking about and engaging
with sound in ways emblematic of the period. In covering this wide
collection of thinkers, Tkaczyk draws on an impressive array of archival
materials---published materials, but also laboratory notebooks, letters,
lecture notes, journals, and others---in a variety of languages and
across a range of locations, though primarily focused around Berlin,
Geneva, Paris, and Vienna. In drawing on this range of thinkers, Tkaczyk
offers a sense of an international network of thinkers in conversation
about issues raised by thinking with and about sound and a sense of how
the ``two cultures'' of science and the humanities engaged which each
other around this blossoming topic.

As Tkaczyk observes, the early twentieth century was an important period
of growth and tension around questions of science. Psychology, for
instance, had begun splitting away from philosophy---its 

\enlargethispage{2\baselineskip}

\vspace*{3em}

\noindent{\emph{History of Media Studies}, vol. 4, 2024}


 \end{titlepage}

% \vspace*{2em} | to use if abstract spills over



\noindent parent
discipline---towards the end of the nineteenth century. Working to
establish itself as a separate discipline, many of its practitioners
began to embrace the tools and perspectives of the natural sciences,
particularly in the establishment of new psychological laboratories.
Especially in the early twentieth century, this led to some interesting
and lively tensions between ``scientific'' and ``humanistic'' sorts of
approaches to psychological and philosophical phenomena. In the United
States, for instance, Carl Seashore, whose work Tkaczyk mentions as a
parallel and extension of the European approaches she explores, worked
to establish a laboratory and scientific approach for research on the
aesthetic questions that had been studied by philosophers. Investigating
the psychology of beauty, for instance, Seashore used developing
scientific instruments of measurement to study such question as what
makes for a good vibrato or a beautiful musical tone.\footnote{Brenton
  J. Malin, ``Not Just Your Average Beauty: Carl Seashore and the
  History of Communication Research in the United States,''
  \emph{Communication Theory} 21, no. 3 (2011); Carl E. Seashore, ``The
  Natural History of the Vibrato,'' \emph{Proceedings of the National
  Academy of Sciences} 17, no. 12 (1931); Seashore, ``A Voice
  Tonoscope,'' \emph{University of Iowa Studies in Psychology} 3 (1902);
  Seashore, ``The Measure of a Singer,'' \emph{Science} 35, no. 893
  (1912); Seashore, ``Phonophotography in the Measurement of the
  Expression of Emotion in Music and Speech,'' \emph{The Scientific
  Monthly} 24, no. 5 (1927).} Such an approach was closely aligned with
that of Carl Stumpf, in particular, whom Tkaczyk explains was working to
establish an early form of ``laboratory humanities.'' Even as
psychologists worked to separate themselves from philosophy, they
remained committed to many of the same sorts of questions and concerns.

For Tkaczysk, sound becomes a general thread uniting these kinds of
questions across geography, approach, and topic. As such, she finds
questions of sound in places where they might not always be immediately
apparent. This has the interesting benefit of reframing some well-known
thinkers via their notions of sound. Media and communication scholars,
to whom Tkaczysk's work is partially addressed, will be familiar with
Sigmund Freud for how his notions of psychoanalysis are taken up in a
range of approaches to film and media; the same is true of Ferdinand de
Saussure's notions of semiotics. In drilling down on Freud's and
Saussure's notions of sound, however, Tkaczysk excavates another way of
seeing these thinkers engaged with and around media questions. Arguing
that much of the contemporary understanding of Saussure's ideas has been
overdetermined by his later \emph{Course in General Linguistics} (itself
reconstructed from student notes), Tkaczysk chooses instead to focus on
his earlier writings, where we see him building on and departing from
existing work in linguistics. One of Saussure's chief interventions,
according to Tkaczysk, was pushing linguistics away from the study of
written language towards that of oral and spoken speech. Here, Saussure
engaged with and benefitted from technologies such as the phonograph as
he developed his own approach that, in the style of the time, worked to
cross ``humanistic'' and ``scientific'' questions. That Saussure pushed
for a notion of a living language adapted to and from the ``collective
mind of a linguistic community'' (75) is both an interesting way to see
Saussure engaged with sound and a very useful means of understanding his
later ideas of signification as laid out in the \emph{Course in General
Linguistics}.

Here and elsewhere, Tkaczysk provides additional biographical and
contextual information that helps frame these thinkers' discussions of
sound as well as their work more broadly. For instance, to explain the
ways that the ego responds to sounds in the environment, Freud suggested
that it was wearing a ``hearing cap,'' which, Tkaczyk explains, was a
device worn in the 1920s by telephone switchboard operators and pilots
in order to allow for hands-free communication (37--38). Likewise, in
discussing Freud's analysis of ``Anna O.,'' who had a psychosomatic
reaction to hearing rhythmic music played, Tkaczysk offers that the
Viennese waltz was popular at the time. As Tkaczyk recounts,
``rhythmically simple but fast-moving (a three-quarter measure at
approximately 60 beats per measure), the Viennese waltz required dancers
to stress the first beat and passionately anticipate the second. The
thrill of the waltz and the close contact of the dancers drew strong
moral objections'' (41) that likely influenced Anna O.'s response to it.
Similarly, we learn that philosopher Henri Bergson's father was a
pianist and composer, which presumably influenced his own discussion of
the brain as a piano (87). The physicist Ernst Mach, who studied how
sound waves functioned at various speeds, was himself a pianist who also
wrote about musical perception (111). In all of these ways, Tkaczyk does
a nice job of illustrating how various scholars came to focus on sound
and how a focus on sound came to shape other elements of their thinking
and theory construction.

As these examples illustrate, Tkaczyk does a nice job of bringing
together the history of science---including the history of the social
sciences---with the history of various scientific instruments and
techniques in ways similar to such turn-of-this-century sound studies
luminaries as Jonathan Sterne.\footnote{Jonathan Sterne, \emph{The
  Audible Past: Cultural Origins of Sound Reproduction} (Durham, NC:
  Duke University Press, 2003).} Alongside the ``hearing cap,'' Tkaczyk
discusses such technical devices as the ``laryngograph,'' a device that
measured changes in the larynx and was used to study reading and speech
production, and the ``acoustometer,'' ``a precursor to modern sound
level meters'' (126), developed by physicist Sigmund Exner to measure
sound reverberations in auditoriums. Tkaczyk also discusses the
importance of such developing technologies as telephones, radios,
microphones, speakers, and amplifiers, in the production and study of
speech. Although Tkaczyk sees Friedrich Kittler's claims about the role
of the phonograph in shaping the scientific and media culture of the
twentieth century as too deterministic, her discussion of this and these
other technologies nonetheless offers an important image of how such
technologies shape and are shaped by the inquiries to which they are
put. This is especially important in this early twentieth century moment
of ``applied research,'' when researchers were not only, like Exner,
developing their own technologies, but often marketing and selling them
to the broader public as well.

Without falling into a technologically deterministic perspective, it is
possible to see the complex ways that the technologies and techniques of
various approaches can follow from and shape particular cultural and
political values. When early twentieth century social scientists
attempted to separate themselves from philosophy, that often meant
employing more presumably ``objective'' means of investigation,
eschewing the sorts of personal or subjective approaches that
characterized much of philosophy. This could and did create the kinds of
hybrid spaces of laboratory humanities and humanistic science that
Tkaczyk discusses. But treating speech as an abstract, general topic,
divorced from some of its more human elements, could also create grave
ethical challenges, as illustrated by the so-called ``monster study''
conducted under the supervision of Wendell Johnson at Carl Seashore's
University of Iowa, in which graduate student Mary Tudor attempted to
make orphaned children stutter by criticizing their speech
practice.\footnote{Mary Tudor, ``An Experimental Study of the Effect of
  Evaluative Labeling on Speech Fluency'' (master\textquotesingle s
  thesis, University of Iowa, 1939).} Tkaczyk finds similar problems in
her own analysis. In one study conducted in Germany, a group of
musicologists, linguists, and anthropologists led by Carl Stumpf
recorded and analyzed ``the voices of prisoners of war interned in
German camps between 1915 and 1918.'' As Tkaczyk explains, ``many of the
resulting recordings served preservationist and colonialist interests,
tied to the objective of systemically investigating the world's
languages, dialects, and musics,'' through which this commission, ``most
of whom were Germans, claimed interpretive sovereignty over these
samples of speech from all over the world'' (189--90). This connected to
desires to create a standardized dialect of German and eventually tied
into notions of radio speech and served various aspects of Nazi
propaganda.

\emph{Thinking with Sound} is a useful and important book in media and
cultural history and science and technology studies. While it is
extensively researched, Tkaczyk's writing here leans much more heavily
on her compelling narrative than it does her impressive archival
resources. This makes for an eminently readable discussion that should
speak to a wide range of readers. Likewise, the ``sound'' of this book
plays an organizational and rhetorical role as much as it does a
specific area or topic of study. For Tkaczyk, it encapsulates Freud's
and Bergson's thinking about images in the mind or brain and Einstein's
conception of time in addition to more literal acoustic waves measured
by Stumpf or Exner. This is sound history as cultural and intellectual
history writ large. For all of these reasons, \emph{Thinking with Sound}
is a compelling story that should be of great interest to those both
within and beyond sound studies.




\section{Bibliography}\label{bibliography}

\begin{hangparas}{.25in}{1} 



~Malin, Brenton J. ``Not Just Your Average Beauty: Carl Seashore and the
History of Communication Research in the United States.''
\emph{Communication Theory} 21, no. 3 (2011): 299--316.

~Seashore, Carl E. ``The Measure of a Singer.'' \emph{Science} 35, no.
893 (1912): 201--12.

~Seashore, Carl E. ``The Natural History of the Vibrato.''
\emph{Proceedings of the National Academy of Sciences} 17, no. 12
(1931): 623--26.

~Seashore, Carl E. ``Phonophotography in the Measurement of the
Expression of Emotion in Music and Speech.'' \emph{The Scientific
Monthly} 24, no. 5 (1927): 463--71.

~Seashore, Carl E. ``A Voice Tonoscope.'' \emph{University of Iowa
Studies in Psychology} 3 (1902):~18--28.

Sterne, Jonathan. \emph{The Audible Past: Cultural Origins of Sound
Reproduction} (Durham, NC: Duke University Press, 2003).

Tudor, Mary. ``An Experimental Study of the Effect of Evaluative
Labeling on Speech Fluency.'' Master's thesis, University of Iowa, 1939.



\end{hangparas}


\end{document}