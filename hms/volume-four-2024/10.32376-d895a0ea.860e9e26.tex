% see the original template for more detail about bibliography, tables, etc: https://www.overleaf.com/latex/templates/handout-design-inspired-by-edward-tufte/dtsbhhkvghzz

\documentclass{tufte-handout}

%\geometry{showframe}% for debugging purposes -- displays the margins

\usepackage{amsmath}

\usepackage{hyperref}

\usepackage{fancyhdr}

\usepackage{hanging}

\hypersetup{colorlinks=true,allcolors=[RGB]{97,15,11}}

\fancyfoot[L]{\emph{History of Media Studies}, vol. 4, 2024}


% Set up the images/graphics package
\usepackage{graphicx}
\setkeys{Gin}{width=\linewidth,totalheight=\textheight,keepaspectratio}
\graphicspath{{graphics/}}

\title[Elizabeth Fox]{Elizabeth Fox: Intellectual Biography and History of a Field of Study} % longtitle shouldn't be necessary

% The following package makes prettier tables.  We're all about the bling!
\usepackage{booktabs}

% The units package provides nice, non-stacked fractions and better spacing
% for units.
\usepackage{units}

% The fancyvrb package lets us customize the formatting of verbatim
% environments.  We use a slightly smaller font.
\usepackage{fancyvrb}
\fvset{fontsize=\normalsize}

% Small sections of multiple columns
\usepackage{multicol}

% Provides paragraphs of dummy text
\usepackage{lipsum}

% These commands are used to pretty-print LaTeX commands
\newcommand{\doccmd}[1]{\texttt{\textbackslash#1}}% command name -- adds backslash automatically
\newcommand{\docopt}[1]{\ensuremath{\langle}\textrm{\textit{#1}}\ensuremath{\rangle}}% optional command argument
\newcommand{\docarg}[1]{\textrm{\textit{#1}}}% (required) command argument
\newenvironment{docspec}{\begin{quote}\noindent}{\end{quote}}% command specification environment
\newcommand{\docenv}[1]{\textsf{#1}}% environment name
\newcommand{\docpkg}[1]{\texttt{#1}}% package name
\newcommand{\doccls}[1]{\texttt{#1}}% document class name
\newcommand{\docclsopt}[1]{\texttt{#1}}% document class option name


\begin{document}

\begin{titlepage}

\begin{fullwidth}
\noindent\Large\emph{History of Communication Studies across the Americas
} \hspace{19mm}\includegraphics[height=1cm]{logo3.png}\\
\noindent\hrulefill\\
\vspace*{1em}
\noindent{\Huge{Elizabeth Fox: Intellectual Biography and\\\noindent History of a Field of Study\par}}

\vspace*{1.5em}

\noindent\LARGE{Yamila Heram} \href{https://orcid.org/0000-0002-9209-4571}{\includegraphics[height=0.5cm]{orcid.png}}\par\marginnote{\emph{Yamila Heram and Santiago Gándara, ``Elizabeth Fox: Intellectual Biography and History of a Field of Study,'' \emph{History of Media Studies} 4 (2024), \href{https://doi.org/10.32376/d895a0ea.860e9e26}{https://doi.org/ 10.32376/d895a0ea.860e9e26}.} \vspace*{0.75em}}
\vspace*{0.5em}
\noindent{{\large\emph{Universidad de Buenos Aires}, \href{mailto:yaheram@yahoo.com.ar}{yaheram@yahoo.com.ar}\par}} \marginnote{\href{https://creativecommons.org/licenses/by-nc/4.0/}{\includegraphics[height=0.5cm]{by-nc.png}}}

\vspace*{0.75em} 

\noindent{\LARGE{Santiago Gándara} \href{https://orcid.org/0000-0003-3350-7417}{\includegraphics[height=0.5cm]{orcid.png}}\par}
\vspace*{0.5em}
\noindent{{\large\emph{Universidad de Buenos Aires}, \href{mailto:sjgandar@gmail.com}{sjgandar@gmail.com}\par}}

% \vspace*{0.75em} % third author

% \noindent{\LARGE{<<author 3 name>>}\par}
\vspace*{0.5em}
% \noindent{{\large\emph{<<author 3 affiliation>>}, \href{mailto:<<author 3 email>>}{<<author 3 email>>}\par}}

\end{fullwidth}

\noindent\emph{Translated by William Quinn}

\vspace*{1em}

\hypertarget{abstract}{%
\section{Abstract}\label{abstract}}

The aim of this article is to highlight and recognize the contributions
of one of the pioneers in Latin American communication studies,
Elizabeth Fox, who since the late 1960s in Colombia has investigated
media ownership and inequalities in the flow of information in the
region. At the same time, the description of her intellectual journey
reveals a transnational figure who established relations not only
between different countries in the region but also with universities and
agencies in the United States and Europe. In order to examine her
trajectory, a meta-analysis of her academic publications serves to
identify changes, ruptures, and continuities in her research topics and
theoretical positions; an interview with the author was also conducted
to complement the reconstruction of her trajectory. The main results and
conclusions of the article are synthesized in a recovery of the critical
traditions of the political economy of communication in her works
published in the 1970s.





\enlargethispage{2\baselineskip}

\vspace*{10em}

\noindent{\emph{History of Media Studies}, vol. 4, 2024}


 \end{titlepage}

% \vspace*{2em} | to use if abstract spills over



\hypertarget{introduction}{%
\section{Introduction}\label{introduction}}

To advance our understanding of the contributions made by women pioneers
in communication studies in Latin America, it is necessary not only to
highlight and recognize the place these women have occupied, but also to
recover the perspectives and ways of working that gave rise to
communication studies in the region. In recent years, a series of works
have emerged that examine and reassert the impact of women researchers
on the history of the field; this research is situated in the Latin
American region in general, and particularly in Argentina, Mexico, and
Bolivia.\footnote{Alejandra García Vargas, Nancy Díaz Larrañaga, and
  Larisa Kejval, \emph{Mujeres de la comunicación: Argentina} (Buenos
  Aires: FES Comunicación, 2022); Yamila Heram and Santiago Gándara,
  "Pioneira: As contribuições de Michèle Mattelart para o campo da
  comunicação," \emph{MATRIZes} 14, vol. 3 (2020); Yamila Heram and
  Santiago Gándara, \emph{Pioneers in Latin American Communication
  Studies} (Buenos Aires: Teseo, 2021; Yamila Heram and Santiago
  Gándara, "Visibility and Recognition of Pioneer Women in the Field of
  Latin American Communication: An Analysis of the Trajectory of Mabel
  Piccini,'' \emph{Revista Mediterránea De Comunicación} 12, no. 2
  (2021); Yamila Heram and Santiago Gándara, "Mantener vivo un
  pensamiento crítico: Entrevista a Elizabeth Fox," \emph{Comunicación},
  no. 50 (2024); Claudia Magallanes Blanco and Paola Ricaurte Quijano,
  \emph{Mujeres de la comunicación: México} (Mexico City: FES
  Comunicación, 2022); Guillermo Mastrini, \emph{Margarita Graziano:
  Entre la academia y la acción política} (Buenos Aires: Universidad
  Nacional de General Sarmiento, 2020); Clemencia Rodríguez, Claudia
  Magallanes Blanco, Amparo Marroquín Parducci, and Omar Rincón,
  \emph{Mujeres de la comunicación} (Bogotá: FES Comunicación, 2020);
  Vania Sandoval Arenas, Rigliana Portugal Escóbar, and Sandra Villegas
  Taborga, \emph{Mujeres de la comunicación: Bolivia} (La Paz, Bolivia:
  FES Comunicación, 2022).} These studies reflect a historical need to
recover the intellectual work carried out in the early days of
communication studies by many women who have been overlooked,
marginalized, or excludedin the reconstruction of the field's
history.\footnote{Jesús Arroyave, ``Unveiling the Reasons for
  Asymmetrical Dialogue: Exploring Exclusion in the Field of
  Communication,'' \emph{Comunicación y Sociedad} (2023).} From this
perspective, this work is part of a larger series of investigations into
the trajectories of pioneering women in communication studies in Latin
America.\footnote{Yamila Heram and Santiago Gándara, dirs., "Los
  estudios en comunicación argentinos en las revistas latinoamericanas:
  Tendencias y contratendencias de un campo en disputa (2010--2020)"
  (research project, UBACyT, University of Buenos Aires, 2023--2024).} A
public communication of science project is currently being developed in
which the voices and recollections of these protagonists will be
recovered and presented in the form of a podcast, as part of an effort
to reconstruct the history of communication in the region.

The article presented here has a twofold objective: to highlight the
contributions of one of the pioneers in Latin American communication
studies, Elizabeth Fox, and to recognize and characterize Fox as a
transnational figure, whose career has extended throughout South
America, North America, and Europe---spanning at least eleven countries
(United States, Colombia, Peru, Venezuela, Chile, Mexico, Argentina,
France, Canada, Spain, and Germany)---with some of her main
collaborators also being transnational figures: Ramiro Beltrán,
Schmucler, Waisbord, and in a more limited way, Rose Kohn Goldsen. The
article is part of a recent line of collaboration between scholars from
North, South, and Central America and the Caribbean on "Exclusions in
the History and Historiography of Communication Studies" and "History of
Communication Studies in the Americas," the themes of the 2021 ICA
conference and July 2022 roundtable, respectively. Two special issues
published by the scientific journals \emph{Comunicación y Sociedad} in
Mexico and \emph{MATRIZes} in Brazil give an account of this line of
research that aims to "promote academic dialogue about the history of
media studies in the different national and linguistic contexts of the
Americas, and to open new perspectives for transnational comparative
research.''\textsuperscript{4} Or, in the words of Peter Simonson and his co-authors,
``to trace transnational flows and interregional dynamics that have
constituted communication studies in all its versions throughout the
Americas.''\textsuperscript{5} In this
way, an attempt is made\marginnote{\textsuperscript{4} Raúl Fuentes Navarro, "Presentation: Histories of
  Communication Studies in the Americas," \emph{Comunicación y Sociedad}
  (Fall 2023).} to\marginnote{\textsuperscript{5}\setcounter{footnote}{5} Peter Simonson, Jefferson Pooley, and David Park,
  "The History of Communication Studies across the Americas: A View from
  the United States," \emph{MATRIZes} 17, no. 3 (2023): 191.} begin to settle a debt of imbalance in the
recognition of South--North interconnections in the Americas, as
proposed by Simonson et al.

It is precisely for this reason that we have chosen to focus on the
figure of Elizabeth Fox: not only because of her pioneering perspective
on the political economy of communication, but also because of the
marginal, almost excluded place she has had in the reconstruction of the
communication field itself. Very little has been written about Fox; we
have found a few limited references that remember her for her work in
academic dissemination and publication. For example, Heriberto Muraro
highlights ``the great task of dissemination of our work carried out by
our dear `gringa,' Elizabeth Fox.''\footnote{Delia Crovi Druetta and
  Gustavo Cimadevilla, coords., \emph{Del mimeógrafo a las redes
  digitales: Narrativas, testimonios y análisis del campo comunicacional
  en el 40 aniversario de ALAIC} (Lima: ALAIC, 2022), 152.} Luis Peirano
mentions Fox's support for ``the publication of the first accounts of
research in Latin America,''\footnote{Crovi Druetta and
  Cimadevilla\emph{,} 179.} and Beltrán highlights the outstanding work
of Fox, ``my partner in work and ideals,''\footnote{Luis Ramiro Beltrán,
  \emph{Comunicología de la liberación, desarrollismo y políticas
  públicas} (Madrid: Luces de Galibo, 2014), 94.} in the theoretical
construction of the National Communication Policies (NCPs). Not only are
the references to Fox remarkably scarce, but the first two researchers
seem to praise Fox more for her work in supporting them in their tasks
than for her own achievements. This article seeks to fill in this gap in
the history of the communications field.

In order to give an account of Fox's trajectory, we conducted a
meta-analysis of her academic publications and identified changes,
ruptures, and continuities in her research topics and theoretical
positions, and we also conducted an interview with the
author.\footnote{We are especially grateful to Elizabeth Fox for
  granting us the interview and subsequently answering key questions by
  e-mail.} In bringing together the results of this dual approach, we
were able to recover the history of a critical intellectual current of
the 1970s and the personal trajectory of one of its pioneering figures.
The article begins with a methodological section, followed by an
overview of major moments in Fox's intellectual biography, and finally,
our conclusions.

\hypertarget{methodology}{%
\section{Methodology}\label{methodology}}

This work is part of a larger series on major moments and figures in
communication research in Latin America, and therefore our techniques
and instruments, drawn from that larger project, are not specific to the
case of Elizabeth Fox.

In the construction of Fox's intellectual biography, we postulated three
phases: 1) in the seventies, her first studies on the economic structure
of the media system and her interventions with respect to the National
Communication Policies (NCPs) from the perspective of dependency theory;
2) in the first half of the eighties, the critical balance of the
interventions of the previous period and a reconceptualization---from
new theoretical-methodological paradigms---of the relationship between
communication, culture,\\\noindent and civil society; and 3) from the end of the
1980s to the present, when Fox's trajectory shifts towards research and
the implementation of communication and health programs. We then
situated these phases within the broader context of the field of
communication studies in Latin America.\footnote{Various periodizations
  of the history of the field of communication have been proposed. The
  first, for the regional sphere, was that of Mexican researcher Raúl
  Fuentes Navarro (1992); and one of the most complete (because it
  articulates political-economic processes, theoretical matrices, and
  approaches) is the one formulated by Bolivian researcher Erick Torrico
  Villa (2004). Here, in order to understand Elizabeth
  Fox\textquotesingle s trajectory, we adopt a periodization that
  includes a period of autonomization (mid-sixties and seventies),
  institutionalization (eighties) and professionalization (nineties to
  2000). For a justification of this periodization, see Carlos Mangone,
  "The Bureaucratization of Cultural Analyses," \emph{Zigurat}, no. 4
  (2003).}

In order to produce a comprehensive account of Elizabeth Fox's
contributions to Latin American communication studies, we conducted a
meta-analysis of her academic publications, which allowed us to identify
the main trends in her research, as well as changes, ruptures, and
shifts in focus. In addition, we conducted an interview with the author
in order to reconstruct her trajectory in greater detail and, crucially,
in her own voice. The materials we included in our corpus for analysis
ranged from articles in scientific journals and book chapters to reports
and books authored or co-authored by Fox, from her first publications to
the present. While these selections span her entire career, from her
first publications to the present, particular consideration was given to
recovering her pioneering contributions. We understand, following
Barthes,\footnote{Roland Barthes, \emph{Elementos de semiología} (Buenos
  Aires: Tiempo Contemporáneo, 1971).} that a corpus is a finite
collection of materials whose construction is informed by the analyst's
objectives as well as a certain degree of arbitrariness. Nevertheless,
we have tried to build a corpus that also includes the scarce secondary
sources, and that is broad enough to allow us to reconstruct the
author's trajectory. From this meta-analysis of her publications, we can
identify changes, ruptures, and continuities in her research topics and
theoretical positions.

The technique of interviewing the author was also used both to gather
biographical data not present in secondary sources, and to reconstruct
her intellectual trajectory on the basis of her own testimony. The
interview was conducted by video call since the author lives in
Washington, DC. It followed a semi-structured interview model, with
issues and questions prepared in advance and additional questions
introduced by the interviewer over the course of the seventy-five-minute
interview for the purposes of clarification or to obtain more
information.\footnote{Roberto Hernández-Sampieri, Carlos Fernández, and
  María del Pilar Baptista, \emph{Metodología de la investigación}
  (Mexico City: McGraw Hill, 2014), 403.} Within the category of the
semi-structured interview, we chose to conduct a ``focused interview'':
an informal talk that focuses on a single topic and is usually used to
collect testimonies about some fact or event---in our case, the
interviewee's academic career.\footnote{Robert Merton, Marjorie Fiske,
  and Patricia Kendall, "Propósitos y criterios de la entrevista
  focalizada," \emph{EMPIRIA: Revista de Metodología de Ciencias
  Sociales}, no. 1 (1998).} The purpose of the interview was mainly
exploratory, since as mentioned above, there is very little literature
on the author. Therefore, we focused on getting to know and trying to
periodize her academic work from its beginnings to the present, as well
as on understanding the institutional and academic relationships that
marked her intellectual trajectory.\footnote{For more details, see Heram
  and Gándara, "Keeping Critical Thinking Alive: An Interview with
  Elizabeth Fox."}

\hypertarget{results}{%
\section{Results}\label{results}}

\hypertarget{the-beginnings-the-sixties-and-seventies}{%
\subsection{The Beginnings (The Sixties and
Seventies)}\label{the-beginnings-the-sixties-and-seventies}}

Elizabeth Fox was born in Ithaca, New York, in 1947, and grew up in
Ohio, Massachusetts, and Washington, DC. Her father, Frederic Ewing Fox,
was a Congregationalist minister and journalist who in the 1950s served
as special assistant to President Dwight Eisenhower. Elizabeth Fox spent
most of her teenage years in Washington, but graduated from high school
in Princeton, New Jersey, in 1965. Her studies continued at Vassar
College. During her first years at Vassar, she had two experiences
living in Spanish-speaking countries: in 1966 she worked at a hotel in
Spain, and she also lived for a time in Guadalajara, Mexico, in 1967. In
her third year of college, she spent six months at the Universidad de
Los Andes in Bogotá, Colombia, for the purpose of continuing her studies
in political science. There she met her first husband and decided to
stay in Colombia but changed her career and discipline, beginning
studies in journalism at the Universidad Javeriana.

It is there that the author locates her academic beginnings in the
communications field:

\begin{quote}
I got in touch with a woman who was in Colombia, who came from the Ford
Foundation,\footnote{The Ford Foundation was one of the private US
  institutions with the strongest presence in Latin America starting in
  1962, when it opened its first office in Colombia. In the context of
  the Cold War and the revolutionary processes in the region, it
  increased its contributions to finance research in different fields,
  particularly in the social sciences. Following the denunciations of
  the Camelot Project in the mid-sixties---the financing of social
  science research to identify the causes of possible insurgency
  processes in Latin America---other institutions of Canadian and German
  origin gradually took over as a source of funding for communication
  research.} Rose Kohn Goldsen, a very important figure in the early
years of communication studies in the US. She worked at Cornell
University and formed a study group on communication, to analyze above
all the structure of the media, which was a very important topic and on
which, at that time, there was not much research.
\end{quote}

\noindent Under that umbrella and with Goldsen's guidance, Fox did her
undergraduate thesis on ``the new media communication policy laws in
Colombia, which focused on how television was organized, whether it was
public or private, also highlighting the foreign influence.'' Regarding
her graduation as a journalist, Fox comments: ``I took a couple of extra
years to graduate because I didn't speak Spanish very well. I had to
start from scratch, more or less.''

According to Fox, Goldsen was very relevant in her early formative years
because she organized a communication studies group with some students
from Javeriana---in which the later prominent Colombian sociologist and
writer Azriel Bibliowicz also participated. This first study group,
based in Bogotá, would contribute to the delimitation of one of the key
themes of early communication research in Latin America: National
Communication Policies (NCPs). Goldsen was also relevant not only
because she directed Fox's graduation thesis, but also because she
helped both Fox and Bibliowicz to obtain funds from the Ford Foundation
to pursue graduate studies in the United States, where Fox returned as a
Colombian with a scholarship between 1971 and 1973 to pursue a master's
degree in communications at the University of Pennsylvania. There she
worked with Canadian policy expert and political economist William H.
``Bill'' Melody. At the time, she had a son and lived with her husband
in Princeton, New Jersey, where her husband was on a scholarship.

Goldsen's mentorship, the Ford Foundation scholarship, the back and
forth between Colombia and her country of origin, and that first contact
with the Canadians all helped to shape Fox's incipient transnational
career. In 1973, at a very young age, she returned to Colombia and made
contact with Luis Ramiro Beltrán,\footnote{Luis Ramiro Beltrán
  (1930--2015), a Bolivian-born journalist and researcher, is considered
  one of the pioneering intellectuals in the field of communication
  studies in Latin America and a central figure in the debates and
  proposals for national communication policies in the region. In his
  doctoral thesis, entitled \emph{Communication in Latin America:
  Persuasion for Status Quo or for National Development}? (1970), he
  anticipated his first questioning of the dominant paradigm in US
  research since the 1950s, which conceived of the mass media as
  instruments of modernization, development, and social change. Against
  this perspective, Beltrán would postulate a perspective that was
  sensitive to socio-economic conditions; he would research and denounce
  the monopolistic tendencies of the media system in the region and
  question the relations of dependence with the United States as a
  decisive factor in the underdevelopment of our countries. He was the
  author of numerous publications, many of which he co-wrote with
  Elizabeth Fox. For more information, see José Luis Aguirre Alvis, "La
  investigación para democratizar la comunicación: Los aportes de Luis
  Ramiro Beltrán," \emph{Revista Ciencia y Cultura}, no. 1 (1997).} who
had completed his doctorate in communication at Michigan State
University and was the Latin American representative for the Information
Sciences division of the International Development Research Centre in
Canada.\footnote{The International Development Research Centre (IDRC) is
  a public corporation created by the Parliament of Canada in 1970,
  whose statement of purpose was "to initiate, encourage, support and
  conduct research into the problems of the developing regions of the
  world and into the means of applying and adapting scientific,
  technical and other knowledge to the economic and social advancement
  of those regions." Tahira Gonsalves and Stephen Baranyi,
  \emph{Research for Policy Influence: A History of IDRC Intent}
  (Ottawa: International Development Research Centre, 2003), 5.}

With funding from this Canadian institution, they undertook a series of
investigations in the region, as Fox recounts:

\begin{quote}
{[}Beltrán{]} asked me to go to Venezuela to do the same type of
research that I had done in Colombia. That was in seventy-three,
seventy-four. It was a short stay, a couple of months. Then I came back
and joined IDRC working with Luis Ramiro: we started to develop, among
other things, a program to apply communication research in different
Latin American countries, because Beatriz Solís was already starting in
Mexico, Giselle Munizaga was in Chile.
\end{quote}

\noindent Beltrán and Fox's collaboration is noteworthy, as many of their
publications show, and yet Fox is all but absent from existing accounts
of the field's history and those which are still being written today,
where there remains a tendency to highlight certain figures and neglect
others.\textsuperscript{18} We share with Simonson and his co-authors the conviction
that "exclusions involving gender, race, language, colonialism,
geopolitical location, and institutionally endorsed privilege will be
reproduced in the formal and informal accounts of our
fields\textquotesingle{} pasts."\textsuperscript{19} This is the case of Fox, who, having worked and published numerous
articles together with Beltrán, remains an indistinct figure with few
mentions. For example, in the book \emph{Comunicología de la liberación,
desarrollismo y políticas públicas} (2014) by Luis Ramiro Beltrán, with
a foreword by Manuel Chaparro and an introduction by Alejandro
Barranquero, Fox is mentioned only briefly. Perhaps Fox, as a figure,
has also been marginalized due to her transnational career that has made
her an \emph{outsider} in Latin America and also in the United States.
After all, she was a native of the United States who wrote from Latin
America against the power wielded in the region by her country of
origin.

\newpage Fox\marginnote{\textsuperscript{18} Arroyave, "Unveiling the Reasons for Asymmetric
  Dialogue.''} participated\marginnote{\textsuperscript{19}\setcounter{footnote}{19} Peter D. Simonson, David Park,
  and Jefferson Pooley, "Exclusions/Exclusiones: The Role for History in
  the Field's Reckoning,'' \emph{History of Media Studies} 2 (2022), 1.} in the drafting of reports and documents used for the
preliminary meetings and debates on National Communication Policies
(NCPs) and the New World Information and Communication Order (NWICO).
She also collaborated as an external advisor for the RATELVE
project\footnote{"The Radio and Television Committee was in charge of
  preparing a report on mass cultural production. From his position as
  director of the Committee, Pasquali established dialogues with a range
  of experts from the cultural, political, and intellectual fields, who,
  between November 1974 and May 1975, met in twenty-eight working
  sessions to formulate the broadcasting policy of the Venezuelan State.
  The Ratelve Report was finalized in May 1975, and was subsequently
  published in book form by the bookstore and publishing house SUMA in
  1977 under the title \emph{Proyecto Ratelve}. Emiliano Sánchez
  Narvarte, "Antonio Pasquali y las políticas de comunicación en
  Venezuela (1974--1979)," 5.} and was president of the Colombian
Association of Communication Researchers from 1978 to 1980). Her
contribution---like that of many other researchers---was decisive in the
context of revolutionary processes of social transformation and national
liberation in the post-war period, when the entry of countries belonging
to what was then known as the ``Third World'' weakened the US's defense
of the free flow of information as an ideal. Not only was the inequality
of flows questioned, but an alternative communicational order was also
proposed.

In the period of autonomization of communication studies in the
region,\footnote{Raúl Fuentes Navarro, \emph{Un campo cargado de futuro:
  El estudio de la comunicación en América Latina} (Mexico City:
  FELAFACS, 1992).} institutional resources played a key role in
fostering and contributing to research. In our interview, Fox highlights
in particular the role played by the Canadians through the International
Development Research Centre (ISDR):

\begin{quote}
They wanted to support this research because it was a time when Canada
was running a different media model than the US. Canada was also
supporting NWICO, unlike the US. In fact they were resisting the whole
influx of US programming in their own country.
\end{quote}

\noindent The financing of research projects meant that resources were available
to convene researchers from Latin America, Fox explains:

\begin{quote}
So we financed meetings. We went to Cartagena, to Santa Marta, to Cali,
to Peru. We could not do things in Chile, because the dictatorship was
already in place, but we did them at ILET\footnote{The Latin American
  Institute for Transnational Studies (ILET, in its initials in Spanish)
  was founded in 1975 and had an area of communication and development
  in which the Chileans Juan Somovía (1941) and Fernando Reyes Matta
  (1938) participated, later to be joined by the Argentines Héctor
  Schmucler (1931--2018), Sergio Caletti (1947--2015) and Alcira
  Argumedo (1940--2021), among others. See Facundo Altamirano,
  "Intelectuales, exilio y comunicación en el Instituto Latinoamericano
  de Estudios Transnacionales (ILET) (1975--1984)," \emph{Journal of the
  Red Intercátedras de Historia de América Latina Contemporánea} 7, no.
  13 (2020/2021).} in Mexico, which was then run by Juan Somovía. ILET
became the other focus of critical media studies. Somavía, Rafael
Roncagliolo were there. . . . It was fundamental that Luis Ramiro and I
had a stable institutional base, that I had a telephone, that I was able
to call long distance whenever I wanted, that we had money to finance
meetings . . .
\end{quote}

\hypertarget{her-first-publications-the-seventies}{%
\subsection{Her First Publications (The
Seventies)}\label{her-first-publications-the-seventies}}

The articles Fox produced throughout the 1970s---a time when she was
primarily based in Bogotá but continued to travel to and concern herself
with different sites in Latin America---can be organized around a common
diagnostic throughline. By and large, this work called attention to the
preponderance of data coming from the United States and decried the
disproportionate penetration of that information within the field. The
pioneering nature of these investigations is also revealed in the
absence of a library---authors, bibliography, theories---that could
serve as a general framework. As Fox points out, there was "nothing" in
that library. That is why, she recalls:

\begin{quote}
Rose Kohn Goldsen\textquotesingle s teaching was so important, because
there were such studies in the United States, where they began to
investigate the media. But there were no studies on Latin America. So my
sources were primary, that is, legislation, interviews with congressmen,
with businessmen, with people from educational TV.
\end{quote}

\noindent One of the most representative publications of her stay in Colombia is
"La televisión norteamericana en América Latina" (1974), published in
the journal \emph{Chasqui}, where she examines the state of the art of
television in the region: the profits obtained from the sale of canned
programming; the role played by the networks ABC, CBS, NBC, and
\emph{Time} in providing capital and technical assistance; the
advertising business and the direct sale of programming; as well as the
development of and changes in commercial operations. Her conference
paper ``Políticas Nacionales de Comunicación en América Latina''
summarizes the central aspects of the meeting of experts on
Communication Policies and Planning in Latin America, held in Bogotá in
July 1974. This meeting---together with another held in Quito in 1975 on
the exchange of news---provided input for discussion at the Seminar on
National Communication Policies in Latin America held by the
International Center for Higher Education in Communication in Latin
America (CIESPAL),\footnote{Based in Quito, Ecuador, the International
  Center for Higher Education in Communication in Latin America
  (CIESPAL) was created by UNESCO in 1959 to train journalists. By 1970,
  it had become a center for dissemination and consultation for
  communication researchers.} the Center for Democratic Studies for
Latin America (CEDAL),\footnote{The Costa Rica-based Center for
  Democratic Studies for Latin America (CEDAL), founded in 1968,
  promoted scholarships and publications to undertake communication
  studies, among other topics.} and the Friedrich Ebert
Foundation\footnote{The Friedrich Ebert Foundation, linked to German
  social democracy, was another key institution in financing the
  training of journalists and regional meetings, scholarships, and
  research.} in April 1975 in San José, Costa Rica, in preparation for
the following year's conference.

Together with Luis Ramiro Beltrán, Fox published "La influencia de los
Estados Unidos en la comunicación masiva en América Latina:
desequilibrio en el flujo de información" (1976) for the meeting on Fair
Policy in International Information Exchange in Hawaii; this report was
then requested by CIESPAL to be used in the seminar "La radio y la
televisión" held in San José, Costa Rica, in April 1976, under the
auspices of CEDAL and CIESPAL. Fox and Beltrán also published, in 1980,
an abridged version of the document, with the title "Medios de
comunicación de masas y dominación cultural," which they presented in
1979 at the Symposium on the Role of International Broadcasting,
sponsored by Radio Nederland and held in Hilversum, the Netherlands.

The common denominator in all of these articles---among many others
published during this period---is the search for data and an explanation
of the difficulties in obtaining them, since it is precisely data that
are needed to accompany and support the denunciations of inequality in
the flow of information and the structure of the media. We share some
quotations from these works:

\begin{quote}
It is difficult to obtain unbiased data on the growth of the US
television industry overseas.\footnote{Elizabeth Fox de Cardona, "La
  televisión norteamericana en América Latina," \emph{Chasqui} 6 (1974):
  56.}

The basic document for this conference asked whether there was any clear
evidence of foreign penetration within a country\textquotesingle s
communication system. It seemed to question whether there was
necessarily a communication imbalance between developed and
underdeveloped countries.\footnote{Elizabeth Fox de Cardona and Luis
  Ramiro Beltrán, "La influencia de Estados Unidos en la comunicación
  masiva en América Latina," 53.}

The following pages will examine some of these indicators in order to
briefly illustrate the situation, with emphasis on Latin America and
broadcasting (radio and television), where the available data
allow.\footnote{Luis Ramiro Beltrán and Elizabeth Fox de Cardona, "
  Medios de comunicación de masas y dominación cultural,"
  \emph{Perspectiva} 10 (1980): 88.}

Empirical evidence justifies the dissatisfaction of Third World
countries with the international communication situation. \footnote{Beltrán
  and Fox de Cardona, 91.}
\end{quote}

\noindent The pioneering value of these investigations produced in Latin America
is that they lay bare the economic and cultural domination of the United
States in the region. Fox was among a group of pioneering
researchers---which also included Argentina\textquotesingle s Margarita
Graziano and Mexico\textquotesingle s Fátima Fernández---who sought and
produced data on the structure of the media in order to support their
denunciations and interventions under the imprint of dependency theory
and cultural imperialism. Their shared concerns reflect the same hopes
and later disappointments with respect to the NCPs and NWICO. Both
Graziano and Fox published very similar articles in 1974: the former in
the magazine \emph{Comunicación y Cultura}, the latter in
\emph{Chasqui.} Both seek data, critique, and promote discussions.

Fox's final work in this stage was the book \emph{Comunicación dominada:
Estados Unidos en los medios de América Latina}, co-written with
Beltrán, which published in 1980 as Fox left Colombia for Buenos Aires.
This was her last publication whose primary focus was the analysis of
cultural domination, defined as "a verifiable process of social
influence by which a nation imposes on other countries its set of
beliefs, values, knowledge and norms of behavior, as well as its general
way of life."\footnote{Luis Ramiro Beltrán and Elizabeth Fox, \emph{La
  comunicación dominada} (Mexico City: ILET-Nueva Visión, 1980), 20.}
The book compiled "a large part of the information resulting from the
systematic verification of the phenomenon of communications in Latin
America and its relationship with the United States."\footnote{Beltrán
  and Fox\emph{,} 21.} Fox and Beltrán reviewed a mass of research
undertaken up to that time in the region, which demonstrated the degree
of concentration of the industry (radio, television, advertising,
press), the penetration of US capital in local corporations, and the
falsity of the thesis of the free flow of information, which provided a
flimsy cover for the unilateral diffusion driven by imperialism.
Likewise, in the last chapter, they pointed out the need to formulate
alternative communication policies.

\newpage\hypertarget{from-bogot-to-buenos-aires-disenchantment-and-balance-the-eighties}{%
\subsection{From Bogotá to Buenos Aires:
Disenchantment and Balance\\\noindent (The
Eighties)}\label{from-bogot-to-buenos-aires-disenchantment-and-balance-the-eighties}}

In 1980 Fox left Bogotá and moved to Buenos Aires, where she continued
working for the International Development Research Center until 1984. We
can locate here a second moment in her trajectory based on her
disenchantment with the truncated projects of the NCPs. This can be
explained by the role played by the governments of the region with
respect to the recommendations that they themselves made at the San José
de Costa Rica meeting, which proved to be a pious statement of good
intentions with no real transformative impact on a continent that, to a
large extent, was being governed by military dictatorships. This is how
Luis Gonzaga Motta summarized it six years after the declaration:

\begin{quote}
The proposal of national communication policies, which at the beginning
seemed to the progressive sectors a promising path to follow, must now
be re-discussed in the light of recent experiences and revised as an
alternative for the democratization of communication. The
continent\textquotesingle s theoretical and practical communication
professionals (professors, researchers, journalists, educators, etc.)
must reformulate their own positions of the past decade and reorient
their attitudes and struggles based on concrete experiences.\footnote{Luis
  Gonzaga Motta, "Costa Rica: Six Years Later," \emph{Chasqui,} no. 3
  (1982): 14--15.}
\end{quote}

\noindent In line with these and other disenchantments, the introduction to the
book \emph{Comunicación y democracia en América Latina} (1982)---written
together with Héctor Schmucler and emerging from meetings of the group
on communication of the Latin American Council of Social Sciences
(CLACSO)\footnote{CLACSO is an institution created in 1967 at the
  initiative of UNESCO, with headquarters in Buenos Aires, Argentina,
  whose objectives are related to the promotion, dissemination, and
  exchange of research in the field of social sciences.}---presented a
critical assessment of the communication studies of the previous decade:

\begin{quote}
Dependency theory---which contributed data that had barely been
considered until then---became a rigid, restrictive framework . . . the
absolute responsibility assigned to the external enemy repeatedly
overshadowed the analysis of forces.\footnote{Elizabeth Fox and Héctor
  Schmucler, "Introducción," in \emph{Comunicación y democracia en
  América Latina,} ed. Elizabeth Fox and Héctor Schmucler (Lima: CLACSO,
  1982), 18.}
\end{quote}

\noindent Fox mentions that, in this period, she began to look more at the subject
of social movements and the relationship between communication and
democracy:

\begin{quote}
Well, it was a natural evolution. And I also believe that it was the
contact with the intellectuals from the Southern Cone, from Uruguay,
from Chile, at that time, from Argentina, from Brazil, who were thinking
about rebuilding democracy, about the role of civil society, of the
State, about the nature of the State . . . because it was the moment
when things were beginning, at least in Argentina, to return to
democracy.
\end{quote}

\noindent The book \emph{Comunicación y democracia} proposed other questions and
new paradigms:

\begin{quote}
History demands the refinement, and often the replacement, of the
concepts we use to think about how to redesign society . . . it is not
very useful to think about communication theories without alluding to
the social practices that condition communicative forms.\footnote{Fox
  and Schmucler, 15.}
\end{quote}

\noindent In this regard, the book advocated the relevance of reception studies:

\begin{quote}
The contradictory nature of communication phenomena was not always
emphasized with sufficient force. The stories of the media were often
repeated, and they were interpreted taking into account the way they
were managed by the dominant sectors of local societies or by the
central countries in the international arena. Little effort was directed
at the analysis of the other pole: that of the dominated. In the
venerable sender-receiver dichotomy, the concern of scholars was
generally directed towards the former variable. \footnote{Fox and
  Schmucler, 12.}
\end{quote}

\noindent Along these lines, Fox and Schmucler questioned the place of Power with
a capital P and proposed reflecting in terms of multi-situated power:
"Power as a monolithic and singular nucleus that establishes its
dominion over society as a whole is now regarded as a concept that needs
to be replaced . . . infrequent questions arise about the concept of
hegemony."\footnote{Fox and Schmucler, 15.} In the same sense, the
article "Comunicación y sociedad civil: Una temática incipiente" served
as a balance and projection. Fox asked why study communications from the
perspective of civil society, and the answer situated her in the context
of a region subjected to authoritarian regimes:

\begin{quote}
The suppression or elimination of many of the more traditional forms of
communication is the norm in the countries of the region, with few
exceptions. The question then becomes: what communication processes
continue among the members of society?; how do they receive, send and
process information?; and what are the consequences for the social
fabric?\footnote{Elizabeth Fox," Comunicación y sociedad civil: Una
  temática incipiente," \emph{Crítica \& Utopia} 7 (1982): 1.}
\end{quote}


\noindent As can be seen, these reconsiderations and reconceptualizations of
communicational processes were in line with the shifts that were taking
place in those same years in the field of communication in Latin
America. It can be summed up in terms of the most significant
theoretical shift: from dependency theory to the theory of hegemony, a
re-reading of Gramscian thought which, in turn, connected with the
circulation of British cultural studies in our region.\footnote{We are
  aware that we are just pointing out the main changes---which we label
  as "shifts"---that can be observed in the field of communication in
  Latin America. We can also note the passage from production to
  reception, from media to mediations, from mass culture to popular
  culture and everyday life. To the re-reading of Gramsci and the
  circulation of British cultural studies we should add the works of
  Pierre Bourdieu, Michel Foucault, and Michel de Certeau, among others,
  which also had a widespread reception in the field of communication
  and social sciences in our region during this period.}



During this same period, between 1980 and 1984, Fox served as
vice-president of ALAIC\textsuperscript{40}---with
the presidency in the hands of Patricia Anzola. This was a complicated
moment for the association: it had emerged in 1978 with the objective of
giving "a greater institutional representation of the region before the
United Nations Educational, Scientific and Cultural Organization
(UNESCO) and the\newpage\noindent International Association for Media and Communication
Research (IAMCR),"\textsuperscript{41} but the new Latin American political
context, the weakening\marginnote{\textsuperscript{40} The Latin American Association of
  Communication Researchers (ALAIC) was founded in 1978, with
  headquarters in Caracas, Venezuela, on the initiative of Antonio
  Pasquali, Luis Ramiro Beltrán, Elizabeth Fox, among many others.} of\marginnote{\textsuperscript{41}\setcounter{footnote}{41} Cicilia Krohling, "La presencia de ALAIC en
  la comunidad latinoamericana de Ciencias de la Comunicación,"
  \emph{Telos}, no. 61 (2004): 1.} UNESCO, and \textbf{"}the withdrawal of funds
led to the isolation of ALAIC, although it remained alive in a more
informal way."\footnote{María Victoria Martin and Leila Vicentini,
  "Comunicación y memoria: ALAIC en el contexto latinoamericano,"
  \emph{Oficios Terrestre}, 15/16 (2004): 265.}

Different initiatives were developed in the association, and Fox,
together with Luis Ramiro Beltrán, Luis Peirano, and Patricia Anzola,
worked on a project to make an inventory of academic publications on
communication in Argentina, Brazil, Colombia, Chile, and Peru. Financial
support for this initiative came from the International Development
Research Centre of Canada.

In 1984 Fox moved to Paris, where she lived until 1990, working as a
consultant for different organizations. From then on she gradually moved
away from the field of communication and media: in 2002 she edited,
together with Silvio Waisbord, \emph{Latin Politics Global Media}, which
would be her last book. Fox explains how this project came about:

\begin{quote}
In 1984 I left Buenos Aires, married an American journalist, and went to
Paris. I spent five years there. While I was there, working with Rafael
Roncagliolo at ILET, in Peru, the Germans financed a study for me;
because I left the IDRC, the Germans financed a study to make
comparative policies for Latin America. That was the basis of that book,
which is an edited book on communication policies in Latin America,
which is more of a historical review. I wrote it in Paris.
\end{quote}

\noindent Between 1986 and 1987 Fox worked for the Volkswagen Foundation under the
Communications Policy in Latin America program, between 1987 and 1988
for UNESCO, and in 1988 for the World Bank. In addition, between 1990
and 1991 she held the UNESCO Chair of Communication at the Autonomous
University of Barcelona. Since 1988 she has been involved in the field
of communication and health, where she has worked for the last
thirty-five years.

\hypertarget{intervention-work-involving-communication-and-health-since-the-nineties}{%
\subsection{Intervention Work Involving
Communication and Health\\\noindent (Since the
Nineties)}\label{intervention-work-involving-communication-and-health-since-the-nineties}}

In 1990 Fox moved to Washington, DC, where she worked until 1995 at the
Pan American Health Organization and later for the United States Agency
for International Development (USAID),\footnote{In an interview with
  Beltrán on the history of communication for development, he explains,
  "In the 1980s and 1990s, some international organizations took great
  pains to support the National Communication Program for
  People\textquotesingle s Health and Nutrition, the foundations of
  development. Unicef and USAID made considerable contributions in this
  sense and the Pan American Health Organization (PAHO) made an effort}
 in different
positions and functions: from 1996 to 2004 as Senior Technical Advisor
in Health Communications and Behavior Change; from 2004 to 2011 as
Deputy Director; from 2011 to 2016 as Director of the Office of Health,
Infectious Diseases and Nutrition; and from 2017 to 2019 as Deputy
Coordinator for Maternal and Child Survival. She was also Vice President
of the International Association for Media and Communication Research
(IAMCR) from 1996 to 2000. Finally, from 2007 to 2015\marginnote{to help the ministries of health, giving priority to primary health
  care and education of the people on the main health problems, so that
  they would strive to strengthen the goal of \textquotesingle Health
  for All in the Year 2000.\textquotesingle" Fanny Patricia Franco
  Chávez and Ana María López Rojas, "Una mirada a las raíces de la
  comunicación para el Desarrollo: Entrevista con Luis Ramiro Beltrán
  Salmón," \emph{Signo y Pensamiento}, no. 58 (2011): 172.} she worked as an
adjunct professor of the subject "Communication, Health, and
Development" at the American University in Washington, DC. Fox
elaborates on this shift in her career:

\begin{quote}
I started a career at USAID in public health, as a social scientist,
studying how tuberculosis and malaria programs are organized, how child
and maternal and child health is done, how information is brought to
women. In other words, I became much more involved in applied research.
I had a twenty-five-year career. I retired four years ago, and I was not
going to do anything. However, I went back to work where I am now, at
the Pan American Development Foundation, which is a foundation that
supports social sciences in Latin America. I'm having fun, working more
on democracy programs, transition of democracy, migration, peace and
justice, a little bit of health. But more like human rights and
democracy.
\end{quote}

\noindent To close her pioneering career in communication studies in Latin
America, in 2007 she was named Doctor Honoris Causa by the Pontificia
Universidad Católica del Perú, an academic institution to which she
donated her entire library of Latin American authors twenty years ago.
This is how Fox remembers it:

\begin{quote}
With so much moving, when I was widowed, I packed it all up and gave it
all to the Catholic University of Lima. So it is in their library, at
the Catholic University of Lima, in the School of Communication, because
my friend Luis Peirano was dean at that time. That is why I made a
donation of about two thousand books to the Catholic University. Because
I was not doing research and there were many primary sources.
\end{quote}

\hypertarget{conclusions}{%
\section{Conclusions}\label{conclusions}}

In this article we have presented the professional and academic
trajectory of Elizabeth Fox, with a twofold purpose: on the one hand, to
shed light on her pioneering contributions to the Latin American
communicational field, "trying to situate our fields in their
corresponding local, national and regional scales and the processes of
their institutionalization and development in their respective
historical contexts"\footnote{Maria Immacolata Vassallo de Lopes and
  Raúl Fuentes Navarro, "Histórias da internacionalização do campo de
  estudos da comunicação," \emph{MATRIZes} 17, no. 3(2023): 6.}; and on
the other hand, to recognize Fox as a transnational figure, since, as
mentioned by Simonson and his co-authors,\footnote{Simonson, Pooley, and
  Park, "A história dos estudos de comunicaçãonas Américas," 190.} there
has been a pattern of imbalance in the recognition of South--North
interconnections in the Americas. This is precisely what sparked our
interest in the author: the limited visibility of her contributions both
in Latin America---the region where she worked and lived for more than a
decade---and also in the United States, the author\textquotesingle s
country of origin and where she has lived and worked for more than three
decades.

Elizabeth Fox's intellectual biography spans the history of
communication studies in Latin America, particularly the last decades of
the last century in which she was one of its key figures. What she
called "national communication policies" was not only about
consolidating what would later become one of the most productive Latin
American traditions---that of critical political economy---it also
responded to a critical mass of empirical research that was being
undertaken in each of the countries of the region. Its
results---accumulating data on the US's concentration and penetration in
the region's media---allowed the scholarly community to demonstrate and
denounce this "dominated communication" in times of cultural
imperialism. Fox\textquotesingle s history and story reveal both the
pioneering nature of this research enterprise---where libraries served
as primary sources, to borrow an image from Fox herself---and the
collective dimension of this undertaking, which brought together
researchers from the region and inaugurated the first institutions where
they met to debate and where they had access to financial resources in
an international context that, until the 1980s, was still open to the
debate about unequal information and cultural flows. Fox not only
produced prolific academic literature but also played a central role as
an organizer, bringing together many other researchers in the region to
participate in the institutions that were beginning to take shape and
promoting collaborative interventions in regional and national
communication policies. She played a very important role in the
beginnings of the field, was able to move around to different countries
and continents, and worked alongside other transnational figures such as
Beltrán. Despite this, she is only sporadically mentioned in the
reconstruction of the history of the communication field, both in the
United States and in Latin America.

As early as the 1980s, Fox critically reviewed the perspectives of the
foundational period and began to share the theoretical and
methodological shifts that were developing in the field of communication
in Latin America. Democracy, civil society, and reception were some of
the key words in that era of closure of the processes of social
transformation, of more or less weak democratic transitions, and of the
construction of new consensuses. Even in this context---which would lead
in the 1990s to the consolidation of a neoliberal consensus---Fox
persisted in placing her trust in the communication practices of civil
society.

One fact that stands out in Fox\textquotesingle s account is the role
played by different institutions that financed work, research, and
collective meetings. According to her testimony, she obtained different
types of funding from at least eight institutions at the beginning of
her career: the Ford Foundation (United States), International
Development Research Centre (Canada), CIESPAL (Ecuador), CEDAL (Costa
Rica), Friedrich Ebert Foundation (Germany), Latin American Institute of
Transnational Studies (Mexico), and CLACSO (Argentina), in addition to
actively participating in the Latin American Association of
Communication Researchers (ALAIC), where she eventually rose to the
position of vice-president. Surprisingly, such funding did not limit
research that challenged an unequal international order. A series of
confluences helped to finance projects and research, pay salaries, and
guarantee the necessary conditions for academic production, such as the
organization of events and creation of media networks that enabled
permanent contact and exchange (as Fox recalled: "We could make
long-distance calls.''). In the interview, Fox points out at least two
converging factors. On the one hand, she highlights the "very liberal"
orientation of the Ford Foundation in those early years, which favored
criticism of the functioning of mass media in the region and of US
cultural domination. On the other hand, she points out that Canadian and
German interests coincided with those of the so-called Third World
countries, strengthening support for a New World Information and
Communication Order and leading to greater promotion of research that
worked from that perspective. Canada confronted in a relatively similar
way the growing media and cultural influence of the United States in its
territory, while at the same time supporting a public communication
model opposed to that of its neighbor. Germany, through the Ebert
Foundation, recovered the more liberal traditions of social democracy
and had a public media system. For our part, we add a third factor: in
the 1970s, UNESCO was still receptive to debates on NWICO, to the
formulation of national or regional communication policies, and to the
denunciation of the unequal North--South flow of information. Only a
decade later, in the early 1980s, the withdrawal of the United States
and Great Britain from UNESCO would alter the situation of this
organization and of the European media landscape, which was beginning a
process of privatization of its audiovisual market.

Half a century later, this regional movement that investigated
communication policies is part of a history that deserves to be
recovered. Researchers in this field today must review these archives in
order to avoid crystallizing their exclusions in memory and to instead
update these records and make them available again as the basis for a
critical perspective of communication studies in Latin America. Celia
Del Palacio Montiel proposes not only giving visibility but also
advancing in a process of "searching for connections and analyzing from
the local space . . . but without taking our eyes off the
world."\footnote{Celia Del Palacio Montiel, "History of Communication
  Studies from the Regions of Latin America," \emph{Comunicación y
  Sociedad} (2023): 16.} These "connected histories" of which she speaks
create a tension between homogenizing views and particularist
pretensions, and also mark the next horizon for this critical project,
in part because the processes and demands that gave rise to it are still
open: the profound communicational inequality between the global north
and south, and the lack of communicational democracy in each of our
countries.




\section{Bibliography}\label{bibliography}

\begin{hangparas}{.25in}{1} 


Bibliography}{ Bibliography}}\label{bibliography}}

Altamirano, Facundo. ``Intelectuales, exilio y comunicación en el
Instituto Latinoamericano de Estudios Transnacionales (ILET)
(1975--1984).'' \emph{Revista de la Red Intercátedras de Historia de
América Latina Contemporánea} 7, no. 13 (2020/2021): 250--58.

Aguirre Alvis, José Luis. ``La investigación para democratizar la
comunicación: Los aportes de Luis Ramiro Beltrán.'' \emph{Revista
Ciencia y Cultura}, no. 1 (1997).
\url{http://www.scielo.org.bo/scielo.php?script=sci_arttext\&pid=S2077-33231997000100011}.

Arroyave, Jesús. ``Unveiling the Reasons for Asymmetric Dialogue:
Exploring Exclusion in the Field of Communication.'' \emph{Comunicación
y Sociedad} (2023): 1--21.
\url{https://doi.org/10.32870/cys.v2023.8719}.

Barthes, Roland. \emph{Elementos de semiología}. Buenos Aires: Tiempo
Contemporáneo, 1971.

Beltrán, Luis Ramiro. \emph{Comunicación, política y desarrollo}. Quito:
CIESPAL, 2014.

Beltrán, Luis Ramiro. \emph{Comunicología de la liberación,
desarrollismo y políticas públicas.} Madrid: Luces de Galibo, 2014.

Beltrán, Luis Ramiro, and Elizabeth Fox de Cardona. ``Medios de
comunicación de masas y dominación cultural.'' \emph{Perspectivas} 10,
no. 1 (1980): 84--98.

Beltrán, Luis Ramiro, and Elizabeth Fox de Cardona. \emph{Comunicación
dominada: Estados Unidos en los medios de América Latina}. Mexico City:
Editorial ILET-Nueva Visión, 1980.

Cimadevilla, Gustavo. ``Entrevista a Heriberto Muraro `El viento sopla
por donde quiere.'\,'' \emph{Revista Latinoamericana de Ciencias de la
Comunicación} 15, no. 28 (2017): 148--54.

Crovi Druetta, Delia, and Gustavo Cimadevilla, coords. \emph{Del
mimeógrafo a las redes digitales: Narrativas, testimonios y análisis del
campo comunicacional en el 40 aniversario de ALAIC}. Mexico City: ALAIC,
2018.

Del Palacio Montiel, Celia. ``History of Communication Studies from
Latin American Regions: Connected Histories as a Resource for
Analysis.'' \emph{Comunicación y Sociedad} (2023): 1--19.
\url{https://doi.org/10.32870/cys.v2023.8609}.

Fox, Elizabeth. ``Comunicación y sociedad civil: Una temática
incipiente.'' \emph{Crítica \& Utopía,} no. 7 (1982).

Fox, Elizabeth. \emph{Latin American Broadcasting: From Tango to
Telenovela.} Luton, England: University of Luton Press, 1997.

Fox, Elizabeth. \emph{Media and Politics in Latin America, The Struggle
for Democracy}. London: Sage, 1988.

Fox, Elizabeth. ``Perplejidades compartidas sobre la comunicación
democrática'' {[}Shared perplexities about Democratic communication{]}.
\emph{Crítica \& Utopía,} no. 9 (1983).

Fox, Elizabeth. ``La política de reforma de la comunicación en América
Latina.'' \emph{Diálogos de la comunicación,} no. 21 (1988).

Fox, Elizabeth. ``Políticas nacionales de comunicación en América
Latina.'' In \emph{Políticas de comunicación en sociedades de cambio},
edited by Marco Ordoñez, Elizabeth Fox de Cardona, and Benjamin Ortiz
Brennan, 91--97. San José, Costa Rica: Cuadernos CEDAL, 1975.

Fox, Elizabeth. ``Tres visitas al paradigma de la dependencia
cultural.'' \emph{Chasqui,} no. 44 (1993): 80--87.

Fox de Cardona, Elizabeth. ``Colombia.'' In \emph{Políticas Nacionales
de Comunicación}, edited by Peter Schenkel, 243--86. Quito: Editorial
Época, 1981.

Fox de Cardona, Elizabeth. \emph{Medios de comunicación y política en
América Latina}. Barcelona: Gustavo Gili, 1989.

Fox de Cardona, Elizabeth. ``Situación y Política de Comunicación en
Colombia: El caso de la prensa, la radio y la televisión.''
\emph{Cultura y Comunicación en América Latina}, no. 7 (1982).

Fox de Cardona, Elizabeth. ``La televisión norteamericana en América
Latina.'' \emph{Chasqui}, no. 6 (1974): 53--70.

Fox de Cardona, Elizabeth, and Luis Ramiro Beltrán. ``La influencia de
Estados Unidos en la comunicación masiva en América Latina:
desequilibrios en el flujo de información.'' (Paper prepared for the
Meeting on Fair Policy in International Information Exchange, East-West
Center, East West Communication Institute, Honolulu, Hawaii, 1976).

Fox, Elizabeth, and Hector Schmucler. ``Introducción.'' In
\emph{Comunicación y democracia en América Latina}. Buenos Aires:
CLACSO, 1982.

Fox, Elizabeth, Héctor Schmucler, Patricia Terrero, Giselle Munizaga,
Luis Gonzaga Motta, Luis Peirano, Oswaldo Carpriles, et al.
\emph{Comunicación y democracia en América Latina}. Lima: CLACSO, 1982.

Fox, Elizabeth, and Silvio Waisbord. \emph{Latin Politics, Global
Media.} Austin: University of Texas Press, 2002.

Fuentes Navarro, Raúl. \emph{Un campo cargado de futuro: El estudio de
la comunicación en América Latina}. Mexico City: FELAFACS, 1992.

Fuentes Navarro, Raúl. ``Introduction: Histories of Communication
Studies in the Americas.'' \emph{Comunicación y Sociedad} (2023): 1--5.
\href{https://doi.org/10.32870/cys.v2023.8737}{https://doi.org/10.32870/cys.v2023.8737.}

Franco Chávez, Fanny Patricia, and Ana María López Rojas. ``Una mirada a
las raíces de la comunicación para el desarrollo: Entrevista con Luis
Ramiro Beltrán Salmón.'' \emph{Signo y Pensamiento}, no. 58 (2011):
170--77.

García Vargas, Alejandra, Nancy Díaz Larrañaga, and Larisa Kejval.
\emph{Mujeres de la comunicación: Argentina}. Buenos Aires: FES
Comunicación, 2022.

Gonsalves, Tahira, and Stephen Baranyi. \emph{Research for Policy
Influence: A History of IDRC Intent}. Ottawa: International Development
Research Centre, 2003.

Graziano, Margarita. \emph{Guía teórica primera parte: Carrera de
Ciencia de la Comunicación}. Buenos Aires: University of Buenos Aires,
1997.

Hernández-Sampieri, Roberto, Carlos Fernández, and María del Pilar
Baptista. \emph{Metodología de la investigación}. Mexico City: McGraw
Hill, 2014.

Heram, Yamila, and Santiago Gándara. ``Mantener vivo un pensamiento
crítico: Entrevista a Elizabeth Fox.'' \emph{Comunicación}, no. 50
(2024): 140--51\href{https://doi.org/10.18566/comunica.n50.a07}{.
https://doi.org/10.18566/comunica.n50.a07.}

Heram, Yamila, and Santiago Gándara. ``Pioneira: As contribuições de
Michèle Mattelart para o campo da comunicação.'' \emph{MATRIZes} 14, no.
3 (2020):
51­--68\href{https://doi.org/10.11606/issn.1982-8160.v14i3p51-68}{.
https://doi.org/10.11606/issn.1982-8160.v14i3p51-68.}

Heram, Yamila, and Santiago Gándara. \emph{Pioneras en los estudios
latinoamericanos de Comunicación}. Buenos Aires: Teseo, 2021. \\\hspace{.25in}\href{https://www.teseopress.com/pionerasenlosestudioslatinoamericanosdecomunicacion}{https://www.teseopress.com/pionerasenlosestudioslatino}\\\hspace{.24in}\href{https://www.teseopress.com/pionerasenlosestudioslatinoamericanosdecomunicacion}{americanosdecomunicacion/}.

Heram, Yamila, and Santiago Gándara. ``Visibility and Recognition of
Pioneer Women in the Latin American Communicational Field: An Analysis
of the Trajectory of Mabel Piccini.'' \emph{Revista Mediterránea de
Comunicación} 12, no. 2 (2021):
65--75\href{file:///Users/Emily/Desktop/.\%20https:/doi.org/10.14198/MEDCOM.19151.}{.
https://doi.org/10.14198/MEDCOM.19151.}

Krohling, Cicilia. ``La presencia de ALAIC en la comunidad
latinoamericana de Ciencias de la Comunicación.'' \emph{Telos,} no. 61
(2004). Segunda Época.

Magallanes Blanco, Claudia, and Paola Ricaurte Quijano. \emph{Mujeres de
la comunicación: México.} Mexico City: FES Comunicación, 2022.

Mangone, Carlos. ``La burocratización de los análisis culturales.''
\emph{Zigurat}, no. 4 (2003): 135--58.

Martin, María Victoria, and Leila Vicentini. ``Comunicación y memoria:
ALAIC en el contexto latinoamericano.'' \emph{Oficios terrestres}, no.
15/16 (2004): 262--67.

Mastrini, Guillermo. \emph{Margarita Graziano: Entre la academia y la
acción política}. Buenos Aires: Universidad Nacional de General
Sarmiento, 2020.

Merton, Robert, Marjorie Fiske, and Patricia Kendall. ``Propósitos y
criterios de la entrevista focalizada.'' \emph{EMPIRIA: Revista de
Metodología de Ciencias Sociales}, no. 1 (1998):
215--27. \url{https://doi.org/10.5944/empiria.1.1998.740}.

Motta, Luis Gonzaga. ``Costa Rica: Seis años después.'' \emph{Chasqui,}
no. 3 (1982): 14--19.

Rodríguez, Clemencia, Claudia Magallanes Blanco, Amparo Marroquín
Parducci, and Omar Rincón. \emph{Mujeres de la comunicación}. Bogotá:
FES Comunicación, 2020.

Sánchez Narvarte, Emilio. ``Antonio Pasquali y las políticas de
comunicación en Venezuela (1974--1979).'' \emph{Avatares}, no. 19
(2020): 1--18.

Sandoval Arenas, Vania, Rigliana Portugal Escóbar, and Sandra Villegas
Taborga. \emph{Mujeres de la comunicación: Bolivia}. La Paz, Bolivia:
FES Comunicación, 2022.

Simonson, Peter, David Park, and Jefferson Pooley.
``Exclusions/\\\hspace{.25in}Exclusiones: The Role for History in the Field's
Reckoning.'' \emph{History of Media Studies}, no. 2 (2022).
\url{https://doi.org/10.32376/d895a0ea.cb32b735}.

Simonson, Peter, Jefferson Pooley, and David Park. ``The History of
Communication Studies across the Americas: A View from the United
States.'' \emph{MATRIZes} 17, no. 3 (2023): 189--216.
\url{https://doi.org/10.11606/issn.1982-8160.v17i3p189-216}.

Torrico Villanueva, Erick. \emph{Abordajes y períodos de la teoría de la
comunicación.} Bogotá: Norma, 2004.

Vassallo de Lopes, Maria Immacolata, and Raul Fuentes Navarro.
"Histórias da internacionalização do campo de estudos da comunicação."
\emph{MATRIZes} 17, no. 3 (2023):
5--16. \url{https://doi.org/10.11606/issn.1982-8160.v17i3p5-16}.



\end{hangparas}


\end{document}