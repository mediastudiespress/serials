% see the original template for more detail about bibliography, tables, etc: https://www.overleaf.com/latex/templates/handout-design-inspired-by-edward-tufte/dtsbhhkvghzz

\documentclass{tufte-handout}

%\geometry{showframe}% for debugging purposes -- displays the margins

\usepackage{amsmath}

\usepackage{hyperref}

\usepackage{fancyhdr}

\usepackage{hanging}

\hypersetup{colorlinks=true,allcolors=[RGB]{97,15,11}}

\fancyfoot[L]{\emph{History of Media Studies}, vol. 4, 2024}


% Set up the images/graphics package
\usepackage{graphicx}
\setkeys{Gin}{width=\linewidth,totalheight=\textheight,keepaspectratio}
\graphicspath{{graphics/}}

\title[Review of Hristova, Dorsten, \& Stabile]{\emph{The Ghost Reader: Recovering Women’s Contributions to Media Studies}} % longtitle shouldn't be necessary

% The following package makes prettier tables.  We're all about the bling!
\usepackage{booktabs}

% The units package provides nice, non-stacked fractions and better spacing
% for units.
\usepackage{units}

% The fancyvrb package lets us customize the formatting of verbatim
% environments.  We use a slightly smaller font.
\usepackage{fancyvrb}
\fvset{fontsize=\normalsize}

% Small sections of multiple columns
\usepackage{multicol}

% Provides paragraphs of dummy text
\usepackage{lipsum}

% These commands are used to pretty-print LaTeX commands
\newcommand{\doccmd}[1]{\texttt{\textbackslash#1}}% command name -- adds backslash automatically
\newcommand{\docopt}[1]{\ensuremath{\langle}\textrm{\textit{#1}}\ensuremath{\rangle}}% optional command argument
\newcommand{\docarg}[1]{\textrm{\textit{#1}}}% (required) command argument
\newenvironment{docspec}{\begin{quote}\noindent}{\end{quote}}% command specification environment
\newcommand{\docenv}[1]{\textsf{#1}}% environment name
\newcommand{\docpkg}[1]{\texttt{#1}}% package name
\newcommand{\doccls}[1]{\texttt{#1}}% document class name
\newcommand{\docclsopt}[1]{\texttt{#1}}% document class option name


\begin{document}

\begin{titlepage}

\begin{fullwidth}
\noindent\LARGE\emph{Book review
} \hspace{87mm}\includegraphics[height=1cm]{logo3.png}\\
\noindent\hrulefill\\
\vspace*{1em}
\noindent{\Huge{\emph{The Ghost Reader: Recovering Women’s\\\noindent Contributions to Media Studies}\par}}

\vspace*{1.5em}

\noindent\LARGE{Leonarda García-Jiménez}\par\marginnote{\emph{The Ghost Reader: Recovering Women’s Contributions to Media Studies}, reviewed by Leonarda García-Jiménez and Esperanza Herrero, \emph{History of Media Studies} 4 (2024), \href{https://doi.org/10.32376/d895a0ea.02cbe46d}{https://doi.org/ 10.32376/d895a0ea.02cbe46d}.} \vspace*{0.75em}
\vspace*{0.5em}
\noindent{{\large\emph{Universidad de Murcia}, \href{mailto:leonardagj@um.es address}{leonardagj@um.es address}\par}} \marginnote{\href{https://creativecommons.org/licenses/by-nc/4.0/}{\includegraphics[height=0.5cm]{by-nc.png}}}

\vspace*{0.75em} 

\noindent{\LARGE{Esperanza Herrero}\par}
\vspace*{0.5em}
\noindent{{\large\emph{Universidad de Murcia}, \href{mailto:mariaesperanza.herrero@um.es}{mariaesperanza.herrero@um.es}\par}}

% \vspace*{0.75em} % third author

% \noindent{\LARGE{<<author 3 name>>}\par}
\vspace*{0.5em}
% \noindent{{\large\emph{<<author 3 affiliation>>}, \href{mailto:<<author 3 email>>}{<<author 3 email>>}\par}}

\end{fullwidth}

\vspace*{1em}


\noindent Elena D. Hristova, Aimee-Marie Dorsten, and Carol A.
Stabile, eds. \emph{The Ghost Reader: Recovering
Women's Contributions to Media Studies.} 256 pp.
London: Goldsmiths Press, 2024. \$30 (paperback).

\vspace{1em}

\newthought{One of the main} outcomes of the current wave of feminism is the growing
effort to recover historical female referents in many different areas of
society. This is also happening in the field of communication. This
feminist reconstruction of communication's intellectual history is based
on a fundamental conviction: gender has been a constitutive element of
communication studies from its very beginning.\footnote{Karen Lee
  Ashcraft and Peter Simonson, ``Gender, Work, and the History of
  Communication Research,'' in \emph{The International History of
  Communication Study}, ed. Peter Simonson and David W. Park (New York:
  Routledge, 2015); Sue Curry Jansen, ``The Future Is Not What It Used
  to Be: Gender, History and Communication Studies,''
  \emph{Communication Theory} 3, no. 2 (1993).} The book \emph{Women in
Communication: A Biographical Sourcebook} was a pioneering act in this
line of research.\footnote{Nancy Signorielli, \emph{Women in
  Communication: A Biographical Sourcebook} (Westport, CT: Greenwood
  Press, 1996).} Ever since it was published, many other approaches have
worked towards reclaiming women as an essential piece in the foundation
of communication studies,  especially regarding European and North
American histories.\footnote{See, e.g., Christian Fleck, ``Lazarsfeld's
  Wives or What Happened to Women Sociologists in the Twentieth
  Century,'' \emph{International Review of Sociology} 31, no. 1 (2021);
  Aimee-Marie Dorsten, ``Thinking Dirty: Digging up Three Founding
  Matriarchs of Communication Studies,'' \emph{Communication Theory} 22,
  no. 1 (2012); Leonarda García-Jiménez, ``Female Contributions to
  Communication Theories: A Teaching and Scientific Proposal,''
  \emph{Anàlisi} 65 (2021).} Of particular note is the recent feminist
revival that has also emerged in Latin America.\textsuperscript{4}

  This body of research points to an obvious but often ignored fact: women
have always been present wherever communication and media studies have
been developed, even if most of them later became ghosts or, at best,
simple footnotes. The narrow historiography of the field has erased
their names and, in return, constructed a very masculinized history in
which women have no place.\textsuperscript{5} In
response, a new, valuable reference work has been added to the field: \emph{The Ghost Reader: Recovering Women\textquotesingle s Contributions
to Media Studies} is a book

\enlargethispage{2\baselineskip}

\vspace*{2em}

\noindent{\emph{History of Media Studies}, vol. 4, 2024}


 \end{titlepage}

% \vspace*{2em} | to use if abstract spills over




\noindent that\marginnote{\textsuperscript{4} See, e.g.,
  Clemencia Rodríguez, Claudia Magallanes, Amparo Marroquín, and Omar
  Rincón, \emph{Mujeres de la comunicación} (Berlin: FES Comunicación,
  2021); Yamila Heram and Santiago Gándara, ``Visibilidad y
  reconocimiento a las mujeres pioneras del campo comunicacional
  Latinoamericano,'' \emph{Revista Mediterránea de Comunicación} 12, no.
  2 (2021).} claims\marginnote{\textsuperscript{5}\setcounter{footnote}{5} Dorsten, ``Thinking Dirty.''} that the intellectual history of
communication studies can no longer be told without addressing women
scholars. 
The authors draw on Rebecca Solnit's reflections on ghost
libraries to address the ``ghosts in our canon'' (1--3) those
individuals---those women---who contributed to the construction of media
studies from the margins, excluded or relegated. The ghosts they refer
to are those ever-present but invisible figures, those who often
embodied different traditions of research, and whose loss is nothing
less than a narrowing and homogenization of communication studies
scholarship. Our field's tradition needs to be contested, they claim, by
un-ghosting the stories of many of these forgotten women.

Framed by meta-analysis and feminist epistemologies, the book
reconstructs the biographical profiles and recovers the work of eighteen
women who worked in Europe and North America between the 1930s and
1950s. These were the years in which the field of communication and
media studies began, a field that was not only shaped by the success
stories of its ``founding fathers,'' but also by the ambivalent stories
of success and suffering of these intelligent and strong women---some of
them immigrants---who lived and developed their careers in the turbulent
twentieth century, a century that was particularly hard for women
researchers who did not want to become ghosts confined in private
spaces, but rather wanted to reclaim and regain their place in the
public sphere and in the labor market.

The book is made up of eighteen chapters (with fifteen different
authors) in which some of the stories of pioneering female researchers
in the North American and European tradition are recovered. These are
eighteen women who deserve a place of their own in our intellectual
history; they are Gretel K. Adorno, Violet E. Lavine, Marjorie E. Fiske
Lissance Löwenthal, Shirley G. Du Bois, Herta Herzog, Mae Huettig
Churchill, Marie Jahoda, Romana Javitz, Claudia Jones, Dorothy B. Jones,
Patricia L. Kendall, Eleanor Leacock, Helen M. Lynd, Hortense
Powdermaker, Jeanette S. Smith, Lisa Sergio, Fredi Washington, and Gene
Weltfish. All of them studied communication, or aspects relevant to
understanding communication, while simultaneously inhabiting a hostile
academic environment, much more hostile to women than most professional
spaces, activist circles, or even the media.

In this book, each of these eighteen pioneers of communication research
is recovered from a double perspective: first, the authors propose a
biographical-experiential profile of each of the women; then, an excerpt
of their work is included. We believe that this twofold approach is
appropriate, especially because the recovery of intellectual figures
cannot be done without considering life stories, giving that knowledge
is also a personal and lived experience. Ultimately, the
biographical-experiential approach recovers the personal stories that
are often key for understanding an author's work. Can we fully
understand Herta Herzog's (chapter 6) contribution to audience analysis,
or Mae Huettig Churchill's (chapter 7) critical approach to the US film
industry without understanding the circumstances of their personal
lives? We do not think so, hence the importance of placing each of these
women's works in their own specific and biographical context. For
Herzog's ``intellectual curiosity and analytical bears,'' as Elana
Levine states on page 67, underlie her long research career (e.g., her
study on anti-Semitism cited in the chapter was published just a few
years before her death). Something similar happens with Mae Huettig
Churchill and the rest of the women included in the book. Huettig
Churchill, the daughter of Russian émigré anarchist parents, was also a
person harassed by the FBI for her membership in the Communist Party.
Research and researcher are but one and the same, so knowing her
personal circumstances helps us understand her critique of the low
quality of Hollywood cinema and the threat posed by the ``maze of
intricate relationships'' behind the film industry (including
``distribution and exhibition''), as Aimee-Marie Dorsten points out
(77).

At the same time, the selection of original works authored by women that
is included in each chapter functions as an invitation to read them and
give them a voice. It is them speaking directly to us, after we have
been introduced to their personal lives and struggles. At first glance,
selecting a few pages from an entire career seems like a titanic and
enormously difficult editorial task. We believe there is an unavoidable
risk of oversimplifying an author\textquotesingle s work. However,
\emph{The Ghost Reader: Recovering Women\textquotesingle s Contributions
to Media Studies} rises to the challenge, and we find a brilliant
selection of texts to offer as a first step in getting to know the
authors and their contributions to early media studies. We sincerely
believe that reading a few pages of ``What Do We Really Know about
Daytime Serial Listeners?'' (67--72), one of Herta
Herzog\textquotesingle s most emblematic works, or \emph{Economic
Control of the Motion Picture Industry} (78--83), one of Mae Huettig
Churchill's most important contributions, will make the reader want to
turn to the original sources and continue the dialogue with these
fascinating researchers. If this does not happen, there is a risk that
the ghost will continue to hover over our heads.

The result is not an ordinary book. Rather, it is a sourcebook: a book
designed to be consulted by professors, students, and researchers. In
general, a book written for anyone and everyone who wants to pluralize
their approaches to communication research through women's voices.

\newpage The eighteen European and North American women recovered by \emph{The
Ghost Reader: Recovering Women\textquotesingle s Contributions to Media
Studies} were present in the early moments of the field but were
eventually consciously or unconsciously erased by an androcentric field.
Thanks to this book, these brave and resilient women will no longer be
ghostly shadows in the history of communication and media studies.




\section{Bibliography}\label{bibliography}

\begin{hangparas}{.25in}{1} 



Ashcraft, Karen Lee, and Peter Simonson. ``Gender, Work, and the History
of Communication Research.'' In \emph{The International History of
Communication} Study, edited by Peter Simonson and David W. Park,
47--68. London: Routledge, 2015.

Dorsten, Aimee-Marie. ``Thinking Dirty: Digging up Three Founding
Matriarchs of Communication Studies.'' \emph{Communication Theory} 22,
no. 1 (2012): 25--47.
\url{https://doi.org/10.1111/j.1468-2885.2011.01398}.

Fleck, Christian. ``Lazarsfeld's Wives or What Happened to Women
Sociologists in the Twentieth Century.'' \emph{International Review of
Sociology} 31, no. 1 (2021): 49--71.
\url{https://doi.org/10.1080/03906701.2021.1926672}.

García-Jiménez, Leonarda. ``Female Contributions to Communication
Theories: A Teaching and Scientific Proposal.'' \emph{Anàlisi} 65
(2021): 121--35. \url{https://doi.org/10.5565/rev/analisi.3327}.

Heram, Yamila, and Santiago Gándara. ``Visibilidad y reconocimiento a
las mujeres pioneras del campo comunicacional latinoamericano: Un
análisis de la trayectoria de Mabel Piccini.'' \emph{Revista
Mediterránea de Comunicación} 12, no. 2 (2021): 65--75.
\url{https://doi.org/10.14198/MEDCOM.19151}.

Jansen, Sue Curry. ``The Future Is Not What It Used to Be: Gender,
History and Communication Studies.'' \emph{Communication Theory} 3, no.
2 (1993): 136--48.
\url{https://doi.org/10.1111/j.1468-2885.1993.tb00063.x}.

Rodríguez, Clemencia, Claudia Magallanes, Amparo Marroquín, and Omar
Rincón. \emph{Mujeres de la comunicación}. Berlin: FES Comunicación,
2021.

Signorielli, Nancy. \emph{Women in Communication: A Biographical
Sourcebook}. Westport, CT: Greenwood Publishers, 1996.



\end{hangparas}


\end{document}