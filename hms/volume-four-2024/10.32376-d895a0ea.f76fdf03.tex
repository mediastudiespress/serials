% see the original template for more detail about bibliography, tables, etc: https://www.overleaf.com/latex/templates/handout-design-inspired-by-edward-tufte/dtsbhhkvghzz

\documentclass{tufte-handout}

%\geometry{showframe}% for debugging purposes -- displays the margins

\usepackage{amsmath}

\usepackage{hyperref}

\usepackage{fancyhdr}

\usepackage{hanging}

\hypersetup{colorlinks=true,allcolors=[RGB]{97,15,11}}

\fancyfoot[L]{\emph{History of Media Studies}, vol. 4, 2024}


% Set up the images/graphics package
\usepackage{graphicx}
\setkeys{Gin}{width=\linewidth,totalheight=\textheight,keepaspectratio}
\graphicspath{{graphics/}}

\title[Borderline Cases]{Borderline Cases: Crossing Borders in Canadian Communication Studies, 1960s–1980s} % longtitle shouldn't be necessary

% The following package makes prettier tables.  We're all about the bling!
\usepackage{booktabs}

% The units package provides nice, non-stacked fractions and better spacing
% for units.
\usepackage{units}

% The fancyvrb package lets us customize the formatting of verbatim
% environments.  We use a slightly smaller font.
\usepackage{fancyvrb}
\fvset{fontsize=\normalsize}

% Small sections of multiple columns
\usepackage{multicol}

% Provides paragraphs of dummy text
\usepackage{lipsum}

% These commands are used to pretty-print LaTeX commands
\newcommand{\doccmd}[1]{\texttt{\textbackslash#1}}% command name -- adds backslash automatically
\newcommand{\docopt}[1]{\ensuremath{\langle}\textrm{\textit{#1}}\ensuremath{\rangle}}% optional command argument
\newcommand{\docarg}[1]{\textrm{\textit{#1}}}% (required) command argument
\newenvironment{docspec}{\begin{quote}\noindent}{\end{quote}}% command specification environment
\newcommand{\docenv}[1]{\textsf{#1}}% environment name
\newcommand{\docpkg}[1]{\texttt{#1}}% package name
\newcommand{\doccls}[1]{\texttt{#1}}% document class name
\newcommand{\docclsopt}[1]{\texttt{#1}}% document class option name


\begin{document}

\begin{titlepage}

\begin{fullwidth}
\noindent\Large\emph{History of Communication Studies across the Americas
} \hspace{18mm}\includegraphics[height=1cm]{logo3.png}\\
\noindent\hrulefill\\
\vspace*{1em}
\noindent{\Huge{Borderline Cases: Crossing Borders in\\\noindent Canadian Communication Studies,\\\noindent 1960s–1980s\par}}

\vspace*{1.5em}

\noindent\LARGE{Michael Darroch} \href{https://orcid.org/https://www.orcid.org/0009-0001-4820-8746}{\includegraphics[height=0.5cm]{orcid.png}}\par\marginnote{\emph{Michael Darroch, ``Borderline Cases: Crossing Borders in Canadian Communication Studies, 1960s–1980s,'' \emph{History of Media Studies} 4 (2024), \href{https://doi.org/10.32376/d895a0ea.f76fdf03}{https://doi.org/ 10.32376/d895a0ea.f76fdf03}.} \vspace*{0.75em}}
\vspace*{0.5em}
\noindent{{\large\emph{York University}, \href{mailto:mdarroch@yorku.ca}{mdarroch@yorku.ca}\par}} \marginnote{\href{https://creativecommons.org/licenses/by-nc/4.0/}{\includegraphics[height=0.5cm]{by-nc.png}}}

% \vspace*{0.75em} % second author

% \noindent{\LARGE{<<author 2 name>>}\par}
% \vspace*{0.5em}
% \noindent{{\large\emph{<<author 2 affiliation>>}, \href{mailto:<<author 2 email>>}{<<author 2 email>>}\par}}

% \vspace*{0.75em} % third author

% \noindent{\LARGE{<<author 3 name>>}\par}
% \vspace*{0.5em}
% \noindent{{\large\emph{<<author 3 affiliation>>}, \href{mailto:<<author 3 email>>}{<<author 3 email>>}\par}}

\end{fullwidth}

\vspace*{1em}


\hypertarget{abstract}{%
\section{Abstract}\label{abstract}}

Communication studies has been shaped by the affective contexts of
border cultures and bordering practices. Human experiences of living in
borderlands, of migrating across borders, and the concomitant bridging
of cultural and linguistic contexts have influenced theories, metaphors,
and methods within communication and media studies. The multicultural
and multilingual contexts of Canada and Quebec provide an important case
study for a history of communication and media studies across the
Americas.~This paper explores the history of these fields in the
Canadian context through the lens of bordering practices: Canada--US
relations, but also the cultural and linguistic borderlines between
English Canada, Quebec, and beyond. I develop these themes by exploring
specific cases and considerations of early Canadian scholars and
programs, as well as key organizations and publications. If 1950s and
1960s new media, including primarily television, promised to
de-emphasize official borders, where circuits of media accessibility
began to knit US and Canadian cultural practices together, we should not
neglect the forces of national concerns on media industries or use.~My
goal in this paper is not to contribute another institutional study, but
rather to offer an interpretative lens on intersecting histories of the
field in the context of Canadian pluralisms.

\enlargethispage{2\baselineskip}

\vspace*{4em}

\noindent{\emph{History of Media Studies}, vol. 4, 2024}

\hypertarget{resumen}{%
\section{Resumen}\label{resumen}}

Los estudios en comunicación se han visto configurados por los contextos
afectivos de las culturas fronterizas y las prácticas
\emph{fronterizantes}. ~Las experiencias humanas de vivir en las zonas
fronterizas y de migrar a través de fronteras, así como los puentes que
se tienden entre contextos culturales y lingüísticos, han incidido en
teorías, metáforas y métodos dentro de los estudios de comunicación y
medios. Los contextos multiculturales y multilingües de Quebec nos
proporcionan un importante caso de estudio para una historia de los
estudios de comunicación y medios en las Américas. En este trabajo se
explora la historia de estos campos en el contexto canadiense, con la
mirada puesta en las prácticas \emph{fronterizantes:} las relaciones
entre Canadá y Estados Unidos, pero también las fronteras culturales y
lingüísticas entre la Canadá anglófona, Quebec y más allá. Desarrollo
estos temas explorando casos concretos, teniendo en cuenta las
consideraciones específicas de algunos de los primeros investigadores y
programas de estudio canadienses, así como analizando organizaciones y
publicaciones clave. Si bien los nuevos medios de las décadas de los 50
y 60, sobre todo la televisión, prometían desdibujar las fronteras
oficiales, donde los circuitos de acceso mediático empezaron a fundir
las prácticas culturales de Estados Unidos y Canadá, no debemos
subestimar las fuerzas de los intereses nacionales y su impacto en las
industrias mediáticas y los usos de los medios. Mi objetivo con este
trabajo no es generar otro estudio institucional, sino aportar un
enfoque para interpretar las historias entrecruzadas del campo en el
contexto de los pluralismos canadienses.




 \end{titlepage}

% \vspace*{2em} | to use if abstract spills over

\newthought{The multicultural and} multilingual contexts of Canada and Quebec provide
an important case study for a history of communication and media studies
across the Americas.\footnote{Research for this paper was supported by
  an Insight Grant (2021--2026) from the Social Sciences and Humanities
  Research Council of Canada: \emph{Distributed Networks: Media
  Archaeologies of Educational TV and Communication Studies in Canada,
  1945--1975}.} This paper explores the history of these fields in the
Canadian context through the lens of bordering practices: Canada--US
relations, but also the cultural and linguistic borderlines between
English Canada, Quebec, and beyond. Robert Holub's 1992 study of the
treatment of reception theory, poststructuralism, and deconstruction in
Germany and the United States is instructive for other historiographies
of intellectual traditions that bridge cultural and linguistic divides.
Holub argued that ``although theory presents itself as abstract and
applicable without regard to temporal and geographical boundaries, its
appropriation and understanding were evidently bound to
context.''\footnote{Robert C. Holub, \emph{Crossing Borders: Reception
  Theory, Poststructuralism, Deconstruction} (Madison: University of
  Wisconsin Press, 1992), ix.} According to Holub, ``what matters most
in the appropriation of a theory from a foreign country is how it fits
into an already established constellation in the importing country'' and
that ``the vicissitudes and preferences in theoretical endeavour in the
United States and Germany cannot be adequately understood without
reference to a notion of cross-cultural contextualization.''\footnote{Holub,
  \emph{Crossing Borders}, ix.} The history of communication studies has
long been shaped by such affective contexts of border cultures and
bordering practices. Human experiences of living in borderlands, of
migrating across borders, and the concomitant bridging of cultural and
linguistic contexts have influenced theories, metaphors, and methods
within communication and media studies. This is not surprising since
human experiences of living in borderlands regions or crossing borders
can be both stimulating and terrifying. Communication practices and
material media forms both absorb and circulate the narratives we carry
from such experiences.

Scholars of the history of communication and media study have begun to
address the twentieth-century transnational origins and legacies of
these fields.\footnote{Stefanie Averbeck-Lietz,
  \emph{Kommunikationswissenschaft im internationalen Vergleich:
  Transnationale Perspektiven (Medien, Kultur, Kommunikation)}
  (Wiesbaden, Germany: Springer, 2017); Norm Friesen, ed., \emph{Media
  Transatlantic: Developments in Media and Communication Studies between
  North American and German-Speaking Europe} (Vienna: Springer, 2016);
  Maria Löblich and Stefanie Averbeck-Lietz, ``The Transnational Flow of
  Ideas and \emph{Histoire Croisée} with Attention to the Cases of
  France and Germany,'' in \emph{The International History of
  Communication Study}, ed. Peter Simonson and David W. Park (New York:
  Routledge, 2016), 25--46; David W. Park and Jefferson Pooley, eds.,
  \emph{The History of Media and Communication}} Have transnational flows of intellectual thought influenced
the idea of bordering as a method of theory and practice? In this paper,
I suggest that we have yet to recognize fully the role played by what my
colleague Lee Rodney has called a ``frontier imagination'' in the North
American context.\textsuperscript{5}
For this contribution to \emph{Communication Studies Across the
Americas}, I develop these themes by exploring specific cases and
considerations of early Canadian scholars and programs, as well as key
organizations and publications. If 1950s and 1960s new media including
primarily television promised to usher in a period of de-emphasizing
official borders, where circuits of media accessibility began
increasingly to knit US and Canadian cultural practices together, we
should not neglect the forces of national concerns on media industries
or use. The networked and mediatized characteristics of transborder,
diasporic, and multilingual\marginnote{\emph{Research: Contested
  Memories} (New York: Peter Lang, 2008); Peter Simonson and John Durham
  Peters, ``Communication and Media Studies: History to 1968,'' in
  \emph{International Encyclopedia of Communication}, ed. Wolfgang
  Donsbach (2014); Peter Simonson and David W. Park, eds., \emph{The
  International History of Communication Study} (New York: Routledge,
  2016).} environments\marginnote{\textsuperscript{5}\setcounter{footnote}{5} Lee Rodney, \emph{Looking Beyond Borderlines:
  North America's Frontier Imagination} (New York: Routledge, 2017).} of contemporary Canada reach
beyond the image of English and French Canada as ``two solitudes'' and
reflect more importantly the multicultural landscape of Canada's urban
environments.

After early ``echo chambers'' of communication thought and studies took
shape in the 1940s and 1950s and first university programs were
established in the 1960s and 1970s, a number of historiographies have
been published to ``take stock'' of the field in Canada and Quebec.
These include \emph{Studies in Canadian Communications}, edited by
Donald Theall and Gertrude Robinson and published by the McGill
Programme in Communications;\footnote{Donald Theall and Gertrude
  Robinson, eds., \emph{Studies in Canadian Communications} (Montreal:
  Graduate Program in Communications, McGill University, 1975).} and the
volume \emph{Communication Studies in Canada}, edited by Liora Salter at
Simon Fraser University in Vancouver, a bilingual collection of articles
selected from the founding conference of the Canadian Communication
Association held in Montreal in 1980.\footnote{Liora Salter, ed.,
  \emph{Communication Studies in Canada} (Toronto: Butterworth, 1981).}
Rowland Lorimer and Jean McNulty's \emph{Mass Communication in Canada}
was published in 1987, the first English textbook to introduce students
to themes in communication and media studies from a Canadian
perspective.\footnote{Rowland Lorimer and Jean McNulty, \emph{Mass
  Communication in Canada} (Toronto: McClelland and Stewart, 1987). Now
  in its ninth edition as Mike Gasher, David Skinner, and Natalie
  Coulter, \emph{Media and Communication in Canada: Networks, Culture,
  Technology, Audience} (Toronto: Oxford University Press Canada, 2020).}
Further reflections on the field were published in the \emph{Canadian
Journal of Communication} in 1988 and a special issue dedicated to the
theme in 2000. Studies taking stock of the field in Quebec were also
published in the 1980s in journals including \emph{Communication
Information.}\footnote{Serge Proulx, ``Les communications: Vers un
  nouveau savoir savant?'' \emph{Recherches Sociographiques} 20, no. 1
  (1979): 103--17; Jean-Guy Lacroix and Benoit Lévesque, ``L'émergence
  et l'institutionnalisation de la recherche en communication au
  Quebec,'' \emph{Communication Information} 7, no. 2 (1985); Jean-Guy
  Lacroix and Benoit Lévesque, ``Principaux thèmes et courants
  théoriques dans la littérature scientifique en communication au
  Quebec,'' \emph{Communication Information} 7, no. 3 (1985).} In
Quebec, scholars involved with the founding of communications studies
programs at Francophone universities, such as Roger de la Garde and
Gaëtan Tremblay, have published numerous accounts of institutional
histories, while a more recent generation have increasingly focused on
microhistories of communication studies.\footnote{François Yelle,
  ``Étude de la littérature réflexive de la recherche universitaire
  québécoise en communication médiatique'' (PhD diss., University of
  Montreal, 2004); François Yelle, ``L'histoire des études en
  communication au Québec et le dogme de la rupture, ou l'héritage peu
  célébré des intellectuels canadiens-français des années 1940 et
  1950,'' \emph{Revista Eptic} 19, no. 1 (2017).} These studies have
already laid the foundation for a well-documented comparative
historiography of the field, including institutional histories of first
programs and the key commitments of scholars who founded them. They
emphasize the bridging moments across intellectual traditions in
rhetoric, literary studies, speech communication, journalism, and the
arts broadly, with political economy, sociology, and
psychology.\textsuperscript{11}

My goal in this paper is not to contribute yet another institutional
study, but rather to offer an interpretative lens on intersecting
histories of the field in the context of Canadian pluralisms. To
undertake this study, I first wish to acknowledge my own position as a
white English-speaking Canadian scholar. As a native of Toronto,
Ontario, my personal experience derives from undergraduate and graduate
studies in second languages and literatures, theater and creative arts,
and communication and media studies in Montreal, Quebec from the
mid-1990s to the early 2000s (encompassing Quebec's second referendum
for independence from Canada in 1995), as well as two separate years
studying in Germany. I also spent twelve years, from 2008\marginnote{\textsuperscript{11}\setcounter{footnote}{11} Robert E. Babe, \emph{Canadian Communication
  Thought: Ten Foundational Writers} (Toronto: University of Toronto
  Press, 2000); Roger de la Garde, ``The 1987 Southam Lecture: Mr.
  Innis, Is There Life after the \textquotesingle American
  Empire\textquotesingle?,'' \emph{Canadian Journal of Communication}
  13, no. 5 (1988); Roger de la Garde and François Yelle, ``Coming of
  Age: Communication Studies in Quebec,'' in \emph{Mediascapes: New
  Patterns in Canadian Communication}, ed. Paul Attalah and Leslie Regan
  Shade (Toronto: Thomson-Nelson, 2002); Gertrude Robinson,
  ``Remembering Our Past: Reconstructing the Field of Communication
  Studies,'' \emph{Canadian Journal of Communication} 25, no. 1 (2000);
  Liora Salter, ``Taking Stock: Communication Studies in 1987,''
  \emph{Canadian Journal of Communication} 13, no. 5 (1988); Eugene D.
  Tate, Andrew Osler, Gregory Fouts, and Arthur Siegel, ``The Beginnings
  of Communication Studies in Canada: Remembering and Narrating the
  Past,'' \emph{Canadian Journal of Communication} 25, no. 1 (2000);
  Michael Dorland, ``Knowledge Matters: The Institutionalization of
  Communication Studies in Canada,'' in \emph{Mediascapes: New Patterns
  in Canadian Communication}, ed. Paul Attalah and Leslie Regan Shade
  (Toronto: Thomson-Nelson, 2002); Gaëtan Tremblay, ``Journey of a
  Researcher,'' \emph{Canadian Journal of Communication} 39, no. 1
  (2014); Gregory Taylor and Ray op'tLand, ``Communication Research and
  Teaching in Canada,'' \emph{Publizistik} 64 (2019).} to 2020, as a
faculty member at the University of Windsor, a city located on the
Canadian side of the US-Canada border, opposite the city of Detroit,
Michigan. Here, my interest in multilingualism in cultural and artistic
industries began to influence a growing focus on comparative cultural
borderlands studies, inspired by the unusual cross-border urbanized
context of Windsor-Detroit. Given Detroit's prominence in media
industries, particularly its early facilitation of US cable television
to a breadth of Canadian centres, it is perhaps no coincidence that the
University of Windsor, incorporated in 1962, proposed one of Canada's
earliest programs in communication studies, a Department of
Communication Arts, as early as 1969. It is impossible to live or study
in Windsor without reflecting on the border's influence on a near-daily
basis. These experiences have shaped my own relationship to the history
of communication and media studies. In this contribution, it is not my
goal to develop a single, overarching history of Canadian communication
and media studies, but rather to highlight several intersecting
histories in which taking a leap across cultural divisions was central
to ways in which the field developed in specific locations. The paper
first reviews metaphors of crossing borders and translating cultures as
part of the lived experiences of influential communications and media
scholars. It next considers notions of a ``frontier imagination'' and a
``bi-focal habit of vision'' that drew upon Canada's cultural and
linguistic pluralisms as early communication and media thought took
shape. The paper then dives deeper into the rich disciplinary and
discursive border crossings that accompanied the establishment of
pedagogical programs, institutional frameworks, and sites of research
dissemination. Canada's ``borderline case'' might ultimately be derived
from multi- rather than bi-focal habits of understanding, experience,
and indeed, communication.

\hypertarget{media-metaphors-crossing-borders-and-translating-cultures}{%
\section{Media Metaphors: Crossing Borders and Translating
Cultures}\label{media-metaphors-crossing-borders-and-translating-cultures}}

Crossing borders, translating between cultures, and living in ``margi-\\\noindent
nal'' contexts have long influenced metaphors for media and for
acts of communication.\footnote{Rainer Guldin, ``From Transportation to
  Transformation: On the Use of the Metaphor of Translation within Media
  and Communication Theory,'' \emph{Global Media Journal: Canadian
  Edition} 5, no. 1 (2012).} In the twentieth and twenty-first
centuries, experiences of international conflict and friction in border
zones, historic flows of migration, and the affective implications of
crossing borders and cultural divides have featured prominently in the
research and writings of key theorists. Among these central figures are
scholars such as Harold Innis, who researched cross-border continental
patterns in North American staples industries, and later the history of
communication and media in terms of imperial legacies across space and
time.\footnote{Harold Adams Innis, \emph{Empire and Communications}
  (Oxford: Oxford University Press, 1950); Harold Adams Innis, \emph{The
  Bias of Communication} (Toronto: University of Toronto Press, 1951).}
Innis's studies of empires were driven in part by his personal
experiences in the first World War. The post--World War II European
``intellectual migration'' of cultural scholars to the United States
included Paul Lazarsfeld, Herta Herzog, and members of the Frankfurt
School.\footnote{Laura Fermi, \emph{Illustrious Immigrants: The
  Intellectual Migration from Europe, 1930--1941} (Chicago: University
  of Chicago Press, 1968); Donald Fleming and Bernard Bailyn, eds.,
  \emph{The Intellectual Migration: Europe and America, 1930--1960}
  (Cambridge, MA: Belknap Press of Harvard University Press, 1969);
  Martin Jay, \emph{Permanent Exiles: Essays on the Intellectual
  Migration from Germany to America} (New York: Columbia University
  Press, 1985); Gertrude Robinson, ``The Katz/Lowenthal Encounter: An
  Episode in the Creation of \emph{Personal Influence},'' \emph{Annals
  of the American Academy of Political and Social Science} 608 (2006).}
The profound influence of these scholars can only be considered through
the lens of their existential border crossings, and the imperative to
operate in a second language. Writing about Canada's particular
``marginal'' condition, Donald Theall once noted that the aesthetic
interpretative character of the work of Innis and Marshall McLuhan could
be compared with the dialectic procedures and aesthetic interests of
``the philosophers of the Frankfurt School, who themselves occupied a
marginal position to the United States by continuing to write in German
after their emigration.''\footnote{Donald Theall, ``Communication Theory
  and the Marginal Culture: The Socio-Aesthetic Dimensions of
  Communication Study,'' in \emph{Studies in Canadian Communications},
  ed. Gertrude Robinson and Donald Theall (Montreal: Graduate Program in
  Communications, McGill University, 1975), 19.} Gertrude Robinson (who
along with Theall founded the MA {[}1973{]} and PhD {[}1976{]} programs
in Communication Studies at McGill University) has contributed poignant
remarks on the field's history. Tracing the work of Elihu Katz and Leo
Lowenthal, Robinson (herself originally from Germany) characterized
these scholars as ``border travelers,'' contributing to the
``geographical transfer of ideas'' and also ``setting up new research
institutions, which would utilize their scholarly
expertise.''\footnote{Robinson, ``The Katz/Lowenthal Encounter,'' 94.}
In the 1960s, McLuhan drew on a frontier imagination to build his many
neologisms such as the ``global village'' or ``centres without
margins,'' as well as his arguments about ``media as translators''
across cultures and borderline conditions as interfaces, or intervals of
cultural resonance. In 1980s Germany, particularly in Berlin, it is
arguable that the experience of living at the edge of the Iron Curtain
contributed to new theories of media and the materialities of
communication, including Friedrich Kittler's emphasis on media
materialities in terms of the translatability of information. Perhaps
the most important communications and media theorist of the twentieth
century to have been influenced by his personal history of migration was
Vilém Flusser. Having fled his multilingual home in Prague during the
period of Nazi Germany, Flusser landed in Brazil before producing an
exceptional array of multilinguistic writings on communication and
media, including his emphasis on ``nomadic thinking'' and the experience
of the migrant as a quintessential global citizen.\footnote{Vilém
  Flusser, \emph{Ende der Geschichte, Ende der Stadt?} (Vienna: Picus,
  1992); Vilém Flusser, \emph{Freedom of the Migrant,} trans. Kenneth
  Kronenberg, ed. Anke Finger (Urbana: University of Illinois Press,
  2003).} These are but a few examples of the central role played by
national, cultural, and linguistic border crossing in developing
intellectual and institutional traditions of theoretical and creative
communications and media research.

To pursue two of these examples, McLuhan's writings on the global
village can be juxtaposed with Flusser's thesis of an emergent
posthistorical telematic society. These two central contributors to
theories of communication and media wrote from vastly different vantage
points in the Americas, Canada and Brazil, two countries which, as
Alfred Braz has argued, occupy comparably peripheral statuses in
inter-American discourse.\footnote{Albert Braz, ``Outer America: Racial
  Hybridity and Canada's Peripheral Place in Inter-American Discourse,''
  in \emph{Canada and Its Americas: Transnational Navigations}, ed.
  Winfried Siemerling and Sarah Phillips Casteel (Montreal:
  McGill-Queen's University Press, 2010).} Reflecting their own
histories, McLuhan and Flusser each represent European-oriented
experiences of the Americas, yet they returned time and again to broad
metaphors of borderlines and translations as sites and methods of
interpenetrations between cultures and languages, spaces and cities,
senses, images, and codes, and diverse material media forms themselves.
As Flusser wrote in 1967, ``The problem of translation and
translatability takes on the cosmic dimensions of all existential
issues: it encompasses everything.''\footnote{Vilém Flusser, ``Essays,''
  in \emph{Writings}, ed. Andreas Ströhl, trans. Erik Eisel
  (Minneapolis: University of Minnesota Press, 2002),~194.} The
multilingual landscape of Flusser's writings in German, Czech,
Portuguese, French, and English (requiring his own body of work to be
translated in multiple directions in order for scholars of different
backgrounds to be able to trace the full trajectory of his thought) is
characterized by his personal experiences of displacement,
groundlessness, loss, migration, and unsettlement---first from fleeing
Nazi-occupied, multilingual Prague for the United Kingdom, and later
from landing in São Paulo, Brazil for some twenty years and then
returning to Europe. It was from these vantage points that he developed
theses on cultures and cities as relational fields, and of communication
and media as intricate and overlapping networks of intersubjective
relations. Flusser's commitment to the positive outcomes of what he
termed ``nomadic thinking'' and his capacity to move and write around
and between languages led to his theorization of modes of communication
and media forms as fundamentally translational and processual. His
theories of communication embrace dialogic encounters and translations
as a series of jumps or leaps from one language, experience, code, or
format to another. These theories can be contrasted with McLuhan's focus
on hybrid media, and on forms of cultural, perceptual, and sensual
interpenetrations.\footnote{Rainer Guldin, ``Die Zweite Unschuld:
  Heilsgeschichtliche und Eschatologische Perspektiven im Werk Vilém
  Flussers und Marshall McLuhans,'' \emph{Flusser Studies} 6 (2008);
  Michael Darroch, ``Medial Translations and Human Unsettlements:
  Planetary Urbanisms from McLuhan to Flusser,'' in \emph{Speaking
  Memory: How Translation Shapes Cities}, ed. Sherry Simon (Montreal:
  McGill-Queen's University Press, 2016).}

McLuhan's references to bordering practices and translation focus not on
crossing or overcoming linear borderlines or edges of media forms, but
rather deep intervals of resonance and cultural interpenetration. These
references can also be read in the context of his experiences moving as
a scholar across North America and the United Kingdom from the 1930s to
the 1970s, including sites such as Edmonton, Winnipeg, Cambridge, St.
Louis, Windsor-Detroit, Toronto, and New York. Such cross-border
experiences were ripe for thinking about Canada: the nation's
marginality to the United States, Britain, and continental Europe
arguably framed perceptions and scholarly outlooks across multiple
fields of study, positioning scholars such as Innis and McLuhan as
observers from just outside yet still involved with the maelstroms of
technological change. Canada's self-perceived marginality also
influenced its many institutions of nation-building and governance
through communicational structures and cultural technologies that became
closely aligned with theories of communication and media, as well as
fledgling programs of communication studies, in the mid-twentieth
century. According to Robinson, ``this `symbolic environment' was
constructed by royal commissions and publicly owned Crown corporations,
including the Canadian Broadcasting Corporation (CBC), the National Film
Board (NFB), the Canada Council, the Canadian Radio-television and
Telecommunications Commission (CRTC), and Téléfilm Canada, as well as
the statutory legislation which defines their mandates.''\footnote{Robinson,
  ``Remembering Our Past,'' 107.} Yet the thesis of Canada's
marginality, as Sheryl Hamilton has argued, makes many assumptions about
Canada's place in the world. Is it Canada's marginality, its
``edginess'' at the margins of Empire, or rather its deep cross-border
cultural interpenetrations with the United States and Europe, that drove
scholars to develop studies of communication?\footnote{Sheryl Hamilton,
  ``Considering Critical Communication Studies in Canada,'' in
  \emph{Mediascapes: New Patterns in Canadian Communication}, ed. Paul
  Attalah and Leslie Regan Shade (Toronto: Thomson-Nelson, 2002).}

\hypertarget{frontier-imaginations-cross-border-interpenetrations}{%
\section{Frontier Imaginations, Cross-Border
Interpenetrations}\label{frontier-imaginations-cross-border-interpenetrations}}

McLuhan's treatise on Canada's ``Borderline Case,'' which he first
proposed in 1967 during Canada's centennial celebrations, perhaps
captures the idea that Canada is a site of constant bordered conditions,
marginality, in-betweenness, and dualisms (as framed by author Hugh
MacLennan's famous 1945 novel \emph{The Two Solitudes}). Canada's
position between French and British colonial histories, between the
influence of European and growing US cultural hegemony, and its later
national characterization as simultaneously bilingual and multicultural,
have contributed to a sense of national ambiguity that has also shaped
the history of these fields. McLuhan's proposal characterizes Canada as
many-bordered: historically, geographically, culturally, linguistically,
and symbolically. In his 1975 essay, ``Communication Theory and the
Marginal Culture,'' Theall surmised that the ``communication theory that
arose in Canada . . . arose to a considerable extent as a strategy of
culture and consequently the theorists concerned themselves with
questions of a cultural nature and with a critique of the conflicting
demands of British, Continental and U.S. traditions.''\footnote{Theall,
  ``Communication Theory and the Marginal Culture,'' 10.} Similarly,
Arthur Kroker once argued that a discourse on technology is ``central to
the Canadian imagination'' because it is ``situated \emph{midway}
between the future of the New World and the past of European culture,
between the rapid unfolding of the `technological imperative' in
American empire and the classical origins of the technological dynamo in
European history.''\footnote{Arthur Kroker, \emph{Technology and the
  Canadian Mind: Innis, McLuhan, Grant} (Montreal: New World
  Perspectives, 1984). See also Maurice Charland, ``Technological
  Nationalism,'' \emph{Canadian Journal of Political and Social Theory}
  10, no. 1 (1996).} In Richard Cavell's reconsideration of McLuhan's
``Borderline Case'' essay, he notes that ``even if the cultural threat''
of US hegemony ``has receded . . . the issue of borderlines is still
important culturally and politically.''\footnote{Richard Cavell,
  ``McLuhan's `Borderline Case' Revisted,'' in \emph{Comment comparer le
  Canada avec les États-Unis aujourd'hui}, ed. Hélène Quanquin,
  Christine Lorre-Johnston, and Sandrine Ferré-Rode (Paris: Presses
  Sorbonne Nouvelle, 2009), para. 6.} In my own contributions, I have
argued that Toronto School media theorization in the 1950s---primarily
through McLuhan and Edmund Carpenter's production of the journal
\emph{Explorations} with their colleagues in Toronto---represented
cross-border and transnational intellectual entanglements through the
core research group members' commitment to interdisciplinarity and to
studying patterns that connect peoples and cultures across space and
time, a precedent for contemporary media studies that emphasize shared
methodologies, collaborative projects, experiments in research-creation,
and new critical pedagogies reflecting the changing shape of university
research cultures.\footnote{Michael Darroch, ``Bridging Urban and Media
  Studies: Jaqueline Tyrwhitt and the \emph{Explorations} Group,
  1951--1957,'' \emph{Canadian Journal of Communication} 33, no. 2
  (2008); Michael Darroch, ``The Toronto School: Cross-Border
  Encounters, Intellectual Entanglements,'' in \emph{The International
  History of Communication Studies}, ed. Peter Simonson and David Park
  (London: Routledge, 2016).}

Intertwined with these considerations is the history of communication
and media studies in Quebec, where we must recognize deep
interpenetrations of academic fields with English Canada, but also
arguably a gaze that reached further to anti-colonialist thought in
Latin America, South America, and North Africa. The earliest Canadian
English-language undergraduate and graduate studies programs in
communication studies were established in Quebec: a BA at Concordia
University (then Loyola College) in 1965, and an MA and PhD at McGill
University in 1973--1974 and 1976, respectively. These programs were
established within the bilingual context of Montreal, a city that has
become increasingly intercultural, multilingual, and diasporic since the
1960s. These initiatives paralleled other programs emerging in Ontario,
Saskatchewan, British Columbia, and later the Maritimes. Quebec
universities including Laval University (in Quebec City) and the
University of Montreal also initiated the first French-language programs
and departments in communication studies in the late 1960s and early
1970s.\footnote{University of Montreal, BA {[}1969{]}; Laval University,
  BA {[}1969{]}; University of Quebec at Montreal, BA, {[}1973{]};
  University of Montreal, MA {[}1973{]}.} The sheer number of university
programs in communication and media studies, and adjacent fields, that
emerged in and near Quebec in the 1960s and 1970s invites an important
historiographical examination of scholarly trends across these
political, cultural, and linguistic spaces. Intercultural experiences of
moving between cultural contexts and languages in part underpinned these
foundational programs. Founding program scholars, including Father John
O'Brien at Loyola, Gertrude Robinson at McGill, and James Taylor at the
University of Montreal, had completed doctoral studies in the United
States; others, including Line Ross at Laval and Gaëtan Tremblay at the
University of Quebec at Montreal, had pursued degrees in France.

Lee Rodney's thesis of North America's ``frontier imagination'' provides
a useful framework for considering multidimensional contexts of
communication studies. Rodney develops a critical reassessment of the
role played by the US-Canadian and US-Mexican borderlines, not as linear
boundaries dividing territorial spaces but rather through the historical
and contemporary perceptual imaginaries of these border environments and
the bordering logics of the twentieth and twenty-first centuries. We
must look beyond the idea of the borderline---that is, beyond official
and legal narratives about the boundaries that mark such national
divides, to reconsider how borders also encompass historical division,
political shifts, and racialized experiences of nation and space that
have had tangible effects on different communities. Indeed, as Rodney
and I argued in a research-creation initiative called \emph{Sensing
Borders}, borders are highly mediated and abstract archives of stories
and images that shape political imaginaries and
repercussions.\footnote{See Michael Darroch, Karen Engle, and Lee
  Rodney, ``Introduction: Sensing Borders,'' in ``Ressentir (les
  frontières)/Sensing (Borders),'' special issue,
  \emph{Intermédialités}, no. 34 (Autumn 2019).} McLuhan's thesis of
Canada itself as a ``borderline case'' remains pertinent here. His
metaphor that a border is not ``a connection but an interval of
resonance'' continues to evoke a compelling framework for understanding
cultural borderlands in the twenty-first century.\footnote{Marshall
  McLuhan, ``Canada: The Borderline Case,'' in \emph{The Canadian
  Imagination}, ed. David Staines (Cambridge, MA: Harvard University
  Press, 1977), 226.} McLuhan suggested that Canada's historical
position produced a frontier condition of identity and perspective, a
``space between two worlds'' distinguished not by linear modes of
thought but rather by the metaphors of interval and interface. Such
cultural intervals are both spatial and temporal, but they are not
necessarily apolitical, harmonious, or representative of cultural
hybridity in complacent terms. Rather, borders as intervals of resonance
are collective cultural spaces that also produce abrasions or
irritations within them, requiring the acknowledgement of cultural
differences and necessitating modes of co-existence---or alternatively,
igniting fears, crises, and violence. Canada in this reading is not just
\emph{a} borderline case, but \emph{the} quintessentially postmodern
case constituted by multiple borderlines, many sites of mutual
irritation articulated through language, culture, and communication
technologies. Jody Berland has claimed that the ``consciousness of the
border's arbitrary location'' has led to labeling Canada ``the world's
first postmodern country,'' a country that ``registers the prospect of
reconciliation among multiple identities, `in-process' rather than
complete; `in-between' rather than whole; a `contrapuntal' form rather
than a singular narrative.''\footnote{Jody Berland, \emph{North of
  Empire: Essays on the Cultural Technologies of Space} (Durham, NC:
  Duke University Press, 2009), 52.} These many metaphors lend
themselves to the networked and mediatized characteristics of
transborder, diasporic, or multilingual environments in other regions
and times.

\hypertarget{a-bifocal-habit-of-vision}{%
\section{A ``Bifocal Habit of
Vision''}\label{a-bifocal-habit-of-vision}}

It is worth recalling that one foundational moment in the Canadian
context, an early ``echo chamber'' that pre-dated the formation of
programs or centers of study, started with a 1953 Ford Foundation grant
application submitted by McLuhan, Edmund Carpenter, and their colleagues
that promised to explore the ``Changing Patterns of Language and
Behavior in the New Media of Communication.''\footnote{See La Garde,
  ``The 1987 Southam Lecture,'' 8--9.} Their successful Ford grant
facilitated a two-year graduate seminar in Culture and Communications,
and the launch of the journal \emph{Explorations in Culture and
Communication} that the applicant team co-edited between 1953 and 1958.
The Ford proposal portrayed a specific Canadian dualistic mindset that
drew in part from their reading of the Canadian political economist
Harold Innis. Key moments in Canadians' history, they argued, created
``a bi-focal habit of vision in their culture which makes natural to
their outlook the historical and the scientific, the humanist and the
technological simultaneously.''\footnote{Edmund S. Carpenter, Jaqueline
  Tyrwhitt, H. M. McLuhan, W. T. Easterbrook, and D. C. Williams,
  ``University of Toronto: Changing Patterns of Language and Behavior
  and the New Media of Communication (1953--1955),'' Ford Foundation
  Archives, Rockefeller Archive Center, New York, grant file PA 53--70,
  section 1, 1.} They believed this dualistic sensibility offered a rich
perspective on all of North America, where Canada worked as an ``early
warning system'' for the United States, another border metaphor based on
the DEW-line radar in the Canadian arctic. As I have traced in other
contributions, what later became known as the ``Toronto School'' was set
in motion by a group of scholars from Canada, the United States, and the
United Kingdom who sought to bridge historical, cultural, and
disciplinary borders.\footnote{Darroch, ``Bridging Urban and Media
  Studies''; Darroch, ``The Toronto School.''} They equally embraced
contributions from other transatlantic thinkers such as the Swiss
architectural historian Siegfried Giedion. While they were very
different scholars, Innis and Giedion shared some common research and
pedagogical goals in the late 1940s: Innis's encyclopaedical studies on
\emph{Empire and Communication} (1950) and the \emph{Bias of
Communication} (1951) were compiled in largely the same period that
Giedion scoured patent offices across the United States to develop his
monumental volume \emph{Mechanization Takes Command} (1948). Both
scholars approached space and time as key dimensions of technological
history and cultural analysis. And this should not be surprising in the
wake of World War II, when evolving theories in natural sciences and
physics in space-time relations also shaped works such as Norbert
Wiener's \emph{Cybernetics} (1948). Co-applicant Jaqueline Tyrwhitt, a
British town planner linked to CIAM and the London-based MARS group, and
a longtime colleague and translator of Giedion's work, used her own
voice to shape his writings for consumption within anglophone
countries.\footnote{See Ellen Shoshkes, \emph{Jaqueline Tyrwhitt: A
  Transnational Life in Urban Planning and Design} (Farnham, UK:
  Ashgate, 2013).} Tyrwhitt came to Toronto in 1951 and participated in
the Ford Foundation project bringing scholarly interests from across
disciplines and languages to the table.\footnote{Darroch, ``Bridging
  Urban and Media Studies.''} And Edmund Carpenter, an American
anthropologist who with McLuhan most shaped Toronto School thought,
arrived in 1948 with experiences including his World War II soldier
years in Japan, and with connections to a vast network of scholars in
anthropology and intercultural communications in the United States.
Carpenter actively solicited contributions from scholars such as Ashley
Montagu, David Bidney, and Irving Hallowell for the journal
\emph{Explorations}, for which he acted as chief editor from 1953 to
1958. In 1959, Carpenter left Toronto to serve as chair of a new
anthropology department at San Fernando Valley State College in
Northridge, California (now a campus in the University of California
system), where he helped shape the curriculum in visual and film
anthropology, and across the arts. \emph{Explorations} was borne out of
the Ford Foundation grant and took a bi-focal or even mosaic habit of
vision as its mantra. It was a journalistic experiment meant to
integrate research in anthropology, culture, and communication, and to
provide a mosaic of approaches, studies, and understandings.

While the notion of a ``bi-focal'' habit of vision influenced the early
Toronto School and English-Canadian approaches to communication and
media, the story in Quebec is different. While Innis was English
Canada's most prominent political economist and overall scholar in the
1940s and early 1950s, he was received much more hesitantly in Quebec.
In \emph{Harold Innis in the New Century,} Daniel Salée and, in a
separate chapter, Alain-G Gagnon and Sarah Fortin review the many
reasons that Innis's staples theories and major writings received such a
different audience there.\footnote{Daniel Salée, ``Innis and Quebec: The
  Paradigm That Would Not Be,'' in \emph{Harold Innis in the New
  Century}, ed. Charles R. Acland and William J. Buxton (Montreal:
  McGill-Queen's University Press, 1999); Alain-G Gagnon and Sarah
  Fortin, ``Innis in Quebec: Conjectures and Conjunctures,'' in
  \emph{Harold Innis in the New Century}, ed. Charles R. Acland and
  William J. Buxton (Montreal: McGill-Queen's University Press, 1999).}
Just as Canadian scholars of Innis's generation most often took their
doctorates in the United States, Britain, or France, Quebec
undergraduates also left for the United States and Europe, as well as
English Canada. As Salée notes:

\begin{quote}
In the late 1940s, the first crop of graduates from
Laval\textquotesingle s École des sciences sociales returned from the
United States and English Canada with postgraduate degrees to fill new
academic positions. It was only then that a more positivistic and
theoretically inclined outlook on social questions began to emerge in
Quebec\textquotesingle s sociographical and historiographical
discourse.\footnote{Salée, ``Innis and Quebec,'' 200.}
\end{quote}



\noindent Nevertheless, Innis's preoccupations and outlooks did not widely
resonate with scholars of his generation. ``The scholarly traditions out
of which each emerged, the imperatives of their milieux, and the
characteristic objectives of their respective intellectual endeavours
were so divergent that they would have had little to say to each
other.''\footnote{Salée, 202.}


Innis's scholarship came back into prominence in English Canada in the
early 1960s through political economists such as Mel Watkins at the
University of Toronto. Watkins was, intriguingly, one of the graduate
students in the Culture and Communications Seminar provided by the Ford
grant, working under the supervision of Innisian political economist
Thomas Easterbrook. Watkins became part of a generation of scholars
concerned with Canadian economic dependence on the United States,
inspired by Innis's studies of Canada's economic history.\footnote{For
  example, M. H. Watkins, ``A Staple Theory of Economic
  Growth,''~\emph{Canadian Journal of Economics and Political Science}
  296, no. 2 (1963).} As Salée notes, however, in Quebec the same
generation of scholars working towards radical analyses of economic\newpage\noindent
history and dependence did not share Watkins's reinterpretation of
Innis:

\begin{quote}
Quebec\textquotesingle s young radical intellectuals and academics found
much with which to agree in such an objective, but that was not going
far enough. They explained Quebec\textquotesingle s experience in the
light of its political, economic, and cultural oppression in the
post-Conquest (English)-Canadian state. Their reading of
Quebec\textquotesingle s situation through authors such as {[}Frantz{]}
Fanon and {[}Albert{]} Memmi clearly indicated that they equated
Quebec\textquotesingle s plight with that of Third World colonies and
developing nations.\footnote{Salée, ``Innis and Quebec,'' 204--5.}
\end{quote}

While neo-Innisians such as Watkins sought to reinforce Canadian
economic autonomy through state structures, Quebec's radical political
economic scholars during the province's growing nationalist movement and
Quiet Revolution cast their look farther south and beyond, to
anti-colonialist struggles in North Africa and to Latin American
dependency theory.

Yet a reading of Innis also needs to be positioned through his influence
on McLuhan and on the general orbit of the Toronto School, and the
emerging field of communication studies in the mid-to-late 1960s across
Canada, including Quebec. By the time new programs in communication
studies were being established from 1965 onwards, understandings of his
thought began to circulate among Quebec-based scholars, both anglophone
and Francophone, many of whom were bilingual themselves or at least
willing to work across the linguistic cleavage. This is to note again
that the bi- and multilingualism of many of Quebec scholars facilitated
and influenced emergent programs in this province. If Innis's work
itself was only selectively taken up by Quebec's Francophone scholars in
political economy and political science, the themes of his key 1950s
communications studies were announced and recirculated through the
excitement that McLuhan garnered across Canada and Quebec, first when
McLuhan rose to fame in the early 1960s and again when the translation
of his work into French coincided with Expo 67 in Montreal. With the
dawn of communications programs and related journals in Quebec in the
1960s, Innis began to receive a new readership. Roger de le Garde and
Line Ross, founding members of Laval University's program and the
journal \emph{Communication Information}, later published the only (to
my knowledge) translation of Innis's work in French, the famous first
chapter ``Minerva's Owl'' from \emph{The Bias of Communication},
explaining that ``Innis outlines a new field of communication research
in which media history is related to its cultural, economic, political
and military context.''\footnote{Harold Adams Innis, ``L'Oiseau de
  Minerve,'' trans. by Roger de la Garde and Line Ross,
  \emph{Communication Information} 5, no. 2--3 (1983): 266.} La Garde
would also reconsider Innis directly in his Southam Lecture address to
the Canadian Communication Association in 1987: ``Mr. Innis, Is There
Life after the `American Empire'?''\footnote{La Garde, ``The 1987
  Southam Lecture.''} Gaëtan Tremblay, founding member of the University
of Quebec at Montreal's program in communication (established in 1975)
and first President of the Canadian Communication Association, has also
recognized the profound influence of reading both Innis and McLuhan, the
former when he was an undergraduate student in sociology at Laval
University,\footnote{Gaëtan Tremblay, ``From Marshall McLuhan to Harold
  Innis, or From the Global Village to the World Empire,''
  \emph{Canadian Journal of Communication} 37, no. 4 (2012): 563.} and
the latter after his undergraduate studies and in the heyday of Expo
67.\footnote{Gaëtan Tremblay, ``Journey of a Researcher,''
  \emph{Canadian Journal of Communication} 39, no. 1 (2014): 11.}

\hypertarget{disciplinary-and-discursive-border-crossings}{%
\section{Disciplinary and Discursive Border
Crossings}\label{disciplinary-and-discursive-border-crossings}}

Accounts such as Tremblay's invite us to recall the diversity of
cultural and linguistic experiences for many Canadians, particularly in
key urban centers such as Montreal where both English- and
French-language universities established programs of study, and thus to
the role played not only by the promotion of Canadian and Québécois
linguistic and multicultural policies, but also to the city's everyday
multilingualism and acts of translation. They also invite us to consider
the influence of core translations of communication and media
theorization that facilitated access to scholarship for a rising
generation of scholars in Quebec and across Canada. In Quebec
especially, the French journal \emph{Communications}, launched in 1961
by Roland Barthes, George Friedmann, and Edgar Morin, was a major
resource for studies in mass communications and theories of semiotics.
The translation into French of McLuhan's most influential writings in
the late 1960s by Quebec journalist Jean Paré further promoted the study
of media and communication. The establishment of scholarly journals in
Canada and Quebec also provided sites for translations of intellectual
positions as well as reviews of books across languages: the
\emph{Canadian Journal of Communication}, originally based in Toronto,
and the journal \emph{Communication Information}, based in Quebec City
at Laval University, were launched in 1974 and 1975, respectively. Yet
scholars and students in the bilingual capital of Quebec played a
central role. La Garde and Yelle note that:

\begin{quote}
The newness of the field of communication studies has meant that there
is a relative scarcity of books and articles devoted to it. Until the
end of the 1980s, Quebec students drew on their knowledge of English,
which was fortunately reasonably good, in order to gain access to the
broader disciplinary debates. Of course, certain extracts from classic
American texts of the 1940s, 1950s, and 1960s had been translated into
French by European researchers.\footnote{La Garde and Yelle, ``Coming of
  Age,'' 73.}
\end{quote}

\noindent These include the first significant volumes of French translations of
core writings from US-based scholars which provided even further access
to Francophone students, such as \emph{Sociologie de l'Information:
Textes fondamentaux}, edited by Francis Balle and Jean G. Padioleau in
1972, and \emph{La Nouvelle Communication}, edited in 1981 by Yves
Winkin. These two contributions provided translations of core
communications scholarship, including, among many others, writings by
Harold Lasswell, Talcot Parsons, Elihu Katz, Gregory Bateson, Ray
Birdwhistell, Erving Goffman, and Edward T. Hall.\footnote{I wish to
  express my appreciation here to my colleague François Yelle at the
  University of Sherbrooke for a discussion of key sites and collections
  of communication thought in French translation. See also Yelle's
  (2004) exhaustive bibliography of publications in Europe for
  francophone communication studies, including those published in
  English, those co-authored by a Québécois scholar, and those written
  by an anglophone: ``Étude de la littérature réflexive de la recherche
  universitaire québécoise en communication médiatique'' (PhD diss.,
  University of Montreal, 2004), lxiv--lxxii.}

What became the \emph{Canadian Journal of Communication} (\emph{CJC})
was first launched in 1974 as \emph{Media Probe} by Earle Beatie, a York
University professor of journalism and the journal's first editor.
Beatie explained that there was ``no organization worrying about how
news is presented and how that presentation can be achieved.''
\emph{Media Probe} no doubt provided a nod to McLuhan's theses on the
artist and academic as cultural probe, but also to organizations such as
Pollution Probe that were active at the time. Among the journal's
initial areas of focus were analysis of how information circulates and
questions of access, library sciences and the responsibilities of
libraries, and ``understanding . . . the process of communication, the
roles of communication agencies, and the characteristics of the various
communication media.''\footnote{Earle Beattie, ``What Media Probe Is All
  About,'' \emph{Canadian Journal of Communication} 1, no. 1 (1974): 12.}
\emph{Media Probe} was established in metropolitan Toronto after
meetings in 1973 at York and the then Ryerson Polytechnical Institute.
While the journal was initiated in Toronto, its ``area of coverage'' was
planned to ``include other parts of Canada'' with the suggestion that
``local or regional Media Probes may be organized in other areas and
linking up will constitute a national organization.''\footnote{Beattie,
  11.} By 1978, \emph{Media Probe} had transitioned officially into the
\emph{Canadian Journal of Communication} (\emph{CJC}). A conference held
at the University of Windsor in 1979 facilitated initial discussions for
forming a Canadian Communications Association (CCA) as a learned
society. The \emph{CJC} was recommended to act as the association's
publication arm, but the ``necessity to encompass French language
scholarship in the field'' also led to the concern from ``Quebec
participants . . . that trying to incorporate the two language groups
into a single journal would be inappropriate and impracticable, given
not only the language question, but also the differing areas of emphasis
of French and English language scholars.''\footnote{Stewart Ferguson,
  ``Communication as a Discipline,'' \emph{Canadian Journal of
  Communication} 5, no. 1 (1978): 3.} An alternative affiliation for the
association was proposed with the ``University of Laval, which now
publishes a scholarly journal involving itself with communication
concerns.''\footnote{Ferguson, 3.} This journal, \emph{Communication
Information} (\emph{CI}), launched in 1975 with Line Ross, Michel de
Repentigny, and Roger de la Garde as part of the editorial
team.\footnote{The journal would later be renamed \emph{Communication,
  Information, Médias, Théories} in 1983.} Ultimately, \emph{CI} did not
become officially associated with the Canadian Communications
Association. The \emph{CJC} took on this mantle for the association's
initial years, although there was no legally binding arrangement.

Both journals contributed to disciplinary and linguistic border
crossing. \emph{Media Probe}/\emph{CJC} started as a more modest
publication and until the early 1980s remained largely focused on
English Canadian explorations of media industries, communications
policy, and theory. The journal has moved to the home university of each
subsequent editor. When the second editor, Eugene Tate from St. Thomas
More College, University of Saskatchewan took the reins in January 1982,
the \emph{CJC} began requiring abstracts in English and French, and
published its first full essay in French in the April 1985 issue (vol.
11, no. 4). It was not until Gertrude Robinson and Liss Jeffrey took
over the editorship at McGill University in 1988 that the journal
actively sought more submissions in French with the offer of translating
them where possible, noting as well that \emph{CI} was the leading
publication for Francophone research in communications.\footnote{See
  also Karla Margarita Ramírez y Ramírez's bibliometric analysis of the
  \emph{CJC} and \emph{CI}: ``Analyse bibliométrique des revues
  \emph{Canadian Journal of Communication} et \emph{Communication}
  1974--2005'' (PhD diss., University of Montreal, 2010).}

From the vantage point of Quebec City, the early issues of \emph{CI}
survey Quebec-based communications issues and industries, but also more
directly sought to incorporate research from beyond Quebec's borders. To
be sure, this outreach included reporting on research from English
Canada, the United States, and Europe. The original cover of \emph{CI}
through its first six issues (vols. 1 and 2, 1975--1978) features the
image of a human ear with an eye within it---a likely gesture to
McLuhan's neologism of an ``eye for an ear,'' resembling the inside
cover of \emph{Explorations 8: Verbi-Voco-Visual} (1957), a reference to
the interconnected sensory perceptions afforded by new media and an ode
to the polyphonic writing of James Joyce. However, \emph{CI} quickly
adopted an interest reaching beyond North American and European sources.
As early as its fifth edition (vol. 2, no. 2), \emph{CI} began
publishing abstracts in French, English, and Spanish. The journal's
Winter 1980 edition (vol. 3, no. 2) focused on ``L'information
internationale: commerce ou propagande ?,'' included a primary section
of articles stemming from a symposium on February 24, 1979 on
``L\textquotesingle échange inégal des informations dans le monde: le
cas de l\textquotesingle Amérique Latine,'' including a contribution
from Armand Mattelart after he had left the Pontifical Catholic
University of Chile and returned to France. In 1983, \emph{CI} announced
a partnership with the journal \emph{ININCO} of the Insitutio de
Investigaciones de la Communicación, Universidad Central de Venezuela
(vol. 6, no. 1), with the promise that each journal would publish a
contribution from the other, followed by a similar arrangement with
\emph{Communicación y cultura} in Mexico (vol. 7, no. 1). If English
Canadian communication thought and studies were wrapped up with
east-west cross-border disciplinary- (and nation-) building exercises,
Quebec-based communications scholars were casting their gaze beyond the
continent.

In her 1999 Southam Lecture for the Canadian Communication Association,
Gertrude Robinson set out to ``reconstruct'' the field of Canadian
communication studies. Drawing on previous historical accounts of the
field, she examines three interrelated phases: ``The first inquires into
when and how our field was institutionalized into the university system,
the second probes into the `founders' who set up the 10 graduate
programs, and the third addresses the French-English division of labour
in building our interdiscipline.''\footnote{Robinson, ``Remembering Our
  Past.''} In the first phase, she reminds us (following Roger de la
Garde) of a range of ``echo chambers'' from the 1940s to early 1960s
that predated university programs and other institutional
frameworks.\footnote{See La Garde, ``The 1987 Southam Lecture.''} These
are largely framed within central Canada, between the axes of Toronto
and Montreal. In Toronto, they include the 1950s Ford Grant and
\emph{Explorations} period of Toronto School thinkers described in the
previous section.

In his own 1987 Southam Lecture, La Garde took stock of such echo
chambers in the Quebec context, some twenty years after the first
programs had been institutionalized. These include Radio-Canada's
research division; the Centre catholique national; journals and
magazines including \emph{Cité Libre}~(1950) and \emph{Parti
Pris}~(1960), and le Centre de recherche sur l\textquotesingle opinion
publique (CROP), the province's first privately owned public opinion
firm.\footnote{La Garde, ``The 1987 Southam Lecture.'' See also
  Robinson, ``Remembering Our Past''; Lévesque, ``L'émergence et
  L'institutionnalisation.''} It is essential to recall that this phase
took place during Quebec's Quiet Revolution, overlapping with the launch
of television by CBC/Radio-Canada in 1952, a period of intense
nation-building and modernization during which the state wrestled social
power and educational oversight from the Catholic Church. Taking place
towards the end of this period and coinciding with the launch of
Quebec's first communication studies programs was of course the
much-celebrated world exposition in Montreal, Expo 67.

The founding of the first university programs in communication arts or
studies thus overlapped with the Canadian centennial celebrations of
1967, monumentalized by Expo 67. The bi-focal habit of perception that
inspired Toronto School scholars in the 1950s was in many ways echoed at
these celebrations. McLuhanism was a major influence at Expo 67 after
the publication of \emph{The Gutenberg Galaxy} (1962) and
\emph{Understanding Media} (1964), and their translation into French by
Jean Paré in 1967 and 1968. The first program in Communication Arts in
Canada was founded in 1965 by Father John O'Brien at Loyola, an
English-language Jesuit College in Montreal that graduated students with
BA degrees first through Laval University and later through the
University of Montreal, before the institution merged with Sir George
Williams University to become today's Concordia University. O'Brien
would contribute to designing the Christian Pavilion at Expo 67, which
featured the renowned film \emph{The Eighth Day} by filmmaker and Loyola
Communications faculty member Charles Gagnon. At McGill University,
Donald Theall led a ``McGill Study of Expo 67'' with graduate student
researchers from the Department of English, and later planned to
complete a manuscript on ``Expo 67 as Total Environment'' in the same
year (1976) that Theall and Robinson helped establish the first PhD in
Communication Studies at McGill.\footnote{Monika Kin Gagnon and Janine
  Marchessault, \emph{Reimagining Cinema: Film at Expo 67} (Montreal:
  McGill-Queen's University Press, 2014), 10.} In 1971, Theall also
recruited Jacques Languirand, a Quebec dramatist and novelist, who had a
major presence in radio and television, to join McGill's English
Department, where he taught until 1980. Languirand had helped coordinate
Expo 67, contributing to the design of several key installations: the CN
Pavilion, the Polar Regions, the Ville des solitudes, and most notably,
the Citérama as part of the Man and his Community Pavilion.\footnote{François
  Yelle, ``McLuhan at McGill: a Brief Look at Theall and Languirand''
  (paper presented at the Edgy Media Symposium, University of Windsor,
  Canada, March 1, 2019).}

Michael Dorland reminds us that Robinson's brief reference to the
French-English divide in her 1999 address

\begin{quote}
does at least raise the linguistic/cultural question, which surely
constitutes a crucial defining characteristic of the
institutionalization of communication studies in the Canadian
context---indeed, a characteristic as significant as the role of the
state in setting research priorities and establishing graduate programs.
The ``linguistic divide,'' therefore, reinforces the highly fragmented
nature of communication studies in Canada, a fragmentation already
encouraged by the various disciplines from which different programs
emerged and by their regional location.\footnote{Dorland, ``Knowledge
  Matters,'' 56.}
\end{quote}

\noindent The disciplinary and discursive fragmentation that Dorland alludes to
represents in part the variety of communication studies or cognate
programs across the Canadian horizon. In the 1960s and 1970s, these
include communications-related programming at the University of
Saskatchewan, particularly its Regina campus in the 1960s and early
1970s; the University of Windsor, which established a degree in
Communication Arts in 1969; and Simon Fraser University's Vancouver
campus, which launched a BA program in 1973. Alongside Quebec
institutions, participation from these institutions came together to
form the Canadian Communication Association (CCA) in 1979.

The CCA was first proposed at a meeting of scholars at the University of
Windsor in Spring 1978. In Fall 1978, Donald Theall chaired a steering
committee meeting in Ottawa to consider the formation of the new
academic society. The association itself was established the following
year at the learned societies meeting in Saskatoon, Saskatchewan, with a
board of directors and steering committee chaired by Theall. At the
following learned societies meeting in Montreal in 1980, Gaëtan Tremblay
was elected as the first president of the association. Windsor thus
played an important but underreported role in establishing the CCA as a
cross-nation initiative for the field of Canadian communication studies.
Windsor had in fact been home to McLuhan from 1944 to 1946 when he
taught at the university's predecessor, Assumption College. As I have
argued elsewhere, McLuhan's experience living in the Windsor-Detroit
borderlands, where one of his idols, Wyndham Lewis, was also actively
lecturing, can only have influenced his later concept of borderlines as
intervals of resonance.\footnote{Michael Darroch, ``Border Environments:
  Theorising Media and Culture in the Windsor-Detroit Borderlands,
  1943--1946,'' in ``Borders and Spaces in the English-Speaking World,''
  ed. Jean-Jacques Chardin, special issue, \emph{RANAM: Recherches
  Anglaises et Nord-Américain}, no. 52 (2019).} In 1965, Windsor
commissioned a report by instructors at Wayne State University in
Detroit which recommended a ``suggested structure and application of a
closed circuit television system to the instructional program of the
University of Windsor'' based on research and institutional experience
with educational television at American universities.\footnote{James B.
  Tintera and Stuart K. Bergsma, ``A Report to the University of
  Windsor: The Suggested Structure and Application of a Closed Circuit
  Television System to the Instructional Program of the University of
  Windsor'' (Detroit: Mass Communications Centre, Wayne State
  University, November 30, 1965), original document, Department of
  Communication, Media, and Film, University of Windsor.} In short
order, a Communications Centre was established in 1966 to provide CCTV
and other media services across faculties at the University of Windsor.
By 1968, however, it became clear that media services would become in
demand across the institution and that the study of communication and
media practices would require a different entity. The Department of
Communication Arts ``sprung from the Communications Centre,''
particularly through the leadership of the Centre's first director,
Walter Romanow, who then became the initial Chair of Communication
Arts.\footnote{``The Future of the Media Centre at the University of
  Windsor'' (unpublished manuscript, August 1971), original document,
  Department of Communication, Media, and Film, University of Windsor.}
Windsor grew into a prominent Communications program; faculty took part
in international conferences and contributed to a number of Royal
Commissions, most importantly Ontario's LaMarsh Royal Commission on
Violence in the Communications Industry (1975).



\hypertarget{borderline-cases-multifocal-habits-of-vision}{%
\section{Borderline Cases: Multifocal Habits of
Vision}\label{borderline-cases-multifocal-habits-of-vision}}

Communication studies in Canada from the 1960s to the 1980s was an
environment that bridged disciplinary but also cultural divides, in
Flusser's terms, requiring translative leaps across linguistic and
disciplinary horizons. These acts of border crossing took place
particularly due to the people involved, scholars and students who were
committed to writing and studying in French and English, or to
publishing their scholarly work in both languages. Many of these
scholars came from backgrounds that afforded them the multilinguistic
capital to pursue these avenues; others worked tirelessly to overcome
the linguistic divide, taking pride in even working on translations of
their own work. The stories told by founding figures in the field are
therefore important, even as they at times question and contradict each
other. We may recognize that communication and media studies in Canada
have been shaped by the symbolic environment of the powerful discourses
of Canadian dualisms, including the binational status of English and
French as official languages; and by the many royal and provincial
commissions and Crown corporations established to examine and promote
bilingual cultural and communication\newpage\noindent institutions and infrastructures
(mostly in cities huddled along the physical border).\footnote{Michael
  Dorland, ed., \emph{The Cultural Industries in Canada} (Toronto: James
  Lorimer and Company, 1996), xii.}

McLuhan's notion of Canada's borderline case expressed the perspective
that Canadian identities, during the time when communications studies
programs took root across the country from the 1960s to 1980s, were
ambivalent and post-national. In the twenty-first century, this has
become a more difficult perspective to maintain. Since the beginning of
the century, borders have become more firmly entrenched and militarized,
and offer more hesitantly the same metaphorical resonance of openness,
allegiances, or forms of cultural interpenetration.\footnote{Lee Rodney,
  \emph{Looking Beyond Borderlines: North America's Frontier
  Imagination} (New York: Routledge, 2017), 104--6.} Within Canada, the
``two solitudes'' motif has faded as multicultural policies have
embraced Canada's pluralism, its First Nations, Métis, and Inuit
communities, and as immigration has rapidly risen. Political momentum to
restrict immigration and assert a firmer Canadian identity ensconced in
settler colonial histories and ideological images of Canada as a
northern state today collides with increased recognition of Indigenous
peoples' histories across Canada and decolonial narratives advanced by
scholars, activists, and artists, particularly after the final report of
the 2015 Canada's Truth and Reconciliation Commission.

However, rather than viewing Canadians' ``bi-focal habit of vision,''
celebrated by McLuhan and his \emph{Explorations} colleagues, as out of
step with twenty-first century realities, I would argue that the
metaphorical power of the notion of the borderline case still resides in
its multiplicity. As McLuhan indicated in his 1967 lectures, Canada was
never \emph{a} borderline case, but rather is constituted by multiple
borderlines. Janine Marchessault remarks that McLuhan

\begin{quote}
was able to recognize and understand as radical methodology in Innis's
work {[}that{]} the borderline is historical, charged with emotional
intensities, and in Canada's case is ``porous.'' History written from a
location along the borderline can reanimate and challenge official
history and reassert the effects of time.\footnote{Janine Marchessault,
  \emph{Marshall McLuhan: Cosmic Media} (London: Sage, 2005), 100.}
\end{quote}

\noindent This recognition was amplified by the various proposals for establishing
communications studies in Canada in the late 1960s. To stick with
musical metaphors, borders (national, cultural, linguistic) may be
spaces of intense resonances, but that does not necessarily make them
harmonious. A frontier imagination enables cultural collisions and
consonances, facilitates contradictions and agreements, and provides
sites for processes of conflict, power, force, and aggression to
function against and alongside hospitality, openness, and, indeed,
communication. For the history of Canadian communication studies,
Canada's many borderlines continue to provide fertile ground for
multifocal habits of vision.




\section{Bibliography}\label{bibliography}

\begin{hangparas}{.25in}{1} 



Averbeck-Lietz, Stefanie. \emph{Kommunikationswissenschaft im
Internationalen Vergleich: Transnationale Perspektiven (Medien, Kultur,
Kommunikation).} Wiesbaden, Germany: Springer, 2017.

Babe, Robert E. \emph{Canadian Communication Thought: Ten Foundational
Writers}. Toronto: University of Toronto Press, 2000.

Beattie, Earle. ``What Media Probe Is All About.'' \emph{Canadian
Journal of Communication} 1, no. 1 (1974): 11--12.

Berland, Jody. \emph{North of Empire: Essays on the Cultural
Technologies of Space}. Durham, NC: Duke University Press, 2009.

Braz, Albert. ``Outer America: Racial Hybridity and Canada's Peripheral
Place in Inter-American Discourse.'' In \emph{Canada and Its Americas:
Transnational Navigations}, edited by Winfried Siemerling and Sarah
Phillips Casteel\emph{,} 119--34. Montreal: McGill-Queen's University
Press, 2010.

Carpenter, Edmund S., Jaqueline Tyrwhitt, H. M. McLuhan, W. T.
Easterbrook, and D. C. Williams. ``University of Toronto: Changing
Patterns of Language and Behavior and the New Media of Communication
(1953--1955).'' Ford Foundation Archives, Rockefeller Archive Center,
New York. Grant File PA 53--70, Section 1, 1--11.

Cavell, Richard. ``McLuhan's `Borderline Case' Revisted.'' In
\emph{Comment comparer le Canada avec les États-Unis aujourd'hui},
edited by Hélène Quanquin, Christine Lorre-Johnston, and Sandrine
Ferré-Rode, 25--50. Paris: Presses Sorbonne Nouvelle, 2009.

Charland, Maurice. ``Technological Nationalism.'' \emph{Canadian Journal
of Political and Social Theory} 10, no. 1 (1996): 196--220.

Darroch, Michael. ``Border Environments: Theorising Media and Culture in
the Windsor-Detroit Borderlands, 1943--1946.'' In ``Borders and Spaces
in the English-Speaking World,'' edited by Jean-Jacques Chardin. Special
issue, \emph{RANAM: Recherches anglaises et nord-américain}, no. 52
(2019): 11--32.

Darroch, Michael. ``Bridging Urban and Media Studies: Jaqueline Tyrwhitt
and the \emph{Explorations} Group, 1951--1957.'' \emph{Canadian Journal
of Communication} 33, no. 2 (2008): 147--63.

Darroch, Michael. ``Giedion and Explorations: Confluences of Space and
Media in Toronto School Theorisation.'' In \emph{Media Transatlantic:
Developments in Media and Communication Studies between North American
and German-Speaking Europe}, edited by Norm Friesen, 63--90. Vienna:
Springer, 2016.

Darroch, Michael. ``Medial Translations and Human Unsettlements:
Planetary Urbanisms from McLuhan to Flusser.'' In \emph{Speaking Memory:
How Translation Shapes Cities}, edited by Sherry Simon, 169--88.
Montreal: McGill-Queen's University Press, 2016.

Darroch, Michael. ``The Toronto School: Cross-Border Encounters,
Intellectual Entanglements.'' In \emph{The International History of
Communication Studies}, edited by Peter Simonson and David W. Park,
276--301. London: Routledge, 2016.

Darroch, Michael, Karen Engle, and Lee Rodney. ``Introduction: Sensing
Borders.'' In ``Ressentir (les frontières)/Sensing (Borders).'' Special
issue, \emph{Intermédialités}, no. 34 (Autumn 2019).
\url{https://doi.org/10.7202/1070869ar}.

Dorland, Michael, ed. \emph{The Cultural Industries in Canada}. Toronto:
James Lorimer and Company, 1996.

Dorland, Michael. ``Knowledge Matters: The Institutionalization of
Communication Studies in Canada.'' In \emph{Mediascapes: New Patterns in
Canadian Communication}, edited by Paul Attalah and Leslie Regan Shade,
46--64. Toronto: Thomson-Nelson, 2002.

Ferguson, Stewart. ``Communication as a Discipline.'' \emph{Canadian
Journal of Communication} 5, no. 1 (1978): 1--5.

Fermi, Laura. \emph{Illustrious Immigrants: The Intellectual Migration
from Europe, 1930--1941}. Chicago: University of Chicago Press, 1968.

Fleming, Donald, and Bernard Bailyn, eds. \emph{The Intellectual
Migration: Europe and America, 1930--1960.} Cambridge, MA: Belknap Press
of Harvard University Press, 1969.

Flusser, Vilém. \emph{Ende der Geschichte, Ende der Stadt?} Vienna:
Picus, 1992.

Flusser, Vilém. ``Essays.'' In \emph{Writings}, edited by Andreas
Ströhl, translated by Erik Eisel, 192--96. Minneapolis: University of
Minnesota Press, 2002.

Flusser, Vilém. \emph{Freedom of the Migrant.} Translated by Kenneth
Kronenberg. Edited by Anke Finger. Urbana, IL: University of Illinois
Press, 2003.

Friesen, Norm, ed. \emph{Media Transatlantic: Developments in Media and
Communication Studies between North American and German-Speaking
Europe}. Vienna: Springer, 2016.

``The Future of the Media Centre at the University of Windsor.''
Unpublished manuscript, August 1971. Original Document, Department of
Communication, Media, and Film, University of Windsor.

Gagnon, Alain-G, and Sarah Fortin. ``Innis in Quebec: Conjectures and
Conjunctures.'' In \emph{Harold Innis in the New Century}, edited by
Charles R. Acland and William J. Buxton, 209--24. Montreal:
McGill-Queen's University Press, 1999.

Giedion, Sigfried. \emph{Mechanization Takes Command}. Cambridge, MA:
Harvard University Press, 1948.

Guldin, Rainer. ``Die Zweite Unschuld: Heilsgeschichtliche und
Eschatologische Perspektiven im Werk Vilém Flussers und Marshall
McLuhans.'' \emph{Flusser Studies} 6 (2008): 1--24.

Guldin, Rainer. ``From Transportation to Transformation: On the Use of
the Metaphor of Translation within Media and Communication Theory.''
\emph{Global Media Journal: Canadian Edition} 5, no. 1 (2012): 39--52.

Hamilton, Sheryl. ``Considering Critical Communication Studies in
Canada.'' In \emph{Mediascapes: New Patterns in Canadian Communication},
edited by Paul Attalah and Leslie Regan Shade, 4--26. Toronto:
Thomson-Nelson, 2002.

Holub, Robert C. \emph{Crossing Borders: Reception Theory,
Poststructuralism, Deconstruction}. Madison: University of Wisconsin
Press, 1992.

Innis, Harold Adams. \emph{The Bias of Communication}. Toronto:
University of Toronto Press, 1951.

Innis, Harold Adams. \emph{Empire and Communications}. Oxford: Oxford
University Press, 1950.

Innis, Harold Adams. ``L'Oiseau de Minerve.'' Translated by Roger de la
Garde and Line Ross. \emph{Communication Information} 5, no. 2--3
(1983): 266--97.

Jay, Martin. \emph{Permanent Exiles: Essays on the Intellectual
Migration from Germany to America}. New York: Columbia University Press,
1985.

Kin Gagnon, Monika, and Janine Marchessault. \emph{Reimagining Cinema:
Film at Expo 67.} Montreal: McGill-Queen's University Press, 2014.

Kroker, Arthur. \emph{Technology and the Canadian Mind: Innis, McLuhan,
Grant}. Montreal: New World Perspectives, 1984.

Lacroix, Jean-Guy, and Benoit Lévesque. ``L'émergence et
l'institution- nalisation de la recherche en communication au Quebec.''
\emph{Communication Information} 7, no. 2 (1985): 7--31.

Lacroix, Jean-Guy, and Benoit Lévesque. ``Principaux thèmes et courants
théoriques dans la littérature scientifique en communication au
Quebec.'' \emph{Communication Information} 7, no. 3 (1985): 153--211.

La Garde, Roger de, and François Yelle. ``Coming of Age: Communication
Studies in Quebec.'' In \emph{Mediascapes: New Patterns in Canadian
Communication}, edited by Paul Attalah and Leslie Regan Shade,
65\emph{--}86. Toronto: Thomson-Nelson, 2002.

Löblich, Maria, and Stefanie Averbeck-Lietz. ``The Transnational Flow of
Ideas and \emph{Histoire Croisée} with Attention to the Cases of France
and Germany.'' In \emph{The International History of Communication
Study}, edited by Peter Simonson and David W. Park, 25--46. New York:
Routledge, 2016.

Lorimer, Rowland, and Jean McNulty. \emph{Mass Communication in Canada}.
Toronto: McClelland and Stewart, 1987. Now in its ninth edition as
Gasher, Mike, David Skinner, and Natalie Coulter. \emph{Media and
Communication in Canada: Networks, Culture, Technology, Audience}.
Toronto: Oxford University Press Canada, 2020.

McLuhan, Marshall. ``Canada: The Borderline Case.'' In \emph{The
Canadian Imagination}, edited by David Staines, 226--48. Cambridge, MA:
Harvard University Press, 1977.

McLuhan, Marshall. \emph{The Gutenberg Galaxy: The Making of Typographic
Man}. Toronto: University of Toronto Press, 1962.

McLuhan, Marshall. \emph{La Galaxie Gutemberg: La Génèse de
L\textquotesingle homme Typographique}. Translated by Jean Paré.
Montreal: Éditions H. M. H. Ltée, 1967.

McLuhan, Marshall. ``The Marfleet Lectures (1967): Part 1; Canada the
Borderline Case.'' In \emph{Understanding Me}: \emph{Lectures and
Interviews}, edited by Stephanie McLuhan and David Staines, 104‒22.
Toronto: McClelland \& Stewart, 2003.

McLuhan, Marshall. \emph{Pour Comprendre Les Medias}. Translated by Jean
Paré. Montreal: Éditions H. M. H. Ltée, 1968.

McLuhan, Marshall. \emph{Understanding Media: The Extensions of Man}.
New York: McGraw Hill, 1964.

McLuhan, Marshall, and Edmund Carpenter, eds. ``Verbi-Voco-Visual.''
Special issue, \emph{Explorations}, no. 8 (1957).

Park, David W., and Jefferson Pooley, eds. \emph{The History of Media
and Communication Research: Contested Memories.} New York: Peter Lang,
2008.

Proulx, Serge. ``Les communications: Vers un nouveau savoir savant?''
\emph{Recherches sociographiques} 20, no. 1 (1979): 103--17.

Ramírez y Ramirez, Karla Margarita. ``Analyse bibliométrique des revues
\emph{Canadian Journal of Communication} et \emph{Communication}
1974--2005.'' PhD diss., University of Montreal, 2010.

Robinson, Gertrude. ``The Katz/Lowenthal Encounter: An Episode in the
Creation of Personal Influence.'' \emph{Annals of the American Academy
of Political and Social Science} 608 (November 2006): 76--96.

Robinson, Gertrude. ``Remembering Our Past: Reconstructing the Field of
Communication Studies.'' \emph{Canadian Journal of Communication} 25,
no. 1 (2000).

Rodney, Lee. \emph{Looking Beyond Borderlines: North America's Frontier
Imagination.} New York: Routledge, 2017.

Salée, Daniel. ``Innis and Quebec: The Paradigm That Would Not Be.'' In
\emph{Harold Innis in the New Century}, edited by Charles R. Acland and
William J. Buxton, 196--208. Montreal: McGill-Queen's University Press,
1999.

Salter, Liora, ed. \emph{Communication Studies in Canada.} Toronto:
Butterworth, 1981.

Salter, Liora. ``Taking Stock: Communication Studies in 1987.''
\emph{Canadian Journal of Communication} 13, no. 5 (1988): 23--45.

Shoshkes, Ellen. \emph{Jaqueline Tyrwhitt: A Transnational Life in Urban
Planning and Design}. Farnham, UK: Ashgate, 2013.

Simonson, Peter, and David W. Park, eds. \emph{The International History
of Communication Study}. New York: Routledge, 2016.

Simonson, Peter, and John D. Peters. ``Communication and Media Studies:
History to 1968.'' In \emph{International Encyclopedia of
Communication}, edited by Wolfgang Donsbach. Article published September
1, 2014. https://doi.org/10.1002/9781405186407.wbiecc087.pub2.

Tate, Eugene D., Andrew Osler, Gregory Fouts, and Arthur Siegel. ``The
Beginnings of Communication Studies in Canada: Remembering and Narrating
the Past.'' \emph{Canadian Journal of Communication} 25, no. 1 (2000).

Taylor, Greg, and Ray op'tLand. ``Communication Research and Teaching in
Canada.'' \emph{Publizistik} 64 (2019): 79--101.

Theall, Donald. ``Communication Theory and the Marginal Culture: The
Socio-Aesthetic Dimensions of Communication Study.'' In \emph{Studies in
Canadian Communications}, edited by Gertrude Robinson and Donald Theall,
7--26. Montreal: Graduate Program in Communications, McGill University,
1975.

Tintera, James B., and Stuart K. Bergsma. ``A Report to the University
of Windsor: The Suggested Structure and Application of a Closed Circuit
Television System to the Instructional Program of the University of
Windsor.'' Detroit: Mass Communications Centre, Wayne State University,
November 30, 1965. Original Document, Department of Communication,
Media, and Film, University of Windsor.

Tremblay, Gaëtan. ``From Marshall McLuhan to Harold Innis, or From the
Global Village to the World Empire.'' \emph{Canadian Journal of
Communication} 37, no. 4 (2012): 561--75.

Tremblay, Gaëtan. ``Journey of a Researcher.'' \emph{Canadian Journal of
Communication} 39, no. 1 (2014): 9--28.

Watkins, M. H. ``A Staple Theory of Economic Growth.''~\emph{Canadian
Journal of Economics and Political Science} 296, no. 2 (1963): 141--58.

Wiener, Norbert. \emph{Cybernetics}. Cambridge, MA: MIT Press, 1948.

Yelle, François. ``Étude de la littérature réflexive de la recherche
universitaire québécoise en communication médiatique.'' PhD diss.,
University of Montreal, 2004.

Yelle, François. ``L'histoire des études en communication au Québec et
le dogme de la rupture, ou l'héritage peu célébré des intellectuels
canadiens-français des années 1940 et 1950.'' \emph{Revista Eptic} 19,
no. 1 (2017): 115--35.

Yelle, François. ``McLuhan at McGill: A Brief Look at Theall and
Languirand.'' Paper presented at the Edgy Media Symposium, University of
Windsor, Canada, March 1, 2019.



\end{hangparas}


\end{document}