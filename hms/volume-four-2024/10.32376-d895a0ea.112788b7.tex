% see the original template for more detail about bibliography, tables, etc: https://www.overleaf.com/latex/templates/handout-design-inspired-by-edward-tufte/dtsbhhkvghzz

\documentclass{tufte-handout}

%\geometry{showframe}% for debugging purposes -- displays the margins

\usepackage{amsmath}

\usepackage{hyperref}

\usepackage{fancyhdr}

\usepackage{hanging}

\hypersetup{colorlinks=true,allcolors=[RGB]{97,15,11}}

\fancyfoot[L]{\emph{History of Media Studies}, vol. 4, 2024}


% Set up the images/graphics package
\usepackage{graphicx}
\setkeys{Gin}{width=\linewidth,totalheight=\textheight,keepaspectratio}
\graphicspath{{graphics/}}

\title[Notes for Historicizing]{Notes for Historicizing the Disintegrated Internationalization of Communication Studies in Latin America} % longtitle shouldn't be necessary

% The following package makes prettier tables.  We're all about the bling!
\usepackage{booktabs}

% The units package provides nice, non-stacked fractions and better spacing
% for units.
\usepackage{units}

% The fancyvrb package lets us customize the formatting of verbatim
% environments.  We use a slightly smaller font.
\usepackage{fancyvrb}
\fvset{fontsize=\normalsize}

% Small sections of multiple columns
\usepackage{multicol}

% Provides paragraphs of dummy text
\usepackage{lipsum}

% These commands are used to pretty-print LaTeX commands
\newcommand{\doccmd}[1]{\texttt{\textbackslash#1}}% command name -- adds backslash automatically
\newcommand{\docopt}[1]{\ensuremath{\langle}\textrm{\textit{#1}}\ensuremath{\rangle}}% optional command argument
\newcommand{\docarg}[1]{\textrm{\textit{#1}}}% (required) command argument
\newenvironment{docspec}{\begin{quote}\noindent}{\end{quote}}% command specification environment
\newcommand{\docenv}[1]{\textsf{#1}}% environment name
\newcommand{\docpkg}[1]{\texttt{#1}}% package name
\newcommand{\doccls}[1]{\texttt{#1}}% document class name
\newcommand{\docclsopt}[1]{\texttt{#1}}% document class option name


\begin{document}

\begin{titlepage}

\begin{fullwidth}
\noindent\Large\emph{History of Communication Studies across the Americas} \hspace{20mm}\includegraphics[height=1cm]{logo3.png}\\
\noindent\hrulefill\\
\vspace*{1em}
\noindent{\Huge{Notes for Historicizing the Disintegrated\\\noindent Internationalization of Communication\\\noindent Studies in Latin America\par}}

\vspace*{1.5em}

\noindent\LARGE{Raúl Fuentes Navarro} \href{https://orcid.org/0000-0001-6494-8122}{\includegraphics[height=0.5cm]{orcid.png}}\par\marginnote{\emph{Raúl Fuentes Navarro, ``Notes for Historicizing the Disintegrated Internationalization of Communication Studies in Latin America,'' \emph{History of Media Studies} 4 (2024), \href{https://doi.org/10.32376/d895a0ea.112788b7}{https://doi.org/ 10.32376/d895a0ea.112788b7}.} \vspace*{0.75em}}
\vspace*{0.5em}
\noindent{{\large\emph{Universidad de Guadalajara}, \href{mailto:raul@iteso.mx}{raul@iteso.mx}\par}} \marginnote{\href{https://creativecommons.org/licenses/by-nc/4.0/}{\includegraphics[height=0.5cm]{by-nc.png}}}

% \vspace*{0.75em} % second author

% \noindent{\LARGE{<<author 2 name>>}\par}
% \vspace*{0.5em}
% \noindent{{\large\emph{<<author 2 affiliation>>}, \href{mailto:<<author 2 email>>}{<<author 2 email>>}\par}}

% \vspace*{0.75em} % third author

% \noindent{\LARGE{<<author 3 name>>}\par}
% \vspace*{0.5em}
% \noindent{{\large\emph{<<author 3 affiliation>>}, \href{mailto:<<author 3 email>>}{<<author 3 email>>}\par}}

\end{fullwidth}

\vspace*{5em}


\hypertarget{abstract}{%
\section{Abstract}\label{abstract}}

This essay characterizes a ``disintegrated internationalization'' as the
main trend in the development of communication studies in Latin America.
This hypothetical approach is based on critical readings of multiple
bibliographical sources generated in the most recent decades and
published in Spanish, Portuguese, or English, both in the form of essays
and empirical studies, somewhat supporting a historical sociology. The
systematic and rigorously analyzed and interpreted documentation,
following the methodological example of Luis Ramiro Beltrán, allows us
to advance in the recognition of some factors that separate national
``academic fields'' from their integration in broader scales---Latin
American or global. The conclusion proposes what the elaboration of a
historical narrative needs to recognize to be more useful: the
complexity and multiplicity of the historical-social processes that have
been interwoven and interdetermined in each of the spatiotemporal stages
that it is pertinent to define as Latin America's own.


\hypertarget{abstract}{%
\section{Resumen}\label{resumen}}

Este ensayo caracteriza una "internacionalización desintegrada" como la principal tendencia en el desarrollo de los estudios de comunicación en América Latina. Este planteamiento hipotético se basa en lecturas críticas de múltiples fuentes bibliográficas generadas en las décadas más recientes y publicadas en español, portugués o inglés, tanto en forma de ensayos como de estudios empíricos, apoyándose en cierto modo en una sociología histórica. La documentación sistemática y rigurosamente analizada e interpretada, siguiendo el ejemplo metodológico de Luis Ramiro Beltrán, permite avanzar en el reconocimiento de algunos factores que separan los "campos académicos" nacionales de su integración en escalas más amplias -latinoamericanas o globales. La conclusión propone lo que la elaboración de una narrativa histórica necesita reconocer para ser más útil: la complejidad y multiplicidad de los procesos histórico-sociales que se han entretejido e interdeterminado en cada una de las etapas espacio-temporales que es pertinente definir como propias de América Latina.


 




 \end{titlepage}

% \vspace*{2em} | to use if abstract spills over

\hypertarget{introduction}{%
\section{Introduction}\label{introduction}}

\newthought{These ``notes'' emerge} from a decades-long trajectory researching the
processes of institutionalization of the study of communication and the
constitution of a correlative \emph{academic field,}\footnote{Pierre
  Bourdieu, ``La specificité du champ scientifique et les conditions
  sociales du progrès de la raison,'' \emph{Sociologie} \emph{et}
  \emph{Societés} 7, no. 1 (1975); Bourdieu, \emph{Homo Academicus}
  (Stanford: Stanford University Press, 1988); Bourdieu, \emph{Science
  of Science and Reflexivity} (Chicago: University of Chicago Press, 2004).} situated first in Mexico, and which in some aspects and
occasions has been extended towards broader scales, as ``Latin
America,''\footnote{Raúl Fuentes-Navarro, \emph{La emergencia de un
  campo académico: Continuidad utópica y estructuración científica de la
  investigación de la comunicación en México} (Guadalajara: ITESO /
  Universidad de Guadalajara, 1998); Fuentes-Navarro, \emph{Un campo
  cargado de futuro: El estudio de la comunicación en América Latina}
  (Mexico City: FELAFACS, 1992); Fuentes-Navarro ``La investigación de
  la comunicación en América Latina: Condiciones y perspectivas para el
  siglo XXI,'' \emph{Comunicación y Sociedad}, no. 36 (1999).}
eventually as ``Ibero-America,'' or even as an ``Inter-American''
macro-region that includes Canada, the United States (of America), and
the other countries of origins and languages other than Spanish and
Portuguese in the continent.\footnote{During the most recent decade,
  this trajectory has found great stimulus and impulse in the context of
  the international academic community developed on the initiative of
  Park, Pooley, and Simonson---most significantly, through the Communication History Division
  of the International Communication Association (ICA)
  (\url{https://www.icahdq.org/}), the ``History of Media Studies''
  working group of the Consortium for History of Science, Technology,
  and Medicine (CHSTM) (\url{https://www.chstm.org/media-studies}), and
  the journal \emph{History of Media Studies}
  (\url{https://hms.mediastudies.press/}); and, more specifically,
  through the July 2022 virtual roundtable they organized on the
  ``History of Communication Studies in the Americas,'' at which an
  earlier draft of this essay was presented. It is important for the
  author to acknowledge the highly qualified and careful reviewers who
  read the essay at that stage and offered thoughtful recommendations
  with deep implications for the work, particularly in terms of its
  theoretical framework; most of these recommendations, for that reason,
  can only be appropriately addressed at a future occasion.} Although
these denominations are just convenient heuristics and therefore
contingent, they are never innocuous, since they refer to a complex,
challenging, and ``enigmatic'' spatiotemporal
multidimensionality.\footnote{Octavio Ianni, \emph{Enigmas da
  modernidade-mundo} (Rio de Janeiro: Civilização Brasileira, 2000).}

As noted elsewhere,\textsuperscript{5} ``the term `Latin
America' was probably invented by the French in their attempts in the
nineteenth century to colonize the Americas to the south of the Rio
Grande.''\textsuperscript{6} It has been used at times as
a kind of counterpart to the term ``manifest destiny,'' coined by US
journalists and politicians to justify as ``God's will'' the annexation
of territories and the military interventions abroad, but above all as a
common identity mark for the more than 650 million
inhabitants\textsuperscript{7} of twenty modern
countries on the continent. ``Ibero-America,'' in turn, refers to
another regional composition, more historical-cultural than
geographical, as it includes Spain and Portugal, the imperial ``mother
countries'' of Latin America, together with the countries that were once
their overseas colonies, which increases the current reference
population to more than seven hundred million people, that is, almost 9
percent of the world\textquotesingle s total.

The intellectual dimension of this work takes a cue from the thesis of
Mexican historian and philosopher Edmundo O'Gorman (1906--1995),
originally published in 1958, where he writes that ``America was not
discovered, but invented.'' Therefore, the key to understanding America
lies not in its identity but in its \emph{historical sense}, that is,
not in the past but in the future. O'Gorman pointed out that ``America,
in effect, was invented under the physical species of `continent' and
under the historical species of `the new world.' It emerged, then, as a
given physical entity, already made and unalterable, and as a moral
entity endowed with the possibility of realizing itself in the order of
historical being.''\textsuperscript{8} Discarding the idea that the world was
an island with three parts (Europe, Asia, and Africa) was an Iberian
contribution, and rejecting that the European model was the pinnacle of
civilization was an Anglo-Saxon one. Together, these are the two
``liberations'' in which ``the greatness of the invention of America,
the\marginnote{\textsuperscript{5} Raúl Fuentes-Navarro,
  ``Institutionalization and Internationalization of the Field of
  Communication Studies in Mexico and Latin America,'' in \emph{The
  International History of Communication Study}, ed. Peter Simonson and
  David W. Park (New York: Routledge, 2016), 338.} double\marginnote{\textsuperscript{6} Francisco Salzano and Maria C. Bortolini, \emph{The
  Evolution and Genetics of Latin American Populations} (Cambridge:
  Cambridge University Press, 2002), 328.} step,\marginnote{\textsuperscript{7} Updated as of July 2023.} decisive\marginnote{\textsuperscript{8}\setcounter{footnote}{8} Edmundo O'Gorman, \emph{La invención de
  América: Investigación acerca de la estructura histórica del nuevo
  mundo y del sentido de su devenir}, 2nd ed. (Mexico City: Fondo de
  Cultura Económica, 1977), 152.} and irreversible, in the fulfillment of the
ecumenical program of the Culture of the West'' can be
understood.\footnote{O'Gorman, \emph{La invención de América}, 159.}
Seen from this perspective, in sum, colonial conservatism, and not ``the
Western,'' was and is the main enemy of the ``historical realization''
of America,\footnote{Julimar del Carmen Mora Silva, ``Utopias and
  Dystopias of Our History: Historiographical Approximation to `The
  Latin American' in the Mexican Social Thought of the 20th Century
  (Edmundo O'Gorman, Guillermo Bonfil Batalla and Leopoldo Zea),''
  \emph{História} \emph{da} \emph{Historiografia} 11, no. 28 (2018):
  209.} a polemical position that, although it has never been
predominant in historical-philosophical debates, retains a high
heuristic value in sociocultural terms and is worth pondering as a
contextual background for the institutionalization of academic fields
such as that of communication studies.\footnote{Raúl Fuentes-Navarro,
  ``Apresentação: Comunicação e fronteiras; Geografias e espaços
  simbólicos das práticas comunicativas na América Latina,'' in
  \emph{Fronteiras culturais e práticas comunicativas,} ed. Daniela
  Cristiane Ota and Marcus Paulo da Silva (Campo Grande, Brazil: Editora
  UFMS, 2023).} In addition to the imperative of escaping this
``colonial conservatism,'' it can be argued that even without
specialized training in the historical disciplines, as professionals in
higher education in communication we understand the need to strengthen
knowledge platforms that facilitate the historicization of our objects
of research and learning,\footnote{Benedict Anderson, \emph{Comunidades
  imaginadas: Reflexiones sobre el origen y la difusión del
  nacionalismo} (Mexico City: Fondo de Cultura Económica, 1993);
  Immanuel Wallerstein, ``From Sociology to Historical Social Science:
  Prospects and Obstacles,'' \emph{British Journal of Sociology} 51, no.
  1 (2000).} through a dense contextualization, in time and space, of
sociocultural processes that are obviously anything but simple and
straightforward, one-dimensional or constant.

Although it might seem obvious that the processes of constitution of
scientific, cultural, and educational institutions recognizable as
``academic fields'' have been diversely related to each other throughout
their history, the transnational factors of influence and
inter-determination between such unequal entities as the American
nation-states, their development, or their culture, are scarcely
recognized. Consequently, this paper proposes to heuristically identify
some of the most relevant institutionalization trends of the field on
the American continental scale, analyzable as differential structuring
processes and their mutual constants and influences, that is, their
``internationalization,'' in different spaces and times, as academic
proposals that have been articulated in a very special way by academic
associations, and thus have counteracted the disarticulating influences
of other factors.\footnote{Associations, as well as publications and
  university educational programs, are a concrete product of the
  interactions between projects and the sociocultural dynamics of the
  academic fields to which they belong and of the groups of subjects
  they represent. As such, they are \emph{objective referents} of the
  processes of institutionalization. Beyond their utility as objects of
  critical reflection, these processes are also a reflection of the
  objectives that motivate university formative interventions. As these
  processes are communicatively mediated, they can therefore be
  interpreted---from}

This idea is proposed as part of a line of work that has been developed
within the frameworks of a critical, sociocultural, processual, and
historical-structural sociology of science, based on contributions from
several authors,\textsuperscript{14} all
read from a Latin American perspective, and sustains some points of
coincidence with better known analyses in the international literature
that refer to other countries and regions.\textsuperscript{15}

The reflective and critical contributions of particular individuals and
academic communities are at the same time indispensable references for
the historical reconstruction of the field in the Latin American region
and require us to recognize that both the field itself and its study are
complex and heterogeneous, and that therefore simultaneous tendencies of
convergence and fragmentation are manifested\marginnote{a praxeological model of institutionalized
  collective agency (consisting of the exercise of ``the social
  production of meaning on the social production of meaning'')---as key
  factors in the generation of the dynamics of the field. Raúl
  Fuentes-Navarro, ``La producción social de sentido sobre la producción
  social de sentido: De un marco epistemológico a un modelo metodológico
  mediado por la metainvestigación,'' in \emph{Experiências
  metodológicas na comunicação}, ed. Laura Wottrich and Nísia M. do
  Rosário (São Paulo: Pimenta Cultural, 2022).} in\marginnote{\textsuperscript{14} Bourdieu, \emph{Science of Science and
  Reflexivity}; Anthony Giddens, \emph{The Constitution of Society:
  Outline of the Theory of Structuration} (Berkeley: University of
  California Press, 1984); Andrew Abbott, \emph{Methods of Discovery:
  Heuristics for the Social Sciences} (New York: W. W. Norton, 2004);
  Abbott, \emph{Processual Sociology} (Chicago: University of Chicago
  Press, 2016); Immanuel Wallerstein, \emph{Impensar las ciencias
  sociales: Límites de los paradigmas decimonónicos} (Mexico City: Siglo
  XXI / CIICH UNAM, 1998); Wallerstein, ``From Sociology to Historical
  Social Science''; Wallerstein, \emph{The Uncertainties of Knowledge}
  (Philadelphia: Temple University Press, 2004); Enrique E.
  Sánchez-Ruiz, \emph{Medios de difusión y sociedad: Notas críticas y
  metodológicas} (Guadalajara: Universidad de Guadalajara, 1992);
  Sánchez-Ruiz, ``Recuperar la crítica: Algunas reflexiones personales
  en torno al estudio de las industrias culturales en Iberoamérica en
  los últimos decenios,'' in \emph{Qué pasa con el estudio de los
  medios: Diálogo con las ciencias sociales en Iberoamérica}, ed.
  Sánchez-Ruiz (Sevilla, Spain: Comunicación Social ediciones y publicaciones,
  2011); Fernando Henrique Cardoso and Enzo Faletto, \emph{Dependencia y
  desarrollo en América Latina} (Mexico City: Siglo XXI, 1969).} academic\marginnote{\textsuperscript{15}\setcounter{footnote}{15} Manuel Parés i
  Maicas, ed., ``La recerca europea em Comunicació Social,''
  \emph{Anàlisi, quaderns de comunicació i cultura}, no. 27 (1997);
  Robert T. Craig, ``Communication Theory as a Field,''
  \emph{Communication Theory} 9, no. 2 (1999); Craig, ``Communication as
  a Field and Discipline,'' in vol. 2 of \emph{The International
  Encyclopedia of Communication}, ed. Wolfgang Donsbach (New York:
  Blackwell, 2008); Timothy Glander, \emph{Origins of Mass Communication
  Research during the Cold War: Educational Efforts and Contemporary
  Implications} (Mahwah, NJ: Lawrence Erlbaum, 2000); Juha Koivisto and
  Peter D. Thomas, eds., \emph{Mapping}} production and
in its meta-research.\textsuperscript{16} In this sense, the hypothetical
characterization of communication research in Latin America as subject
to a ``disintegrated internationalization,'' and the aim to promote a
collective reflection based on systematic and rigorously analyzed
documentation, ``in the manner of Luis Ramiro Beltrán
(1974),''\textsuperscript{17} incorporate, among others, contributions
by Jesús Martín-Barbero, José Marques de Melo, Carlos
Gómez-Palacio, Guillermo Orozco, Gustavo Adolfo León, and Erick Torrico,\textsuperscript{18}
as well as those contained in several empirical studies of scientific
production in the field.\textsuperscript{19}

\hypertarget{convergences-divergences-and-fragmentations-disintegrated-internationalization}{%
\section{Convergences, Divergences, and Fragmentations: Disintegrated\\\noindent Internationalization}\label{convergences-divergences-and-fragmentations-disintegrated-internationalization}}

While allusions to the relations between national entities often refer
to the past---a point in time before it became commonplace to speculate
that the very idea of the nation-state, a category of Western modernity,
was ``on the verge of extinction'' and that attention should be paid to
the processes of transition from the ``transnational'' to the
``post-national''---the geopolitics of the most recent decades requires
a more critical reconsideration.\textsuperscript{20} It can be argued from this point
of view that the existence in Latin America of diverse and changing
national (not all ``nationalist'') patterns of development\textsuperscript{21} has not substantially
modified the relations between states, which are for the most part
peaceful despite multiple ongoing conflicts.\textsuperscript{22} However, the general conditions and rhetoric of
``integration,'' a subject and term that frequently appears in Latin
American communication research of the 1970s and 1980s,\textsuperscript{23} as well as in much of the intergovernmental discourse of the
time, have substantially changed.

Within this general historical context, communication teaching and
research that can be properly called ``Latin American'' have been
supported by institutions created precisely for that purpose, especially
CIESPAL (International Center for Higher Studies in Journalism {[}later,
Communication{]} for Latin America), an international organization
operating since 1959 with headquarters in Quito; ALAIC (Latin American
Association of Communication Researchers), established in 1977; and
FELAFACS (Latin American Federation of Social Communication Schools and
Faculties), established in 1981. More than others, these three
institutions have been assumed as fundamental referents for articulating
reflection with action, attention to national processes with the
construction of continental links, and the formulation of critical
proposals of broad coverage---all on a regional scale. Many of these
proposals have been adopted as common challenges by academic communities
as disparate as those that have developed over\marginnote{\emph{Communication and Media Research:
  Paradigms, Institutions, Challenges} (Helsinki: University of
  Helsinki, Communication Research Center, 2008); Maria Löblich and
  Andreas Matthias Scheu, ``Writing the History of Communication
  Studies: A Sociology of Science Approach,'' \emph{Communication
  Theory} 21, no. 1 (2011); José Luis Piñuel-Raigada, \emph{La docencia y la
  investigación universitarias en torno a la comunicación como objeto de
  estudio en Europa y América Latina}, Colección Cuadernos Artesanos de
  Latina 15 (La Laguna, Spain: Sociedad Latina de Comunicación Social,
  2011); Silvio Waisbord, ``Communication Studies without Frontiers?
  Translation and Cosmopolitanism across Academic Cultures,''
  \emph{International Journal of Communication} 10 (2016); Waisbord,
  \emph{Communication: A Post-Discipline} (Cambridge, MA: Polity Press,
  2019); Jefferson D. Pooley, ``The Four Cultures: Media Studies at the
  Crossroads,'' \emph{Social Media and Society} 2, no. 1 (2016); Pooley,
  ``Die abnehmende Bedeutung des disziplinären Gedächtnisses: Der Fall
  der Kommunikationsforschung,'' in \emph{Handbuch
  kommunikationswissenschaftliche Erinnerungsforschung}, ed. Christian
  Pentzold and Christine Lohmeier (Berlin: De Gruyter, 2023); Stefanie
  Averbeck-Lietz, ``Communication Studies Beyond the National:
  Connections and Disconnections Between Research Communities and How to
  Study Them,'' \emph{Global Media Journal: German Edition} 2, no. 2
  (2012); Stefanie Averbeck-Lietz and Sarah Cordonnier, ``French and
  German Theories of Communication: Comparative Perspectives with Regard
  to the Social and the Epistemological Body of Science,'' in \emph{The
  Handbook of Global Interventions in Communication Theory}, ed.
  Yoshitaka Miike and Jing Yin (New York: Routledge, 2022); Marton
  Demeter, Dina Vozab, and Francisco José Segado-Boj, ``From
  Westernization to Internationalization: Research Collaboration
  Networks of Communication Scholars from Central and Eastern Europe,''
  \emph{International Journal of Communication} 17 (2023).} the\marginnote{\textsuperscript{16} Raúl Fuentes-Navarro, ``Investigación y
  meta-investigación sobre comunicación en América Latina,''
  \emph{MATRIZes} 13, no. 1 (2019).} last\marginnote{\textsuperscript{17} Raúl Fuentes-Navarro, ``La investigación de la
  comunicación en América Latina: una internacionalización
  desintegrada,'' \emph{Oficios Terrestres}, no. 31 (2014);
  Fuentes-Navarro, ``Tendencias regionales y transnacionales de la
  investigación de la comunicación en América Latina,'' in
  \emph{Tejiendo nuestra historia: investigación de la comunicación en
  América Latina}, ed. Delia Crovi Druetta and Raúl Trejo Delarbre
  (Mexico City: UNAM, 2018).} half century in
Latin America, in a context that has been characterized as one of
``multiple disarticulation'' and ``triple marginality'' of communication
studies with respect to the social sciences, to scientific research in
general, and to the priorities of national development.\textsuperscript{24}

According to a summary of diverse sources, modern reflection on ``the
social'' began in Latin America between the last decades of the
nineteenth and the first decades of the twentieth centuries, in the form
of scholarly studies of a philosophical, legal, and historical
nature.\textsuperscript{25} The first studies on communication, referring to
journalism, followed this general pattern.\textsuperscript{26} Later, Latin America received
as part of the ``modernization'' process in the 1950s and 1960s the
dominant theories and methods of the North American social sciences
(empiricism, functionalism, diffusionism, and
developmentalism).\textsuperscript{27} In the field of
communication studies, research on effects, audiences, public opinion,
and the like then spread, along with the expansion of mass media,
advertising, and the commercial model of mass communication, all
imported as well from the United States.

In particular, the ``diffusion of innovations'' approach, first assessed
in the US Midwest, was extensively employed in the rural areas of
several Latin American countries, notably Colombia, Brazil, and Mexico,
to investigate the effects of small-scale social transformation
projects.\textsuperscript{28} The research thus conducted
was often characterized as ``dependent'' because Latin American scholars
pursuing graduate studies in the United States, upon returning to their
countries, served as mere field agents in large US-led
projects.\textsuperscript{29} In fact, although most of the academic
communication research in Latin America in the 1950s and 1960s was
conducted or directed by---or under the influence of---American
researchers,\textsuperscript{30} there were also
notable examples of critical contributions to theory and methodology by
some Latin Americans, such as Beltrán and Díaz-Bordenave, contributions
that were clearly recognized by their US colleagues.\textsuperscript{31}

The eventual emergence of ``hybrid perspectives'' in Latin America from
the ``empirical and critical schools,''\textsuperscript{32} could be imagined based
on the revitalizing movement of Latin American social sciences,
especially originating in Santiago de Chile, where several important
international research, teaching, and planning institutions, such as
FLACSO (Latin American Faculty of Social Sciences) and CEPAL (Economic
Commission for Latin America and the Caribbean), were operating. The
triumph of the Cuban revolution in 1959 was key to promoting critical
thinking in the academy and in societies, since it showed that there was
an option for socialist development within reach, which was viewed with
much optimism at first, given the multiple injustices, imbalances, and
contradictions evident in Latin American\marginnote{\textsuperscript{18} Jesús Martín-Barbero, ``Retos a la investigación
  de comunicación en América Latina,'' \emph{Comunicación y Cultura},
  no. 9 (1982); Martín-Barbero, ``Pensar la comunicación en
  Latinoamérica,'' \emph{Redes: Revista do Desenvolvimento Regional},
  no. 10 (2014); José Marques de Melo, ``La investigación
  latinoamericana en Comunicación,'' \emph{Chasqui}, no. 11 (1984);
  Marques de Melo, \emph{Entre el saber y el poder: Pensamiento
  comunicacional latinoamericano} (Monterrey, Mexico: Comité Regional
  Norte de Cooperación con la UNESCO, 2007); Carlos Gómez-Palacio y
  Campos, ``The Origins and Growth of Mass Communication Research in
  Latin America'' (PhD diss., Stanford University, 1989); Guillermo
  Orozco, \emph{La investigación de la comunicación dentro y fuera de
  América Latina: Tendencias, perspectivas y desafíos del estudio de los
  medios} (La Plata, Argentina: Universidad Nacional de La Plata, 1997);
  Gustavo Adolfo León Duarte, \emph{La nueva hegemonía en el pensamiento
  latinoamericano de la comunicación: Un acercamiento a la producción
  científica de la escuela latinoamericana de la comunicación}
  (Hermosillo, Mexico: Universidad de Sonora, 2007); Erick Torrico
  Villanueva, \emph{La comunicación pensada desde América Latina
  (1960--2009)} (Salamanca, Spain: Comunicación Social, ediciones y
  publicaciones, 2016).} countries.\marginnote{\textsuperscript{19} Steven H Chaffee, Carlos
  Gómez-Palacio, and Everett M. Rogers, ``Mass Communication Research in
  Latin America: Views from Here and There,'' \emph{Journalism \& Mass
  Communication Quarterly} 67, no. 4 (1990); Robert Huesca and Brenda
  Dervin, ``Theory and Practice in Latin American Alternative
  Communication Research,'' \emph{Journal of Communication} 44, no. 4
  (1994); Sarah Anne Ganter and Félix Ortega, ``The Invisibility of
  Latin American Scholarship in European Media and Communication
  Studies: Challenges and Opportunities of De-Westernization and
  Academic Cosmopolitanism,'' \emph{International Journal of
  Communication} 13 (2019); Martín Becerra and Florencia Enghel,
  ``Pluralismo agonista en la internacionalización de los estudios
  latinoamericanos de la comunicación: Reflexiones a partir de la
  práctica,'' \emph{Comunicación y Medios} 30, no. 43 (2021); Santiago
  Gándara and Yamila Heram, ``Los estudios Latinoamericanos de
  comunicación (2000--2018): Consolidación académica, estancamiento
  burocrático o dispersión temática?''}  At the same time,
both in communication studies and in the rest of the social sciences,
the search for pertinence of the analysis to the complex Latin American
realities made some scholars think that it was possible to generate
totally original and ``autochthonous'' theory, methodology, and even
epistemology, independent or even opposed to ``Western scientific
colonialism,'' a position that paradoxically, with European or American
support, continues to have an important presence today. But there is
also still present the conviction that the best Latin American
contributions to the social sciences have been the product of ``creative
syntheses'' of elements of diverse origins with locally generated
elements and facts pertinent to the concrete social reality, its
processes, and transformations.\textsuperscript{33} The theory or approach of dependency, and the
innovative approach of Paulo Freire's \emph{Pedagogy of the Oppressed},
strongly associated with Liberation Theology, are two good examples of
this.\textsuperscript{34}

On the bases of these and other antecedents---which it should be
emphasized were mostly ``external'' to the universities---the
institutionalization of communication research in Latin America and the
constitution of the corresponding academic field were clearly manifested
from the 1970s onwards. In documents from those years, in which for the
first time the diagnoses and strategic programs for the development of
``Latin American communication research'' were formulated, analyses can
be found that, updating the data, could very well refer to the present
day, although they undoubtedly also evidence the radical changes that
the decades have generated in the referents, contexts, and premises
adopted to guide this articulated development. Among such
``foundational'' documents are the final report of the Seminar on
Communication Research in Latin America organized by CIESPAL in La
Catalina, Costa Rica in September 1973\textsuperscript{35}; and the paper presented
by Luis Ramiro Beltrán at the IAMCR International Scientific Conference
in Leipzig a year later. Beltrán's text, entitled ``Communication
Research in Latin America: The Blindfolded Inquiry?''\textsuperscript{36}
was based on the documentation presented at the CIESPAL seminar and
similarly emphasized the scientific and social foundations that should
be developed for the area. The La Catalina seminar assumed a clearly and
forcefully normative tone in its report:

\begin{quote}
The main objective of research should be the critical analysis of the
role of communication at all levels of functioning, without omitting its
relations with internal domination and external dependence and the study
of new channels, media, messages, communication situations, etc., which
contribute to the process of social transformation.\textsuperscript{37}
\end{quote}

\noindent Undoubtedly, the drafters of this text were aware of the conditions
necessary to advance this objective. Particularly striking is their
warning\marginnote{\emph{Astrolabio}, no. 27 (2021);
  Francisco Segado-Boj, Juan José Prieto-Gutiérrez, and Jesús
  Díaz-Campo, ``Redes de coautorías de la investigación Española y
  Latinoamericana en comunicación (2000--2019): Cohesión interna y
  aislamiento transcontinental,'' \emph{Profesional de la información}
  30, no. 3 (2021); Jesús Arroyave Cabrera and Rafael González Pardo,
  ``Bibliometric Research on Communication in Scientific Journals in
  Latin America (2009--2018),'' \emph{Comunicar} 30, no. 70 (2022);
  María Elena Rodríguez-Benito, María Esther Pérez-Peláez, and Teresa
  Martín-García, ``Investigación en comunicación: Diferencias entre
  Península Ibérica y América Latina,'' \emph{Cuadernos.info}, no. 54
  (2023).} that\marginnote{\textsuperscript{20} Arie Kacowitz, ``América
  Latina en el mundo: Globalización, regionalización y fragmentación,''
  \emph{Nueva Sociedad}, no. 124 (2008); Andrés Serbin, ed.,
  \emph{América Latina y El Caribe Frente a un nuevo orden mundial:
  Poder, globalización y respuestas regionales} (Buenos Aires: Icaria
  Editorial / Ediciones CRIES, 2018).} ``up\marginnote{\textsuperscript{21} Anderson,
  \emph{Comunidades imaginadas}, 99--101.} to\marginnote{\textsuperscript{22} A 2022 BBC report
  (\url{https://www.bbc.com/mundo/noticias-america-latina-59579795})
  mentions seven ongoing territorial disputes, five of which have
  required intervention from the International Court of Justice at The
  Hague: Guyana vs. Venezuela over Essequibo; the border between Belize
  and Guatemala; Colombia vs. Nicaragua over the archipelago of San
  Andres, Providencia, and Santa Catalina; Chile vs. Bolivia over the
  Silala River; Honduras and Nicaragua vs. El Salvador over the Gulf of
  Fonseca; Argentina vs. Chile over the Drake Passage; and, of course,
  Argentina vs. the United Kingdom over the Falklands/Malvinas Islands
  and Antarctica.} now,\marginnote{\textsuperscript{23} Peter
  Schenkel, ``La importancia del consenso latinoamericano,'' \emph{Nueva
  Sociedad}, no. 15 (1974); Schenkel\emph{, La integración
  latinoamericana y el desarrollo} (Quito: CIESPAL, 1984);
  Fuentes-Navarro, \emph{Un campo cargado de futuro}; Luis Núñez and
  Beatriz Solís, eds., \emph{Comunicación, identidad e integración
  Latinoamericana} (Mexico City: FELAFACS / CONEICC / Universidad
  Iberoamericana, 1994); Néstor García-Canclini, ed., \emph{Culturas en
  globalización: América Latina-Europa-Estados Unidos; Libre comercio e
  integración} (Caracas: Nueva Sociedad /}  Latin America does not have a sufficient
number of specialists in research, since there is not even an
institution specialized in the training of high-level experts in this
field.''\textsuperscript{38} This lack would obviously be a determining factor
in attending to the three areas of research that should be considered a
priority: the formulation, refinement, and testing of theories and
methods on the various aspects of ``the communication process and its
relationship with the process of social transformation; the role of
communication in education; and the role of communication in popular
organization and mobilization.''\textsuperscript{39}

Context cannot be ignored here: amid the world oil supply crisis of the
1970s, Latin American countries were subject to the internal and
external contradictions of the Cold War and the ideological, political,
and economic polarization associated with that world order. Thus was the
incipient and precariously institutionalized field of Latin American
communication research marked by all the contradictions that---sometimes
with extreme violence---characterized the social dynamics in which
communication tended to be instrumentalized rather than researched. And
in this context, Luis Ramiro Beltrán, with greater clarity despite his
subtlety than the speakers at the La Catalina seminar, formulated what
can be considered the \emph{essential tension} of communication research
in Latin America throughout its very intense and complex history: the
relationship between scientific rigor and dogmatic thinking. This
``tension,'' which is much more than a mere epistemological problem, is
an eloquent synthesis of the various dimensions of ``disintegrated
internationalization.'' Beltrán's work, in fact, ended with critical
comments on ``the mythology of a science free of values'' and on ``the
risk of dogmatism,'' whether in the form of the postulates of classical
liberalism or Marxism. In particular, the opposition between the rigor
of science and the political commitment to the transformation of
reality---referring directly to the polemic that had just begun between
the groups of researchers led by Armand Mattelart in Chile and Eliseo
Verón in Argentina---gave rise to a final crucial question:

\begin{quote}
Could this mean that Latin American communication research will one day
run into danger of substituting ideologically conservative and
methodologically rigorous functionalism with unrigorous radicalism? May
the patient reader of this already too lengthy report kindly provide an
answer. And may that answer give us lucid clues as whether Latin
American's communication research will cease to be the blind-folded
search which it appears at times to have been . . . be that blindfold of
any color.\textsuperscript{40}
\end{quote}

\noindent Despite the fact that this ``essential tension'' that can be identified
in the ``blindfolded research'' tends easily to be reduced to a
Manichean opposition,\marginnote{Seminario de Estudios de la
  Cultura / CLACSO, 1996); Franz Portugal Bernedo, ed., \emph{La
  investigación en comunicación social en América Latina 1970--2000}
  (Lima: Universidad Nacional Mayor de San Marcos, 2000); Carlos Véjar
  Pérez-Rulfo, ed., \emph{Globalización, Comunicación e integración
  Latinoamericana} (Mexico City: Plaza y Valdés / CIICH UNAM / UACM,
  2006).} to\marginnote{\textsuperscript{24} Enrique
  E. Sánchez-Ruiz and Raúl Fuentes-Navarro, \emph{Algunas condiciones
  para la investigación científica de la comunicación en México},
  Cuadernos Huella 17 (Guadalajara: ITESO, 1989), 12--13.} an\marginnote{\textsuperscript{25} Guillermo Boils Morales and Antonio Murga Frassinetti,
  eds., \emph{Las ciencias sociales en América Latina} (Mexico City:
  UNAM, 1979).} authoritarian\marginnote{\textsuperscript{26} Marques de Melo,
  ``La investigación Latinoamericana.''} \marginnote{\textsuperscript{27} Raúl Trejo-Delarbre, ``Seis décadas de
  investigación Latinoamericana sobre comunicación: Una propuesta de
  periodización,'' in \emph{Tejiendo nuestra historia: Investigación de
  la comunicación en América Latina}, ed. Delia Crovi Druetta and Raúl
  Trejo Delarbre (Mexico City: UNAM, 2018), 322.}``all\marginnote{\textsuperscript{28} Everett M. Rogers, ``Communication and Development:
  The Passing of the Dominant Paradigm,'' in \emph{Communication and
  Development: Critical Perspectives}, ed. Everett M. Rogers (Beverly
  Hills, CA: SAGE, 1976); Raúl Fuentes-Navarro, ``Everett M. Rogers
  (1931--2004) y la investigación Latinoamericana de la comunicación,''
  \emph{Comunicación y Sociedad} 4 (2005).} or\marginnote{\textsuperscript{29} Pablo González Casanova, \emph{Las categorías del
  desarrollo económico y la investigación en ciencias sociales} (Mexico
  City: UNAM, 1977).} nothing,''\marginnote{\textsuperscript{30} Luis Ramiro Beltrán, ``Alien Premises, Objects,
  and Methods in Latin American Communication Research,'' in
  \emph{Communication and Development: Critical Perspectives}, ed.
  Everett M. Rogers (Beverly Hills, CA: SAGE, 1976).} the\marginnote{\textsuperscript{31} Raúl
  Fuentes-Navarro, ``Latin American Interventions to the Practice and
  Theory of Communication and Social Development: On the Legacy of Juan
  Díaz Bordenave,'' in \emph{The Handbook of Global Interventions in
  Communication Theory}, ed. Yoshitaka Miike and Jing Yin (New York:
  Routledge, 2022).} very\marginnote{\textsuperscript{32} Everett M. Rogers,
  ``The Empirical and the Critical Schools of Communication Research,''
  \emph{ICA Communication Yearbook} 5 (1982).}
evolution of Beltrán's thought and discourse, as well as his continued
influence as an interlocutor or mediator between the academic
communities and the national and international public agencies of the
communications sector, maintained for decades a ``critical openness''
that was not the most common attitude in the field, but which, due to
his recognized intellectual authority, provided opportunities for
dialogue and mutual respect among many of the agents of the nascent
specialty, not yet clearly differentiated from journalism. Two
anthologies of his texts and interventions attest to this ability finely
cultivated by Beltrán.\textsuperscript{41} But finally, it is unquestionable that both at the
epistemological or methodological level and at the (axiological) level
of the transforming action of social communication systems and
practices, processes of fragmentation or multiple divergence have
replaced in Latin America the typical polarizations of other times,
unfortunately without reducing the risks of dogmatism.

\hypertarget{the-disjointed-development-of-latin-american-communicational-thought}{%
\section{The Disjointed Development of ``Latin American\\\noindent
Communicational
Thought''}\label{the-disjointed-development-of-latin-american-communicational-thought}}

In the academic spaces opened in several Latin American universities in
the 1970s, within journalism schools or outside them, including the
first postgraduate programs in communication, there was a proliferation
of theoretical-methodological perspectives imported and appropriated in
some way in each place. Within a few years, other European currents of
thought on the social---together with the influence of the ``Frankfurt
School,'' spread by pioneering authors such as Antonio
Pasquali,\textsuperscript{42} and the classical Marxism
already present in Latin America---were incorporated into the academic
field of communication studies, which was already incipiently
configured. These European influences included structuralism\textsuperscript{43} with linguistic roots and its
developments in semiology, psychoanalysis, and sociology, as well as the
influential ``structuralist'' Marxism of Louis Althusser and his
followers.

Then came the ``rediscovery'' of Antonio Gramsci's thought, particularly
in relation to popular culture studies, and the French school of
discourse analysis, among other contributions. But this constant flow of
analytical frameworks, which very often became nothing more than
intellectual fads, hindered the development of rational debates that
would focus the critical discussion of such frameworks to support their
relevance. However, on other occasions and in other places, these and
other analytical frameworks have been critically adopted, incorporated
into the intellectual heritage, and made relevant to the understanding
of Latin American realities through articulated empirical\marginnote{\textsuperscript{33} Sánchez-Ruiz, \emph{Medios de
  difusión y sociedad}.} research\marginnote{\textsuperscript{34} Fernando Henrique Cardoso and Enzo Faletto,
  \emph{Dependencia y desarrollo en América Latina} (Mexico City: Siglo
  XXI, 1969); Paulo Freire, \emph{Pedagogía del oprimido} (Mexico City:
  Siglo XXI, 1970).} and\marginnote{\textsuperscript{35} CIESPAL, ``Seminario
  sobre la investigación de la comunicación en América Latina: Informe
  provisional,'' \emph{Chasqui}, no. 4 (1973).}
practical\marginnote{\textsuperscript{36} Luis
  Ramiro Beltrán, ``Communication Research in Latin America: The
  Blindfolded Inquiry?'' (paper presented at International Association
  for Media and Communication Research, Leipzig, East Germany, 1974).} action\marginnote{\textsuperscript{37} CIESPAL,
  ``Seminario sobre la investigación de la comunicación,'' 15.}---central\marginnote{\textsuperscript{38} CIESPAL, ``Seminario sobre la investigación de la
  comunicación,'' 25.} formative\marginnote{\textsuperscript{39} CIESPAL, ``Seminario sobre la
  investigación de la comunicación,'' 18.} tasks\marginnote{\textsuperscript{40} Beltrán, ``Communication Research in Latin
  America,'' 40.} of\marginnote{\textsuperscript{41} Luis Ramiro Beltrán, \emph{Investigación
  sobre comunicación en Latinoamérica: Inicio, trascendencia y
  proyección} (La Paz: Plural Editores / Universidad Católica Boliviana,
  2000); Beltrán, \emph{Comunicación, política y desarrollo: Selección
  de textos publicados en la revista Chasqui entre 1982 y 2009} (Quito:
  CIESPAL, 2014).} graduate\marginnote{\textsuperscript{42} Antonio Pasquali, \emph{Comunicación y cultura de
  masas} (Caracas: Monte Avila, 1972); Pasquali\emph{, Comprender la
  comunicación} (Caracas: Monte Avila, 1978).} programs\marginnote{\textsuperscript{43}\setcounter{footnote}{43} Eliseo
  Verón, \emph{Conducta, estructura y comunicación}, 2nd ed. (Buenos
  Aires: Tiempo Contemporáneo, 1972).} and
fundamental objectives of research centers such as ININCO (Institute for
Communication Research, Central University of Venezuela) or ILET (Latin
American Institute for Transnational Studies), which promoted the strong
and vital presence of the ``Latin American critical current'' in
UNESCO's programs that led in 1980 to the well-known \emph{MacBride
Report}.\footnote{Sean MacBride et al., \emph{Un solo mundo, voces
  múltiples: Comunicación e información en nuestro tiempo} (Mexico City:
  Fondo de Cultura Económica, 1980).}

However, the focus of global attention and concern on ``communication''
amid the major world economic, political, and ideological-cultural
crises of the 1980s and onward, as well as the growth and institutional
consolidation of the academic field and the proliferation of multi-,
inter-, trans- and even ``post-disciplinary'' perspectives, generated in
Latin America a growing series of tensions that, in their most positive
aspect, were manifested in valuable contributions from Jesús
Martín-Barbero, Eliseo Verón, Renato Ortiz, and Néstor García
Canclini,\footnote{Jesús Martín-Barbero, \emph{De los medios a las
  mediaciones: Comunicación, cultura y hegemonía} (Mexico City: Gustavo
  Gili, 1987); Eliseo Verón, \emph{La semiosis social: Fragmentos de la
  una teoría de la discursividad} (Buenos Aires: Gedisa, 1988); Renato
  Ortiz, \emph{Cultura Brasileira e identidade nacional} (São Paulo:
  Brasiliense, 1985); Ortiz, \emph{A moderna tradição brasileira} (São
  Paulo: Brasiliense, 1988); Néstor García-Canclini, \emph{Culturas
  híbridas: Estrategias para entrar y salir de la modernidad} (Mexico
  City: Grijalbo, 1989).} among others, whose work brought together the
social sciences and culture and society studies and would come to be
identified (although not by them) as ``Latin American communicational
thought.'' This label itself merits some additional reflection.

One of the most elaborate and internationally appreciated formulations
of the label ``communicational thought'' is that of Bernard Miège,
clearly located in the French debate for the academic legitimization of
the Information and Communication Sciences, and sustained in the face of
the double tension between ``discipline'' and ``interdiscipline'' on the
one hand, and intellectual consistency and its instrumental uses on the
other. But ``the condition of this communicational thought is still
profoundly undecided,'' since it is at the same time an organizer of
scientific, reflexive, or professional practices and a response to the
demands of states and large organizations and an inspiration for changes
in them. ``In one word,'' Miège writes, ``{[}communicational thought{]}
can be at the origin of or accompany changes in cultural practices or
modes of dissemination or acquisition of knowledge.''\footnote{Bernard
  Miège, \emph{El pensamiento comunicacional} (Mexico City: Universidad
  Iberoamericana, Cátedra UNESCO de Comunicación, 1996), 9--10.} This
``profound indecision'' is also implicit in the version of the label
formulated in Brazil by José Marques de Melo\textsuperscript{47} about
Latin American ``communicational thought,'' although it refers more
emphatically to an explicit socio-political project, based on a highly
debatable but influential diagnosis:

\begin{quote}
The affirmation of the Latin American gaze, vindicating the
sociocultural identity of studies and research that for half a century
have been in the process of development in our mega-region, corresponds
to the purpose of confronting the traditional complex of the colonized.
Reflecting a type of congenital dependence, this distortion of
personality supports the production of theoretical frameworks generated
in ecologies that are distanced from our ways of being, thinking and
acting. Faced with challenges of this nature, the academic segment of
communication\marginnote{\textsuperscript{47}\setcounter{footnote}{47} José Marques de
  Melo, ``Communication Research: New Challenges of the Latin American
  School,'' \emph{Journal} \emph{of} \emph{Communication} 43, no. 4
  (1993); Marques de Melo, \emph{Entre el saber y el poder}.} in Latin America does not always react positively,
adopting a defensive behavior instead of occupying its rightful space at
the forefront of the world scientific community.\footnote{Marques de
  Melo, \emph{Entre el saber y el poder,} 16--17.}
\end{quote}

\noindent Although it could be mistaken as a re-edition of ``the blindfolded
inquiry,'' this Manichean formulation may have had more of a
``provocative'' intention, as part of a polemical and strategic
disposition of the leadership exercised by Marques de Melo to rescue the
``Latin American School'' in the face of the ``great fascination for
digital technologies and for the relations of sociability cultivated
through global computer networks'' of the new generation of students and
scholars of communication, and to the ``compulsory connection'' to the
process of scientific and technological globalization, towards which we
would be being led, ``in tune with the heralds of cultural
globalization, but without awareness of its effects, especially by the
gradual erosion of our regional/national identities.''\footnote{Marques
  de Melo, \emph{Entre el saber y el poder,} 383.} Although the Latin
American School ``has not yet conquered hegemony'' in the study of
communication in Latin America, its future ``depends basically on the
generational transition that is now in process,'' Marques de
Melo\footnote{Marques de Melo, \emph{Entre el saber y el poder,} 380,
  382.} affirmed. On this conviction, perhaps, he dedicated a great
effort throughout his life to the creation and institutional
strengthening of academic associations of different scales of coverage
(Brazilian, Lusophone, Latin American, Ibero-American).

Even earlier, in the context of the turn of the century, Latin America
had to ``accept that the times are not for synthesis,'' as
Martín-Barbero formulated it, and that it would be necessary to
``advance gropingly, without a map or with only a nocturnal map . . . a
map not for escape but for the recognition of the situation from the
mediations and the subjects,''\footnote{Martín-Barbero, \emph{De los
  medios a las mediaciones}, 229.} a position that apparently coincided
with the change of perspective pointed out in the journal
\emph{Comunicación y Cultura} by Héctor Schmucler, one of its editors,
although the differences between them were later emphasized. Schmucler
proposed in 1984 that ``communication is not everything, but it must be
spoken from everywhere; it must cease to be a constituted object, to be
an objective to be achieved. From culture . . . communication will have
a meaning transferable to everyday life.''\footnote{Héctor Schmucler,
  ``Un proyecto de comunicación/cultura,'' \emph{Comunicación y
  cultura}, no. 12 (1984): 8.}

Few of the texts that appeared in those years with self-critical reviews
of the past and prefigurations of the future of the field, written by
several of the most influential Latin American researchers, were
optimistic or inspired enthusiastic action. Martín-Barbero himself came,
a few years later, to promote with others a very radical criticism of
the academic field, and more precisely of the tendencies he perceived in
universities, related to ``the slowness, and even stagnation, of a
critical thinking that, entangled in the internal discussions of the
academy and ideological inertia, is unable to closely accompany the
transformations of the social and cultural real.''\footnote{Jesús
  Martín-Barbero, ed., \emph{Entre saberes desechables y saberes
  indispensables: Agendas de país desde la Comunicación} (Bogotá: Centro
  de Competencia en Comunicación para América Latina, 2009), 6.} In view
of this, Martín-Barbero and others argued, research should be better
linked to a ``country agenda.'' Despite the growth and institutional
consolidation of ``training'' programs in communication in all Latin
American countries, and some progress in research, ``linkage,'' and
internationalization, inertia and dissatisfaction have been constant, at
the same time as changes in communication systems, technologies,
policies, habits, and applications have proliferated.

It is then evident that the use of terms such as ``communicational
thought'' or ``country agenda'' is usually associated with a position in
a debate, in a struggle for domination (and denomination) of the field,
and that is why they serve to reconstruct a history in which certain
contributions are considered more valuable or significant than others:
to justify the perspectives adopted in the present, and from there to
draw lines of development and future action. Something similar can be
said to be proposed, albeit in different and divergent ways, by two
books of great erudition and critical acuity that appeared in the last
decade of the twentieth century outside the Latin American region, but
also of relevance there, since both display a history of the main
sources of influence exerted over the Latin American academic field and
communicational thought: from a French angle, \emph{The Invention of
Communication} by Armand Mattelart,\footnote{Armand Mattelart, \emph{La
  invención de la comunicación} (Mexico City: Siglo XXI, 1995).} and
from an American perspective, \emph{Speaking into the Air: A History of
the Idea of Communication} by John Durham Peters.\footnote{John Durham
  Peters, \emph{Speaking into the Air: A History of the Idea of
  Communication} (Chicago: University of Chicago Press, 1999).} The two
books have been translated to Spanish and Mattelart's also to
Portuguese. By way of contrast, Beltrán and co-authors published
\emph{La Comunicación antes de Colón: Tipos y formas en Mesoamérica y
Los Andes} (2008)\emph{,} a wonderful historical reconstruction of
ancient communication practices held by several civilizations before the
fifteenth century European arrival to ``America.''\footnote{Luis Ramiro
  Beltrán et al., \emph{La comunicación antes de Colón: Tipos y formas
  en Mesoamérica y los Andes} (La Paz: Centro Interdisciplinario
  Boliviano de Estudios de la Comunicación, 2008).} Not surprisingly,
this book has not been translated to other languages, and even the
Spanish edition is difficult to get outside Bolivia since there is no
digital edition.

However, the predominant tension in the Latin American academic field
since the 1990s centers not on ``historicization'' but rather on
diverging debates on the ``abandonment of critical
premises''\footnote{Carlos Ossandón, Claudio Salinas, and Hans Stange,
  \emph{La impostura crítica: Desventuras de la investigación en
  comunicación} (Salamanca, Spain: Comunicación Social, ediciones y
  publicaciones / ICEI Universidad de Chile, 2019).}; the adoption of
the ``inevitable validity'' of the laws of the market in the field of
research\footnote{Carlos Hoevel, \emph{La industria académica: La
  universidad bajo el imperio de la tecnocracia global} (Buenos Aires:
  Teseo, 2021).}; the dispersion of approaches to the multiple cultural
``mediations'' of social practices and ``mediatizations'' in
society\footnote{Mario Carlón, \emph{Circulación del sentido y
  construcción de colectivos: En una sociedad hipermediatizada} (San
  Luis, Argentina: Universidad Nacional de San Luis, 2020).}; or in
other directions, among which ``technologization'' has an important
place.\textsuperscript{60} In its most general lines, in the
second decade of the twenty-first century, this \emph{fragmentation}
could very well continue to describe Latin American communication
research, which has nevertheless advanced considerably in extension and
recognition, following very diverse and even divergent patterns of
academic institutionalization\marginnote{\textsuperscript{60}\setcounter{footnote}{60} Luciano Sanguinetti, \emph{Las revoluciones de la
  comunicación: Información, conocimiento y cultura; resistencia y
  hegemonía} (La Plata, Argentina: Universidad Nacional de La Plata,
  Facultad de Periodismo, 2021).} depending on the country in question, and
showing complex and opaque structural features that ``foster
asymmetrical pluralism throughout the region, such that certain Latin
American scholars are overrepresented in the most prestigious
international publications while others barely figure.''\footnote{Becerra
  and Enghel, ``Pluralismo agonista en la internacionalización,'' 31.}

Faced with the accumulation of complaints and data about the scarce
participation of Latin American production in the most globally
recognized academic journals, Florencia Enghel and Martín
Becerra\footnote{Florencia Enghel and Martín Becerra, ``Here and There:
  (Re)Situating Latin America in International Communication Theory,''
  \emph{Communication Theory} 28, no. 2 (2018).} coordinated a special
edition of \emph{Communication Theory} and produced an interesting
critical reflection on the matter that was subsequently published in
Spanish in Chile. In their journal article, the authors recognize that
inequity exists and should not be ignored by those who promote it or
benefit directly or indirectly from it. ``But paying attention to the
structural dimension means also taking into account the limitations
coming from Latin America itself.''\footnote{Becerra and Enghel,
  ``Pluralismo agonista en la internacionalización,'' 31.} The three
``structural'' factors they point out as determinants are, first, the
general context in which communication studies are produced in Latin
America, characterized ``by the scarcity of non-commercial public
service media systems'' and by ``the hyper-concentration of private
media ownership''; second, in relation to academic systems, the
``absence of public policies and resources---material and
symbolic---aimed at fostering the wide dissemination of knowledge
produced at national, regional and international levels''; and third,
that institutional pressure ``for Latin American communication and media
studies to adopt thematic concerns and publication formats typical of
academic work in the North erodes plurality and diversity for South and
North alike.''\footnote{Becerra and Enghel, ``Pluralismo agonista en la
  internacionalización,'' 30--31.}

  \enlargethispage{\baselineskip}

This last sentence by Becerra and Enghel, although they do not make it
explicit, can also be interpreted with regard to the development of
graduate programs (master's and doctorates) in communication in Latin
America, their remarkable growth since the 1990s, and the structural
conditions that characterize them differentially. Coordinated by Maria
Immacolata Vassallo de Lopes in 2011 as part of the Ibero-American
Forums organized by Confibercom,\footnote{Ibero-American Confederation
  of Scientific and Academic Communication Associations (Confederación
  Iberoamericana de Asociaciones Científicas y Académicas en
  Comunicación).} the study identified 249 master's and thirty-eight
doctoral programs in communication operating under very different
conditions and university regimes in nineteen Latin American countries.
Despite this large number, most were less than ten years old and had few
provisions for their ``internationalization.'' The Forum sought to
cooperatively meet the following \emph{initial} objectives:

\begin{quote}
1) To draw up a descriptive inventory of the postgraduate programs in
Communication operating in the Ibero-American region, understanding as
such the study programs aimed at obtaining doctoral and/or
master\textquotesingle s degrees, officially recognized by national
higher education regulations.

2) To identify national trends in the development of postgraduate
systems in Communication during the last decade, within the framework of
the educational legislation of each country, including especially the
evaluation and accreditation systems in force.

3) To explore the institutional frameworks of academic cooperation and
exchange and of the national mechanisms in force and susceptible to
being used for the internationalization of the programs.\footnote{Maria
  Immacolata Vassallo de Lopes, ed., \emph{Posgrados en comunicación en
  Iberoamérica: Políticas nacionales e internacionales} (São Paulo:
  Confibercom, 2012), 8.}
\end{quote}

Unfortunately, this Ibero-American project would not be renewed or
updated on an ongoing basis, nor was the project taken up by other
associations. The other two academic projects would suffer the same
abandonment, as did the international teams formed to develop the
``Forums'' of ``Public Policies of Communication'' and of ``Scientific
Journals of Communication Sciences,'' which together with that of
``Graduate Studies,'' were designed by a group of researchers led by
José Marques de Melo.\footnote{Margarida Maria Krohling-Kunsch, ed.,
  \emph{La comunicación en Iberoamérica: Políticas científicas y
  tecnológicas, posgrado y difusión del conocimiento} (Quito: CIESPAL /
  Confibercom, 2013).} Although the purpose was to build an entity that
would bring together the national and regional associations already
existing in Ibero-America, and the format designed for it seemed very
appropriate to strengthen internationalization, its institutionalization
lost continuity and viability in a brief time, coinciding with Marques
de Melo's illness and death in 2018.

\hypertarget{latin-american-academic-organizations-and-the-internationalization-of-the-field}{%
\section{Latin American Academic Organizations and the\\\noindent
Internationalization of the
Field}\label{latin-american-academic-organizations-and-the-internationalization-of-the-field}}

Over five decades, ALAIC and FELAFACS have alternately gone through
periods of growth and stability and periods of precarity, as has
CIESPAL, but with differences due to the latter's character as an
international organization which is subject to different structural
factors. At their best these associations have strengthened and expanded
the ``field'' through their contributions to the consolidation of trends
of institutional development and regional integration of academic
studies on communication; while, at their worst, they have failed to
close evident ``gaps.'' The associations' different compositions and
priority areas of intervention were conceived from the outset as
complementary, and on several occasions their leaders have prioritized
collaboration over the division of tasks, and many of their members have
been simultaneously or successively members or officials of both.
However, there have also been episodes of estrangement, mutual
discredit, or even political confrontation, such as when the Latin
American support for the IAMCR presidential candidate was divided in
1996.

As for other regions of the world, IAMCR has been an academic
organization of significant importance for the development of the
academic field of communication in Ibero-America.\footnote{Jörg Becker
  and Robin Mansell, eds., \emph{Reflections on the International
  Association for Media and Communication Research: Many Voices, One
  Forum} (Cham, Switzerland: Palgrave Macmillan, 2023).} Since its
foundation in 1957, due to direct promotion by UNESCO, IAMCR has
included members from Ibero-America, both individual and institutional,
and although it has never been notably intense, this participation has
been constant and has remained a factor that has ``modeled'' the
functions of international associations of regional scale. According to
Kaarle Nordenstreng, there is no doubt that the initiative to create
IAMCR was dominated by Europeans, particularly the French, ``but
colleagues from countries such as Brazil, Peru, Uruguay, Egypt, Israel,
India, Indonesia, Japan, Australia, the United States and Canada also
intervened.''\footnote{Kaarle Nordenstreng, ``Institutional Networking:
  The Story of the International Association for Media and Communication
  Research (IAMCR),'' in \emph{The History of Media and Communication
  Research: Contested Memories}, ed. David W. Park and Jefferson D.
  Pooley (New York: Peter Lang, 2008), 229.} Since 1968, when IAMCR met
in Pamplona, its conferences have been hosted by an Ibero-American city
eleven times: four in Spain, two in Brazil, two in Mexico, and one each
in Argentina, Venezuela, and Colombia---which constitutes 25 percent of
the venues in that period.

Simonson and Park assert that transnational academic associations ``have
been major forces in facilitating the flow of ideas and people,
solidifying hegemonic and counter-hegemonic paradigms and political
orientations to communication research and social networks of
scholars''\footnote{Peter Simonson and David W. Park, eds., \emph{The
  International History of Communication Study} (New York: Routledge,
  2016), 69--70.}; and Miquel de Moragas emphasizes the importance of
academic cooperation in countering the competitiveness that ``the logics
of current science policy tend to prioritize'' and urges that
globalization be understood not as unification but as ``interconnection
of nodes of influence.''\footnote{Miquel de Moragas, ``Las asociaciones
  de investigación de la comunicación: Funciones y retos'' (paper
  presented at the Encuentro Internacional de Asociaciones Académicas de
  Comunicación, Bilbao, Spain, 2014): 6.} By following these and other
confluent leads, it becomes increasingly clear that the focus of
attention on tensions and counter-positions is much more illuminating of
international institutionalization processes than unidirectional flows
of influences or resources, or the defense of national historical
``exceptionalisms.'' On the other hand, it also confirms that it is
methodologically convenient to define both spatial and temporal scales
to adequately and diversely contextualize transnationalization
processes.\footnote{Maria Löblich and Stefanie Averbeck-Lietz, ``The
  Transnational Flow of Ideas and \emph{Histoire Croisée} with Attention
  to the Cases of France and Germany,'' in \emph{The International
  History of Communication Study}, ed. Peter Simonson and David W. Park
  (New York: Routledge, 2016).}

It has already been pointed out that ALAIC was founded and developed in
its early years as a forum for ``counter-hegemonic action,'' and at the
same time ``intra-regional cooperation,'' in a socio-political context
of resistance to authoritarianism (particularly to the military
governments of the Southern Cone) and to what was then referred to as
``cultural imperialism.''\footnote{Martin Carnoy, \emph{La educación
  como imperialismo cultural} (Mexico City: Siglo XXI, 1977); Ariel
  Dorfman, \emph{Reader's nuestro que estás en la tierra: Ensayos sobre
  el imperialismo cultural} (Mexico City: Nueva Imagen, 1980).} Given
the scarcity and fragility of university institutions and academic
centers for communication research in Latin America during the sixties,
seventies, and eighties, the research and ``militant theory'' that
characterized the region were strongly international, or even
transnational, but were far removed from the more orthodox \emph{canons}
of scientific-academic institutionalization and university teaching.
Regarding communication and culture in Latin America until at least the
1990s, Luiz Gonzaga Motta writes that

\begin{quote}
\ldots{} the Weberian institutionalization of science (consecration of
behaviors of the ``scientific community'' through the assimilation of
social roles proper to science, such as political disinterest,
rationality, and emotional neutrality) has not taken place, at least not
in the North American molds. Scientific activity in communication during
the last three decades (perhaps excepting Brazil), has not been
institutionalized even in terms of the installation of a proper and
accepted place for research. In fact, the most significant scientific
production in this area took place, and still takes place, outside the
mechanisms of the state (universities, techno-bureaucracy,
etc.)\footnote{Luiz Gonzaga Motta, ``Las revistas de comunicación en
  América Latina: Creación de la teoría militante,'' \emph{Telos}, no.
  19 (1989): 150--51.}
\end{quote}

\noindent In this line, ALAIC was constituted in response to a growing interest in
socio-political processes related to the ``media'' yet not oriented
toward ``academization.'' As such, ALAIC promoted the articulation of
intellectual positions and research projects developed from thematically
specialized centers and institutes, with a focus on ``opposing'' social
movements, even as it also promoted the realization and publication of
bibliographic accounts of ``research in social communication'' and the
formation or incorporation of national associations of researchers.
Nonetheless, and for many reasons, ALAIC's activities decreased in
intensity and scope from the mid-1980s on, to the extent that a process
of ``reconstitution'' had to be initiated in 1988 on new bases and
impulses---precisely those typical of an international academic
organization, with not only socio-political but also scientific and
professional references. This institutional ``transformation'' has
endured such trials as the periodical renewal of its leadership, the
strengthening and expansion of joint projects with other regional and
global associations in the field, the biennial organization of
international congresses and seminars, and the maintenance of two
scientific journals: \emph{Revista Latinoamericana de Ciencias de la
Comunicación}, with forty-two issues edited in Portuguese and Spanish
since 2004; and \emph{Journal of Latin American Communication Research},
in English and Spanish, with eleven issues published since 2011. Despite
the growing disproportion between Brazil and the rest of the Latin
American countries in the number of qualified participants and the
degrees of institutionalization of the field,\footnote{Maria Immacolata
  Vassallo de Lopes and Richard Romancini, ``History of Communication
  Study in Brazil: The Institutionalization of an Interdisciplinary
  Field,'' in \emph{The International History of Communication Study},
  ed. Peter Simonson and David W. Park (New York: Routledge, 2016).}
ALAIC has managed to consolidate itself as a ``Latin American space,''
open both to the very diverse national and local realities of the region
and to contact and collaboration with academic bodies from other regions
of the planet.

Although its scope of action is not primarily research, FELAFACS has
also been a major driving force for academic support and dissemination
of Latin American research ever since its formation in October 1981. Its
work in bringing together university training programs, especially
undergraduate, and their national associations, and articulating them
through joint projects on a regional scale, had a determining influence
on the institutionalization of the academic field. The activities of
FELAFACS have been widely varied and influential, and have covered the
twenty countries of the region, although these activities have decreased
in recent years. Of note are the Latin American \emph{Encuentros}
(meetings) which FELAFACS convenes every two or three years, the
countless workshops and training seminars it organizes for teachers and
scholars across many different universities, cities, and countries, and
its publication of the journal \emph{Diá-logos de la
Comunicación}.\footnote{In its print version, fifty-seven issues
  published from 1987 to 2007 (after sixteen issues of \emph{Bulletin}),
  and then in digital format until issue ninety-two in 2016.
  Unfortunately, the collection is not currently available on the
  internet.} Together these resources have made it possible to
articulate and address the problems of teaching and, to a lesser extent,
of communication research around concrete, complex, and often rapidly
changing social situations. In an unpublished, internal ``evaluative
report'' from 1997, Wolfgang Donsbach described three basic functions,
``manifest and latent,'' fulfilled by the Federation:

\begin{quote}
First, the organization is a forum and a communicational network for
Latin American faculties of social communication. In this function, the
organization creates something like a common identity of Latin American
faculties of social communication.

Secondly, FELAFACS is an educational institution, dedicated to the
``training of trainers.'' In this sense, the organization creates
quality. This quality refers on the one hand to the teaching and
research work and on the other hand to the level of graduates of the
careers represented in FELAFACS.

Thirdly, FELAFACS fulfills the function of an institution that promotes
research. In this capacity, FELAFACS can contribute to obtaining more
knowledge about specific Latin American phenomena and problems in the
field of public communication.\footnote{Wolfgang Donsbach, ``Proyecto
  Federación Latinoamericana de Facultades de Comunicación Social
  (FELAFACS), Informe Evaluativo'' (unpublished report, Fundación Konrad
  Adenauer, Buenos Aires, 1997), 76, photocopy of a letter addressed to
  FELAFACS's president.}
\end{quote}

\noindent It is instructive to weigh this evaluation against FELAFACS's original
objectives. As Donsbach's description here suggests, the organization
was ``overtaken'' by the growth in the number of schools and students of
communication and the modification of national support policies in
several countries, but also by the generalized predominance of
professional training projects different from the ``humanistic'' ones
that motivated its constitution and by the very transformation of social
communication systems in local, national, and transnational contexts. It
remains to be seen, twenty-five years later, whether the efforts of the
current directors of the Federation will lead to renewed perspectives of
action and reflection as influential as those achieved in the past. The
``alliance'' with CIESPAL to jointly organize the First Congress of
Latin American Communication CIESPAL/FELAFACS in October 2023 will have
to have that renewing effect for both institutions and for the field.
Many agents of Latin American ``communication fields'' other than the
strictly academic or scientific ones were invited to join panels and
seminars set up for reflecting upon the ``Millennium Goals'' and to
participate in dialogues among diverse communities of ``communicators.''

For more than four decades, ALAIC and FELAFACS have been responsible for
facilitating the participation of researchers, professors, students, and
professionals in Latin American meetings that can be considered
``massive'' by bringing together hundreds or even thousands of
registered participants in both formal and informal academic activities.
Amid travel restrictions imposed by the COVID-19 pandemic, the two
institutions were able to replace face-to-face gatherings with
``online'' activities while ensuring a return to in-person formats as
soon as possible. It will soon be necessary to evaluate which of these
formats ought to be replicated, but surely ``live personal
communication'' will have to be preserved and strengthened in the field
of ``communication.''

For now, it should be noted that, during the last thirty years, Latin
America has regularly held one or two of these meetings per year, either
an ALAIC Congress or a FELAFACS \emph{Encuentro}, or both. There have
been a total of thirty-five Latin American meetings, held in fourteen
different countries: seven in Colombia; five in Peru, Mexico, and
Brazil; two in Chile, Argentina, and Uruguay; one in Panama, Venezuela,
Bolivia, Puerto Rico, Cuba, Costa Rica, and Ecuador. In all these
meetings, a Latin American community of agents of different generations
and specialties has been consolidated and strengthened, and a
predictable yet ambivalent result of this consolidation has been the
fragmentation and dispersion of the academic field, given that these
newly formed communities are tendentially specialized and have less and
less in common.

Obviously, this situation is perfectly common---in the disciplines and
in the specialties of research or action---and the way out may be the
same as was suggested by sociologist Craig Calhoun in the context of an
ICA conference some years ago:

\begin{quote}
In this heterogeneous field what is needed is not pressure towards
conformity but the production of more and better connections between
different lines of work. . . . {[}Theory{]} has a special role to play
in this, but asking the big questions that connect different lines of
work is something that goes far beyond the domain of theory.\footnote{Craig
  Calhoun, ``Communication as Social Science (and More),''
  \emph{International Journal of Communication} 5 (2011).}
\end{quote}

\hypertarget{towards-an-integrative-internationalization-of-the-latin-american-academic-field}{%
\section{Towards an Integrative Internationalization of the Latin\\\noindent
American Academic
Field}\label{towards-an-integrative-internationalization-of-the-latin-american-academic-field}}

The hypothetical characterization in recent decades of communication
research as subject to a ``disintegrated internationalization,'' and the
aim to promote a collective historical reflection based on systematic
and rigorously analyzed documentation recovered in this work, were also
the central axes of the inaugural keynote speech which this author
delivered under the title ``Memory and Historicity of Communication
Research in Latin America'' at the Fortieth Congress of the Sociedade
Brasileira de Estudos Interdisciplinares da Comunicação (INTERCOM),
thanks to a generous invitation aimed at ``internationalizing'' the
conference.\footnote{Raúl Fuentes-Navarro, ``Memoria e historicidad de
  la Investigación en Comunicación en América Latina'' (Opening Conference
  at the Congresso de la Sociedade Brasileira de Estudos
  Interdisciplinares da Comunicação {[}INTERCOM{]}, Curitiba, Brazil,
  September 6, 2017).} Both arguments imply the conviction and the
consequent recognition that the academic field of communication is
diverse and heterogeneous in Latin America, and therefore simultaneous
trends of convergence and fragmentation in academic production are
frequently and clearly manifested. What the main Latin American academic
associations have managed to consolidate in more than four decades is
still open to an undetermined future in terms of regional
``integration,'' but inescapably also to the ``international'' dynamics
of the field in the broadest sense---that is, on the ``global'' scale,
or at least on the scale of Western academia. One critical movement,
which has had its own international correspondences, has been the
recognition of women in the histories of national and Latin American
communication fields. Several books on this subject have been edited
recently, mainly following the initiative of Omar Rincón from the
Friedrich Ebert Stiftung Program for Latin America and the Caribbean,
whose anthology \emph{Mujeres de la Comunicación}\footnote{Clemencia
  Rodríguez et al., eds., \emph{Mujeres de la comunicación} (Bogotá: FES
  Comunicación, 2020); Vania Sandoval Arenas et al., eds., \emph{Mujeres
  de la comunicación: Bolivia} (La Paz, Bolivia: FES Comunicación,
  2022); Claudia Magallanes Blanco and Paola Ricaurte Quijano, eds.,
  \emph{Mujeres de la comunicación: México} (Mexico City: FES
  Comunicación, 2022); Alejandra García Vargas, Nancy Díaz Larrañaga,
  and Larisa Kejval, eds., vol. 1, \emph{Mujeres de la comunicación:
  Argentina} (Buenos Aires, FES Comunicación, 2022); Clemencia
  Rodríguez, Amparo Marroquín Parducci, and Omar Rincón, eds.,
  \emph{Mujeres de la comunicación 2: América Latina y el Caribe}
  (Bogotá: FES Comunicación, 2023). In 2021, \emph{Pioneras en los
  estudios latinoamericanos de comunicación}, edited by Yamila Heram and
  Santiago Gándara, appeared in Argentina. Among the six books, all of
  them with digital open circulation, 128 profiles of outstanding Latin
  American women in the field, were published.} offers a collection of
valuable yet stylistically heterogeneous profiles of women researchers.

While some scholars ``remain convinced that the field primarily reflects
external paradigms and concerns,'' Latin American development cannot be
reduced to a projection of foreign contributions.\footnote{Waisbord,
  ``Communication Studies without Frontiers?,'' 876.} Over a decade ago,
Miquel de Moragas pointed out that communication research in Latin
America ``is not homogeneous,'' and yet it responds to a particular,
common imperative: ``sharing diversity and deconstructing theoretical
apparatuses on communication based on the foreign experience of the
great metropolises of the developed Western world.''\footnote{Miquel de
  Moragas, \emph{Interpretar la comunicación: Estudios sobre medios en
  América y Europa} (Barcelona: Gedisa, 2011), 302.} It does not seem
viable or convenient to insist on a mere ``epistemological''
reconstruction of the evolution of ideas. Nor would it suffice to merely
acknowledge individual contributions, however extraordinary they may be,
or to rely solely on a deterministic explanation---wherein macroeconomic
and geopolitical factors are offered as an explanation for specific
modalities of cultural or even ideological reproduction of reductionist
notions of communication. Instead, the elaboration of a consistent and
guiding historical narrative needs to recognize the complexity and
multiplicity of the social-historical processes that have been
interwoven and interdetermined at each of the spatiotemporal scales that
it is relevant to define as one's own. As part of the social world,
``which is continually being made, un-made, re-made,'' the
social-scientific discourse implies a perpetual self-revision of its
premises and its proposals, its methods and its articulations with the
other practices and social structures it takes as its object.\footnote{Abbott,
  \emph{Processual Sociology}.} And thus, reflexively and critically,
through communication and not in some other way, will the international
integration Latin American communication scholarship be strengthened and
advanced.




\section{Bibliography}\label{bibliography}

\begin{hangparas}{.25in}{1} 



Abbott, Andrew. \emph{Methods of Discovery: Heuristics for the Social
Sciences}. New York: W. W. Norton, 2004.

Abbott, Andrew. \emph{Processual Sociology.} Chicago: University of
Chicago Press, 2016.

Anderson, Benedict. \emph{Comunidades imaginadas: Reflexiones sobre el
origen y la difusión del nacionalismo.} Mexico City: Fondo de Cultura
Económica, 1993.

Arroyave Cabrera, Jesús, and Rafael González Pardo. ``Bibliometric
Research on Communication in Scientific Journals in Latin America
(2009--2018).'' \emph{Comunicar} 30, no. 70 (2022): 85--96.
\url{https://doi.org/10.3916/C70-2022-07}.

Averbeck-Lietz, Stefanie. ``Communication Studies Beyond the National:
Connections and Disconnections Between Research Communities and How to
Study Them.'' \emph{Global Media Journal: German Edition} 2, no. 2
(2012): 1--10.
\url{https://www.globalmediajournal.de/index.php/gmj/article/view/113}.

Averbeck-Lietz, Stefanie, and Sarah Cordonnier. ``French and German
Theories of Communication: Comparative Perspectives with Regard to the
Social and the Epistemological Body of Science.'' In \emph{The Handbook
of Global Interventions in Communication Theory,} edited by Yoshitaka
Miike and Jing Yin, 373--92. New York: Routledge, 2022.
\url{https://doi.org/10.4324/9781003043348-28}.

Becerra, Martín, and Florencia Enghel. ``Pluralismo agonista en la
internacionalización de los estudios latinoamericanos de la
comunicación: Reflexiones a partir de la práctica.'' \emph{Comunicación
y Medios} 30, no. 43 (2021): 24--35.
\url{https://doi.org/10.5354/0719-1529.2021.60718}.

Becker, Jörg, and Robin Mansell, eds. \emph{Reflections on the
International Association for Media and Communication Research: Many
Voices, One Forum.} Cham, Switzerland: Palgrave Macmillan, 2023.
\url{https://doi.org/10.1007/978-3-031-16383-8}.

Beltrán, Luis Ramiro. ``Communication Research in Latin America: The
Blindfolded Inquiry?'' Paper presented at the International Association
for Media and Communication Research, Leipzig, East Germany, 1974.

Beltrán, Luis Ramiro. ``Alien Premises, Objects, and Methods in Latin
American Communication Research.'' In \emph{Communication and
Development: Critical Perspectives}, edited by Everett M. Rogers,
15--42. Beverly Hills, CA: Sage, 1976.

Beltrán, Luis Ramiro. \emph{Investigación sobre comunicación en
Latinoamérica: Inicio, trascendencia y proyección.} La Paz: Plural
Editores / Universidad Católica Boliviana, 2000.

Beltrán, Luis Ramiro. \emph{Comunicación, política y desarrollo:
Selección de textos publicados en la revista Chasqui entre 1982 y 2009.}
Quito: CIESPAL, 2014.

Beltrán, Luis Ramiro, Karina Herrera, Esperanza Pinto, and Erick
Torrico. \emph{La comunicación antes de Colón: Tipos y formas en
Mesoamérica y los Andes.} La Paz: Centro Interdisciplinario Boliviano de
Estudios de la Comunicación, 2008.

Boils Morales, Guillermo, and Antonio Murga Frassinetti, eds. \emph{Las
ciencias sociales en América Latina.} Mexico City: UNAM, 1979.

Bourdieu, Pierre. ``La specificité du champ scientifique et les
conditions sociales du progrès de la raison.'' \emph{Sociologie et
Societés} 7, no. 1 (1975): 91--118.

Bourdieu, Pierre. \emph{Homo Academicus.} Stanford: Stanford University
Press, 1988.

Bourdieu, Pierre. \emph{Science of Science and Reflexivity.} Chicago:
University of Chicago Press, 2004.

Calhoun, Craig. ``Communication as Social Science (and More).''
\emph{International Journal of Communication} 5 (2011): 1479--96.
\href{https://ijoc.org/index.php/ijoc/article/view/1331/622}{https:
/ijoc.org/index.php/ijoc/article/view/1331/622}.

Cardoso, Fernando Henrique, and Enzo Faletto. \emph{Dependencia y
desarrollo en América Latina.} Mexico City: Siglo XXI, 1969.

Carlón, Mario. \emph{Circulación del sentido y construcción de
colectivos: En una sociedad hipermediatizada.} San Luis, Argentina:
Universidad Nacional de San Luis, 2020.

Carnoy, Martin. \emph{La educación como imperialismo cultural.} Mexico
City: Siglo XXI, 1977.

Chaffee, Steven H., Carlos Gomez-Palacio, and Everett M. Rogers. ``Mass
Communication Research in Latin America: Views from Here and There.''
\emph{Journalism \& Mass Communication Quarterly} 67, no. 4 (1990):
1015--24. \url{https://doi.org/10.1177/107769909006700402}.

CIESPAL. ``Seminario sobre la investigación de la comunicación en
América Latina: Informe provisional.'' \emph{Chasqui: Revista
Latinoamericana de Comunicación}, no. 4 (1973): 11--25.
\url{https://revistachasqui.org/index.php/chasqui/article/view/2358}.

Craig, Robert T. ``Communication Theory as a Field.''
\emph{Communication Theory} 9, no. 2 (1999): 119--61.
\url{https://doi.org/10.1111/j.1468-2885.1999.tb00355.x}.

Craig, Robert T. ``Communication as a Field and Discipline.'' In vol. 2
of \emph{The International Encyclopedia of Communication,} edited by
Wolfgang Donsbach, 675--88. New York: Blackwell, 2008.

Demeter, Marton, Dina Vozab, and Francisco José Segado-Boj. ``From
Westernization to Internationalization: Research Collaboration Networks
of Communication Scholars from Central and Eastern Europe.''
\emph{International Journal of Communication} 17 (2023): 1211--31.
\url{https://ijoc.org/index.php/ijoc/article/view/20176/4046}.

Donsbach, Wolfgang. ``Proyecto Federación Latinoamericana de Facultades
de Comunicación Social (FELAFACS), Informe Evaluativo.'' Unpublished
report, Fundación Konrad Adenauer, Buenos Aires, 1997. Photocopy of a
letter addressed to FELAFACS's president.

Dorfman, Ariel. \emph{Reader's nuestro que estás en la tierra: Ensayos
sobre el imperialismo cultural.} Mexico City: Nueva Imagen, 1980.

Enghel, Florencia, and Martín Becerra. ``Here and There: (Re)Situating
Latin America in International Communication Theory.''
\emph{Communication Theory} 28, no. 2 (2018): 111--30.
\url{https://academic.oup.com/ct/article-abstract/28/2/111/4994890}.

Freire, Paulo. \emph{Pedagogía del oprimido}. Mexico City: Siglo XXI,
1970.

Fuentes-Navarro, Raúl. \emph{La emergencia de un campo académico:
Continuidad utópica y estructuración científica de la investigación de
la comunicación en México}. Guadalajara: ITESO/Universidad de
Guadalajara, 1998.

Fuentes-Navarro, Raúl. \emph{Un campo cargado de futuro: El estudio de
la comunicación en América Latina}. Mexico City: FELAFACS, 1992.

Fuentes-Navarro, Raúl. ``La investigación de la comunicación en América
Latina: condiciones y perspectivas para el siglo XXI.''
\emph{Comunicación y Sociedad}, no. 36 (1999): 105--32.

Fuentes-Navarro, Raúl. ``Everett M. Rogers (1931--2004) y la
investigación Latinoamericana de la comunicación.'' \emph{Comunicación y
Sociedad} 4 (2005): 93--125.
\url{https://www.comunicacionysociedad.cucsh.udg.mx/index.php/comsoc/article/view/4096/3853}.

Fuentes-Navarro, Raúl. ``La investigación de la comunicación en América
Latina: Una internacionalización desintegrada.'' \emph{Oficios
Terrestres}, no. 31 (2014): 11--22.
\url{http://perio.unlp.edu.ar/ojs/index.php/oficiosterrestres}.

Fuentes-Navarro, Raúl. ``Institutionalization and Internationalization
of the Field of Communication Studies in Mexico and Latin America.'' In
\emph{The International History of Communication Study,} edited by Peter
Simonson and David W. Park, 325--45. New York: Routledge, 2016.

\pagebreak Fuentes-Navarro, Raúl. ``Memoria e historicidad de la Investigación en
Comunicación en América Latina.'' Opening Conference at the Congresso de la
Sociedade Brasileira de Estudos Interdisciplinares da Comunicação
(INTERCOM), Curitiba, Brazil, September 6, 2017.

Fuentes-Navarro, Raúl. ``Tendencias regionales y transnacionales de la
investigación de la comunicación en América Latina.'' In \emph{Tejiendo
nuestra historia: Investigación de la comunicación en América Latina,}
edited by Delia Crovi Druetta and Raúl Trejo Delarbre, 295--315. Mexico
City: UNAM, 2018.

Fuentes-Navarro, Raúl. ``Investigación y meta-investigación sobre
comunicación en América Latina.'' \emph{MATRIZes} 3, no. 1 (2019):
27--48. \url{http://dx.doi.org/10.11606/issn.1982-8160.v13i1p27-48}.

Fuentes-Navarro, Raúl. ``La producción social de sentido sobre la
producción social de sentido: De un marco epistemológico a un modelo
metodológico mediado por la metainvestigación.'' In \emph{Experiências
metodológicas na comunicação,} edited by Laura Wottrich and Nísia M. do
Rosário, 343--56. São Paulo, Brazil: Pimenta Cultural, 2022.
\url{https://doi.org10.31560/pimentacultural/2022.95514}.

Fuentes-Navarro, Raúl. ``Latin American Interventions to the Practice
and Theory of Communication and Social Development: On the Legacy of
Juan Díaz Bordenave.'' In \emph{The Handbook of Global Interventions in
Communication Theory,} edited by Yoshitaka Miike and Jing Yin, 310--23.
New York: Routledge, 2022.
\url{https://doi.org/10.4324/9781003043348-23}.

Fuentes-Navarro, Raúl. ``Apresentação: Comunicação e fronteiras;
Geografias e espaços simbólicos das práticas comunicativas na América
Latina.'' In \emph{Fronteiras Culturais e Práticas Comunicativas,}
edited by Daniela Cristiane Ota and Marcus Paulo da Silva, 13--33. Campo
Grande, Brazil: Editora UFMS, 2023.

Gándara, Santiago, and Yamila Heram. ``Los estudios latinoamericanos de
comunicación (2000--2018): Consolidación académica, estancamiento
burocrático o dispersión temática?'' \emph{Astrolabio}, no. 27 (2021):
276--97. \url{https://doi.org/10.55441/1668.7515.n27.26556}.

Ganter, Sarah Anne, and Félix Ortega. ``The Invisibility of Latin
American Scholarship in European Media and Communication Studies:
Challenges and Opportunities of De-Westernization and Academic
Cosmopolitanism.'' \emph{International Journal of Communication} 13
(2019): 68--91.

García-Canclini, Néstor. \emph{Culturas híbridas: Estrategias para
entrar y salir de la modernidad.} Mexico City: Grijalbo, 1989.

\pagebreak García-Canclini, Néstor, ed. \emph{Culturas en globalización: América
Latina-Europa-Estados Unidos; libre comercio e integración.} Caracas:
Nueva Sociedad / Seminario de Estudios de la Cultura / CLACSO, 1996.

García Vargas, Alejandra, Nancy Díaz Larrañaga, and Larisa Kejval, eds.
Vol. 1, \emph{Mujeres de la comunicación: Argentina.} Buenos Aires, FES
Comunicación, 2022.

Giddens, Anthony. \emph{The Constitution of Society: Outline of the
Theory of Structuration.} Berkeley: University of California Press,
1984.

Glander, Timothy. \emph{Origins of Mass Communication Research during
the Cold War: Educational Efforts and Contemporary Implications.}
Mahwah, NJ: Lawrence Erlbaum, 2000.

Gómez-Palacio y Campos, Carlos. ``The Origins and Growth of Mass
Communication Research in Latin America.'' PhD diss., Stanford
University, 1989.

González-Casanova, Pablo. \emph{Las categorías del desarrollo económico
y la investigación en ciencias sociales}. Mexico City: UNAM, 1977.

Heram, Yamila, and Santiago Gándara, eds. \emph{Pioneras en los estudios
Latinoamericanos de comunicación.} Buenos Aires: Teseo Press, 2021.

Hoevel, Carlos. \emph{La industria académica: La universidad bajo el
imperio de la tecnocracia global.} Buenos Aires: Teseo, 2021.

Huesca, Robert, and Brenda Dervin. ``Theory and Practice in Latin
American Alternative Communication Research.'' \emph{Journal of
Communication} 44, no. 4 (1994): 53--73.

Ianni, Octavio. \emph{Enigmas da modernidade-mundo.} Rio de Janeiro:
Civilização Brasileira, 2000.

Kacowitz, Arie. ``América Latina en el mundo: globalización,
regionalización y fragmentación.'' \emph{Nueva Sociedad}, no. 124
(2008): 112--23.

Koivisto, Juha, and Peter D. Thomas, eds. \emph{Mapping Communication
and Media Research: Paradigms, Institutions, Challenges.} Helsinki:
University of Helsinki, Communication Research Center, 2008.

Krohling-Kunsch, Margarida Maria, ed. \emph{La comunicación en
Iberoamérica: Políticas científicas y tecnológicas, posgrado y difusión
del conocimiento.} Quito: CIESPAL / Confibercom, 2013.

León Duarte, Gustavo Adolfo. \emph{La nueva hegemonía en el pensamiento
latinoamericano de la comunicación: Un acercamiento a la producción
científica de la escuela latinoamericana de la comunicación.}
Hermosillo, Mexico: Universidad de Sonora, 2007.

Löblich, Maria, and Andreas Matthias Scheu. ``Writing the History of
Communication Studies: A Sociology of Science Approach.''
\emph{Communication Theory} 21, no. 1 (2011): 1--22.
\url{https://doi.org/10.1111/j.1468-2885.2010.01373.x}.

Löblich, Maria, and Stefanie Averbeck-Lietz. ``The Transnational Flow of
Ideas and \emph{Histoire} \emph{Croisée} with Attention to the Cases of
France and Germany.'' In \emph{The International History of
Communication Study}, edited by Peter Simonson and David W. Park,
25--46. New York: Routledge, 2016.

MacBride, Sean, et al. \emph{Un solo mundo, voces multiples:
Comunicación e Información en Nuestro Tiempo.} Mexico City: Fondo de
Cultura Económica, 1980.

Magallanes Blanco, Claudia, and Paola Ricaurte Quijano, eds.
\emph{Mujeres de la comunicación: México.} Mexico City: FES
Comunicación, 2022.

Marques de Melo, José. ``La investigación Latinoamericana en
comunicación.'' \emph{Chasqui: Revista Latinoamericana de Comunicación},
no. 11 (1984): 4--11.

Marques de Melo, José. ``Communication Research: New Challenges of the
Latin American School.'' \emph{Journal of Communication} 43, no. 4
(1993): 182--90.

Marques de Melo, José. \emph{Entre el saber y el poder: Pensamiento
comunicacional latinoamericano.} Monterrey, Mexico: Comité Regional
Norte de Cooperación con la UNESCO, 2007.

Martín-Barbero, Jesús. ``Retos a la investigación de comunicación en
América Latina.'' \emph{Comunicación y Cultura}, no. 9 (1982): 99--114.

Martín-Barbero, Jesús. \emph{De los medios a las mediaciones:
Comunicación, cultura y hegemonía}. Mexico City: Gustavo Gili, 1987.

Martín-Barbero, Jesús, ed. \emph{Entre saberes desechables y saberes
indispensables: Agendas de país desde la comunicación}. Bogotá: Centro
de Competencia en Comunicación para América Latina, 2009.

Martín-Barbero, Jesús. ``Pensar la comunicación en Latinoamérica.''
\emph{Redes: Revista do Desenvolvimento Regional}, no. 10 (2014):
20--39.

Mattelart, Armand. \emph{La invención de la comunicación.} Mexico City:
Siglo XXI, 1995.

Miège, Bernard. \emph{El pensamiento comunicacional.} Mexico City:
Universidad Iberoamericana, Cátedra UNESCO de Comunicación, 1996.

Mora Silva, Julimar del Carmen. ``Utopias and Dystopias of Our History:
Historiographical Approximation to `The Latin American' in the Mexican
Social Thought of the 20th Century (Edmundo O'Gorman, Guillermo Bonfil
Batalla and Leopoldo Zea).'' \emph{História da Historiografia} 11, no.
28 (2018): 195--218. \url{https://doi.org/10.15848/hh.v0i28.1232}.

Moragas, Miquel de. \emph{Interpretar la comunicación: Estudios sobre
medios en América y Europa.} Barcelona: Gedisa, 2011.

Moragas, Miquel de. ``Las asociaciones de investigación de la
comunicación: Funciones y retos.'' Paper presented at the Encuentro
Internacional de Asociaciones Académicas de Comunicación, Bilbao, Spain,
2014.

Motta, Luiz Gonzaga. ``Las revistas de comunicación en América Latina:
Creación de la teoría militante.'' \emph{Telos}, no. 19 (1989): 147--51.

Nordenstreng, Kaarle. ``Institutional Networking: The Story of the
International Association for Media and Communication Research
(IAMCR).'' In \emph{The History of Media and Communication Research:
Contested Memories}, edited by David W. Park and Jefferson Pooley,
225--48. New York: Peter Lang, 2008.

Núñez Gornés, Luis, and Beatriz Solís Leree, eds. \emph{Comunicación,
identidad e integración latinoamericana: VII Encuentro Latinoamericano
de Facultades de Comunicación Social.} Mexico City: FELAFACS / CONEICC /
Universidad Iberoamericana, 1994.

O'Gorman, Edmundo. \emph{La invención de América: Investigación acerca
de la estructura histórica del nuevo mundo y del sentido de su devenir}.
2nd ed. Mexico City: Fondo de Cultura Económica, 1977.

Orozco Gómez, Guillermo. \emph{La investigación de la comunicación
dentro y fuera de América Latina: Tendencias, perspectivas y desafíos
del estudio de los medios.} La Plata, Argentina: Universidad Nacional de
La Plata, 1997.

Ortiz, Renato. \emph{Cultura Brasileira e identidade nacional.} São
Paulo: Brasiliense, 1985.

Ortiz, Renato. \emph{A moderna tradição Brasileira}. São Paulo:
Brasiliense, 1988.

Ossandón, Carlos, Claudio Salinas, and Hans Stange. \emph{La impostura
crítica: Desventuras de la investigación en Comunicación.} Salamanca,
Spain: Comunicación Social, ediciones y publicaciones / ICEI Universidad
de Chile, 2019.

Parés i Maicas, Manuel, ed. ``La recerca europea em Comunicació
Social.'' \emph{Anàlisi, quaderns de comunicació i cultura}, no. 21
(1997): 1--283.

Pasquali, Antonio. \emph{Comunicación y cultura de masas.} Caracas:
Monte Avila, 1972.

Pasquali, Antonio. \emph{Comprender la comunicación.} Caracas: Monte
Avila, 1978.

Peters, John Durham. \emph{Speaking into the Air. A History of the Idea
of Communication.} Chicago: University of Chicago Press, 1999.

Piñuel-Raigada, José Luis. \emph{La docencia y la investigación
universitarias en torno a la Comunicación como objeto de estudio en
Europa y América Latina}. Colección Cuadernos Artesanos de Latina 15. La
Laguna, Spain: Sociedad Latina de Comunicación Social, 2011.

Pooley, Jefferson D. ``The Four Cultures: Media Studies at the
Crossroads.'' \emph{Social Media and Society} 2, no. 1 (2016): 1--4.
\url{https://doi.org/10.1177/2056305116632777}.

Pooley, Jefferson D. ``Die abnehmende Bedeutung des disziplinären
Gedächtnisses: Der Fall der Kommunikationsforschung.'' In
\emph{Handbuch} \emph{kommunikationswissenschaftliche}
\emph{Erinnerungsforschung}, edited by Christian Pentzold and Christine
Lohmeier, 369--90. Berlin: De Gruyter, 2023.

Portugal Bernedo, Franz, ed. \emph{La investigación en comunicación
social en América Latina 1970--2000.} 2nd ed. Lima: Universidad Nacional
Mayor de San Marcos, 2000.

Rodríguez, Clemencia, Claudia Magallanes Blanco, Amparo Marroquín
Parducci, and Omar Rincón, eds. \emph{Mujeres de la comunicación.}
Bogotá: FES Comunicación, 2020.

Rodríguez, Clemencia, Amparo Marroquín Parducci, and Omar Rincón, eds.
\emph{Mujeres de la comunicación 2: América Latina y el Caribe.} Bogotá:
FES Comunicación, 2023.

Rodríguez-Benito, María Elena, María Esther Pérez-Peláez, and Teresa
Martín-García. ``Investigación en comunicación: Diferencias entre
Península Ibérica y América Latina.'' \emph{Cuadernos.info}, no. 54
(2023): 182--204. \url{https://doi.org/10.7764/cdi.54.51309}.

Rogers, Everett M. ``Communication and Development: The Passing of the
Dominant Paradigm.'' In \emph{Communication and Development: Critical
Perspectives,} edited by Everett M. Rogers, 213--40. Beverly Hills, CA:
SAGE, 1976.

Rogers, Everett M. ``The Empirical and the Critical Schools of
Communication Research.'' \emph{ICA Communication Yearbook} 5 (1982):
125--44.

Salzano, Francisco M., and Maria C. Bortolini. \emph{The Evolution and
Genetics of Latin American Populations.} Cambridge: Cambridge University
Press, 2002.

Sánchez-Ruiz, Enrique E. \emph{Medios de difusión y sociedad: Notas
críticas y metodológicas}. Guadalajara: Universidad de Guadalajara,
1992.

Sánchez-Ruiz, Enrique E. ``Recuperar la crítica: Algunas reflexiones
personales en torno al estudio de las industrias culturales en
Iberoamérica en los últimos decenios.'' In \emph{Qué pasa con el estudio
de los medios:} \emph{Diálogo con las ciencias sociales en
Iberoamérica,} edited by Enrique E. Sánchez-Ruiz, 121--75. Sevilla, Spain:
Comunicación Social ediciones y publicaciones, 2011.

Sánchez-Ruiz, Enrique E., and Raúl Fuentes-Navarro. \emph{Algunas
condiciones para la investigación científica de la comunicación en
México.} Cuadernos Huella 17. Guadalajara: ITESO, 1989.

Sandoval Arenas, Vania, Rigliana Portugal Escobar, and Sandra Villegas
Taborga, eds. \emph{Mujeres de la comunicación: Bolivia.} La Paz,
Bolivia: FES Comunicación, 2022.

Sanguinetti, Luciano. \emph{Las revoluciones de la comunicación:
Información, conocimiento y cultura; resistencia y hegemonía.} La Plata,
Argentina: Universidad Nacional de La Plata, Facultad de Periodismo,
2021.

Schenkel, Peter. ``La importancia del consenso Latinoamericano.''
\emph{Nueva Sociedad}, no. 15 (1974): 3--10.

Schenkel, Peter. \emph{La integración latinoamericana y el desarrollo}.
Cuadernos de Chasqui 1. Quito: CIESPAL, 1984.

Schmucler, Héctor. ``Un proyecto de comunicación/cultura.''
\emph{Comunicación y Cultura}, no. 12 (1984): 3--8.

Segado-Boj, Francisco, Juan José Prieto-Gutiérrez, and Jesús Díaz-Campo.
``Redes de coautorías de la investigación Española y Latinoamericana en
comunicación (2000--2019): Cohesión interna y aislamiento
transcontinental.'' \emph{Profesional de la Información} 30, no. 3
(2021): 1--34. \url{https://doi.org/10.3145/epi.2021.may.05}.

Serbin, Andrés, ed. \emph{América Latina y el Caribe Frente a un nuevo
orden mundial: Poder, globalización y respuestas regionales.} Buenos
Aires: Icaria Editorial / Ediciones CRIES, 2018.

Simonson, Peter, and David W. Park, eds. \emph{The International History
of Communication Study}. New York: Routledge, 2016.

Torrico Villanueva, Erick. \emph{La comunicación pensada desde América
Latina (1960--2009)}. Salamanca, Spain: Comunicación Social, ediciones y
publicaciones, 2016.

Trejo-Delarbre, Raúl. ``Seis décadas de investigación latinoamericana
sobre comunicación: Una propuesta de periodización.'' In \emph{Tejiendo
nuestra historia: Investigación de la comunicación en América Latina,}
edited by Delia Crovi Druetta and Raúl Trejo Delarbre, 312--52. Mexico
City: UNAM, 2018.

Vassallo de Lopes, Maria Immacolata, ed. \emph{Posgrados en comunicación
en Iberoamérica: Políticas nacionales e internacionales.} São Paulo:
Confibercom, 2012.

Vassallo de Lopes, Maria Immacolata, and Richard Romancini. ``History of
Communication Study in Brazil: The Institutionalization of an
Interdisciplinary Field.'' In \emph{The International History of
Communication Study}, edited by Peter Simonson and David W. Park,
346--66. New York: Routledge, 2016.

Véjar Pérez-Rulfo, Carlos, ed. \emph{Globalización, comunicación e
integración Latinoamericana.} Mexico City: Plaza y Valdés / CIICH UNAM /
UACM, 2006.

Verón, Eliseo. \emph{Conducta, estructura y comunicación}. 2nd ed.
Buenos Aires: Tiempo Contemporáneo, 1972.

Verón, Eliseo. \emph{La semiosis social: Fragmentos de una teoría de la
discursividad.} Buenos Aires: Gedisa, 1988.

Waisbord, Silvio. ``Communication Studies without Frontiers? Translation
and Cosmopolitanism across Academic Cultures.'' \emph{International
Journal of Communication} 10 (2016): 868--86.

Waisbord, Silvio. \emph{Communication: A Post-Discipline.} Cambridge,
MA: Polity Press, 2019.

\pagebreak Wallerstein, Immanuel. \emph{Impensar las ciencias sociales: Límites de
los paradigmas decimonónicos.} Mexico City: Siglo XXI / CIICH UNAM,
1998.

Wallerstein, Immanuel. ``From Sociology to Historical Social Science:
Prospects and Obstacles.'' \emph{British Journal of Sociology} 51, no. 1
(2000): 25--35.

Wallerstein, Immanuel. \emph{The Uncertainties of Knowledge.}
Philadelphia: Temple University Press, 2004.



\end{hangparas}


\end{document}