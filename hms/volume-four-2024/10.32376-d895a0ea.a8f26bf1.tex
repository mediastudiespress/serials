% see the original template for more detail about bibliography, tables, etc: https://www.overleaf.com/latex/templates/handout-design-inspired-by-edward-tufte/dtsbhhkvghzz

\documentclass{tufte-handout}

%\geometry{showframe}% for debugging purposes -- displays the margins

\usepackage{amsmath}

\usepackage{hyperref}

\usepackage{fancyhdr}

\usepackage{hanging}

\hypersetup{colorlinks=true,allcolors=[RGB]{97,15,11}}

\fancyfoot[L]{\emph{History of Media Studies}, vol. 4, 2024}


% Set up the images/graphics package
\usepackage{graphicx}
\setkeys{Gin}{width=\linewidth,totalheight=\textheight,keepaspectratio}
\graphicspath{{graphics/}}

\title[Americas Introduction]{The History of Communication Studies across the Americas: An Introduction} % longtitle shouldn't be necessary

% The following package makes prettier tables.  We're all about the bling!
\usepackage{booktabs}

% The units package provides nice, non-stacked fractions and better spacing
% for units.
\usepackage{units}

% The fancyvrb package lets us customize the formatting of verbatim
% environments.  We use a slightly smaller font.
\usepackage{fancyvrb}
\fvset{fontsize=\normalsize}

% Small sections of multiple columns
\usepackage{multicol}

% Provides paragraphs of dummy text
\usepackage{lipsum}

% These commands are used to pretty-print LaTeX commands
\newcommand{\doccmd}[1]{\texttt{\textbackslash#1}}% command name -- adds backslash automatically
\newcommand{\docopt}[1]{\ensuremath{\langle}\textrm{\textit{#1}}\ensuremath{\rangle}}% optional command argument
\newcommand{\docarg}[1]{\textrm{\textit{#1}}}% (required) command argument
\newenvironment{docspec}{\begin{quote}\noindent}{\end{quote}}% command specification environment
\newcommand{\docenv}[1]{\textsf{#1}}% environment name
\newcommand{\docpkg}[1]{\texttt{#1}}% package name
\newcommand{\doccls}[1]{\texttt{#1}}% document class name
\newcommand{\docclsopt}[1]{\texttt{#1}}% document class option name


\begin{document}

\begin{titlepage}

\begin{fullwidth}
\noindent\Large\emph{History of Communication Studies across the Americas
} \hspace{18mm}\includegraphics[height=1cm]{logo3.png}\\
\noindent\hrulefill\\
\vspace*{1em}
\noindent{\Huge{The History of Communication Studies\\\noindent across the Americas: An Introduction\par}}

\vspace*{1.5em}

\noindent\LARGE{David W. Park}  \href{https://orcid.org/0000-0001-7019-1525}{\includegraphics[height=0.5cm]{orcid.png}}\par\marginnote{\emph{David W. Park, Jefferson Pooley, Peter Simonson, and Esperanza Herrero, ``The History of Communication Studies across the Americas: An Introduction,'' \emph{History of Media Studies} 4 (2024), \href{https://doi.org/10.32376/d895a0ea.a8f26bf1}{https://doi.org/ 10.32376/d895a0ea.a8f26bf1}.} \vspace*{0.75em}}
\vspace*{0.5em}
\noindent{{\large\emph{Lake Forest College}, \href{mailto:park@lakeforest.edu}{park@lakeforest.edu}\par}} \marginnote{\href{https://creativecommons.org/licenses/by-nc/4.0/}{\includegraphics[height=0.5cm]{by-nc.png}}}

\vspace*{0.75em} 

\noindent{\LARGE{Jefferson Pooley}  \href{https://orcid.org/0000-0002-3674-1930}{\includegraphics[height=0.5cm]{orcid.png}}\par}
\vspace*{0.5em}
\noindent{{\large\emph{Muhlenberg College}, \href{mailto:pooley@muhlenberg.edu}{pooley@muhlenberg.edu}\par}}

\vspace*{0.75em} % third author

\noindent{\LARGE{Peter Simonson}  \href{https://orcid.org/0000-0001-7156-467X}{\includegraphics[height=0.5cm]{orcid.png}}\par}
\vspace*{0.5em}
\noindent{{\large\emph{University of Colorado Boulder}, \href{mailto:peter.simonson@colorado.edu}{peter.simonson@colorado.edu}\par}}

\vspace*{0.75em} % third author

\noindent{\LARGE{Esperanza Herrero}  \href{https://orcid.org/0000-0001-5926-2142}{\includegraphics[height=0.5cm]{orcid.png}}\par}
\vspace*{0.5em}
\noindent{{\large\emph{Universidad de Murcia}, \href{mailto:mariaesperanza.herrero@um.es}{mariaesperanza.herrero@um.es}\par}}

\end{fullwidth}

\vspace*{1em}

\hypertarget{abstract}{%
\section{Abstract}\label{abstract}}

This special section investigates the history of communication and media studies across national and linguistic contexts in the Americas. It maps transnational entanglements that have shaped communication inquiry in the multiple forms it has taken in South and North America and the Caribbean. At the same time, the section’s articles attend to political, institutional, and cultural dynamics that shaped the field in different national and local contexts. In so doing, the special section throws light on topics and regions that have received little attention in English-language literature, and draws attention to historic lines of hegemony, exclusion, resistance, and alternative traditions of research across the hemisphere. In this editors’ introduction, we outline the origins of the collective effort, connect it to

\vspace*{4em}

\noindent{\emph{History of Media Studies}, vol. 4, 2024}


\newpage\noindent parallel projects in two Latin American journals, and introduce the outstanding essays that follow.

\hypertarget{resumen}{%
\section{Resumen}\label{resumen}}

Esta sección especial investiga la historia de los estudios sobre comunicación y medios en los diferentes contextos nacionales y/o lingüísticos de las Américas. Para ello, mapea los múltiples enlaces transnacionales que han dado forma a la investigación de la comunicación en sus vertientes norte y sudamericanas, así como en el Caribe. A la vez, los artículos de esta sección prestan atención particularmente a distintos contextos nacionales y locales. De esta manera, esta sección especial nos permite iluminar algunos de los temas y algunas de las regiones que han permanecido oscurecidas por la literatura anglofona, prestando especial atención a las líneas históricas de hegemonía, exclusión y resistencia, así como a las tradiciones de investigación alternativas que se han dado en el hemisferio. En esta introducción a cargo de los editores, señalamos los orígenes de este esfuerzo colectivo, lo conectamos con otros proyectos similares que se han dado en dos revistas latinoamericanas y, finalmente, introducimos los maravillosos ensayos que conforman esta sección especial.






 \end{titlepage}

Most of the essays in this special section have their genesis in a South-North collaboration
begun in late 2021. Earlier that year, the editors of this journal organized a virtual preconference for the meetings of the International
Communication Association (ICA), ``Exclusions in the History and
Historiography of Communication Studies.''\footnote{``Exclusions in the
  History and Historiography of Communication Studies/Exclusiones en la
  Historia e Historiografía de los Estudios de Comunicación,''
  International Communication Association preconference, May 26--27,
  2021, virtual.} The gathering was of a piece with the journal's
mission to decenter the centers that have traditionally structured the
historiography of the fields of media and communication
studies---especially around geographical region, language, gender, race,
and the legacies of colonialism. The preconference turned out to be one
of those pandemic-era events in which limitations---we couldn't meet in
person---afforded new possibilities. Latin Americans and other scholars
who would have been unlikely to attend ICA participated in the
gathering, and Zoom conferencing made simultaneous Spanish-English
interpretation easier to pull off. That event led to a special section
of this journal, with essays in Spanish and English.\footnote{Peter
  Simonson, David W. Park, and Jefferson Pooley, eds., ``Exclusions in
  the History of Media Studies/Exclusiones en la historia de los
  estudios de medios,'' special section, \emph{History of Media Studies}
  2 (2022).} It also raised the question as to whether it might be
productive to investigate the complex, politically fraught history of
communication and media studies within the geopolitical context of the
Americas writ large. That notion led to a second virtual conference in
June 2022, ``The History of Communication Studies across the
Americas.''\footnote{``Historia de los Estudios de Comunicación en las
  Américas/História dos Estudos de Comunicação nas Américas/History of
  Communication Studies across the Americas,'' History of Media Studies
  roundtable, July 12, 2022, virtual.} It was a collaborative effort by
three open-access journals published in three different countries:
\emph{History of Media Studies} from the US, \emph{MATRIZes} from
Brazil, and \emph{Comunicación y Sociedad} from Mexico.\footnote{The
  initial incubator for the project was a Spanish-language working group
  formed at \emph{History of Media Studies} with three advisory board
  members (Raúl Fuentes Navarro and Claudia Magallanes Blanco of Mexico
  and Mariano Zarowsky of Argentina) and one of the journal's editors
  (Pete Simonson from the US). Fuentes, doyen of the historiography of
  communication studies in Latin America, proposed and facilitated the
  collaboration with the other two journals and their editors: Maria
  Immacolata Vassallo de Lopes, editor of \emph{MATRIZes,} and Gabriela
  Gómez Rodriguez, editor of \emph{Comunicación y Sociedad}.} Twenty-two
scholars from eleven different countries participated, with simultaneous
interpretation in Spanish, Portuguese, and English.\textsuperscript{5}

One of the main aims of the conference---and by extension this special
section---was to promote communities of inquiry across regions and
languages that had not often been in dialogue with one another.
Professional associations in the Americas have facilitated some kinds\marginnote{\textsuperscript{5}\setcounter{footnote}{5} Participants
  were from Argentina, Bolivia, Brazil, Canada, Colombia, Costa Rica, El
  Salvador, Jamaica, Mexico, Puerto Rico, and the US, with simultaneous
  interpretation by Bárbara Barisch and her Argentinian colleagues.} of
contact but limited others. As Raúl Fuentes Navarro discusses in his
contribution to this special issue, the Asociación Latinoamericana de
Investigadores de la Comunicación (ALAIC) has provided forums and
created networks that span Spanish- and Portuguese-speaking Latin
America. The International Communication Association (ICA) has
historically done something similar for scholars from North America and
Western Europe who publish in English, even as it has often furthered US
hegemony.\footnote{Thomas Wiedemann and Michael Meyen,
  ``Internationalization through Americanization: The Expansion of the
  International Communication Association\textquotesingle s Leadership
  to the World,'' \emph{International Journal of Communication} 10
  (2016): 1489--1509.} Neither of those organizations has historically
included the Anglophone (or Francophone) Caribbean in its ken.
Additionally, national professional associations, particularly in the
larger countries, have exercised their own kinds of centripetal force.
The organizers of the 2022 conference asked if ``the Americas'' might
provide a shared, if essentially contested, intellectual space for
scholars from national contexts that have typically not been in
conversation with one another. Was there interest in trying to develop a
multi-layered map of the history of the field that traversed the region,
South to North?


This ``Americas'' project is one of several recent collaborations
dedicated to advancing North-South dialogue without erasing intellectual
and geopolitical specificity. Several years ago, US and German
historians of the field edited volumes with global
aspirations.\footnote{Peter Simonson and David W. Park, eds., \emph{The
  International History of Communication Study} (New York: Routledge,
  2016); and Stefanie Averbeck-Lietz, ed.,
  \emph{Kommunikationswissenschaft im internationalen Vergleich:
  Transnationale Perspektiven} (Wiesbaden, Germany: Springer Fachmedien
  Wiesbaden, 2017).} More recently, an international group of critical
scholars, many with ties to the Global South, have turned to history as
a way of decolonizing the field and drawing out previously marginalized
histories from around the world.\footnote{See, for example, a pair of
  linked preconferences: ``Media and Communication Studies in a Global
  Context: A Critical History,'' International Communication Association
  preconference, Toronto, May 25, 2023; and ``Repressed Histories in
  Communication and Media Studies,'' International Communication
  Association preconference, Gold Coast, Australia, June 20, 2024.}
Meanwhile, as the European Union has officially encouraged research
collaborations with Latin America and the Caribbean, scholars have built
bridges between the regions through collaborations among professional
associations. One result was a joint editorial effort by ALAIC and the
European Communication Research and Education Association (ECREA), which
published a volume on intellectual traditions of Latin American and
European communication studies.\footnote{Fernando Oliveira Paulino et
  al., eds., \emph{Research Traditions in Dialogue: Communication
  Studies in Latin America and Europe} (Porto, Portugal: Porto~Editora,
  2020). For other studies cutting across European and Latin American
  communication studies, see Sarah~Anne Ganter and Félix Ortega,
  ``The~Invisibility of Latin American Scholarship in~European Media and
  Communication Studies:~Challenges and Opportunities of
  De-Westernization~and Academic Cosmopolitanism,''~\emph{International
  Journal of~Communication}~13~(2019); Ana Rayen Dall'Orso,
  ``Investigación de la~Comunicación en Iberoamérica: Una
  Paleta~Diversa,~\emph{Cuadernos.info},\emph{~}no. 53 (2022); and María
  Elena Rodríguez~Benito,~María Esther Pérez-Peláez, and Teresa Martín
  García, ``Investigación~en Comunicación: Diferencias~entre Península
  Ibérica y América Latina,''~\emph{Cuadernos.info}, no. 54 (2023).} 
Owing to significant sociocultural and linguistic similarities, there
have also been multiple initiatives within Iberoamerican intellectual
networks investigating shared histories of communication studies among
Spanish- and Portuguese-speaking countries. Several of these networks
have focused on the historical and contemporary gendering of
communication research, including FEMICOMI (Analisis de los Roles
Femeninos en la Investigacion de la Comunicación en Iberoamerica
{[}Analysis of Females' Roles in Communication Research in
Iberoamerica{]}, begun in 2022) and IBERFEMCOM (Red Iberoamericana de
Investigacion en Comunicación y Feminismo {[}Iberoamerican Network for
Social Justice Research in Communication and Feminism{]}, begun in
2017). Other initiatives have opened complementary spaces for dialogue
on the history and present state of communication studies across
North-South lines in Iberoamerica.\footnote{See, for example,
  ``Comunicar {[}en{]} la Historia: Panorama científico de la Historia
  de la Comunicación Social en Iberoamérica; Intersecciones y marcos
  comparados,'' AE-IC \& AsHisCom conference, June 17--18, 2021,
  virtual; and ``IV Doctoral AE-IC: Taller iberoamericano de
  investigación en comunicación,'' AE-IC predoctoral conference,
  Pontevedra, Spain, June 15--16, 2023.}

These recent collaborations are set within a contemporary moment of
belated reckoning with the structural exclusions, inequalities, and
injustices that have helped constitute communication studies. Among the
many fronts for the reckoning are the historiographies and collective
memories of the field. We have yet to fully acknowledge, much less
historically unearth, all the ways that gender, race, language,
colonialism, geopolitical location, and institutionally sanctioned
privilege have shaped formal and informal accounts of our field's pasts.
Reversing these processes and recovering lost pasts requires multiple
methodologies and theoretical frameworks---from feminism and
transnational studies to the historical sociology of knowledge, critical
race theory, decolonial/postcolonial thought, and other geopolitically
informed critique, all of which are represented in this special
section's contributions.\footnote{This case is made more extensively in
  Peter Simonson, David W. Park, and Jefferson Pooley,
  ``Exclusions/Exclusiones: The Role for History in the Field's
  Reckoning,'' \emph{History of Media Studies} 2 (2022).}

\hypertarget{three-journals-in-dialogue}{%
\section{Three Journals in
Dialogue}\label{three-journals-in-dialogue}}

This \emph{History of Media Studies} special section is a companion to
rich collections published by \emph{MATRIZes} and \emph{Comunicación y
Sociedad} last year.\footnote{Raúl Fuentes Navarro, ed., ``Historias de
  los Estudios de Comunicación en las Américas,'' special section,
  \emph{Comunicación y Sociedad} 20 (2023), with essays in Spanish and
  English; and Maria Immacolata Vassallo de Lopes and Raúl Fuentes
  Navarro, eds., ``Histórias da internacionalização do campo de estudos
  da comunicação,'' special issue, \emph{MATRIZes} 17, no. 3 (2023),
  with contributions in Portuguese, Spanish, and English.} All three
special sections have their roots in the 2022 conference on ``The
History of Communication Studies across the Americas.'' Like the
conference itself, the three-journal collaboration is an enactment in
practice of a commitment to multi-lingual, cross-hemispheric cooperation
in charting the field's intersecting histories. Widely respected and
well-established journals, \emph{MATRIZes} (Brazil) and
\emph{Comunicación y Sociedad} (Mexico) are communication studies
exemplars of the pioneering Latin American tradition of fee-free open
access publishing.\footnote{See, for example, Dominique Babini, ``Toward
  a Global Open-Access Scholarly Communications System: A Developing
  Region Perspective,'' in \emph{Reassembling Scholarly Communications},
  ed. Martin Paul Eve and Jonathan Gray (Cambridge, MA: MIT Press,
  2020).} The two journals are models for our own commitments to diamond
open access and multi-lingualism---and, crucially, to our mission to
ventilate the US provincialism of much English-language historiography.
Here we identify shared themes in the two other collections, with the
aim to relate those themes to the six papers published here.

The three collections are bound, first, by the participation of Raúl
Fuentes Navarro, a leading historian of the field who sits on all three
journals' editorial boards. Fuentes Navarro's introductory essay leads
off the \emph{Comunicación y Sociedad} special section, which includes
three additional contributions, each published in both Spanish and
English.\footnote{Raúl Fuentes Navarro, ``Historias de los estudios de
  comunicación en las Américas/Histories of Communication Studies in the
  Americas,'' \emph{Comunicación y Sociedad} 20 (2023); Jesús Arroyave, ``Develando las razones del diálogo asimétrico. Explorando la
  exclusión en el campo de la comunicación/Unveiling the Reasons for
  Asymmetrical Dialogue: Exploring Exclusion in the Field of
  Communication,'' \emph{Comunicación y Sociedad} 20 (2023); Celia Del
  Palacio Montiel, ``Historia de los estudios de comunicación desde las
  regiones de América Latina. Las historias conectadas como recurso para
  el análisis/The History of Communication Studies from Regions of Latin
  America: Connected Histories as a Resource for the Analysis,''
  \emph{Comunicación y Sociedad} 20 (2023); and Eliseo R. Colón Zayas,
  ``Estudios de comunicación desde el pensamiento caribeño:
  Contribuciones de Luis Ramiro Beltrán, Frantz Fanon y Stuart Hall
  sobre desarrollo e identidad cultural/Communication Studies from
  Caribbean Thought: Contributions of Luis Ramiro Beltrán, Frantz Fanon
  and Stuart Hall on Development and Cultural Identity,''
  \emph{Comunicación y Sociedad} 20 (2023).} The \emph{MATRIZes} issue,
introduced by Maria Immacolata Vassallo De Lopes and Fuentes Navarro,
contains fourteen essays from a globe-spanning range of
contributors.\textsuperscript{15} The \emph{MATRIZes} collection
has the widest scope: the journal commissioned essays from a number of
scholars who had not participated in the 2022 conference, and cast the
issue under the broader rubric of ``Histórias da internacionalização do
campo de estudos da comunicação/Histories of the Internationalization of
the Field of Communication Studies.'' Despite the distinctive scopes,
the two special sections touch on a handful of shared themes.

The first is the most complicated to draw out, related as it is to the
foundational ``Americas'' framing of the 2022 conference. The premise, a
tentative one by design, was to cast the hemisphere as a space that is
both shared and contested. A small number of papers in the
\emph{MATRIZes} and \emph{Comunicación y Sociedad} collections set their
papers in full hemispheric relief.\textsuperscript{16} Only four of the seventeen papers mention the
``Americas'' at all, however, and one of these, by Mexican press
historian Celia del Palacio Montiel, takes up the frame in order to
criticize\marginnote{\textsuperscript{15} Maria Immacolata Vassallo De Lopes and Raúl
  Fuentes Navarro, ``Histórias da internacionalização do campo de
  estudos da comunicação/Histories of the Internationalization of the
  Field of Communication Studies,'' \emph{MATRIZes} 17, no. 3 (2023);
  Muniz Sodré, ``A ruptura paradigmática da comunicação/A Paradigmatic
  Rupture in Communication,'' \emph{MATRIZes} 17, no. 3 (2023); Paulo
  Serra, ``O espaço ibero-americano de ciências da comunicação e as
  epistemologias do Sul/The Ibero-American Space of Communication
  Sciences and the Epistemologies of the South,'' \emph{MATRIZes} 17,
  no. 3 (2023); Erick Rolando Torrico Villanueva, ``Colonialidade do
  saber na internacionalização dos estudos sobre comunicação: Abordagem
  do caso da América Latina,'' \emph{MATRIZes} 17, no. 3 (2023);
  Francisco Rüdiger,``Adeus à crítica?: passado e presente da teoria e
  método na pesquisa em comunicação de massa/Farewell to Critique? Past
  and Present of Theory and Method in Mass Communication Research,''
  \emph{MATRIZes} 17, no. 3 (2023); Carlos Sandoval García, ``Textos,
  audiencias y medios de comunicación: La persistencia de las
  preguntas,'' \emph{MATRIZes} 17, no. 3 (2023); Gustavo Adolfo
  León-Duarte, ``Cruces y límites en la investigación sobre
  comunicación: El sentido práctico interdisciplinar,'' \emph{MATRIZes}
  17, no. 3 (2023); Miquel de Moragas Spà, ``Investigar la comunicación:
  Entre el pasado y la prospectiva,'' \emph{MATRIZes} 17, no. 3 (2023);
  Delia Crovi Druetta, ``Travesía de la comunicación latinoamericana
  hacia su internacionalización,'' \emph{MATRIZes} 17, no. 3 (2023);
  Fernando Oliveira Paulino, ``América Latina, internacionalização e
  reciprocidade acadêmica/Latin America, Internationalization, and
  Academic Reciprocity,'' \emph{MATRIZes} 17, no. 3 (2023); Peter
  Simonson, Jefferson Pooley, and David Park, ``The History of
  Communication Studies across the Americas: A View from the United
  States,'' \emph{MATRIZes} 17, no. 3 (2023); Gabriela Rosa Cicalese,
  ``Internacionalización y raíces identitarias de la comunicación en
  Argentina,'' \emph{MATRIZes} 17, no. 3 (2023); Stefanie
  Averbeck-Lietz, ``On (Missing) Links between German, Latin American,
  and French Mediatization Research: Reflections on Diverse Research
  Milieus and Their Traditions,'' \emph{MATRIZes} 17, no. 3 (2023); Eva
  Da Porta, ``La internacionalización de la investigación en
  comunicación:} its application.\textsuperscript{17} ``Is it possible, even
pertinent, to carry out a history of communication studies throughout
the Americas?'' Palacio Montiel doesn't rule out the possibility in the
future, but worries that a ``generalizing project'' centered on the
hemisphere could render the distinctive characteristics and histories of
Latin American communication research invisible.\textsuperscript{18} Most of the other articles endorse, if only
implicitly, Palacio Montiel's warning, through their framing choices.
They place the Latin American tradition at their center, with other
regions---including Europe, the US, and the Global North at large---set
in complicated relief. The geographic footprint of the collections
centers on Latin America, with a dotted line north to the US, and across
the Atlantic to the Iberian peninsula and on to France.\textsuperscript{19}
Anglophone Canada, Quebec, and the French and Anglophone Caribbean
rarely appear.\textsuperscript{20}

There are a number of very good reasons for the two collections'
centering of Latin America. The region is the principal remit of
\emph{MATRIZes} and \emph{Comunicación y Sociedad}, and the journals
publish in its dominant languages.\textsuperscript{21}
Most of the seventeen contributors are based in Latin America, moreover,
and all but two of the papers were authored in Spanish or
Portuguese.\textsuperscript{22} Consider, too, that the \emph{MATRIZes} special issue was
cast in broad, ``internationalization'' terms, without foregrounding the
``Americas'' frame.\textsuperscript{23} That was, of course, an
editorial decision, and we note, too, the muted uptake of the
``Americas'' formulation among the papers originating in the 2022
conference.

Thus we want to gesture at a complementary reason, in the spirit of
self-reflexivity and with the positionalities of the US editors of this
journal foregrounded.\textsuperscript{24} One theme that animates most of the seventeen
articles, in different ways, is the structural inequality that has
marked the development and reception of Latin American communication
research.\textsuperscript{25}
As noted above, the US does indeed appear in a number of the
collections' essays, including our own. The context of that treatment
reflects, in various ways, the US role as colonialist, hemispheric
hegemon, and intellectual imperialist. A handful of papers linger on the
imposition, in the early postwar decades, of a US model of communication
research---quantitative, putatively universalistic, but rooted (often
covertly) in the US Cold War project.\textsuperscript{26} Latin American resistance, in
the 1970s and 1980s, to the mainstream US field, including its
``modernization'' paradigm, is also widely registered across the
essays.\textsuperscript{27} Many papers highlight, too, the
growth of homegrown intellectual coordinates, rooted in the region's
specific histories and the field's creative incorporation of critical
European thought.\textsuperscript{28} This ``rich, hybrid tradition'' (``rica
tradición híbrida''), to borrow Silvio Waisbord's phrase, flourished
alongside the establishment of regional associations and other forms of
patterned\marginnote{Algunas notas críticas y una propuesta,''
  \emph{MATRIZes} 17, no. 3 (2023); and Silvio Waisbord, ``¿Cómo
  enfrentar las desigualdades de la academia global en los estudios de
  comunicación?: colaboración, crítica y curiosidad,'' \emph{MATRIZes}
  17, no. 3 (2023). Articles with English-language translation have
  their English titles included above.} exchange.\textsuperscript{29}
Many contributions, finally, take up the structural
perversions---silences and warpings---of the neoliberal ``world academic
system''\marginnote{\textsuperscript{16} See, for example, Colón
  Zayas, ``Estudios de comunicación desde el pensamiento caribeño''; and
  Simonson, Pooley, and Park, ``The History of Communication Studies
  across the Americas.''} with\marginnote{\textsuperscript{17} See Arroyave, ``Develando las
  razones del diálogo asimétrico''; Colón Zayas, ``Estudios de
  comunicación desde el pensamiento caribeño''; Palacio Montiel,
  ``Historia de los estudios de comunicación desde las regiones de
  América Latina''; and Simonson, Pooley, and Park, ``The History of
  Communication Studies across the Americas.''} gathering\marginnote{\textsuperscript{18} Palacio
  Montiel, ``Historia de los estudios de comunicación desde las regiones
  de América Latina,'' 2. The quotations are from the English-language
  version of the article.} momentum\marginnote{\textsuperscript{19} On
  Spain, Portugal, and France in particular, see Serra, ``O espaço
  ibero-americano de ciências''; Averbeck-Lietz, ``On (Missing) Links
  between German, Latin American, and French Mediatization Research'';
  Moragas Spà, ``Investigar la comunicación''; and Colón Zayas,
  ``Estudios de comunicación desde el pensamiento caribeño.''} over the last three decades.\textsuperscript{30} Here again the US is implicated---as
a pillar of the Global North formation masquerading as
``international,'' and as the spear's tip of English-language hegemony.
All the while, from the early postwar decades through to the present,
the overwhelming majority of US scholars have remained blissfully
oblivious to the work of their Latin American counterparts.

We\marginnote{\textsuperscript{20} In the Caribbean context, two important
  exceptions are Colón Zayas, ``Estudios de comunicación desde el
  pensamiento caribeño''; and Da Porta, ``La internacionalización de la
  investigación en comunicación.''} centered our own \emph{MATRIZes} contribution around this theme, the
colonialist mix of US imperialism and indifference in its scholarly
relations with Latin America. Our approach was to highlight the unmarked
universalism of US historiography, in general and vis-à-vis Latin
America. ``\emph{The} urgent task for historians of US communication
studies,'' we wrote, ``is to provincialize and particularize the field
as it has developed in that country and situate it within international
movements of ideas, institutions, and peoples that have constituted the
field globally.''\textsuperscript{31} The
\emph{History of Media Studies} journal was founded with similar aims in
mind. Thus we organized the 2022 ``Across the Americas'' conference with
the hope that a pan-American frame might underwrite an overdue reckoning
with South-North entanglements within the hemisphere. At the same time
we registered some worries that such a project could lead to a ``new
master narrative,'' as inadvertently influenced by our position as
white, male US scholars.\textsuperscript{32}

One lesson we take from the \emph{MATRIZes} and \emph{Comunicación y
Sociedad} collections is to listen to these worries---to approach any
such pan-American historiographical project with humility and in light
of the hemisphere's histories of structural power dynamics and erasures.
This means, among other things, foregrounding the distinctive histories
of\marginnote{\textsuperscript{21} That many papers in both
  journals are translated into English is a reflection of the language's
  growing global hegemony in the ``internationalized'' neoliberal
  academy---a theme addressed in a number of the collections' articles.}  Latin\marginnote{\textsuperscript{22} The authors based outside Latin America are
  Stefanie Averbeck-Lietz (Germany), Miquel de Moragas Spà (Spain), Paul
  Serra (Portugal), Silvio Waisbord (US), and the authors of this
  introduction, all based in the US. The two papers authored in English
  are Averbeck-Lietz, ``On (Missing) Links between German, Latin
  American, and French Mediatization Research''; and Simonson, Pooley,
  and Park, ``The History of Communication Studies across the
  Americas.''} American communication research, with respect, in particular,
for a historiography written for and by the region's scholars. It also
means attending to our own positions, and those of the authors in all
three collections, in a ``global'' academic system that remains
thoroughly Western, in the face of (largely symbolic) calls for the
field's ``de-Westernization.'' Another way of saying this is that any
history of communication studies in the Americas must also be a
historical sociology of academic knowledge, one sensitive, in
particular, to epistemological erasures past and ongoing. This is a
theme that, fittingly, animates a large number of the \emph{MATRIZes}
and \emph{Comunicación y Sociedad} contributions.\textsuperscript{33}

\hypertarget{the-special-section-entangled-histories-across-the-americas}{%
\section{The Special Section: Entangled Histories across the
Americas}\label{the-special-section-entangled-histories-across-the-americas}}\marginnote{\textsuperscript{23} Vassallo de Lopes and Fuentes Navarro, in
  their \emph{MATRIZes} introduction, do mention the ``Americas'' in the
  context of the 2022 conference. ``Histórias da internacionalização do
  campo de estudos da comunicação,'' 9.}

Given the multifarious entanglements that inform the history of media
and communication studies across the Americas, the articles in this
special section chart numerous\marginnote{\textsuperscript{24} In the prose that follows, ``we'' refers
  to the three US editors of \emph{History of Media Studies}---Park,
  Pooley, and Simonson.} means\marginnote{\textsuperscript{25} See especially Arroyave, ``Develando las razones del
  diálogo asimétrico''; Serra, Paulo. ``O espaço ibero-americano de
  ciências da comunicação e as epistemologias do Sul''; Torrico
  Villanueva, ``Colonialidade do saber na internacionalização dos
  estudos sobre comunicação''; and Waisbord, ``¿Cómo enfrentar las
  desigualdades de la academia global en los estudios de comunicación?''} by\marginnote{\textsuperscript{26} See Colón Zayas,
  ``Estudios de comunicación desde el pensamiento caribeño,'' 2--5;
  Crovi Druetta, ``Travesía de la comunicación latinoamericana hacia su
  internacionalización,'' 159--62; Torrico Villanueva, ``Colonialidade
  do saber na internacionalização dos estudos sobre comunicação,''
  65--68; and Simonson, Pooley, and Park, ``The History of Communication
  Studies across the Americas,'' 196--99.} which\marginnote{\textsuperscript{27} See for example, Arroyave, ``Develando las razones del
  diálogo asimétrico,'' 10--11; Colón Zayas, ``Estudios de comunicación
  desde el pensamiento caribeño''; Crovi Druetta, ``Travesía de la
  comunicación latinoamericana hacia su internacionalización,'' 163--64;
  and Rüdiger, ``Adeus à crítica?''} to describe and consider
their topics.

The\marginnote{\textsuperscript{28} See Averbeck-Lietz, ``On (Missing) Links
  between German, Latin American, and French Mediatization Research,''
  259--62; Rüdiger, ``Adeus à crítica?''; and Waisbord, ``¿Cómo
  enfrentar las desigualdades de la academia global en los estudios de
  comunicación?'' 296--99.} section begins with Nova Gordon-Bell's history of communication and
media studies in the Anglophone Caribbean, a contribution that models
how to chart some of the shared and contested spaces and ideas one finds
in the North-South conversation, with particular attention to the
effects of colonial rule. In her essay, Gordon-Bell takes institutions
to be ideological tools for domination and control, a position that
seems quite fitting for a region where, as she relates, British colonial
rule took knowledge to be something that could only come from the mother
country. This approach to knowledge informed the functioning of the
University College of the West Indies, which became the University of
the West Indies in 1962, concurrent with Jamaican independence. The
legacies of this colonial system remain in many ways, but the story that
Gordon-Bell shares is one where communication study in the Anglophone
Caribbean took inspiration from the Non-Aligned Movement and UNESCO's
New World Information and Communication Order (NWICO) proposal.
Jamaica's Prime Minister Michael Manley took up the Non-Aligned cause,
and university education in communication assumed a focus on providing
professional training for journalists and other media workers. CARIMAC,
the Caribbean Institute of Mass Communication, was set up at the
University of the West Indies on the updrafts provided by the political
mood of the Non-Aligned movement and of democratic socialism. W. Aggrey
Brown, director of CARIMAC from 1979--2002 and media polymath with
numerous connections to local media, developed a first-year course for
students at CARIMAC. Gordon-Bell compares this course to a Trojan Horse,
with its seemingly innocuous\marginnote{\textsuperscript{29} Waisbord, ``¿Cómo enfrentar las
  desigualdades de la academia global en los estudios de comunicación?''
  On the history of new associations and patterned exchanges in the
  1970s and 1980s, see, for example, Crovi Druetta, ``Travesía de la
  comunicación latinoamericana hacia su internacionalización,'' 166--67;
  Moragas Spà, ``Investigar la comunicación,'' 146; and Paulino,
  ``América Latina, internacionalização e reciprocidade acadêmica.''} title (Communication, Culture, and
Caribbean Society) belying its critical bite, with a focus on questions
of power, media ownership, and Caribbean identity and politics. Readings
included Paolo Freire's \emph{Pedagogy of the Oppressed}, Jeremy
Tunstall's \emph{The Media Are American}, and the MacBride
Report.\textsuperscript{34} Since the 1970s, global political configurations,
local concerns, and the pressures placed on academe have pushed Jamaican
communication study away from more critical impulses to accommodate more
job-relevant training and North American certification requirements.
Gordon-Bell closes by musing on the importance of knowledge generation
from the Anglophone Caribbean to center on change.

\newpage Much\marginnote{\textsuperscript{30} Arroyave,
  ``Develando las razones del diálogo asimétrico''; Porta, ``La
  internacionalización de la investigación en comunicación,'' 277--79;
  Waisbord, ``¿Cómo enfrentar las desigualdades de la academia global en
  los estudios de comunicación?''} as Gordon-Bell positions the Anglophone Caribbean as a focal point
for understanding hemispheric and global dynamics of domination and
resistance, Yamila Heram and Santiago Gándara take an individual---the
US-born Elizabeth Fox, a pioneering critical political economist of the
media---as their subject. They show how Fox's peregrinations across
eleven countries\marginnote{\textsuperscript{31} Simonson, Pooley, and Park, ``The History of
  Communication Studies across the Americas,'' 190--91.} make\marginnote{\textsuperscript{32} Simonson, Pooley, and Park, 191.} her\marginnote{\textsuperscript{33}\setcounter{footnote}{33} See,
  especially, Serra, ``O espaço ibero-americano de ciências da
  comunicação e as epistemologias do Sul,'' and also Arroyave,
  ``Develando las razones del diálogo asimétrico,'' 13--15; León-Duarte,
  ``Cruces y límites en la investigación sobre comunicación''; and
  Torrico Villanueva, ``Colonialidade do saber na internacionalização
  dos estudos sobre comunicação,'' 59--61.} an\marginnote{\textsuperscript{34}\setcounter{footnote}{34} Paulo Freire, \emph{Pedagogy of the Oppressed}
  (Harmondsworth, UK: Penguin, 1972); Jeremy Tunstall, \emph{The Media
  Are American} (New York: Columbia University Press, 1977); and
  International Commission for the Study of Communication Problems,
  \emph{Many Voices, One World: Towards a New, More Just, and More
  Efficient World Information and Communication Order} (New York:
  UNESCO, 1980).} exemplary ``transnational figure,'' and one
who was likely to collaborate with other transnational intellectuals.
Heram and Gándara link all of this to Fox's intellectual and
institution-building legacy. Working from a meta-analysis of Fox's work,
and from semi-structured interviews with Fox herself, the authors
provide a revealing portrait of Fox's emergent focus on topics in media
studies that seem to have been prompted by her own border-crossing
proclivities. These are apparent in the three moments the authors
explore in Fox's career: from (1) her early studies (when she lived in
Bogotá) on media economics and her interventions on behalf of National
Communication Policies (NCPs) and the New World Information and
Communication Order (NWICO) to (2) her more reflective and
reconceptualizing work in the 1980s (when she lived in Buenos Aires and
then Paris), to (3) her most recent work (much of it conducted in
Washington, D.C.) on the implementation of health programs. Heram and
Gándara point out that, like many transnational figures and like many
women in academe, Fox has suffered from a distinct lack of visibility in
both North and South American academic contexts, an injustice that this
article may help to redress.

The goal of identifying meaningful strands of intra-hemispheric
influence can also be served by interrogating borders themselves, and by
holding up borders as generative moments in constructing meaning.
Michael Darroch makes borders and bordering behavior central to his
article's consideration of the Canadian and Québecois histories of
communication studies, where he asserts that border-crossings (literal
and figurative) took on a centrality all their own at moments of the
communication field's development. It is the ``frontier imagination''
that Darroch identifies at work in the North American media studies
context, an imagination within which borders supply us with some of the
imaginary substrate that shaped how communication came to be studied.
Darroch calls on the ideas and example of noted border-crosser Vilém
Flusser---whose own diasporic story found him born in Prague, fleeing to
London, moving to Brazil, then back to Europe---for terminology to
center the acts of translation, of nomadic thinking, and of dialogue
that set this article in motion. Borders and their crossing occupied an
important place in the work of Marshall McLuhan and of Harold Innis, and
also play an important role in the parallax one finds in the two
scholars' differential uptake in Quebec and in Anglophone Canada.
Darroch then turns to discursive and disciplinary border-crossing at
work in the journal \emph{Media Probe}, which would later become the
\emph{Canadian Journal of Communication} (\emph{CJC}), and at the
University of Laval--based journal \emph{Communication Information}
(\emph{CI}), later titled \emph{Communication, Information, Médias,
Théories}. The same productivity of borders can be identified at work in
the founding of the Canadian Communication Association. Darroch leaves
us with a vivid impression of the possibilities associated with what he
calls ``multifocal habits of vision,'' where the duality often
associated with borders is replaced with an appreciation of the humbling
layers of multivocality we can retrieve from properly contextualized
histories of media studies.

Raúl Fuentes Navarro, as noted above, helped set in motion this special
section, as well as its companion collections in \emph{MATRIZes} and
\emph{Comunicación y Sociedad}. His article here features a
characteristically full-throttled---if also nuanced---institutional
perspective on the history of inter-American media and communication
studies. He begins with careful stage-setting, noting the historicity of
the terminology we use to refer to the Americas, and then links this
discursive instability to what he calls ``disintegrated
internationalism''---a dynamic at work, he writes, in the institutional
housings for communication studies across the hemisphere. Fuentes
Navarro concentrates our attention on three major Latin America--focused
institutions: Centro Internacional de Estudios Superiores de
Comunicación para América Latina (CIESPAL), ALAIC, and Federación
Latinoamericana de Facultades de Comunicación (FELAFACS). The three
institutions acted as agents of the disintegrated internationalism that
Fuentes Navarro points to, serving in some ways as conduits for
transnational influence while also preserving autochthonous Latin
American intellectual tendencies, as well as essential national
differences in scholarship across the region. Rejecting a zero-sum model
for US--Latin American relations in the context of the history of
communication and media studies, Fuentes Navarro avers that we should
(following the model and words of Luis Ramiro Beltrán) appreciate the
complexity at work in how academic labor comes to be organized, with an
avoidance of both dogmatism and of the self-deluding quest for a science
``free of values.''

The inter-American connections that Fuentes Navarro brings to life in an
institutional framework can also be seen at work on the discursive
level. In a translation of a previously-published article, Erick Torrico
Villanueva shows how inter-America communication studies has developed
discursively, centered on a positivist scientific inquiry that emerged
from Western---and in particular US---scholarly traditions.\footnote{Erick
  Rolando Torrico Villanueva, ``La comunicación `occidental,'\,''
  \emph{Oficios Terrestres}, no. 32 (2015).} This narrow focus on one
way of knowing led to ``abysmal thinking,'' where any other epistemology
is positioned as something other than real knowledge. Torrico Villanueva
explores how this understanding of knowledge traditions around the world
finds expression in communication study in the US, Europe, and in
Iberoamerica through a critical review of major textbooks. He finds
relatively little in the way of reflexivity concerning intellectual
standpoint, and an overwhelming reliance upon the ideas of figures from
the US and from Western Europe. Torrico Villanueva concludes that the
work before us will require seeing communication and its study through
lenses less completely tinged by US and Western European ideas and
practices.~

The focus on dominant understandings of inter-American relations in the
history of media studies cues Afonso de Albuquerque's carefully
calibrated contribution, which applies the idea of \emph{intellectual
imperialism} to these transnational relations. Albuquerque begins the
special section's sixth and final entry with an observation that
academic interest in cultural imperialism has waned in recent
decades---perhaps because cultural imperialism itself has been so
thoroughly assimilated across global academic culture. Albuquerque sets
cultural imperialism alongside related concepts, notably \emph{media
imperialism}---in which media outlets become tools that imperial powers
use to assert power over others---and \emph{intellectual
imperialism}---where powerful countries impose their own ways of knowing
on other countries. Albuquerque describes how academic institutions
(including universities, philanthropic organizations, and journals)
exert a collective influence on the contours of work (academic and
para-academic) that holds up communication scholarship from the US as
the model for Latin America. Albuquerque illustrates this mode of
intellectual imperialism with a case study of the (University of
Texas--based) Knight Center for Journalism in the Americas, especially
as the center has operated through the Brazilian organization ABRAJI
(Associação Brasileira de Jornalismo Investigativo {[}Brazilian
Investigative Journalism Association{]}). US models for journalism
study, practice, and training, put to work in the political culture of
Brazil, contributed powerfully to the destabilization of the country's
politics. Albuquerque's broader point is the continuing relevance of
intellectual imperialism in making sense of the field's transnational
history.

Taken together, these articles make a strong, if qualified, case for
``the Americas'' as a frame of reference. Without smoothing over
patterns of dominance, without sealing off the region from other sources
of influence, and without positioning ``the Americas'' as a stable
referent, this special section of \emph{History of Media Studies} gives
voice to linguistic, discursive, methodological, and institutional modes
of transnational and intra-national influence and stability. The section
does not inaugurate this reflection, but it does pick up strands of
scholarly attention and intellectual influence in order to braid them
together under the hemispheric rubric. ``The Americas,'' in essays
collected here, stands as a provocation to explore something other than
national or global tendencies in the field's historiography---to
consider how these received histories might be re-charted, as it were,
from the Southern Cone to points north.




\section{Bibliography}\label{bibliography}

\begin{hangparas}{.25in}{1} 



Arroyave, Jesús. ``Develando las razones del diálogo asimétrico:
Explorando la exclusión en el campo de la comunicación/Unveiling the
Reasons for Asymmetrical Dialogue: Exploring Exclusion in the Field of
Communication.'' \emph{Comunicación y Sociedad} 20 (2023): 1--21.
\url{https://doi.org/10.32870/cys.v2023.8719}.

Averbeck-Lietz, Stefanie, ed. \emph{Kommunikationswissenschaft im
internationalen Vergleich: Transnationale Perspektiven}. Wiesbaden,
Germany: Springer Fachmedien Wiesbaden, 2017.

Averbeck-Lietz, Stefanie. ``On (Missing) Links between German, Latin
American, and French Mediatization Research: Reflections on Diverse
Research Milieus and their Traditions.'' \emph{MATRIZes} 17, no. 3
(2023): 241--72.
\url{https://doi.org/10.11606/issn.1982-8160.v17i3p241-272}.

Babini, Dominique. ``Toward a Global Open-Access Scholarly
Communications System: A Developing Region Perspective.'' In
\emph{Reassembling Scholarly Communications}, edited by Martin Paul Eve
and Jonathan Gray, 331--41. Cambridge, MA: MIT Press, 2020.
\url{https://doi.org/10.7551/Mitpress/11885.003.0033}.

Cicalese, Gabriela Rosa. ``Internacionalización y raíces identitarias de
la comunicación en Argentina.'' \emph{MATRIZes} 17, no. 3 (2023):
217--40. \url{https://doi.org/10.11606/issn.1982-8160.v17i3p217-240}.

Colón Zayas, Eliseo R. ``Estudios de comunicación desde el pensamiento
caribeño: Contribuciones de Luis Ramiro Beltrán, Frantz Fanon y Stuart
Hall sobre desarrollo e identidad cultural/Communication Studies from
Caribbean Thought: Contributions of Luis Ramiro Beltrán, Frantz Fanon
and Stuart Hall on Development and Cultural Identity.''
\emph{Comunicación y Sociedad} 20 (2023): 1--25.
\url{https://doi.org/10.32870/cys.v2023.8628}.

``Comunicar {[}en{]} la Historia: Panorama científico de la Historia de
la Comunicación Social en Iberoamérica; Intersecciones y marcos
comparados.'' AE-IC \& AsHisCom Conference, June 17--18, 2021, virtual.

Crovi Druetta, Delia. ``Travesía de la comunicación latinoamericana
hacia su internacionalización.'' \emph{MATRIZes} 17, no. 3 (2023):
155--72. \url{https://doi.org/10.11606/issn.1982-8160.v17i3p155-172}.

Da Porta, Eva. ``La internacionalización de la investigación en
comunicación: Algunas notas críticas y una propuesta.'' \emph{MATRIZes}
17, no. 3 (2023): 273--94.
\url{https://doi.org/10.11606/issn.1982-8160.v17i3p273-294}.

``Exclusions in the History and Historiography of Communication
Studies/Exclusiones en la Historia e Historiografía de los Estudios de
Comunicación.'' International Communication Association preconference,
May 26--27, 2021, virtual.
\url{https://hms.mediastudies.press/pub/schedule}.

Freire, Paulo. \emph{Pedagogy of the Oppressed}. Harmondsworth, UK:
Penguin, 1972.

Fuentes Navarro, Raúl, ed.~``Historias de los Estudios de Comunicación
en las Américas.'' Special section, \emph{Comunicación y Sociedad} 20
(2023).
\url{https://comunicacionysociedad.cucsh.udg.mx/index.php/comsoc/issue/view/v2023}.

Fuentes Navarro, Raúl. ``Historias de los estudios de comunicación en
las Américas/Histories of Communication Studies in the Americas.''
\emph{Comunicación y Sociedad} 20 (2023): 1--5.
\url{https://doi.org/10.32870/cys.v2023.8737}.

Ganter, Sarah Anne, and Félix Ortega. ``The~Invisibility of Latin
American Scholarship in~European Media and Communication
Studies:~Challenges and Opportunities of De-Westernization~and Academic
Cosmopolitanism.''~\emph{International Journal
of~Communication}~13~(2019).
\url{https://ijoc.org/index.php/ijoc/article/view/8449/2523.}

``Historia de los Estudios de Comunicación en las Américas/História dos
Estudos de Comunicação nas Américas/History of Communication Studies
across the Americas.'' History of Media Studies roundtable, July 12,
2022, virtual. \url{https://hms.mediastudies.press/americas-roundtable}.

International Commission for the Study of Communication Problems.
\emph{Many Voices, One World: Towards a New, More Just, and More
Efficient World Information and Communication Order}. New York: UNESCO,
1980.

``IV Doctoral AE-IC: Taller iberoamericano de investigación en
comunicación.'' AE-IC predoctoral conference, Pontevedra, Spain, June
15--16, 2023.

León-Duarte, Gustavo Adolfo. ``Cruces y límites en la investigación
sobre comunicación: El sentido práctico interdisciplinar.''
\emph{MATRIZes} 17, no. 3 (2023): 117--40.
\url{https://doi.org/10.11606/issn.1982-8160.v17i3p117-140}.

``Media and Communication Studies in a Global Context: A Critical
History.'' International Communication Association preconference,
Toronto, May 25, 2023.

Moragas Spà, Miquel de. ``Investigar la comunicación: Entre el pasado y
la prospectiva.'' \emph{MATRIZes} 17, no. 3 (2023): 143--54.
\url{https://doi.org/10.11606/issn.1982-8160.v17i3p143-154}.

Oliveira Paulino, Fernando. ``América Latina, internacionalização e
reciprocidade acadêmica/Latin America, Internationalization, and
Academic Reciprocity.'' \emph{MATRIZes} 17, no. 3 (2023): 173--85.
\url{https://doi.org/10.11606/issn.1982-8160.v17i3p173-185}.

Oliveira Paulino, Fernando, Gabriel Kaplún, Miguel Vicente-Mariño, and
Leonardo Custodio, eds. \emph{Research Traditions in Dialogue:
Communication Studies in Latin America and Europe.} Porto, Portugal:
Porto~Editora, 2020.
\url{https://www.alaic.org/wp-content/uploads/2022/02/Research-Traditions-in-Dialogue.pdf}.

Palacio Montiel, Celia Del. ``Historia de los estudios de comunicación
desde las regiones de América Latina: Las historias conectadas como
recurso para el análisis/The History of Communication Studies from
Regions of Latin America: Connected Histories as a Resource for the
Analysis.'' \emph{Comunicación y Sociedad} 20 (2023): 1--19.
\url{https://doi.org/10.32870/cys.v2023.8609}.

Rayen Dall'Orso, Ana. ``Investigación de la~Comunicación en
Iberoamérica: Una Paleta~Diversa.''~\emph{Cuadernos.info},\emph{~}no. 53
(2022). \url{http://dx.doi.org/10.7764/cdi.53.53863}.

``Repressed Histories in Communication and Media Studies.''
International Communication Association preconference, Gold Coast,
Australia, June 20, 2024.

Rodríguez~Benito,~María Elena, María Esther Pérez-Peláez, and Teresa
Martín García. ``Investigación~en Comunicación: Diferencias~entre
Península Ibérica y América Latina.''~\emph{Cuadernos.info},~no. 54
(2023). \url{https://doi.org/10.7764/cdi.54.51309}.

Rüdiger, Francisco. ``Adeus à crítica?: passado e presente da teoria e
método na pesquisa em comunicação de massa/Farewell to Critique? Past
and Present of Theory and Method in Mass Communication Research.''
\emph{MATRIZes} 17, no. 3 (2023): 73--99.
\url{https://doi.org/10.11606/issn.1982-8160.v17i3p73-99}.

Sandoval García, Carlos. ``Textos, audiencias y medios de comunicación:
La persistencia de las preguntas.'' \emph{MATRIZes} 17, no. 3 (2023):
101--16. \url{https://doi.org/10.11606/issn.1982-8160.v17i3p101-116}.

Serra, Paulo. ``O espaço ibero-americano de ciências da comunicação e as
epistemologias do Sul/The Ibero-American Space of Communication Sciences
and the Epistemologies of the South.'' \emph{MATRIZes} 17, no. 3 (2023):
29--54. \url{https://doi.org/10.11606/issn.1982-8160.v17i3p29-54}.

Simonson, Peter, David W. Park, and Jefferson Pooley, eds. ``Exclusions
in the History of Media Studies/Exclusiones en la historia de los
estudios de medios.'' Special section, \emph{History of Media Studies} 2
(2022). \url{https://hms.mediastudies.press/volume-two}.

Simonson, Peter, David W. Park, and Jefferson Pooley.
``Exclusions/Exclusiones: The Role for History in the Field's
Reckoning.'' \emph{History of Media Studies} 2 (2022): 1--27. .

Simonson, Peter, and David W. Park, eds. \emph{The International History
of Communication Study}. New York: Routledge, 2016.

Simonson, Peter, Jefferson Pooley, and David Park. ``The History of
Communication Studies across the Americas: A View from the United
States.'' \emph{MATRIZes} 17, no. 3 (2023): 189--216.
\url{https://doi.org/10.11606/issn.1982-8160.v17i3p189-216}.

Sodré, Muniz. ``A ruptura paradigmática da comunicação/A Paradigmatic
Rupture in Communication.'' \emph{MATRIZes} 17, no. 3 (2023): 19--27.
\url{https://doi.org/10.11606/issn.1982-8160.v17i3p19-27}.

Torrico Villanueva, Erick Rolando. ``Colonialidade do saber na
internacionalização dos estudos sobre comunicação: Abordagem do caso da
América Latina.'' \emph{MATRIZes} 17, no. 3 (2023): 55--72.
\url{https://doi.org/10.11606/issn.1982-8160.v17i3p55-72}.

Torrico Villanueva, Erick Rolando. ``La comunicación `occidental.'\,''
\emph{Oficios Terrestres}, no. 32 (2015): 3--23.
\url{http://www.perio.unlp.edu.ar/ojs/index.php/oficiosterrestres/article/view/2381}.

Tunstall, Jeremy. \emph{The Media Are American}. New York: Columbia
University Press, 1977.

Vassallo de Lopes, Maria Immacolata, and Raúl Fuentes Navarro,
eds.~``Histórias da internacionalização do campo de estudos da
comunicação.'' Special issue, \emph{MATRIZes} 17, no. 3 (2023).
\url{https://www.revistas.usp.br/matrizes/issue/view/13086}.

Vassallo de Lopes, Maria Immacolata, and Raúl Fuentes Navarro.
``Histórias da internacionalização do campo de estudos da
comunicação/Histories of the Internationalization of the Field of
Communication Studies.'' \emph{MATRIZes} 17, no. 3 (2023): 5--16.
\url{https://doi.org/10.11606/issn.1982-8160.v17i3p5-16}.

Waisbord, Silvio. ``¿Cómo enfrentar las desigualdades de la academia
global en los estudios de comunicación?: colaboración, crítica y
curiosidad.'' \emph{MATRIZes} 17, no. 3 (2023): 295--315.
\url{https://doi.org/10.11606/issn.1982-8160.v17i3p295-315}.

Wiedemann Thomas, and Michael Meyen. ``Internationalization through
Americanization: The Expansion of the International Communication
Association\textquotesingle s Leadership to the World.''
\emph{International Journal of Communication} 10 (2016): 1489--1509.
\url{https://ijoc.org/index.php/ijoc/article/view/4504}.



\end{hangparas}


\end{document}