% see the original template for more detail about bibliography, tables, etc: https://www.overleaf.com/latex/templates/handout-design-inspired-by-edward-tufte/dtsbhhkvghzz

\documentclass{tufte-handout}

%\geometry{showframe}% for debugging purposes -- displays the margins

\usepackage{amsmath}

\usepackage{hyperref}

\usepackage{fancyhdr}

\usepackage{hanging}

\hypersetup{colorlinks=true,allcolors=[RGB]{97,15,11}}

\fancyfoot[L]{\emph{History of Media Studies}, vol. 4, 2024}


% Set up the images/graphics package
\usepackage{graphicx}
\setkeys{Gin}{width=\linewidth,totalheight=\textheight,keepaspectratio}
\graphicspath{{graphics/}}

\title[Forgetting / Cybernetics]{Forgetting / Cybernetics} % longtitle shouldn't be necessary

% The following package makes prettier tables.  We're all about the bling!
\usepackage{booktabs}

% The units package provides nice, non-stacked fractions and better spacing
% for units.
\usepackage{units}

% The fancyvrb package lets us customize the formatting of verbatim
% environments.  We use a slightly smaller font.
\usepackage{fancyvrb}
\fvset{fontsize=\normalsize}

% Small sections of multiple columns
\usepackage{multicol}

% Provides paragraphs of dummy text
\usepackage{lipsum}

% These commands are used to pretty-print LaTeX commands
\newcommand{\doccmd}[1]{\texttt{\textbackslash#1}}% command name -- adds backslash automatically
\newcommand{\docopt}[1]{\ensuremath{\langle}\textrm{\textit{#1}}\ensuremath{\rangle}}% optional command argument
\newcommand{\docarg}[1]{\textrm{\textit{#1}}}% (required) command argument
\newenvironment{docspec}{\begin{quote}\noindent}{\end{quote}}% command specification environment
\newcommand{\docenv}[1]{\textsf{#1}}% environment name
\newcommand{\docpkg}[1]{\texttt{#1}}% package name
\newcommand{\doccls}[1]{\texttt{#1}}% document class name
\newcommand{\docclsopt}[1]{\texttt{#1}}% document class option name


\begin{document}

\begin{titlepage}

\begin{fullwidth}
\noindent\LARGE\emph{Overlay Re-publication
} \hspace{66mm}\includegraphics[height=1cm]{logo3.png}\\
\noindent\hrulefill\\
\vspace*{1em}
\noindent{\Huge{Forgetting / Cybernetics\par}}

\vspace*{1em}

\noindent\LARGE{Diego Gómez-Venegas} \href{https://orcid.org/0000-0001-5640-204X}{\includegraphics[height=0.5cm]{orcid.png}}\par\marginnote{\emph{Diego Gómez-Venegas, ``Forgetting / Cybernetics,'' \emph{History of Media Studies} 4 (2024), \href{https://doi.org/10.32376/d895a0ea.c519a1b0}{https://doi.org/ 10.32376/d895a0ea.c519a1b0}.} \vspace*{0.75em}}
\vspace*{0.5em}
\noindent{{\large\emph{Humboldt-Universität zu Berlin}, \href{mailto:diego@gomezvenegas.com}{diego@gomezvenegas.com}\par}} \marginnote{\href{https://creativecommons.org/licenses/by-nc/4.0/}{\includegraphics[height=0.5cm]{by-nc.png}}}

% \vspace*{0.75em} % second author

% \noindent{\LARGE{<<author 2 name>>}\par}
% \vspace*{0.5em}
% \noindent{{\large\emph{<<author 2 affiliation>>}, \href{mailto:<<author 2 email>>}{<<author 2 email>>}\par}}

% \vspace*{0.75em} % third author

% \noindent{\LARGE{<<author 3 name>>}\par}
% \vspace*{0.5em}
% \noindent{{\large\emph{<<author 3 affiliation>>}, \href{mailto:<<author 3 email>>}{<<author 3 email>>}\par}}

\end{fullwidth}



\hypertarget{abstract}{%
\section{Abstract}\label{abstract}}

This article aims to trace epistemological connections between
cybernetics and German media theory by emphasizing the notion of
\emph{forgetting}. This notion is presented as a central condition for
problematizing, first, the human-machine coupling pointed out by
cybernetics and, second, the rise of the machines that has concerned
German media theory for decades. To this end, the article draws a line
connecting an early text by Friedrich Kittler and Heinz von Foerster's
work on memory---a connection that will also be discussed in the light
of findings and statements by Moritz Hiller and Jan Müggenburg. Finally,
this article will outline a (still hypothetical) way to problematize the
techno-epistemological scope of Project Cybersyn through the lens of
such a notion.\footnote{This is an updated version---mostly in terms of
  language and structure, but also slightly in terms of content---of an
  article published in early-2020 in the first issue of Chronus Art
  Center's \emph{CAC/CAFAcat Editorial} journal. Diego Gómez-Venegas,
  ``Forgetting / Cybernetics. History of Media Studies,''
  \emph{CAC/CAFAcat Editorial} 1, no. 1 (2020). Republished with
  permission. This article is also part of the author's doctoral
  research on Project Cybersyn, developed at the Institute for
  Musicology and Media Studies, Humboldt-Universität zu Berlin.}

  \hypertarget{resumen}{%
\section{Resumen}\label{resumen}}

Este artículo trata de rastrear las conexiones epistemológicas entre la cibernética y la teoría mediática alemana, enfatizando la noción de olvido. Esta idea se presenta como una condición central para problematizar, en primer lugar, el vínculo humanos-máquinas señalado por la cibernética y, en segundo lugar, el auge de las máquinas que ha preocupado la teoría mediática alemana durante décadas. Para ello, el artículo traza una línea que conecta un texto temprano de Friedrich Kittler con el trabajo de Heinz von Foerster sobre la memoria--una conexión que también se discutirá a la luz de los hallazgos y las aseveraciones de Mortiz Hiller y Jan Müggenburg. Finalmente, el artículo señala una manera (todavía hipotética) de problematizar el alcance tecno-epistemológico del Proyecto Cybersyn a través de la óptica de esta noción. 




\enlargethispage{2\baselineskip}

\vspace{1em}

\noindent{\emph{History of Media Studies}, vol. 4, 2024}


 \end{titlepage}

 \newthought{In 2013, a large group} of scholars gathered at New York University's
Deutches Haus to commemorate the work and life of the late media
theorist Friedrich Kittler. There, Bernhard Dotzler, who gave the only
German-language presentation\footnote{See Bernhard Dotzler, ``The Sirens
  Go Silent - Friedrich Kittler Part 11: Bernhard Dotzler,'' filmed
  March 2013 at Deutsches Haus at NYU, New York, NY, video, 48:34,
  \url{https://youtu.be/_uZJg8lwfJU}.} of what would later be translated
as \emph{Idiocy, Forgetting, and Outdatedness,}\footnote{See Bernhard
  Dotzler, ``Idiocy, Forgetting, and Outdatedness: Friedrich Kittler's
  Avant-Gardism and the `Time for Other Stories','' in \emph{The
  Technological Introject: Friedrich Kittler between Implementation and
  the Incalculable}, eds. Jeffrey Champlin et al (New York: Fordham
  University Press, 2018), 35--45.} emphasized what he calls Kittler's
``avant-gardism.'' Namely, the approach that the author of the
\emph{Aufschreibesysteme} unfolded---there, in the now distant days of
the late-1970s---by declaring that it was already ``time for other
stories.''\footnote{Friedrich Kittler, ``Forgetting,'' \emph{Discourse}
  3 (1981{[}1979{]}): 116; quoted in Dotzler, ``Idiocy, Forgetting, and
  Outdatedness,'' 38.} Accordingly, by moving beyond and somehow
escaping from literary studies, Kittler would pave the way for what
would later be known as German media theory---the field that would
remain recursively connected to cybernetic thinking from its
foundations.\footnote{See Jan Müggenburg, ``Bats in the Belfry: On the
  Relationship of Cybernetics and German Media Theory,'' \emph{Canadian
  Journal of Communication} 42 (2017): 467--84.} Thus, through an
intervention that forced his audience to automatically switch languages
for encoding and decoding on the fly, or simply invited them ``to feel
free to leave,''\footnote{Dotzler, ``The Sirens Go Silent - Friedrich
  Kittler Part 11,'' 2:22.} Dotzler reminds us of the structural
centrality of the notion of \emph{forgetting} in Kittler's radical move.

In that spirit, this article will look back to a time before the
publication of Kittler's pivotal habilitation thesis, \emph{Discourse
Networks} {[}\emph{Aufschreibesysteme}{]},\footnote{See Friedrich
  Kittler, \emph{Aufschreibesysteme 1800/1900} (München: Fink, 1985).
  Also, see Friedrich Kittler, \emph{Discourse Networks 1800/1900}
  (Stanford: Stanford University Press, 1990).} and will pay particular
attention to Dotzler's remarks in order to trace the nodal point that
would help us to witness, to understand, and ideally to problematize,
the connection between cybernetics and German media theory. But this
will also imply looking at Moritz Hiller's research\footnote{See Moritz
  Hiller, ``Unter Aufschreibesystemen: `Eine Adresse im
  \sout{Adressbuch} IC der Kultur,'\,'' \emph{Metaphora} 1 (2015):
  II/1--26.} when he claims that, contrary to what Kittler later
claimed,\footnote{See John Armitage, ``From Discourse Networks to
  Cultural Mathematics: An Interview with Friedrich A. Kittler,''
  \emph{Theory, Culture \& Society} 23, no. 7-8 (2006): 17--38.} the
founding discussion around the notion of \emph{Discourse Networks} was
neither influenced by Shannon's mathematical theory of
communication,\footnote{See Claude E. Shannon and Warren Weaver,
  \emph{The Mathematical Theory of Communication} (Urbana: University of
  Illinois Press, 1998 {[}1949{]}).} nor was a ``free
application''\footnote{Armitage, ``From Discourse Networks to Cultural
  Mathematics,'' 19.} of it\emph{.} Rather, as Jan Müggenburg has
pointed out,\footnote{See Müggenburg, ``Bats in the Belfry,'' 475--78.}
it was rather a development influenced by an early familiarity with the
work of cyberneticians such as Heinz von Foerster.

Consequently, the central issue that this article seeks to address lies
in the fact that, beyond any implicit emphasis on the
\emph{transmission} and \emph{processing} functions of media technology
that Friedrich Kittler had deployed by stressing, with Shannon, the role
of statistics, probabilities, noise, and channel, the actual operation
that activated the link between German media theory and cybernetics
would be the \emph{problem of memory}. In other words, given that
``transfer and storage are two sides of one coin,''\footnote{Wolfgang
  Ernst, ``Archives in Transition: Dynamic Media Memories,'' in
  \emph{Digital Memory and the Archive} (Minneapolis: University of
  Minnesota Press, 2013), 100.} as Wolfgang Ernst has pointed out, what
seems to define not only the link between cybernetics and German media
theory, but perhaps the very essence of each of these fields, is the key
ambivalence between \emph{transmission} and \emph{storage} that media
technologies produce. Therefore, the human-machine coupling---insofar as
it is understood here as a central object of research for both
cybernetics and German media theory---will be considered in these pages
as the actual manifestation of the aforementioned nodal point; i.e., as
the true embodiment of the scope of cybernetics, and even more so, of
its entangled existence around and within German media theory. Thus, the
sentence ``Media determine our situation,''\footnote{Friedrich Kittler,
  \emph{Gramophone, Film, Typewriter} (Stanford: Stanford University
  Press, 1999), xxxix.} one of Kittler's most well-known statements and
perhaps the (deceptive) starting point of his explicit analytical
program on media technologies, can find its rationale a few hundred
pages back.\footnote{The above quote---``Media determine our
  situation''---marks the actual beginning of Kittler's book
  \emph{Gramophone, Film, Typewriter}, and so it could be argued that
  all the reasoning that supports such a statement is to be found in the
  remaining chapters and pages of that book. Even more, it could be also
  argued that this work marks the beginning of Kittler's second
  period---the one concerned with media technologies---and thus, that
  the cybernetic strands of his work must be found there. This article
  contends, however, that it is his earlier work that draws the
  theoretical complex that, as a \emph{hinge} between his literary and
  media theoretical periods, points to Kittler's cybernetic program. On
  Kittler\textquotesingle s periods, see Geoffrey Winthrop-Young,
  ``Introduction,'' in \emph{Kittler and the Media} (Cambridge: Polity
  Press, 2011), 1--7.} Paradoxically, despite the warning of the Berlin
School of Media Archaeology and, of course, of Kittler himself, it is
through the study of written texts that this article will address these
questions. For this reason, what follows should not necessarily be seen
as a media archaeology, but rather as a kind of genealogy.

\hypertarget{forgetting}{%
\section{Forgetting}\label{forgetting}}

In 1979, in a small anthology entitled \emph{Texthermeneutik:
Aktualität, Geschichte, Kritik} {[}Text-Hermeneutics: Present, History,
Critique{]}, the 36-year-old (still) literary scholar Friedrich Kittler
presented an article succinctly titled \emph{Vergessen
}{[}Forgetting{]}.\footnote{See Friedrich Kittler, ``Vergessen,'' in
  \emph{Texthermeneutik: Aktualität, Geschichte, Kritik}, ed. Ulrich
  Nassen (Paderborn: Ferdinand Schöningh, 1979), 195--221\emph{.}} Two
years later, in 1981, an English translation appeared in the US journal
\emph{Discourse.}\footnote{See Kittler, ``Forgetting.''} There, Kittler
laid out an early set of arguments that, as the recursion of time now
reveals, would not only outline a critique of the traditions that
sustained the field that demarcated the scholar's own position at the
time, but would also constitute an act of radical departure that would,
in turn, mark the attempt to draw a new field---a field whose boundaries
would be as fragmented as to seek and promote the struggle between the
old world of letters and the not-so-old realm of circuits. Such a task,
however, begins by looking at Nietzsche. It is in the work of the
(first) philosopher and (then) genealogist that Kittler finds the thread
to begin weaving the notion of \emph{forgetting} as a key element for
understanding the media-conditioned scope of the processes of memory in
contemporary cultures. Interestingly, this connection invoked an
\emph{untimely}---perhaps a too early one\footnote{In the German
  article, \emph{Vergessen}, Kittler cites Nietzsche's work \emph{Vom
  Nutzen und Nachteil der Historie für das Leben} {[}On the Utility and
  Liability of History for Life{]} from the collection
  \emph{Unzeitgemässe Betrachtungen} {[}Unfashionable Observations, or
  Untimely Meditations{]}, which was originally published in Leipzig in
  1874. However, in the English translation of this article,
  \emph{Forgetting}, the editors of the journal \emph{forgot} to use the
  already translated and published English version of Nietzsche's work,
  and instead allowed the translators to automatically translate---that
  is, without paying attention to already archived storage devices---the
  quote in question. In this article, as it will be seen below, such a
  particular act of \emph{forgetting} is challenged.}---reflection on
the then impossible systems of control and communication between the
animal and the human being:

\begin{quote}
The human being might ask the animal: `Why do you just look at me like
that instead of telling me about your happiness?' The animal wanted to
answer, `Because I always immediately forget what I wanted to say'---but
it had already forgotten this answer and hence said nothing, so that the
human being was left to wonder.\footnote{Friedrich Nietzsche, ``On the
  Utility and Liability of History for Life,'' in \emph{Unfashionable
  Observations} (Stanford: Stanford University Press, 1995), 87.}
\end{quote}

\noindent Kittler thus embarks himself on a project, on a journey, that will lead
him to analyze the techniques and technologies that would allow humans
to distance themselves from other animals: speaking, reading, writing,
and thus \emph{storage devices}. ``So,'' Kittler argues, ``it is only on
paper that the `human being' originates, this being, \emph{per
definitionem} distinct from the animal.''\footnote{Kittler
  ``Forgetting,'' 90.} In other words, what preoccupies Kittler in the
ante-penultimate decade of the 20th century, are the material processes
that not only define the human condition, but even more so configure the
transitions and transformations of such a condition. For if these
material processes, these techniques and technologies, change, then it
is not only not strange, but expected, that human beings will also
change as a consequence.

However, the German scholar's main concern is that these elements have
historically been taken for granted, as if ``memory {[}were{]}
considered an attribute or even a peculiarity of the `human
being.'\,''\footnote{Kittler ``Forgetting,'' 90.} But on the contrary,
precisely because we forget, and moreover because we have entrusted our
memory and knowledge to what he calls ``storage devices'' and
``mnemo-techniques,''\footnote{Kittler ``Forgetting,'' 90.} Kittler
initially sees in discourse analysis---not only Foucault's, but also
Nietzsche's---the procedure for inquiring into these \emph{writing
systems}. But such an approach also implies a critique, and thus it is
possible to argue that another mode of analysis must underlie the
researcher's general program:

\begin{quote}
Discourse analysis, by contrast, means to let the `human being' be. I
forget every day whether I forget or remember. But that is not the
question. The question is where and how those memory systems function
that philosophy ascribed to the `human being.'\footnote{Kittler
  ``Forgetting,'' 90.}
\end{quote}

\noindent And later, \emph{literally} announcing his departure, he continues:

\begin{quote}
Archives themselves provide plenty of material to archivize. Only
imperial myths propagate the belief that sentences are eternal once they
have been hewn into stone, once they have become lapidary. No storage
device operates in isolation. Archives are hooked up with other
archives, directly or via interfaces, and are themselves archivized in
other archives. Archives require input and output stations (even if
these be just sense organs and brains). Archives contain mechanisms that
bring about and/or prevent the erasure of their data. The development of
electronic computers has merely provided precise terms and circuit
diagrams for factors which come into play in all cases of
archivizing.\footnote{Kittler ``Forgetting,'' 93.}
\end{quote}

\noindent 
For Kittler, therefore, no more simple discourse analyses---as if that
had ever been the case---because even archives have to operate in
networks in order to avoid \emph{forgetting}, or rather to prevent us
humans from doing so. Ironically, however, insightful Friedrich knows
that not everything has been said or written. This is because Kittler's
main criticism of Foucauldian discourse analysis is precisely what the
old-fashioned ``archeologist simply forgot''\footnote{Friedrich Kittler,
  \emph{Gramophone, Film, Typewriter} (Stanford: Stanford University
  Press, 1999), 5.}: that in the 20th century such networks went far
beyond books and letters. Thus, in 1979, just\newpage\noindent as he would put it almost
15 years later\footnote{See Friedrich Kittler, ``The World of The
  Symbolic -- A World of The Machine,'' in \emph{Literature, Media,
  Information Systems}, ed. John Johnston (Amsterdam: OPA Amsterdam
  B.V., 1997), 133. See also Friedrich Kittler, ``Die Welt des
  Symbolischen -- eine Welt der Maschine,'' in \emph{Draculas
  Vermächtnis. Technische Schriften} (Leipzig: Reclam, 1993), 62.} in
the midst of his media-analytical phase, Kittler refers to computer data
storage devices as the actual material manifestation of the processes
that regulate the circulation of information---at that time still mostly
characterized as knowledge---in modern cultures. According to the
fugitive literary scholar,\footnote{According to Moritz Hiller, by
  1976---the year in which Friedrich Kittler would have first publicly
  referred to his research subject in technological terms---the scholar
  would have already spent years experimenting with electrotechnical
  media. See Hiller, ``Unter Aufschreibesystemen,'' II--8.} programmable
read-only memories (PROMs), on the one hand, and random-access memories
(RAMs), on the other, become the key operational model and thus a
critical media-epistemological way to characterize the flow, the
circuit, the exchange of information between archives and human beings.
Since PROMs are the set of deeper permanent instructions to initiate
operations within the system, and RAMs are the devices to store only the
necessary data to operate in the present, this new coupling between
humans and archival (information) systems emerges as conditioned by the
ambivalent nature of a programmable permanence and a permanent
transition.\footnote{See Ernst, ``Archives in Transition,'' 99.}

\begin{quote}
People, once simply PROM's who were programmed once and for all through
baptism, village schools, and the order of estates, became RAM's. In
order to supply storage space for new books, new knowledge, new
programs, information had to be made erasable---and according to
Nietzsche understanding is the erasure of signifiers.\footnote{Kittler
  ``Forgetting,'' 99.}
\end{quote}

\noindent In other words, what Kittler is able to discern at this early stage of
his career is that when, in the mid-1900s, archives become technological
information systems, the only thing that remains permanent is not
knowledge but the commands to access it---the entry barriers or
protocols---and then knowledge itself becomes information that is
constantly erasable and rewritable---that is, \emph{forgettable}. What
is more, by being plugged into these systems, our very being as humans
becomes not only a product, but the embodiment of such a technological
\emph{forgetting}. Whereas in earlier epochs, in many cultures, people
were encouraged, if not forced, to learn by heart the knowledge they
could acquire from the archives and storage devices to which they had
access---i.e., libraries and books---in modern times, not only the
ever-increasing amount of information, but also the very techno-logical
structures that sustained it, would have turned people into RAM modules.
From then on, the very act of searching for the desired information
implied running the protocols that would make this procedure possible,
and looking for the requested data points one address \emph{at a time.}
But since all the many addresses and data points that would have to be
consulted before a successful search could be carried out would occupy
valuable and limited memory capacity, such \emph{at a time} would always
imply \emph{forgetting} the previously registered data, and then it
would also literally mean operating in\newpage\noindent one and only one time---as in a
random-access memory, in a present that unfolds as presence.

Then, people learn to learn by technologically \emph{forgetting}. Here
seems to lie the cybernetic core of Friedrich Kittler's media
theoretical program.

In this regard, there are two paths that might show that the emergence
of this program may also respond to concrete connections between the
fields in question---literary studies and cybernetics. Thus, following
what Jan Müggenburg has shown,\footnote{Although Jan Müggenburg
  emphasizes the fact that cybernetic thinking and theories became
  popular among German postmodernists in the 1980s, thus influencing
  German media theory from then on, he also points out, following Moritz
  Hiller (2015), that Kittler may have received the influence of
  cybernetics indirectly, already in the 1970s, through the works of
  Watzlawick, Luhmann, and Schmidt. However, this article suggests that
  Kittler may have been directly exposed to the work of Heinz von
  Foerster already in that decade, obscure and indirect references
  notwithstanding. See Müggenburg, ``Bats in the Belfry,'' 475--78.}
this article argues that both of these paths, as the conjecture has it,
lead to Heinz von Foerster's work on memory.

On the one hand, and even beyond the particularly precise conceptual
coincidence, Kittler's ``Forgetting'' offers a specific indication that
allows this article to strongly suggest that Kittler might have
indirectly referred to von Foerster's lecture \emph{Quantum Mechanical
Theory of Memory}\footnote{See Heinz von Foerster, ``Quantum Mechanical
  Theory of Memory,'' in \emph{Cybernetics \textbar{} Kybernetik: The
  Macy-Conferences 1946-1953. Volume I / Band I. Transactions /
  Protokolle}, ed. Claus Pias (Zurich-Berlin: Diaphanes, 2003), 98--121.}
in his arguments. Comparing the library system with computational
memories and procedures, and further referring to a ``cunning reader''
or user of such a library, who in the scholar's view ``is an address
selector of the sort that is hooked up to the latest generations of IBM
computers,'' Kittler adds that when this search system overcomes
``harmless books'' and libraries, ``the {[}computational{]} address
selector equipped with a \emph{randomness generator} sends the incoming
data to \emph{free positions}, the exact address of which does not
appear at any of the many output stations.''\footnote{Kittler,
  ``Forgetting,'' 94. {[}Emphasis in the published English translation.
  However not present in the original in German{]}.} It is precisely
this techno-logical explanation---the one with \emph{randomness
generators} and \emph{free positions}---that gives this article the
space to unfold its conjecture.

What Heinz von Foerster, the Austrian-American cybernetician, presented
in the aforementioned lecture at the Macy Conferences in 1949, was the
outline of a theory of memory in three steps: the phenomenological
(explained by quantum mechanical means), the psychological, and the
biophysical.\footnote{See von Foerster, ``Quantum Mechanical Theory of
  Memory,'' 98.} The second step---which may have been of particular
interest to Kittler given his ongoing attention to poststructuralism and
psychoanalysis\footnote{Kittler, ``Forgetting,'' 94.}---explains
diagrammatically how the mental procedure by which a human being
memorizes a series of nonsense syllables would work. Such a procedure is
defined by this nonsense information being ``fixed on a certain carrier,
many of which may be in the brain ready to be impregnated by such an
elementary impression.''\footnote{von Foerster, ``Quantum Mechanical
  Theory of Memory,'' 100.} Von Foerster calls these
ready-to-be-impregnated elements ``free carriers,''\footnote{von
  Foerster, ``Quantum Mechanical Theory of Memory,'' 101.} and goes on
to point out that it is possible to ``assume that such a carrier is not
able to carry forever its impregnation but only during a certain time
and decays after time τ to a free carrier.''\footnote{von Foerster,
  ``Quantum Mechanical Theory of Memory,'' 101.} In other words, if
\emph{nonsense syllables} are tantamount to \emph{random data}, which is
sent to \emph{free positions} or \emph{carriers}, then Kittler's early
media theory of \emph{forgetting} could have found its origin right at
the dawn of cybernetics. And, of course, this correlation, this
conjecture, does not simply obey the interpretation of meanings, or even
the equivalence of syntactic structures, but the exact matching of
characters. Says von Foerster:

\begin{quote}
Some time ago I was trying to work out a relation between the physical
and the psychological time. Certainly, both these times would be
proportional to each other if our memory would work like a
tape-recorder: any incoming information would be stored indefinitely.
Recall of a certain event would give exactly the same time structure as
previously observed. We know, however, that isn't so. As time elapses we
lose a certain amount of information by forgetting. Hence I tried to
start with a simple theory of forgetting.\footnote{von Foerster,
  ``Quantum Mechanical Theory of Memory,'' 98.}
\end{quote}

\noindent What is more, in his lecture the cybernetician argues that in order to
develop such a theory, he needs ``a psychological process which deals
with impressions of which the elements are as independent as possible of
each other.''\footnote{von Foerster, ``Quantum Mechanical Theory of
  Memory,'' 98.} Interestingly, von Foerster finds his case study---and
this point takes this article already to the second path of its
argument---in the work on memory developed by the German psychologist
Hermann Ebbinghaus.\footnote{von Foerster, ``Quantum Mechanical Theory
  of Memory,'' 99.}

As Moritz Hiller points out, while the first part of Friedrich Kittler's
seminal work \emph{Aufschreibesysteme} {[}Discourse Networks{]} was
practically finished by 1976, the second part, on ``the language of
technical communication,'' would be written ``between the end of 1979
and May 1982.''\footnote{Hiller, ``Unter Aufschreibesystemen,'' II--10.
  {[}Translation by the author{]}.} In other words, the article
\emph{Vergessen}, published around July 1979,\footnote{See Ulrich
  Nassen, ``Vorwort,'' in \emph{Texthermeneutik: Aktualitat, Geschichte,
  Kritik}, ed. Ulrich Nassen (Paderborn: Ferdinand Schöningh, 1979), 7.}
could be read as the epistemic hinge that draws the radical turn that
characterizes Kittler's work, and perhaps the new field of German media
theory as a whole. For, once again in the case of the
\emph{Aufschreibesysteme}, one can observe that Kittler devotes an
entire section of the second part of this book to Ebbinghaus's
psychophysics:

\begin{quote}
Nietzsche and Ebbinghaus presupposed forgetfulness, rather than memory
and its capacity, in order to place the medium of the soul against a
background of emptiness or erosion. A zero value is required before acts
of memory can be quantified. Ebbinghaus banned introspection and thus
restored the primacy of forgetting on a theoretical level. On the one
hand, there was Nietzsche's delirious joy at forgetting even his
forgetfulness; on the other, there was a psychologist who forgot all of
psychology in order to forge its algebraic formula.\footnote{Kittler,
  \emph{Discourse Networks}, 207.}
\end{quote}

\noindent As Heinz von Foerster pointed out in his 1949 Macy Conference lecture,
he would ``use results observed by Ebbinghaus during his study of the
forgetting process'' where ``the experimenter teaches a group of
subjects 100 nonsense syllables until everyone knows these syllables by
heart.''\footnote{von Foerster, ``Quantum Mechanical Theory of Memory,''
  99.} According to von Foerster, the experimenter would have examined
the subjects daily, plotting the number of syllables remembered on a
graph whose baseline would be a function of time. Thus, if ``any
observed event leaves an impression which can be divided into a lot of
elementary impressions,'' where ``any event leads initially to number
\emph{N}\textsubscript{0} of elementary impressions,'' then it would be
possible to state that ``{[}a{]}fter a certain time \emph{t} the number
of existing elementary impressions may be called \emph{N.}'' What the
cybernetician is looking for, then, ``is a function which connects the
number \emph{N} with number \emph{N}\textsubscript{0} and the time
\emph{t},'' that is to say, the ``forgetting-coefficient.''\footnote{von
  Foerster, ``Quantum Mechanical Theory of Memory,'' 98.}

The human being forgets, and such a process can be scientifically
quantified in order to know exactly both the amount of information that
can be humanly memorized---either consciously, unconsciously, or
hallucinatorily,\footnote{von Foerster, ``Quantum Mechanical Theory of
  Memory,'' 105.} and the time it takes for this information to
disappear from our being. And what Friedrich Kittler knew in his
all-too-contemporary journey was that, ironically, in order to control
the forgetting-coefficient, the only resource left to humans was to be
permanently coupled to machines whose internal operations were based
precisely on processes of \emph{forgetting}---``{[}t{]}his is how
electronic memories forget the `human being.'\,''\footnote{Kittler,
  ``Forgetting,'' 94.}

\hypertarget{cybernetics}{%
\section{Cybernetics}\label{cybernetics}}

Therein lies the paradox that runs through cybernetics and German media
theory---a paradox that prevents us from answering in a single movement
whether such a coupling would respond to a case of negative or rather
positive feedback. And just as Norbert Wiener---one of the founding
fathers of cybernetics---acknowledged J. Clerk Maxwell's paper on
governors as a cornerstone in the prehistory of this interdisciplinary
field,\footnote{See Norbert Wiener, ``Introduction,'' in
  \emph{Cybernetics: or Control and Communication in the Animal and the
  Machine} (Cambridge: MIT Press, 1985 {[}1948/1961{]}), 11.} perhaps
this article will benefit from a brief digression to try to prove how
rooted the idea of \emph{forgetting} might be in cybernetics.

Although Maxwell's work on governors---the mechanical devices used in
the 18th and 19th centuries as ``regulators of
machinery''\footnote{See J. Clerk Maxwell, ``On Governors,''
  \emph{Proceedings of the Royal Society of London} 16 (1867-1868): 271.}---is
more clearly related to the realm of physics and mechanical engineering,
its conceptual scope can shed some light on the issue of feedback in the
way it is discussed here. Accordingly, this work will prove essential in
the long run for measuring the extent of cybernetics\footnote{See Otto
  Mayr, ``Maxwell and the Origins of Cybernetics,'' \emph{Isis} 62, no.
  4 (1971): 424--44.} and, more importantly here, the human-machine
coupling in which \emph{forgetting} operates. Maxwell offers a
conceptual distinction between \emph{moderators} and \emph{governors}.
In the former, the ``the corrective action {[}\ldots{]} is directly
proportional to {[}the{]} overspeed''\footnote{Mayr, ``Maxwell and the
  Origins of Cybernetics,'' 427.} to be regulated in a given machine,
this process of regulation thus being prone to receive the internal
malfunctions of the machine in question. The latter, on the other hand,
the actual object of Maxwell's interest, would be those devices
constituted by an ``additional {[}and independent{]} mechanism that
translates any output error into a corrective action that increases
steadily until the output error has entirely disappeared.''\footnote{Mayr,
  ``Maxwell and the Origins of Cybernetics,'' 427.} This description
offers us an interesting point of reflection, because if such
independent devices are conceptually equivalent, or at least similar, to
the carriers that are ``not able to carry forever its impregnation but
only during a certain time,'' which then decay and become ``a free
carrier,''\footnote{von Foerster, ``Quantum Mechanical Theory of
  Memory,'' 101.} ready to be impregnated by the immediately following
amount of information---as if the previous one had never existed---it
would be fair to argue that the essential structure of the notion of
\emph{forgetting} can already be found in the early rise of machines,
whose increasing autonomy cybernetics described and helped to improve.
If this is the case, then perhaps it would be possible to say that what
cyberneticians like von Foerster did was to identify a kind of autonomy
whose operational nature was already present in human biological
behavior---an operational nature that scholars like Kittler would later
place at the center of an enterprise aimed at updating the humanities,
by redescribing the human condition itself.

\enlargethispage{\baselineskip}

When Norbert Wiener, the Mexican physician Arturo Rosenblueth, and
Wiener's research assistant Julian Bigelow wrote \emph{Behavior, Purpose
and Teleology}\footnote{See Arturo Rosenblueth, Norbert Wiener, and
  Julian Bigelow, ``Behavior, Purpose and Teleology,'' \emph{Philosophy
  of Science} 10, no. 1 (January 1943): 18--24.} in Cambridge,
Massachusetts, in 1943---more than 70 years after Maxwell's paper on
governors---an early yet inevitable problematization and categorization
of the human and machine (co-)existence was set in motion. Through a
conceptual analysis of behavior, the authors sought to emphasize the
role of purpose and the importance of negative feedback---teleology in
their terminology---in constructing a transversal understanding of the
modes in which both organisms and machines respond to their environment,
and pursue goals in such a context.\footnote{See Rosenblueth, Wiener,
  and Bigelow, ``Behavior, Purpose and Teleology,'' 24.} Thus,
cybernetics begins to emerge as an analytical way of thinking that
focuses on the becoming of the entities that populate this world, in
which biological and machine structures not only merge, but also
recursively explain each other.\footnote{See Rosenblueth, Wiener, and
  Bigelow, ``Behavior, Purpose and Teleology,'' 22.} Moreover, the
contribution of Maxwell's study of governors to science in general drew
a thin but strong line that allowed these 1940s researchers to argue
that every purposeful behavior in the world is driven by negative
feedback; i.e., by ``the margin of error at which {[}an{]} object {[}or
organism{]} stands at a given time with reference to a relatively
specific goal,''\footnote{Rosenblueth, Wiener, and Bigelow, ``Behavior,
  Purpose and Teleology,'' 19.} where such an error is informed by the
output of the object-organism, and the margin is then fed back to it in
the form of input. Consequently, even if these researchers would state
that, unlike human organisms, ``{[}l{]}earning and memory'' in machines
would remain ``quite rudimentary'' for a while, what Heinz von Foerster
proposed six years later with his ``forgetting-coefficient'' should be
seen here as a substantial escalation of the Massachusetts team's
argument.

The question remains, however, whether von Foerster's and Kittler's
\emph{forgetting} would obey purposeful behavior in the long run. For
both the cybernetician and the media theorist would assert that such a
process, \emph{forgetting}, prevails even in unconscious behavior. On
the one hand, von Foerster argued that in the human organism ``sensory
receptors'' can also be seen as ``short-term'' carriers ``which transmit
consciously or unconsciously their impregnation immediately to the
carriers''\footnote{von Foerster, ``Quantum Mechanical Theory of
  Memory,'' 105.} of the memory; an argument that, from the point of
view of this analysis, renders unclear the existence of a control
mechanism that regulates the margin of error between the output of the
human body and its eventual goal, if it does not simply declare its
absence---all this, of course, with the understanding that the
cybernetician did not really discuss the human-machine coupling in his
lecture at the Macy Conference, but only human memory by cybernetic
means. On the other hand, Kittler would explicitly ``attach'' machines
to the human body in order to develop his approach to \emph{forgetting}.
In this line, he will insist on that such a process requires
acknowledging that ``consciousness is only the imaginary interior view
of media standards.''\footnote{Kittler, ``The World of The Symbolic -- A
  World of The Machine,'' 132.} In other words, Kittler's media
theoretical program even radicalizes cybernetics by not only suggesting
but stating that humans have delegated the control mechanisms that
regulate the difference between their outputs and purposes---voluntarily
or not, consciously or not---to machines. ``Le monde symbolique, c'est
le monde de la machine,''\footnote{Jacques Lacan, \emph{Le séminaire II:
  Le moi dans la théorie de Freud et dans la technique de la
  psychanalyse} (Paris: Le Seuil, 1978), 63, quoted in Kittler,
  ``Vergessen,'' 202.} Kittler wrote in \emph{Vergessen}, because, as he
well knew, by the 1970s Ebbinghaus's nonsense syllables had already been
replaced, through the materialization of ``presence and
absence,''\footnote{Jacques Lacan, ``Psychoanalysis and cybernetics, or
  on the nature of language,'' in \emph{The Seminar of Jacques Lacan,
  Book II, The Ego in Freud's Theory and in the Technique of
  Psychoanalysis 1954-1955}, ed. Jacques-Alain Miller, trans. Sylvana
  Tomaselli (New York: W.W. Norton \& Company, 1991), 303.} by ASCII
series of seven or eight bits. Nevertheless, the question remains: How
is it that such an apparently unconscious technological
delegation---given that it operates as a continuous feedback---avoids
the emergence of ``clumsy behavior'' derived from the feedback, which
becomes ``positive instead of negative for certain frequencies of
oscillation''?\footnote{Rosenblueth, Wiener, and Bigelow, ``Behavior,
  Purpose and Teleology,'' 20.}

Thus, the human might have asked the machine: Why are you just reading
my output like that instead of telling me about your presence? But the
human, unconsciously, had already forgotten about awareness and such
presence, and so said nothing---leaving themself to wonder.

\hypertarget{cybernetic-synergy}{%
\section{Cybernetic Synergy}\label{cybernetic-synergy}}

Turing tests aside,\footnote{See Alan M. Turing, ``Computing Machinery
  and Intelligence,'' \emph{Mind: A Quarterly Review of Psychology and
  Philosophy} 59, no. 236 (October 1950): 433--60.} it is worth
insisting on and recalling that the problem of memory, both in ``the
animal and the machine,'' has concerned cybernetics since its nominal
foundation: ``A very important function of the nervous system, and, as
we have said, a function equally in demand for computing machines, is
that of \emph{memory}, the ability to preserve the results of past
operations for use in the future.''\footnote{Norbert Wiener, ``Computing
  Machines and the Nervous System,'' in \emph{Cybernetics: or Control
  and Communication in the Animal and the Machine} (Cambridge: MIT
  Press, 1985 {[}1948/1961{]}), 121. {[}Emphasis in the original{]}.}
But as Norbert Wiener also knew---in the wake of Alan Turing's
work\footnote{See Wiener, ``Computing Machines and the Nervous System,''
  125.}---on the radical threshold of this machine-driven era that he so
eloquently analyzed and described, a correlation between past and future
could only be activated through the mediation of a memory that could
``record quickly, be read quickly, and be erased quickly.''\footnote{Wiener,
  ``Computing Machines and the Nervous System,'' 121.} In other words,
already in the 1940s, if not even in the previous decade,\footnote{See
  Alan M. Turing, ``On Computable Numbers, With and Application to the
  Entscheidungsproblem,'' \emph{Proceedings of the London Mathematical
  Society} s2-42, no. 1 (January 1937): 230--65.} the forerunners of
cybernetics understood that any ``permanent record'' that was to
constitute an analytical source for ``future behavior,''\footnote{See
  Wiener, ``Computing Machines and the Nervous System,'' 121.} had to be
processed through technologies of erasure. Even more, Wiener argued that
just as a ``short-time memory'' could be implemented in electrical
circuits by using devices such as ``\emph{telegraph-type
repeaters},''\footnote{Wiener, ``Computing Machines and the Nervous
  System,'' 122. {[}Emphasis in the original{]}.} there was already
sufficient evidence at that time ``to believe that {[}something
similar{]} happens in our brains during the retention of
impulses.''\footnote{Wiener, ``Computing Machines and the Nervous
  System,'' 122.}

However, despite any possible implicit connection between the
developments unfolded in Princeton, (any) Cambridge, or Manchester, the
explicit articulation and inclusion of the notion of \emph{forgetting}
as a key element of the field of cybernetics is done only by Heinz von
Foerster.

The Austrian-American cybernetician reaffirmed this premise during his
later days in Pescadero, California, in the preface to his book
\emph{Understanding Understanding} (2003). Given that it is only through
learning and understanding that it is possible to discover that ``one
forgets an amount of data proportional to the amount of data one has in
store at any one time,'' a truly mathematical mind would conclude very
early, even as a schoolboy, that such a proportion ``corresponded to
some sort of logarithmic decay of memory.''\footnote{Heinz von Foerster,
  ``Preface,'' in \emph{Understanding Understanding: Essays on
  Cybernetics and Cognition} (New York: Springer-Verlag, 2003), v.}
Moreover, since it is only by browsing through bookshelves\footnote{See
  Kittler, ``Forgetting,'' 93--94.} that one arrives at the material
proof of the fact that the operations of the human mind can be
interrogated by algebraic analyses, and thus that it responds to
them---such as in ``a graph showing a decaying line labeled
`Ebbinghaus's Forgetting Curve'\,''\footnote{von Foerster, ``Preface,''
  v.}---the epistemological core of cybernetics cannot be disassociated
from a genealogy of knowledge and its apparatuses.

Thus, apart from the experiments of the psychologist Hermann Ebbinghaus,
when it comes to cybernetic apparatuses as such, it might seem that
there are not many applied cases that can help us to genealogically and
archaeologically investigate how the operational constitution of the
human-machine coupling---perhaps the main condition of contemporary
media cultures---is based on the cybernetic operation of
\emph{forgetting}. There is, however, at least one case that is useful.

In 1971, an epistolary exchange between technologists in Chile and
England activated a plan to design a system that would apply the
principles of management cybernetics to the Chilean economy.\footnote{See
  Diego Gómez-Venegas, ``Cybersyn y la memoria simbólica del papel,''
  \emph{Artnodes,} no. 23 (January 2019).} This country had just
undergone unprecedented socialist political reforms that included the
nationalization of several companies. This process required the
implementation of truly sophisticated management techniques if the
government was to effectively keep pace with the increasing complexity
of the state-run economy. Thus, this correspondence began with the
technological director of the Chilean Development Agency, Fernando
Flores, who asked the management cybernetics expert Stafford Beer to
guide the implementation of his own work in the Chilean context. The
exchange continued with the British cybernetician typing his response to
Flores's call, which was not only positive, but enthusiastic.\footnote{See
  Alberto Mayol, ``Cuando la utopía es el fórceps para alumbrar una
  nueva era,'' in \emph{The Counterculture Room}, ed. FabLab Santiago
  (Barcelona: Polígrafa, 2017), 24--25.} Thus, the project later known
as Cybersyn was configured as an enterprise consisting of four parts: 1)
a network of teletypewriters called Cybernet, which would place an
input/output node in each nationalized company participating in the
system; 2) a central processing node called Cyberstride, which would
statistically analyze the data sent by the companies; 3) a simulation
suite called CHECO, which would model possible scenarios for the local
economy, based on relations between the processed information and
international economic flows; and 4) an operations room called the
Opsroom, where seven senior government officials and experts would
discuss and make data-driven decisions, which would then be fed back to
the nodes in the companies, always through the system's
infrastructure.\footnote{See Eden Medina, \emph{Cybernetic
  Revolutionaries: Technology and Politics in Allende's Chile}
  (Cambridge: The MIT Press, 2011), 96.} And while important scholarship
has been written about the socio-technical and political aspects that
surrounded the case and would have defined its scope,\footnote{See Claus
  Pias, ``Unruhe und Steuerung. Zum utopischen Potential der
  Kybernetik,'' in \emph{Die Unruhe der Kultur. Potentiale des
  Utopischen}, ed. Jörn Rosen and Michael Fehr (Weilerswist-Metternich:
  Velbrück Wissenschaft, 2004); ``Der Auftrag. Kybernetik und Revolution
  in Chile,'' in \emph{Politiken der Medien}, ed. Daniel Gethmann and
  Markus Stauff (Zurich-Berlin: Diaphanes, 2005); Sebastian Vehlken,
  ``Environment for Decision -- Die Medialität einer kybernetischen
  Staatsregierung. Eine medienwissenschaftliche Untersuchung des
  Projekts Cybersyn in Chile 1971-73'' (master\RL{'}s thesis,
  Ruhr-Universität Bochum, 2004); Eden Medina, ``Designing Freedom,
  Regulating a Nation: Socialist Cybernetics in Allende\RL{'}s Chile,''
  \emph{Journal of Latin American Studies} 38, no. 3 (2006): 571--606;
  \emph{Cybernetic Revolutionaries.}} this article argues that not
enough has yet been said about the actual cybernetic human-machine
coupling that Cybersyn could have set in motion.

Hence the importance of \emph{forgetting} as a (first) conceptual
apparatus linking cybernetics and German media theory. And for the same
reason, there is the importance of Project Cybersyn as perhaps one of
the few applied cases of cybernetics that could help us to witness
whether such a theoretical apparatus transcends its \emph{symbolic}
status and proves to be as \emph{real} as it is
\emph{imaginary}\textsubscript{78}---that
is, truly technological.

In each input/output node of the Cybernet, a human being would type a
series of information according to the protocols implemented by the
system, by the project. Thus, through the persistent severity of the
Q-W-E-R-T-Y keyboards, through the implacable grids drawn by the punched
paper tapes of the teletypewriter, this procedure would silently
eradicate any space for semantics and interpretation. The bodies of the
typists would then become, perforation by perforation, a surface\marginnote{\textsubscript{78}\setcounter{footnote}{78} This refers to Jacques Lacan's registers of
  the real, the symbolic, and the imaginary which Friedrich Kittler
  translated into media theory in order to explain how the technological
  functions of transmission, processing, and storage define our
  situation. See Kittler ``The World of the Symbolic -- A World of The
  Machine,'' 135. {[}Note added in the 2024 update for clarification{]}.} of
inscription---an always erasable surface of inscription. Attached to the
machine by a procedure that divided every meaning into individual
symbols, and thus, into a code of presence and absence that could only
be read by other machines, ``humans change their position.''\footnote{Kittler,
  \emph{Gramophone, Film, Typewriter}, 210.} This is why random access
memory seems to have been a key and distributed cybernetic component of
Project Cybersyn.\footnote{In an earlier paper, I argued that the
  typists behind Cybernet's teletypewriters had to be considered
  read-only memory modules, or ROMs. I was wrong. This was a
  misinterpretation of Kittler's article ``The World of the Symbolic --
  A World of the Machine,'' which, ironically, I formulated before
  including ``Forgetting'' in my research. According to Kittler, every
  typist, or more radically, every human being, must be considered here
  as a RAMs. See Gómez-Venegas, ``Cybersyn y la memoria simbólica del
  papel''; and Kittler, ``Forgetting.''} But again, this is still a
conjecture. More research is needed to determine the exact nature of the
protocols mentioned above, and thus, the exact structure of the series
of information that the typists typed into their teletypewriters. In the
same way, a media archaeological work on such a device is also
necessary. This is so, because we must be able to structurally define
the input/output operations of the teletypewriter with precision if we
want to technologically understand the data-driven human-machine
coupling that this device could have activated. However, what we already
know---e.g., that Cybersyn's telex network proved to be reliable and
strong,\footnote{See Medina, \emph{Cybernetic Revolutionaries}, 149.}
that a specific protocol was designed for each industrial sector and
company, and that every day a typist sat in front of a teletypewriter
device to serially enter the company's daily
operations\footnote{Isaquino Benadof (former Cyberstride chief
  programmer), email message to the author, August 21, 2018.}---allows
this article to outline a preliminary hypothesis, suggesting that such a
network would have constituted an always erasable, an always rewritable,
network of information and telecommunications where device and human
being were part of a single but interconnected random access node, where
the technological operation of \emph{forgetting} was
sovereign.\footnote{A friendly reminder to the reader that the original
  version of this article was published in early-2020---but actually
  written in late 2018---and that all of the research necessary to turn
  this informed conjecture into an archivally documented investigation
  has already been done---a process that, in turn, has refined and even
  corrected some of the statements in this and the next paragraph. Until
  the dissertation containing this final research is published, see
  Diego Gómez-Venegas, ``Encoding from/to the Real: On Cybersyn's
  Symbolic Politics of Transmission,'' in \emph{Frictions: Inquiries
  into Cybernetic Thinking and Its Attempts towards
  Mate{[}real{]}ization} , ed. Diego Gómez-Venegas (Lüneburg: meson
  press, 2023), 91--117. {[}Note added in the 2024 update for
  clarification{]}.}

Similarly, the central processing node, an IBM 360/50 mainframe computer
connected to a teletype machine that served as an input peripheral,
received the data coming from the Cybernet nodes in the form of punched
paper tapes.\footnote{Benadof, email message to the author.} There, a
software suite programmed in Assembler and PL360---plus some routines
written in Cobol and Fortran where punched paper cards were also
fundamental---allowed the daily statistical analysis of the
data.\footnote{Benadof, email message to the author.} In other words,
the permanent procedures of the system, its ROM, either written in the
machine itself or stored on perforated pieces of paper that the machine
could handle and decode, would ensure that \emph{forgetting} would not
be forgotten.

If ``it is only on paper that the `human being' originates,'' it may be
that, as a reborn or renewed species, ``humans change their position''
through it as well. Thus, it seems that German media theory and
cybernetics---as an essential composite---can help us understand how and
why media cultures have moved from storing to remembering to erasing to
\emph{forgetting}; and, perhaps more importantly, how this technological
process came to determine our relationship to the past, the present, and
the future---our condition as historical or, rather,\newpage\noindent
more-than-historical beings. A case like Cybersyn provides the facts;
Kittler's and von Foerster's theories of \emph{forgetting} provides the
methods.



In summary, this article has first attempted to genealogically trace the
emergence and relevance of the notion of \emph{forgetting} in both
German media theory and in cybernetics. To this end, connections were
drawn between the early works of Friedrich Kittler and Heinz von
Foerster. This was done in the wake of Jan Müggenburg's article ``Bats
in the Belfry,'' in which he points out that von Foerster's thinking
could have reached Kittler indirectly through Siegfried J. Schmidt, for
example.\footnote{Müggenburg, ``Bats in the Belfry,'' 476--77.}
Alternatively, however, this text has re-read and examined Kittler's
``Forgetting'' and von Foerster's \emph{Quantum Mechanical Theory of
Memory} to suggest a more direct influence, the ramifications of which
can also be seen in Kittler's \emph{Discourse Networks}, as well as in
\emph{Gramophone, Film, Typewriter}, and \emph{The World of the Symbolic
-- A World of the Machine}. Second, this article has also studied Moritz
Hiller's \emph{Unter Aufschreibesystemen} in order to insist with him
that just as Shannon's \emph{Mathematical Theory of Communication} is
not a founding element of German media theory, the \emph{problem of
memory}, and more precisely the question of \emph{forgetting}, signals
such a founding root. Accordingly, the fact that this question---as a
thin, sometimes hard to see, but nevertheless strong, matted
thread---crosses German media theory and cybernetics weaving a web of
relations, can ultimately be seen as a unique fabric that constitutes
the common core of both fields. Finally, this article has attempted to
outline a preliminary re-examination of Project Cybersyn through the
question of \emph{forgetting}, in order to not only point out the
centrality that this transversal fabric would play in the analysis of
media cultures and their antecedents, but also to suggest that such a
project could prove critical in assessing the scope of the notion of
\emph{forgetting}, and, similarly, the performance of German media
theories and cybernetics when it comes to the analysis of modern
knowledge and its technologies.

As a result, some questions are only sketched here, not in search of
immediate answers, but in an attempt to delineate future areas of
investigation. For example, only two of Cybersyn's
subprojects---Cybernet and Cyberstride---have been preliminarily
discussed in these pages, deliberately avoiding the spectacular Opsroom,
the part of the project that has received, unfairly, all the attention
in the past. And while I have stated in previous articles that the
Opsroom must be forgotten in order to grasp the true cybernetic scope of
Project Cybersyn,\footnote{See Gómez-Venegas, ``Cybersyn y la memoria
  simbólica del papel.''} a more nuanced approach will make it possible
to understand that only by focusing our attention and archaeological
gaze on efforts like Cybernet and Cyberstride will we be able to fully
comprehend the role that spaces and devices such as the Opsroom would
have played in the configuration of technologies of \emph{forgetting},
as well as in the transitional state of \emph{the archive}, and even
more so in the permanent but somehow always forgotten presence of
cybernetics in our cultures.

``Black out.''\footnote{Kittler, ``Forgetting,'' 116.} In times of
cybernetics, when presence equals present in a constant feedback loop,
``memory is literally permanently in transition.''\footnote{Ernst,
  ``Archives in Transition,'' 97.}




\section{Bibliography}\label{bibliography}

\begin{hangparas}{.25in}{1} 



Dotzler, Bernhard. ``The Sirens Go Silent - Friedrich Kittler Part 11:
Bernhard Dotzler.'' Filmed March 2013 at Deutsches Haus at NYU, New
York, NY, video, 48:34. \url{https://youtu.be/_uZJg8lwfJU}.

Dotzler, Bernhard. ``Idiocy, Forgetting, and Outdatedness: Friedrich
Kittler's Avant-Gardism and the `Time for Other Stories'.'' In \emph{The
Technological Introject: Friedrich Kittler between Implementation and
the Incalculable}, edited by Jeffrey Champlin et al, 35--45. New York:
Fordham University Press, 2018.

Armitage, John. ``From Discourse Networks to Cultural Mathematics: An
Interview with Friedrich A. Kittler.'' \emph{Theory, Culture \& Society}
23, no. 7--8 (2006): 17--38.

Ernst, Wolfgang. ``Archives in Transition: Dynamic Media Memories.'' In
\emph{Digital Memory and the Archive}, edited by Jussi Parikka, 95--101.
Minneapolis: University of Minnesota Press, 2013.

Gómez-Venegas, Diego. ``Forgetting / Cybernetics. History of Media
Studies.'' \emph{CAC/CAFAcat Editorial} 1, no. 1 (2020).

Gómez-Venegas, Diego. ``Encoding from/to the Real: On Cybersyn's
Symbolic Politics of Transmission.'' In \emph{Frictions: Inquiries into
Cybernetic Thinking and Its Attempts towards Mate{[}real{]}ization},
edited by Diego Gómez-Venegas, 91--117. Lüneburg: meson press, 2023.

Gómez-Venegas, Diego. ``Cybersyn y la memoria simbólica del papel.''
\emph{Artnodes}, no. 23 (January 2019).

Hiller, Moritz. ``Unter Aufschreibesystemen: `Eine Adresse im IC der
Kultur'.'' \emph{Metaphora} 1 (2015): II/1--26.

Kittler, Friedrich. ``Vergessen.'' In \emph{Texthermeneutik: Aktualität,
Geschichte, Kritik}, edited by Ulrich Nassen, 195--221. Paderborn:
Ferdinand Schöningh, 1979.

Kittler, Friedrich. ``Forgetting.'' \emph{Discourse} 3 (1981{[}1979{]}):
88--121.

Kittler, Friedrich. \emph{Aufschreibesysteme 1800/1900}. München: Fink,
1985.

Kittler, Friedrich. \emph{Discourse Networks 1800/1900}. Stanford:
Stanford University Press, 1990.

Kittler, Friedrich. ``The World of The Symbolic -- A World of The
Machine.'' In \emph{Literature, Media, Information Systems}, edited by
John Johnston, 130--46. Amsterdam: OPA Amsterdam B.V., 1997.

Kittler, Friedrich. ``Die Welt des Symbolischen --- eine Welt der
Maschine.'' In \emph{Draculas Vermächtnis. Technische Schriften},
58--80. Leipzig: Reclam, 1993.

Kittler, Friedrich. \emph{Gramophone, Film, Typewriter}. Stanford:
Stanford University Press, 1999.

Lacan, Jacques. \emph{Le séminaire II: Le moi dans la théorie de Freud
et dans la technique de la psychanalyse}. Paris: Le Seuil, 1978. Quoted
in Kittler, ``Vergessen.''

Lacan, Jacques. ``Psychoanalysis and cybernetics, or on the nature of
language.'' In \emph{The Seminar of Jacques Lacan, Book II, The Ego in
Freud's Theory and in the Technique of Psychoanalysis 1954-1955}, edited
by Jacques-Alain Miller, translated by Sylvana Tomaselli, 294--308. New
York: W.W. Norton \& Company, 1991.

Mayol, Alberto. ``Cuando la utopía es el fórceps para alumbrar una nueva
era.'' In \emph{The Counterculture Room}, ed. FabLab Santiago, 16--43.
Barcelona: Polígrafa, 2017.

Mayr, Otto. ``Maxwell and the Origins of Cybernetics.'' \emph{Isis} 62,
no. 4 (Winter 1971): 424--44.

Maxwell, J. Clerk. ``On Governors,'' \emph{Proceedings of the Royal
Society of London} 16 (1867--1868): 270--83.

Medina, Eden. ``Designing Freedom, Regulating a Nation: Socialist
Cybernetics in Allende's Chile.'' \emph{Journal of Latin American
Studies} 38, no. 3 (2006): 571--606.

Medina, Eden. \emph{Cybernetic Revolutionaries: Technology and Politics
in Allende's Chile}. Cambridge: The MIT Press, 2011.

Müggenburg, Jan. ``Bats in the Belfry: On the Relationship of
Cybernetics and German Media Theory.'' \emph{Canadian Journal of
Communication} 42 (2017): 467--84.

Nassen, Ulrich. ``Vorwort.'' In \emph{Texthermeneutik: Aktualität,
Geschichte, Kritik}, edited by Ulrich Nassen, 7. Paderborn: Ferdinand
Schöningh, 1979.

Nietzsche, Friedrich. ``On the Utility and Liability of History for
Life.'' In \emph{Unfashionable Observations}, 83--168. Stanford:
Stanford University Press, 1995.

Pias, Claus. ``Unruhe und Steuerung. Zum utopischen Potential der
Kybernetik.'' In \emph{Die Unruhe der Kultur. Potentiale des
Utopischen}, edited by Jörn Rosen and Michael Fehr, 313--25.
Weilerswist-Metternich: Velbrück Wissenschaft, 2004.

Pias, Claus. ``Der Auftrag. Kybernetik und Revolution in Chile.'' In
\emph{Politiken der Medien}, edited by Daniel Gethmann and Markus
Stauff, 131--53. Zurich-Berlin: Diaphanes, 2005.

Rosenblueth, Arturo, Norbert Wiener and Julian Bigelow. ``Behavior,
Purpose and Teleology.'' \emph{Philosophy of Science} 10, no. 1 (January
1943): 18--24.

Shannon, Claude E. and Warren Weaver. \emph{The Mathematical Theory of
Communication}. Urbana: University of Illinois Press, 1998 {[}1949{]}.

Turing, Alan M. ``On Computable Numbers, With an Application to the
Entscheidungsproblem.'' \emph{Proceedings of the London Mathematical
Society} s2-42, no. 1 (January 1937): 230--65.

Turing, Alan M. ``Computing Machinery and Intelligence.'' \emph{Mind: A
Quarterly Review of Psychology and Philosophy} 59, no. 236 (October
1950): 433--60.

Vehlken, Sebastian. ``Environment for Decision -- Die Medialität einer
kybernetischen Staatsregierung. Eine medienwissenschaftliche
Untersuchung des Projekts Cybersyn in Chile 1971--73.'' Master's thesis,
Ruhr-Universität Bochum, 2004.

von Foerster, Heinz. ``Quantum Mechanical Theory of Memory.'' In
\emph{Cybernetics \textbar{} Kybernetik: The Macy-Conferences 1946-1953.
Volume I / Band I. Transactions / Protokolle}, edited by Claus Pias,
98--121. Zurich-Berlin: Diaphanes, 2003.

von Foerster, Heinz. ``Preface.'' In \emph{Understanding Understanding:
Essays on Cybernetics and Cognition}, v-ix. New York: Springer-Verlag,
2003.

Wiener, Norbert. ``Introduction.'' In \emph{Cybernetics: or Control and
Communication in the Animal and the Machine}, 1--29. Cambridge: MIT
Press, 1985 {[}1948/1961{]}.

Wiener, Norbert. ``Computing Machines and the Nervous System.'' In
\emph{Cybernetics: or Control and Communication in the Animal and the
Machine}, 116--32. Cambridge: MIT Press, 1985 {[}1948/1961{]}.

Winthrop-Young, Geoffrey. ``Introduction.'' In \emph{Kittler and the
Media}, 1--7. Cambridge: Polity Press, 2011.



\end{hangparas}


\end{document}