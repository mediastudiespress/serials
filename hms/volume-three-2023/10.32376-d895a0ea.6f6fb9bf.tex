% see the original template for more detail about bibliography, tables, etc: https://www.overleaf.com/latex/templates/handout-design-inspired-by-edward-tufte/dtsbhhkvghzz

\documentclass{tufte-handout}

%\geometry{showframe}% for debugging purposes -- displays the margins

\usepackage{amsmath}

\usepackage{hyperref}

\usepackage{fancyhdr}

\usepackage{hanging}

\hypersetup{colorlinks=true,allcolors=[RGB]{97,15,11}}

\fancyfoot[L]{\emph{History of Media Studies}, vol. 3, 2023}


% Set up the images/graphics package
\usepackage{graphicx}
\setkeys{Gin}{width=\linewidth,totalheight=\textheight,keepaspectratio}
\graphicspath{{graphics/}}

\title[Understanding the German Media System]{Understanding the German Media System with the Help of Bourdieu and Elias: Historical Sociology of Press-Political Relations in Germany} % longtitle shouldn't be necessary

% The following package makes prettier tables.  We're all about the bling!
\usepackage{booktabs}

% The units package provides nice, non-stacked fractions and better spacing
% for units.
\usepackage{units}

% The fancyvrb package lets us customize the formatting of verbatim
% environments.  We use a slightly smaller font.
\usepackage{fancyvrb}
\fvset{fontsize=\normalsize}

% Small sections of multiple columns
\usepackage{multicol}

% Provides paragraphs of dummy text
\usepackage{lipsum}

% These commands are used to pretty-print LaTeX commands
\newcommand{\doccmd}[1]{\texttt{\textbackslash#1}}% command name -- adds backslash automatically
\newcommand{\docopt}[1]{\ensuremath{\langle}\textrm{\textit{#1}}\ensuremath{\rangle}}% optional command argument
\newcommand{\docarg}[1]{\textrm{\textit{#1}}}% (required) command argument
\newenvironment{docspec}{\begin{quote}\noindent}{\end{quote}}% command specification environment
\newcommand{\docenv}[1]{\textsf{#1}}% environment name
\newcommand{\docpkg}[1]{\texttt{#1}}% package name
\newcommand{\doccls}[1]{\texttt{#1}}% document class name
\newcommand{\docclsopt}[1]{\texttt{#1}}% document class option name


\begin{document}

\begin{titlepage}

\begin{fullwidth}
\noindent\LARGE\emph{French-German Communication Research
} \hspace{27mm}\includegraphics[height=1cm]{logo3.png}\\
\noindent\hrulefill\\
\vspace*{1em}
\noindent{\Huge{Understanding the German Media System with the Help of Bourdieu and Elias:\\\noindent Historical Sociology of Press-Political\\\noindent Relations in Germany\par}}

\vspace*{1.5em}

\noindent\LARGE{Nicolas Hubé} \href{https://orcid.org/0000-0001-7226-7546}{\includegraphics[height=0.5cm]{orcid.png}}
\par\marginnote{\emph{Nicolas Hubé, ``Understanding the German Media System with the Help of Bourdieu and Elias: Historical Sociology of Press-Political Relations in Germany,'' \emph{History of Media Studies} 3 (2023), \href{https://doi.org/10.32376/d895a0ea.6f6fb9bf}{https://doi.org/ 10.32376/d895a0ea.6f6fb9bf}.} \vspace*{0.75em}}
\vspace*{0.5em}
\noindent{{\large\emph{Université de Lorraine}, \href{mailto:nicolas.hube@univ-lorraine.fr}{nicolas.hube@univ-lorraine.fr}\par}} \marginnote{\href{https://creativecommons.org/licenses/by-nc/4.0/}{\includegraphics[height=0.5cm]{by-nc.png}}}

% \vspace*{0.75em} % second author

% \noindent{\LARGE{<<author 2 name>>}\par}
% \vspace*{0.5em}
% \noindent{{\large\emph{<<author 2 affiliation>>}, \href{mailto:<<author 2 email>>}{<<author 2 email>>}\par}}

% \vspace*{0.75em} % third author

% \noindent{\LARGE{<<author 3 name>>}\par}
% \vspace*{0.5em}
% \noindent{{\large\emph{<<author 3 affiliation>>}, \href{mailto:<<author 3 email>>}{<<author 3 email>>}\par}}

\end{fullwidth}

\vspace*{1em}


\hypertarget{abstract}{%
\section{Abstract}\label{abstract}}

A scientist's work in and about a foreign country implies a distancing
from his or her own analytical routines. This paper aims to present a
reflexive approach toward the use of ``French'' media studies patterns
on a German case, and the heuristic gains of mixing them with the German
ones. Starting from a sociological paradox and the surprising lack of
research on German journalistic institutions, the aim of this paper is
to show that the academic socialization of a French researcher can make
heuristic contributions to research on Germany. This paper is based on a
long-term historical sociology book (starting in 1918) on interactions
between press and politics in Germany, and especially the continuum of
both the political journalists' association (nowadays:
\emph{Bundespressekonferenz}) and governmental spokesperson agencies
(nowadays: \emph{Bundespresseamt}). The paper first briefly presents the
two national analytical traditions as starting points which have to be
overcome in this specific research on Germany. The following section
deals more precisely with the new framework of the research on the
relations between journalists and politicians. The aim, finally, is to
show how this method of bringing together French journalism and
political sociology (influenced by a constructivist and historical


\enlargethispage{2\baselineskip}

\vspace*{2em}

\noindent{\emph{History of Media Studies}, vol. 3, 2023}


 \end{titlepage}

\noindent approach) with the systemic German media studies approach can lead to
specific advances in our understanding of the structure of the national
public sphere.


\vspace*{2em}



\hypertarget{introduction-german-specificity-in-an-international-comparison}{%
\section{Introduction: German Specificity in an International Comparison}\label{introduction-german-specificity-in-an-international-comparison}}

Twenty years ago, Raymond Kuhn and Erik Neveu wanted to challenge ``the
effects of analytical routines on the approach to the study of political
journalism by political communication researchers.''\footnote{Erik Neveu
  and Raymond Kuhn, ``Political Journalism: Mapping the Terrain,'' in
  \emph{Political Journalism: New Challenges, New Practices}, ed.
  Raymond Kuhn and Erik Neveu (London: Routledge, 2002), 1.} A
scientist's work in and about a foreign country implies a distancing
from his or her own analytical routines. The comparative approach is not
only a theoretical one, it is a practice and a journey in various
foreign environments, in several countries. To be engaged in
non-standardized international research is a permanent intellectual
rearrangement,\footnote{Michel de Certeau, \emph{The Practice of
  Everyday Life} (Berkeley: University of California Press, 1984).} a
practice of different academic cultures, and of immersion in academic
fields that are not all homologous. A scholar writing about Germany from
France (and vice versa) always carries with him ideas, prerequisites,
and analytical traditions from which he engages in his work. This is all
the more true when the two research traditions are very
different.\footnote{Nicolas Hubé, ``À la recherche d'une universalité du
  journalisme: la \emph{Journalistik} allemande,'' \emph{Revue française
  des sciences de l'information et de la communication}, no. 19 (2020);
  Lisa Bolz, ``Recherches sur le journalisme en France et en Allemagne:
  un dialogue impossible?,'' \emph{Revue française des sciences de
  l'information et de la communication}, no. 18 (2019). See also Lisa
  Bolz in this issue.} The research project described in the current
paper grew from a sociologist's enigma.\footnote{Nicolas Hubé, \emph{La
  politique des chemins courts: Un siècle de relations entre
  journalistes et communicants gouvernementaux en Allemagne
  (1918--2018)} (Vulaines-sur-Seine: Editions du Croquant, 2022). For a
  first step, see Hubé, ``Understanding the Off-The-Record as a Social
  Practice: German Press-Politics Relations Seen from France,''
  \emph{Laboratorium: Russian Review of Social Research} 9, no. 2
  (2017).} This paper proposes a reflexive approach toward the use of
``French'' media studies patterns on a German case, and identifies
several heuristic gains of such a mixed approach.

German journalism appears to be an almost singular case. One image is
strongly anchored: In Germany, distance between journalists and
politicians is the dominant mode. In national\footnote{Klaus-Dieter
  Altmeppen and Martin Löffelholz, ``Zwischen Verlautbarungsorgan und
  `vierter Gewalt': Strukturen, Abhängigkeiten und Perspektiven des
  politischen Journalismus,'' in \emph{Politikvermittlung und Demokratie
  in der Mediengesellschaft}, ed. Ullrich Sarcinelli (Bonn:
  Bundeszentrale für politische Bildung, 1998); Hans-Matthias
  Kepplinger, \emph{Journalismus als Beruf} (Wiesbaden: VS Verlag,
  2011).} or comparative international studies,\textsuperscript{6} German journalists
commonly declare that they are the least subject to competition and the
pressures of commercialization, and their professional beliefs are often
permeated by a strong critical sense. Moreover, the political conception
of their work is one of a professional commitment to
democracy.\textsuperscript{7} However, as soon as the research shifts from
perceptions of the journalistic role to understanding them through
practices, the picture becomes much more varied.\textsuperscript{8} In March 2014, for example, the
Constitutional Court issued a judgment criticizing the over-political
nature of appointments to the board of the public channel ZDF, as this
distorted the distribution of editorial jobs, which are otherwise
arranged systematically as a ``ticket'' with one journalist close to the
government party and one deputy from the opposition party.\textsuperscript{9}
Comparing the German and American press corps, Matthias Revers observes
that German journalists seem to be much less confrontational with
politicians than their\marginnote{\textsuperscript{6} Wolfgang
  Donsbach and Thomas Patterson, ``Political News Journalists:
  Partisanship, Professionnalism, and Political Roles in Five
  Countries,'' in \emph{Comparing Political Communication: Theories,
  Cases and Challenges}, ed. Frank Esser and Barbara Pfetsch (Cambridge:
  Cambridge University Press, 2004); Thomas Hanitzsch and Rosa Berganza,
  ``Explaining Journalists' Trust in Public Institutions Across 20
  Countries: Media Freedom, Corruption, and Ownership Matter Most,''
  \emph{Journal of Communication} 62, no. 5 (2012).} North\marginnote{\textsuperscript{7} Donsbach and Patterson, ``Political News
  Journalists,'' 261--64.}\marginnote{\textsuperscript{8} Michael Meyen
  and Claudia Riesmeyer, \emph{Diktatur des Publikums: Journalisten in
  Deutschland} (Konstanz: UVK, 2009).} American\marginnote{\textsuperscript{9}\setcounter{footnote}{9} Valérie
  Robert, ``Staatsfreiheit ou intervention de l'État? Le modèle allemand
  de l'audiovisuel public,'' \emph{Sur le journalism} 2, no. 2 (2013).} counterparts.\footnote{Matthias
  Revers, \emph{Contemporary Journalism in the US and Germany: Agents of
  Accountability} (New York: Palgrave MacMillan, 2017).} Wegmann and
Mehnert underline that ``beyond personal acquaintances, there exists a
structural connection between politics and the media.''\footnote{Nikolaus
  Wegmann and Ute Mehnert, ``\emph{Scoop-o-mania}, l'introduction du
  scoop dans la vie politique allemande,'' \emph{Le Temps des Médias,}
  no. 7 (2006): 148--49.} However, this conclusion is not particularly
widespread in the literature. Niklas Luhmann's critique remains very
systemic, and not empirically based,\footnote{Niklas Luhmann, \emph{The
  Reality of the Mass Media} (Stanford: Stanford University Press,
  2000).} and critical media studies are historically rare in
Germany,\footnote{Maria Löblich, Niklas Venema, and Elisa Pollack,
  ``West Berlin's Critical Communication Studies and the Cold War: A
  Study on Symbolic Power from 1948 to 1989,'' \emph{History of Media
  Studies} 2 (2022).} whereas in France the critical sociology approach
(sometimes quite normative) is widespread. Recently, this criticism has
been most strongly and almost exclusively voiced inside the political
field by extreme right-wing populist movements.\footnote{Kristoffer Holt
  and André Haller, ``What Does `Lügenpresse' Mean? Expressions of Media
  Distrust on PEGIDA's Facebook Pages,'' \emph{Politik} 20, no. 4
  (2017).}

The German specificity does not end at this paradox. At the governmental
level, a centralized and hierarchically important government spokesman's
department was set up at the end of the First World War. This department
has been ongoing since then. Despite its high degree of brutality, which
has no comparison with current institutions, the \emph{Reichsministerium
für Volksaufklärung und Propaganda}, the institution headed by Joseph
Goebbels from 1933 to 1945, is part of a state apparatus
continuum.\footnote{Elke Fröhlich, ``Joseph Goebbels: Profil de sa
  propagande (1926--1939),'' in \emph{Joseph Goebbels} \emph{Journal
  1933--1939}, ed. Pierre Ayçoberry and Barbara Lambauer (Paris:
  Tallandier, 2007); Daniel Mühlenfeld, ``Vom Kommissariat zum
  Ministerium: Zur Gründungsgeschichte des Reichsministeriums für
  Volksaufklärung und Propaganda,'' in \emph{Hitlers Kommissare:
  Sondergewalten in der nationalsozialistischen Diktatur}, ed. Rüdiger
  Hachtmann and Winfried Süß (Konstanz: Wallstein Verlag, 2006).} In the
political information field, since the postwar period German journalists
have had a central instrument for the regulation of news production
under their own responsibility: the \emph{Bundespressekonferenz}
(hereafter, BPK). Founded in 1949 together with the Federal Republic,
the BPK gathers all parliamentary journalists working for the German
media and shares its building with the Association of Foreign
Journalists (\emph{Verein der ausländischen Presse}---hereafter, VAP).
This unique institution gathers journalists covering politics from the
federal capital. They are correspondents sent by their editorial offices
in Berlin, Frankfurt am Main, Hamburg, Munich, Cologne, or Mainz. The
BPK has been reproduced at the level of each state, as the
``\emph{Landespressekonferenz}.'' The reason for this replication is the
high level of autonomy of the Länder (states) in the federal system. The
governments of the Länder have significant powers, and the regional
press is particularly influential. Another specificity is that the BPK
is run by journalists. The government is only invited to explain its
policy to journalists three times a week. It is represented by the State
Secretary responsible for the spokesman's office and ministerial
spokesmen. With lower attendance, ministers can take direct part in
these press conferences (forty-four times in 2016 and twenty-eight times
in 2017), and very seldom the Chancellor (one time in both
years).\footnote{Hubé, \emph{La politique des chemins courts}, 321.} The
conferences are opened, moderated, and then closed by a member of the
BPK's administrative board, with no say from the State Secretary. Over
time, a gradual construction of corporatist institutions has been
observed: political journalists' associations such as \emph{Verein
Berliner Presse} and \emph{Reichsverband der deutschen Presse} under
Weimar and \emph{Bundespressekonferenz} (BPK) for the FRG; the German
Journalists' Association representing and defending journalists'
professional and democratic role in the FRG (\emph{Deutsche
Journalistenverband} (DJV); and spokesperson's offices (\emph{Vereinigte
Presseabteilung der Reichsregierung} under Weimar and
\emph{Bundespresseamt} in the FRG). Such an institutionalization of
press-political relations is highly unusual compared to the situations
seen elsewhere in the so-called consolidated democracies.

Rather surprisingly, these places (and particularly the BPK) have not
been the object of specific investigations.\footnote{There are some
  works produced by journalists: Gunnar Krüger, \emph{``Wir sind doch
  kein exklusiver Club!'': Die Bundespressekonferenz in der Ära
  Adenauer} (Münster: Lit, 2005); Tissy Bruns, \emph{Republik der
  Wichtigtuer: Ein Bericht aus Berlin} (Bonn: Bundeszentrale für
  politische Bildung, 2007). Some others are dedicated to specific
  issues: Christian Nuernbergk and Jan-Hinrik Schmidt, ``Twitter im
  Politikjournalismus: Ergebnisse einer Befragung und Netzwerkanalyse
  von Hauptstadtjournalisten der Bundespressekonferenz,''
  \emph{Publizistik} 65, no. 1 (2020).} Comparative work focuses on the
Länder level,\footnote{Denise Burgert, \emph{Politisch-mediale
  Beziehungsgeflechte: Ein Vergleich politikfeldspezifische
  Kommunikationskulturen in Deutschland und Frankreich} (Berlin, LIT
  Verlag, 2010); Revers, \emph{Contemporary Journalism in the US and
  Germany.}} while few studies focus on federal communication over
time.\footnote{A number of edited books study these particular moments:
  Klaus Arnold et al., eds., \emph{Von der Politisierung der Medien zur
  Medialisierung des Politischen? Zum Verhältnis von Medien,
  Öffentlichkeiten und Politik im 20. Jahrhundert} (Leipzig: Leipziger
  Universitätsverlag, 2010); Frank Bösch and Norbert Frei, eds.,
  \emph{Medialisierung und Demokratie im 20. Jahrhundert} (Göttingen:
  Wallstein Verlag, 2006); Bernhard Fulda, \emph{Press and Politics in
  the Weimarer Republic} (Oxford: Oxford University Press, 2013); Thomas
  Mergel, \emph{Parlamentarische Kultur in der Weimarer Republik}
  (Düsseldorf: Droste Verlag, 2002).} The only exception is the period
of National Socialism, which is particularly well-documented.\footnote{Ernest
  K. Bramsted, \emph{Goebbels und die nationalsozialistische Propaganda
  1925--1945} (Frankfurt: Fischer, 1971); Fröhlich, ``Joseph Goebbels'';
  Elke Fröhlich, ``Joseph Goebbels, un propagandiste profiteur de
  guerre,'' \emph{Joseph Goebbels} \emph{Journal 1939­--1942} (Paris:
  Tallandier, 2009); Mühlenfeld, ``Vom Kommissariat zum Ministerium'';
  Bernd Sösemann, ed., \emph{Propaganda: Medien und Öffentlichkeit in
  der NS-Diktatur} (Stuttgart: Franz Steiner Verlag, 2011);} However, the strong institutionalization of
press-political relations surprised observers of press departments as
early as the interwar period.\textsuperscript{21} But the German case is understudied
relative to other countries. In the UK, the parliamentary press gallery
is where the framing of the political situation is
co-produced.\textsuperscript{22} The rapid expansion of a
business sector of communication and the liberal organization of the
press reinforced the trend towards the emergence of a so-called
\emph{public relations democracy.}\textsuperscript{23} Washington
is a priori similar to Berlin/Bonn: a federal capital strongly
structured by the proximity of interactions, made up of interpersonal
relations and revolving doors.\textsuperscript{24} The
press corps shares important sociabilities.\textsuperscript{25} However, observers point out that
there has been a significant escalation in the control over and
distancing of journalistic mechanisms by press officers in the
US.\textsuperscript{26} Press-political relations are much more
clearly defined by tension and competition with politicians (and among
journalists) than in Germany, where journalists tend to seek collective
group management via their press associations, without one media holding
a dominant position over another.\textsuperscript{27} The strong
presidentialization of the American and French political systems, as
well as the economic competition for news, reinforces the phenomenon of
personalization and the game-frame in the journalistic coverage of
politics, in contrast to German parliamentarism.\textsuperscript{28} Japan seems to be
the polar opposite of this press-political relationship, with absolute
control over sources and freedom of the press seemingly not the goal,
according to a comparative study of Germany and Japan.\textsuperscript{29} Press clubs
(\emph{kisha kurabu}) operate formally in the same way as press
galleries in the UK or Australia.\textsuperscript{30} But
the combined effects of a one-party government in power for nearly sixty
years, the very closed functioning of elite circles in Japan---where
press, industry, and political leaders go to similar private schools and
clubs---and the extreme concentration of the media structure exchanges
in a way that is not favorable to journalists.\textsuperscript{31}

\newpage The\marginnote{Matthias
  Weiß, ``Journalisten: Worte als Taten,'' in \emph{Karrieren im
  Zwielicht: Hitlers Eliten nach 1945}, ed. Norbert Frei (Frankfurt:
  Campus Verlag, 2001).} situation\marginnote{\textsuperscript{21} Michel Stankovitch, ``Les
  Services de presse des gouvernements et de la S.D.N'' (PhD diss.,
  Université de Paris, 1939).} most\marginnote{\textsuperscript{22} Jeremy Tunstall, \emph{The Westminster Lobby
  Correspondents: Sociological Study of National Political Journalism}
  (London: Routledge \& Kegan Paul, 1970).} akin\marginnote{\textsuperscript{23} Aeron Davis, \emph{Public
  Relations Democracy: Public Relations, Politics and the Mass Media in
  Britain} (Manchester: Manchester University Press, 2002).} to\marginnote{\textsuperscript{24} Ralph Bläser, ``Ménage à trois:
  la pertinence géographique des relations de lobbying entre les
  ONG-Bankwatch, l'État national et la Banque mondiale à Washington
  D.C.,'' \emph{L\textquotesingle espace politique}, no. 1 (2007).} that\marginnote{\textsuperscript{25} Timothy Crouse,
  \emph{The Boys on the Bus} (New York: Random House, 2003); Kendall
  Hoyt and Frances S. Leighton, \emph{Drunk Before Noon: The
  Behind-the-Scenes Story of the Washington Press Corps} (Englewood
  Cliffs, NJ: Prentice Hall, 1979).} in\marginnote{\textsuperscript{26} Anderson C. W., Leonard Downie, and Michael Schudson,
  \emph{The News Media: What Everyone Needs to Know} (Oxford: Oxford
  University Press, 2016), 116.} Japan\marginnote{\textsuperscript{27} Revers, \emph{Contemporary
  Journalism in the US and Germany,} 189.} is\marginnote{\textsuperscript{28} Frank Esser,
  Carsten Reinemann, and David Fan, ``Spin-doctoring in British and
  German Election Campaigns,'' \emph{European Journal of Communication}
  15, no. 2 (2000); Peter Van Aelst et al., ``Personalization,'' in
  \emph{Comparing Political Journalism}, ed. Claes de~Vreese, Frank
  Esser, and David Hopmann (London: Routledge, 2017).} probably\marginnote{\textsuperscript{29} Takashi
  Jitsuhara, ``Guarantee of the Right to Freedom of Speech in Japan---A
  Comparison with Doctrines in Germany,'' in \emph{Contemporary Issues
  in Human Rights Law,} ed. Yumiko Nakanishi (Singapore: Springer,
  2018); Steven Borowiec, ``Writers of Wrongs: Have Japan's Press Clubs
  Created Overly Cosy Relationships Between Business Leaders and the
  Press?,'' \emph{Index on Censorship} 45, no. 2 (2016).} to\marginnote{\textsuperscript{30} Jane O\textquotesingle Dwyer,
  ``Japanese Kisha Clubs and the Canberra Press Gallery: Siblings or
  Strangers,'' \emph{Asia Pacific Media Educator} 1, no. 16 (2005).} be found in
Brussels.\textsuperscript{32} Journalism has been
institutionalized through the press corps meetings at the European
Commission's Berlaymont building. Relations are founded on the principle
of a peaceful co-production of information. But this system was
established by and for the Commission, and journalists are driven by a
faith in the European federalist project.\textsuperscript{33} The structure
of the European bureaucratic field has made it more difficult to voice
political opposition and to seek competitive information from political
rivals.\textsuperscript{34} Moreover, political
journalists are part of neither a specific national nor a European
journalistic field,\textsuperscript{35} unlike
German parliamentary journalists. Contrary to their colleagues in
Brussels, they do not have to justify their interest in reporting on
government policy.

At this stage, it is both the institutionalization of these exchanges
and their corporative functioning that distinguishes Germany. Hallin and
Mancini have described German journalism as
corporatist-democratic.\textsuperscript{36} Indeed, journalists
prefer a cooperative management of relations with sources to a
competitive struggle between colleagues. The point here is not to say
that journalists are \emph{not} competing for news access, but that they
are \emph{particularly} aware of the value of maintaining the
collective. Everything seems to indicate that actors agree to preserve a
monopoly on and control over the political game, which is usually
regulated by transactions, profit sharing, and a collusive desire to
regulate competition, to limit how and where conflicts are expressed
and, sometimes, to protect the positions of opponents.\textsuperscript{37} However, Hallin and
Mancini's model is only based on indicators. It says nothing about the
sociohistorical roots of the institutions and how they actually operate
in such a model.

The aim of this paper is to show---with these analytical gaps and the
lack of a socio-historical explanation in mind---that the academic
socialization of a French researcher can make heuristic contributions to
research on Germany. The structure of the paper is, first, to briefly
present the two national analytical traditions as reflexive starting
points which I tried to overcome in my research on Germany. The
following section deals more precisely with the framework of my research
on the relations between journalists and politicians. The aim, finally,
is to enumerate some specific ways in which this framework can advance
our understanding of the structure of the national public sphere and to
show this reality in practice---mixing Habermas with the more specific
French uses of Norbert Elias and Pierre Bourdieu. I stress the
advantages of combining French journalism and political sociology
(influenced by a constructivist and historical approach) with the
systemic German media studies approach.

\hypertarget{overcoming-two-national-analytical-traditions}{%
\section{Overcoming\marginnote{\textsuperscript{31} César Castellvi,
  ``Les Clubs de presse au Japon: Le journaliste, l'entreprise et ses
  sources,'' \emph{Sur le journalism} 8, no. 2 (2019); William Nester,
  ``Japan\textquotesingle s Mainstream Press: Freedom to Conform?,''
  \emph{Pacific Affairs} 62, no. 1 (1989); Yamamoto Taketoshi, ``The
  Press Clubs of Japan,'' \emph{Journal of Japanese Studies} 15, no. 2
  (1989).} Two National Analytical
Traditions}\label{overcoming-two-national-analytical-traditions}}

Looking at\marginnote{\textsuperscript{32} Olivier Baisnée, ``Reporting the European Union: A
  Study in Journalistic Boredom,'' in \emph{Political Journalism in
  Transition: Western Europe in a Comparative Perspective}, ed. Raymond
  Kuhn and Rasmus Kleis Nielsen (London: I.B. Taurus \& Co., 2013); Anke
  Offerhaus, \emph{Die Professionalisierung des deutschen
  EU-Journalismus: Expertisierung, Inszenierung und
  Institutionalisierung der europäischen Dimension im deutschen
  Journalismus} (Wiesbaden: Springer VS, 2011).} the\marginnote{\textsuperscript{33} Martin Herzer,
  \emph{The Media, European Integration and the Rise of Euro-Journalism,
  1950s--1970s} (Basingstoke: Palgrave Macmillan, 2019).} national\marginnote{\textsuperscript{34} Didier Georgakakis and Jay Rowell, eds., \emph{The
  Field of Eurocracy: Mapping the EU Staff and Professionals}
  (Basingstoke: Palgrave Macmillan, 2013).} academic\marginnote{\textsuperscript{35} Florian Tixier, ``En quête de
  professionnalisme: L'Association des journalistes européens, des
  spécialistes de l'Europe aux journalistes spécialisés,'' in \emph{Les
  Médiations de l'Europe politique}, ed. Philippe Aldrin et al.
  (Strasbourg: Presses Universitaires de Strasbourg, 2014).} routine,\marginnote{\textsuperscript{36} Daniel Hallin and Paolo Mancini,
  \emph{Comparing Media Systems: Three Models of Media and Politics}
  (Cambridge: Cambridge University Press, 2004).} this\marginnote{\textsuperscript{37}\setcounter{footnote}{37} Frederick
  George Bailey, \emph{Stratagems and Spoils: A Social Anthropology of
  Politics} (Oxford: Westview Press, 2001), 55.} article cannot, of
course, reflect the full diversity of those studies in Germany or in
France, but is rather a reflexive examination of these two academic
communities. It is built on twenty years of comparative research on
journalism and political communication in Germany, a dual PhD in
political science in journalism at Sciences Po Strasbourg and the Freie
Universität Berlin,\footnote{Nicolas Hubé, \emph{Décrocher la ``Une'':
  Le choix des titres de première page de la presse quotidienne en
  France et en Allemagne (1945--2005)} (Strasbourg: Presses
  universitaires de Strasbourg, 2008).} as well as a two-and-a-half-year
experience as a lecturer on these subjects at the Europa Universität
Viadrina (Frankfurt/Oder) from 2013 to 2015. However, it is within this
dialogue between two academic worlds that my ``particular'' approach has
been built. A second factor completes and complexifies the analytical
work: the intersection of political sociology with communication
sciences in France.

\hypertarget{journalism-and-political-studies-in-france-and-germany}{%
\subsection{Journalism and Political Studies in
France and
Germany}\label{journalism-and-political-studies-in-france-and-germany}}

It is precisely at the intersection of these two approaches that I have
attempted to build my research program. A first comparison of German and
French traditions indicates that, while they both focus on questions
concerning the democratic public space, the methods and traditions of
media analysis can hardly be compared.\footnote{Hubé, ``À la recherche
  d'une universalité du journalisme.'' See also Lisa Bolz in this issue.}
As the historical perspective offered in Lisa Bolz's contribution to
this special section illustrates, there is a modelled and tendentially
disincarnated relationship between groups of actors (media, politicians,
and public opinion) in Germany; whereas, in France, there is a more
qualitative (and critical) understanding of the exchanges between these
groups, which tends to overlook the public to focus instead on
journalists, their characteristics, their resources, and their
professional organizations. If these major questions are equivalent to
those of the German \emph{Journalistik}, the latter is distinguished by
a more abstract production, on the one hand seeking to characterize
these transformations in the form of models and systems, and on the
other hand using a more quantitative and internationalized empirical
approach which serves to establish statistical laws. This permanent
quest for a democratic balance between public opinion, politics, and the
media is as much the result of the history of university reconstruction
(and its funding methods) after World War II as it is a strictly
theoretical question. The weight of the Allied presence (especially the
US) in the funding of universities, the intense pressure of the Cold
War, and the anti-communism of the founders made it difficult, if not
impossible, for critical thought to emerge, and it was confined to the
Frankfurt School and a few marginal places after 1968.\footnote{Löblich,
  Venema, and Pollack, ``Critical Communication Studies and the Cold
  War.''} The German research community intends to model this triptych
of public, political professionals, and journalists in the form of a
system.\footnote{Carsten Brosda and Christian Schicha, ``Interaktion von
  Politik, Public Relations und Journalismus,'' in \emph{Politische
  Akteure in der Mediendemokratie: Politiker in den Fesseln der
  Medien?}, ed. Heribert Schatz, Patrick Rössler, and Jörg-Uwe Nieland
  (Wiesbaden: Westdeutscher Verlag, 2002); Barbara Pfetsch,
  \emph{Politische Kommunikationskultur: Politische Sprecher und
  Journalisten in der Bundesrepublik und den USA im Vergleich}
  (Wiesbaden: Westdeutscher Verlag, 2003).} In 2011, Hans-Mathias
Kepplinger stated in his introduction to the ``challenges of journalism
research'' that:

\begin{quote}
Mass communication represents a highly differentiated and highly
interconnected sub-system of the society system, which is highly
distinct from its environment. The other subsystems are politics,
economy, science, etc. . . . A basis {[}for this systemic autonomy{]} is
the recognition by the Constitutional Court that freedom of the press is
constitutive to democracy.\footnote{Kepplinger, \emph{Journalismus als
  Beruf}, 9.}
\end{quote}

More than in France, one of the issues for the \emph{Journalistik} is to
determine which of the three components has the greatest influence over
the other two, and a second is to be able to measure it. These
contributions from a more systemic approach have undoubtedly influenced
my way of working on this subject. And this is all the more true as the
discipline has more systematically comparative, quantitative, and
internationalized approaches than those in France. Conversely, the
sociology of the journalists' work and of the actors in their
organizations is less developed there. In France, works based on a
sociographic approach to journalism abound. The main research on the
profession is carried out through large surveys of press card holders.
Though the contours of German journalism have been regularly observed
since the 1980s, the data are rarely studied from the perspective of
their social attributes, and the notion of habitus derived from Bourdieu
is rather flexible and used to describe professional roles rather than
to articulate social positions and attitudes.\footnote{Thomas Hanitzsch,
  ``Deconstructing Journalism Culture: Toward a Universal Theory,''
  \emph{Communication Theory} 17, no. 4 (2007).} French work highlights
the heterogeneity of the ways in which journalism is practiced, while
the universalist vision of journalism is what is mostly sought by the
\emph{Journalistik.}\footnote{Theirs are newer approaches. See a first
  synthesis here: Stefanie Averbeck-Lietz and Michael Meyen, eds.,
  \emph{Handbuch nicht standardisierte Methoden in der
  Kommunikationswissenschaft} (Wiesbaden: Springer VS, 2016).} It
assumes its methodological openness and its very clear inter- and
transdisciplinary character (between communication science, political
science, and sociology).\footnote{See Lisa Bolz in this issue.}

More generally, French research has adopted a more qualitative approach,
more systematically using interviews or ethnographic observations. The
objective is to understand journalistic production as a collective
action, involving journalists' work with and against their sources (and
to understand the struggles between sources), their colleagues in the
newsrooms, as well as their specialist colleagues. One very important
text for journalism studies in France was the translation of Philipp
Schlesinger's text in the journal \emph{Réseaux} in 1992,\footnote{Philip
  Schlesinger, ``Rethinking the Sociology of Journalism: Source
  Strategies and the Limits of Media-Centrism,'' in \emph{Public
  Communication: The New Imperatives}, ed. Margorie Ferguson (London:
  SAGE Publications, 1990). This text has been translated and published
  in a special issue on ``Journalists' Sociology'' in the journal
  \emph{Réseaux}, no. 51 (1992).} which is still regularly referred to
today to argue for more clearly constructivist and structural approaches
to understand the exchanges between journalists and sources.\textsuperscript{47}
Consequently, from my sociological perspective, the perimeter of
political communication has changed and expanded hand in hand with
contemporary transformations of\marginnote{\textsuperscript{47}\setcounter{footnote}{47} In
  the last decade, three special issues have been published: Clément
  Desrumaux and Jérémie Nollet, eds., ``Le travail politique par et pour
  les médias,'' \emph{Réseaux}, no. 187 (2014); Philippe Juhem and Julie
  Sedel, eds., \emph{Agir par la parole: Porte-paroles et asymétries de
  l'espace public} (Rennes: Presses Universitaires de Rennes, 2016);
  Nicolas Kaciaf and Jérémie Nollet, eds., ``Journalisme: retour aux
  sources,'' \emph{Politiques de communication}, no. 1 (2013).} the political process.\footnote{For a
  synthesis, see Philippe Aldrin and Nicolas Hubé, \emph{Introduction à
  la communication politique} (Bruxelles: De Boeck, 2022).} Observations
of economic or technical changes are made as closely as possible to the
daily work of journalists. In Germany, the observation of editorial
structures is somewhat unusual. Despite an early tradition in newsroom
observation that has been mostly published in German,\footnote{Elke
  Grittman, ``Organisationeller Kontext,'' in \emph{Grundlagentexte zur
  Journalistik,} ed. Irene Neverla, Elke Grittmann, and Monika Pater
  (Konstanz: UVK, 2002); Manfred Rühl, \emph{Die Zeitungsredaktion als
  organisiertes soziales System} (Fribourg: Universitätsverlag Freiburg,
  1979); and Manfred Rühl, ``Organisatorischer Journalismus: Tendenzen
  der Redaktionsforschung,'' in \emph{Grundlagentexte zur Journalistik},
  ed. Irene Neverla, Elke Grittmann, and Monika Pater (Konstanz: UVK,
  2002).} and despite Frank Esser's contributions showing how editorial
organization influences the management of publishing and commercial
choices in German and British newsrooms, this approach has not been
pursued.\footnote{Frank Esser, ``Editorial Structures and Work
  Principles in British and German Newsrooms,'' \emph{European Journal
  of Communication} 13, no. 3 (1998); Brigitte Hofstetter and Philomen
  Schoenhagen, ``When Creative Potentials are Being Undermined By
  Commercial Imperatives,'' \emph{Digital Journalism} 5, no. (2017);
  Brigitte Hofstetter and Philomen Schoenhagen, ``Wandel redaktioneller
  Strukturen und journalistischen Handelns'' {[}Changing newsroom
  structures and journalistic routines{]}, \emph{Studies in
  Communication and Media} 3, no. 2 (2014); Lars Rinsdorf and Laura
  Theiss, ``Leidenschaftliche Amateur*innen oder kühle Profis: Zum
  Integrationspotenzial der freien Mitarbeiter*innen lokaler
  Tageszeitungen,'' in \emph{Integration durch Kommunikation: Jahrbuch
  der Publizistik- und Kommunikationswissenschaft}, ed. Volker Gehrau,
  Anni Waldherr, and Armin Scholl (Münster: Deutsche Gesellschaft für
  Publizistik- und Kommunikationswissenschaft e.V.; Westfälische
  Wilhelms-Universität Münster, Institut für Kommunikationswissenschaft,
  2019).} Moreover, although Germany is described as a model of a
democratic and corporatist country, few studies have focused on its
professional associations' struggles to define the contours of the labor
market, in contrast to France, where many studies have explored the
evolution of practices over a long socio-historical period.\footnote{A
  few of the most significant texts on this approach may be cited here:
  Ivan Chupin, \emph{Les écoles du journalisme: Les enjeux de la
  scolarisation d'une profession}}

A look at the textbooks of both countries provides information on this
difference. In Germany, the systemic approach is often represented
according to the ``onion model.''\textsuperscript{52} The media system is structured in
four layers: media actors (the role context), media statements (the
functional context), media institutions (the structural context), and
the media system itself (the normative context).\textsuperscript{53} When in 2007 researchers
published \emph{Journalism Theory: New Generation}, they did not
challenge these systemic conceptions. The authors sought to explore new
approaches in order to bridge the gap between the micro- and
macro-sociological aspects.\textsuperscript{54} Conversely, in France, Erik Neveu
devotes two chapters of his \emph{Journalism Sociology} to ``the field
of journalism today'' and to ``journalists at work,'' and Rémy Rieffel
devotes a long chapter to ``media professionals'' in his \emph{Media
Sociology.}\textsuperscript{55} More systemic and comparative models
are only rarely present in these works. When they deviate from a
theoretical and/or quantitative approach, German scholars claim to be
using ``non-standardized methods in communication science.''\textsuperscript{56} Ironically, the authors of this 2016
textbook have close academic collaborations with French
scholars.\textsuperscript{57}

\hypertarget{the-sociological-turn-of-french-political-science}{%
\subsection{The Sociological Turn of French
Political
Science}\label{the-sociological-turn-of-french-political-science}}

With regard to political communication, recent French textbooks also
claim to be based on \emph{political communication
sociology.}\textsuperscript{58} At the
beginning of the 2000s, several researchers at the intersection of
sociology, media and communication studies (\emph{sciences de
l'information et de la communication} (SIC)), and political science
advocated a sociology-based approach to these topics in order to
separate research from publications by professionals in
communication.\textsuperscript{59} New journals were
created to complement this approach (\emph{Questions de communication}
in 2002 and \emph{Politiques de communication} in 2013). For these
authors, the goal is to focus on both the dissemination of knowledge and
the generalization of\marginnote{\emph{(1889--2018)} (Rennes: Presses
  universitaires de Rennes, 2018); Nicolas Kaciaf, \emph{Les Pages
  ``Politique'': Histoire du journalisme politique dans la presse
  française (1945--2006)} (Rennes: Presses Universitaires de Rennes,
  2013); Sandrine Lévêque, \emph{Les Journalistes sociaux: Histoire et
  sociologie d'une spécialité journalistique} (Rennes: Presses
  Universitaires de Rennes, 2000); Arnaud Mercier,
  ``L'institutionalisation de la profession des journalistes,''
  \emph{Hermès}, no. 13--14 (1994); Denis Ruellan, \emph{Les ``Pro'' du
  journalisme: De l'état au statut, la construction d'un espace
  professionnel} (Rennes: Presses universitaires de Rennes, 1997).} the\marginnote{\textsuperscript{52} Siegfried Weischenberg,
  \emph{Journalistik: Theorie und Praxis aktueller Medienkommunikation;
  Band 2: Medientechnik, Medienfunktionen, Medienakteure} (Opladen:
  Westdeutscher Verlag, 1995), 69ff.} professional\marginnote{\textsuperscript{53} Bolz,
  ``Recherches sur le journalisme,'' 19.} tools\marginnote{\textsuperscript{54} Klaus-Dieter Altmeppen, Thomas
  Hanitzsch, and Carsten Schlüter, eds., \emph{Journalismustheorie: Next
  Generation: Soziologische Grundlegung und theoretische Innovation}
  (Wiesbaden: VS Verlag, 2007), 12.} related\marginnote{\textsuperscript{55} Érik Neveu, \emph{Sociologie du journalisme}
  (Paris: La Découverte, 2019); Rieffel Rémy, \emph{Sociologie des
  médias} (Paris: Ellipses, 2015).} to\marginnote{\textsuperscript{56} Stefanie
  Averbeck-Lietz and Michael Meyen, eds., \emph{Handbuch nicht
  standardisierte Methoden}.} political\marginnote{\textsuperscript{57} Stefanie Averbeck-Lietz, Fabien Bonnet, and Jacques
  Bonnet, ``Le discours épistémologique des Sciences de l'information et
  de la communication,'' \emph{Revue française des sciences de
  l'information et de la communication}, no. 4 (2014); Thomas Wiedemann
  and Michael Meyen, eds., \emph{Pierre Bourdieu und die
  Kommunikationswissenschaft: Internationale Perspektiven} (Köln:
  Herbert von Halem Verlag, 2013).}
communication,\marginnote{\textsuperscript{58} Aldrin and Hubé; \emph{Introduction à la
  communication politique}; Philippe Riutort, \emph{Sociologie de la
  communication politique} (Paris: La Découverte, 2020).} but\marginnote{\textsuperscript{59}\setcounter{footnote}{59} Eric Darras, ``Division du travail politiste et
  travail politiste de division: L'exemple de la communication,'' in
  \emph{La science politique: Une et multiple}, ed. Eric Darras and
  Olivier Philippe (Paris: L'Harmattan, 2004); Arnaud Mercier, ``La
  communication politique en France: Un champ de recherche qui doit
  encore s'imposer,'' \emph{L'Année sociologique}, no. 51 (2001);
  Philippe Riutort, ``Sociologiser la communication politique?,''
  \emph{Politique et Sociétés} 26, no. 1 (2007).} also---following a processual approach to the
phenomenon, and using the vocabulary of Norbert Elias---on the
complexity of the interdependence chains that link the different
participants in the political process today. This work pays particular
attention to historical contexts, social structures, and
interdependencies between the professional worlds of politics, the
media, and academic research. These elements together condition the
perceptions, the action logics, and the practices of political
communication.

A particular aspect of this research program is its emphasis on a more
explicitly (micro)sociological perspective. This research agenda has
been inspired by major contributions from the history and the sociology
of situations, actors, and institutions to the understandiong of
political processes. Indeed, unlike German political science, which also
tends to focus on the formulation of empirical answers to normative
questions,\textsuperscript{60} French political science underwent a
significant sociological shift from its public law and institutionnalist
perspective at the turn of the 1980s.\textsuperscript{61} Scholars use Max Weber,
Pierre Bourdieu, and Emile Durkheim as much as Norbert Elias, to which
we can add the translation of the text by Peter Berger and Thomas
Luckmann, \emph{The Social Construction of Reality}, in 1986, which was
immediately adopted by a section of political sociology. In 1988,
\emph{Politix: Journal of the Social Sciences of Politics} was launched.
Today it is one of the leading mainstream journals of political science
in France, alongside the \emph{Revue Française de Science
Politique.}\textsuperscript{62}

Politics is defined in the broader sense proposed by Pierre
Bourdieu---that is, as the character of ``any action aimed at
transforming perception categories.''\textsuperscript{63}
These studies stress the professionalization of the political function,
complementing Weber with Bourdieu's analysis.\textsuperscript{64} Political agents live both of and for politics. From a
more Durkheim-inspired perspective, the academic discipline starts from
the presupposition of a social division of political work. In order to
function and thus to calm, if not pacify, social conflict, this power
relationship presupposes the broadest possible acceptance of the
political order. In order to obtain the consent of the governed to this
specific arrangement of relations between members of society, this
research seeks to understand how authority tends to monopolize
``legitimating discourse''\textsuperscript{65} by exclusively claiming the authority to state the
basis of the social order in order to make it appear rational,
desirable, and sacred. Introductory classes in political science are
traditionally dispensed as part of political sociology in law faculties.
The first few hours are devoted to the emergence of the modern state,
based on Marx, Weber, or Elias, while the functioning of institutions is
left to law professors in constitutional law classes. The perspective of
historical sociology\marginnote{\textsuperscript{60} Herfried Münkler and Skadi Krause, ``Geschichte und
  Selbstverständnis der Politikwissenschaft in Deutschland,'' in
  \emph{Politikwissenschaft: Ein Grundkurs}, ed. Herfried Münkler
  (Hamburg: Rowohlt, 2003).} has\marginnote{\textsuperscript{61} The bestselling handbooks
  for students in political sciences and law are: Jacques Lagroye,
  \emph{Sociologie politique}, 6th ed. (Paris: Presses de Sciences Po \&
  Dalloz, 2012), originally published in 1991; Philippe Braud,
  \emph{Sociologie politique}, 14th ed. (Paris: LGDJ, 2020), originally
  published in 1992; and Dormagen Jean-Yves and Daniel Mouchard,
  \emph{Introduction à la sociologie politique}, 5th ed. (Bruxelles: De
  Boeck, 2019), originally published in 2007. The first of these
  authors, Jacques Lagroye, died in 2013.} entered\marginnote{\textsuperscript{62} PhDs in political sociology represent 40 percent of
  political science PhDs published in France each year. Delphine Dulong,
  ``Le champ politique selon Bourdieu,'' \emph{Politika}, June 11, 2020,
  \url{https://www.politika.io/fr/article/champ-politique-bourdieu}.} fully\marginnote{\textsuperscript{63} Pierre Bourdieu,
  \emph{Sociologie générale}, vol. 2, \emph{Cours au collège de France
  1983--1986} (Paris: Raisons d'agir/Seuil, 2016), 150 (my translation).} into\marginnote{\textsuperscript{64} Pierre Bourdieu,
  \emph{Language and Symbolic Power} (Cambridge, MA: Harvard University
  Press, 1991).} the\marginnote{\textsuperscript{65}\setcounter{footnote}{65} Jacques Lagroye, ``La
  légitimation,'' in \emph{Traité de science politique}, vol. 1\emph{,}
  ed. Madeleine Grawitz and Jean Leca (Paris: Presses Universitaires de
  France, 1985).} social sciences of
politics (notably through the journals \emph{Les Annales} and
\emph{Genèses: Sciences sociales et histoire}, founded in
1990),\footnote{See the handbook: Yves Déloye, \emph{Sociologie
  historique du politique} (Paris: La Découverte, 2017).} while in
Germany it has been largely abandoned by sociologists, political
scientists, and media researchers alike, and reserved for
historians.\footnote{George Steinmetz, ``Field Theory and
  Interdisciplinarity: History and Sociology in Germany and France
  during the Twentieth Century,'' \emph{Comparative Studies in Society
  and History} 59, no. 2 (2017).}

These studies acknowledge that the political process and its degree of
acceptance by the governed vary greatly and that the problem of
communication is not addressed in the same way in different social
environments. Contemporary democracies are not just a regime of checks
and balances.\footnote{Bernard Manin, \emph{The Principles of
  Representative Government} (Cambridge: Cambridge University Press,
  1997).} Founded on the avoidance of physical violence and on the ideal
of representative government, these regimes are rooted in three
principles: election and independence of the government; freedom of
opinion and public expression of the governed; and the testing of
political choices through discussion.\footnote{Ivan Chupin, Nicolas
  Hubé, and Nicolas Kaciaf, \emph{Histoire politique et économique des
  médias en France} (Paris: La Découverte, 2012).} Having said that,
participants in the political process cover a very broad social
spectrum, since in principle all persons or groups of persons engaged in
the defense of an interest or cause are---at least legally---in a
position to make their voices heard in the public debate and to
influence public decisions. In this movement of rationalization of
political activities, ``political communication'' can be analyzed as a
continuous process whereby political entrepreneurs are equipped with
cognitive, technical, and instrumental qualities with the aim of
reducing the uncertainty of the conquest and/or the exercise of
power.\footnote{Aldrin and Hubé, \emph{Introduction à la communication
  politique}.} Following Weber, by selecting news, the journalist is
already ``a type of professional politician''\footnote{Max Weber,
  \emph{The} \emph{Vocations Lectures} (Indianapolis/Cambridge: Hackett
  Publishing Company, 2004 {[}1919{]}), 58.} and not a representative of
a power separated from politics.

This implies that one ought not to analyze press-political relations in
terms of degree of (in)dependence, but rather to consider them as being
\emph{interdependent}, and to situate them in the particular social
figuration in which they evolve.\textsuperscript{72} In
itself, this approach is not specifically French, but it is part of a
more general paradigm at the intersection of sociological and historical
institutionalism\textsuperscript{73} which is more common in
France and defended, in particular in the US, by authors with a
Bourdieusian filiation.\textsuperscript{74} As Benson writes, ``the challenge is to bring the same
sophisticated analysis to bear on understanding media as an independent
variable, as part of the process of political meaning making rather than
just a convenient indicator of the outcome.''\textsuperscript{75}

Any research on journalism in France, even if it is German, inevitably
requires considering the object from this ``national'' academic
perspective, as it is also the case from the sociological-political
dimension of the object. But having been integrated for two decades into
the German scientific field, the epistemological approach is inevitably
influenced by its own specific perspectives (for instance, the weight\marginnote{\textsuperscript{72} Michael Schudson, ``Autonomy
  from What?,'' in \emph{Bourdieu and the Journalistic Field}, ed.
  Rodney Benson and Erik Neveu (Cambridge, MA: Polity, 2005).} of\marginnote{\textsuperscript{73} Peter A. Hall and Rosemary C. R. Taylor,
  ``Political Science and the Three New Institutionalisms,''
  \emph{Political Studies} 44, no. 5 (1996).}
Luhmann\marginnote{\textsuperscript{74} For example: Rodney Benson, ``News
  Media as a `Journalistic Field': What Bourdieu Adds to New
  Institutionalism, and Vice Versa,'' \emph{Political Communication} 23,
  no. 2 (2006). See also the contributors to \emph{Bourdieu and the
  Journalistic Field}, ed. Rodney Benson and Erik Neveu (Cambridge, MA:
  Polity, 2005).} and\marginnote{\textsuperscript{75}\setcounter{footnote}{75} Rodney Benson,
  ``Bringing the Sociology Back In,'' \emph{Political Communication} 21,
  no. 3 (2004): 276.} Habermas in the field of communication studies).

\hypertarget{journalists-and-politicians-as-associate-rivals}{%
\section{Journalists and Politicians as
Associate-Rivals}\label{journalists-and-politicians-as-associate-rivals}}

Thanks to these academic traditions, I observed the long process of
institutionalization of the rules of information exchange between the
two groups. This paper refers to my research based on a series of
fifty-one interviews held with journalists who are active or retired
members of the BPK, politicians (including Bundestag members,
parliamentary group president, and/or ministers), and communication
staff and spokespeople for parties and ministries between 2003 and 2015.
In addition to these interviews, it is informed by my observations of
\emph{Bundespressekonferenz} meetings in 2010 and 2015 and of the
communication staff of various parties during the 2017 federal election
campaign.\footnote{Hubé, \emph{La politique des chemins courts.}} Its
analysis also draws on the archives of the \emph{Bundespresseamt}
(Bundesarchiv Koblenz) 1949--1985, the \emph{Vereinigte Presseabteilung
der Reichsregierung} (Bundesarchiv Berlin) 1918--1933, the
\emph{Bundespressekonferenz} (Berlin) 1949--1985, the \emph{Verein der
ausländischen Presse} 1970--1975, and the administrative documents filed
in the documentation department of the \emph{Bundespresseamt}. The 1985
cut-off date was the maximum access limit for open archives according to
the thirty-year limit at the time of the investigation and according to
the classification of these data by the institutions (BPA and BPK). Some
more recent documents could be consulted, when they had not been sent to
the archives, and the interviews largely made up for this period.

In order to investigate the work done by the journalists of the BPK and
the Government Communication Service (BPA), I established a \emph{first
theoretical principle}. In Berlin, like everywhere else, professional
politicians and journalists need to meet and spend time with each other.
This type of interdependence, where one needs the other to exchange
information for publicity, is consubstantial with a social space of
practices that is strongly heteronomous, and that is commonly referred
to as the public sphere.\footnote{Schudson, ``Autonomy from What?''} One
of the central issues for politicians involves access to the market of
symbolic political goods over which they compete with journalists. The
means of access to this public sphere are structured very differently
depending on the social and political configuration. These factors have
been mentioned by Hallin and Mancini. But one must also consider more
rarely studied factors such as external constraints (military
occupations, peace negotiations, wars) and internal contestation
(revolutionary strikes, attempted putsches, demonstrations, terrorism,
etc.) which have weighed heavily on German national politics since 1918.
In other words, the approach that I am proposing takes us out of a
widespread media-centrism to focus more particularly on the coproduction
of political discourses.\footnote{Schlesinger, ``Rethinking the
  Sociology of Journalism.''}

\hypertarget{understanding-the-co-production-of-symbolic-goods}{%
\subsection{Understanding the Co-production of
Symbolic
Goods}\label{understanding-the-co-production-of-symbolic-goods}}

Researchers usually refer to these exchanges using a dance metaphor.
These scholars are interested above all in the strategic dimension of
these relations, seeking to understand the interplay of
influence,\footnote{Jesper Strömbäck and Lars W. Nord, ``Do Politicians
  Lead The Tango: A Study of the Relationship between Swedish
  Journalists and Their Political Sources in the Context of Election
  Campaigns,'' \emph{European Journal of Communication} 21, no. 2
  (2006).} the strategic calculations,\footnote{Marc-François Bernier,
  \emph{Les fantômes du Parlement: L'utilité des sources anonymes chez
  les courriéristes parlementaires} (Sainte-Foy: Presses de l'Université
  Laval, 2000); Jonathan Cohen, Yariv Tsafti, and Tamir Sheafer, ``The
  Influence of Presumed Media Influence in Politics: Do Politicians'
  Perceptions of Media Power Matter?,'' \emph{Public Opinion Quaterly}
  72, no. 2 (2008).} as well as the representations of one group by the
other.\footnote{Isabelle Borucki, \emph{Regieren mit Medien:
  Auswirkungen der Medialisierung auf die Regierungskommunikation der
  Bundesregierung von 1982--2010} (Opladen: Barbara Budrich Verlag,
  2014); Jean Charron, \emph{La production de l'actualité: Une analyse
  stratégique des relations entre la presse parlementaire et les
  autorités politiques au Québec} (Québec: Boréal, 1994); Karen Ross,
  ``Danse Macabre: Politicians, Journalists, and the Complicated Rumba
  of Relationships,'' \emph{International Journal of Press/Politics} 15,
  no. 3 (2010).} They observe these interactions in order to understand
the process by which journalists distance themselves from their sources,
and the framework of these interactions. These investigations focus on
various places in France and in English-speaking countries: the
corridors of Parliament,\footnote{Tunstall, \emph{The Westminster Lobby
  Correspondents}; Bernier, \emph{Les fantômes du Parlement}; Burgert,
  \emph{Politisch-mediale Beziehungsgeflechte;} Charron, \emph{La
  production de l'actualité}; Revers, \emph{Contemporary Journalism in
  the US and Germany}; Sharon Dunwoody and Steven Shields, ``Accounting
  for Patterns of Selection of Topics in Statehouse Reporting,''
  \emph{Journalism Quaterly} 63 (1986).} the backstage of
political\footnote{Crouse, \emph{The Boys on the Bus}; Nicolas Kaciaf,
  ``Des dissidences aux confidences: Ce que couvrir un parti veut
  dire,'' \emph{L\textquotesingle informel pour informer: Les
  journalistes et leurs sources}, ed. Jean-Baptiste Legavre (Paris:
  L\textquotesingle Harmattan/Pepper, 2014).} or military
campaigns,\textsuperscript{84} or the rooms
of European summits.\textsuperscript{85} They show that interactions
between journalists and politicians are neither random nor driven solely
by professional ethics. They are subject to permanent negotiations at
the limit of what is sayable, feasible, and newsworthy. But these
studies focus on these relationships from a solely strategic and
generally ahistorical perspective.

My first theoretical choice was to postulate that journalists and
politicians are associate-rivals, contributing together to the symbolic
production of politics. Following a more interactionist analysis, this
oxymoron has the advantage of naming and explaining the ``types of
non--zero sum social games, made of intertwining and entanglements'' in
which journalists and politicians get involved. They are in a
``competitive-cooperative'' relationship, typical of certain power
relations.\textsuperscript{86} As associates, they contribute to the
visibility of politics, its rules and frameworks, as well as the issues
that structure the political game. As rivals, they follow divergent
interests and expectations: for politicians, the desire to ensure high
visibility, a positive image, an advantageous framing; and for
journalists, the desire to keep their independence in this framing,
their critical sense, up to the possibility of carrying out
investigations. But this relationship is not only made up of
interactions and strategic coups, it is structured by different
factors:\textsuperscript{87} a multipolar
configuration (competition between political actors, between departments
within an editorial office, between media and types of media, etc.)
framed by different watchdog audiences (press council---the German
\emph{Presserat}, ethics council, legal bodies, etc.) that can ``to a
greater or lesser extent impose compliance with `obligations,'
particularly moral ones, and, in so doing, stretch their
relationships.''\textsuperscript{88}

\newpage During\marginnote{\textsuperscript{84} Emmanuelle Gatien, ``\,`Un peu comme la pluie': La
  co-production relative de la valeur d'information en temps de
  guerre,'' \emph{Réseaux}, no. 157--58 (2009); Thomas Hanitzsch,
  ``Kriegskorrespondenten entmystifizieren,'' in
  \emph{Kriegskorrespondenten}, ed. Barbara Korte and Horst Tonn
  (Wiesbaden: VS Verlag für Sozialwissenschaften, 2007).} these\marginnote{\textsuperscript{85} Florian Tixier, ``Concurrences et
  coopérations pour la production de l'information européenne,''
  \emph{Sur le journalism} 8, no. 1 (2019).} exchanges,\marginnote{\textsuperscript{86} Jean-Baptiste Legavre, ``Entre conflit et
  coopération: Les journalistes et les communicants comme
  `associés-rivaux,'\,'' \emph{Communication \& langages}, no. 169
  (2011): 106. See also Revers, \emph{Contemporary Journalism in the US
  and Germany}, 113--52.} the\marginnote{\textsuperscript{87} Desrumaux and Nollet, ``Le travail politique par et
  pour les médias''; Juhem and Sedel, \emph{Agir par la parole}; Kaciaf
  and Nollet, ``Journalisme: retour aux sources.''} role\marginnote{\textsuperscript{88}\setcounter{footnote}{88} Legavre, ``Entre conflit et cooperation,''
  123.} of all public institutions, companies,
political parties, associations, and trade unions with respect to the
media is, on the one hand, to capture the media's attention, to promote
their worldviews, to share positive representations of their ``values,''
and, on the other hand, to define the limits of the visible and the
invisible and to prevent the dissemination of information that might
contradict official messages. The permanent tension surrounding these
exchanges owes much to public debates and to the way these debates are
shaped by an ever-evolving definition of what is acceptable. The work
carried out by sources and journalists is analyzed by this paper in
equal parts.

\hypertarget{institutionalizing-press-political-relations-codifying-the-off-the-record}{%
\subsection{Institutionalizing Press-Political
Relations: Codifying the\\\noident
Off-the-Record}\label{institutionalizing-press-political-relations-codifying-the-off-the-record}}

Looking at these relations and, especially, the off-the-record rules,
Germany is again somewhat particular. The highly formalized separation
between confidential information and official information is based on
the institutionalization and codification of the procedures of exchange.
From an international and comparative perspective, the most surprising
thing is that these procedures are respected to such an extent. The
question that arises from all of this is the following: How can we
explain that the collective benefits of calculability and predictability
linked to codification ultimately prevail without discussion over the
individual interests of journalists and political actors to break the
rules?\footnote{Pierre Bourdieu, ``Codification,'' in \emph{In Other
  Words: Essays Toward a Reflexive Sociology} (Stanford: Stanford
  University Press, 1990).} If individual journalists have various ways
of accessing information, here the question focuses upon those
collective moments of transmitting such confidential material (BPK
meetings, dinners, etc.). Indeed, from a rational point of view, it may
be in their interest to publish information in the name of citizens'
right to this information, of economic competition between newspapers,
or of the quest for mediatization of politicians. But they don't do it.
These relationships are institutionalized through the enactment of a set
of rules, the complexity of which is based on the fact that their
practical principles are embedded in moral principles.\footnote{Bailey,
  \emph{Stratagems and Spoils,} 16.} It is, as we have said, a set of
meetings and exchanges of information and conversations (more or less
formal) in a non-public context, but which imposes a moral constraint on
its users: to maintain in secrecy a practice that is potentially
suspected of complicity and regularly denounced.

The codification process works in two ways.\footnote{Jean-Baptiste
  Legavre, ``\emph{Off the record}: Mode d\textquotesingle emploi
  d\textquotesingle un instrument de coordination,'' \emph{Politix}, no.
  19 (1992).} First, the definition of public arenas: Each place has its
type of interaction, type of information provided (public vs.
confidential arenas), and roles played by the protagonists. Journalists
and politicians or PR people emphasize the contractual and procedural
dimension of these interactions. The second way of regulating this
social space involves a codification of the publication of press
releases. \emph{Unter drei} corresponds to situations where confidences
may not be published under any circumstances. It is numbered three
because it comes after two other regulatory mechanisms laid down in the
BPK's bylaws, all subject to sanction by means of exclusion from the
group.

\begin{quote}
§ 16 (1) Information is given at press conferences: \emph{unter 1}
{[}under 1: on-record{]}, for general use; or \emph{unter 2} {[}under 2:
off-record{]}, for use with no mention of source or name of informant;
or \emph{unter 3} {[}under 3: background{]}, confidential.

(2) Informants may state how their information is to be handled.
Association members and press conference participants are bound by this
classification of the information. If no statement is made, the material
is considered to be for general use. Any breach of these rules
concerning the classification of information may lead to exclusion from
the association or withdrawal of accreditation as permanent
guest.\footnote{Extract from the BPK's bylaws, our translation.
  ``Satzung'' {[}Bylaws{]}, Bundespressekonferenz (website),
  \url{https://www.bundespressekonferenz.de/verein/satzung}.}
\end{quote}

\enlargethispage{\baselineskip}

The recurring question for a journalist, then, is if and when
information can be divulged. In French media, off-the-record information
is rarely \emph{unter drei}: It is given by a politician in the
hope---or at least the knowledge---that journalists will release it,
while the politician simultaneously seeks to cover themselves so that,
if a controversy blows up, they can say that it didn't come from them.
An impossible task, of course. Most of the time the breach of the
off-the-record rule is tacit, since the actors know---i.e., have
internalized---the boundaries of what is possible, and also know each
other. But in order to be certain, journalists often get together after
off-the-record conversations to agree on what they have heard and
whether it can be released, as long as a colleague seeking a competitive
scoop does not release it first. But in Germany, for a journalist, the
threat of exclusion is a sword of Damocles. What maintains the symbolic
order is the fact that this exclusion is the work of journalists alone.
That is how BPK journalists protect the procedure and maintain a refined
system of inclusion and exclusion, selecting new entrants and excluding
undesirables---i.e., those who do not follow the rules. Within archives
and internal notes, I have noticed that the code has nearly always been
complied with. Breaches of confidence in these circumstances have
consequences for both the culprit and the group. Trust can only be
guaranteed because the organization ensures that these spaces are
relatively independent by bringing ``into play new, impersonal, motives
for action.''\footnote{Niklas Luhmann, \emph{Trust and Power} (New York:
  John Wiley and Sons, 1979), 93. See also Bourdieu, ``Codification.''}
So breaching this trust amounts to unravelling the ``internally
guaranteed security'' provided by the organization that enables
politicians to associate with journalists.\footnote{Niklas Luhmann,
  \emph{Trust and Power}.} In other words, to avoid reintroducing
mistrust, journalists' reactions are constrained by the need to maintain
the proper functioning of the organization.

The culturalist idea of a typically German conformity to the rules is
hardly satisfactory. The explanation for this very strong codification
instead lies in the very limited figuration of parliamentary politics in
the confined space of federal politics, as Norbert Elias noted when
writing about court society.\footnote{Norbert Elias, \emph{The Court
  Society} (Dublin: University College Dublin Press, 2006).} German
journalists do not talk much about politics behind the scenes, and they
are much less able to demonstrate political strategies and tricks than
their counterparts in other countries.\footnote{De~Vreese, Esser, and
  Hopmann, \emph{Comparing Political Journalism.}} In this way, German
journalists can maintain a framework of trust within this federal
area---described as a spaceship---where all is known and observed. This
codification of the inner circle symbolically guarantees the
independence of both groups. Since 1918, journalists and politicians
have sought to build and maintain a stabilized frame of interaction.
During the National Socialist period, Goebbels's \emph{Reichsministerium
für Volksaufklärung und Propaganda} transformed the
\emph{Reichspressekonferenz} into a brutal top-down
institution,\footnote{Fröhlich, ``Joseph Goebbels: Profil de sa
  propagande (1926--1939)''; Mühlenfeld, ``Vom Kommissariat zum
  Ministerium.''} but contributed to this state apparatus continuum by
maintaining bureaucratic offices and an institutional budget and by
delegating leadership to a civil servant.\footnote{Weiß, ``Journalisten:
  Worte als Taten''; Nicolas Hubé, ``L'\emph{Öffentlichkeitsarbeit}~ou
  la propagande au service de la démocratie: La définition du travail de
  communication gouvernementale dans les premières années de la RFA
  (1949--1953),'' \emph{Histoire, économie \& société} 41, no. 4 (2022).}

Based on what Goffman called \emph{keying}---i.e., the transformation of
a set of conventions of a given activity (here, governmental press
conferences) into something patterned\footnote{Erving Goffman,
  \emph{Frame Analysis: An Essay on the Organization of Experience}
  (Boston: Northeastern University Press, 1986), 43--44.}---this
interactional frame has always had to demonstrate its filiation to
democratic theory (in particular, respect for the division of powers) in
order to reach its goal of legitimizing politics and political
journalism. This codification of exchanges was, at the same time, the
guarantee of a pacification of exchanges between the two groups with
their intense political relations, notably during the Weimar Republic.
All the efforts by journalists' professional associations were focused
on gaining the necessary latitude to regulate these interactions solely
under the control of journalists. The symbolic tour de force of the
\emph{Bundespressekonferenz} in 1949 succeeded in building a corporatist
monopoly for managing press-political relations.

\hypertarget{understanding-the-structuring-of-the-national-public-sphere-in-practice}{%
\section{Understanding the Structuring of the National Public Sphere\\\noindent
in
Practice}\label{understanding-the-structuring-of-the-national-public-sphere-in-practice}}

By choosing to observe interactions from the perspective of their
long-term institutionalization, the focus shifts. Journalistic and
political actors do not, simply through strategic calculation, have a
sufficient structuring force that is independent of the socio-political
context. This would be tantamount to arguing that, over the full period,
each actor perceives the separation of powers and democracy issues in
exactly the same manner. Thus, my second theoretical choice: finding the
sociogenesis of these exchanges and understanding the place occupied by
the instruments designed to control the expression of opinions.

\hypertarget{habermas-revisited-the-invention-of-the-uxf6ffentlichkeitsarbeit}{%
\subsection{Habermas Revisited: The Invention
of the
Öffentlichkeitsarbeit}\label{habermas-revisited-the-invention-of-the-uxf6ffentlichkeitsarbeit}}

Understanding the legitimization of politics requires analyzing and
grasping the structural transformations of the national public sphere.
Both the state and journalism are institutions that have been socially
constructed as a result of a long-term process.\footnote{Michael
  Schudson, ``The `Public Sphere' and Its Problems: Bringing the State
  (Back) In,'' \emph{Notre Dame Journal of Law, Ethics \& Public Policy}
  8, no. 2 (1994).} The democratic issue is at stake in the structuring
of this public sphere, in order to defend it against external or
internal political threats, and to convince citizens and journalists of
its validity. I chose to observe these relations in the realm of
historical bifurcations in order to clarify their genetic constitution,
to think about the paths taken---in other words, to understand what the
structuring of these exchanges would have been like or against what
backdrop they were built.\footnote{Pierre Bourdieu, \emph{On the State:
  Lectures at the Collège de France 1989--1992} (Cambridge, MA: Polity
  Press, 2015).} In line with this constructivist approach, no
distinction was made a priori between the practices of ``propaganda,''
``\emph{Öffentlichkeitsarbeit},'' public communication, and public
relations.\footnote{Caroline Ollivier-Yaniv, ``De l'opposition entre
  `propagande' et `communication publique' à la définition de la
  politique du discours,'' \emph{Quaderni}, no. 72 (2010).} It is
obvious from our archives that these labels differ less in essence than
in how they (de-)legitimize a practice.

Rather than seeking the normative and dialogical dimensions of the
democratic structure, this paper attempts to place this public sphere in
its material and historical configuration. In 1962, Habermas showed the
contemporary degradation of the public sphere by stressing the erosion
of the critical capacity of citizens, the commercialization of
information, the double rise of the state and of the great educational
and economic bureaucracies which threatened the private sphere and
perverted the original principles of a dialogical public sphere
undergoing the colonization of the lived world by the mass
media.\footnote{Jürgen Habermas, \emph{The} \emph{Structural
  Transformation of the Public Sphere} (Cambridge, MA: MIT Press, 1989).}
This sphere was structured by political actors and a public of
practitioners (journalists, associations, interest groups) and citizens
who challenged governmental work. But Habermas's thoughts are all about
the affirmation of a ``re-feudalization of the public
sphere,''\footnote{Jürgen Habermas, \emph{Structural Transformation of
  the Public Sphere}, 195.} subordinated to the influence of propaganda
and commercial advertising. This public sphere (\emph{Öffentlichkeit})
was affected by its historical formation after a lengthy process of
imposition of worldviews and material investments. This work on the
public sphere---\emph{Öffentlichkeitsarbeit} in German---was the
concrete action of the actors engaged around Adenauer within the state
apparatus. Habermas's critical and normative expression must be
understood in this critical context, shared by many intellectuals in the
early days of the FRG,\footnote{See the work of one of the founders of
  post-war German political science, Ralf Dahrendorf, \emph{Gesellschaft
  und Demokratie in Deutschland} (Munich: Piper, 1965).} whereby German
democracy \emph{à la} Adenauer was a form of ``enlightened absolutism''
in which journalists should have taken a more critical stance, looking
for more factual and investigative information. The fear was that this
\emph{Öffentlichkeitsarbeit} was merely a continuation of Goebbels-style
propaganda in a new guise of democratic respectability. Political
parties led the first campaigns showing strong distance from the
National Socialist regime, which was omnipresent and
cumbersome.\footnote{Thomas Mergel, \emph{Propaganda nach Hitler: Eine
  Kulturgeschichte des Wahlkampfs in der Bundesrepublik 1949--1990}
  (Göttingen: Wallstein Verlag, 2010).} The governments all had in mind
the ``necessity'' to act on public opinion. This was part of the
organizational continuity of the state apparatus. Thus, successive
governments in 1953, in 1968, and in 1977 intended to create a new
Ministry of Information and Communication. But political and
journalistic opposition rendered this impossible, or even
taboo.\footnote{Hubé, ``L'\emph{Öffentlichkeitsarbeit}~ou la propagande
  au service de la démocratie.''}

In this respect, although no new ministry was created to control public
opinion, democratic Germany is a rather unique case because of the
(relative) persistence of its structures and its state apparatus in
charge of the legitimization of power. In contrast to France,\footnote{Didier
  Georgakakis\emph{, La République contre la propagande: aux origines
  perdues de la communication d'État en France (1917--1940)} (Paris:
  Economica, 2004)\emph{;} Caroline Ollivier-Yaniv, \emph{L'État
  communicant} (Paris: Presses Universitaires de France, 2000).} neither
parliamentarianism nor authoritarian propaganda seems to have been an
obstacle to the institutionalization of state thinking and a state
apparatus in charge of the enactment of symbolic goods. Journalists, on
the other hand, established themselves both as a competing group,
identically claiming a monopoly on the diffusion of symbolic goods, and
as an associate in the defense of a form of ``State Reason''
guaranteeing the freedom of the press. Nothing in that period, however,
could allow one to predict the result of this competitive
struggle---namely: which group would succeed in imposing its definition
of the situation, or which group would contribute its own sense of
social stability.\footnote{George Steinmetz, ``Bourdieusian Field Theory
  and the Reorientation of Historical Sociology,'' in \emph{The Oxford
  Handbook of Pierre Bourdieu}, ed. Thomas Medvetz and Jeffrey Sallaz
  (Oxford: Oxford University Press, 2018), 612.}

After the Empire, the first German democratic experiment took place in
the context of a ``rudimentary State,'' according to Elias,\footnote{Norbert
  Elias, \emph{The Germans: Power Struggles and the Development of
  Habitus in the Nineteenth and Twentieth Centuries} (New York: Columbia
  University Press, 1996), 218.} with a fragmented public sphere that
was highly competitive and divided, and where journalists contributed to
these very high political tensions.\footnote{Fulda, \emph{Press and
  Politics.}} In this configuration, a ``thought of the State'' (as
Bourdieu would say\footnote{Pierre Bourdieu, ``Rethinking the State:
  Genesis and Structure of the Bureaucratic Field,'' \emph{Sociological
  Theory 12}, no. 1 (1994).}) emerged around two emergencies for
legitimizing the new regime: on the one hand, the need to persuade the
citizens, and on the other hand, the necessary enrollment of spokesmen
for this opinion (namely, the journalists). Achieving this enterprise
meant registering these relations in a network of strongly pacified
interactions. This idea of the ``threat'' to or ``defense'' of the
regime was strongly anchored in the generation of political and
journalistic agents active between 1920 and 1950, and lent sociological
sense to the continuities that were observed between 1918 and
1949.\footnote{Eckhard Jesse, \emph{Systemwechsel in Deutschland:
  1918/19---1933---1945/49---1989/90} (Bonn: Bundeszentrale für
  politische Bildung, 2013).} In this period, the context of revolution
and then totalitarianism, as well as the failure of the first democratic
experience of Weimar, were both constraints and conditions of success
for the establishment of renewed forms of press-political relations
after 1949. The experience of a totalitarian and brutal practice from
1933 to 1945 made the establishment of a dedicated ministry symbolically
unthinkable ex-post, or at least gave journalists moral arguments to
declare these projects immediately illegitimate.

\hypertarget{courtization-of-individuals-in-a-federal-capital}{%
\subsection{Courtization of Individuals in a
Federal
Capital}\label{courtization-of-individuals-in-a-federal-capital}}

This paper proposes to reconsider the concrete mode of functioning of
this public sphere, at the mesosociological level of
organizations.\footnote{Benson, ``Bringing the Sociology Back In,'' 280.}
It is possible to understand these relations not primarily from the
point of view of the democratic theory of the separation of powers but
rather as a social construction of actors objectively interacting and
subjectively engaged in these relations. In particular, it is a matter
of analyzing the social mechanisms by which such a figuration produces
what Elias calls a \emph{courtization} of the agents in
competition\footnote{Elias, \emph{The Court Society.}}---in other words,
more or less pacified interactions between groups that are competing but
constrained by their co-presence. Press-political exchanges do not take
place in a theoretical public sphere but perform this public sphere
``whose places, spaces, forms, scenes and moments must be analyzed
respectively.''\footnote{Patrick Boucheron and Nicolas Offenstadt, ``Une
  histoire de l'échange politique,'' in \emph{L'espace public au
  Moyen-Âge: Débats autour de Jürgen Habermas}, ed. Patrick Boucheron
  and Nicolas Offenstadt (Paris: Presses Universitaires de France,
  2011), 17.} Power is as much staged as it is embodied by men and women
in interaction and who are interdependent.

Moreover, the structuring of (political) activity ``has to be explicated
in terms of its spatiality as well as its temporality,'' experienced by
actors themselves.\footnote{Anthony Giddens, \emph{The Constitution of
  Society: Outline of the Theory of Structuration} (Berkeley: University
  of California Press, 1986), 118.} This approach was thus also
nourished by the recent findings from the so-called ``spatial turn'' of
the social sciences in order to shed light on the historical
articulations of a world of relations, which also occurred in a specific
territory that could be objectified.\footnote{Doris Bachmann-Medick,
  \emph{Cultural Turns: Neuorientierungen in den Kulturwissenschaften}
  (Reinbek: Rowohlt, 2018).} Spatiality is not only an outcome; it is
also a part of the explanation.\footnote{Doreen Massey, ``Introduction:
  Geography Matters,'' in \emph{Geography Matters}, ed. Doreen Massey
  and John Allen (Cambridge: Cambridge University Press, 1984), 4.} In
contrast to a state-national construction, like those of Paris and
London, centralized around a royal court and a state nobility in a
geographical space that concentrates both the political-administrative
elites and the economic power, ``Berlin is a young city.''\footnote{Elias,
  \emph{The Germans}, 9.} In our case, one of the specificities is that,
in the course of its history, both groups---journalists and
politicians---sought to settle in a dedicated place at the heart of
power. This location in the capital is more than anecdotal because it
changes the interaction settings every time. In concrete terms, the
question arose as to whether political and parliamentary journalists
should move into a common building (or not), and whether they should
procure it or build it for the \emph{Pressehaus} under Weimar or the
BPK. This was not self-evident. It required negotiations to obtain funds
or land to set up the journalists' group as close as possible to the
political institutions. Moreover, only those political journalists
covering federal politics from the federal seat can be members of the
BPK. This material dimension gave a different meaning to the corporatist
system by installing it spatially. Here again, these socio-geographical
figurations and the presence of journalists and politicians in precise
places were not spontaneous or ``natural''; they were the object of a
construction in the proper sense of the term: that of a desired and
physically identifiable proximity.

If the Weimar figuration was characterized by an extreme polarization of
press titles,\footnote{Fulda, \emph{Press and Politics,} 223.} the
strategy of the government was to create what we will call the
conditions for pacification by coalescence. The rulers of 1918 (mainly)
and the political agents after 1949 had only one idea: to create a dense
network of interdependent relationships where rapid access to each other
would allow the construction of what the actors themselves would end up
calling a \emph{politics of the short paths,}\footnote{Ernst Ney,
  President of the Bundespressekonferenz, to Dr. Hans Daniels, City
  Mayor of Bonn, 26 July 1978, Vorstand {[}Board of Directors{]}
  14.02.1978 bis 12.02.1979, Bundespressekonferenz Archives, Berlin.}
the only one capable of pacifying these exchanges, involving proximity
and permanent exchanges. The embedding of the interactions within
particular institutions---i.e., the \emph{courtization} of the actors of
this parliamentary democracy---allowed the preservation of a regulated
and disciplined game, in spite of everything.\footnote{Elias, \emph{The
  Germans,} 288--97.} It was also necessary to provide these
representation professionals with distinctive places---the government
quarter (\emph{Regierungsviertel}) and its multiple reserved venues or
moments---where this parliamentary etiquette could be practiced and
where journalists and parliamentarians could rub shoulders. This is made
analytically possible by the cross-referencing and discussion of Elias's
and Bourdieu's theories in French political sociology, notably used by
French media scholars like Erik Neveu.

After 1945, with the Allied presence on German soil and the fear of
international reprimands, the idea gradually took hold that all the
actors in these institutions (including the opposition and journalists)
had the same sense of ``responsibilities'' and the same respect for the
``democratic frames'' fixed by the Fundamental Law and guaranteed by the
Karlsruhe Court. This demonstration was achieved, on the one hand,
through the ritualization of press conferences within the BPK, and on
the other hand, through the codification of the framework of exchanges,
which closed the border between the private world of informal relations
and the public world of television interviews, for example.

\hypertarget{revisiting-field-theory-an-interstitial-field-within-the-federal-state}{%
\subsection{Revisiting Field Theory: An
Interstitial Field Within the Federal
State}\label{revisiting-field-theory-an-interstitial-field-within-the-federal-state}}

Working on these figuration changes also required a reconsideration of
the generality of field theory.\footnote{Gisèle Sapiro, ``Le champ
  est-il national? La théorie de la différenciation sociale au prisme de
  l\textquotesingle histoire globale,'' \emph{Actes de la recherche en
  sciences sociales}, no. 200 (2013): 85.} As mentioned, the
sociogenesis of press-political relations in Germany led to the
construction of an autonomous space within the field of federal power,
that of the production of symbolic governmental goods for the general
public. It is indeed a field---a structured, relatively autonomous
space---of objective relations, which cannot be reduced to the
interactions between social agents competing for the definition of a
situation and constrained by ``the mediation of the representation that
people have of the structure.''\footnote{Pierre Bourdieu,
  \emph{Sociologie générale,} vol. 1, \emph{Cours au collège de France
  1981--1983} (Paris: Raisons d'agir/Seuil, 2015), 558.} This field is
based on the durable constitution of a state sector (and its apparatus)
dedicated to the production of symbolic goods for the public. It is the
object of a permanent struggle for the delimitation of the legitimate
actors who can participate in it, as well as the ways of acting and
speaking within this space designed for positions and stances. Like
parliamentarianism, it is a space for the expression of a certain
\emph{consensus}, or rather for the disciplining of the legitimate
expression of \emph{dissensus.}\footnote{De~Vreese, Esser, and Hopmann,
  \emph{Comparing Political Journalism.}} The historical originality of
this construction rests on a corporatist-democratic system, that is to
say, concretely, on a monopolization by the journalists' organizations
in charge of the coverage of parliamentary politics (\emph{Verein
Berliner Presse, Reichsverband der deutschen Presse,} and
\emph{Bundespressekonferenz}) of the expression and transmission of
governmental information, protecting itself against competitive
struggles between editorial offices.\footnote{Revers, \emph{Contemporary
  Journalism.}} This forces the state apparatus all the more strongly to
organize a relatively unified production of government statements in
return. It is these ``relational and dynamic properties, in their own
historicity and temporality,''\footnote{Sapiro, ``Le champ est-il
  national?,'' 85.} that lead us to speak of the field of governmental
symbolic politics. This term is more appropriate to that of the
political/media field, which is too imprecise to describe the effects of
circulation of political statements on the media space and the relations
that structure them. These relations between the spaces and the effects
of intersectoral conversion have to be studied.

However, this field is neither independent nor autonomous. In many ways,
it is a field that can be described as weak or interstitial.\footnote{Stephanie
  Mudge and Antoine Vauchez, ``Building Europe on a Weak Field: Law,
  Economics, and Scholarly Avatars in Transnational Politics,''
  \emph{American Journal of Sociology} 118, no. 2 (2012); Antoine
  Vauchez, ``Interstitial Power in Fields of Limited Statehood:
  Introducing a `Weak Field' Approach to the Study of Transnational
  Settings,'' \emph{International Political Sociology} 5, no. 3 (2011);
  Thomas Medvetz, ``Les think tanks dans le champ du pouvoir
  étasunien,'' \emph{Actes de la recherche en sciences sociales}, no.
  200 (2013).} This term describes an empirical reality and inscribes
these relationships in a larger social structure. It is a ``systemic''
concept that refers to the way the field of power functions.\footnote{Medvetz,
  ``Les think tanks,'' 55.} The concept of the interstitial field
implies that multiple fields are in competition for the control of these
practices. The weakness of this field lies in this interstitial
position, where the actors are caught up in the logics of action of
their own fields, but where goods, norms, and knowledge are exchanged
and capital and positions are converted from one field to another. After
having carried out prosopographical work on the trajectories of
journalists and government communicators, one can show that the actors
are sufficiently bound by these relatively autonomous rules of the game
to speak of a field, but no profession or professional group is able to
impose its rules and precepts to structure its center of
gravity.\footnote{Vauchez, ``Interstitial Power in Fields,'' 342.}
Parliamentary journalists are very largely dependent upon their
inclusion in the national journalistic field. On the other hand, the
national journalistic field alone does not cover the area of political
journalism, which is practiced from the headquarters of the editorial
offices and is scattered throughout the Federal Republic. The
functioning of parliamentary journalism is constrained by the logic of
the parliamentary game. The work of the spokespersons is also doubly
dependent on the rules of the political and bureaucratic fields and
their interconnection, especially because bureaucratic appointments are
linked to electoral contingencies.

Finally, it allows us to rethink the theory in its institutional
anchorage. The centrality and independence of this field of interaction
is relative, because it owes much to the federal organization of
political institutions and editorial offices. Federalism weakens the
magnetic centrality of this field, because not all political and
journalistic careers take place in Bonn or Berlin, and not all
\emph{cursus honorum} are oriented towards the federal
center.\footnote{Martin Baloge and Nicolas Hubé, ``Coproduire les biens
  politiques: Journalistes et politiques en comparaison dans des
  contextes centralisés et fédéraux,'' \emph{Savoir/Agir}, no. 46
  (2019).} Paradoxically, it is also this federal logic that allows this
field to remain autonomous, by guaranteeing the closure of this field.
The concept allows us to describe the logics in terms of career as well
as the reconversions of capital, resources, and knowledge that can take
place in this field and can then be reinvested in other fields
(journalistic, but especially bureaucratic and political). The concept
gives meaning to the institutionalization of press-political relations.

\hypertarget{conclusion}{%
\section{Conclusion}\label{conclusion}}

Understanding interactions between the press and politics hence requires
an understanding of the space of co-production of political discourse
and its structuring, the interdependence and the rationalization of the
political work oriented towards the media as a relational arrangement
between these two groups of actors (at least). One of the issues at
stake in this pacification of exchanges concerns access to the market of
symbolic political goods over which the political authorities compete
with journalists and over which different state sectors compete with
each other. The tour de force of the state-national constructions of
modern states is to have been able to concentrate the instruments of
legitimization and to develop (or attempt to do so) a state apparatus to
support this process.\footnote{Jacques Lagroye, ``La légitimation,'' in
  \emph{Traité de science politique}, vol. 1, ed. Madeleine Grawitz and
  Jean Leca (Paris: Presses Universitaires de France, 1985).} This
sociology of political communication is part of a historical sociology
of the State, where it is understood that journalists participate in the
field of power.\footnote{Aldrin and Hubé, \emph{Introduction à la
  communication politique}; Schudson, ``The `Public Sphere' and Its
  Problems.''}

This approach is a dominant one in French political communication
studies. Adopting reflexive thinking on my own French media studies
routines, I tried here to present the benefits of both national
approaches. Far from rejecting either the French or the German approach,
this investigation was only made possible by taking advantage of each.
My objective was to propose a sociology of the mediatization of politics
that would combine the comprehensive sociology of journalistic work and
the political sociology of federal (and above all parliamentary and
governmental) power, observed over time, and integrate contributions of
media studies in France and Germany and French political sociology,
largely influenced by a constructivist and historical sociology
approach. The more German perspective of a comparative and more
institutional analysis of politics led to an exploration of the
functioning of democracy and parliamentarianism. The more French
approach of the sociology of journalism has placed journalists and
politicians in a process of interdependency rather than independence.
Finally, the weight of historical political sociology provides a path to
follow the construction and evolution of media systems in relation to
those of political systems.




\section{Bibliography}\label{bibliography}

\begin{hangparas}{.25in}{1} 



Aldrin, Philippe, and Nicolas Hubé. \emph{Introduction à la
communication politique}. Bruxelles: De Boeck, 2022.

Altmeppen, Klaus-Dieter, and Martin Löffelholz. ``Zwischen
Verlautbarungsorgan und `vierter Gewalt': Strukturen, Abhängigkeiten und
Perspektiven des politischen Journalismus.'' In \emph{Politikvermittlung
und Demokratie in der Mediengesellschaft}, edited by Ulrich Sarcinelli,
97--123. Bonn: Bundeszentrale für politische Bildung, 1998.

Altmeppen, Klaus-Dieter, Thomas Hanitzsch, and Carsten Schlüter, eds.
\emph{Journalismustheorie: Next Generation; Soziologische Grundlegung
und theoretische Innovation}. Wiesbaden: VS Verlag, 2007.

Anderson, C. W., Leonard Downie, and Michael Schudson. \emph{The News
Media: What Everyone Needs to Know}. Oxford: Oxford University Press,
2016.

Arnold, Klaus, Christoph Classen, Susanne Kinnebrock, Edgar Lersch, and
Hans-Ulrich Wagner, eds. \emph{Von der Politisierung der Medien zur
Medialisierung des Politischen? Zum Verhältnis von Medien,
Öffentlichkeiten und Politik im 20. Jahrhundert}. Leipzig: Leipziger
Universitätsverlag, 2010.

Averbeck-Lietz, Stefanie, and Meyen Michael, eds. \emph{Handbuch nicht
standardisierte Methoden in der Kommunikationswissenschaft}. Wiesbaden:
Springer VS, 2016.

Averbeck-Lietz, Stefanie, Fabien Bonnet, and Bonnet Jacques. ``Le
discours épistémologique des Sciences de l\textquotesingle information
et de la communication.'' \emph{Revue française des sciences de
l'information et de la communication} 4 (2014).
\url{https://doi.org/10.4000/rfsic.823}.

Bachmann-Medick, Doris. \emph{Cultural Turns: Neuorientierungen in den
Kulturwissenschaften}. Reinbek: Rowohlt, 2018.

Bailey, Frederick George. \emph{Stratagems and Spoils: A Social
Anthropology of Politics}. Oxford: Westview Press, 2001.

Baisnée, Olivier. ``Reporting the European Union: A Study in
Journalistic Boredom.'' In \emph{Political Journalism in Transition:
Western Europe in a Comparative Perspective}, edited by Raymond Kuhn and
Rasmus Kleis Nielsen, 131--50. London: I.B. Taurus \& Co., 2013.

Baloge, Martin, and Nicolas Hubé. ``Coproduire les biens politiques:
Journalistes et politiques en comparaison dans des contextes centralisés
et fédéraux.'' \emph{Savoir/Agir}, no. 46 (2019): 57--64.

Benson, Rodney. ``Bringing the Sociology Back In.'' \emph{Political
Communication} 21, no. 3 (2004): 275--92.

Benson, Rodney. ``News Media as a `Journalistic Field': What Bourdieu
Adds to New Institutionalism, and Vice Versa.'' \emph{Political
Communication} 23, no. 2 (2006): 187--203.

Bernier, Marc-François. \emph{Les fantômes du Parlement: L'utilité des
sources anonymes chez les courriéristes parlementaires}. Sainte-Foy:
Presses de l'Université Laval, 2000.

Bläser, Ralph. ``Ménage à trois: la pertinence géographique des
relations de lobbying entre les ONG-Bankwatch, l'État national et la
Banque mondiale à Washington D.C.'' \emph{L\textquotesingle espace
politique}, no. 1 (2007).
\url{https://doi.org/10.4000/espacepolitique.303}.

Blumler, Jay. ``The Fourth Age of Communication.'' \emph{Politiques de
communication}, no. 6 (2016): 19--30.

Bolz, Lisa. ``Recherches sur le journalisme en France et en Allemagne:
un dialogue impossible?'' \emph{Revue française des sciences de
l'information et de la communication}, no. 18 (2019).
\url{https://doi.org/10.4000/rfsic.7702}.

Borowiec, Steven. ``Writers of Wrongs: Have Japan's Press Clubs Created
Overly Cosy Relationships Between Business Leaders and the Press?''
\emph{Index on Censorship} 45, no. 2 (2016): 48--50.

Borucki, Isabelle. \emph{Regieren mit Medien: Auswirkungen der
Medialisierung auf die Regierungskommunikation der Bundesregierung von
1982--2010}. Opladen: Barbara Budrich Verlag, 2014.

Bösch, Frank, and Norbert Frei, eds. \emph{Medialisierung und Demokratie
im 20. Jahrhundert.} Göttingen: Wallstein Verlag, 2006.

Boucheron, Patrick, and Nicolas Offenstadt, eds. \emph{L'espace public
au Moyen-Âge: Débats autour de Jürgen Habermas}. Paris: Presses
Universitaires de France, 2011.

Bourdieu, Pierre. ``Codification.'' In \emph{In Other Words: Essays
Toward a Reflexive Sociology}. Translated by Matthew Adamson, 76--86.
Stanford: Stanford University Press, 1990.

Bourdieu, Pierre. \emph{Language and Symbolic Power}. Cambridge, MA:
Harvard University Press, 1991.

Bourdieu, Pierre. ``Rethinking the State: Genesis and Structure of the
Bureaucratic Field.'' \emph{Sociological Theory} 12, no. 1 (1994):
1--18.

Bourdieu, Pierre. \emph{Sociologie générale.} Vol. 1, \emph{Cours au
collège de France 1981--1983}. Paris: Raisons d'agir/Seuil, 2015.

Bourdieu, Pierre. \emph{Sociologie générale}. Vol. 2, \emph{Cours au
collège de France 1983--1986.} Paris: Raisons d'agir/Seuil, 2016.

Bourdieu, Pierre. \emph{On the State: Lectures at the Collège de France
1989--1992.} Cambridge, MA: Polity Press, 2015.

Bramsted, Ernest K. \emph{Goebbels und die nationalsozialistische
Propaganda 1925--1945}. Frankfurt: Fischer, 1971.

Braud, Philippe. \emph{Sociologie politique}. Paris: LGDJ, 2020.

Brosda, Carsten, and Christian Schicha. ``Interaktion von Politik,
Public Relations und Journalismus.'' In \emph{Politische Akteure in der
Mediendemokratie: Politiker in den Fesseln der Medien?}, edited by
Heribert Schatz, Patrick Rössler, and Jörg-Uwe Nieland, 41--64.
Wiesbaden: Westdeutscher Verlag, 2002.

Bruns, Tissy. \emph{Republik der Wichtigtuer: Ein Bericht aus Berlin}.
Bonn: Bundeszentrale für politische Bildung, 2007.

Burgert, Denise. \emph{Politisch-mediale Beziehungsgeflechte: Ein
Vergleich politikfeldspezifische Kommunikationskulturen in Deutschland
und Frankreich}. Berlin: LIT Verlag, 2010.

Büttner, Ursula. \emph{Weimar: Die überforderte Republik 1918--1933}.
Bonn: Bundeszentrale für politische Bildung, 2008.

Castellvi, César. ``Les Clubs de presse au Japon: Le journaliste,
l'entreprise et ses sources.'' \emph{Sur le journalisme} 8, no. 2
(2019): 124--37.

Charron, Jean. \emph{La production de l'actualité: Une analyse
stratégique des relations entre la presse parlementaire et les autorités
politiques au Québec}. Québec: Boréal, 1994.

Chupin, Ivan. \emph{Les écoles de journalisme: Les enjeux de la
scolarisation d\textquotesingle une profession (1889--2018)}. Rennes:
Presses universitaires de Rennes, 2018.

Chupin, Ivan, Nicolas Hubé, and Nicolas Kaciaf. \emph{Histoire politique
et économique des médias en France.} Paris: La Découverte, 2012.

Cohen, Jonathan, Yariv Tsafti, and Tamir Sheafer. ``The Influence of
Presumed Media Influence in Politics: Do Politicians' Perceptions of
Media Power Matter?'' \emph{Public Opinion Quaterly} 72, no. 2 (2008):
331--44.

Crouse, Timothy. \emph{The Boys on the Bus}. New York: Random House,
2003.

Dahrendorf, Ralf. \emph{Gesellschaft und Demokratie in Deutschland}.
Munich: Piper, 1965.

Darras, Eric. ``Division du travail politiste et travail politiste de
division: L'exemple de la communication.'' In \emph{La science
politique: Une et multiple}, edited by Eric Darras and Olivier Philippe.
Paris: L'Harmattan, 2004.

Davis, Aeron. \emph{Public Relations Democracy: Public Relations,
Politics and the Mass Media in Britain.} Manchester: Manchester
University Press, 2002.

de Certeau, Michel. \emph{The Practice of Everyday Life.} Berkeley:
University of California Press, 1984.

Déloye, Yves. \emph{Sociologie historique du politique.} Paris: La
Découverte, 2017.

Desrumaux, Clément, and Jérémie Nollet, eds. ``Le travail politique par
et pour les médias.'' \emph{Réseaux}, no. 187 (2014).

Donsbach, Wolfgang, and Thomas Patterson. ``Political News Journalists:
Partisanship, Professionalism, and Political Roles in Five Countries.''
In \emph{Comparing Political Communication: Theories, Cases and
Challenges,} edited by Frank Esser and Barbara Pfetsch, 251--70.
Cambridge: Cambridge University Press, 2004.

Dormagen, Jean-Yves, and Daniel Mouchard. \emph{Introduction à la
sociologie politique}. Bruxelles: De Boeck, 2019.

Dunwoody, Sharon, and Steven Shields. ``Accounting for Patterns of
Selection of Topics in Statehouse Reporting.'' \emph{Journalism
Quaterly} 63 (1986): 488--96.

Elias, Norbert. \emph{The Court Society}. Dublin: University College
Dublin Press, 2006. First published 1969 by Hermann Luchterhand Verlag
(Darmstadt/Neuwied).

Elias, Norbert. \emph{The Germans: Power Struggles and the Development
of Habitus in the Nineteenth and Twentieth Centuries.} New York:
Columbia University Press, 1996.

Esser, Frank. ``Editorial Structures and Work Principles in British and
German Newsrooms.'' \emph{European Journal of Communication} 13, no. 3
(1998): 375--405.

Esser, Frank, Carsten Reinemann, and David Fan. ``Spin-doctoring in
British and German Election Campaigns.'' \emph{European Journal of
Communication} 15, no. 2 (2000): 209--39.

Fröhlich, Elke. ``Joseph Goebbels: Profil de sa propagande
(1926--1939).'' In \emph{Joseph Goebbels: Journal 1933--1939}, edited by
Pierre Ayçoberry and Barbara Lambauer, 17--53. Paris: Tallandier, 2007.

Fulda, Bernhard. \emph{Press and Politics in the Weimarer Republic}.
Oxford: Oxford University Press, 2013.

Gatien, Emmanuelle. ``\,`Un peu comme la pluie': La co-production
relative de la valeur d'information en temps de guerre.''
\emph{Réseaux}, no. 157--58 (2009): 61--88.

Georgakakis, Didier. ``Le gouvernement des esprits: concurrence
internationale, comparatisme et développement de la propagande
d\textquotesingle État en Europe (1917--1940).'' In \emph{Les sciences
de gouvernement en Europe}, edited by Olivier Ihl, Martin Kaluszynski,
and Gilles Pollet, 53--57. Paris: Economica, 2003.

Georgakakis, Didier\emph{. La République contre la propagande: aux
origines perdues de la communication d\textquotesingle État en France
(1917--1940).} Paris: Economica, 2004\emph{.}

Georgakakis, Didier, and Jay Rowell, eds. \emph{The Field of Eurocracy:
Mapping the EU Staff and Professionals}. Basingstoke: Palgrave
Macmillan, 2013.

Giddens, Anthony. \emph{The Constitution Of Society: Outline of the
Theory of Structuration}. Berkeley: University of California Press,
1986.

Goffman, Erving. \emph{Frame Analysis: An Essay on the Organization of
Experience.} Boston: Northeastern University Press, 1986.

Grittman, Elke. ``Organisationeller Kontext.'' In \emph{Grundlagentexte
zur Journalistik}, edited by Irene Neverla, Elke Grittmann, and Pater
Monika, 291--301. Konstanz: UVK, 2002.

Gurevitch, Michael, Stephen Coleman, and Jay Blumler. ``Political
Communication: Old and New Media Relationship.'' \emph{The Annals of the
American Academy of Political and Social Science} 625, no. 1 (2009):
164--81.

Habermas, Jürgen. \emph{The Structural Transformation of the Public
Sphere.} Cambridge, MA: MIT Press, 1989.

Hägerstrand, Torsten. ``Aspects of the of the Spatial Structure of
Social Communication and the Diffusion of Information.'' \emph{Papers of
the Regional Science Association} 16, no. 1 (1965): 16--27.

Hall, Peter A., and Rosemary C. R. Taylor. ``Political Science and the
Three New Institutionalisms.'' \emph{Political Studies} 44, no. 5
(1996): 936--57.

Hallin, Daniel, and Paolo Mancini. \emph{Comparing Media System: Three
Models of Media and Politics}. Cambridge: Cambridge University Press,
2004.

Hanitzsch, Thomas. ``Kriegskorrespondenten Entmystifizieren.'' In
\emph{Kriegskorrespondenten}, edited by Barbara Korte and Horst Tonn,
39--58. Wiesbaden: VS Verlag für Sozialwissenschaften, 2007.

Hanitzsch, Thomas. ``Deconstructing Journalism Culture: Toward a
Universal Theory.'' \emph{Communication Theory} 17, no. 4 (2007):
367--85.

Hanitzsch, Thomas, and Rosa Berganza. ``Explaining Journalists' Trust in
Public Institutions Across 20 Countries: Media Freedom, Corruption, and
Ownership Matter Most.'' \emph{Journal of Communication} 62, no. 5
(2012): 794--814.

Herzer, Martin. \emph{The Media, European Integration and the Rise of
Euro-Journalism, 1950s--1970s}. Basingstoke: Palgrave Macmillan, 2019.

Hofstetter, Brigitte, and Philomen Schoenhagen. ``Wandel redaktioneller
Strukturen und journalistischen Handelns'' {[}Changing newsroom
structures and journalistic routines{]}. \emph{Studies in Communication
and Media} 3, no. 2 (2014): 228--52.

Hofstetter, Brigitte, and Philomen Schoenhagen. ``When Creative
Potentials Are Being Undermined By Commercial Imperatives.''
\emph{Digital Journalism} 5, no. 1 (2017): 44--60.

Holt, Kristoffer, and André Haller. ``What Does `Lügenpresse' Mean?
Expressions of Media Distrust on PEGIDA's Facebook Pages.''
\emph{Politik} 20, no. 4 (2017): 42--57.
\url{https://doi.org/10.7146/politik.v20i4.101534}.

Hoyt, Kendall, and Frances S. Leighton. \emph{Drunk before Noon: The
Behind-the-Scenes Story of the Washington Press Corps}. Englewood
Cliffs, NJ: Prentice Hall, 1979.

Hubé, Nicolas. ``À la recherche d'une universalité du journalisme: la
\emph{Journalistik} allemande.'' \emph{Revue française des sciences de
l'information et de la communication}, no. 19 (2020).
\url{https://doi.org/10.4000/rfsic.9269}.

Hubé, Nicolas. ``Understanding the Off-The-Record as a Social Practice:
German Press-Politics Relations Seen from France.'' \emph{Laboratorium:
Russian Review of Social Research} 9, no. 2 (2017): 4--29.

Hubé, Nicolas. \emph{Décrocher la ``Une'': Le choix des titres de
première page de la presse quotidienne en France et en Allemagne
(1945--2005)}. Strasbourg: Presses universitaires de Strasbourg, 2008.

Hubé, Nicolas. \emph{La politique des chemins courts: Un siècle de
relations entre journalistes et communicants gouvernementaux en
Allemagne (1918--2018)}. Vulaines-sur-Seine: Editions du Croquant, 2022.

Hubé, Nicolas. ``L'\emph{Öffentlichkeitsarbeit}~ou la propagande au
service de la démocratie: La définition du travail de communication
gouvernementale dans les premières années de la RFA (1949--1953).''
\emph{Histoire, économie \& société} 41, no. 4 (2022): 9--26.
\url{https://doi.org/10.3917/hes.224.0009}.

Jesse, Eckhard. \emph{Systemwechsel in Deutschland:
1918/19---1933---1945/49---1989/90}. Bonn: Bundeszentrale für politische
Bildung, 2013.

Jitsuhara, Takashi. ``Guarantee of the Right to Freedom of Speech in
Japan---A Comparison with Doctrines in Germany.'' In \emph{Contemporary
Issues in Human Rights Law,} edited by Yumiko Nakanishi, 169--91.
Singapore: Springer, 2018.

Juhem, Philippe, and Julie Sedel, eds. \emph{Agir par la parole:
Porte-paroles et asymétries de l'espace public}. Rennes: Presses
Universitaires de Rennes, 2016.

Jungblut, Peter. ``Unter vier Reichskanzlern: Otto Hammann und die
Pressepolitik der deutschen Reichsleitung 1890 bis 1916.'' In
\emph{Propaganda: Meinungskampf, Verführung und politische Sinnstiftung
1789--1989}, edited by Ute Daniel and Wolfram Siemann, 101--16.
Frankfurt am Main: Fischer Taschenbuch Verlag, 1994.

Kaciaf, Nicolas, and Jérémie Nollet, eds. ``Journalisme: retour aux
sources.'' \emph{Politiques de communication}, no. 1 (2013): 5--34.

Kaciaf, Nicolas. ``Des dissidences aux confidences: Ce que couvrir un
parti veut dire.'' In \emph{L\textquotesingle informel pour informer:
Les journalistes et leurs sources}, edited by Jean-Baptiste Legavre,
21­--37. Paris: L\textquotesingle Harmattan/Pepper, 2014.

Kaciaf, Nicolas. \emph{Les Pages ``Politique'': Histoire du journalisme
politique dans la presse française (1945--2006)}. Rennes: Presses
Universitaires de Rennes, 2013.

Kepplinger, Hans-Mathias. \emph{Journalismus als Beruf}. Wiesbaden: VS
Verlag, 2011.

Krüger, Gunnar. ``\emph{Wir sind doch kein exklusiver Club!'': Die
Bundespressekonferenz in der Ära Adenauer}. Münster: Lit, 2005.

Lagroye, Jacques, Bastien François, and Frédéric Sawicki.
\emph{Sociologie politique}. Paris: Presses de Sciences Po \& Dalloz,
2012.

Lagroye, Jacques. ``La légitimation.'' In \emph{Traité de science
politique}, vol. 1, edited by Madeleine Grawitz and Jean Leca. Paris:
Presses Universitaires de France, 1985.

Legavre, Jean-Baptiste. ``Off the record: Mode d\textquotesingle emploi
d\textquotesingle un instrument de coordination.'' \emph{Politix}, no.
19 (1992): 135--48.

Legavre, Jean-Baptiste, ed. \emph{L\textquotesingle informel pour
informer: Les journalistes et leurs sources}. Paris:
L\textquotesingle Harmattan/Pepper, 2014.

Legavre, Jean-Baptiste. ``Entre conflit et cooperation: Les journalistes
et les communicants comme `associés-rivaux.'\,'' \emph{Communication \&
langages}, no. 169 (2011): 105--23.

Lévêque, Sandrine. \emph{Les journalistes sociaux: Histoire et
sociologie d'une spécialité journalistique}. Rennes: Presses
Universitaires de Rennes, 2000.

Löblich, Maria, Niklas Venema, and Elisa Pollack. ``West Berlin's
Critical Communication Studies and the Cold War: A Study on Symbolic
Power from 1948 to 1989.'' \emph{History of Media Studies} 2 (2022).
\url{https://doi.org/10.32376/d895a0ea.d0db9590}.

Luhmann, Niklas. \emph{Trust and Power}. New York: John Wiley and Sons,
1979.

Luhmann, Niklas. \emph{The Reality of the Mass Media}. Stanford:
Stanford University Press, 2000.

Manin, Bernard. \emph{The Principles of Representative Government}.
Cambridge: Cambridge University Press, 1997.

Massey, Doreen. ``Introduction: Geography Matters.'' In \emph{Geography
Matters,} edited by Doreen Massey and John Allen, 1--11. Cambridge:
Cambridge University Press, 1984.

Medvetz, Thomas. ``Les think tanks dans le champ du pouvoir étasunien.''
\emph{Actes de la recherche en sciences sociales}, no. 200
(2013):~44--55.

Mercier, Arnaud. ``L'institutionalisation de la profession des
journalistes.'' \emph{Hermès}, no. 13--14 (1994): 219--35.

Mercier, Arnaud. ``La communication politique en France : Un champ de
recherche qui doit encore s'imposer.'' \emph{L'Année sociologique}, no.
51 (2001): 355--66.

Mergel, Thomas. \emph{Parlamentarische Kultur in der Weimarer Republik.}
Düsseldorf: Droste Verlag, 2002.

Mergel, Thomas. \emph{Propaganda nach Hitler: Eine Kulturgeschichte des
Wahlkampfs in der Bundesrepublik, 1949--1990}. Göttingen: Wallstein
Verlag, 2010.

Meyen, Michael, and Claudia Riesmeyer. \emph{Diktatur des Publikums:
Journalisten in Deutschland.} Konstanz: UVK, 2009.

Mudge, Stephanie, and Antoine Vauchez. ``Building Europe on a Weak
Field: Law, Economics, and Scholarly Avatars in Transnational
Politics.'' \emph{American Journal of Sociology} 118, no. 2 (2012):
449--92.

Mühlenfeld, Daniel. ``Vom Kommissariat zum Ministerium: Zur
Gründungsgeschichte des Reichsministeriums für Volksaufklärung und
Propaganda.'' In \emph{Hitlers Kommissare: Sondergewalten in der
nationalsozialistischen Diktatur,} edited by Rüdiger Hachtmann and
Winfried Süß, 72--92. Konstanz: Wallstein Verlag, 2006.

Münkler, Herfried, and Skadi Krause. ``Geschichte und Selbstverständnis
der Politikwissenschaft in Deutschland.'' In \emph{Politikwissenschaft:
Ein Grundkurs}, edited by Herfried Münkler, 13--54. Hamburg: Rowohlt,
2003.

Nester, William. ``Japan\textquotesingle s Mainstream Press: Freedom to
Conform?'' \emph{Pacific Affairs} 62, no. 1(1989): 29--39.

Neveu, Érik. \emph{Sociologie du journalisme}. Paris: La Découverte,
2019.

Neveu, Érik, and Raymond Kuhn. ``Political Journalism: Mapping the
Terrain.'' In \emph{Political Journalism: New Challenges, New
Practices}, edited by Raymond Kuhn and Erik Neveu, 1--21. London:
Routledge, 2002.

Nuernbergk, Christian, and Jan-Hinrik Schmidt. ``Twitter im
Politikjournalismus: Ergebnisse einer Befragung und Netzwerkanalyse von
Hauptstadtjournalisten der Bundespressekonferenz.'' \emph{Publizistik}
65, no. 1 (2020): 41--61.

O'Dwyer, Jane. ``Japanese Kisha Clubs and the Canberra Press Gallery:
Siblings or Strangers.'' \emph{Asia Pacific Media Educator} 1, no. 16
(2005): 1--16.

Ollivier-Yaniv, Caroline. ``De l'opposition entre `propagande' et
`communication publique' à la définition de la politique du discours.''
\emph{Quaderni}, no. 72 (2010): 87--99.

Ollivier-Yaniv, Caroline. \emph{L'État communicant}. Paris: Presses
Universitaires de France, 2000.

Pfetsch, Barbara. \emph{Politische Kommunikationskultur: Politische
Sprecher und Journalisten in der Bundesrepublik und den USA im
Vergleich.} Wiesbaden: Westdeutscher Verlag, 2003.

Revers, Matthias. \emph{Contemporary Journalism in the US and Germany:
Agents of Accountability.} New York: Palgrave MacMillan, 2017.

Rieffel, Rémy. \emph{Sociologie des médias}. Paris: Ellipses, 2015.

Rinsdorf, Lars, and Laura Theiss. ``Leidenschaftliche Amateur*innen oder
kühle Profis: Zum Integrationspotenzial der freien Mitarbeiter*innen
lokaler Tageszeitungen.'' In \emph{Integration durch Kommunikation:
Jahrbuch der Publizistik- und Kommunikationswissenschaft,} edited by
Volker Gehrau, Anni Waldherr, and Armin Scholl, 57--67. Münster:
Deutsche Gesellschaft für Publizistik- und Kommunikationswissenschaft
e.V.; Westfälische Wilhelms-Universität Münster, Institut für
Kommunikationswissenschaft, 2019.

Riutort, Philippe. ``Sociologiser la communication politique?,''
\emph{Politique et Sociétés} 26, no. 1 (2007): 77--95.

Riutort, Philippe. \emph{Sociologie de la communication politique}.
Paris: La Découverte, 2020.

Robert, Valérie. ``Staatsfreiheit ou intervention de l'État? Le modèle
allemand de l'audiovisuel public.'' \emph{Sur le journalisme} 2, no. 2
(2013): 118--31.

Ross, Karen. ``Danse Macabre: Politicians, Journalists, and the
Complicated Rumba of Relationships.'' \emph{The International Journal of
Press/Politics} 15, no. 3 (2010): 272--94.

Ruellan, Denis. \emph{Les ``Pro'' du journalisme: De
l\textquotesingle état au statut, la construction d\textquotesingle un
espace professionnel}. Rennes: Presses universitaires de Rennes, 1997.

Rühl, Manfred. ``Organisatorischer Journalismus: Tendenzen der
Redaktionsforschung.'' In \emph{Grundlagentexte zur Journalistik},
edited by Neverla Irene, Grittmann Elke, and Pater Monika, 303--20.
Konstanz: UVK, 2002.

Rühl, Manfred, \emph{Die Zeitungsredaktion als organisiertes soziales
System}. Fribourg: Universitätsverlag Freiburg, 1979.

Sapiro, Gisèle. ``Le champ est-il national? La théorie de la
différenciation sociale au prisme de l\textquotesingle histoire
globale.'' \emph{Actes de la recherche en sciences sociales}, no. 200
(2013): 70--85.

Schlesinger, Philip. ``Rethinking the Sociology of Journalism: Source
Strategies and the Limits of Media-Centrism.'' In \emph{Public
Communication: The New Imperatives}, edited by Margorie Ferguson,
61--83. London: SAGE Publications, 1990.

Schudson, Michael. ``Autonomy from What?'' In \emph{Bourdieu and the
Journalistic Field}, edited by Rodney Benson and Erik Neveu, 214--23.
Cambridge, MA: Polity, 2005.

Schudson, Michael. ``The `Public Sphere' and Its Problems: Bringing the
State (Back) In.'' \emph{Notre Dame Journal of Law, Ethics \& Public
Policy} 8, no. 2 (1994): 529--46.

Sösemann, Bernd, ed. \emph{Propaganda: Medien und Öffentlichkeit in der
NS-Diktatur}. Stuttgart: Franz Steiner Verlag, 2011.

Stankovitch, Michel. ``Les Services de presse des gouvernements et de la
S.D.N.'' PhD diss., Université de Paris, 1939.

Steinmetz, George. ``Field Theory and Interdisciplinarity: History and
Sociology in Germany and France during the Twentieth Century.''
\emph{Comparative Studies in Society and History} 59, no. 2 (2017):
477--514.

Steinmetz, George. ``Bourdieusian Field Theory and the Reorientation of
Historical Sociology.'' In \emph{The Oxford Handbook of Pierre
Bourdieu}, edited by Thomas Medvetz and Jeffrey Sallaz. Oxford: Oxford
University Press, 2018.

Strömbäck, Jesper, and Lars W. Nord. ``Do Politicians Lead the Tango: A
Study of the Relationship between Swedish Journalists and Their
Political Sources in the Context of Election Campaigns.'' \emph{European
Journal of Communication} 21, no. 2 (2006): 147--64.

Taketoshi, Yamamoto. ``The Press Clubs of Japan.'' \emph{The Journal of
Japanese Studies} 15, no. 2 (1989): 371--88.

Tixier, Florian. ``Concurrences et coopérations pour la production de
l'information européenne.'' \emph{Sur le journalisme} 8, no. 1 (2019):
40--53.

Tixier, Florian. ``En quête de professionnalisme: L'Association des
journalistes européens, des spécialistes de l'Europe aux journalistes
spécialisés.'' In \emph{Les Médiations de l\textquotesingle Europe
politique,} edited by Philippe Aldrin, Nicolas Hubé, Caroline
Ollivier-Yaniv, and Jean-Michel Utard, 285--305. Strasbourg: Presses
Universitaires de Strasbourg, 2014.

Tunstall, Jeremy. \emph{The Westminster Lobby Correspondents:
Sociological Study of National Political Journalism.} London: Routledge
\& Kegan Paul, 1970.

Van Aelst, Peter, Tamir Sheafer, Nicolas Hubé, and Stylianos
Papathanassopoulos. ``Personalization.'' In \emph{Comparing Political
Journalism}, edited by Claes de~Vreese, Frank Esser, and David Hopmann,
112--30. London: Routledge, 2017.

Vauchez, Antoine. ``Interstitial Power in Fields of Limited Statehood:
Introducing a `Weak Field' Approach to the Study of Transnational
Settings.'' \emph{International Political Sociology} 5, no. 3 (2011):
340--45.

Weber, Max. \emph{The Vocations Lectures}. Cambridge, MA: Hackett
Publishing Company, 2004. First published 1919 by Duncker \& Humblot
(Munich).

Wegmann, Nikolaus, and Ute Mehnert. ``\emph{Scoop-o-mania},
l'introduction du scoop dans la vie politique allemande.'' \emph{Le
Temps des Médias,} no. 7 (2006): 148--49.

Weischenberg, Siegfried. \emph{Journalistik: Theorie und Praxis
aktueller Medienkommunikation; Band 2: Medientechnik, Medienfunktionen,
Medienakteure.} Opladen: Westdeutscher Verlag, 1995.

Weiß, Matthias. ``Journalisten: Worte als Taten.'' \emph{Karrieren im
Zwielicht: Hitlers Eliten nach 1945,} edited by Norbert Frei, 241--301.
Frankfurt: Campus Verlag, 2001.

Wiedemann, Thomas, and Michael Meyen, eds. \emph{Pierre Bourdieu und die
Kommunikationswissenschaft: Internationale Perspektiven}. Köln: Herbert
von Halem Verlag, 2013.



\end{hangparas}


\end{document}