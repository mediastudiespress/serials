% see the original template for more detail about bibliography, tables, etc: https://www.overleaf.com/latex/templates/handout-design-inspired-by-edward-tufte/dtsbhhkvghzz

\documentclass{tufte-handout}

%\geometry{showframe}% for debugging purposes -- displays the margins

\usepackage{amsmath}

\usepackage{hyperref}

\usepackage{fancyhdr}

\usepackage{hanging}

\hypersetup{colorlinks=true,allcolors=[RGB]{97,15,11}}

\fancyfoot[L]{\emph{History of Media Studies}, vol. 3, 2023}


% Set up the images/graphics package
\usepackage{graphicx}
\setkeys{Gin}{width=\linewidth,totalheight=\textheight,keepaspectratio}
\graphicspath{{graphics/}}

\title[Review of Innis]{Empire and Communications (2022 Edition), reviewed by Chris Russill} % longtitle shouldn't be necessary

% The following package makes prettier tables.  We're all about the bling!
\usepackage{booktabs}

% The units package provides nice, non-stacked fractions and better spacing
% for units.
\usepackage{units}

% The fancyvrb package lets us customize the formatting of verbatim
% environments.  We use a slightly smaller font.
\usepackage{fancyvrb}
\fvset{fontsize=\normalsize}

% Small sections of multiple columns
\usepackage{multicol}

% Provides paragraphs of dummy text
\usepackage{lipsum}

% These commands are used to pretty-print LaTeX commands
\newcommand{\doccmd}[1]{\texttt{\textbackslash#1}}% command name -- adds backslash automatically
\newcommand{\docopt}[1]{\ensuremath{\langle}\textrm{\textit{#1}}\ensuremath{\rangle}}% optional command argument
\newcommand{\docarg}[1]{\textrm{\textit{#1}}}% (required) command argument
\newenvironment{docspec}{\begin{quote}\noindent}{\end{quote}}% command specification environment
\newcommand{\docenv}[1]{\textsf{#1}}% environment name
\newcommand{\docpkg}[1]{\texttt{#1}}% package name
\newcommand{\doccls}[1]{\texttt{#1}}% document class name
\newcommand{\docclsopt}[1]{\texttt{#1}}% document class option name


\begin{document}

\begin{titlepage}

\begin{fullwidth}
\noindent\LARGE\emph{Book review
} \hspace{88mm}\includegraphics[height=1cm]{logo3.png}\\
\noindent\hrulefill\\
\vspace*{1em}
\noindent{\Huge{\emph{Empire and Communications} (2022 Edition)\par}}

\vspace*{1.5em}

\noindent\LARGE{reviewed by Chris Russill} \href{https://orcid.org/0009-0001-4921-0151}{\includegraphics[height=0.5cm]{orcid.png}}\par\marginnote{\emph{\emph{Empire and Communications} (2022 Edition), reviewed by Chris Russill, \emph{History of Media Studies} 3 (2023), \href{https://doi.org/10.32376/d895a0ea.9fd5d1ed}{https://doi.org/ 10.32376/d895a0ea.9fd5d1ed}.} \vspace*{0.75em}}
\vspace*{0.5em}
\noindent{{\large\emph{Carleton University}, \href{mailto:ChrisRussill@cunet.carleton.ca}{ChrisRussill@cunet.carleton.ca}\par}} \marginnote{\href{https://creativecommons.org/licenses/by-nc/4.0/}{\includegraphics[height=0.5cm]{by-nc.png}}}



\end{fullwidth}

\vspace*{1em}


\noindent\small{Harold A. Innis. \emph{Empire and Communications}, edited and with an
introduction by William J. Buxton. 288 pp., index. Toronto: University
of Toronto Press, 2022. \$45 (paper).}



\vspace*{0.25em}



\newthought{The invasion of} Ukraine by the Russian military and its inescapable
reminder of the aggressions, murder, and genocide that shapes the
colonial history of the Bloodlands all but ensures that questions of
empire and imperialism will not soon slip scholarly attention. These
events have not only reshaped the geopolitics of colonial ambition in
strikingly overt ways, but foreground the importance of communication to
questions of war and occupation more generally.

In this respect, the 2022 appearance of a revised edition of Harold
Innis's text, \emph{Empire and Communications}, is timely given his
interest in the conditions through which empires emerge, extend, endure,
and dissolve. His understanding of historiography, knowledge, and
communication repays careful consideration, especially his concerns with
the incapacities of scholars to set aside the biases of the empires they
inhabit.

As a revised version of the 1972 edition edited by Mary Quail Innis, the
2022 reissue adds a nostalgic cover, expanded index, and updated
bibliography, as well as an introduction by Innis scholar William
Buxton. In this update, like several before it, the text is mostly
unchanged. (Let us not speak of the Godfrey edition.\footnote{The
  Godfrey edition preserved Mary Quail Innis's improvements to the
  original text in an uncredited fashion, inserted editorial subheadings
  to reorganize the prose in an apparent attempt to aid readers, and
  added illustrations of questionable value, among other problems. For a
  concise summary, see Alison Beale, ``Book Review: Empire and
  Communications,'' \emph{Canadian Journal of Communication} 15, no. 1
  (1990).})

This proliferation of editions brings the political and
institutional investments of the field into the foreground, as each
iteration raises anew questions about the stakes of working with Innis's
understanding of communication. The key concepts and familiar arguments
are too well known to rehearse at length and in some cases beside the
point. The question of the early/late or A series/B series periodization
of Innis might safely be set aside without the need to establish longer
continuities across the wider oeuvre or sort out his relationship to
Marshall McLuhan. (A series/B series is a division of Innis's
scholarship proposed by Havelock (1981) that distinguishes between the
writing on economic concerns (A series) and his later work on
theoretical questions of philosophical history (B series).)

One result of the accumulating editions is that Innis's technique of
aggregation shines through. If \emph{Empire and Communications} is
itself a product of aggregation, it should be no surprise that those
most engaged with the text

\enlargethispage{2\baselineskip}

\vspace*{2em}

\noindent{\emph{History of Media Studies}, vol. 3, 2023}


\end{titlepage}




\noindent cannot resist adding notes when passing it
forward. Innis loved to aggregate statements relevant to his
perspective,\footnote{Liam Cole Young, ``The McLuhan-Innis Field: In
  Search of Media Theory,'' \emph{Canadian Journal of Communication} 44,
  no. 4 (2019).} a practice of recording information mostly untroubled
by the arguments of the texts he consulted (xxiv), a characteristic also
applauded by McLuhan in his 1972 introduction, republished with this
edition. While the text continues to accumulate minutiae mined by
editors, the main throughlines remain as stark as ever and suggest that
the concepts, arguments, and mode of analysis are less important than
the problem of empire that Innis articulated in its communicative
dimensions. (Innis used the terminology of empire consistently
throughout the text and attended mainly to the large-scale
administrative structures involved in its durability.) The dynamics of
empires are illuminated through attention to the communicative systems
in which they are embedded. In this respect, Innis developed a
materialist conception of communication that remixed our received
knowledge of civilization to pose novel questions about the durability
and dissolution of British empire.

Buxton's introduction notes that Innis was uninterested in founding a
field and that he took up questions of communication and media primarily
as a way of ``reconfiguring economic history'' (xiv, note 41). Buxton
drives the point home by attending to the internal economy of Innis's
work in the 1940s, his wider research and reading practices, and his
correspondence with peers. The argument is compelling and contributes to
the main lines of Innisian scholarship by foregrounding an expert
understanding of the archive broadly understood. \emph{Empire and
Communications} is one part of a broader cluster of research that Innis
developed to address issues that were central to ``the economic history
of British Empire'' (xvii). Buxton also illustrates in passing how the
Beit Lectures were not just an occasion for the oral medium, but part of
wider institutional circuits through which imperial knowledge was
circulated. (Innis delivered these lectures in 1948 and they were the
origin of the published text.)

Buxton's most interesting technique is to re-read Innis through the
marginalia he left in the original text (Mary Quail Innis made these
handwritten notes available to scholars by incorporating them into
footnotes in the 1972 edition she edited, a feature retained in this new
version.) It is a departure from usual approaches to Innis and a
delightful way to reimagine the main contributions of the book. The
results are perhaps not so surprising, insofar as they appear to reduce
to a few key points: Innis's recognition that he had downplayed the
violent role of war and the military, his desire to better integrate the
different civilizational clusters in terms of an analysis that would
amplify the exceptionalism of Greek culture, and his hints for expanding
the evidentiary base and reading that informed the early chapters in
particular. These last kinds of glosses confirm (to my mind at least)
that the relentless aggregation of material was not just a starting
point or rudimentary form of historical analysis but central to Innis's
way of tracing the networks of empire.

Buxton deserves credit for putting this gently modified version of Innis
as a theorist of economic history and British empire before us, and for
leaving open the implications of this depiction for communication and
media studies. In addition, the way Buxton tracks the shifts from
civilizational description to media thematics across chapters, and
otherwise clarifies some of the opaque elements of the book, is
immensely helpful. Still, the degree to which Buxton remains aloof from
other efforts to mobilize Innis's work strikes me as curious. Buxton
mostly refrains from engaging communication and media scholarship, even
when puncturing some of the mythos about Innis that circulate in that
space. As a result, he appears to avoid some of the difficulties that
accrue when the significance of a scholar is situated as foundational to
a field of scholarship; however, by indexing \emph{Empire and
Communications} so tightly to one man's intellectual biography, rather
than the machinations of empire in which he was embedded, or the
formalization of communication expertise within which those machinations
are extended, the question of how our intellectual habits, archives, and
exclusions are shaped by empire is somewhat constrained. To be sure, as
Paul Heyer once put it, the connection between British empire and what
is presented in \emph{Empire and Communications} is ``drawn in loose
fashion,''\footnote{Paul Heyer, \emph{Harold Innis} (Lanham, MD: Rowman
  \& Littlefield Publishers, 2003), 45.} but Innis did force these
questions into the foreground of communication theory.

Buxton's hesitation to engage the field is set aside when Innis's notion
of history is discussed. Buxton's introduction encourages readers to
align Innis with Hegel's approach to philosophical history, an
orientation that distinguishes an accumulation of information from the
interpretations afforded by reflection as these attune us to the wider
significance of historical change (xiii). It is an inclination to Hegel
in apparent if unspoken tension with the many kinds of media and
infrastructural materialism that Innis has inspired, not to mention
McLuhan's own introduction to the volume (the first sentence of which
distinguishes Innis's thought from Hegelian forms of abstraction). I
found it interesting that Buxton mobilized Menahem Blondheim's approach
to Innis and Hegel, itself a rebuke to conventional ways of reading
Innis in the field, as a key to understanding the text today.
Nevertheless, it is curious that this was done with little context or
historical commentary on the last two decades of scholarship. I also
wonder if tethering Innis to Hegel overrides the crucial realization by
Innis that the apparent development of Canada's quasi-political autonomy
from British empire was only a prelude to its transformation by the
forms of colonial integration fostered by post--World War II projects of
American imperialism. Innis, it seems to me, is better placed among
scholars emphasizing the intimacies produced by empire than with Hegel.

The problem is posed more precisely when we recall with Buxton how Innis
situated his interest in empire. The introduction covers this ground
with admirable clarity. Innis was concerned with the way present
circumstances impress themselves on civilizational scholarship, a
problem he viewed as shaping most studies of the subject in an
unacknowledged or unconscious way. The growing interest in
civilizational history, Innis noted, was probably a result of the nature
of our civilization, a feature of which was the accumulation and storage
of information in a context of extensive mechanization. It was through
these elements that Innis understood the integration of Canadian
geographies into a continental empire, a set of historical processes
sketched by subsequent scholarship, including Jody Berland,\footnote{Jody
  Berland, \emph{North of Empire} (Durham, NC: Duke University Press,
  2009).} Shirley Roburn,\footnote{Shirley Roburn, ``Innis and
  Environmental Politics: Practical Insight from the Yukon,'' in
  \emph{Harold Innis and the Canadian North: Appraisals and
  Contestations}, ed. William Buxton (Montreal: McGill-Queens University
  Press, 2013).} and Michael Stamm,\footnote{Michael Stamm, \emph{Dead
  Tree Media: Manufacturing the Newspaper in Twentieth-Century North
  America} (Baltimore, MD: Johns Hopkins University Press, 2018).} among
others. The pulp and paper industry, a conglomeration of mechanical,
energy, and chemical expertise, was the material network through which
Canada's ecology was reshaped to support the informational demands of a
rapidly extending American empire. It was, in brief, a colonial
experience that brought communication into the foreground of
intellectual thought as the medium through which empires expand, endure,
and dissolve.

If resituated within wider genealogies of empire, or used as a means to
explore them, the question of how to receive the histories offered in
\emph{Empire and Communications} re-acquires some of the political
intensity with which Innis invested his work. Innis was more interested
in aggregating information relevant to the historicization of empire
than in an orderly presentation of facts within a theoretical framework
or archival system of classification. Whether the overflowing nature of
the text results from the hurried writing of a scholar doing too much,
or from a patient willingness to reflect the contradictions and
fractures in knowledge that colonial powers produce, Innis's remix of
accepted expertise, administrative data, and personal reflection in
exploration of the longer continuities and undercurrents of empire has
called forth a variety of responses that are admittedly difficult to
organize in efficient fashion.

Of these possible responses, I am most fascinated by the way aggregated
material often explodes rather than confirms the philosophical histories
that scholars have used to ground the field of communication and media
studies. These might be constitutive exclusions, or what Wendy
Willems\footnote{Wendy Willems, ``Provincializing Hegemonic Histories of
  Media and Communication Studies: Toward a Genealogy of Epistemic
  Resistance in Africa,'' \emph{Communication Theory} 24, no. 4 (2014).}
refers to as baffling silences in the manner of Michel-Rolph
Trouillot---and one is reminded here of Pierre-Franklin
Tavarès's\footnote{Pierre-Franklin Tavarès, "Hegel et
  l\textquotesingle abbé Grégoire: question noire et révolution
  française,'' \emph{Annales Historiques de la Révolution Française} 3,
  no. 4 (1993).} and Susan Buck-Morss's\footnote{Susan Buck-Morss,
  ``Hegel and Haiti,'' \emph{Critical Inquiry} 26, no. 4 (2000).}
recovery of Hegel's fascination with slave rebellion and anti-colonial
war in Haiti when fashioning his accounts of freedom, slavery, and world
history, a fact elided by most Hegelian scholarship and historiography
more generally.\footnote{See Michel-Rolph Trouillot, \emph{Silencing the
  Past: Power and the Production of History} (Boston: Beacon Press,
  1995); Lisa Lowe, \emph{The Intimacies of the Four Continents}
  (Durham, NC: Duke University Press, 2015); Armond Towns, ``The (Black)
  Elephant in the Room: McLuhan and the Racial," \emph{Canadian Journal
  of Communication} 44, no. 4 (2019); Wendy Willems, ``Provincializing
  Hegemonic Histories of Media and Communication Studies: Toward a
  Genealogy of Epistemic Resistance in Africa,'' \emph{Communication
  Theory} 24, no. 4 (2014).} More recently, Darin Barney\footnote{Darin
  Barney, ``Infrastructure and the Form of Politics,'' \emph{Canadian
  Journal of Communication} 46, no. 2 (2021).} has drawn scholarship on
settler colonialism and Indigenous resurgence together with Innisian
strands of study to illuminate how the historicity of empire animates
the biases of an infrastructure that operates as an administrative
medium of a colonial state. The implication of education in the
administration of such infrastructure is well-known and documented in
government and church archives, in settlement agreements, and in the
experiences and memories shared publicly with those composing the
reports of the Truth and Reconciliation Commission (TRC) of Canada. The
role of education in policies described by the TRC as cultural genocide
has meant challenging refusals to acknowledge the historicity of the
present, a form of contestation evident in the vandalizing, toppling,
and removal of the Egerton Ryerson statue on the campus of what is now
called Toronto Metropolitian University (TMU). Ryerson, an educational
administrator at Victoria College (now part of University of Toronto),
crafted arguments that were influential in the design of Canada's
residential school system and had been honored in higher educational
institutions in Toronto, most notably when TMU was still named Ryerson
University. These facts, widely known and largely uncontested, acquired
new significance when the erasure of Indigenous presence by educational
institutions was challenged by claims connecting these historical events
to the present. The destruction of the monument to Ryerson soon
followed.

The forgetting of the colonial divisions of humanity through which
official history is often constituted is not only belied by the
historical records of state administrative systems that document and
manage the inequalities through which wealth, advantage, and movement
are regulated, but as Lisa Lowe\footnote{Lisa Lowe, \emph{The Intimacies
  of the Four Continents} (Durham, NC: Duke University Press, 2015).}
teaches us, also interrupts knowledge of how British and European empire
was constituted through relationships with Africa, Asia, and America
that are rarely documented in our field. It is in this context that
Innis's desire to contest the absorption of Canada into an
Anglo-American form of empire remains valuable, not as an argument for
cultural nationalism or claim that Canadians are unique in their
capacity to make American empire more perceptible, but as one way into
the wider histories of which our world is composed. As a communication
student in Ontario, Canada, I knew of Ryerson only as an eponym to a
press once sold to a US publisher in 1970, an event that crystalized
intense fears about US cultural imperialism, as the rising sales of
textbooks to schools, colleges, universities, and libraries had
encouraged American companies to enter the market.\footnote{George L.
  Parker, ``The Sale of Ryerson Press: The End of the Old Agency System
  and Conflicts over Domestic and Foreign Ownership in the Canadian
  Publishing Industry, 1970--1986,'' \emph{Papers of The Bibliographical
  Society of Canada} 40, no. 2 (2002): 18.} In the political ferment
around the sale, a literary figure draped a US flag on the Ryerson
statue to protest this apparent loss of intellectual and cultural
autonomy. Once mobilized as a symbol to defend education against
integration into empire, the statue fell when a fuller range of empire's
intimacies were connected through the history of education it
materialized.

\emph{Empire and Communications} remains valuable in such contexts
because its abstractions never double as claims to transcend empire,
never assume a position detached from its dynamics, and never suggest a
progressive arc to historical change. Yes, there are lots of other
histories that do this work too. Yet, its departure from form
illustrated how the material conditions for knowledge production were
themselves organized by administrative structures that subsumed
communication and the forms of historical evidence that are
available.\footnote{Shirley Roburn, ``Innis and Environmental Politics:
  Practical Insight from the Yukon,'' in \emph{Harold Innis and the
  Canadian North: Appraisals and Contestations}, ed. William J. Buxton
  (Montreal: McGill-Queens University Press, 2013).}

On this approach, the ceaseless parade of details in the text is not
contained by any of the abstractions, arguments, or narratives one might
wish to read out of historical changes, but is simply an aggregation of
evidence pointing to the administration of communication that became
central to the durability of empires and our knowledge of them. What is
of crucial importance, to my mind, is the way this aggregation of
seemingly disparate details challenges prevailing forms of historical
reflection, then and now. Innis dissociates historical information from
the reflection and arguments of many of the sources that he consulted, a
technique that not only moved the wider administration of time and space
into the foreground of historical development, but did so as a politics
of knowledge perhaps more inclined to aggregate than hold apart the
intimacies generated by empire.




\section{Bibliography}\label{bibliography}

\begin{hangparas}{.25in}{1} 



Barney, Darin. ``Infrastructure and the Form of Politics.''
\emph{Canadian Journal of Communication} 46, no. 2 (2021): 225--46.

Beale, Alison. ``Book Review: Empire and Communications.''
\emph{Canadian Journal of Communication} 15, no. 1 (1990): 115--17.

Berland, Jody. \emph{North of Empire}. Durham, NC: Duke University
Press, 2009.

Blondheim, Menahem. ``Discovering `The Significance of Communication':
Harold Adams Innis as Social Constructivist.'' \emph{Canadian Journal of
Communication} 29, no. 2 (2004): 119--44.

Buck-Morss, Susan. ``Hegel and Haiti.'' \emph{Critical Inquiry} 26, no.
4 (2000): 821--65.

Havelock, Eric A. ``Harold Innis: The Philosophical Historian.''
\emph{ETC: A Review of General Semantics} 38, no. 3 (1981): 255--68.
http://www.jstor.org/stable/42575548.

Heyer, Paul. \emph{Harold Innis}. Lanham, MD: Rowman \& Littlefield,
2003.

Lowe, Lisa. \emph{The Intimacies of the Four Continents}. Durham, NC:
Duke University Press, 2015.

Parker, George L. ``The Sale of Ryerson Press: The End of the Old Agency
System and Conflicts over Domestic and Foreign Ownership in the Canadian
Publishing Industry, 1970--1986.'' \emph{Papers of The Bibliographical
Society of Canada} 40, no. 2 (2002): 7--56.
\url{https://doi.org/10.33137/pbsc.v40i2.18269}.

Roburn, Shirley. ``Innis and Environmental Politics: Practical Insight
from the Yukon.'' In \emph{Harold Innis and the Canadian North:
Appraisals and Contestations}, edited by William J. Buxton, 295--325.
Montreal: McGill-Queens University Press, 2013.

Stamm, Michael. \emph{Dead Tree Media: Manufacturing the Newspaper in
Twentieth-Century North America}. Baltimore: Johns Hopkins University
Press, 2018.

Tavarès, Pierre-Franklin. ``Hegel et l\textquotesingle abbé Grégoire:
question noire et révolution française.'' \emph{Annales Historiques de
la Révolution Française} 3, no. 4 (1993): 491--509.

Towns, Armond. ``The (Black) Elephant in the Room: McLuhan and the
Racial." \emph{Canadian Journal of Communication} 44, no. 4 (2019):
545--54. \url{https://doi.org/10.22230/cjc.2019v44n4a3721}.

Trouillot, Michel-Rolph. \emph{Silencing the Past: Power and the
Production of History}. Boston: Beacon Press, 1995.

Willems, Wendy. ``Provincializing Hegemonic Histories of Media and
Communication Studies: Toward a Genealogy of Epistemic Resistance in
Africa.'' \emph{Communication Theory} 24, no. 4 (2014): 415--34.
\url{https://doi.org/10.1111/comt.12043}.

Young, Liam Cole. ``The McLuhan-Innis Field: In Search of Media
Theory.'' \emph{Canadian Journal of Communication} 44, no. 4 (2019):
527­--44.



\end{hangparas}


\end{document}