% see the original template for more detail about bibliography, tables, etc: https://www.overleaf.com/latex/templates/handout-design-inspired-by-edward-tufte/dtsbhhkvghzz

\documentclass{tufte-handout}

%\geometry{showframe}% for debugging purposes -- displays the margins

\usepackage{amsmath}

\usepackage{hyperref}

\usepackage{fancyhdr}

\usepackage{hanging}

\hypersetup{colorlinks=true,allcolors=[RGB]{97,15,11}}

\fancyfoot[L]{\emph{History of Media Studies}, vol. 3, 2023}


% Set up the images/graphics package
\usepackage{graphicx}
\setkeys{Gin}{width=\linewidth,totalheight=\textheight,keepaspectratio}
\graphicspath{{graphics/}}

\title[Review of Geoghegan]{\emph{Code: From Information Theory to French Theory}} % longtitle shouldn't be necessary

% The following package makes prettier tables.  We're all about the bling!
\usepackage{booktabs}

% The units package provides nice, non-stacked fractions and better spacing
% for units.
\usepackage{units}

% The fancyvrb package lets us customize the formatting of verbatim
% environments.  We use a slightly smaller font.
\usepackage{fancyvrb}
\fvset{fontsize=\normalsize}

% Small sections of multiple columns
\usepackage{multicol}

% Provides paragraphs of dummy text
\usepackage{lipsum}

% These commands are used to pretty-print LaTeX commands
\newcommand{\doccmd}[1]{\texttt{\textbackslash#1}}% command name -- adds backslash automatically
\newcommand{\docopt}[1]{\ensuremath{\langle}\textrm{\textit{#1}}\ensuremath{\rangle}}% optional command argument
\newcommand{\docarg}[1]{\textrm{\textit{#1}}}% (required) command argument
\newenvironment{docspec}{\begin{quote}\noindent}{\end{quote}}% command specification environment
\newcommand{\docenv}[1]{\textsf{#1}}% environment name
\newcommand{\docpkg}[1]{\texttt{#1}}% package name
\newcommand{\doccls}[1]{\texttt{#1}}% document class name
\newcommand{\docclsopt}[1]{\texttt{#1}}% document class option name


\begin{document}

\begin{titlepage}

\begin{fullwidth}
\noindent\LARGE\emph{Book review
} \hspace{88mm}\includegraphics[height=1cm]{logo3.png}\\
\noindent\hrulefill\\
\vspace*{1em}
\noindent{\Huge{\emph{Code: From Information Theory to French\\\noindent Theory}\par}}

\vspace*{1.5em}

\noindent\LARGE{reviewed by Ido Ramati}\par\marginnote{\emph{Code: From Information Theory to French Theory}, reviewed by Ido Ramati, \emph{History of Media Studies} 3 (2023), \href{https://doi.org/10.32376/d895a0ea.a9c39117}{https://doi.org/ 10.32376/d895a0ea.a9c39117}. \vspace*{0.75em}}
\vspace*{0.5em}
\noindent{{\large\emph{Hebrew University of Jerusalem}, \href{mailto:ido.ramati@mail.huji.ac.il}{ido.ramati@mail.huji.ac.il}\par}} \marginnote{\href{https://creativecommons.org/licenses/by-nc/4.0/}{\includegraphics[height=0.5cm]{by-nc.png}}}

% \vspace*{0.75em} % second author

% \noindent{\LARGE{<<author 2 name>>}\par}
% \vspace*{0.5em}
% \noindent{{\large\emph{<<author 2 affiliation>>}, \href{mailto:<<author 2 email>>}{<<author 2 email>>}\par}}

% \vspace*{0.75em} % third author

% \noindent{\LARGE{<<author 3 name>>}\par}
% \vspace*{0.5em}
% \noindent{{\large\emph{<<author 3 affiliation>>}, \href{mailto:<<author 3 email>>}{<<author 3 email>>}\par}}

\end{fullwidth}

\vspace*{1em}


\noindent Bernard Dionysius Geoghegan. \emph{Code: From Information Theory to French
Theory}. 272 pp., 47 figs., index. Durham, NC: Duke
University Press, 2023. \$26.95 (paper).

\vspace{2em}

\noindent\emph{Code} traces the ways in which the notion of code grew to be a
remarkably influential concept, both across interrelated theories and
schools---mainly cybernetics, information theory, and structuralist and
post-structuralist thought---and outside of academic research.
\emph{Code} forefronts ``code'' as the leading component in a range of
cross-disciplinary thinking in terms of communication and with
communicative models. Following the various paths from which the concept
of code emerged and through which it diffused, the book uncovers the
political motivations that gave rise to cybernetics, and the ways in
which the cybernetic perspective has influenced our understanding of
cultural and social phenomena, from speech, literature, and hermeneutics
to kinship and illness. In \emph{Code}, Geoghegan suggests a broad yet
nuanced narrative which unearths the crises out of which cybernetics and
information theory emerged---colonial control, World War II and the
Holocaust---discovering in them the human-machine and human-human
entanglements that accompanied the rise of these theories and eventually
also became their legacies in subsequent paradigms. In this, \emph{Code}
presents a compelling case which exemplifies how the dissemination of
knowledge and the circulation of ideas within scientific structures is
always already committed to political, social, and cultural impetuses.



\enlargethispage{2\baselineskip}

\vspace*{2em}

\noindent{\emph{History of Media Studies}, vol. 3, 2023}


 \end{titlepage}

% \vspace*{2em} | to use if abstract spills over


\noindent \emph{Code} joins previous seminal studies that narrated the rise and
influence of cybernetics and information theory---among them,
Galison,\footnote{Peter Galison, ``The Ontology of the Enemy: Norbert
  Wiener and the Cybernetic Vision,''~\emph{Critical Inquiry}~21, no. 1
  (1994).} Liu\footnote{Lydia H. Liu, \emph{The Freudian Robot: Digital
  Media and the Future of the Unconscious} (Chicago: The University of
  Chicago Press, 2010).} and Medina,\footnote{Eden Medina,
  \emph{Cybernetic Revolutionaries: Technology and Politics in
  Allende\textquotesingle s Chile} (Cambridge, MA: MIT Press, 2011).} to
mention only a few---and adds to them a reading of this story from
within: what led to the cybernetic way of thinking, what were its
sources, and how these eventually shaped subsequent theories and schools
of thought. This trajectory follows the trans-Atlantic axis through
which cybernetic ideas were diffused and transmitted synchronically as
well as diachronically, beginning with its sources in European and North
American thought and research which preceded the famous Macy conferences
of the late 1940s and early 1950s, and following the dispersal of these
ideas and methodologies across several research fields, institutions,
and policy-making agencies. By looking at the before and after of
cybernetics, the book suggests a longer historical perspective on the
enduring impact of the cybernetic prism and its principal ideas, but
also highlights the political, racial, and other social themes that
continued to shape the production of knowledge, up to the present day.
It is possible to understand this move offered in the book in cybernetic
terms: The immigration of ideas, and people, across that trans-Atlantic
axis and back, and their continuing and powerful impact on Western
thought, may be described as a cybernetic loop of influence that
generates feedback, intersections, and even metaphorical noises which
nourish both theory and practice.

Although the boundaries of the category ``cybernetics'' are
questionable, the book wisely uses the phrase \emph{cybernetic
apparatus} to include in its analyses ``the network of institutions,
methods, techniques, researchers, conferences, instruments,
laboratories, clinics, infrastructure, and jargon mobilized around
cybernetic themes'' (13). This allows the discussion to deal with both
human and non-human contributions to the rise of information theory and
to investigate the distribution and transfiguration of complex ideas and
models. For example, the notion of \emph{technocracy}---which is
commonly associated with a generally inflexible and insensitive
adherence to mechanisms or systems---serves in \emph{Code} to reveal a
much more nuanced depiction of thinking with and through technologies,
which is inherently ``politically motivated'' even when it purports to
be ``a supposedly nonpolitical and neutral tool of governance'' (15).
Carefully grappling with such key terms and phrases, \emph{Code} unpacks
the ways in which cybernetics is political down to its core and from its
very beginnings. In my mind, this argument is of great importance to the
history of media studies for two main reasons. First, it restates the
central role cybernetics and its promotion of thinking technologically
had in the emergence of key ideas across almost all the disciplines in
the human sciences. Secondly, it simultaneously refutes the widespread
association of cybernetics and information theory with the supposed
``cold gaze'' of science, informed by the quasi-neutrality of machines
and technical models. In this sense, it also provides theoretical depth,
and a longer historical perspective on how media are inherently
political machines, to research on contemporary information
technologies---which, at their most basic levels, rely on cybernetic
models.

The analysis suggested in \emph{Code} follows the transfiguration of
cybernetic terms and keywords---such as code, communication, computing,
feedback, and control---as well as the influence of the cybernetic
mindset on ways of thinking in patterns and structures that became the
pillars of theories such as structuralism and post-structuralism. The
wide range covered by the book, in jointly discussing the parallel
developments in fields such as anthropology, sociology, psychology
(including psychoanalysis), semiotics, and literary theory, reflects the
extensive and enduring influence of cybernetics. The book depicts the
travels of ideas and people, and delves into famous scholarly
cooperations and the mutual influences they absorbed: from Weaver's
co-optation and ``translation'' into socially meaningful concepts of
Shannon's servomechanical designs that eventually became the influential
mathematical model of communication, through how Mead and Bateson
together and separately developed their methodologies of pattern
discovering, continuing with how Jakobson and Lévi-Strauss inspired each
other in formulating their own versions of structural analysis, and
following with how Lacan (and his disciples) and Barthes implemented
some of these ideas, while transfiguring them, in their theories and
research.

Here lies, to my mind, the main take for the history of media studies
and for the historiography of science in general. The depiction provided
by \emph{Code} sheds light not only on the nature of direct influence
and mutual inspiration but also on how free and creative understandings
as well as misuses of ideas---which might be considered, in cybernetic
terms, as metaphorical ``noise'' as I suggested above---serve as
productive and thought-provoking maneuvers that eventually lead to the
development of theories and methodologies. And this is another level on
which \emph{Code} embodies and reflects the principles of its own
subject matter, because it exemplifies how the communication and
development of knowledge, while indeed depending on systems and
structures (e.g., grants offered by institutions, conferences bringing
researchers together with similar interests), is more messy than
structurally organized. Examples of this may be found in the inner
feedback-loops of influence and translation of knowledge which gave rise
to post-structuralism, out of and against structuralism, and in how
``French Theory'' (as it has come to be known in the English-speaking
world) actually emerged from the cybernetic influence on Continental
thought. In a sense, the book suggests, post-structuralism was also
post-cybernetic, harnessing the logic of the machine to critique
communication systems (for example, in Kristeva's or Baudrillard's
philosophies).

This last point opens a new trajectory for considering contemporary
algorithmic culture, and especially its current, still-emerging
manifestations in generative technologies. These are based on machine
learning and natural language processing models, which are initially,
again, cybernetic systems. What will ``code'' become in the age of
autonomic coding AI agents? What will be the political and social stakes
of such technologies? While contemporary research has already started to
provide some preliminary answers to these questions, the historical
analysis provided in \emph{Code} also suggests that for future
historians of media studies, the retrospective description of current
technological developments would have to consider the complexity of
beyond-human communicative concepts.




\section{Bibliography}\label{bibliography}

\begin{hangparas}{.25in}{1} 



Galison, Peter. ``The Ontology of the Enemy: Norbert Wiener and the
Cybernetic Vision.''~\emph{Critical Inquiry}~21, no. 1 (1994): 228--66.

Liu, Lydia H. \emph{The Freudian Robot: Digital Media and the Future of
the Unconscious}. Chicago: University of Chicago Press, 2010.

Medina, Eden. \emph{Cybernetic Revolutionaries Technology and Politics
in Allende\textquotesingle s Chile.} Cambridge, MA: MIT Press, 2011.



\end{hangparas}


\end{document}