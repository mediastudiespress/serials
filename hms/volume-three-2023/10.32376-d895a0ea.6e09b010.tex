% see the original template for more detail about bibliography, tables, etc: https://www.overleaf.com/latex/templates/handout-design-inspired-by-edward-tufte/dtsbhhkvghzz

\documentclass{tufte-handout}

%\geometry{showframe}% for debugging purposes -- displays the margins

\usepackage{amsmath}

\usepackage{hyperref}

\usepackage{fancyhdr}

\usepackage{hanging}

\hypersetup{colorlinks=true,allcolors=[RGB]{97,15,11}}

\fancyfoot[L]{\emph{History of Media Studies}, vol. 3, 2023}

\usepackage{tabu}
\usepackage{longtable}[=v4.13]

% Set up the images/graphics package
\usepackage{graphicx}
\setkeys{Gin}{width=\linewidth,totalheight=\textheight,keepaspectratio}
\graphicspath{{graphics/}}

\title[From Victims to Economic Assets]{From Victims to Economic Assets: Training Women in an Emerging Digital Society During the Late 1970s to the Mid-1990s} % longtitle shouldn't be necessary

% The following package makes prettier tables.  We're all about the bling!
\usepackage{booktabs}

% The units package provides nice, non-stacked fractions and better spacing
% for units.
\usepackage{units}

% The fancyvrb package lets us customize the formatting of verbatim
% environments.  We use a slightly smaller font.
\usepackage{fancyvrb}
\fvset{fontsize=\normalsize}

% Small sections of multiple columns
\usepackage{multicol}

% Provides paragraphs of dummy text
\usepackage{lipsum}

% These commands are used to pretty-print LaTeX commands
\newcommand{\doccmd}[1]{\texttt{\textbackslash#1}}% command name -- adds backslash automatically
\newcommand{\docopt}[1]{\ensuremath{\langle}\textrm{\textit{#1}}\ensuremath{\rangle}}% optional command argument
\newcommand{\docarg}[1]{\textrm{\textit{#1}}}% (required) command argument
\newenvironment{docspec}{\begin{quote}\noindent}{\end{quote}}% command specification environment
\newcommand{\docenv}[1]{\textsf{#1}}% environment name
\newcommand{\docpkg}[1]{\texttt{#1}}% package name
\newcommand{\doccls}[1]{\texttt{#1}}% document class name
\newcommand{\docclsopt}[1]{\texttt{#1}}% document class option name


\begin{document}

\begin{titlepage}

\begin{fullwidth}
\noindent\LARGE\emph{Article
} \hspace{99mm}\includegraphics[height=1cm]{logo3.png}\\
\noindent\hrulefill\\
\vspace*{1em}
\noindent{\Huge{From Victims to Economic Assets: Training Women in an Emerging Digital Society\\\noindent During the Late 1970s to the Mid-1990s\par}}

\vspace*{1.5em}

\noindent\LARGE{Rosalía Guerrero Cantarell} \href{https://orcid.org/0000-0003-1071-1941}{\includegraphics[height=0.5cm]{orcid.png}}\par\marginnote{\emph{Rosalía Guerrero Cantarell, Carmen Flury, and Michael Geiss, ``From Victims to Economic Assets: Training Women in an Emerging Digital Society During the Late 1970s to the Mid--1990s,'' \emph{History of Media Studies} 3 (2023), \href{https://doi.org/10.32376/d895a0ea.6e09b010}{https://doi.org/ 10.32376/d895a0ea.6e09b010}.} \vspace*{0.75em}}
\vspace*{0.5em}
\noindent{{\large\emph{Zurich University of Teacher Education},\\\noindent \href{mailto:rosalia.guerrero@ekhist.uu.se}{rosalia.guerrero@ekhist.uu.se}\par}} \marginnote{\href{https://creativecommons.org/licenses/by-nc/4.0/}{\includegraphics[height=0.5cm]{by-nc.png}}}

\vspace*{0.75em} 

\noindent{\LARGE{Carmen Flury} \href{https://orcid.org/0000-0002-3759-6595}{\includegraphics[height=0.5cm]{orcid.png}}\par}
\vspace*{0.5em}
\noindent{{\large\emph{Zurich University of Teacher Education},\\\noindent \href{mailto:carmen.flury@phzh.ch}{carmen.flury@phzh.ch}\par}}

\vspace*{0.75em} % third author

\noindent{\LARGE{Michael Geiss}\par}
\vspace*{0.5em}
\noindent{{\large\emph{Zurich University of Teacher Education},\\\noindent \href{mailto:michael.geiss@phzh.ch}{michael.geiss@phzh.ch}\par}}

\end{fullwidth}

\vspace*{1em}


\hypertarget{abstract}{%
\section{Abstract}\label{abstract}}

The European Community (EC), afraid of lagging behind the United States
and Japan in the technology race, developed a series of strategies in
the 1980s and 1990s to enhance their position in the field of
information and telecommunication technologies (ICT). One key strategy
was the training and retraining of citizens to deal with the shortage of
a highly skilled labor force. At the same time, women's issues were
being raised as part of the EC's political agenda, with a particular
emphasis on the effects of technological change on the employment of
women. We claim that despite the advantageous conditions for promoting
women's advanced education in ICT, the EC's strategy was to promote only
basic computer literacy, following a dis-

\enlargethispage{2\baselineskip}

\vspace*{2em}

\noindent{\emph{History of Media Studies}, vol. 3, 2023}

 \end{titlepage}


\noindent course that victimized women,
due to their alleged lack of skills and flexibility. Although by the
late 1980s, the view of women as
constituting potential economic assets
in the EC's economy became more prominent, the concrete measures for
women's education and training did not substantially change and remained
scantily funded and short-sighted throughout the whole period.


\hypertarget{resumen}{%
\section{Resumen}\label{resumen}}

La Comunidad Europea (EC), temerosa de quedarse rezagada con respecto a Estados Unidos y Japón en la carrera tecnológica, puso en marcha una serie de estrategias en las décadas de 1980 y 1990 para mejorar su posición en el campo de la tecnología de la información y las telecomunicaciones (ICT). Una de estas estrategias fue la de formar y actualizar la formación de los ciudadanos para compensar la falta de mano de obra altamente cualificada. Simultáneamente, en esta época, los problemas de las mujeres empezaban a formar parte de la agenda política de la EC, con un énfasis especial en los efectos que el cambio tecnológico implicaba para la ocupación femenina. Entendemos que, pese a las condiciones ideales que se dieron para favorecer la formación de mujeres en ICT, la estrategia de la EC fue la de fomentar exclusivamente competencias básicas del uso de ordenadores, así como desarrollar un discurso que victimizaba a las mujeres por unas supuestas carencias en habilidades y flexibilidad para adaptarse a la transición tecnológica. Aunque hacia el final de los 1980s la concepción de las mujeres como activos para la economía de la EC se afianzó, las medidas concretas para la educación y la formación femenina no solo no cambiaron de manera sustancial, sino que, durante toda la etapa, se mantuvieron cortoplacistas e insuficientemente financiadas.

\vspace*{2em}

\newthought{The European Community} (EC) regarded the advent of new information
technologies in the 1970s as an urgent matter, involving complex
economic, industrial and technical questions. Concerning the workforce,
the EC identified two main challenges: the devaluation of existing
qualifications and the shortage of highly-qualified specialists in
Information and Communications Technology (ICT).\footnote{Commission of
  the European Communities, ``The Changing Nature of Employment: New
  Forms, New Areas,'' \emph{Social Europe} 1 (1988).} In light of the
then-dominant Knowledge-Based Economy (KBE) paradigm that incorporated
ideas, such as level of education, skills, and human capital,\footnote{Sami
  Moisio, \emph{Geopolitics of the Knowledge-Based Economy} (New York:
  Routledge, 2018), 15.} the EC redefined education and training to
match the demand of transferable and specific skills required in the
economy. European policies, following a human capital approach, were
framed as a strategy to supply the labor market with an adequate stock
of human capital to meet the demands of the KBE.\footnote{Manfred
  Wallenborn, ``Vocational Education and Training and Human Capital
  Development: Current Practice and Future Options,'' \emph{European
  Journal of Education} 45, no. 2 (2010).}

In a technological society, as suggested by sociologist Andrew Barry,
``technical change {[}is{]} the model for political intervention'' and
``specific technologies dominate our sense of the kinds of problems that
government and policies must address, and the solutions that we must
adopt.''\footnote{Andrew Barry, \emph{Political Machines: Governing a
  Technological Society} (London: Athlone Press, 2001), 2.} This
explains the emergence of European technology policies in the 1980s,
aimed at creating a European system of innovation. Policymakers assumed
that technology was related to knowledge and, therefore, depended on a
mix of skills and capabilities of people, as well as institutions.

While research institutions and technical facilities were
essential,\footnote{John Peterson and Margaret Sharp, \emph{Technology
  Policy in the European Union} (London: Macmillan, 1998), 49--50.} the
development of human capital was equally important. The new economic
structure required a pool of individuals ready to undergo training to
master the rapidly developing computer technologies.

Concurrently, an international gender equality movement was becoming
established.\footnote{See Angela Glasner, ``Gender and Europe: Cultural
  and Structural Impediments to Change,'' in \emph{Social Europe}, ed.
  Joe Bailey (New York: Longman, 1991).} Equal opportunities in
education and work constituted an urgent demand for groups working on
women's issues, which led to the development of equality policies in
these areas, both at national and European levels.\footnote{See Mary Ann
  Danowitz Sagaria, \emph{Women, Universities, and Change: Gender
  Equality in the European Union and the United States} (New York:
  Palgrave Macmillan, 2007).}

The timing seemed perfect, and the conditions ripe, for creating
suitable educational and training policies that targeted women's
inequality and addressed the shortage of highly skilled workers in the
area of ICT. According to the human capital paradigm, targeting women's
education and training in new technologies might appear to be an
unequivocal choice. However, we claim that the EC repeatedly pigeonholed
women as a socially-excluded group, particularly prone to unemployment
due to their poor qualifications and labor market behavior.

In the early years of the microchip revolution, both women and men faced
the consequences of inadequate training in new ICT. However, women often
encountered the additional challenge of not being recognized as
potential highly skilled workers. This view had far-reaching
consequences for the content and scope of education and training
initiatives. This historical inconsistency is at the core of this
article.

We seek to understand the role that the EC assigned women in building a
European technological society and how this role was encouraged through
training and education policies. In the following sections, we add
nuance to Jacky Brine's assumption that the EC's patriarchal approach
trumped its economic rationale.\footnote{Jacky Brine, ``Equal
  Opportunities and the European Social Fund: Discourse and Practice,''
  \emph{Gender and Education} 7, no.1 (1995); Jacky Brine,
  \emph{UnderEducating Women: Globalizing Inequality} (Philadelphia:
  Open University Press, 1999).} We show, instead, how in the 1980s and
early 1990s, a gendered interpretation of the labor market permeated the
EC's policy decisions. This interpretation was based on assumptions
regarding women's interests, skills, and capabilities. This EC's bias
resulted in education and training measures that endorsed women's
subordinated economic role in the advent of the so-called ICT
revolution.

This study centers on the era spanning roughly from 1970 onwards, during
which Western governments and the EC acknowledged the developmental
potential of emerging information technologies. This period also saw a
growing emphasis on citizen education and training for effective
utilization and innovation of these technologies. The period under study
ends with the late 1990s and the dot-com boom, which necessitated novel
competencies and enterprise models and thus marked a shift in European
strategy towards training and education. This article is based on an
analysis of the EC's guidelines, programs, and policies on women's
education and training in ICT. The central documents consist of the
social policy of the Commission of the European Communities (CEC),
including equal opportunities and social action programs, documentation
on initiatives, such as IRIS and NOW, and the \emph{Women of Europe
Newsletter}. Additionally, we use reports, Council resolutions,
conferences, Commission recommendations, EC publications, and archival
documents of the EC's Committee on Women's Rights and Technological
Change from the Historical Archives of the European Union in Florence.

While none of the action programs and initiatives dealt entirely with
education and training in technology for women, they all considered the
question to some extent. In all these documents, we analysed how the
problem of women and new technologies was formulated, the vision of the
technology society they advocated, and the strategies proposed to
achieve this vision. From the series of documents used in this study,
the action programs and targeted initiatives (IRIS, NOW) are the most
concrete in terms of their stance on the question of women's education
and training in ICT, as they present a strategy of action---with varying
levels of implementation in the member states. Although these do not
constitute binding legislation, they are considered pre-law instruments
and agenda setters.\footnote{See Linda Senden, \emph{Soft Law in
  European Community Law} (Oxford: Hart, 2004), 128.}

Although the EC is composed of several bodies with various prerogatives
and interests, in this article, we treat the EC as a cohesive entity.
While we refer to specific actors within the EC, such as the European
Social Fund, the EC Commission, or the Centre for the Development of
Vocational Training (CEDEFOP), we do not follow specific interest groups
and individual behaviour. Our focus is on the policy outcomes---that is,
the final programs and policies (whether legally binding or not) that
the EC introduced. By analyzing how action programs and EC initiatives
present the role of women in a technological society, we aim to bring
attention to a paradoxical issue: why, in the age of human capital, the
pursuit of efficiency in education and training, as well as the untapped
potential of women in the ICT field, seem to have been disregarded.

We draw on Teresa Rees's model of gender equality, both to characterize
the EC guidelines and strategies in women's education training in the
field of ICT and to periodize them. Rees identifies an equal treatment
model during the 1970s, a positive action model in the 1980s, and a
third model based on the principle of mainstreaming beginning in the
mid-1990s.\footnote{Teresa Rees, \emph{Mainstreaming Equality in the
  European Union: Education, Training and Labour Market Policies}
  (London: Routledge, 1998).} However, the period covered by our study
ends before the mainstreaming model was properly established in
education and research policy. Apart from identifying the gender
equality approach to which the studied strategies belong, we examine the
content of the policies and measures regarding women's training and
education in ICT and locate the shifts therein.

\hypertarget{state-of-the-research}{%
\section{State of the Research}\label{state-of-the-research}}

While European education and technology policies have been the subject
of intense research, gender issues have been largely
overlooked.\footnote{Ilkka Kauppinen, ``The European Round Table of
  Industrialists and the Restructuring of European Higher Education,''
  \emph{Globalisation, Societies and Education} 12, no. 4 (2014); Marina
  Cino Pagliarello, \emph{Ideas and European Education Policy,
  1973--2020: Constructing the}} Likewise, historical research on European gender policies and
gender mainstreaming programs has paid little attention to technological
change and the European competitiveness agenda. The present contribution
fills this gap by bringing these two perspectives together.

The\marginnote{\emph{Europe of Knowledge?} (London: Palgrave
  Macmillan, 2022); Simone Paoli, ``The European Community and the Rise
  of a New Educational Order (1976--1986),'' in \emph{Contesting
  Deregulation: Debates, Practices and Developments in the West since
  the 1970s}, ed. Knud Andersen and Stefan Müller (Oxford: Berghahn,
  2017).} shift of the Commission of the European Communities towards a new,
interventionist technology policy in the mid-1970s has already been
studied to a great extent.\footnote{Laurent Warlouzet, \emph{Governing
  Europe in a Globalizing World: Neoliberalism and its Alternatives
  Following the 1973 Oil Crisis} (London: Routledge, 2018).} Historical
and comparative research have considered both earlier and more recent
action programs.\footnote{Laurent Warlouzet, ``Towards a European
  Industrial Policy? The European Economic Community (EEC) Debates,
  1957--1975,'' in \emph{Industrial Policy in Europe after 1945: Wealth,
  Power and Economic Development in the Cold War}, ed. Christian Grabas
  and Alexander Nützenadel (London: Palgrave Macmillan, 2014).} In this
context, Bornschier and Parker have shown how the attempt to cope
politically with the advent of the microchip produced a new governance
model, which ultimately resulted in a Single Market policy and expanded
the competencies of European institutions.\footnote{Volker
  Bornschier,``Western Europe's Move toward Political Union,'' in
  \emph{State-Building in Europe: The Revitalization of Western European
  Integration}, ed. Volker Bornschier (Cambridge: Cambridge University
  Press, 2000); Simon Parker, ``Esprit and Technology Corporatism,'' in
  \emph{State-Building in Europe: The Revitalization of Western European
  Integration}, ed. Volker Bornschier (Cambridge: Cambridge University
  Press, 2000).}

The new technology policy was accompanied by numerous initiatives to
reform vocational education, further training, and higher education.
Comparative research has taken this commitment into account and
systematically examined the European Commission's ``action program
approach'' in educational governance.\textsuperscript{15} The new European technology policy not
only targeted research and development, but also had an increasingly
important educational component.\textsuperscript{16}

However, the attempt to create a highly qualified workforce and thus
increase the competitiveness of the member states was only one side of
the story. At the same time, digital change brought into focus different
groups that were considered especially vulnerable, including women.
Jacky Brine has carried out several studies on the history of Europeanmeasures. In relation to new technologies, she claims that the\marginnote{\textsuperscript{15} Hubert Ertl,
  \emph{European Union Initiatives in Education and Vocational Training:
  The Development and Impact of the Programme Approach} (Paderborn:
  Paderborn University, 2002).}
gender\marginnote{\textsuperscript{16}\setcounter{footnote}{16} Alan Clarke,
  ``Competitiveness, Technological Innovation and the Challenge to
  Europe,'' in \emph{The Learning Society: Challenges and Trends}, ed.
  Peter Raggatt, Richard Edwards, and Nick Small (London: Routledge,
  1996); Anne Corbett, \emph{Universities and the Europe of Knowledge:
  Ideas, Institutions and Policy Entrepreneurship in European Union
  Higher Education Policy, 1955--2005} (London: Palgrave Macmillan,
  2005).} 
interest of the EC in training women in this field declined from the
1980s onwards. She argues, moreover, that regardless of the form of the
initiatives, the hierarchy of male power remained, as women were
encouraged into male-dominated professions that were in decline, rather
than those of the future.\footnote{Jacky Brine, ``The European Social
  Fund and the Vocational Training of Unemployed Women: Questions of
  Gendering and Re-Gendering,'' \emph{Gender and Education} 4, no. 1--2
  (1992).}

Brine argues repeatedly that the EC had a primarily economic incentive
for the creation and implementation of measures to support women's
training and retraining.\footnote{Brine, ``The European Social Fund'';
  Jacky Brine, ``European Education and Training Policy for
  Under-Educated Unemployed People,'' \emph{International Studies in
  Sociology of Education} 7, no. 2 (1997); Jacky Brine, ``The European
  Union's Discourse of `Equality' and its Education and Training Policy
  within the Post-Compulsory Sector,'' \emph{Journal of Education
  Policy} 13, no. 1 (1998).} She shows how the EC used a discursive
economic rationale when establishing policies or advice related to
women's training. However, she finds it surprising that although the
economic rationale was predominant, and there were liberal economic
reasons to promote women's training, the rigid structure of the EC seems
to have prevented an approach that actually benefitted women.\footnote{Brine,
  \emph{UnderEducating.}} The result, for Brine, was that there was a
mismatch between the areas lacking a skilled workforce and the areas in
which women were encouraged to seek training.\footnote{Brine, ``Equal
  Opportunities.''} Thus, the training which women underwent did not
lead to employment, despite this being the main purpose of the European
Social Fund (ESF) programs.

In this article, we propose a more nuanced perspective to the argument
that the inflexible gendered structure of the EC has superseded economic
justification. Rather, we suggest that a gendered understanding of the
workings of the labor market was embedded in the policy-making process.
Perceptions of women's inherent abilities, skills, and interests were
incorporated into the understanding of the needs of the economy. We
argue that in the specific measures proposed to support women's training
and education in new technologies, women were earmarked as a useful
reserve of low-skilled workers, equipped for and willing to use the
technological innovations introduced into traditional workplaces, rather
than as potential highly skilled innovators and creators, driving
technological development.

\hypertarget{the-changing-role-of-the-ec-background}{%
\section{The Changing Role of the EC:
Background}\label{the-changing-role-of-the-ec-background}}

The EC's initial focus in the 1960s was on trade and increasing
prosperity among its member states, but in the 1970s, global changes led
to economic crises and a greater need for political cooperation. The
1980s saw a rise in youth unemployment, prompting a new focus on
education and training as a strategy for economic recovery.\footnote{Michel-Pierre
  Chélini and Laurent Warlouzet, eds., \emph{Calmer les prix} {[}Slowing
  down prices{]} (Paris: Presses de Sciences Po, 2017).} At the same
time, the European Court of Justice, in a series of rulings based on
Article 128 of the Treaty of Rome, extended the interpretation of the
term ``vocational,'' thereby giving the Commission the prerogative to
assume competence in the field of higher and continuing
education.\footnote{Hubert Ertl, ``European Union Programmes for
  Education and Vocational Training: Development and Impact,''
  \emph{SKOPE Research Paper}, no. 42 (2003): 8.} This enabled EC
institutions to adopt legislation in this field that would be legally
binding on the member states. The European Court of Justice's rulings
based on Article 128 subsequently provided the legal basis for a series
of programs launched by the Commission and the Council from the
mid-1980s onwards.\footnote{Ertl, ``European Union Initiatives,'' 20.}

The decision to achieve an economic convergence, which led to the
realization of the Economic and Monetary Union (EMU) in 1988, was
preceded by the first enlargement of the EEC to include Denmark,
Ireland, and the UK, and to the development of common policies.
\footnote{Emmanuel Mourlon-Druol, ``The EMS as an External Anchor in
  Inflation-Prone Countries,'' in \emph{Calmer les Prix} {[}Slowing down
  prices{]}, ed. Michel-Pierre Chélini (Paris: Presses de Sciences Po,
  2017), 280.} However, limitations on the EU's role in education meant
that it was predominantly left in the hands of national actors. The
adoption of the Maastricht Treaty in 1992 established the EU and widened
the scope of the ESF to include education and vocational
training.\footnote{The Maastricht Treaty was signed by Belgium, Denmark,
  France, Germany, Greece, Ireland, Italy, Luxembourg, Portugal, Spain,
  the Netherlands, and the United Kingdom. Nick Adnett and Stephen
  Hardy, \emph{The European Social Model: Modernisation or Evolution?}
  (Cheltenham: Edward Elgar Publishing, 2005), 4.} The education aims of
the EU were restricted to developing a European dimension, promoting
mobility and cooperation between educational establishments, exchanging
information, and encouraging distance learning.\footnote{Brad Blitz,
  ``From Monnet to Delors: Educational Cooperation in the European
  Union,'' \emph{Contemporary European History} 12, no. 2 (2003): 210.}
The education of women in technology was affected by this limitation in
that there was no framework for supranational cooperation on the issue
and no consensus on how to approach it. While some national programs
were implemented, there was no broader cooperation on the matter.

\hypertarget{women-and-the-new-information-and-communications-technologies}{%
\section{Women and the New Information and Communications\\\noindent
Technologies}\label{women-and-the-new-information-and-communications-technologies}}

The phenomenon that this article examines is related to the gendering of
ICT occupations. The ways in which the EC dealt with the training and
education of women in computer technology are illustrative of a general
tendency that Cynthia Cockburn identified when she studied the gendered
nature of technology in the 1980s. Cockburn argues that women have been
excluded from technical occupations and judged as technically
incompetent.\footnote{Cynthia Cockburn, ``The Material of Male Power,''
  \emph{Feminist Review}, no. 9 (1981).} In this study, we corroborate
this tendency in the overall European strategy of ICT training.

In the early years of computing, women had an active role. The
best-known cases are the United Kingdom and the United States, where
research has found that women were an integral part of the development
of the computer industry, as they engaged in processing and analyzing
data, using mechanical calculators initially, and were part of the
building and programming of electronic computers at a later stage.\footnote{Janet Abbate, ``Women and Gender and the History of
  Computing,'' \emph{IEEE Annals of the History of Computing} 24, no. 4
  (October--December 2003); Paul E. Ceruzzi, ``When Computers Were
  Human,'' \emph{Annals of the History of}}

Some of these women had a college education, and the majority had
technical training. However, by the 1960s, women were increasingly
regarded as unsuitable for programming work and were not given the
opportunities to\marginnote{\emph{Computing} 13, no. 3 (1991);
  David Alan Grier, ``Human Computers: The First Pioneers of the
  Information Age,'' \emph{Endeavour} 25, no. 1 (2001); Mar Hicks,
  \emph{Programmed Inequality: How Britain} \emph{Discarded Women
  Technologists and Lost Its Edge in Computing} (Cambridge: MIT Press,
  2017); Jennifer S. Light, ``When Computers Were Women,''
  \emph{Technology and Culture} 40, no. 3 (July 1999).} occupy positions that they would be the most qualified
to perform; instead, they were given dead-end jobs with low salaries and
little influence over their tasks.\footnote{Hicks, \emph{Programmed
  Inequality}.} By the late 1970s and 1980s, when the need for computer
education and training for European citizens was raised, the existing
expertise and the potential of women were not considered in the
development of education and training programs in ICT.

This oversight is surprising since this was a period when gender issues
were voiced and discussed worldwide for the first time in history. The
International Women's Year and the United Nations Conference in Mexico
City in 1975 put gender equality and women's issues on the radar
internationally. The period 1975--1985 was subsequently designated as
the United Nations Decade for Women, which stimulated policymaking in
gender equality worldwide, not least in the EC. The Treaty of Rome in
1957, with its article 119, establishing sex equality, had given the EC
the prerogative to deal with issues of gender equality.

Nonetheless, the first European legislation targeting women only came
into force in the 1970s, guaranteeing equal pay, equal treatment, and
equal welfare.\footnote{Wendy Stokes, \emph{Women in Contemporary
  Politics} (Cambridge: Polity, 2005), 219.} The principle of equal
treatment for men and women applied to the areas of employment,
vocational training, and working conditions.\textsuperscript{31} The first parliamentary body specifically focused on women
was an ad hoc\marginnote{\textsuperscript{31}\setcounter{footnote}{31} ``Council
  Directive 76/207/EEC on the Implementation of the Principle of Equal
  Treatment for Men and Women as Regards Access to Employment,
  Vocational Training and Promotion, and Working Conditions,''
  \emph{Official Journal of the European Communities} L 39/40 (February
  9, 1976).} committee, created in 1979, to prepare for a debate on the
situation of women. This committee became the Committee of Enquiry into
the Situation of Women in Europe, and in 1984, the Committee on Women's
Rights. The debate waged in these committees was the basis of the
recommendations integrated into the first Community Action Programme for
Equal Opportunities in the early 1980s.\footnote{Stokes, \emph{Women in
  Contemporary Politics}, 217.}

Two EC bodies were particularly relevant in the promotion of measures in
training and education in new technologies targeting women: the ESF and
the CEDEFOP. The former was established as an employment policy
instrument aimed at supporting EC workers by funding training programs,
among other strategies.\footnote{Brine, ``Equal Opportunities.''} Women
were recognized as an eligible group for assistance in 1971.\footnote{CEC,
  ``Community Initiatives on Vocational Training for Women,''
  \emph{Social Europe} 1 (1988): 27; Brine, ``The European Social
  Fund.''} However, it was not until 1978 that the ESF began to fund
projects for the education and training of women over twenty-five. In
1980, the ESF funded training programs, based on two priorities:
projects to train women in areas where they had been underrepresented
and projects to help women become more qualified. The initial priority
projects were all funded, amounting to 14 million ECU, but the funds for
the second priority projects were cut to 7.1 million ECU. The funded
projects benefitted a total of 10,577 women from seven
countries.\footnote{CEC, ``The Changing European Communities,''
  \emph{Women of Europe} 17 (November--December 1980): 49.}

The CEDEFOP began operations in 1977 as a decentralized agency in charge
of supporting the development and implementation of European vocational
education and training policies. The agency's priorities were the
training and employment of young people, vocational training for women,
the establishment of a documentation and information service, and the
creation of a research program.\footnote{CEC, \emph{Report on the
  Development of the Social Situation in the Communities in 1977}
  (Brussels: CEC, 1978), 22.}

The functioning of both the ESF and the CEDEFOP was primarily connected
to the member states' government structures. Prior to 1977, the ESF was
administered by the EC; however, after the reforms, the member states
could integrate ESF-funded training into their employment
policies.\footnote{Brine, ``The European Social Fund,'' 153.}

During the 1970s---the era of the gender equality model as identified by
Rees---the ESF provided funding for studies on women in employment based
on the surveys of programs implemented by certain member states. The
studies concluded that, in order to promote equality in the labor
market, women required enhanced specialized education and training
services and increased guidance; in addition, a change in women's
attitudes in relation to taking up a career later in life was also
necessary.\footnote{CEC, ``The European Community and Work for Women,''
  \emph{Information Service for Women's Organisations and Press}, no.
  584/X/77 (September 1977).} Moreover, the CEDEFOP reviewed programs
already implemented within the member states in vocational training and
disseminated information regarding these. Unlike the ESF, the CEDEFOP
had no funding function, but sought, through cooperation with the member
states and other social and institutional actors, to disseminate an
agenda on the diversification of women's occupational choices, training
for women in industrial and technical jobs, and the upgrading of
traditionally female occupations.\footnote{Suzanne Seeland, \emph{Equal
  Opportunities and Vocational Training---13 Years On: The Results of
  CEDEFOP's Programme for Women 1977--90} (Berlin: CEDEFOP, 1991), 1.}
Both institutions aimed to leverage existing good practices to avoid the
cost of developing programs from the ground up.

In 1980, for example, the CEDEFOP organized a seminar in Brussels, in
which all member states presented the results of their pilot programs on
women's vocational training, with a specific focus on the topic of
innovation. The innovative projects encompassed fields such as building
techniques, tile-laying, crane-operating, and even the provision of care
for the elderly. The CEDEFOP did not report on any project linked to the
new technologies in this seminar. One of the main lessons resulting from
this event, according to the CEDEFOP, was that an integral element of
the success of innovation in women's training depended on women's own
motivation.\footnote{CEC, ``Women's Employment: How to Innovate,''
  \emph{Women of Europe}, no. 17 (November/December 1980).} The view of
the woman's role in new technologies was not re-evaluated from a
European perspective. As we can observe, the promoted and funded
programs mirrored local views and eventually reproduced a national bias
on the types of training women were encouraged to undergo. The positive
action model that Rees identifies in this period was locally formulated
and implemented.



\hypertarget{womens-education-and-training-in-new-technologies-as-european-social-policy}{%
\section{Women's Education and Training in New Technologies as\\\noindent
European Social
Policy}\label{womens-education-and-training-in-new-technologies-as-european-social-policy}}

The European strategy to cope with information technology was manifold.
While a critical aspect of the strategy involved fostering industrial
growth through initiatives like ESPRIT, RACE, EUREKA, and
others,\footnote{See Carmen Flury, Michael Geiss, and Rosalía Guerrero,
  ``Building the Technological European Community through Education:
  European Mobility and Training Programmes in the 1980s,''
  \emph{European Educational Research Journal} (December 2020); Wayne
  Sandholtz, \emph{High-Tech Europe: The Politics of International
  Cooperation} (Los Angeles: University of California Press, 1992).}
which relied heavily on the availability of a highly skilled workforce,
there was also a need to provide flexible labor that could adapt to new
skill requirements and embrace innovations, even if they were not
capable of contributing to the creation and development of new
technologies. At the same time, there was a need to create a large
domestic market and a social environment prepared for information
technology, for which the training of workers, users, and the general
public was essential.\footnote{CEC, ``Europe and the New Information
  Technology,'' \emph{European File}, no. 3 (March 1980).} A
reorientation of vocational training not only meant the re-training of
engineers and scientists, but a professional re-adaptation of the
unemployed, by strengthening the links between education, vocational
training, and work.\footnote{CEC, ``Micro-Electronics and Employment,''
  \emph{European File}, no. 16 (October 1980).}

Thus, preparing for new technologies not only involved educating the
highly skilled workforce but also creating a computer-literate workforce
that would be more willing to accept technical changes at work and in
other areas. Rhetorically, there was never a clear distinction between
training in technology for women and men, but in practice, this was the
case.

The EC used a discourse that depicted women in the labor market
predominantly in a position of victimhood. In a survey published in 1980
on the situation of European women in paid employment, the European
Commission's Committee on Women's Rights indicated that 13 percent of
employed women had felt discriminated against in matters of pay,
recruitment, promotion, and training. These results were the basis of
further measures to help working women.\footnote{CEC, ``Survey: European
  Women in Paid Employment,'' \emph{Women of Europe}, no. 17
  (November--December 1980).} The ESF also reported in 1980 that women
continued to be disadvantaged in the area of employment.\footnote{CEC,
  \emph{Tenth Report on the Activities of the European Social Fund 1981
  Financial Yea}r, COM (82) 420 (Brussels: CEC, 1982).} Although the
Commission stated that the highest priority would be given to projects
that enabled women to find jobs in new occupations where they were
underrepresented and in sectors where they had been the victims of mass
redundancies during the reporting period, the projects funded by the ESF
concerned occupations traditionally reserved for men, such as industrial
production, construction, metalwork, and woodwork. The report does not
mention examples of training for women in new technologies. However, the
ESF did fund three pilot project schemes related to technical progress:
a training program for firms to adapt to new technologies, training for
unemployed mature engineers in computer-aided design and manufacture,
and training operators to become instructors for new data-processing
professionals. In these projects, women were not specifically mentioned,
as the women-focused pilot projects were related to the training of
rural women for self-employed agricultural work. This report illustrates
the point we seek to make. Women are mentioned as victims of
discrimination in the labor market, for which they are given support in
the form of training. However, they are not being addressed as potential
assets capable of working with new technologies.

While the share of research and development investment in the EC more
than tripled, and programs such as the Framework Programmes for Research
and Technology (FRAMEWORK) materialized, these had no special provisions
for women until the Fourth Framework Programme (1994--1999) that
specifically dealt with social exclusion.\footnote{Jong-Tsong Chiang,
  ``From Industry Targeting to Technology Targeting: A Policy Paradigm
  Shift in the 1980s,'' \emph{Technology in Society} 15, no. 4 (1993):
  346; Kim Junic and Yoo Yaewook, ``Science and Technology Policy
  Research in the EU: From Framework Programme to HORIZON 2020,''
  \emph{Social Sciences} 8, no. 5 (2019); Rees, \emph{Mainstreaming},
  128.} Although this problem was overlooked in the higher-level
political arena, youth organizations identified this issue early on. In
a 1980 forum, youth organizations claimed that women should not be only
a marginal labor force, as they currently were, but should be workers in
their own right.\footnote{CEC, ``The Youth Forum: Young Women's
  Employment,'' \emph{Women of Europe} 17 (November--December 1980).}

A commissioned report by sociologist Jonathan I. Gershuny on technical
innovation and women's work in the EEC called for policies that would
prevent the situation of women from worsening. The proposal was for
women to be part of work-sharing schemes. Such schemes required specific
training that would provide them with a broader mix of
skills.\footnote{Jonathan I. Gershuny, \emph{Technical Innovation and
  Women's Work in the EEC: A Medium-Term Perspective; A Briefing for the
  EEC} (Sussex: SPRU, 1980).} This strategy, deemed the most suitable
realm for EC initiatives, involved training women in diverse but less
complex roles rather than providing them with specialized training,
thereby turning them into interchangeable workers.

The concerns voiced in investigations like these on the effects of new
technologies on women were included in the equal opportunities and
social action programs from the 1980s, although this topic was not the
focus of the programs. The New Community Action Programme on the
Promotion of Equal Opportunities for Women and Men (1982--85)
highlighted the field of employment, with the objective of achieving
equal treatment. The already precarious situation of women, the
Commission explained, was affected by the introduction of
manpower-saving technologies in secretarial work and retailing. Thus,
action in initial and further training, as well as vocational guidance,
should aim to encourage women into non-traditional areas and prepare
them for new information technologies. The action program invited member
states to impose measures at the national level. Apart from a general
call to undertake action to improve equal opportunities in education,
guidance, and training, the program requested one concrete measure from
the member states to promote women's mastering of new technologies: to
provide ``further training for women in employment within firms, with a
view to improving their prospects for promotion . . . other than
unskilled jobs, especially those involving new
technologies'';\footnote{CEC, ``A New Community Action Programme on the
  Promotion of Equal Opportunities for Women 1982--85,'' supplement,
  \emph{Bulletin of the European Communities} (1982), 21.} on the EC
side, the Commission proposed extended action of the ESF and the
CEDEFOP.

In this spirit, and after the adoption of the resolution on equality of
opportunities in the field of education, a meeting took place between
the Ministers of Education in 1985 that led to an action program on
equal opportunities for girls and boys in education. One of the four
goals of this program was to ``encourage girls to participate as much as
boys in new and expanding sectors, within both education and vocational
training, such as the new information technologies . . .''\footnote{CEC,
  ``Resolution of the Council and of the Ministers of Education, Meeting
  within the Council of 3 June 1985 Containing an Action Programme for
  Equal Opportunities for Girls and Boys in Education,'' \emph{Official
  Journal of the European Communities} C 166 (1985): 130.} The
objectives of the action program would be achieved by promoting
awareness, offering counselling and disseminating information, but also
through more concrete measures, such as introducing new technologies to
boys and girls from the end of primary school onwards, opening schools
to working life, and making female role models working in
non-traditional fields available to pupils.\footnote{CEC, ``Resolution
  of the Council,'' 2­--3.} The action program was to be implemented
within the constitutional possibilities of each member state, with the
funds they had available and to the extent that their respective
educational systems allowed. The implementation of this program took
place through national projects, without a common European understanding
of objectives, target populations, or evaluation
methodologies.\footnote{Sylvie Osterrieth, abstract of \emph{Equal
  Opportunities and New Information Technologies, Evaluation on the
  Projects 1987--1989} (Luxembourg: Office for Official Publications of
  the European Communities, 1991), 1.} While the national projects were
funded by each member state, the EC provided assistance through
different bodies for the organization of contests among member states,
as well as colloquiums, surveys, and research publications.\footnote{See
  CEC, ``Activities of the Commission of the European Communities in the
  Fields of Education, Training and Youth Policy During 1987,''
  supplement 5, \emph{Social Europe} (1988): 16--17.}

The formulation of this program followed an equal treatment approach,
promoting girls to match boys in their interest to pursue technical
education. Moreover, the assumption behind the program was that gender
inequality in technology was due to women's lack of interest and
confidence and general prejudices in relation to women's abilities to do
certain jobs. Thus, the program was primarily intended to fight gender
stereotypes in different target groups, including among girls
themselves, as well as in teachers, parents, and teacher trainers, from
pre-school to university level.\footnote{CEC, \emph{Cooperation in
  Education in the European Union 1976--1994}, (Luxembourg: Office for
  Official Publications of the European Communities, 1994), 14.}

Since women's work was regarded as particularly affected by ICT, their
training in this area was explicitly recognized as a
priority.\footnote{Felix Rauner, \emph{Women Study Microcomputer
  Technology} (Berlin: CEDEFOP, 1985), 13.} The EC Commission resorted
to a positive action approach when referring to their goal of providing
job opportunities to special groups, such as older workers, people with
disabilities, and women. Pilot and demonstration actions, already in
place locally, would increase the opportunities of groups with specific
needs in other member states, if broadly disseminated.\footnote{CEC,
  \emph{Vocational Training and New Information Technologies: New
  Community Initiatives during the Period 1983--1987}, information memo
  P-29, 1982.} The CEDEFOP also took a positive action approach, stating
that the development of vocational training in microelectronics must
consider the needs of specific groups of women, including young women
without training, women who wished to return to the occupation they were
trained for following a career break due to family commitments, and
women from ethnic minorities.\footnote{Rauner, \emph{Women Study
  Microcomputer Technology}, 7.}

In a similar vein, the Medium-Term Social Action Programme, adopted in
1984, proposed measures for the employment of young people and women
linked to ICT, which included the acquisition of new skills, basic and
continuous training, and information.\footnote{L. Wallyn, ``The Social
  Policy of the Community and Participation of the Social Partners in
  Decision-Making at European Level,'' \emph{Social Europe} 1 (1988):
  15.} The focus of this program was to combat the economic recession
and the employment crisis. The high rate of unemployment of young people
and women was interpreted as a problem that affected the social balance
of the member states. One way to address this problem was to
\emph{reconvert} these groups to the new technologies.\footnote{EC
  Council, ``Conclusions of the Council Concerning a Community
  Medium-Term Social Action Programme,'' \emph{Official Journal of the
  European Communities} C175 (1984): 2 (our emphasis).} Hence, the fight
against unemployment and the training of vulnerable groups in ICT went
hand in hand. By categorizing the European workforce into different
groups and making them the object of specific measures, the EC
implemented a positive action model.

A report from the EC Commission, presented at the United Nations World
Conference in 1985, summarized the EC work in relation to gender
equality. This report stressed that the advent, development, and spread
of technology were the main challenges to women's work. It highlighted
the role of the ESF in financially supporting the diversification of
women's occupations, the training of women for occupations using new
technologies, and the creation of women's cooperatives in this sector.
According to the report, resolutions had been adopted with measures to
\emph{help} women deal with new technologies in the field of work. These
included basic training in the use of new technologies, information
events, and guidance.\footnote{CEC, \emph{Communication from the
  Commission to the Council on the Community Participation in the World
  Conference to Review and Appraise the Achievements of the United
  Nations Decade for Women: Equality, Development and Peace}, COM (85)
  256 (Brussels: CEC, 1985), 8--9 (our emphasis).} By referring to women
as a group in need of help and by offering basic training to avoid
unemployment, the low-skilled nature of women's work was taken for
granted. No specific measures were developed to change the gendered
structure of work or to promote women's entry into high-skilled
positions.

Moreover, the resource allocation for the developed measures was meagre.
By 1988, the total number of programs funded by the ESF and targeting
women represented only 1.5 percent of total fund approvals.\footnote{CEC,
  ``Community Initiatives on Vocational Training for Women,''
  \emph{Social Europe} 1 (1988): 27.} A positive action model was used
to help women cope with technological change and to have the opportunity
to retain or secure a job, but not to achieve high-level qualifications
or highly skilled employment.

The ESF-funded projects that took place within the member states were
formulated within the context of each state's national infrastructure,
educational and vocational training systems, and gender regimes. The
following examples illustrate how the role of women in the technological
society was viewed locally. The Belgian Commission for Women's Work, for
instance, stated that women's educational level was insufficient for the
new information society, and that it was necessary to invest in
education and training programs that would equip them to obtain
\emph{highly-qualified} jobs.\footnote{Camille Pichault,
  \emph{Technologies et Emploi des Femmes,} Proceedings of the
  Commission Belge du Travail des Femmes, October 15, 1981, FDE-164,
  Historical Archives of the European Union (our emphasis).} In
Luxembourg, free computer courses were offered by a technical school to
encourage girls into scientific and technical professions.\footnote{CEC,
  ``Country to Country,'' \emph{Women of Europe Newsletter} 67 (December
  1990--January 1991): 26.} In the United Kingdom, the Butcher Report on
the situation of ICT experts in the labor market highlighted that ICT
companies could not ignore the intellectual resources offered by women.
This report addressed not only the question of girls' subject choices at
school level but also the situation of women already in
employment.\footnote{Bill Johnstone, ``Women Still Lose Out in the
  Hi-Tech World,'' \emph{The Week}, June 10, 1986, 29.} According to
British research from 1982, women's retraining in new technologies
should be favored to enable them to participate in the new jobs created
by technological developments.\footnote{CEC, ``Microelectronics and
  Women,'' \emph{Women of Europe} 26 (May--July 1982): 49.} A French
study displayed a more cautious attitude. The authors warned of the
potential consequences of encouraging women to attain higher
qualifications. This strategy would not work, according to the study,
because of the prevailing gender discrimination in the labor market.
Such a strategy would only be successful once a change in power
structures had occurred and men were participating to a greater extent
in the domestic domain.\footnote{``Nouvelles Technologies: Deux études
  controversables sur les femmes parviennent à des conclusions
  differéntes,'' \emph{Rapport CREW} 5 (1982), FDE-164, Historical
  Archives of the European Union, 112.} While the study highlighted the
futility of training women, it did not offer proposals to change social
and economic structures in their favor. In 1987, the conclusion of a
French women's association meeting was that women's aptitudes could be
useful in the future. Should they acquire the necessary technical
skills, their ``precision, dexterity, speed and flexibility'' could be
sought after traits in the firms of the future.\footnote{CEC, ``Country
  to Country: France,'' \emph{Women of Europe} 51 (November--December
  1987): 14.} Skills that have traditionally been attached to women and
low-status jobs could, in their view, be useful for the new ICT jobs.
However, the type of positions for which these skills could be useful
were not specified. Women's organizations in the Nordic countries, the
Mediterranean countries, Italy, the United Kingdom, and France organized
conferences, workshops, and symposiums in 1987 to discuss the role of
women in the area of technology; some of these included professional
women who discussed the role of women in the field of new technologies
in higher education.\footnote{CEC, \emph{Women of Europe} 51
  (November--December 1987).} These examples show the divergence of
positions in relation to the training of women in ICT in the member
states. While locally, certain actors noted the need for the education
and training of women in highly skilled areas of ICT, others were of the
opinion that women were vulnerable participants in the labor market with
skills and aptitudes appertaining to their sex.

As a result of a report from an expert network of equal opportunities
regarding ICT projects in the member states, the Second Medium Term
Community Programme on Equal Opportunities (1986--1990) made education
and training a priority.\footnote{CEC, ``The Community Action
  Programmes,'' supplement, \emph{Women of Europe} 25 (October 1986):
  90.} The network saw the current situation in Europe as ``an
industrial revolution as important and far reaching in its consequences
as the industrial revolution of the 19th and 20th century'' and argued
that, while women in the past had failed to find a better position for
themselves in the new labor market and society, an opportunity existed
to avoid such a situation in the current revolution, through education
and training in the new technologies.\footnote{Evelyn Sullerot,
  \emph{Diversification of Vocational Choices for Women} (Luxembourg:
  Office for Official Publications of the European Communities, 1987):
  7--8.}

According to the program, the resources offered by the ESF, hitherto
only used to a limited extent, should be put to better use in networking
and publicity. New technologies constituted one of the fields of action
of the program, which entailed specific provisions on education,
training, and employment in this area. The program called for vertical
mobility in the specific sectors of the new technologies and encouraged
appointing women to positions of responsibility in future-looking
industries. The ESF, the program stipulated, should play a major role in
the achievement of these objectives.\footnote{Odile Quintin, ``Equal
  Opportunities for Women: Medium Term Community Programme 1987--1990,''
  supplement 2, \emph{Social Europe} (1986): 54--62.} In this context,
the CEDEFOP was called upon to work on a network of projects related to
training in new technologies with a view to achieving career
advancement.\textsuperscript{72} Positive action was recommended in ICT so
that women\marginnote{\textsuperscript{72}\setcounter{footnote}{72} CEC, ``Equal Opportunities for Women: Medium Term
  Community Programme 1986--1990,'' supplement, \emph{Women of Europe},
  no. 23 (December 1985): 9.} could ``respond to the technological challenge on an equal
footing.'' However, despite the specific focus on training and education
in new technologies in this program, no indications were offered as to
what kind of training should be given, and full responsibility was given
to the member states to decide the content and level of the training. A
request to the member states to produce statistics on the proportion of
women in the area of ICT and the levels they occupied,\footnote{CEC,
  ``Equal Opportunities for Women,'' 12.} suggests an acknowledgement of
structural inequality in this field, but the measures proposed did not
address this issue. The equality of opportunities problem continued to
be discussed in terms of underrepresentation in vocational training. The
cause was assumed to be that women rarely chose technical training and
opted for traditionally feminine occupations. For this reason, help in
the form of counselling, information, awareness, and role models was
presented as a sufficient and adequate solution. The most concrete
measures named in this context were the provision of appropriate
guidance (equal opportunities counsellors), studies, seminars, and
consciousness-raising campaigns.\footnote{CEC, \emph{Commission
  Communication on Vocational Training for Women,} COM (87) 155
  (Brussels: CEC, 1987).}

Although Brine identifies a break between the first and second action
programs on equal opportunities in relation to women's training in new
technologies---with the former emphasizing women's training specifically
in the fields of computers, electronics, and office work and the latter
focusing on women's training in fields in which they were
underrepresented\footnote{Brine, ``The European Union's Discourse of
  `Equality'.''}---we observe a generally unchanged discourse. These
programs presented the new technologies as a phenomenon that threatened
women's positions in the labor market. To prevent women's social
exclusion, these programs promoted the widening of occupational choices
for women; equal treatment of women and men with regard to access to
education, training, and employment; and the provision of technical
subjects for both girls and boys from the early stages of basic
education.

When concrete measures were proposed in policies, these often took the
form of positive action, targeting specific groups at risk of
unemployment. The most striking feature of the programs and initiatives
was that they constantly referred to women as a vulnerable group,
threatened by the new technologies and requiring support and guidance.
Hence, we observe arguments linked to the equal treatment approach
co-existing with arguments appealing to the positive action approach.
Additionally, although programs and initiatives often mentioned women's
economic role, this role was not based on women's potential to become
part of the skilled workforce to develop the ICT industry, but on the
fear that women's likely unemployment could pose a risk to the EC's
social cohesion.

\hypertarget{womens-education-and-training-in-new-technologies-for-a-new-economic-role}{%
\section{Women's Education and Training in New Technologies for a\\\noindent New
Economic
Role}\label{womens-education-and-training-in-new-technologies-for-a-new-economic-role}}

IRIS, a community network on vocational training, was launched in 1988,
following the report of an expert group representing the twelve member
states, to evaluate the implementation of the Commission recommendation
on vocational training for women, adopted in 1987.\footnote{CEC, ``IRIS:
  The New Community Network of Demonstration Projects on Vocational
  Training for Women,'' \emph{Social Europe} 2 (1989).} This was the
first major women-specific initiative of the EC.

IRIS aimed at encouraging women's training and supporting innovation in
this area. A function of IRIS was to complement national initiatives on
vocational training for women and help spread good examples.\footnote{CEC,
  ``IRIS: The New Community Network.''} IRIS sought to involve different
actors in the goal of improving women's access to employment and
vocational training. These included educational authorities, schools,
vocational centers, industry representatives, government authorities at
different levels and women's organizations.\footnote{CEC, ``Commission
  Communication on Vocational,'' 10.} Out of all the projects supported
by the IRIS network, 36 percent involved the training of women in new
technologies.\footnote{Publications Office of the European Union,
  \emph{IRIS}, Summaries of EU Legislation, n.d., accessed April 15,
  2020.} The first seventy-two projects focused on areas in which women
had been underrepresented, such as building and electronic and
mechanical engineering, but also on certain fields in which women had
already made their mark, such as banking and crafts.\footnote{CEC, ``The
  Changing European Community,'' \emph{Women of Europe} 57
  (November--December 1988): 4.} Even though IRIS's explicit objective
was to develop sustainable strategies and methods for long-term change,
the size and scope of the funding did not match this ambition. IRIS was
allocated 0.75 million ECU for the period 1988--95, which is
significantly lower than the budget of other contemporary programs (see
Table 1). COMETT, the initiative created to develop technological skills
and increase the community's competitiveness, received 206.6 million ECU
during the same period. This reinforces Rees's claim that the positive
measures from the 1980s ``tended to be precariously funded, short term
and piecemeal.''\footnote{Teresa Rees, ``Reflections on the Uneven
  Development of Gender Mainstreaming in Europe,'' \emph{International
  Feminist Journal of Politics} 7, no. 4 (2005): 555--74.}

\vspace{.2in}

\noindent{\large \emph{Table 1. Budgets for Education and Training Action Programmes}}\footnote{Rees, \emph{Mainstreaming Equality, 91.}}


\vspace*{1em}



\tabulinesep=1.1mm
{\begin{longtabu} to 1\textwidth { X[l] X[c] X[l]} 
\emph{Programme} & \emph{Budget ECU (millions)} & \emph{Description} \\
\endfirsthead
\emph{Programme} & \emph{Budget ECU (millions)} & \emph{Description} \\
\endhead
COMETT (1986--95) & 206.6 & Cooperation between universities and
industry in the field of technology \\
FORCE (1991--94) & 31.3 & Development of continuing vocational
training \\
PETRA (1988--94) & 79.7 & Vocational training of young people \\
ERASMUS (1987--95) & 307.5 & Mobility of university students \\
TEMPUS (1990--94) & 197 & Trans-European mobility for university
students \\
LINGUA (1990--94) & 68.6 & Promotion of foreign language competence \\
EUROTECNET (1990--94) & 7 & Promotion of innovation in vocational
training resulting from technological change \\
YOUTH FOR EUROPE (1988--94) & 32.2 & Promotion of youth exchanges \\
IRIS (1988--95) & 0.75 & Network of vocational training projects for
women \\
\end{longtabu}}

\vspace{0.2in}

The establishment of IRIS was linked to the completion of the internal
market in 1992. The first round of the initiative was planned to run
until that year. According to the Commission, the internal Single Market
required the community's economies to make the best use of their human
resources, regardless of gender.\footnote{CEC, ``IRIS: The New Community
  Network,'' 51.} The Commission considered IRIS an innovative labor
market initiative, which, if successful, would set an example for other
areas in the community. It also expected that the grassroot strategies
that proved beneficial for the labor market would be disseminated
broadly within the community. This means that the objective was not
necessarily intended to be ground-breaking and provide women with skills
that would elevate their position as workers, but to place them where
they were needed in the economy and to assess whether the innovative
schemes brought benefits for the European economy.

IRIS supported many ICT projects with a focus on office skills despite
not explicitly advocating for technology-related initiatives. However,
the training was not provided for highly skilled office positions. An
evaluation of IRIS concludes that the funded projects did not lead to
changes in training systems nationally, and only 20 percent of the
projects adopted innovations obtained through the network.\footnote{PA
  Cambridge Economic Consultants, \emph{An Evaluation of the IRIS
  Network (Final Report)}, Commission of the European Communities
  (1992), 123.}

New arguments concerning women's training began to appear in the late
1980s and became more prominent during the 1990s. Within the context of
IRIS's activities, the involvement of a variety of actors in the network
offered new perspectives in the debate on women's training. That society
needed to integrate women armed with the required qualifications into
the modernization process of enterprises emerged as an innovative
idea.\footnote{CEC, ``Seminars,'' \emph{Women of Europe Newsletter 11}
  (June--August 1990).} This concept was discussed in depth at the
Toledo seminar of 1989 on the evaluation of the community policy on
equal opportunities, in which representatives of the EC, the member
states, women's groups, and industry participated. Women were not
referred to as a vulnerable group here but as a potential ``crucial
element in the workforce over the next few years.''\footnote{CEC, ``The
  Evaluation of Community Policy on Equal Opportunities: Toledo, April
  1989,'' \emph{Social Europe} 3 (1989): 63.} In 1992, a seminar
organized by IRIS at the Forum on Women's Vocational Training in Europe,
discussed the question of women as a vital resource. In the opening
message, the President of the Commission of the European Communities,
Jacques Delors, stressed that women represented a source of labor and
skills in the establishment of the Single Market. Apart from
strengthening enterprises, Delors stated that the training of women was
vital in the fight against poverty. In the seminar, the increasing
feminization of poverty was noted, as well as the fact that many women
required personal development and pre-training elements in their
training efforts.\footnote{CEC, ``IRIS Fair 1992: Women, a Vital
  Resource,'' \emph{Women of Europe Newsletter} 29 (October 1992).}
Although there was a shift in the view of the role played by women in
the economy, their economic vulnerability continued to be noted, and
many of the proposed measures were not considerably different from
previous measures: to ensure women's access to new technologies and to
develop pedagogical strategies to teach them science and technology.
Remarkably, ICT was still viewed as an area free from sexual
discrimination and as a potential equalizer. Thus, encouraging women
into this area would automatically widen their occupational
choices.\footnote{CEC, ``The Evaluation of Community Policy,'' 64--65.}

The initiatives of the 1990s reflected a consolidation of the earlier
arguments of equal treatment that prioritized placing women on an equal
footing with men, while later initiatives acknowledged women's key role
in the emerging digital society, viewing them as an integral part of
economic change. In the Third Action Programme on Equal Opportunities
(1991--1996), the view of women's training as a mine of potential
skills, indispensable for Europe's economic growth, became explicit.
This program framed the work carried out by women in both social and
economic terms. On the one hand, it stated that working women were
beneficial for Europe's social cohesion and called for efforts to
counter gender inequality by fighting the underrepresentation of women
in certain occupations, and by ensuring that women also benefitted from
the economic and technological advantages of the Single
Market.\footnote{CEC, \emph{Commission Proposes Third Action Programme
  on Equal Opportunities for Men and Women}, press release P/90/76,
  1990.} On the other hand, it emphasized that women's skills were
valuable assets for the economy and constituted a potential human
resource, hitherto underutilized.\footnote{CEC, ``Community Initiative
  for Women,'' \emph{Women of Europe} \emph{Newsletter} 11 (June--August
  1990).} However, in contrast to the previous action programs, there
was no specific mention of new technologies in this one.\footnote{Brine,
  \emph{UnderEducating,} 86.}

\enlargethispage{\baselineskip}

As an integral element of the Third Action Programme on Equal
Opportunities, the ESF established New Opportunities for Women (NOW)
(1991--1995) to promote vocational training and employment for women.
With a grant of 120 million ECU for the first three years, which was
provided primarily by the ESF,\footnote{CEC, ``The Changing European
  Community,'' \emph{Women of Europe Newsletter} 68 (February--May
  1991): 1.} NOW aimed to increase the participation of women in the
labor market, particularly those in long-term unemployment and those
with entrepreneurial ambitions. For the period 1994 to 1999, NOW was
renewed and granted a budget of 360 million ECU.\footnote{CEC,
  \emph{Commission Communication on Incorporating Equal Opportunities
  for Women and Men into all Community Policies and Activities}, COM
  (96) 67 (Brussels: CEC, 1996), 17.}

The Council stated that projects in ESF programs, such as NOW, that
could anticipate skill needs and promote equal opportunities for men and
women in the labor market should receive support. Hence, following the
same line as the Action Programme, NOW emphasized the importance of
providing women with the appropriate skills for the market, as well as
skills that could enhance their economic position, such as
entrepreneurial skills. Nevertheless, the promotion of training in new
technologies declined noticeably, despite a shift that recognized women
as more relevant economic players in the community. According to Brine,
the promotion of women's training focused on manual trades more than
anything.

However, an example of a NOW project within the ICT area from the late
1990s is the establishment of a telework center in a rural French
community. This center employed about twenty people, mostly women,
performing secretarial and administrative tasks. The EU regional policy
officer mentioned this project as an example of the policy support given
to women in the area of employment. The fact that the commissioner chose
this project to exemplify the kind of support given to women is
illustrative of the point we are making. Even the projects considered
most innovative in supporting women's participation in ICT tended to
reinforce traditional gender roles.\footnote{CEC, ``Regional and Local
  Policy Opens Way for Women,'' \emph{Women of Europe} 70 (May 1997): 4.}

The program created to promote innovation in the field of vocational
training, resulting from technological change, EUROTECNET (1990--1994),
made some provision for equality of opportunities. This program stated
that equal opportunities for men and women should be promoted, and
particularly the access of women to types of training with significant
technological content.\footnote{Rees, \emph{Mainstreaming Equality,}
  132.} In practice, however, only around 5 percent of the projects
addressed women specifically.\footnote{Rees, \emph{Mainstreaming
  Equality,} 144.} Thus, regardless of whether the program particularly
targeted women or training in new technologies, substantial resources to
support women's training in technology were never provided.

The aforementioned initiatives followed the logic of equal treatment and
positive action approaches that Rees identified in the 1970s and 1980s,
respectively. However, we observe intertwined arguments following both
these logics in the programs launched in the 1980s and 1990s.
Additionally, we identify a new argument: Women were not only a
vulnerable group and victims of technology but were also assets for the
European economy's growth. However, despite the apparent recognition of
the significance of women in the labor market and the economy, this was
not reflected in the budget dedicated to women's training initiatives or
projects. The EC budget for 1990--1992, in relation to objectives 3 and
4 of the ESF,\footnote{Related to long-term unemployment and the
  occupational integration of young people, respectively. Aquille
  Hannequart, ed., \emph{Economic and Social Cohesion in Europe: A New
  Objective} (London: Routledge, 1992), xxiv.} allocated only 0.4
percent to measures to assist women encountering difficulties in the
labor market. Only 7 percent of the entire EC budget was allocated to
the ESF,\footnote{Brine, ``The European Social Fund.''} demonstrating
that the proportion allocated to women and to measures regarding women's
employment was almost negligible.

\hypertarget{concluding-remarks}{%
\section{Concluding Remarks}\label{concluding-remarks}}

With the introduction of the gender mainstreaming principle into EC
policy processes, which entailed the integration of a gender perspective
into all domains and actions of the EC, the drafting of policy in most
areas changed drastically, including education and
technology.\footnote{Isabelle Bruno, Sophie Jacquot, and Lou Mandin,
  ``Europeanization through its Instrumentation: Benchmarking,
  Mainstreaming and the Open Method of Co-ordination . . . Toolbox or
  Pandora's Box?,'' \emph{Journal of European Public Policy} 13, no. 4
  (2006): 524.}

\newpage From 1999, the FRAMEWORK programs started to include concrete measures
to foster gender equality in science and technology. Although the scope
and outcomes of this change go beyond the timeframe of this article, we
note that it took more than twenty years for the EC to consider the
gender dimension of science and technology policy, instead of
restricting it to social policy.\textsuperscript{100} 

The\marginnote{\textsuperscript{100}\setcounter{footnote}{100} For a discussion on the
  effects of mainstreaming in research policy, see: Lut Meargert and
  Emanuela Lombardo, ``Resistance to Implementing Gender Mainstreaming
  in EU Research Policy,'' in ``The Persistent Invisibility of Gender in
  EU Policy,'' ed. Elaine Weiner and Heather MacRae, \emph{European
  Integration Online Papers}, special issue 1, no. 18, article 5 (2014).} European Communities developed technology and science policies, as
well as vocational training and education programs, as a response to the
challenges posed by ICT technologies from the 1970s onwards. These
initiatives aimed at improving the competitiveness of European industry
and tackling social problems. In the EC's view, a gap had been created
in the previous twenty years between their own technological industries
and those of the United States and Japan. The objective of the EC's
programs was to catch up with these countries, while dealing with the
technical, economic, and social consequences of the emergence of new
technologies.\footnote{CEC, \emph{Vocational Training and New
  Information Technologies.}} At the same time, gender issues were
becoming institutionalized, and instruments were developed to increase
gender equality in the EC. While there was concern regarding the effects
of new technologies on women, particularly in relation to their labor
market position, the measures and initiatives developed in the field of
training and education to promote the participation of women in ICT
aimed primarily at preventing women's unemployment and creating a pool
of computer literate and flexible workers.

The education and training initiatives and measures we have analyzed,
reveal the transformation of a view of the role of women in the European
labor market and society, within the context of technological change.
The gender question was often discussed in relation to unemployment and
its social and economic consequences for the community, particularly in
the 1970s. For most of this period, the equality model of equal
treatment and positive action were intertwined. As women were considered
weak players in the labor market, they required special attention,
protection, and access to training opportunities, particularly in light
of the changes brought about by technological change.

It seems counterintuitive that although the EC needed to increase the
pool of skilled workers in new technologies and despite the fact that
women had been an unexploited human resource in the 1940s and 1950s and
had played a vital role in the development of the computer industry,
there was no substantial endeavor to encourage women's full
participation in ICT or to promote their role as creators and
developers; instead, they were viewed as end users.

Brine identifies a mismatch between the needs of the labor market and
the training provided to women within the framework of EC initiatives on
vocational education and training.\footnote{Brine, ``The European Social
  Fund.''} The way in which the question of women's education and
training in technology was formulated in the initiatives and other
documents in this area, suggests that women were, for a long time, not
considered as potential highly skilled workers, and all projects to
integrate them into the field of ICT at any level were allocated very
limited budgets.

Targeting underrepresentation has not necessarily meant promoting
women's entry into highly skilled occupations, but traditionally has
often focused on male-dominated occupations, the importance of which has
been in decline in the technological society. The question of
representation obscured the fact that the field of ICT required both
highly skilled and low-skilled workers, therefore merely encouraging
women to learn about computers was not sufficient to promote their full
participation in the creation, management, and use of ICT at a high
level.

The gender models fostered by the EC in the development of education and
training initiatives were limited to \emph{helping} women move from
virtually no training to basic literacy in ICT. This was facilitated by
\emph{allowing} them to gain access to new technologies at different
levels of education and training through short programs with reduced
budgets, and by supporting projects that were often driven or
administered by actors (women's organizations, trade unions, or firms)
that could not ensure the follow-up or institutionalization of the
initiatives in official training and educational structures.

However, we identify a shift at the end of the 1980s. The programs and
initiatives focused on the economic role of women and recognized the
advantages of providing them with skills that could enable their
participation in the more strategic areas of the economy, such as the
ICT field. The gender equality approaches that we observed in earlier
years, based on equal treatment and positive action, were nevertheless
intertwined with the new arguments. The shift was, however, only
rhetorical, as a significant investment of resources in initiatives
targeting women did not take place.

Therefore, we claim that although the challenge of new technologies led
the EC to craft a series of measures to meet new skill demands and
increase competitiveness, the potential of women was underestimated over
a considerable period. By labeling women as victims of technology and
emphasizing their vulnerability in the labor market, the EC justified
initiatives and measures that merely relieved their weakened position
through short-term, underfunded efforts.


\section{Bibliography}\label{bibliography}

\begin{hangparas}{.25in}{1} 



Abbate, Janet. ``Women and Gender and the History of Computing.''
\emph{IEEE Annals of the History of Computing} 24, no. 4 (2003): 4--8.
\href{https://10.0.4.85/MAHC.2003.1253885}{https://10.1109/MAHC.2003.1253885}.

Adnett, Nick, and Stephen Hardy. \emph{The European Social Model:
Modernisation or Evolution?} Cheltenham: Edward Elgar Publishing, 2005.

Barry, Andrew. \emph{Political Machines: Governing a Technological
Society.} London: Athlone Press, 2001.

Blitz, Brad. ``From Monnet to Delors: Educational Cooperation in the
European Union.'' \emph{Contemporary European History} 12, no. 2 (2003):
197--212. \url{https://doi.org/10.1017/S0960777303001140}.

Bornschier, Volker. ``Western Europe's Move toward Political Union.'' In
\emph{State-Building in Europe: The Revitalization of Western European
Integration}, edited by Volker Bornschier, 3--37. Cambridge: Cambridge
University Press, 2000.

Brine, Jacky. ``Equal Opportunities and the European Social Fund:
Discourse and Practice.'' \emph{Gender and Education} 7, no. 1 (1995):
9--22.

Brine, Jacky. ``European Education and Training Policy for
Under-Educated Unemployed People.'' \emph{International Studies in
Sociology of Education} 7, no. 2 (1997): 229--45.
\url{https://doi.org/10.1080/09620219700200008}.

Brine, Jacky. ``The European Social Fund and the Vocational Training of
Unemployed Women: Questions of Gendering and Re-Gendering.''
\emph{Gender and Education} 4, no. 1--2 (1992): 149--62.
\url{https://doi.org/10.1080/0954025920040110}.

Brine, Jacky. ``The European Union's Discourse of `Equality' and its
Education and Training Policy within the Post-Compulsory Sector.''
\emph{Journal of Education Policy} 13, no. 1 (1998): 137--52.
\url{https://doi.org/10.1080/0268093980130109}.

Brine, Jacky. \emph{UnderEducating Women: Globalizing Inequality.}
Philadelphia: Open University Press, 1999.

Bruno, Isabelle, Sophie Jacquot, and Lou Mandin. ``Europeanization
through its Instrumentation: Benchmarking, Mainstreaming and the Open
Method of Co-ordination . . . Toolbox or Pandora's Box?'' \emph{Journal
of European Public Policy} 13, no. 4 (2006): 519--36.
\url{https://doi.org/10.1080/13501760600693895}.

Commission of the European Communities. ``Activities of the Commission
of the European Communities in the Fields of Education, Training and
Youth Policy During 1987.'' Supplement 5, \emph{Social Europe} (1988).

Commission of the European Communities. ``The Changing European
Community.'' \emph{Women of Europe} 17 (November--December 1980):
49--53.

Commission of the European Communities. ``The Changing European
Community.'' \emph{Women of Europe} 57 (November--December 1988): 3--5.

Commission of the European Communities. ``The Changing European
Community.'' \emph{Women of Europe Newsletter} 68 (February--May 1991):
1--6.

Commission of the European Communities. ``The Changing Nature of
Employment: New Forms, New Areas.'' \emph{Social Europe} 1 (1988):
75--79.

Commission of the European Communities. ``The Community Action
Programmes.'' Supplement, \emph{Women of Europe} 25 (October 1986):
88--102.

Commission of the European Communities. ``Commission Proposes Third
Action Programme on Equal Opportunities for Men and Women.'' Press
release P/90/76. 1990.

Commission of the European Communities. \emph{Communication from the
Commission to the Council on the Community Participation in the World
Conference to Review and Appraise the Achievements of the United Nations
Decade for Women: Equality, Development and Peace}. COM (85) 256.
Brussels: Commission of the European Communities, 1985.

Commission of the European Communities. ``Community Initiative for
Women.'' \emph{Women of Europe} \emph{Newsletter} 11 (June--August
1990): 1--2.

Commission of the European Communities. ``Community Initiatives on
Vocational Training for Women.'' \emph{Social Europe} 1 (1988): 27--29.

Commission of the European Communities. \emph{Cooperation in Education
in the European Union 1976--1994}. Luxembourg: Office for Official
Publications of the European Communities, 1994.

Commission of the European Communities. ``Community Initiatives on
Vocational Training for Women.'' \emph{Social Europe} 1 (1988): 27--29.

Commission of the European Communities. ``Country to Country.''
\emph{Women of Europe Newsletter} 67 (December 1990--January 1991):
12--31.

Commission of the European Communities. ``Country to Country: France.''
\emph{Women of Europe} 51 (November--December 1987): 14--15.

Commission of the European Communities. ``Equal Opportunities for Women:
Medium Term Community Programme 1986--1990.'' Supplement, \emph{Women of
Europe}, no. 23 (December 1985).

Commission of the European Communities. ``Europe and the New Information
Technology.'' \emph{European File}, no. 3 (March 1980): 1--7.

Commission of the European Communities. ``The European Community and
Work for Women.'' \emph{Information Service for Women's Organisations
and Press}, no. 584/X/77 (September 1977).
\url{http://aei.pitt.edu/33594/1/A394.pdf}.

Commission of the European Communities. ``The Evaluation of Community
Policy on Equal Opportunities: Toledo, April 1989.'' \emph{Social
Europe} 3 (1989): 63--66.

Commission of the European Communities. \emph{Incorporating Equal
Opportunities for Women and Men into All Community Policies and
Activitie}s. COM (96) 67. Brussels: Commission of the European
Communities, 1996.

Commission of the European Communities. ``IRIS: The New Community
Network of Demonstration Projects on Vocational Training for Women.''
\emph{Social Europe} 2 (1989): 49--52.

Commission of the European Communities. ``IRIS Fair 1992: Women, a Vital
Resource.'' \emph{Women of Europe Newsletter} 29 (October 1992): 3.

Commission of the European Communities. ``Micro-Electronics and
Employment.'' \emph{European File}, no. 16 (October 1980): 1--7.

Commission of the European Communities. ``Microelectronics and Women.''
\emph{Women of Europe} 26 (May--July 1982).

Commission of the European Communities. ``A New Community Action
Programme on the Promotion of Equal Opportunities for Women 1982--85.''
Supplement 1, \emph{Bulletin of the European Communities} (1982).

Commission of the European Communities. ``Regional and Local Policy
Opens Way for Women.'' \emph{Women of Europe} 70 (May 1997): 4.

Commission of the European Communities. \emph{Report on the Development
of the Social Situation in the Communities in 1977}. Brussels:
ECSC/EEC/EAEC, 1978. \url{http://aei.pitt.edu/10246/1/10246.pdf}.

Commission of the European Communities. ``Resolution of the Council and
of the Ministers of Education: Meeting within the Council of 3 June
1985.'' \emph{Official Journal of the European Communities}, no. C 166
(1985).
\url{https://eur-lex.europa.eu/legal-content/EN/TXT/PDF/?uri=CELEX:41985X0507\&rid=1}.

Commission of the European Communities. ``Seminars.'' \emph{Women of
Europe Newsletter} 11 (June--August 1990): 2--3.

Commission of the European Communities. ``Survey: European Women in Paid
Employment.'' \emph{Women of Europe} 17 (November--December 1980):
54--56.

Commission of the European Communities. \emph{Tenth Report on the
Activities of the European Social Fund 1981 Financial Year}. COM (82)
420. Brussels: Commission of the European Communities, 1982.

Commission of the European Communities. ``Vocational Training and New
Information Technologies: New Community Initiatives during the Period
1983--1987.'' Information memo P-29/82. 1982.
\url{http://aei.pitt.edu/31221/}.

Commission of the European Communities. \emph{Vocational Training for
Women}. COM (87) 155. Brussels: Commission of the European Communities,
1987.

Commission of the European Communities. \emph{Women of Europe} 51
(November--December 1987).

Commission of the European Communities. ``Women's Employment: How to
Innovate.'' \emph{Women of Europe}, no. 17 (November--December 1980).

Commission of the European Communities. ``The Youth Forum: Young Women's
Employment.'' \emph{Women of Europe} 17 (November--December 1980):
52--56.

Ceruzzi, Paul E. ``When Computers Were Human.'' \emph{Annals of the
History of Computing} 13, no. 3 (1991): 237--44.
\url{https://doi.org/10.1109/MAHC.1991.10025}.

Chélini, Michel-Pierre, and Laurent Warlouzet, eds. \emph{Calmer les
prix} {[}Slowing down prices{]}. Paris: Presses de Sciences Po, 2017.

Chiang, Jong-Tsong. ``From Industry Targeting to Technology Targeting: A
Policy Paradigm Shift in the 1980s.'' \emph{Technology in Society} 15,
no. 4 (1993): 341--57.
\url{https://doi.org/10.1016/0160-791X(93)90007-B}.

Clarke, Alan. ``Competitiveness, Technological Innovation and the
Challenge to Europe.'' In \emph{The Learning Society: Challenges and
Trends}, edited by Peter Raggatt, Richard Edwards, and Nick Small,
59--67. London: Routledge, 1996.

Cockburn, Cynthia. ``The Material of Male Power.'' \emph{Feminist
Review}, no. 9 (1981): 41--58. \url{https://doi.org/10.2307/1394914}.

Corbett, Anne. \emph{Universities and the Europe of Knowledge: Ideas,
Institutions and Policy Entrepreneurship in European Union Higher
Education Policy, 1955--2005.} London: Palgrave Macmillan, 2005.

Danowitz Sagaria, Mary Ann. \emph{Women, Universities, and Change:
Gender Equality in the European Union and the United States.} New York:
Palgrave Macmillan, 2007.

EC Council. ``Conclusions of the Council Concerning a Community
Medium-Term Social Action Programme.'' \emph{Official Journal of the
European Communities} C175 (1984).
\url{https://eur-lex.europa.eu/legal-content/EN/TXT/PDF/?uri=OJ:C:1984:175:FULL\&from=EN}.

Ertl, Hubert. \emph{European Union Initiatives in Education and
Vocational Training: The Development and Impact of the Programme
Approach}. Wirtschaftspädagogische Beiträge 5. Paderborn: Paderborn
University, 2002.

Ertl, Hubert. ``European Union Programmes for Education and Vocational
Training: Development and Impact.'' \emph{SKOPE Research Paper,} no. 42
(2003).

Flury, Carmen, Michael Geiss, and Rosalía Guerrero. ``Building the
Technological European Community through Education: European Mobility
and Training Programmes in the 1980s.'' \emph{European Educational
Research Journal} 20, no. 3 (December 2020): 348--64.
\url{https://doi.org/10.1177/1474904120980973}.

Gershuny, Jonathan I. \emph{Technical Innovation and Women's Work in the
EEC: A Medium-Term Perspective; A Briefing for the EEC.} Sussex: SPRU,
1980.

Glasner, Angela. ``Gender and Europe: Cultural and Structural
Impediments to Change.'' In \emph{Social Europe}, edited by Joe Bailey,
70--103. New York: Longman, 1991.

Grier, David Alan. ``Human Computers: The First Pioneers of the
Information Age.'' \emph{Endeavour} 25, no. 1 (2001): 28--32.
\url{https://doi.org/10.1016/S0160-9327(00)01338-7}.

Hannequart, Aquille, ed. \emph{Economic and Social Cohesion in Europe: A
New Objective}. London: Routledge, 1992.

Hicks, Mar. \emph{Programmed Inequality: How Britain} \emph{Discarded
Women Technologists and Lost Its Edge in Computing.} Cambridge, MA: MIT
Press, 2017.

Johnstone, Bill. ``Women Still Lose Out in the Hi-Tech World.''
\emph{The Week}, June 10, 1986.

Junic, Kim, and Yoo Yaewook. ``Science and Technology Policy Research in
the EU: From Framework Programme to HORIZON 2020.'' \emph{Social
Sciences} 8, no. 5 (2019): 153--62.
\url{https://doi.org/10.3390/socsci8050153}.

Kauppinen, Ilkka. ``The European Round Table of Industrialists and the
Restructuring of European Higher Education.'' \emph{Globalisation,
Societies and Education} 12, no. 4 (2014): 498--519.
\url{https://doi.org/10.1080/14767724.2013.876313}.

Light, Jennifer S. ``When Computers Were Women.'' \emph{Technology and
Culture} 40, no. 3 (1999): 455--83.
\url{https://www.jstor.org/stable/25147356}.

Meargert, Lut, and Emanuela Lombardo. ``Resistance to Implementing
Gender Mainstreaming in EU Research Policy.'' In ``The Persistent
Invisibility of Gender in EU Policy,'' edited by Elaine Weiner and
Heather MacRae. Special issue, \emph{European Integration Online Papers}
1, no. 18 (2014): 1--21.
\url{http://eiop.or.at/eiop/texte/2014-005a.htm}.

Moisio, Sami. \emph{Geopolitics of the Knowledge-Based Economy}. London:
Routledge, 2018.

Mourlon-Druol, Emmanuel. ``The EMS as an External Anchor in
Inflation-Prone Countries.'' In \emph{Calmer les prix} {[}Slowing down
prices{]}, edited by Michel-Pierre Chélini and Laurent Warlouzet,
273--94. Paris: Presses de Sciences Po, 2017.

``Nouvelles Technologies: Deux études controversables sur les femmes
parviennent à des conclusions differéntes.'' \emph{Rapport CREW} 5
(1982). FDE-164, Historical Archives of the European Union.

Osterrieth, Sylvie. \emph{Equal Opportunities and New Information
Technologies: Evaluation on the Projects 1987--1989.} Luxembourg: Office
for Official Publications of the European Communities, 1991.

PA Cambridge Economic Consultants. \emph{An Evaluation of the IRIS
Network (Final Report)}, Commission of the European Communities, 1992.

Paoli, Simone. ``The European Community and the Rise of a New
Educational Order (1976--1986).'' In \emph{Contesting Deregulation:
Debates, Practices, and Developments in the West since the 1970s},
edited by Knud Andersen and Stefan Müller, 138--52. New York: Berghahn,
2017.

Parker, Simon. ``Esprit and Technology Corporatism.'' In
\emph{State-Building in Europe: The Revitalization of Western European
Integration}, edited by Volker Bornschier, 93--121. Cambridge: Cambridge
University Press, 2000.

Peterson, John, and Margaret Sharp, \emph{Technology Policy in the
European Union.} London: Macmillan, 1998.

Pichault, Camille. \emph{Technologies et Emploi des Femmes.} Proceedings
of the Commission Belge du Travail des Femmes\emph{,} October 15, 1981.
FDE-164, Historical Archives of the European Union.

Publications Office of the European Union. \emph{IRIS}, Summaries of EU
Legislation, n.d. Accessed April 15, 2020.
\url{https://eur-lex.europa.eu/legal-content/EN/TXT/?uri=LEGISSUM:c11018a}.

Quintin, Odile. ``Equal Opportunities for Women: Medium Term Community
Programme 1987--1990.'' Supplement 2, \emph{Social Europe} (1986):
54--70.

Rauner, Felix. \emph{Women Study Microcomputer Technology.} Berlin:
CEDEFOP, 1985.

Rees, Teresa. \emph{Mainstreaming Equality in the European Union:
Education, Training and Labour Market Policies}. London: Routledge,
1998.

Rees, Teresa. ``Reflections on the Uneven Development of Gender
Mainstreaming in Europe.'' \emph{International Feminist Journal of
Politics} 7, no. 4 (2005): 555--74.
\url{https://doi.org/10.1080/14616740500284532}.

Sandholtz, Wayne. \emph{High-Tech Europe: The Politics of International
Cooperation}. Los Angeles: University of California Press, 1992.

Seeland, Suzanne. \emph{Equal Opportunities and Vocational Training---13
Years On: The Results of CEDEFOP's Programme for Women 1977--90}.
Berlin: CEDEFOP, 1991.

Senden, Linda. \emph{Soft Law in European Community Law.} Oxford: Hart,
2004.

Sullerot, Evelyn. \emph{Diversification of Vocational Choices for
Women.} Luxembourg: Office for Official Publications of the European
Communities, 1987.

Wallenborn, Manfred. ``Vocational Education and Training and Human
Capital Development: Current Practice and Future Options.''
\emph{European Journal of Education} 45, no. 2 (2010): 181--98.
\url{https://www.jstor.org/stable/40664660}.

Wallyn, L. ``The Social Policy of the Community and Participation of the
Social Partners in Decision-Making at European Level.'' \emph{Social
Europe} 1 (1988): 13--20.

Warlouzet, Laurent. \emph{Governing Europe in a Globalizing World:
Neoliberalism and its Alternatives Following the 1973 Oil Crisis.}
London: Routledge, 2018.

Warlouzet, Laurent. ``Towards a European Industrial Policy? The European
Economic Community (EEC) Debates, 1957--1975.'' In \emph{Industrial
Policy in Europe after 1945: Wealth, Power and Economic Development in
the Cold War}, edited by Christian Grabas and Alexander Nützenadel,
213--35. London: Palgrave Macmillan, 2014.



\end{hangparas}


\end{document}