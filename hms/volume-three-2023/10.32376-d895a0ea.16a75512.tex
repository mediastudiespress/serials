% see the original template for more detail about bibliography, tables, etc: https://www.overleaf.com/latex/templates/handout-design-inspired-by-edward-tufte/dtsbhhkvghzz

\documentclass{tufte-handout}

%\geometry{showframe}% for debugging purposes -- displays the margins

\usepackage{amsmath}

\usepackage{hyperref}

\usepackage{fancyhdr}

\usepackage{hanging}

\hypersetup{colorlinks=true,allcolors=[RGB]{97,15,11}}

\fancyfoot[L]{\emph{History of Media Studies}, vol. 3, 2023}


% Set up the images/graphics package
\usepackage{graphicx}
\setkeys{Gin}{width=\linewidth,totalheight=\textheight,keepaspectratio}
\graphicspath{{graphics/}}

\title[Review of Becker and Mansell (eds.)]{\emph{Reflections on the International Association for Media and Communication Research}} % longtitle shouldn't be necessary

% The following package makes prettier tables.  We're all about the bling!
\usepackage{booktabs}

% The units package provides nice, non-stacked fractions and better spacing
% for units.
\usepackage{units}

% The fancyvrb package lets us customize the formatting of verbatim
% environments.  We use a slightly smaller font.
\usepackage{fancyvrb}
\fvset{fontsize=\normalsize}

% Small sections of multiple columns
\usepackage{multicol}

% Provides paragraphs of dummy text
\usepackage{lipsum}

% These commands are used to pretty-print LaTeX commands
\newcommand{\doccmd}[1]{\texttt{\textbackslash#1}}% command name -- adds backslash automatically
\newcommand{\docopt}[1]{\ensuremath{\langle}\textrm{\textit{#1}}\ensuremath{\rangle}}% optional command argument
\newcommand{\docarg}[1]{\textrm{\textit{#1}}}% (required) command argument
\newenvironment{docspec}{\begin{quote}\noindent}{\end{quote}}% command specification environment
\newcommand{\docenv}[1]{\textsf{#1}}% environment name
\newcommand{\docpkg}[1]{\texttt{#1}}% package name
\newcommand{\doccls}[1]{\texttt{#1}}% document class name
\newcommand{\docclsopt}[1]{\texttt{#1}}% document class option name


\begin{document}

\begin{titlepage}

\begin{fullwidth}
\noindent\LARGE\emph{Book review
} \hspace{88mm}\includegraphics[height=1cm]{logo3.png}\\
\noindent\hrulefill\\
\vspace*{1em}
\noindent{\Huge{\emph{Reflections on the International Association\\\noindent for Media and Communication Research}\par}}

\vspace*{1.5em}

\noindent\LARGE{reviewed by Michael Meyen} \href{https://orcid.org/0000-0001-6037-5574}{\includegraphics[height=0.5cm]{orcid.png}}\par\marginnote{\emph{Reflections on the International Association for Media and Communication Research}, reviewed by Michael Meyen \emph{History of Media Studies} 3 (2023), \href{https://doi.org/10.32376/d895a0ea.16a75512}{https://doi.org/ 10.32376/d895a0ea.16a75512}. \vspace*{0.75em}}
\vspace*{0.5em}
\noindent{{\large\emph{LMU Munich}, \href{mailto:michael.meyen@ifkw.lmu.de}{michael.meyen@ifkw.lmu.de}\par}} \marginnote{\href{https://creativecommons.org/licenses/by-nc/4.0/}{\includegraphics[height=0.5cm]{by-nc.png}}}

% \vspace*{0.75em} % second author

% \noindent{\LARGE{<<author 2 name>>}\par}
% \vspace*{0.5em}
% \noindent{{\large\emph{<<author 2 affiliation>>}, \href{mailto:<<author 2 email>>}{<<author 2 email>>}\par}}

% \vspace*{0.75em} % third author

% \noindent{\LARGE{<<author 3 name>>}\par}
% \vspace*{0.5em}
% \noindent{{\large\emph{<<author 3 affiliation>>}, \href{mailto:<<author 3 email>>}{<<author 3 email>>}\par}}

\end{fullwidth}

\vspace*{1em}


\noindent Jörg Becker and Robin Mansell, eds., \emph{Reflections on the International Association for
Media and Communication Research: Many Voices, One
Forum}. 558 pp., figs., index. Cham, Switzerland:
Palgrave Macmillan 2023. \$38 (paper)

\newthought{When this book} arrived in my mailbox, I first celebrated. More than 500
pages of the discipline's history. 500 pages of IAMCR history. A brick.
A milestone. One of the most important academic associations in the
field takes five years to delve into its own past and hold up a mirror
to media and communication research. The table of contents is full of
big names and good acquaintances. For someone like me, this is both a
celebration and a promise.

I have to say that I started studying journalism in the communist GDR
and then wanted to understand in the larger Germany why a completely
different academic approach was practiced there under the same name. Why
did the Leipzig professors before 1989 concentrate on journalistic
skills and on optimizing training for editorial departments instead of
asking about media content and media effects like their colleagues in
the West? Why do psychological theories, data analysis, and statistics
dominate in Mainz or Munich? Why is the U.S. considered the ultimate
here and not France or Italy?

I quickly realized that the field's history is worthwhile beyond my own
personal curiosity because there is little competition. Communication is
too far removed from the power pole of the academic field to be relevant
to historians. The only people interested in the subject are members of
the communication community itself. This professional community, in
turn, is too small to afford specialists in

\enlargethispage{2\baselineskip}

\vspace*{2em}

\noindent{\emph{History of Media Studies}, vol. 3, 2023}


 \end{titlepage}

% \vspace*{2em} | to use if abstract spills over



\noindent its own past. In Germany,
for example, most institutes do not even have relevant courses.


This means that the history of the communication field is written by a
few professionals (for example: Dave Park, Jeff Pooley, Pete Simonson,
Christopher Simpson) and by amateurs who are actually at home in other
subdisciplines and who, as a rule, pursue a self-interest when they
become part-time historians. Hanno Hardt (1934--2011), for example, a
professor at the University of Iowa and later in Ljubljana, produced two
books at once to advance what he saw as an overdue paradigm shift in
U.S. mass communication research.\footnote{Hanno Hardt, \emph{Critical
  Communication Studies: Communication, History and Theory in America}
  (London: Routledge, 1992); Hardt, \emph{Social Theories of the Press:
  Constituents of Communication Research, 1840s to 1920s} (Lanham, MD:
  Rowman \& Littlefield, 2001).} Others write history to secure a place
for themselves there. Diversity of sources, self-reflection, and perhaps
even a theoretical perspective that allows scholarly work to be placed
in a context that includes geopolitics and national interests as well as
university structures, relationships with neighboring
disciplines,\footnote{Maria Löblich and Andreas Scheu, ``Writing the
  History of Communication Studies: A Sociology of Science Approach,''
  \emph{Communication Theory} 21, no. 1 (2011).} or biographical
imprints almost inevitably fall by the wayside in such works.

In my own studies, being an outsider as an immigrant from East Germany
helped as much as the media history research I used to advance my
academic career. I had the tools and did not have to promote the subject
as a whole or a particular school of thought, nor mentors, teachers, or
even myself.

Probably this explains why the editors of the brick did not ask me to
contribute. This book is an account that is supposed to put the IAMCR in
a favorable light. The subtitle already testifies to this goal: ``Many
Voices, One Forum'' sounds more like a TV commercial than an academic
endeavor. The way it comes about is also reminiscent of the
collaboration between companies and advertising agencies. IAMCR's
Executive Board appointed a commission at the 2018 annual meeting in
Portland, Oregon, which then sought out topics and authors, repeatedly
consulting with IAMCR leadership. The result is a book that tells how
IAMCR would like to see itself today. This includes the pride in the
MacBride Report and everything that belongs to the New World Information
and Communication Order (NWICO) and the Non-Aligned Movement (five
contributions), as well as the extensive fading out of the competing
organization International Communication Association (ICA) and thus of
the U.S. hegemony in media research.

In their introduction, the two editors concede ``gaps and imbalances''
(xi), but they mean above all ``gender balance'' and the representation
of authors from the Global South (xi). No doubt: Despite all efforts to
be objective and scientifically true, historical research also depends
on social position and personal experience, but historical scholarship
has developed criteria in dealing with sources that allow the reader to
classify and evaluate the results. To put it another way: Actually, it
should not matter whether the history of IAMCR or the communication
field is written by a man from the U.S. or by a woman from Uganda.

How far the transfiguration can go when media researchers become
historians of their own lives can already be studied in this
introduction. Jörg Becker claims that members of the IAMCR who are based
in larger countries ``have tended to have intrinsic, rather than career,
motivations'' (viii). Further in the text, ``It is also precisely the
reason that there have been so many unconditional internationalists in
IAMCR'' (viii). Anyone who has ever been to an ICA annual meeting will
probably shake their head. In my interviews with ICA Fellows, I learned
that careers tend to be a waste product of intrinsic
motivation.\footnote{Michael Meyen, ``57 Interviews with ICA Fellows,''
  \emph{International Journal of Communication} 6 (2012).} The IAMCR has
been accused in many conversations, at least between the lines, of lax
quality standards and the politicization of research.

If you think this is exaggerated, read the article ``IAMCR and Russia,''
written by Kaarle Nordenstreng, born in 1941, longstanding professor in
Tampere, Finland, from 1972 to 1988 Vice-President of IAMCR and, from
1976 to 1990, President of the International Organization of
Journalists, an association dominated by the Eastern Bloc. There is no
doubt that Nordenstreng is an excellent expert on Russia. He has also
been present at almost every meeting in Moscow or St. Petersburg that I
have attended. Nevertheless, why did the History Commission of the IAMCR
not ask a Russian? Nordenstreng's text provides the answer. He seriously
claims that it was only personal interests that led Soviet researchers
to the IAMCR (297), thus defending himself against the accusation of
having been a Trojan horse of the East for almost two decades.

Why is it worthwhile to work on a professional history that goes beyond
justifying one's own decisions and thus legitimizing oneself, and that
is also more than a photo album in which one leafs through on quiet
evenings to remember the good old days? I myself have always learned the
most for my own work when I have asked about dependencies, and here
especially about the instrumentalization of science by politics and
business. This is very well documented for the history of the emergence
of media research in the United States.\footnote{Christopher Simpson,
  \emph{Science of Coercion: Communication Research and Psychological
  Warfare, 1945--1960} (Oxford: Oxford University Press, 1996); Jeff
  Pooley, ``The New History of Mass Communication Research,'' in
  \emph{The History of Media and Communication Research}, ed. David W.
  Park and Jeff Pooley (New York: Peter Lang, 2008).} When one knows the
interest that the military, corporations, and intelligence agencies have
in our work, by no means only in major wars, and how this interest has
corrupted even some of those at the top of the annals of the
professional community, then one is more likely to be immune to overt
and covert attempts at influence.

Many of the authors found by the History Commission of the IAMCR are
either unaware of the state of the art in the field's historiography or
deliberately ignore it. This certainly has to do with the fact that most
of them rarely work on the history of science, and probably also with
the desire to promote their own school, their own country, their own
idols. For a specialist like me, this is most evident in the topics
where I myself have analyzed archives, testimonies, publications. I
could therefore spend hours here on the two contributions about Germany,
but I prefer to point out that there is a contribution about Pakistan,
but none about South Africa or Australia.

This already brings me to the credit side. An account of 500 pages is a
treasure trove for the few people who are specialized on the field's
history, even if the critical distance to the subject matter is missing.
There are the photos, from which every book about the past lives. And
there is plenty of material that will be indispensable as a source for
future trials---from reports on theoretical traditions, countries, and
regions to documentation of institutional changes in the IAMCR to
biographical sketches (George Gerbner, Dallas W. Smythe, Herbert I.
Schiller, Stuart Hall, James Halloran). My personal highlight was the
very personal story by Slavko Splichal, who outlines his own
intellectual journey, providing something every historian needs. That's
why I was able to celebrate once again after reading it.




\section{Bibliography}\label{bibliography}

\begin{hangparas}{.25in}{1} 



Hardt, Hanno. \emph{Critical Communication Studies: Communication,
History and Theory in America}. London: Routledge, 1992.

Hardt, Hanno. \emph{Social Theories of the Press: Constituents of
Communication Research, 1840s to 1920s}. Lanham, MD: Rowman \&
Littlefield, 2001.

Löblich, Maria, and Andreas Scheu. ``Writing the History of
Communication Studies: A Sociology of Science Approach.''
\emph{Communication Theory} 21, no. 1 (2011): 1--21.
\url{https://doi.org/10.1111/j.1468-2885.2010.01373.x}.

Meyen, Michael. ``57 Interviews with ICA Fellows,'' \emph{International
Journal of Communication} 6 (2012): 1460--1886.
\url{https://ijoc.org/index.php/ijoc/article/viewFile/1650/764}.

Pooley, Jeff. ``The New History of Mass Communication Research,'' in
\emph{The History of Media and Communication Research}, edited by David
W. Park and Jeff Pooley, 43--69. New York: Peter Lang, 2008.

Simpson, Christopher. \emph{Science of Coercion: Communication Research
and Psychological Warfare, 1945--1960}. Oxford: Oxford University Press,
1996.



\end{hangparas}


\end{document}