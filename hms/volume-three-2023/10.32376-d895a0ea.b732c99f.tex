% see the original template for more detail about bibliography, tables, etc: https://www.overleaf.com/latex/templates/handout-design-inspired-by-edward-tufte/dtsbhhkvghzz

\documentclass{tufte-handout}

%\geometry{showframe}% for debugging purposes -- displays the margins

\usepackage{amsmath}

\usepackage{hyperref}

\usepackage{fancyhdr}

\usepackage{hanging}

\hypersetup{colorlinks=true,allcolors=[RGB]{97,15,11}}

\fancyfoot[L]{\emph{History of Media Studies}, vol. 3, 2023}


% Set up the images/graphics package
\usepackage{graphicx}
\setkeys{Gin}{width=\linewidth,totalheight=\textheight,keepaspectratio}
\graphicspath{{graphics/}}

\title[Journalism Studies in France and in Germany]{Journalism Studies and journalism education in France and in Germany} % longtitle shouldn't be necessary

% The following package makes prettier tables.  We're all about the bling!
\usepackage{booktabs}

% The units package provides nice, non-stacked fractions and better spacing
% for units.
\usepackage{units}

% The fancyvrb package lets us customize the formatting of verbatim
% environments.  We use a slightly smaller font.
\usepackage{fancyvrb}
\fvset{fontsize=\normalsize}

% Small sections of multiple columns
\usepackage{multicol}

% Provides paragraphs of dummy text
\usepackage{lipsum}

% These commands are used to pretty-print LaTeX commands
\newcommand{\doccmd}[1]{\texttt{\textbackslash#1}}% command name -- adds backslash automatically
\newcommand{\docopt}[1]{\ensuremath{\langle}\textrm{\textit{#1}}\ensuremath{\rangle}}% optional command argument
\newcommand{\docarg}[1]{\textrm{\textit{#1}}}% (required) command argument
\newenvironment{docspec}{\begin{quote}\noindent}{\end{quote}}% command specification environment
\newcommand{\docenv}[1]{\textsf{#1}}% environment name
\newcommand{\docpkg}[1]{\texttt{#1}}% package name
\newcommand{\doccls}[1]{\texttt{#1}}% document class name
\newcommand{\docclsopt}[1]{\texttt{#1}}% document class option name


\begin{document}

\begin{titlepage}

\begin{fullwidth}
\noindent\LARGE\emph{French-German Communication Research
} \hspace{27mm}\includegraphics[height=1cm]{logo3.png}\\
\noindent\hrulefill\\
\vspace*{1em}
\noindent{\Huge{Journalism Studies and Journalism\\\noindent Education in France and in Germany\par}}

\vspace*{1.5em}

\noindent\LARGE{Lisa Bolz} \href{https://orcid.org/0000-0001-5255-6404}{\includegraphics[height=0.5cm]{orcid.png}}\par\marginnote{\emph{Lisa Bolz, ``Journalism Studies and journalism education in France and in Germany,'' \emph{History of Media Studies} 3 (2023), \href{https://doi.org/10.32376/d895a0ea.b732c99f}{https://doi.org/ 10.32376/d895a0ea.b732c99f}.} \vspace*{0.75em}}
\vspace*{0.5em}
\noindent{{\large\emph{Sorbonne University}, \href{mailto:lisa.bolz@sorbonne-universite.fr}{lisa.bolz@sorbonne-universite.fr}\par}} \marginnote{\href{https://creativecommons.org/licenses/by-nc/4.0/}{\includegraphics[height=0.5cm]{by-nc.png}}}

% \vspace*{0.75em} % second author

% \noindent{\LARGE{<<author 2 name>>}\par}
% \vspace*{0.5em}
% \noindent{{\large\emph{<<author 2 affiliation>>}, \href{mailto:<<author 2 email>>}{<<author 2 email>>}\par}}

% \vspace*{0.75em} % third author

% \noindent{\LARGE{<<author 3 name>>}\par}
% \vspace*{0.5em}
% \noindent{{\large\emph{<<author 3 affiliation>>}, \href{mailto:<<author 3 email>>}{<<author 3 email>>}\par}}

\end{fullwidth}

\vspace*{1em}


\hypertarget{abstract}{%
\section{Abstract}\label{abstract}}

Journalism research and journalism education are highly interlinked, but
in each country or in each cultural context there is a certain way of
teaching and researching journalism. France and Germany, despite their
proximity, have two different histories and developments regarding these
topics that have an impact on today's journalism studies in both
countries. While journalism is one of the main research topics in German
communication studies (\emph{Kommunikationswissenschaft}), this is not
the case in France within French communication studies (\emph{Sciences
de l'information et de la communication, SIC}). A look at today's
research topics and perspectives underlines the differences between
these two countries: Whereas German journalism studies are more theory
oriented with some dominant theoretical perspectives and approaches,
French journalism studies are less structured around major theories.


\enlargethispage{2\baselineskip}

\vspace*{18em}

\noindent{\emph{History of Media Studies}, vol. 3, 2023}


 \end{titlepage}

% \vspace*{2em} | to use if abstract spills over

 \hypertarget{introduction}{%
\section{Introduction}\label{introduction}}

Studying journalism in France is not the same as studying the same topic
in Germany. It is not only a matter of language; the differences go well
beyond that, extending to different research traditions, different
turning points, the influence of certain research personalities on the
academic landscape, and certain ways of conceptualizing the research
topic and research methods that vary between the two countries on either
side of the Rhine. Although it seems that France and Germany have
similar media and journalism systems,\footnote{Daniel C. Hallin and
  Paolo Mancini, \emph{Comparing Media Systems: Three Models of Media
  and Politics} (Cambridge: Cambridge University Press, 2004).} a closer
look reveals numerous differences. We find, of course, reasons for these
differences in history, politics, and culture. More precisely, an
analysis of different dynamics in the development of journalism research
and the interplay with academic journalism education allows the
differences that exist today between two research communities to be
better understood. These two research communities rarely cooperate in
the field of communication research in general and journalism research
in particular.\footnote{Stefanie Averbeck-Lietz, Fabien Bonnet, Sarah
  Cordonnier, and Carsten Wilhelm, ``Communication Studies in France:
  Looking for a `Terre du milieu'?'' \emph{Publizistik} 64 (2019).}

Of course, there are many examples of French-German cooperation in the
field of journalism---among them ARTE, binational university degree
programs, programs for young journalists, French-German associations,
etc. However, in the scientific community, few researchers are working
at this intersection.\footnote{Cf. the Franco-German University's
  workshop series for young scholars, ``Exploration transnationale des
  milieux de communication franco-allemands: science, design, culture
  numérique, journalism,'' Program description, the Franco-German
  University (website), last modified July 12, 2021.} The lack of a
French-German dialogue in the field of journalism studies might seem
astonishing from an outside perspective. A look at the history of German
and French journalism studies and journalism education provides a better
understanding of the different pathways that the disciplines have taken
in each country and their impact (or not) on academic journalism
education.

Any research into the epistemology of a research field, the development
of its institutionalization, the evolution of methodological approaches,
and the different scientific discourses requires that the researcher
take a step back and distance themself from their own academic field and
their own academic routines.\footnote{Nicolas Hubé, ``À la recherche
  d'une universalité du journalisme: la \emph{Journalistik}
  allemande,''~\emph{Revue française des sciences de l'information et de
  la communication} 19 (2020).} Especially given the risks of personal
bias, it is impossible for the current study to do justice to more than
a century of journalism research in Germany and in France. This
contribution can only be an attempt to shed light on the parallels and
differences between two research landscapes that are not as homogenous
as they might seem.

The different developments and institutionalizations make it difficult
to name the academic disciplines. The German
\emph{Kommunikationswissenschaft} can be translated as communication
science, especially as both (the German and the Anglo-Saxon strand of
communication science) are epistemologically and methodologically
relatively close. And the German field of journalism research, called
\emph{Journalismusforschung} or \emph{Journalistik} is relatively close
to international journalism studies. A closer look at the editorial
boards of international journals or the representatives of international
associations in the field of journalism studies shows that German
scholars are internationally active and present, which underlines the
similarities between the academic disciplines. It is more difficult to
translate the names of the disciplines on the French side: The
translation of the French \emph{sciences de l'information et de la
communication} is ``sciences of information and communication,'' and
there is no equivalent to journalism studies in French, as will be
explained later. When naming the German and French academic disciplines
in English, one has to keep in mind the complexity behind the
denominations. Even though journalism research preceded the
institutionalized communication studies in both countries, it makes
sense to consider the linkage between the research field of journalism
studies and the contemporary institutionalization of the field within
communication studies in France and in Germany.

This article is a possible response to the main question as to why there
is no German-French dialogue in the field of journalism studies and how
the development of journalism studies and university journalism training
in France and Germany might have an impact on journalism studies. As a
first step, it is important to comprehend the different developments of
journalism research in France and in Germany and to understand the
institutionalization of journalism research as well as the role of
journalism research within communication science. Then, in a second
step, an analysis of the development of journalism research and academic
journalism education reveals a particular relationship between the two:
Their dynamics are different in each country. Finally, in the third
chapter, a closer look at aspects of past and current French and German
journalism research accentuates the differences between them, while also
revealing the possible complementarity as well.

\enlargethispage{\baselineskip}

\hypertarget{the-history-of-journalism-studies-in-france-and-germany}{%
\section{The History of Journalism Studies in France and
Germany}\label{the-history-of-journalism-studies-in-france-and-germany}}

Today, journalism studies is an international field of studies, being
one of the biggest divisions within international scientific
associations such as the ICA or ECREA, and there are research
communities in many countries. But the origins and the development of
journalism studies differ between countries, in this case France and
Germany. A look at the history of the academic disciplines and the place
of journalism studies within the national communication associations
enables us to better understand why there are such different
perspectives on journalism, which even today are hardly perceived in the
other country.

\hypertarget{the-beginnings-in-germany-and-france-in-the-first-half-of-the-twentieth-century-journalism-as-a-new-field-of-study}{%
\subsection{The Beginnings in Germany and
France in the First Half of the\\\noindent Twentieth Century: Journalism as a New
Field of
Study}\label{the-beginnings-in-germany-and-france-in-the-first-half-of-the-twentieth-century-journalism-as-a-new-field-of-study}}

The theories of Max Weber are considered by some journalism scholars to
be a part of German journalism research as Weber's research concerned
journalism and media among other things.\footnote{Siegfried
  Weischenberg, \emph{Max Weber und die Entzauberung der Medienwelt:
  Theorien und Querelen---eine andere Fachgeschichte} (Wiesbaden:
  Springer, 2012), 14; Gilles Bastin, ``La presse au miroir du
  capitalisme moderne: Un projet d\textquotesingle enquête de Max Weber
  sur les journaux et le journalisme,'' \emph{Réseaux} 109, no. 5
  (2001).} In 1910, Weber conceptualized and presented a research plan
to study the press from a sociological point of view.\footnote{Weischenberg,
  78.} Even though the study was never completed, it can be interpreted
as the first demand for journalism to be studied using empirical
research methods. His wish was first fulfilled approximately sixty years
later.\footnote{Martin Löffelholz, ``Theorien des Journalismus: Eine
  historische, metatheoretische und synoptische Einführung,'' in
  \emph{Theorien des Journalismus: Ein diskursives Handbuch}, ed. Martin
  Löffelholz (Wiesbaden: Springer, 2004), 21.} At first, academic
research into journalism and newspapers, the so-called
\emph{Zeitungswissenschaft} (``science of newspapers''), had a
historical and normative approach and journalism was considered to be
the result of the talent and the capabilities of a few
individuals.\footnote{Löffelholz, 39.}

Thoughts on journalism go well beyond the foundation of the academic
institutions. Even though different authors throughout the centuries
wrote about communication, newspapers, or journalism, the first
systematic approach to journalism in Germany (which did not yet exist as
a nation-state at this time) might be the text ``Über Zeitungen'' (About
Newspapers) written by Joachim von Schwarzkopf in 1795.\footnote{Heinz
  Pürer, ``Zur Fachgeschichte der Kommunikationswissenschaft in
  Deutschland,'' \emph{Biographisches Lexikon der
  Kommunikationswissenschaft} (October 2017).} Several authors, such as
Karl Knies and Robert Prutz, during the nineteenth century analyzed the
new profession that emerged at this time and provided academic texts on
topics like press legislation, the formation of public opinion, the
history of German journalism, and symbolic communication.\footnote{Hanno
  Hardt, \emph{Social Theories of the Press: Constituents of
  Communication Research, 1840s to 1920s} (Lanham, MD: Rowman \&
  Littlefield, 2002); Jürgen Wilke, ``Von der Zeitungskunde zur
  Integrationswissenschaft: Wurzeln und Dimensionen im Rückblick auf
  hundert Jahre Fachgeschichte der Publizistik-, Medien- und
  Kommunikationswissenschaft in Deutschland,'' \emph{M\&K} 64, no. 1
  (2016).} In France, journalism was also the topic of several
publications throughout the centuries about the role of journalists or
press history, demonstrating that there was already interest in studying
journalism, both academically and scientifically, in both countries
before the foundation of the first academic departments.

The first German journalism departments were founded at the beginning of
the twentieth century. In 1916, Karl Bücher founded the first German
department in the large field of communication studies. The Institut für
Zeitungskunde in Leipzig offered an academic program for future
journalists as well as a place to study journalism and
newspapers.\footnote{Erik Koenen, ed., \emph{Die Entdeckung der
  Kommunikationswissenschaft: 100 Jahre kommunikationswissenschaftliche
  Fachtradition in Leipzig; Von der Zeitungkunde zur Kommunikations- und
  Medienwissenschaft} (Köln: Herbert von Halem Verlag, 2016).} The
second department was founded three years later in Münster (1919), the
so-called Lektorat für Zeitungskunde under the direction of the
journalist Friedrich Castelle and with Karl D'Ester as an academic
teacher.\footnote{Bettina Maoro, \emph{Die Zeitungswissenschaft in
  Westfalen 1914--45: Das Institut für Zeitungswissenschaften in Münster
  und die Zeitungswissenschaft in Dortmund} (Munich: K.G. Saur, 1987).}
Both are today two of the biggest departments for communication studies
in Germany. In both cases, the scientific interest in journalism
resulted in the creation of departments that would soon work not only on
journalism research but on communication as well.

The lack of empirical studies on journalism and journalists in the early
phase of the German newspaper science called \emph{Zeitungswissenschaft}
was a topic that caused Ferdinand Tönnies in 1930 to publicly criticize
the lack of a sociological approach in the \emph{Zeitungswissenschaft}
and to describe it as a part of sociology. Emil Dovifat, however, was of
the opinion that \emph{Zeitungswissenschaft} should be an independent
academic discipline and couldn't imagine it being studied as a mere
aspect of different disciplines such as sociology, literature,
economics, and psychology.\footnote{Emil Dovifat,
  \emph{Zeitungswissenschaft, Band I: Allgemeine Zeitungslehre} (Berlin:
  Walter de Gruyter \& Co., 1931).} The dispute showed the first steps
towards the institutionalization of journalism studies as an academic
discipline and the emergence of a scientific community. From that
beginning until 1960, the German \emph{Zeitungswissenschaft} passed
through four stages: Between 1890 and 1925, scholars began to identify
the newspaper as their common research object. During the second stage
(1925--1933), researchers intensified research on newspapers and started
to create a scientific network around this object of study. The third
phase (1933--1945) was overshadowed by World War II and Nazi ideology,
and so the fourth stage was about the reconstruction of journalism
research after 1945.\footnote{Stefanie Averbeck and Arnulf Kutsch,
  ``Thesen zur Geschichte der Zeitungs- und Publizistikwissenschaft
  1900--1960,'' \emph{Medien \& Zeit} 17, no. 2--3 (2002); Stefanie
  Averbeck and Arnulf Kutsch, \emph{Zeitung, Werbung, Öffentlichkeit:
  Biographisch-systematische Studien zur Frühgeschichte der
  Kommunikationsforschung} (Köln: Herbert von Halem Verlag, 2005), 12.}

In the first half of the twentieth century, communication studies or
journalism studies did not exist as an institutionalized research field
in France, but different projects and initiatives nonetheless emerged as
a result of academic interest in journalism. In 1937, the then
Université de Paris founded the Institute of Press Science (Institut de
Science de la Presse, ISP), which disappeared during World War II and
was again founded in 1945, becoming in 1951 the French Press Institute
(Institut Français de Presse, IFP), one of the leading French research
centers for journalism and media studies today. Its founding father and
first director, Fernand Terrou, sought international exchanges and
participated in a conference in Strasbourg where he was part of an
``Interim Committee'' that convened the founding conference of the
International Association for Media and Communication Research (IAMCR)
at UNESCO in December 1957.\footnote{Cees Hamelink and Kaarle
  Nordenstreng, ``Overview of IAMCR History: Looking at History through
  the International Association for Media and Communication Research
  (IAMCR),'' IAMCR (website).} Terrou then became the first president of
the IAMCR. In 1946, Terrou also founded the French academic journal for
journalism research, \emph{Études de presse,} which he later co-edited
with other IFP members.

It is notable that the early French journalism research was
internationally oriented, with Terrou being involved in international
associations and with UNESCO's Department of Mass Communication, for
example.

\newpage\hypertarget{new-developments-in-the-1960s-a-turning-point-in-germany-a-new-institutionalized-field-of-study-in-france}{%
\subsection{New Developments in the 1960s: A
Turning Point in Germany,\\\noindent a New Institutionalized Field of Study in
France}\label{new-developments-in-the-1960s-a-turning-point-in-germany-a-new-institutionalized-field-of-study-in-france}}

Although German journalism studies has always been a field of research
that integrates different fields of expertise and academics, in 1960
Werner Schöllgen introduced the notion of ``integrating science''
(\emph{Integrationswissenschaft})\footnote{Wilke, 76.}---meaning an
academic discipline that would include research approaches and
perspectives from different fields of research such as sociology,
history, or political science. The 1960s also marked a turning point in
German journalism studies, which transitioned from a normative human
science (\emph{Geisteswissenschaft}) to an academic field that
incorporated sociological and psychological perspectives as well as
the---at that time in particular, quantitative---methods of empirical
social research.\footnote{Löffelholz, 46.} The inspiration came from
observation of US communication research, which led German journalism
studies to reposition itself as a social science.\footnote{Maria
  Löblich, \emph{Die empirisch-sozialwissenschaftliche Wende in der
  Publizistik- und Zeitungswissenschaft} (Köln: Herbert von Halem
  Verlag, 2010).} Different researchers described this transition as a
liberating and new and necessary recommencement after World War II,
while others are still criticizing the neglect of certain perspectives
and research methods.\footnote{Löblich, 13--14.}

In France, the first center for communication studies (Centre d'Études
des Communications de Masse, CECMAS) was founded in 1960 with the aim of
conducting research into ``massive phenomena of our contemporary society
such as press, radio, television, cinema, advertising'' with all their
different dimensions: ``economic, sociological, ideological, even
anthropological.''\footnote{Roland Barthes, ``Le centre
  d\textquotesingle études des communications de masse: Le C.E.C.MAS,''
  \emph{Annales: Economies, sociétés, civilisations} 16, no. 5 (1961):
  991.} Methodologically, the research center concentrated on content
analysis, but Roland Barthes indicated that this would not be sufficient
as the so-called mass media had a language that needed to be analyzed
along with the content.\footnote{Barthes, 992.} Despite collaborative
work on popular culture, researchers did not want to found an academic
discipline that would concentrate on communication phenomena.\footnote{Stefanie
  Averbeck, ``Über die Spezifika `nationaler Theoriediskurse':
  Kommunikationswissenschaft in Frankreich,'' in \emph{Theorien der
  Medien- und Kommunikationswissenschaft: Grundlegende Diskussionen,
  Forschungsfelder und Theorieentwicklungen}, ed. Carsten Winter,
  Andreas Hepp, and Friedrich Krotz (Wiesbaden: Verlag für
  Sozialwissenschaften, 2008).} But the CECMAS paved the way for the
later establishment of the new academic discipline. Roland Barthes, for
example, would be one of the founding members of the French
\emph{sciences de l'information et de la communication} as an academic
field that was officially created in 1975, and CECMAS's academic
journal, \emph{Communications}, is to this day one of the major French
journals in communication science.

\hypertarget{the-science-of-communication-and-journalism-studies-since-the-1970s}{%
\subsection{The Science of Communication and
Journalism Studies since\\\noindent the
1970s}\label{the-science-of-communication-and-journalism-studies-since-the-1970s}}

As media forms expanded, so too did the field of communication studies,
and journalism was no longer the only center point in Germany. With the
renaming of the national association from ``Deutsche Gesellschaft für
Publizistik- und Zeitungswissenschaft'' (created in 1963)---with an
emphasis on journalism---to ``Deutsche Gesellschaft für Publizistik- und
Kommunikationswissenschaft'' (1972), communication science in Germany
formally became a field of study that covered a huge variety of topics.

The historical development in Germany from journalism to the wider field
of communication was not mirrored in France. The French \emph{sciences
de l'information et de la communication} (SIC) were founded by Roland
Barthes, Robert Escarpit, and Jean Meyriat and have a strong literature
tradition.\footnote{Jean-François Tétu, ``Sur les origines littéraires
  des sciences de l'information et de la communication,'' in \emph{Les
  origines des sciences de l'information et de la communication: Regards
  croisés}, ed. Robert Boure (Lille: Presses universitaires du
  Septentrion, 2002).} During the first conference, Escarpit explained
the new way of thinking and analyzing information and communication:
information as data and communication as a permanent process.\footnote{Robert
  Escarpit, ``Pour une nouvelle épistémologie de la communication''
  (introductory presentation, Premier congrès français des sciences de
  l'information et de la communication {[}first convention of the French
  information and communication sciences conference{]}, Compiègne, April
  21, 1978).} The pluralistic-disciplinary approach to communication
changed into a discipline that distinguished itself from others while
still maintaining a pluralistic-theoretical interdisciplinary approach.
It was not the research topic---communication---that was (and is)
specific to the new field of study, but how researchers think and work
in their discipline. Yves Jeanneret and Bruno Ollivier summarize this by
focusing on two main aspects: the idea that information and
communication are one ensemble and an emphasis on interdisciplinary
research.\footnote{Yves Jeanneret and Bruno Ollivier, ``Introduction:
  Les Sic en perspective,'' \emph{Hermès} 38 (2004): 88.} Specific to
French communication research is the pragmatic-cultural-semiotic
approach to analyzing simultaneous interaction on different levels, such
as public, interpersonal, or mediated communication.\footnote{Averbeck,
  212.}

Löffelholz describes Manfred Rühl's ``Die Zeitungsredaktion als
organisiertes soziales System'' (The Newsroom as an Organized Social
System) (1969) as the turning point in German journalism theory. Instead
of an individualistic approach---that is to say, journalists in the
center of research---Rühl describes journalism as a system.\footnote{Löffelholz,
  53.} According to Rühl, newsrooms are based on the structures of roles
within the newsroom and decision-making processes during the
work.\footnote{Manfred Rühl, \emph{Die Zeitungsredaktion als
  organisiertes soziales System} (Berlin: Bertelsmann
  Universitätsverlag, 1969).} This meant no longer seeing journalism as
a purely talent-based profession and seeing newspapers within their
societal structures. Journalism research in Germany thus converted from
an ideology regarding the profession to ``modern empirical-analytical
journalism research.''\footnote{Löffelholz, 46.}

Journalism studies in France, in contrast, is a quite recent field of
studies, established for the most part in the 1990s.\footnote{Nicolas
  Pélissier and François Demers, ``Recherches sur le journalisme: Un
  savoir dispersé en voie de structuration,'' \emph{Revue française des
  sciences de l'information et de la communication} 5 (2014).} At that
time, different researchers in France studied the development,
institutionalization, and societal challenges of journalism and the
journalistic profession during the first half of the twentieth century.
Denis Ruellan underlines that the first studies in the 1990s
concentrated on different aspects of journalism while highlighting that
journalists functioned as a ``group''\footnote{Denis Ruellan, \emph{Les
  ``Pro'' du journalisme: De l\textquotesingle état au statut, la
  construction d\textquotesingle un espace professionnel} (Rennes:
  Presses universitaires de Rennes, 1997).} with its own structures and
a voluntarily composed collective dynamic. While journalism was the
starting point of the history of communication science in Germany, it
was the opposite in France: A specific interest in journalism emerged
approximately three decades after the creation of the CECMAS. Journalism
studies as such still do not exist in France, which results in the lack
of visibility of journalism research in France.\footnote{Pélissier and
  Demers, para. 5.}

Nicolas Pélissier and François Demers identify three time periods in the
development of French journalism research. During the first phase
(1937--1976), researchers developed a common understanding of press and
journalism, even though they marginalized journalism research at the
same time. The French Institute for Press (Institut Français de Presse,
IFP) played a crucial role during the first period, as it was one of the
few places dedicated to journalism and press with the first doctoral
dissertations on journalism in France in the 1970s. The second stage
(1976--1996) was characterized by the affirmation of academic knowledge
on journalism. Researchers began to work on different topics within
journalism research and they started to interact and align with
researchers abroad. Topics of research included other media as well,
especially television. Even though different academic disciplines
(political science, sociology, economics, linguistics, etc.) began
working on journalism and contributed to a dispersed field of research,
the second phase of journalism research at the beginning of the 1990s is
marked by its initial structuring. The authors see the beginning of the
third phase in journalism studies as being kickstarted by Pierre
Bourdieu, who presented and published his famous \emph{Sur la
télévision} in 1996. (See Benjamin Krämer's
\href{https://hms.mediastudies.press/pub/kramer-bourdieu/}{contribution}
in this Special Section.) Different research methods and perspectives
were applied to journalism research (content analysis, discourse
analysis, qualitative research methods, narratology, anthropological and
experimental approaches, constructivism). In contrast to Germany, French
journalism research generally uses qualitative methods to understand the
``mechanisms of journalistic productions as a collective
action.''\footnote{Hubé, para. 21.} Since 1996, different coalitions and
cooperation have enabled new dynamics within the field of journalism
studies, which will be explained in the next chapter.

\hypertarget{journalism-research-and-the-structure-of-academia-in-germany-and-france}{%
\subsection{Journalism Research and the
Structure of Academia in Germany\\\noindent and
France}\label{journalism-research-and-the-structure-of-academia-in-germany-and-france}}

German communication studies in general and journalism studies in
particular are much more internationally oriented than in France. Many
in Germany (and German-speaking colleagues) are on the editorial boards
of the international journals and are representatives of different
associations and divisions on an international level. Without diving
deeper into the different ways in which international careers and
publication activities are recognized by the scientific community in
France and in Germany, it is obvious that international activities are
very visible within the German community of communication science,
whereas international activities in France have traditionally been
limited to Francophone communities.\footnote{However, the SFSIC has
  institutionalized links outside these historic communities in the past
  ten years, connecting with other national associations, such as DGPuK
  and SGKM, as well as with international associations such as ICA and
  ECREA.}

The German association of communication studies (DGPuK) holds an annual
conference, and each of its divisions, such as the division of
journalism studies, holds an annual conference as well. As many German
scholars attend international conferences (ICA, ECREA) as well as local
ones, and journalism studies divisions exist both internationally and
locally, it is quite easy to identify the scholars that are working in
the same field. The annual meetings enable researchers to create bonds
beyond the presentations. The German journalism division (Fachgruppe
Journalistik/Journalismusforschung) was founded in 1991 and is one of
the biggest divisions within the DGPuK. A statement describing the
group's aims lists a wide range of interests that researchers are
working on, such as: ``journalistic practices and contents, the
structures that shape journalism, the general framework of journalism
and its role in society as well as the relationship between journalism
and its public.''\footnote{DGPuK, ``Selbstverständnis der
  DGPuK-Fachgruppe Journalistik/Journalismusforschung'' (mission
  statement, Hamburg, September 24, 2020).} In other words, this
division is for researchers who are working on theory and on empirical
questions regarding journalism, on the academic and practical education
of journalists, as well as on the application of scientific knowledge in
journalistic practice. All in all, \emph{Journalistik}---German
journalism studies---combines ``different theoretical perspectives with
a variety of empirical and normative approaches.''\footnote{DGPuK,
  ``Selbstverständnis.''} The authors of the statement underlined the
importance of international research and debates; however, in reality,
researchers tend to prioritize exchange between countries in which the
research approach is similar and where the main research language is
English. This is certainly one of the main reasons why there is little
French-German exchange within journalism studies.

Even though the SFSIC (Société Française des Sciences de l'Infor-\\\noindent mation
et de la Communication) is the singular national communication
association in France and plays an important role in planning and
organizing academic careers in the field of communication science,
researchers do not necessarily connect via events organized by the
SFSIC. There are some formal interest groups, called GER (Groupes
d'Etudes et de Recherche), but they have to be renewed every two years
after examination. There are not as many formal and institutionalized
divisions as in Germany, and there is no French interest group dedicated
to journalism research (yet), nor are there annual conferences,
dissertation prizes, junior groups, and the like comparable to those
that exist within the ICA and the ECREA. Academic research in France is
organized in research units (\emph{laboratoires de recherche}) that
organize smaller conferences, research seminars, or working groups. In
the field of journalism studies---which does not exist as such in
France---there are two working groups or consortiums that should be
mentioned to give an idea of research dynamics on journalism in France:
REJ and the GIS Journalisme.

The REJ (Réseau d'études sur le journalisme) is an international
research network of journalism scholars spanning France, Quebec, Brazil,
and Mexico that was created in 1999 with the aim of developing a
theoretical framework in order to unite different methodological
approaches. Hence, all kinds of (new) journalism and (new) journalism
practices were taken into consideration, without establishing a specific
definition of journalism.\footnote{Fabio Henrique Pereira, Tredan
  Olivier, and Langonné Joël, ``Penser les mondes du journalisme,''
  \emph{Hermès} 82, no. 3 (2018): 101--2. See also Roselyne Ringoot and
  Jean-Michel Utard, ``Introduction,'' in \emph{Le journalisme en
  invention: Nouvelles pratiques, nouveaux acteurs} (Rennes: Presses
  universitaires de Rennes, 2006).} The GIS Journalisme (Groupement
d'Intérêt Scientifique Journalisme) was created in 2010 as French
journalism scholars wished to give more visibility and vitality to
journalism research in France, which at that point in time was done by
individuals scattered around France without any clear common structure.
Its founding members came from four French universities: CARISM
(Paris-Panthéon-Assas University), CRAPE (University of Rennes 1), ELICO
(Lumière University Lyon 2), and GRIPIC (Sorbonne University). The
members of these four distinct research units together organized five
conferences between 2011 and 2017, allowing scholars to exchange their
knowledge of this relatively new field of study. There are, of course,
other research collectives, such as the Brazil-France-Francophone
Belgium Journalism Research Conference, or the French-Brazilian
collaboration called MEJOR, or the conferences for young scholars in
journalism (\emph{Jeunes chercheur·es en journalisme}). In addition, the
French-Québecois journal \emph{Les Cahiers du Journalisme}, as well as
the international and multilingual (English, French, Portuguese, and
Spanish) journal \emph{Sur le journalisme, About journalism, Sobre
jornalismo} provide other ways of increasing the visibility of French
journalism studies.

Whereas in Germany journalism studies is a major field of study in
communication science with one formal division within the national
association of communication studies, French journalism research and the
visibility of a French journalism academic community relies more on
individual initiative and efforts. The frequently deplored lack of
international visibility of French journalism scholars can be explained
by the structure of the French academic system and by how academic
recruitment works in France. As young scholars in France do not
necessarily need international experience or international publications
to get a tenured academic position (though this might be an asset and it
becomes more and more important as labs need international competence to
have good evaluations), there is no specific need to spend time abroad
or to publish internationally.

Of course, each country has its own way of creating and structuring
scientific communities, and France and Germany are only two examples
that illustrate how scholars and exchange are organized within the field
of journalism studies. The structure of the main scientific association,
the way recruitment works in different countries, and the role and
importance of certain individuals regarding the structure of the
scientific community all have an impact on research topics and research
habits as well. In the case of France and Germany, it is striking to see
how journalism studies and communication studies have had such different
dynamics in their development; and these were caused by differences in
the history of the academic disciplines in each country, the point of
view and influence of certain individuals, the timing of certain
decisions, and traditions within the scientific communities.

\hypertarget{journalism-education-in-france-and-germany}{%
\section{Journalism Education in France and
Germany}\label{journalism-education-in-france-and-germany}}

Both in France and in Germany, there are many ways to become a
journalist and there is no (formal) need to have a certain degree or to
have studied at a journalism school. There are more or less common ways
to enter the profession of journalism, but in both countries, there are
numerous examples of journalists who have not followed a journalistic
degree program or any other kind of formal professional training.
However, the connection between university journalism education and
journalism studies in France and in Germany differs slightly and this
difference is indicative of variations within journalism studies in each
country. In France, journalism studies is strongly linked to journalism
education and related activities. Before taking a closer look at
journalism studies and its link to journalism education, it is important
to understand the main ways of becoming a journalist in France and in
Germany, since academic journalism education doesn't have the same
significance in both countries.

\hypertarget{becoming-a-journalist-in-germany-choosing-between-traineeship-journalism-schools-and-academic-journalism-education}{%
\subsection{Becoming a Journalist in Germany:
Choosing between Traineeship,\\\noindent Journalism Schools, and Academic
Journalism
Education}\label{becoming-a-journalist-in-germany-choosing-between-traineeship-journalism-schools-and-academic-journalism-education}}

In Germany, there are three main ways to become a journalist: A
traineeship (\emph{Volontariat}) at a media organization, which usually
takes about eighteen to twenty-four months, enables prospective
journalists to learn and work in a newsroom and to contribute directly
to the daily journalistic tasks. Further training complements the
professional experience, even though the standards and benefits can
vary.\footnote{Matthias Kurp, ``Volontariat: Reformstau auf dem
  Königsweg: Diskussion mit Michael Geffken, Annette Hillebrand,
  Christian Lindner, Ulrich Pätzold, Maximiliane Rüggeberg,'' in
  \emph{Dokumentation: IQ-Herbstforum; Qualität und Qualifikation:
  Impulse zur Journalistenausbildung} (Berlin, 2013), 24.} The second
path to a career as a journalist in Germany is through journalism
school. Many of these were founded in the 1980s when different media
organizations decided to take care of the education of their future
colleagues. In addition, there are journalism schools that belong to an
association or an institution.\footnote{Walther von La Roche, Gabriele
  Hooffacker, and Klaus Meier, ``Journalistenschulen,'' in
  \emph{Einführung in den praktischen Journalismus: Journalistische
  Praxis} (Wiesbaden: Springer, 2013).} Studying at one of these
journalism schools sometimes does not require a degree, but
realistically, the candidates need a university degree in order to enter
the most prestigious schools, such as the German Journalism School in
Munich (Deutsche Journalisten Schule). The third main way to get
journalism training is by studying at one of the universities that offer
academic journalism training (\emph{Journalistik}). In the 1970s a
greater exchange regarding journalism education took place with the aim
to reform journalism training and to install university degrees that
would meet the new demands required of journalists. The universities of
Dortmund and Munich were the first to really combine theory and
practice. These are the three most common pathways to a career in
journalism, but there are many other possibilities and smaller programs
for professional journalism education, too.

\hypertarget{becoming-a-journalist-in-france-fourteen-elite-schools-for-journalism-education}{%
\subsection{Becoming a Journalist in France:
Fourteen Elite Schools for\\\noindent Journalism
Education}\label{becoming-a-journalist-in-france-fourteen-elite-schools-for-journalism-education}}

In France, the main pathway into journalism is by studying journalism at
a university. Even though there are over a hundred journalism
degree--granting programs in France, many journalists hope to break into
the industry by studying at one of the fourteen journalism schools that
are ``recognized'' by the profession (``\emph{reconnues par la
profession}'')---even though only about 20 percent of new journalists
come from one of these schools.\footnote{Ivan Chupin, \emph{Les écoles
  du journalisme: Les enjeux de la scolarisation d\textquotesingle une
  profession (1889--2018)} (Rennes: Presses universitaires de Rennes,
  2018), 9.} The ``recognition'' by the profession means that the
degrees at those fourteen schools are more valued by media organizations
than other degree programs, which means that graduates have better
prospects in journalism. As most of these schools offer a master's
degree, students in France start their journalism training right after a
three-year bachelor's degree.

\hypertarget{history-of-journalism-education-and-journalism-studies}{%
\section{History of Journalism Education and Journalism
Studies}\label{history-of-journalism-education-and-journalism-studies}}

It is, of course, impossible to take into consideration all of the
journalism degree programs in these two countries. In the following
sections, I will return to just a few aspects of journalism education
that are directly linked to journalism studies---which is to say,
university journalism programs in Germany and the journalism schools in
France.

\hypertarget{the-history-and-development-of-academic-journalism-education-in-germany}{%
\subsection{The History and Development of
Academic Journalism Education\\\noindent in
Germany}\label{the-history-and-development-of-academic-journalism-education-in-germany}}

Both in France and in Germany, the first attempts to establish
journalism education go back to the nineteenth century. Authors and
journalists wrote about their experiences in journalism and gave advice
to future journalists. J. H. Wehle, for example, gave insights into
journalism and newspapers as well as the work of journalists in order to
teach a new generation of journalists that would be able to meet new
demands in the journalistic field.\footnote{J. H. Wehle, \emph{Die
  Zeitung: Ihre Organisation und Technik} (Vienna: A. Hartleben's
  Verlag, 1883).} And in 1899, Richard Wrede founded a private
journalism school in Berlin and published a handbook shortly afterwards
in 1902.\footnote{Martin Löffelholz, ``Kommunikatorforschung:
  Journalistik,'' in \emph{Öffentliche Kommunikation: Studienbücher zur
  Kommunikations- und Medienwissenschaft}, ed. Günter Bentele,
  Hans-Bernd Brosius, and Otfried Jarren (Wiesbaden: Verlag für
  Sozialwissenschaften, 2003), 29.} Despite these early attempts and the
journalism focus of the first academic institutions that today are among
the biggest departments for communication studies, Günter Kieslich's
working paper from 1970 on problems of journalistic education
(``Probleme der journalistischen Aus- und Fortbildung'') marks the
beginning of academic journalism education in Germany.\footnote{Michael
  Steinbrecher, ``Alte Werte, neue Kompetenzen: was sich in der
  Journalistenausbildung ändern muss,'' in \emph{Dokumentation:
  IQ-Herbstforum; Qualität und Qualifikation: Impulse zur
  Journalistenausbildung} (Berlin, 2013), 11; Walter Hömberg,
  ``Expansion und Differenzierung: Journalismus und
  Journalistenausbildung in den vergangenen drei Jahrzehnten,'' in
  \emph{Journalistenausbildung für eine veränderte Medienwelt:
  Diagnosen, Institutionen, Projekte}, ed. Klaus-Dieter Altmeppen and
  Walter Hömberg (Wiesbaden: Westdeutscher Verlag, 2002), 18.} This text
led to the ``Memorandum of Journalism Education'' (``Memorandum der
Journalistenausbildung'') that the German Press Council published one
year later, describing journalism education in Germany and asking for
specialized university degrees and academic professional training. A
second memorandum two years later (1973) insisted even more on the
necessity of (applied) academic journalism education and marked a
turning point regarding the perception of the profession of journalists,
which was now no longer considered purely talent based.\footnote{Wolfgang
  Donsbach, ``Hausaufgaben noch immer nicht gemacht: Versäumnisse und
  Konzepte der Journalismusforschung,'' in \emph{Didaktik der
  Journalistik: Konzepte, Methoden und Beispiele aus der
  Journalistenausbildung}, ed. Beatrice~Dernbach and Wiebke~Loosen
  (Wiesbaden: Springer, 2012), 33.}

These texts were the beginning of a larger discourse on academic
journalism education. The German association for communication studies
(DGPuK) organized a first conference on this topic in 1976, at which the
curricula from three universities (Munich, Dortmund, Hohenheim) that
combined theory and practice were presented.\footnote{Hömberg, 18.}
Different German universities developed different ways of incorporating
applied journalism education into their curricula. It is impossible to
explain every characteristic of all these degree programs as they are
quite different, but two German universities have each implemented their
own interesting program in an effort to combine theory and practice: the
University of Dortmund and the University of Munich. The so-called
Dortmund Model (Dortmunder Modell) favored the ``integration of theory
and practice''\footnote{Steinbrecher, 13.} by integrating a practical
traineeship (\emph{Volontariat}) into the academic curriculum that
allows the department to build extended partnerships with media
organizations in order to regularly adapt the study program.\footnote{Ulrich
  Pätzold, ``Die Anfänge in Dortmund: eine Erfolgsgeschichte mit viel
  Glück,'' in \emph{Journalismus und Öffentlichkeit}, ed. Tobias
  Eberwein and Daniel Müller (Wiesbaden: Verlag für
  Sozialwissenschaften, 2010).} In Munich, the department for
communication research cooperates with the German journalism school
(Deutsche Journalistenschule).\textsuperscript{48}

In the\marginnote{\textsuperscript{48}\setcounter{footnote}{48} Michael Meyen and Manuel
  Wendelin, eds., \emph{Journalistenausbildung, Empirie und
  Auftragsforschung: Neue Bausteine zu einer Geschichte des Münchners
  Institution für Kommunikationswissenschaft; Mit einer Bibliographie
  der Dissertationen von 1925 bis 2007} (Köln: Herbert von Halem Verlag,
  2008).} 1990s, Siegfried Weischenberg, Klaus-Dieter Altmeppen, and Martin
Löffelholz described the main skills that journalists would need in
order to be professional journalists. They identified journalistic
knowledge (\emph{Fachkompetenz}) as important---a skill which included,
for example, the ability to investigate, to select the right news item,
and to write news, as well as knowledge about media economics, media
politics, media law, media history, and media technique. The second
competence these authors describe is intermediation
(\emph{Vermittlungskompetenz}), which they specify is the ability to
articulate and to present news for a certain public or audience, as well
as knowledge of genres and formats. The third competence refers to
specific knowledge (\emph{Sachkompetenz}) that journalists need in order
to write news for society: knowledge about their specialty and knowledge
of societal issues (sociology, politics, etc.), as well as knowledge of
sources, scientific work, and research methods.\footnote{Siegfried
  Weischenberg, Klaus-Dieter Altmeppen, and Martin Löffelholz, \emph{Die
  Zukunft des Journalismus: Technologische, ökonomische und
  redaktionelle Trends} (Opladen: Westdeutscher Verlag, 1994), 48.}
Later researchers extended this framework by adding further competences
that became important as a result of changes in the media landscape.
Some of these additions included technical and entrepreneurial
competences\footnote{Löffelholz, ``Theorien des Journalismus,'' 29.} and
an understanding of professional values. Explaining why prospective
journalists would need specific instruction and how this knowledge would
be necessary for working as a journalist was important to Wolfgang
Donsbach, who called for interdisciplinary ``team-teaching.''\footnote{Donsbach,
  42.} All in all, there is a rich tradition of academic exchange about
journalism education in Germany which attempts to combine theory and
practice.

The question of how to integrate theory and practice is one of the main
concerns of academic journalism education. Whereas journalism schools do
not have to justify their curriculum, since their main goal (and their
legitimation) is the professional education of future
journalists,\footnote{Klaus-Dieter Altmeppen and Walter Hömberg,
  ``Traditionelle Prämissen und neue Ausbildungsangebote: Kontinuitäten
  oder Fortschritte in der Journalistenausbildung?'' in
  \emph{Journalistenausbildung für eine veränderte Medienwelt:
  Diagnosen, Institutionen, Projekte}, ed. Klaus-Dieter Altmeppen and
  Walter Hömberg (Wiesbaden: Westdeutscher Verlag, 2002), 9.} it is more
complicated for academic or university journalism training programs,
which try to offer practical training through external teachers and
exchanges or internships with media companies, for example. Even though
the metaphor of a zipper\footnote{Michael Haller, ``Didaktischer
  Etikettenschwindel? Die Theorie-Praxis-Verzahnung in der
  Journalistik,'' in \emph{Didaktik der Journalistik: Konzepte, Methoden
  und Beispiele aus der Journalistenausbildung}, ed. Beatrice Dernbach
  and Wiebke Loosen (Wiesbaden: Springer, 2012), 48.} describes the
ambition of combining theory and practice, practical elements are
usually additional to the theoretical studies rather than being really
integrated into the discussion about journalism.\footnote{Altmeppen and
  Hömberg, 9.}

\hypertarget{the-history-and-the-development-of-journalism-schools-in-france}{%
\subsection{The History and the Development of
Journalism Schools in
France}\label{the-history-and-the-development-of-journalism-schools-in-france}}

Even though Delphine Girardin wrote a theatrical piece about a
journalism school in 1839, and she was not the only one during the
nineteenth century to envision journalism education in
France,\footnote{Thomas Ferenczi, \emph{L'invention du journalisme en
  France: Naissance de la presse moderne à la fin du XIXème siècle}
  (Paris: Payot, 1996), 251--52; Christophe Charle, \emph{Le siècle de
  la presse (1830--1939)} (Paris: Seuil, 2004), 217--21.} the first
French journalism school was founded in 1899 by Jeanne Weill, better
known as Dick May.\textsuperscript{56} This happened at a time when journalism
had a great impact\marginnote{\textsuperscript{56}\setcounter{footnote}{56} Vincent Goulet, ``\,`Transformer la société
  par l'enseignement social': La trajectoire de Dick May entre
  littérature, sociologie et journalisme,'' \emph{Revue d'Histoire des
  Sciences Humaines} 19 (2008).} on French society (e.g., the Dreyfus affair) and when
journalism underwent many changes---for example, the shift from
political and literature journalism towards reportage journalism and a
more popular journalism for a broad readership. The peak of this
development was during the so-called golden press era (``\emph{âge d'or
de la presse}'').\footnote{Michael Palmer, ``L'âge d'or de la presse,''
  \emph{Le Temps des médias} 27, no. 2 (2016).} Strictly speaking, the
first journalism school was one of four departments of the School of
Social Science (École des hautes études sociales), but it was less
popular than the other departments. Dick May was interested in
journalism and the emerging field of sociology and combined both by
teaching the newly founded social science within the journalism
program.\footnote{Vincent Goulet, ``Dick May et la première école de
  journalisme en France: Entre réforme sociale et
  professionnalisation,''~\emph{Questions de communication} 16 (2009).}
Well known journalists such as Jules Cornély, Henry Fouquier, and Jules
Claretie, as well as two historians, Alphonse Aulard and Charles
Seignobos, were among the educational advisory board.\footnote{Ferenczi,
  254.} A journalistic article about Dick May and the new journalism
school from December 1, 1899 even mentions Gabriel Tarde, who wrote
about newspapers and public opinion,\footnote{See Elihu Katz,
  ``Influence et réception chez Gabriel Tarde: Un paradigme pour la
  recherche sur l'opinion et les communications,'' in \emph{La
  réception}, ed. Cécile Méadel (Paris: CNRS Éditions, 2009).} as one of
the teachers.\footnote{\emph{``L'école du journalisme,'' La Femme:
  journal bi-mensuel}, December 1, 1899, 182.} Contemporary journalists
criticized the new journalism program as the profession was still viewed
as talent based.\footnote{Chupin, 54.} Before the rise of formal
journalism education, journalistic writing was not yet perceived as a
technique but rather as a writing style that could be learned from older
journalists.\footnote{Ruellan, para. 4.24.} Hence, Robert de Jouvenel, a
member of the labor union of French journalists (Syndicat national des
journalistes, SNJ) writing in 1920, criticized the idea of journalism
schools that would undermine the secrets of newspaper
production.\footnote{Ruellan, para. 4.25.} When in 1929 the labor union
was asked to help develop the curriculum of the Parisian journalism
school, the labor union, in return, demanded that a greater emphasis be
put on practical teaching as the conferences did not seem to prepare the
students for later professional life as journalists.

At the same time, the labor union was also interested in a theoretical
discussion about journalism, especially Georges Bourdon, who had
returned from Germany, where he had heard about the newly founded Berlin
institute for press research, the Deutsches Institut für Zeitungskunde
(1927)\emph{.} In 1929, he founded the center for journalism studies
(Centre d'études journalistiques)---which would later inspire the
creation of the Institut de Science de la Presse, ISP---and envisioned a
press science.\footnote{Ruellan, para. 4.27.} Denis Ruellan points out
that the labor union's interventions gave it control in journalism
education as well as in scientific discourse on journalism, with still
the same goal in mind: the recognition of journalism as a collective
profession.\footnote{Ruellan, para. 4.28.} After World War II, the ISP
was transformed into the French Institution for Press (Institut Français
de Presse, IFP). Nicolas Pélissier and François Demers describe how
academic interest decreased as journalists instead turned to the newly
founded journalism school.\footnote{Pélissier and Demers, para. 15.} The
ESJ (École supérieure de journalisme) in Lille is the oldest journalism
school in France, founded in 1924. This was followed in 1945 by the
foundation of the Parisian journalism school CFJ (Centre de formation
des journalistes). Both schools gained official recognition by the
journalistic profession in 1956. Other journalism schools were founded
after World War II.\footnote{For the history of journalism schools in
  France, see Chupin, \emph{Les écoles du journalisme}.}

In France, journalism research and journalism education have strong
links, which can be explained by looking at the structure of academic
research as well as at the collaboration between researchers and
journalists. However, journalism research was for a long time not done
within journalism schools.\footnote{Groupe de Recherche sur les Enjeux
  de la Communication (GRESEC), ``La recherche sur le journalisme:
  Apports et perspectives,'' \emph{Les Enjeux de l'information et de la
  communication}, no. 1 (2005).} Today, in France, faculty members are
responsible for certain degree programs or part of them. This means that
during the recruitment process for a specific position, the candidate's
experience and research agenda have to fit with the needs of the
institution. Colleagues that are responsible for a journalism degree
program are usually researchers with a proven expertise in journalism
research that would be needed for teaching in journalism programs and
for linking journalism research and journalism education,\footnote{Jacques
  Walter et al., eds., \emph{Dynamiques des recherches en sciences de
  l'information et de la communication} (Conférence permanente des
  directeur.trices d'Unité de Recherche en Sciences de l'information et
  de la communication {[}CPDirSIC{]}, 2018).} although there are cases
where journalists are recruited in order to supervise a degree program.

Today, the fourteen main journalism schools in France are grouped as the
Conférence des écoles de journalisme (CEJ) and are all ``recognized'' by
the national committee of employers and the trade unions (Commission
paritaire nationale de l'emploi des journalistes, CPNEJ). For students,
this status is a guarantee that they will have a quality education and
an easier start in their professional career thanks to privileged
pathways into the media organizations (stipends, awards, etc.). The
collaboration between these fourteen journalism schools as members of
the CEJ enables them to have a collective voice. For example, during the
COVID-19 pandemic, their collective effort allowed journalism students
to continue their practical journalism sessions even during the strict,
nation-wide lockdown or to return to campus earlier than other students
as the technical equipment available there was indispensable for their
education. Once a year the CEJ schools take part in a national
conference on journalistic professions (Conférence nationale des métiers
du journalism, CNMJ), where researchers, journalists, and others discuss
current topics in journalism. In 2019, members of the CEJ organized the
World Journalism Education Congress, which took place in Paris. Another
annual meeting, the Assises du journalisme, independent of the CEJ,
reunites researchers and journalists to discuss a main topic that
changes each year. These different working groups show how journalism
education is linked to a greater debate concerning journalism, in which
journalism researchers are involved.

The dynamics between journalism studies and journalism education in
France and in Germany are quite different: In Germany, journalism was
the main interest that led Bücher to found the first institute to
analyze journalism in 1916. This was the beginning of German
\emph{Zeitungswissenschaft}, which later became German communication
studies. In numerous debates, conferences, and publications, German
researchers exchange views about the relationship between journalism
studies and journalism education and try to find the right balance
between practice and theory. In France, journalism education preceded
the French science of communication, and journalism research does not
exist as a division within the French communication science association
(SFSIC). The orientation of journalism schools towards practical
education has led to less research in the field of journalism studies,
which might explain why the majority of research on journalism has been
conducted outside journalism schools\footnote{Groupe de Recherche sur
  les Enjeux de la Communication (GRESEC).} and why journalism research
in France is so scattered.

\hypertarget{journalism-studies-in-france-and-germany-today}{%
\section{Journalism Studies in France and Germany
Today}\label{journalism-studies-in-france-and-germany-today}}

Journalism studies in France and Germany have certain commonalities, and
yet when looked at in detail, they are quite different. As seen above,
the thought processes underlying the early phases of communication
science in Germany (\emph{Kommunikationswissenschaft}) and
\emph{sciences de l'information et de la communication} in France were
not the same. In Germany, there was a need to better understand
journalism at the beginning of the twentieth century, whereas mass media
related questions were the starting point of a new academic discipline
in France. Several researchers underline the differences between the two
sciences of communication as they are represented in each
country:\footnote{Averbeck, ``Über die Spezifika `nationaler
  Theoriediskurse'\,''; Sarah Cordonnier and Hedwig Wagner, ``La
  discipline au prisme des activités internationales dans les
  trajectoires de chercheurs en France et en Allemagne,'' \emph{Hermès}
  (2013): 133--35; Lisa Bolz, ``Recherches sur le journalisme en France
  et en Allemagne, un dialogue impossible? Regards croisés sur des
  méthodologies et des développements divergents,'' \emph{Revue
  française des sciences de l'information et de la communication} 18
  (2019); Averbeck-Lietz et al., ``\,`Terre du milieu'?''; Hubé, ``À la
  recherche d'une universalité.''} The German
\emph{Kommunikationswissenschaft} addresses questions regarding public
and mass media communication using empirical research methods, whereas
the French SIC have a broader understanding of social communication when
analyzing the mediation of signification through communication
processes.\footnote{Averbeck, 212.}

Even though it is impossible to point out every development within
journalism studies in a given country, a look at the tendencies of
journalism research and the major readings in journalism studies helps
one to better understand the \emph{esprit} of journalism studies in
France and in Germany, especially against the background of the history
of its development in alliance with academic journalism education.
Journalism studies in France and Germany are geographically close, but
differ in many respects---with notable differences in their
methodological approaches and core texts, for instance.\footnote{Averbeck;
  Bolz; Hubé.} The object of analysis---journalism---is the same, but
the way researchers approach and analyze it differs between France and
Germany. It is quite astonishing to see that certain topics are well
studied in one country, while receiving little attention in the other.

Especially in Germany, many researchers analyzed the history of the
academic discipline, and the history and development of journalism
studies in particular, and divided the academic field into several
different approaches to journalism research. Several handbooks and
collaborative books attempt to provide an overview of the many different
methodological and theoretical approaches to journalism while at the
same time presenting a survey of the large field of journalism as an
academic discipline and offering insights into the evolution of
journalism research and journalism theory in Germany. In fact, the
profusion of handbooks and other such texts in Germany shows that
journalism studies in Germany are more structured than in
France.\footnote{Hubé, para. 7.}

There are different kinds of theories and theoretical approaches to
journalism research in Germany.\footnote{Löffelholz, 21.} While the
early days of German journalism research saw more normative
considerations and empiricism, the field from the 1960s on has primarily
adopted a sociological orientation, and today's research is mainly
theory driven with several theoretical perspectives and approaches.
Based on the overviews written by Martin Löffelholz (2004) as well as by
Martin Löffelholz and Liane Rothenberger (2016), four main perspectives
might best describe current research within today's German journalism
studies: functionalist system theories, (critical) action theory,
integrative social theories, and cultural studies.\footnote{Löffelholz,
  62; Löffelholz, ``Paradigmengeschichte der Journalismusforschung,'' in
  Martin Löffelholz and Liane Rothenberger, eds., \emph{Handbuch
  Journalismustheorien} (Wiesbaden: Springer, 2016), 54. For an English
  (but older and shorter and therefore less detailed) text, see: Thomas
  Hanitzsch, ``Journalism Research in Germany: Origins, Theoretical
  Innovations and Future Outlook,'' \emph{Brazilian Journalism Research}
  2, no. 1 (2006).} This list represents neither a chronological
development nor a hierarchy among the theories but instead represents
the coexistence of multiple theories and the ``discontinuous development
of a multi-perspective.''\footnote{Löffelholz, ``Theorien des
  Journalismus,'' 35.} Löffelholz states that journalism theory is
neither a linear process nor a process with different ``revolutionary''
phases, but rather is characterized by the existence of different
theoretical perspectives at the same time, even though the
empirical-analytical perspective has been the dominant one in Germany
since World War II.\footnote{Löffelholz, 35.}

The first perspective is mainly based on Niklas Luhmann's theory of
social systems (\emph{Systemtheorie}), which has had a major influence
on German journalism studies since the 1990s. The system approach to
journalism enabled researchers to understand and study journalism within
society and journalism as a system in and of itself,\footnote{Frank
  Marcinkowski, \emph{Publizistik als autopoietisches System: Politik
  und Massenmedien; Eine systemtheoretische Analyse} (Wiesbaden:
  Springer, 1993); Bernd Blöbaum, \emph{Journalismus als soziales
  System: Geschichte, Ausdifferenzierung und Verselbständigung}
  (Wiesbaden: Springer, 1994); Matthias Kohring, ``Journalismus als
  soziales System: Grundlagen einer systemtheoretischen} alongside other systems such as politics
or the economy. Social systems, according to Luhmann, are systems of
communication and are defined by the boundary between themselves and the
environment, between the interior and the exterior. The distinction
between systems is a distinction of meaning (\emph{Sinn}). Considering
journalism as a system, therefore, is to define boundaries that
distinguish journalism from other\marginnote{Journalismustheorie,'' in \emph{Theorien des Journalismus}, ed. Martin
  Löffelholz (Wiesbaden: Springer, 2004); Bernd Blöbaum, ``Die Struktur
  des Journalismus in systemtheoretischer Perspektive,'' in
  \emph{Theorien des Journalismus}, ed. Martin Löffelholz (Wiesbaden:
  Springer, 2004); Alexander Görke, ``Programmierung, Netzwerkbildung,
  Weltgesellschaft: Perspektiven einer systemtheoretischen
  Journalismustheorie,'' in \emph{Theorien des Journalismus}, ed. Martin
  Löffelholz (Wiesbaden: Springer, 2004); Wiebke Lossen, ``Journalismus
  als (ent-)differenziertes Problem,'' in \emph{Handbuch
  Journalismustheorien}, ed. Martin Löffelholz and Liane Rothenberger
  (Wiesbaden: Springer, 2016).} areas\footnote{Matthias Kohring,
  ``Komplexität ernst nehmen: Grundlagen systemtheoretischer
  Journalismustheorie,'' in \emph{Theorien des Journalismus}, ed. Martin
  Löffelholz (Opladen: Westdeutscher Verlag, 2000); Kohring,
  ``Autopoiesis und Autonomie des Journalismus: Zur notwendigen
  Unterscheidung von zwei Begriffen,'' \emph{Communication Socialis} 34,
  no. 1 (2001).} and to define its structure and functions. The systemic
approach underlines the stability of journalistic structures and
routines and allows us to understand why new media technologies are not
always immediately adopted as ``innovation as well as tradition enable
evolution.''\footnote{Löffelholz, 27.} Different scholars have suggested
different ways of applying the theory of social systems to
journalism.\footnote{Armin Scholl and Siegfried Weischenberg,
  \emph{Journalismus in der Gesellschaft: Theorie, Methodologie und
  Empirie} (Wiesbaden: Verlag für Sozialwissenschaften, 1998), 76.}
Siegfried Weischenberg introduced an analytical framework that enabled
empirical research on journalism according to the system-theoretical
dictum. He suggested four axes for the analysis of journalism: media
systems (societal frameworks, historical foundation, professional and
ethical standards), media institutions (economic, political,
organizational, and technical contexts), media statements (information
sources, formats, construction of reality) and media actors (demography
of journalists, political orientation, understanding of the journalistic
role, professionalization).\footnote{Siegfried Weischenberg,
  \emph{Journalistik: Theorie und Praxis aktueller Medienkommunikation;
  Band 1, Mediensysteme, Medienethik, Medieninstitutionen} (Opladen:
  Westdeutscher Verlag, 1992), 69.} This way of perceiving and studying
journalism is still a current framework among German journalism
scholars. During the 2019 conference of the German association for
communication science (DGPuK) in Münster, researchers even organized a
panel to discuss system theory and journalism (``\emph{Journalismus als
System} revisited''). In France, in contrast, this approach is not found
in journalism research. On the one hand, Luhmann's books \emph{Social
Systems} and \emph{Theory of Society} were only translated into French
in 2011 and 2021, and therefore do not (yet) play an important role in
French communication and journalism research. On the other hand, the
perspective of social systems does not coincide with the \emph{esprit}
of French journalism research, which emerged with a more materialistic
perspective on journalism and journalistic products taking into
consideration both the technical dimension of communication devices and
social interaction (``\emph{conception informationnelle}'').\footnote{Nicolas
  Pélissier, \emph{Journalisme: avis de recherches; La production
  scientifique française dans son contexte international} (Bruxelles:
  Bruylant, 2008), 4.}

Along with the functionalist system theory, action theories are another
of the major perspectives on journalism in German journalism studies.
Whereas functionalist system theory excludes the individual actor in the
theoretical framework, action theories take into consideration
individual and collective actors. Action theory focuses on the
``formalized processes which build the frame of reference for
journalistic activities and the consumption of news by the
public.''\footnote{Hanitzsch, 45.} Based on Habermas's \emph{Theory of
Communicative Action}, certain researchers explain journalistic action
as ``social action in both its everyday life and its systemic
contexts.''\footnote{Hanitzsch, 45.} Susanne Fengler, for example,
examines economic-rational motives of actors withing journalistic
contexts.\footnote{Susanne Fengler, ``Journalismus als rationales
  Handeln,'' in \emph{Handbuch Journalismustheorien}, ed. Martin
  Löffelholz and Liane Rothenberger (Wiesbaden: Springer, 2016).}
Hans-Jürgen Bucher insists that analyzing journalistic actions and
system theory can be complementary, as journalists are not only acting
on a personal or individual level but also within social structures.
Certain tendencies in journalism cannot be reduced to individual actions
and have to be considered as collective dynamics---for example,
digitalization within journalism.\footnote{Hans-Jürgen Bucher,
  ``Journalismus als kommunikatives Handeln: Grundlagen einer
  handlungstheoretischen Journalismustheorie,'' in \emph{Theorien des
  Journalismus}, ed. Martin Löffelholz (Wiesbaden: Springer, 2004).}

The huge importance of the systemic point of view within German
journalism research can eclipse the journalist from the theoretical
framework. In order to go beyond the division between individual and
system, or between individual action and structure, several German
journalism scholars introduced theoretical models that combine macro and
micro level perspectives on journalism, relying, for example, on the
actor-structure dynamics by the sociologist Uwe Schimank.\footnote{For
  more on this, see Hanitzsch.} Christoph Neuberger points out that
journalism should be analyzed on different levels that would include
other sociological concepts beyond social system theory.\footnote{Christoph
  Neuberger, ``Journalismus als systembezogene Akteurskonstellation:
  Grundlagen einer integrativen Journalismustheorie,'' in \emph{Theorien
  des Journalismus}, ed. Martin Löffelholz (Wiesbaden: Springer, 2004).}
Sabine Schäfer is another researcher who seeks to overcome the division
between actor-centered theory and system theory. She suggests that
journalism be thought of as a social field based on the theory of Pierre
Bourdieu.\footnote{Sabine Schäfer, ``Journalismus als soziales Feld: Das
  relationale Denken Pierre Bourdieus als Grundlage für eine
  Journalismustheorie,'' in \emph{Theorien des Journalismus: Ein
  diskursives Handbuch}, ed. Martin Löffelholz (Wiesbaden: Springer,
  2004).}

Some of the most recent journalism research in Germany is influenced by
cultural studies\footnote{See also Andreas Heep, \emph{Cultural Studies
  und Medienanalyse: Eine Einführung} (Wiesbaden: Springer, 2010).} and
focuses more on the journalistic product, the consumer perspective, and
journalism as cultural discourse\footnote{Rudi Renger, ``Journalismus
  als kultureller Diskurs: Grundlagen der Cultural Studies als
  Journalismustheorie,'' in \emph{Theorien des Journalismus: Ein
  diskursives Handbuch}, ed. Martin Löffelholz (Wiesbaden: Springer,
  2004); Margreth Lünenborg, ``Journalismus als kultureller Diskurs,''
  in \emph{Handbuch Journalismustheorien}, ed. Martin Löffelholz and
  Liane Rothenberger (Wiesbaden: Springer, 2016).} and cultural
practice.\footnote{Johannes Raabe, ``Journalismus als kulturelle
  Praxis,'' in \emph{Handbuch Journalismustheorien}, ed. Martin
  Löffelholz and Liane Rothenberger (Wiesbaden: Springer, 2016).}

Even though the research field is quite structured, this should not
undermine the multiplicity of points of view in German journalism
studies. Perspectives beyond the theoretical approaches discussed here
exist, of course. Among these are research on journalism
ethics,\footnote{See, for example, Barbara Thomaß, ``Ethik des
  Journalismus,'' in \emph{Handbuch Journalismustheorien}, ed. Martin
  Löffelholz and Liane Rothenberger (Wiesbaden: Springer, 2016).}
research on journalism and gender topics,\footnote{See, for example,
  Elisabeth Klaus and Margreth Lünenborg, ``Der Wandel des
  Medienangebots als Herausforderung an die Joumalismusforschung:
  Plädoyer fur eine kulturorientierte Annäherung,'' \emph{Medien \&
  Kommunikationswissenschaft} 48, no. 2 (2000); Elisabeth Klaus, ``Von
  Subjekt und System zur Kultur: Theorien zur Analyse der
  Geschlechterverhältnisse im Journalismus,'' in \emph{Theorien des
  Journalismus: Ein diskursives Handbuch}, ed. Martin Löffelholz
  (Wiesbaden: Springer, 2004).} and comparative research,\textsuperscript{98} not to
mention all the research done in other academic disciplines, such as
media studies, for example. Even though German journalism research is
mainly theory based, researchers have also been critical of this way of
doing research, with some arguing that theory building might
predetermine the research and, accordingly, calling for more theory
openness.

As the major research perspectives in German journalism studies are
based on the texts of German sociologists such as Luhmann, Habermas, and
Schimank, whose texts were not received at the same time and in the same
way in France, and as major French texts that influenced French
journalism studies are not read in Germany---with the exception of
Pierre Bourdieu's texts---it is no surprise that each of these countries
has a different approach to journalism. Nevertheless, certain points of
view and research areas seem similar, such as sociological approaches
regarding the main actor in journalism---the journalists---as an
individual or as a collective actor\textsuperscript{99} and within
a journalistic field.\textsuperscript{100} More macro perspectives\textsuperscript{101} were and are also common, as
well as culture-oriented research\textsuperscript{102} and the analysis of gender topics\textsuperscript{103} in French
journalism studies. The field of journalism research in France is not as
structured and institutionalized as in Germany,\marginnote{\textsuperscript{98} See,
  for example, Frank Esser, ``Journalismustheorie und komparative
  Forschung,'' in \emph{Handbuch Journalismustheorien}, ed. Martin
  Löffelholz and Liane Rothenberger (Wiesbaden: Springer, 2016).} and\marginnote{\textsuperscript{99} See, for example, Rémy
  Rieffel, \emph{L'élite des journalistes} (Paris: Presses
  universitaires de France, 1984); Denis Ruellan, \emph{Le
  professionnalisme du flou: Identité et savoir-faire des journalistes
  français} (Grenoble: Presses universitaires de Grenoble, 1993); Eric
  Neveu, \emph{Sociologie du journalisme} (Paris: La Découverte, 2001);
  Roselyne Ringoot and Jean-Michel Utard, \emph{Le Journalisme en
  invention: Nouvelles pratiques, nouveaux acteurs} (Rennes: Presses
  universitaires de Rennes, 2006). See also the issue of \emph{Réseaux}
  on the sociology of journalists, \emph{Réseaux} 1 (1992).} the\marginnote{\textsuperscript{100} See, for example, Rodney Benson and Erik
  Neveu, \emph{Bourdieu and the Journalistic Field} (Cambridge, MA:
  Polity Press, 2005).}\marginnote{\textsuperscript{101} See, for
  example, Michel Mathien, \emph{Les journalistes et le système
  médiatique} (Paris: Hachette, 1992).} journalism\marginnote{\textsuperscript{102} See, for example, Nicolas
  Pélissier, ``Journalisme et études culturelles: de nouveaux
  positionnements de la recherche française?'' \emph{Questions de
  communication} 1 (2010).}
research\marginnote{\textsuperscript{103}\setcounter{footnote}{103} See,
  for example, Virginie Julliard, \emph{De la presse à Internet: la
  parité en question} (Paris: Hermès-Lavoisier, 2012).} community is therefore less visible and cohesive. Whereas
German journalism studies have a theory-based approach to journalism and
media, involving rather strict research protocols (validation or not of
a hypothesis), and aim to systematize empirical research results and to
obtain a holistic perspective; French journalism research values the
heterogeneity of different research approaches.\footnote{Hubé, para.
  20--21.} There is not one common way of doing journalism research in
France or a common methodology. In fact, the openness of journalism
research with respect to other academic disciplines has been one of the
main developments in the field since 1996. The qualitative approach, as
well as the disciplinary openness (interdisciplinary and
transdisciplinary) in France, is seen as a lack of methodological rigor
by some German researchers, whereas others perceive this as a chance to
pursue other research questions.\footnote{Stefanie Averbeck-Lietz,
  Fabien Bonnet, and Jacques Bonnet, ``Le discours épistémologique des
  Sciences de l\textquotesingle information et de la communication,''
  \emph{Revue française des sciences de l'information et de la
  communication} 4 (2014).}

While discussion of theory, models, and empiricism is important in
German journalism studies and defines this field of research,\footnote{Hans
  Matthias Kepplinger, ``Problemdimensionen des Journalismus:
  Wechselwirkung von Theorie und Empirie,'' in \emph{Theorien des
  Journalismus: Ein diskursives Handbuch}, ed. Martin Löffelholz
  (Wiesbaden: Springer, 2004); Johannes Raabe, ``Theoriebildung und
  empirische Analyse: Überlegungen zu einer hinreichend theorieoffenen,
  empirischen Journalismusforschung,'' in \emph{Theorien des
  Journalismus: Ein diskursives Handbuch}, ed. Martin Löffelholz
  (Wiesbaden: Springer, 2004).} French journalism research is structured
by a large panel of topics that are the core identity of the academic
discourse as well as the volitional openness regarding disciplines and
research methods. The association of directors of the French research
units in communication science (Conférence permanente des
directeur.trices d'Unité de Recherche en Sciences de l'information et de
la communication, CPDirSIC) has provided an overview of current research
in the field of journalism. The five main orientations they have
identified are: social and economic aspects of media organizations, the
morphology and working conditions of the journalistic profession, media
coverage and media representation of events and social identities, media
as places of discussion, and the interdisciplinary dimension.\footnote{Walter
  et al., 19­--36.}

There is more and more research on journalism within French
communication research,\footnote{Christine Leteinturier, ``Journalistes
  et journalismes en France: Bibliographie analytique 1990--2012,''
  Université Panthéon-Assas, 2015.} and journalism is still an
interdisciplinary research topic in France. In particular, historians,
sociologists, and scholars of literary studies participate in the
general research into journalism which underlines the interdisciplinary
history of French journalism research. Pélissier and Demers identify
three characteristics of French journalism research: the concentration
on the activity of journalistic production, which mainly means a focus
on journalists and the media message; the collection as well as the
editing and the dissemination of journalistic news; and, finally, the
internal interactions within newsrooms and external interaction between
journalists and other actors in society.\textsuperscript{109} The focus on the writing and editing process,\textsuperscript{110} with a discourse analytical
approach\textsuperscript{111}
or a semiotic-pragmatic interpretation, as well as on concrete and daily
journalistic practices, is specific to French journalism research in
comparison to Germany: \emph{écritures} and \emph{pratiques} of
journalism are the two terms\marginnote{\textsuperscript{109} Pélissier and Demers,
  para. 5--9.} that\marginnote{\textsuperscript{110} Jean-François
  Tétu and Maurice Mouillaud, \emph{Le journal quotidien} (Lyon: Presses
  universitaires de Lyon, 1989); Marc Lits and Adeline Wrona,
  ``Permanence et renouveau des recherches sur l'écriture
  journalistique,''~\emph{Revue française des sciences de l'information
  et de la communication} 5 (2014).} characterize\marginnote{\textsuperscript{111}\setcounter{footnote}{111} Patrick Charaudeau, \emph{Le discours d'information
  médiatique: La construction du miroir social} (Paris: Nathan, 1997).} a major part of
contemporary journalism research in France, especially within the
academic discipline of the French SIC.

\hypertarget{conclusion}{%
\section{Conclusion}\label{conclusion}}

It is striking to know that two neighboring countries have such
different histories regarding the development of journalism studies and
points of view concerning journalism research. On the one hand, German
journalism studies position themselves as an empirical social science
that uses empirical research methods. Theoretical frameworks shape the
scientific discourse. This was not always the case, as the
\emph{Zeitungswissenschaft} at the beginning of the twentieth century
was more historically oriented. The empirical turning point after World
War II was not only inspired by the reception of American empirical
social research, but also a way to leave behind Nazi ideology.
Journalism and newspapers that were the first research topics in Germany
underwent the same evolution towards an empirical orientation.

In France, research on journalism mainly came after the
institutionalization of French communication science, and mass media
were the primary topic of interest in the 1960s. The history of
journalism education and the orientation towards the practical side of
the profession enables us to understand why it took so long before
journalism research was an important part of communication studies in
France and why journalism research remains scattered today.

Research practices and research habits depend on many factors, whether
it be the history and the development of the discipline as an academic
institution, certain turning points, or the influence of earlier
researchers and their opinions. The different dynamics between
journalism research and journalism education in France and Germany hark
back to such factors and can explain the gap that seems to exist between
the two counties, despite efforts on the part of some communication
science scholars actively involved in the French-German academic
discussion today to establish a ``terre de milieu'' where not only the
differences but also the complementarities would be
acknowledged.\footnote{Averbeck-Lietz et al., ``\,`Terre du milieu'?,''
  363--80.}

All in all, the different dynamics between journalism studies and
journalism education, as well as between France and Germany, oblige
researchers to recognize the impact that their own education and
integration into a particular academic discipline can have on their
ability to truly understand the habits of a different research
community.




\section{Bibliography}\label{bibliography}

\begin{hangparas}{.25in}{1} 



Altmeppen, Klaus-Dieter, and Walter Hömberg. ``Traditionelle Prämissen
und neue Ausbildungsangebote: Kontinuitäten oder Fortschritte in der
Journalistenausbildung?'' In \emph{Journalistenausbildung für eine
veränderte Medienwelt: Diagnosen, Institutionen, Projekte}, edited by
Klaus-Dieter Altmeppen and Walter Hömberg, 7--16. Wiesbaden:
Westdeutscher Verlag, 2002.

Averbeck, Stefanie. ``Über die Spezifika `nationaler Theoriediskurse':
Kommunikationswissenschaft in Frankreich.'' In \emph{Theorien der
Medien- und Kommunikationswissenschaft: Grundlegende Diskussionen,
Forschungsfelder und Theorieentwicklungen}, edited by Carsten Winter,
Andreas Hepp, and Friedrich Krotz, 211--28. Wiesbaden: Verlag für
Sozialwissenschaften, 2008.

Averbeck-Lietz, Stefanie, Fabien Bonnet, and Jacques Bonnet. ``Le
discours épistémologique des Sciences de l\textquotesingle information
et de la communication.'' \emph{Revue française des sciences de
l'information et de la communication} 4 (2014).
\url{https://doi.org/10.4000/rfsic.823}.

Averbeck-Lietz, Stefanie, Fabien Bonnet, Sarah Cordonnier, and Carsten
Wilhelm. ``Communication Studies in France: Looking for a `Terre du
milieu'?'' \emph{Publizistik} 64, (2019): 363--80.

Averbeck, Stefanie, and Arnulf Kutsch. ``Thesen zur Geschichte der
Zeitungs- und Publizistikwissenschaft 1900--1960,'' \emph{medien \&
zeit} 17, no. 2--3 (2002): 57--66.

Averbeck, Stefanie, and Arnulf Kutsch. \emph{Zeitung, Werbung,
Öffentlichkeit: Biographisch-systematische Studien zur Frühgeschichte
der Kommunikationsforschung}. Köln: Herbert von Halem Verlag, 2005.

Barthes, Roland. ``Le centre d\textquotesingle études des communications
de masse: Le C.E.C.MAS.'' In \emph{Annales: Economies, sociétés,
civilisations} 16, no. 5 (1961): 991--92.
\url{https://doi.org/10.3406/ahess.1961.420775}.

Bastin, Gilles. ``La presse au miroir du capitalisme moderne: Un projet
d\textquotesingle enquête de Max Weber sur les journaux et le
journalisme.'' \emph{Réseaux} 109, no. 5 (2001): 172--208.

Benson, Rodney, and Erik Neveu. \emph{Bourdieu and the Journalistic
Field}. Cambridge, MA: Polity Press, 2005.

Blöbaum, Bernd. \emph{Journalismus als soziales System: Geschichte,
Ausdifferenzierung und Verselbständigung.} Wiesbaden: Springer, 1994.

Blöbaum, Bernd. \emph{Zwischen Redaktion und Reflexion: Die Integration
von Theorie und Praxis in der Journalistenausbildung}. Münster: LiT,
2000.

Blöbaum, Bernd. ``Die Struktur des Journalismus in systemtheoretischer
Perspektive.'' In \emph{Theorien des Journalismus}, edited by Martin
Löffelholz, 201--15. Wiesbaden: Springer, 2004.

Bolz, Lisa. ``Recherches sur le journalisme en France et en Allemagne,
un dialogue impossible? Regards croisés sur des méthodologies et des
développements divergents.'' \emph{Revue française des sciences de
l'information et de la communication} 18 (2019).
\url{https://doi.org/10.4000/rfsic.7702}.

Bucher, Hans-Jürgen. ``Journalismus als kommunikatives Handeln:
Grundlagen einer handlungstheoretischen Journalismustheorie.'' In
\emph{Theorien des Journalismus}, edited by Martin Löffelholz, 263--85.
Wiesbaden: Springer, 2004.

Champagne, Patrick. \emph{La double dépendance: Sur le journalisme.}
Paris: Raisons d'agir, 2016.

Charaudeau, Patrick. \emph{Le discours d'information médiatique: La
construction du miroir social}. Paris: Nathan, 1997.

Charle, Christophe. \emph{Le siècle de la presse (1830--1939)}. Paris:
Seuil, 2004.

Chupin, Ivan. \emph{Les écoles du journalisme: Les enjeux de la
scolarisation d'une profession (1889--2018)}. Rennes: Presses
universitaires de Rennes, 2018.

Cordonnier, Sarah, and Hedwig Wagner. ``La discipline au prisme des
activités internationales dans les trajectoires de chercheurs en France
et en Allemagne.'' \emph{Hermès}, (2013): 133--35.

Deutscher Presserat {[}German Press Council{]}. \emph{Memorandum zur
Journalistenausbildung} {[}Memorandum on journalism training{]}.
Bonn-Bad Godesberg, 1971.

Deutscher Presserat {[}German Press Council{]}. \emph{Neues Memorandum
für einen Rahmenplan zur Journalistenausbildung} {[}New memorandum for a
journalism training framework{]}. Düsseldorf, 1973.

Deutsche Gesellschaft für Publizistik- und Kommunikationswissenschaft
(DGPuK). ``Selbstverständnis der DGPuK-Fachgruppe
Journalistik/Journalismusforschung.'' Mission statement DGPuK general
meeting of September 24, 2020.
\url{https://www.dgpuk.de/de/selbstverst\%C3\%A4ndnis.html-9}.

Donsbach, Wolfgang. ``Hausaufgaben noch immer nicht gemacht:
Versäumnisse und Konzepte der Journalismusforschung.'' In \emph{Didaktik
der Journalistik: Konzepte, Methoden und Beispiele aus der
Journalistenausbildung}, edited by Beatrice~Dernbach and Wiebke~Loosen,
31--44. Wiesbaden: Springer, 2012.

Dovifat, Emil. \emph{Zeitungswissenschaft, Band I: Allgemeine
Zeitungslehre}. Berlin: Walter de Gruyter \& Co., 1931.

Escarpit, Robert. ``Pour une nouvelle épistémologie de la
communication.'' Introductory presentation at the Premier congrès
français des sciences de l'information et de la communication {[}first
convention of the French Information and Communication Sciences
Conference{]}, Compiègne, April 21, 1978.
http://palimpsestes.fr/communication/escarpit1.htm.

Esser, Frank. ``Journalismustheorie und komparative Forschung.'' In
\emph{Handbuch Journalismustheorien}, edited by Martin Löffelholz and
Liane Rothenberger, 111--30. Wiesbaden: Springer, 2016.

Fengler, Susanne. ``Journalismus als rationales Handeln.'' In
\emph{Handbuch Journalismustheorien}, edited by Martin Löffelholz and
Liane Rothenberger, 234--48. Wiesbaden: Springer, 2016.

Ferenczi, Thomas. \emph{L'invention du journalisme en France: Naissance
de la presse moderne à la fin du XIXème siècle}. Paris: Payot, 1996.

The Franco-German University. ``Exploration transnationale des milieux
de communication franco-allemands: science, design, culture numérique,
journalisme.'' Program description. Last modified July 12, 2021.
\href{https://www.dfh-ufa.org/fr/?research=explorations-transnationale-des-milieux-de-communication-franco-allemands-science-design-culture-numerique-jounalisme\&noredirect=fr\_FR}{https://www.dfh-ufa.org/fr/?research=} \href{https://www.dfh-ufa.org/fr/?research=explorations-transnationale-des-milieux-de-communication-franco-allemands-science-design-culture-numerique-jounalisme\&noredirect=fr\_FR}{explorations-transnationale-des-milieux-de-communication-franco-allemands-science-design-culture-numerique-jounalisme} \href{https://www.dfh-ufa.org/fr/?research=explorations-transnationale-des-milieux-de-communication-franco-allemands-science-design-culture-numerique-jounalisme\&noredirect=fr\_FR}{\&noredirect=fr\_FR}.

Görke, Alexander. ``Programmierung, Netzwerkbildung, Weltgesellschaft:
Perspektiven einer systemtheoretischen Journalismustheorie.'' In
\emph{Theorien des Journalismus}, edited by Martin Löffelholz, 233--47.
Wiesbaden: Springer, 2004.

Goulet, Vincent. ``\,`Transformer la société par
l\textquotesingle enseignement social': La trajectoire de Dick May entre
littérature, sociologie et journalisme.'' \emph{Revue
d\textquotesingle Histoire des Sciences Humaines} 19 (2008): 117--42.
\url{https://doi.org/10.3917/rhsh.019.0117}.

Goulet, Vincent. ``Dick May et la première école de journalisme en
France: Entre réforme sociale et professionnalisation.''~\emph{Questions
de communication} 16 (2009): 27--44.
\url{https://doi.org/10.4000/questionsdecommunication.81}.

Groupe de Recherche sur les Enjeux de la Communication (GRESEC), ``La
recherche sur le journalisme: Apports et perspectives.'' \emph{Les
Enjeux de l\textquotesingle information et de la communication}, no. 1
(2005): 109--28. \url{https://doi.org/10.3917/enic.005.1000}.

Haller, Michael. ``Die zwei Kulturen: Joumalismustheorie und
journalistische Praxis.'' In \emph{Theorien des Journalismus: Ein
diskursives Handbuch}, edited by Martin Löffelholz, 106--29. Wiesbaden:
Springer, 2004.

Haller, Michael. ``Didaktischer Etikettenschwindel? Die
Theorie-Praxis-Verzahnung in der Journalistik.'' In \emph{Didaktik der
Journalistik: Konzepte, Methoden und Beispiele aus der
Journalistenausbildung}, edited by Beatrice Dernbach and Wiebke Loosen,
45--58. Wiesbaden: Springer, 2012.

Hallin, Daniel C., and Paolo Mancini. \emph{Comparing Media Systems:
Three Models of Media and Politics}. Cambridge: Cambridge University
Press, 2004.

Hamelink, Cees, and Kaarle Nordenstreng, ``Overview of IAMCR History:
Looking at History through the International Association for Media and
Communication Research (IAMCR).'' IAMCR (website).
\url{https://iamcr.org/node/3578\#n1}.

Hanitzsch, Thomas. ``Journalism Research in Germany: Origins,
Theoretical Innovations and Future Outlook.'' \emph{Brazilian Journalism
Research} 2, no. 1 (2006).

Hardt, Hanno. \emph{Social Theories of the Press: Constituents of
Communication Research, 1840s to 1920s}. Lanham, MD: Rowman \&
Littlefield, 2002.

Heep, Andreas. \emph{Cultural Studies und Medienanalyse: Eine
Einführung}. Wiesbaden: Springer, 2010.

Hömberg, Walter. ``Expansion und Differenzierung: Journalismus und
Journalistenausbildung in den vergangenen drei Jahrzehnten.'' In
\emph{Journalistenausbildung für eine veränderte Medienwelt: Diagnosen,
Institutionen, Projekte}, edited by Klaus-Dieter Altmeppen and Walter
Hömberg, 17--30. Wiesbaden: Westdeutscher Verlag, 2002.

Hubé, Nicolas. ``À la recherche d'une universalité du journalisme: la
\emph{Journalistik} allemande.''~\emph{Revue française des sciences de
l'information et de la communication} 19 (2020).
\url{https://doi.org/10.4000/rfsic.9269}.

Jeanneret, Yves, and Bruno Ollivier. ``Introduction: Les Sic en
perspective.'' \emph{Hermès} 38 (2004): 86--88.
\url{https://doi.org/10.4267/2042/9429}.

Julliard, Virginie. \emph{De la presse à Internet: la parité en
question}. Paris: Hermès-Lavoisier, 2012.

Katz, Elihu. ``Influence et réception chez Gabriel Tarde: Un paradigme
pour la recherche sur l'opinion et les communications.'' In \emph{La
réception}, edited by Cécile Méadel, 23--40, Paris: CNRS Éditions, 2009.

Kepplinger, Hans Matthias. ``Problemdimensionen des Journalismus:
Wechselwirkung von Theorie und Empirie.'' In \emph{Theorien des
Journalismus: Ein diskursives Handbuch}, edited by Martin Löffelholz,
87--106. Wiesbaden: Springer, 2004.

Klaus, Elisabeth, and Margreth Lünenborg. ``Der Wandel des
Medienangebots als Herausforderung an die Joumalismusforschung: Plädoyer
fur eine kulturorientierte Annäherung.'' \emph{Medien \&
Kommunikationswissenschaf} 48, no. 2 (2000): 188--211.

Klaus, Elisabeth. ``Von Subjekt und System zur Kultur: Theorien zur
Analyse der Geschlechterverhältnisse im Journalismus.'' In
\emph{Theorien des Journalismus: Ein diskursives Handbuch}, edited by
Martin Löffelholz, 377--92. Wiesbaden: Springer, 2004.

Koenen, Erik, ed. \emph{Die Entdeckung der Kommunikationswissenschaft:
100 Jahre kommunikationswissenschaftliche Fachtradition in Leipzig; Von
der Zeitungkunde zur Kommunikations- und Medienwissenschaft}. Köln:
Herbert von Halem Verlag, 2016.

Kohring, Matthias. ``Komplexität ernst nehmen: Grundlagen
systemtheoretischer Journalismustheorie.'' In \emph{Theorien des
Journalismus}, edited by Martin Löffelholz, 153--68. Opladen:
Westdeutscher Verlag, 2000.

Kohring, Matthias. ``Autopoiesis und Autonomie des Journalismus: Zur
notwendigen Unterscheidung von zwei Begriffen.'' \emph{Communication
Socialis} 34, no. 1 (2001): 77--89.

Kohring, Matthias. ``Journalismus als soziales System: Grundlagen einer
systemtheoretischen Journalismustheorie.'' In \emph{Theorien des
Journalismus}, edited by Martin Löffelholz, 185--200. Wiesbaden:
Springer, 2004.

Kurp, Matthias. ``Volontariat: Reformstau auf dem Königsweg: Diskussion
mit Michael Geffken, Annette Hillebrand, Christian Lindner, Ulrich
Pätzold, Maximiliane Rüggeberg.'' In \emph{Dokumentation:
IQ-Herbstforum; Qualität und Qualifikation: Impulse zur
Journalistenausbildung}, 24--25. Berlin: October 14, 2013.

Leteinturier, Christine. ``Journalistes et journalismes en France:
Bibliographie analytique 1990--2012.'' Université Panthéon-Assas, 2015.
\url{https://docassas.u-paris2.fr/nuxeo/site/esupversions/099a4f18-e92b-4f57-a1f0-3d544608e97f?inline}.

Lemieux, Cyril. \emph{Mauvaise presse: Une sociologie compréhensive du
travail journalistique et de ses critiques}. Paris: Édition Métailié,
2000.

Lits, Marc, and Adeline Wrona. ``Permanence et renouveau des recherches
sur l'écriture journalistique.''~\emph{Revue française des sciences de
l'information et de la communication} 5 (2014).
\url{https://doi.org/10.4267/2042/9429}.

Löblich, Maria. \emph{Die empirisch-sozialwissenschaftliche Wende in der
Publizistik- und Zeitungswissenschaft}. Köln: Herbert von Halem Verlag,
2010.

Löffelholz, Martin. ``Kommunikatorforschung: Journalistik.'' In
\emph{Öffentliche Kommunikation: Studienbücher zur Kommunikations- und
Medienwissenschaft}, edited by Günter Bentele, Hans-Bernd Brosius, and
Otfried Jarren. Wiesbaden: Verlag für Sozialwissenschaften, 2003.
\href{https://doi.org/10.4267/2042/9429}{https://doi.org/10.1007/978-3-322-80383-2\_3.}

Löffelholz, Martin. ``Theorien des Journalismus: Eine historische,
metatheoretische und synoptische Einführung.'' In \emph{Theorien des
Journalismus: Ein diskursives Handbuch}, edited by Martin Löffelholz,
17--64. Wiesbaden: Springer, 2004.

Löffelholz, Martin, and Liane Rothenberger, eds. \emph{Handbuch
Journalismustheorien}. Wiesbaden: Springer, 2016.

Loosen, Wiebke. ``Journalismus als (ent-)differenziertes Problem.'' In
\emph{Handbuch Journalismustheorien}, edited by Martin Löffelholz and
Liane Rothenberger, 177--89. Wiesbaden: Springer, 2016.

Lünenborg, Margreth. ``Journalismus in der Mediengesellschaft: Ein
Plädoyer für eine integrative Journalistik.'' In \emph{Theorien der
Medien- und Kommunikationswissenschaft: Grundlegende Diskussionen,
Forschungsfelder und Theorieentwicklungen}, edited by Carsten Winter,
Andreas Hepp, and Friedrich Krotz, 269--90. Wiesbaden: Verlag für
Sozialwissenschaften, 2008.

Lünenborg, Margreth. ``Journalismus als kultureller Diskurs.'' In
\emph{Handbuch Journalismustheorien}, edited by Martin Löffelholz and
Liane Rothenberger, 325--38. Wiesbaden: Springer, 2016.

Malik, Maja. \emph{Journalismusjournalismus: Funktion, Strukturen und
Strategien der journalistischen Selbstthematisierung}. Wiesbaden:
Springer, 2004.

Maoro, Bettina. \emph{Die Zeitungswissenschaft in Westfalen 1914--45:
Das Institut für Zeitungswissenschaften in Münster und die
Zeitungswissenschaft in Dortmund}. Munich: K.G. Saur, 1987.

Marchetti, Dominique, and Denis Ruellan. \emph{Devenir journalistes:
Sociologie de l'entrée sur le marché du travail.} Paris: La
Documentation française, 2001.

Marchetti, Dominique, and Géraud Lafarge. ``Les hiérarchies de
l'information: Les légitimités `professionnelles' des étudiants en
journalisme.'' \emph{Sociétés contemporaines} 106 (2017): 21--44.

Marcinkowski, Frank. \emph{Publizistik als autopoietisches System:
Politik und Massenmedien; Eine systemtheoretische Analyse}. Wiesbaden:
Springer, 1993.

Mathien, Michel. \emph{Les journalistes et le système médiatique}.
Paris: Hachette, 1992.

Mercier, Arnaud. ``L'institutionalisation de la profession des
journalistes.'' \emph{Hermès} 13--14 (1994): 219--35.

Meyen, Michael, and Manuel Wendelin, eds. \emph{Journalistenausbildung,
Empirie und Auftragsforschung: Neue Bausteine zu einer Geschichte des
Münchners Institution für Kommunikationswissenschaft; Mit einer
Bibliographie der Dissertation von 1925 bis 2007}. Köln: Herbert von
Halem Verlag, 2008.

Neuberger, Christoph. ``Journalismus als systembezogene
Akteurskonstellation: Grundlagen einer integrativen
Journalismustheorie.'' In \emph{Theorien des Journalismus}, edited by
Martin Löffelholz, 287--303. Wiesbaden: Springer, 2004.

Neveu, Eric. \emph{Sociologie du journalisme}. Paris: La Découverte,
2001.

Palmer, Michael. ``L'âge d'or de la presse.'' \emph{Le Temps des médias}
27, no. 2 (2016): 97--110.

Pätzold, Ulrich. ``Die Anfänge in Dortmund---eine Erfolgsgeschichte mit
viel Glück.'' In \emph{Journalismus und Öffentlichkeit}, edited by
Tobias Eberwein and Daniel Müller. Wiesbaden: Verlag für
Sozialwissenschaften, 2010.

Pélissier, Nicolas. \emph{Journalisme: avis de recherches; La production
scientifique française dans son contexte international}. Bruxelles:
Bruylant, 2008.

Pélissier, Nicolas. ``Journalisme et études culturelles: de nouveaux
positionnements de la recherche française?'' \emph{Questions de
communication} 1 (2010): 273--90. \href{https://doi.org/10.4000/questionsdecommunication.391}{https://doi.org/10.4000/questions} \href{https://doi.org/10.4000/questionsdecommunication.391}{decommunication.391}.

Pélissier, Nicolas, and François Demers. ``Recherches sur le
journalisme: Un savoir dispersé en voie de structuration.'' \emph{Revue
française des sciences de l'information et de la communication} 5
(2014). \url{https://doi.org/10.4000/rfsic.1135}.

Pereira, Fabio Henrique, Tredan Olivier, and Langonné Joël. ``Penser les
mondes du journalisme.'' \emph{Hermès} 82, no. 3 (2018): 99--106.

Pörksen, Bernhard, Wiebke Loosen, and Armin Scholl. ``Paradoxien der
Journalistik Ein Gespräch mit Siegfried Weischenberg.'' In
\emph{Paradoxien des Journalismus: Theorie--Empirie--Praxis; Festschrift
für Siegfried Weischenberg}, edited by Bernhard Pörksen, Wiebke Loosen,
and Armin Scholl, 721--43. Wiesbaden: Springer, 2008.

Pürer, Heinz. ``Zur Fachgeschichte der Kommunikationswissenschaft in
Deutschland.'' \emph{Biographisches Lexikon der
Kommunikationswissenschaft.} (October 2017).
\url{http://blexkom.halemverlag.de/kommunikationswissenschaft-in-deutschland/}.

Raabe, Johannes. ``Theoriebildung und empirische Analyse: Überlegungen
zu einer hinreichend theorieoffenen, empirischen
Journalismusforschung.'' In \emph{Theorien des Journalismus: Ein
diskursives Handbuch}, edited by Martin Löffelholz, 107--28. Wiesbaden:
Springer, 2004.

Raabe, Johannes. ``Journalismus als kulturelle Praxis.'' In
\emph{Handbuch Journalismustheorien}, edited by Martin Löffelholz and
Liane Rothenberger, 339--54. Wiesbaden: Springer, 2016.

Renger, Rudi. ``Journalismus als kultureller Diskurs: Grundlagen der
Cultural Studies als Journalismustheorie.'' In \emph{Theorien des
Journalismus: Ein diskursives Handbuch}, edited by Martin Löffelholz,
359--71. Wiesbaden: Springer, 2004.

Rieffel, Rémy. \emph{L'élite des journalistes}. Paris: Presses
universitaires de France, 1984.

Ringoot, Roselyne, and Jean-Michel Utard. \emph{Le Journalisme en
invention: Nouvelles pratiques, nouveaux acteurs}. Rennes: Presses
universitaires de Rennes, 2006.

Rühl, Manfred. \emph{Die Zeitungsredaktion als organisiertes soziales
System}. Berlin: Bertelsmann Universitätsverlag, 1969.

Rühl, Manfred. ``Des Journalismus vergangene Zukunft: Zur Emergenz der
Journalistik.'' In \emph{Theorien des Journalismus: Ein diskursives
Handbuch}, edited by Martin Löffelholz, 69--86. Wiesbaden: Springer,
2004.

Ruellan, Denis. \emph{Le professionnalisme du flou: Identité et
savoir-faire des journalistes français.} Grenoble: Presses
universitaires de Grenoble, 1993.

Ruellan, Denis. \emph{Les ``Pro'' du journalisme: De
l\textquotesingle état au statut, la construction d\textquotesingle un
espace professionnel}. Rennes: Presses universitaires de Rennes, 1997.
\url{http://books.openedition.org/pur/24592}.

Schäfer, Sabine. ``Journalismus als soziales Feld: Das relationale
Denken Pierre Bourdieus als Grundlage für eine Journalismustheorie.'' In
\emph{Theorien des Journalismus: Ein diskursives Handbuch}, edited by
Martin Löffelholz, 321--34. Wiesbaden: Springer, 2004.

Scholl, Armin, and Siegfried Weischenberg. \emph{Journalismus in der
Gesellschaft: Theorie, Methodologie und Empirie}. Wiesbaden: Verlag für
Sozialwissenschaften, 1998.

Steinbrecher, Michael. ``Alte Werte, neue Kompetenzen: was sich in der
Journalistenausbildung ändern muss.'' In \emph{Dokumentation:
IQ-Herbstforum; Qualität und Qualifikation: Impulse zur
Journalistenausbildung}, 11--23. Berlin, October 14, 2013.

Tétu, Jean-François. ``Sur les origines littéraires des sciences de
l'information et de la communication.'' In \emph{Les origines des
sciences de l'information et de la communication: regards croisés},
edited by Robert Boure, 71--93. Lille: Presses universitaires du
Septentrion, 2002.

Tétu, Jean-François, and Maurice Mouillaud. \emph{Le journal quotidien}.
Lyon: Presses universitaires de Lyon, 1989.

Thomaß, Barbara. ``Ethik des Journalismus.'' In \emph{Handbuch
Journalismustheorien}, edited by Martin Löffelholz and Liane
Rothenberger, 537--50. Wiesbaden: Springer, 2016.

von La Roche, Walther, Gabriele Hooffacker, and Klaus Meier.
``Journalistenschulen.'' In \emph{Einführung in den praktischen
Journalismus: Journalistische Praxis}, edited by Walther von La Roche,
Gabriele Hooffacker, and Klaus Meier, 255--63. Wiesbaden: Springer,
2013. \url{https://doi.org/10.1007/978-3-658-01699-9_15}.

Walter, Jacques, David Douyère, Jean-Luc Bouillon, and Caroline
Ollivier-Yaniv, eds. \emph{Dynamiques des recherches en sciences de
l'information et de la communication}. Conférence permanente des
directeur.trices d'Unité de Recherche en Sciences de l'information et de
la communication (CPDirSIC), 2018.
\href{https://hal.univ-lorraine.fr/hal-01885229/file/dynamiques-recherches-en-information-communication-190627.pdf}{https://hal.univ-lorraine.fr/hal-01885229/file/dynamiques-recherches-en-information-communi-} \href{https://hal.univ-lorraine.fr/hal-01885229/file/dynamiques-recherches-en-information-communication-190627.pdf}{cation-190627.pdf}.

Wehle, J. H. \emph{Die Zeitung: Ihre Organisation und Technik}. Vienna:
A. Hartleben's Verlag, 1883.

Weischenberg, Siegfried, Klaus-Dieter Altmeppen, and Martin Löffelholz.
\emph{Die Zukunft des Journalismus: Technologische, ökonomische und
redaktionelle Trends}. Opladen: Westdeutscher Verlag, 1994.

Weischenberg, Siegfried. \emph{Journalistik: Theorie und Praxis
aktueller Medienkommunikation; Band 1, Mediensysteme, Medienethik,
Medieninstitutionen.} Opladen: Westdeutscher Verlag, 1992.

Weischenberg, Siegfried. \emph{Journalistik: Theorie und Praxis
aktueller Medienkommunikation; Band 2: Medientechnik, Medienfunktionen,
Medienakteure}. Opladen: Westdeutscher Verlag, 1995.

Weischenberg, Siegfried, Martin Löffelholz, and Armin Scholl. ``Merkmale
und Einstellungen von Journalisten.`` \emph{Media Perspektiven} 4
(1994): 154--67.

Weischenberg, Siegfried, Maja Malik, and Armin Scholl. \emph{Die
Souffleure der Mediengesellschaft, Report über die Journalisten in
Deutschland}. Konstanz: UVK Verlagsgesellschaft, 2006.

Weischenberg, Siegfried. \emph{Max Weber und die Entzauberung der
Medienwelt: Theorien und Querelen---eine andere Fachgeschichte}.
Wiesbaden: Springer, 2012.

Wilke, Jürgen. ``Von der Zeitungskunde zur Integrationswissenschaft:
Wurzeln und Dimensionen im Rückblick auf hundert Jahre Fachgeschichte
der Publizistik-, Medien- und Kommunikationswissenschaft in
Deutschland.'' \emph{Medien \& Kommunikationswissenschaft} 64, no. 1
(2016): 74--92.



\end{hangparas}


\end{document}