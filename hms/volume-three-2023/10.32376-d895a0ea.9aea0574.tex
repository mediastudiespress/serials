% see the original template for more detail about bibliography, tables, etc: https://www.overleaf.com/latex/templates/handout-design-inspired-by-edward-tufte/dtsbhhkvghzz

\documentclass{tufte-handout}

%\geometry{showframe}% for debugging purposes -- displays the margins

\usepackage{amsmath}

\usepackage{hyperref}

\usepackage{fancyhdr}

\usepackage{hanging}

\hypersetup{colorlinks=true,allcolors=[RGB]{97,15,11}}

\fancyfoot[L]{\emph{History of Media Studies}, vol. 3, 2023}


% Set up the images/graphics package
\usepackage{graphicx}
\setkeys{Gin}{width=\linewidth,totalheight=\textheight,keepaspectratio}
\graphicspath{{graphics/}}

\title[The (Still) Neglected German-French Milieu]{Understanding and Stimulating the (Still) Neglected German-French Milieu in Communication and Media Studies} % longtitle shouldn't be necessary

% The following package makes prettier tables.  We're all about the bling!
\usepackage{booktabs}

% The units package provides nice, non-stacked fractions and better spacing
% for units.
\usepackage{units}

% The fancyvrb package lets us customize the formatting of verbatim
% environments.  We use a slightly smaller font.
\usepackage{fancyvrb}
\fvset{fontsize=\normalsize}

% Small sections of multiple columns
\usepackage{multicol}

% Provides paragraphs of dummy text
\usepackage{lipsum}

% These commands are used to pretty-print LaTeX commands
\newcommand{\doccmd}[1]{\texttt{\textbackslash#1}}% command name -- adds backslash automatically
\newcommand{\docopt}[1]{\ensuremath{\langle}\textrm{\textit{#1}}\ensuremath{\rangle}}% optional command argument
\newcommand{\docarg}[1]{\textrm{\textit{#1}}}% (required) command argument
\newenvironment{docspec}{\begin{quote}\noindent}{\end{quote}}% command specification environment
\newcommand{\docenv}[1]{\textsf{#1}}% environment name
\newcommand{\docpkg}[1]{\texttt{#1}}% package name
\newcommand{\doccls}[1]{\texttt{#1}}% document class name
\newcommand{\docclsopt}[1]{\texttt{#1}}% document class option name


\begin{document}

\begin{titlepage}

\begin{fullwidth}
\noindent\LARGE\emph{French-German Communication Research
} \hspace{27mm}\includegraphics[height=1cm]{logo3.png}\\
\noindent\hrulefill\\
\vspace*{1em}
\noindent{\Huge{Understanding and Stimulating the (Still)\\\noindent Neglected German-French Milieu in\\\noindent Communication and Media Studies\par}}

\vspace*{1.5em}

\noindent\LARGE{Stefanie Averbeck-Lietz}\par\marginnote{\emph{Stefanie Averbeck-Lietz, Fabien Bonnet, Sarah Cordonnier, and Carsten Wilhelm, ``Understanding and Stimulating the (Still) Neglected German-French Milieu in Communication and Media Studies,'' \emph{History of Media Studies} 3 (2023), \href{https://doi.org/10.32376/d895a0ea.9aea0574}{https://doi.org/ 10.32376/d895a0ea.9aea0574}.} \vspace*{0.75em}}
\vspace*{0.5em}
\noindent{{\large\emph{Universität Greifswald}, \href{mailto:stefanie.averbeck-lietz@uni-greifswald.de}{stefanie.averbeck-lietz@uni-greifswald.de}\par}} \marginnote{\href{https://creativecommons.org/licenses/by-nc/4.0/}{\includegraphics[height=0.5cm]{by-nc.png}}}

\vspace*{0.75em} 

\noindent{\LARGE{Fabien Bonnet}\par}
\vspace*{0.5em}
\noindent{{\large\emph{Université de Bourgogne}, \href{mailto:Fabien.bonnet@u-bourgogne.fr }{Fabien.bonnet@u-bourgogne.fr}\par}}

\vspace*{0.75em} 

\noindent{\LARGE{Sarah Cordonnier} \href{https://www.orcid.org/0000-0003-2442-5356}{\includegraphics[height=0.5cm]{orcid.png}}\par}
\vspace*{0.5em}
\noindent{{\large\emph{Université Lyon 2}, \href{mailto:Sarah.Cordonnier@univ-lyon2.fr}{Sarah.Cordonnier@univ-lyon2.fr}\par}}

\vspace*{0.75em} 

\noindent{\LARGE{Carsten Wilhelm} \href{https://www.orcid.org/0000-0002-5037-4946}{\includegraphics[height=0.5cm]{orcid.png}}\par}
\vspace*{0.5em}
\noindent{{\large\emph{Université de Haute-Alsace}, \href{mailto:carsten.wilhelm@uha.fr}{carsten.wilhelm@uha.fr}\par}}


\end{fullwidth}

\vspace*{1em}


\newthought{The roots of} this special section lie in three joint workshops under the
umbrella of the Franco-German University in 2021.\footnote{Exploration
  transnationale des milieux de communication franco-allemands: science,
  design, culture numérique, journalisme / Deutsch-Französische
  Kommunikationsmilieus? Wissenschaft, Digitale Kultur und Journalismus:
  Transnationale Perspektiven {[}Transnational exploration of
  Franco-German communication environments: Science, design, digital
  culture, journalism{]}, project funded by the German-French
  University, January 7, 2023,
  \url{https://fonderie-infocom.net/research/mcfa/}.} Organized at the
University of Lyon by Sarah Cordonnier, at the University of Mulhouse by
Fabien Bonnet and Carsten Wilhelm, and at the University of Bremen by
Stefanie Averbeck-Lietz, the workshops took place online due to the
constraints of the pandemic. The topics were (1) German-French research
in communication studies in the context of the history of communication
and media studies (Lyon), 2) German-French design and creation analysis
(Mulhouse), and 3) populism and journalism in both countries (Bremen).

Interestingly enough, with these workshops we reached a milieu
\emph{beyond the milieu}---colleagues, like Benjamin Krämer, whom we did
not initially have in mind when enlisting people to work in German-French environments. Through a ``snowball'' system, we asked col-


\enlargethispage{2\baselineskip}

\vspace*{2em}

\noindent{\emph{History of Media Studies}, vol. 3, 2023}


 \end{titlepage}

% \vspace*{2em} | to use if abstract spills over


\noindent leagues
 to identify who might be interested in our workshops and able
to contribute to the topics \emph{and} the languages---French and German
exclusively. This was a demand by the Franco-German University, but had
the advantage of bringing communication and media researchers together
who were able to take on the challenge of reading the other language,
whether French or German. The talks were held in both languages and were
not translated. For an international, English-speaking public, this is
relevant to know: The literature in communication and media studies in
France is rarely translated to English, and scholars of both
communities, French and German, are largely unfamiliar with the writings
on the other side of the Rhine.\footnote{For details, see Stefanie
  Averbeck-Lietz and Sarah Cordonnier, ``French and German Theories of
  Communication: Comparative Perspectives with Regard to the Social and
  the Epistemological Body of Science,'' in \emph{The Handbook of Global
  Interventions in Communication Theory,} ed. Yoshitaka Miike and Jing
  Yin (New York: Routledge, 2022).}


We reached out to and eventually connected with young and also more
senior researchers, whose work on French-German topics we had not been
aware of before. This in itself was very stimulating and generated some
optimism about our aims for a sustainable French-German future, based on
the traditions of communication and media studies in both countries, in
the face of language barriers and differences in their scientific
cultures. Those cultural differences are rooted in distinct historical
developments in German and French media and communication studies. We
will come back to this point.

We first have to thank our authors, Lisa Bolz (Paris)\emph{,} Nicolas
Hubé (Metz)\emph{,} Benjamin Krämer (Munich), and Hedwig Wagner
(Flensburg), who have contributed to the workshops and to this special
section with new, original work. The four contributions complement each
other on a variety of argumentative levels and help us understand the
German, the French, \emph{and} the (still underdeveloped)
\emph{Franco-German} research field as well. Given that this special
section is published in English for an international public, we highly
appreciate the commitment to the emerging Franco-German field made by
the four articles.

In our view it is no coincidence that all of the special section authors
had worked, still work, and/or have studied during their professional
life in France \emph{and} Germany. The same is true for the invited peer
reviewers---Sabine Bosler, Peter Maurer, Irene Preisinger, and Thomas
Weber---who shared their thoughts with the authors in their open
reviews.

We view the individual work of each author and reviewer, when assembled
together, as greater than the sum of its parts. Read together, the
papers make a significant contribution to understanding the history(ies)
of communication and media studies and their characteristics. The
special section represents Franco-German academic research embedded in
the history of the field \emph{in practice.} All of the authors point
out and fill in research gaps in the Franco-German field.

\newpage Nevertheless, a crucial epistemological question remains: How can we see
what is invisible or what is a ``gap,'' and at the same time identify
this gap by naming it? Could it be that the ``gaps'' themselves are
socially and culturally bound? This is not only an epistemological
question but also a resource in establishing and pursuing sustainable
transcultural scientific debate about it. One step in this direction was
represented in our three workshops of 2021. We are grateful to the
Franco-German University for financing the English language editing to
reach a wider academic public.

\hypertarget{diagnosis-and-challenge-the-neglected-long-term-french-german-milieu-in-communication-and-media-studies}{%
\section{Diagnosis and Challenge: The Neglected Long-Term
French-\\\noindent German Milieu in Communication and Media
Studies}\label{diagnosis-and-challenge-the-neglected-long-term-french-german-milieu-in-communication-and-media-studies}}

Of course, trans-European research projects have been funded for
years.\footnote{Sophia Volk, \emph{Comparative Communication Research: A
  Study of the Conceptual, Methodological and Social Challenges of
  International Collaborative Studies in Communication Science}
  (Wiesbaden: Springer VS, 2022).} But no real German-French research
has been conducted along these lines, such that the French SIC
(\emph{Sciences de l'information et de la communication}) and German
\emph{Kommunikations- und Medienwissenschaft} would have produced
long-term projects in common, with transnational scientific relations
that include epistemological, theoretical, and methodological dimensions
with outcomes to build on.

There is a strong lack\footnote{As already noted by Ursula E. Koch,
  ``Zwischen Frankreich und Deutschland vermitteln,'' in ``\emph{Ich
  habe dieses Fach erfunden'': Wie die Kommunikationswissenschaft an die
  deutschsprachigen Universitäten kam; 19 biographische Interviews}, ed.
  Michael Meyen and Maria Löblich (Köln: von Halem, 2004).} of
systematic and sustainable Franco-German research in communication
sciences. The French field is oriented towards the Francophone world,
while the post-war German field is geared strongly towards North
America.\footnote{Michael Meyen, ``The Founding Parents of
  Communication: 57 Interviews with ICA Fellows; An Introduction,''
  \emph{International Journal of Communication} 6 (2012); Stefanie
  Averbeck-Lietz, ``Sciences de l'information et de la Communication in
  Frankreich: Über eine fehlende Grenzüberschreitung zwischen zwei
  Wissenschaftskulturen in Deutschland und Frankreich,''
  \emph{Lendemains: Etudes Comparées sur la France}, no. 39 (2014).}
Within sciences dedicated to communication, German-French researchers
are thus scarce and scattered. This is why the German-French milieu is,
by now, mostly ``invisible'' in both countries, and beyond that also to
other research communities.\footnote{Stefanie Averbeck-Lietz, Fabien
  Bonnet, Sarah Cordonnier, and Carsten Wilhelm, ``Communication Studies
  in France: Looking for a `Terre du milieu'?'' \emph{Publizistik} 64
  (2019).} Averbeck-Lietz and Cordonnier\footnote{Averbeck-Lietz and
  Cordonnier, ``French and German Theories of Communication.''} had to
explain to US colleagues that ``French'' or ``German'' communication and
media research is not at all integrated in \emph{one} trans-European
perspective.

And yet, we are convinced that long-term cross-border communication
research is needed: While reinforcing mutual understanding, it would
also contribute to the intelligence of scientific, social, political,
and cultural communication in both countries. This conviction led us to
produce joint academic endeavors and, finally, to take a more active and
hopefully more visible path over the last few years. Through workshops
and regular meetings aimed at advanced students and researchers from
France and Germany, our goal has been both to understand \emph{and} to
develop a sustainable, truly interconnected transnational, transcultural
Franco-German milieu.

At any rate, several contributors to this special section are active in
institutional as well as in scientific roles for the national
associations of communication scholars in France and Germany, the SFSIC
(\emph{Société Française des Sciences de l'information et de la
communication}) and the DGPuK (\emph{Deutsche Gesellschaft für
Publizistik- und Kommunikationswissenschaft}), including international
relations of both academic societies.

Usually, work on transculturality and the international circulation of
knowledge explores its institutionalized and/or organized modalities,
often neglecting the practical foundations of a professionalized
activity.\footnote{Pierre Mœglin, ed., \emph{Industrialiser l'éducation:
  Anthologie commentée (1913--2012)} (Saint Denis: Presses
  Universitaires de Vincennes, 2016).} The emblematic word ``milieu''
suggests other approaches, with cross-border and cross-disciplinary
forms of inquiry.\footnote{Hartmut Wessler and Stefanie Averbeck-Lietz,
  ``Grenzüberschreitende Medienkommunikation: Konturen eines
  Forschungsfeldes im Prozess der Konsolidierung,'' in
  ``Grenzüberschreitende Medienkommunikation,'' special issue,
  \emph{Medien \& Kommunikationswissenschaft} 2 (2012).}

\enlargethispage{-\baselineskip}

This term ``milieu'' is ``all-purpose,'' with ``its fuzzy logic and its
character as an intermediary object, allowing for dialogue between human
sciences, serving in fact as a revelator of the said and unsaid of
disciplinary choices.''\footnote{Paul Arnould, ``Milieu,'' in
  \emph{Dictionnaire des sciences humaines}, ed. Patrick Savidan and
  Sylvie Mesure (Paris: Presses Universitaires de France, 2006), 775--76
  (our translation).} For us, Franco-German milieus are a form of
``localization,''\footnote{Arjun Appadurai, \emph{Modernity at Large:
  Cultural Dimensions of Globalization} (Minneapolis: University of
  Minnesota Press, 1996).} a social space with practices of thinking,
writing, and ``doing'' science. Thus understood, Franco-German milieus
connect bodies of ideas (methodologies, theories, and concepts on
communication, journalism, and media) and social and professional bodies
of work.\footnote{Stefanie Averbeck-Lietz and Maria Löblich,
  ``Kommunikationswissenschaft vergleichend und transnational: Eine
  Einführung,'' in \emph{Kommunikationswissenschaft im internationalen
  Vergleich: Transnationale Perspektiven,} ed. Averbeck-Lietz
  (Wiesbaden: Springer VS, 2017).} They engage us to articulate and to
contextualize our research and those of colleagues, while paying
attention to their scholarly,\footnote{Stéphane Olivesi, \emph{Sciences
  de l'information et de la communication: Objets, savoirs, discipline}
  (Grenoble: Presses Universitaires de Grenoble, 2006); Stefanie
  Averbeck-Lietz, Jacques Bonnet, and Fabien Bonnet, ``Le discours
  épistémologique des Sciences de l'information et de la
  communication,'' \emph{Revue Française des sciences de l'information
  et de la communication} 4 (2014).} academic,\footnote{Sarah Cordonnier
  and Hedwig Wagner, ``Déployer l'interculturalité: Les étudiants, un
  vecteur pour la réflexion académique sur l'interculturel: Le cas des
  sciences consacrées à la communication et aux médias en France et en
  Allemagne,'' in \emph{Interkulturelle Kompetenz in
  deutsch-französischen Studiengängen}, ed. Gundula Hiller et al.
  (Wiesbaden:} and media-based mediations.\textsuperscript{15}

In this special section, we mainly consider two related milieus: the
scientific and the journalistic ones, two milieus constantly observing
each other professionally (see the contributions by Bolz and Hubé). In
these two cases, we are faced with marginal research fields as far as a
German-French point of view is concerned. Indeed, some topics are
studied in France and Germany, but very often without any comparative or
transnational reference to the neighboring scientific
community.\textsuperscript{16} This neglect is mentioned in all of the
articles published in this special section, so it is a shared social
reality we deal with. At the same time, the articles collected here fill
research gaps in this very field of Franco-German research, while
maintaining a deep interest in the disciplinary background of
communication and media studies from a comparative angle.\textsuperscript{17}

The communication sciences themselves could be seen as a kind of
scientific meta-milieu. This perspective allows us to think about the
situation in both countries by integrating the history of these
disciplines and research fields. In our German-French workshop, we took
communication sciences in their broad, internationally common sense of
``media studies'' as a factual object and as a kind of epistemological
tool of analysis in terms of the \emph{social} and the \emph{cognitive}
body of science.\textsuperscript{18} We
aimed to discuss and consolidate the conditions for a critical and
situated apprehension of the international environments in which we are
involved. In what ways are they constraints and/or spaces\marginnote{Springer VS, 2017); Sarah Cordonnier and Hedwig Wagner,
  ``L'interculturalité académique entre cadrages et interstices: Une
  enquête franco-allemande sur les sciences consacrées à la
  communication,'' in \emph{France-Allemagne: incommunications et
  convergences}, ed. Gilles Rouet and Michael Oustinoff (Paris: CNRS
  Éditions, 2018).}\marginnote{\textsuperscript{15} Carsten Wilhelm
  and Olivier Thévenin, ``The French Context of Internet Studies:
  Sociability and Digital Practice,'' in
  \emph{Kommunikationswissenschaft im internationalen Vergleich:
  Transnationale Perspektiven}, ed. Stefanie Averbeck-Lietz (Wiesbaden:
  Springer VS, 2017).} of\marginnote{\textsuperscript{16} See also Susanne Merkle, \emph{Politischer
  Journalismus in Deutschland und Frankreich: Ein Vergleich
  systemspezifischer Einflüsse und der Debatte um TTIP in der Presse}
  (Wiesbaden: Springer, 2019); Viviane Harkort,
  ``Traduire~l'intraduisible: Les correspondants allemands face à
  l'élection présidentielle~française,'' in \emph{L'élection
  présidentielle de 2022: vers une réinvention~des processus
  démocratiques}?, ed. Philippe Mark and Nicolas Pélissier (Paris:
  L'Harmattan, forthcoming).} freedom,\marginnote{\textsuperscript{17} Concerning
  the need for comparative research, see also Carsten Wilhelm,
  ``Comparer les imaginaires sociaux du numérique en SIC: vers une
  théorie critique située des rationalisations numériques,''
  \emph{Approches Théoriques en Information-Communication} 3, no. 2
  (2021).}
and\marginnote{\textsuperscript{18}\setcounter{footnote}{18} See also Averbeck-Lietz and Löblich,
  ``Kommunikationswissenschaft vergleichend und transnational.''} how can they be recognized and developed?\footnote{In this sense,
  see also ``Exclusions in the History of Media Studies/Exclusiones en
  la historia de los estudios de medios,'' special section\emph{,
  History of Media Studies} 2 (2022).}

The reflection on international scientific environments confronts us
with the complex relationship between what belongs to our common
frameworks (academic institutions, themselves increasingly Europeanized
and internationalized), and what belongs to individual interactions or
affinities---whether intellectual, professional, thematic, or
personal.\footnote{Yoshitaka Miike and Jin Ying, eds., \emph{The
  Handbook of Global Interventions in Communication Theory} (New York:
  Routledge, 2022).}

We paid attention to academic contexts, to the history of communication
sciences in France and Germany, to their disciplinary characteristics,
to their convergences and divergences, and to their challenges today
(digital methodologies, etc.)---to the practices and ways of doing
science in Franco-German environments, with a particular attention to
the experience and subjects of young researchers.

We took special care to invite doctoral students to our workshops, in
order to share their German-French research projects. Furthermore, all
guest editors of this volume had been involved in the
Franco-German-Swiss ``Doctorales,'' organized in 2019 in Mulhouse by the
SFSIC, in partnership with the German DGPuK and the Swiss Communication
Research Association (SGKM). This long-term dimension of supporting
young researchers is crucial to our work. For example, one of the
authors of this special issue, Lisa Bolz, obtained her PhD through a
German-French ``cotutelle de thèse''; she holds a joint degree PhD from
the Universities of Paris and Münster---a practice still rare in
German-French scientific education, at least in the field of
communication and media studies.

\hypertarget{focal-points-in-this-special-section-media-studies-journalism-studies-and-the-fields-of-political-communication-and-populism}{%
\section{Focal Points in this Special Section: Media Studies,\\\noindent
Journalism Studies, and the Fields of Political\\\noindent Communication and
Populism}\label{focal-points-in-this-special-section-media-studies-journalism-studies-and-the-fields-of-political-communication-and-populism}}

As it gathers contributions resulting from our workshops and
discussions, the present \emph{History of Media Studies} special section
is one outcome of our overall project. At the same time, it marks a new
step in our collective enterprise. Indeed, its production involved many
participants from our milieu: first, the \emph{authors} of course, but
also the \emph{reviewers}; all of them hail, as pointed out, in
different ways from German-French environments, where they act as
researchers and as teachers with comparative and/or
transnational/transcultural topics. And, hopefully soon, some
\emph{readers} will be able to identify our group and, if they so wish,
nurture our ``milieu'' by joining it: The process of producing knowledge
is inseparable from the process of producing the ``milieus'' where this
knowledge is relevant and useful. Or: to be more aware that the lingua
franca English is fine, but that translation alone is not abolishing all
barriers stemming from distinctive, long-term histories in the field.

The individual value of each of the four contributions in this volume
comes from the important expertise of the authors, who have deep
insights in comparative research (and teaching and learning) in
French-German contexts. Nevertheless, the present contributions do not
cover or represent the whole German-French communication milieu, even if
this latter domain is small.\footnote{For media systems and media
  content research, see Merkle, ``Politischer Journalismus in
  Deutschland und Frankreich''; for media usage research, see Wilhelm
  and Thévenin, ``The French Context of Internet Studies''; for media
  pedagogics, see Sabine Bosler, ``Politiques publiques et légitimité
  des savoirs en éducation aux médias: une approche comparative
  franco-allemande,'' \emph{Revue Française des Sciences de
  l\textquotesingle Information et de la Communication} 22 (2022).} But
together, they provide an understanding of the ways and conditions that
allow milieus to be both a resource and a complex topic for research:
They are an enactment of research about cross-border milieus, while
being situated within an (albeit thin) German-French milieu.

It is, then, a unique mixture of texts that would, at first glance, seem
to be heterogeneous---but which, in their combined force, gives a sense
of this ``always-under-construction,'' ever-evolving German-French
milieu. The texts overlap, not only by the shared German-French milieus
of their authors, but also by topics and by at least some perspectives.
The emphasis on a Bourdieusian perspective in this special section, for
example, had not been anticipated by the guest editors.

As Benjamin Krämer puts it, at the University of Munich there is a
milieu focusing on Bourdieu. But this is not the case in German
communication studies as a whole. Krämer shows that Bourdieu is still
underestimated and also partly unknown in Germany with regard to
political and populist communication. When it comes to journalism
research and its history, the Bourdieu School is still a leading one in
France but not in Germany, as highlighted by Lisa Bolz and Nicolas Hubé.

We have to take into account that Bolz, Hubé, and Krämer focus more on
communication studies. Hubé is situated at the crossroads with political
science, while Hedwig Wagner invites the reader to learn more about
the---also neglected---French-German milieu in media studies.
Nevertheless, Wagner highlights some French-German research programs, as
Bolz does for journalism research---both integrated not least in broader
aims to strengthen European outlooks and affiliations. The Franco-German
University is involved in some of these programs, which include a
broader normative viewpoint of German-French reconciliation after World
War II.

\enlargethispage{-\baselineskip}

From a more theoretical and methodological viewpoint, it is interesting
to see that not only Pierre Bourdieu but also Norbert Elias is mentioned
as a key to historically sensible research (see Hubé's contribution). In
our estimation, not only Bourdieu but also Elias are widely neglected in
German communication studies---for Elias, with the exception of the
figurational approach developed by Andreas Hepp, Uwe Hasebrink, and
others.\footnote{See Andreas Hepp, Andreas Breiter, and Uwe Hasebrink,
  eds., \emph{Communicative Figurations: Transforming Communications in
  Times of Deep Mediatizations} (London: Palgrave, 2017).}

Some other standard references from the German side are also mentioned
in both milieus: Max Weber and Jürgen Habermas (in the texts of Bolz and
Hubé). Both references as such would---as with Bourdieu and Elias---need
more research on their German-French interlacing, in what is a mostly
divided history of ideas, at least in communication and media research.
For the Weber adoption in France---also mentioned by Hubé---it is
obviously relevant that Weber is coming from the theory of social action
and \emph{verstehende Soziologie,} close to some paradigms in French
sociology and SIC.

Niklas Luhmann (referenced by both Bolz and Hubé)\footnote{See also
  Wilhelm, ``Comparer les imaginaires sociaux.''} is mentioned more as
an antipode to the more process-oriented research on communication
practices and representations in the French tradition. This tradition
(see Wagner and Krämer on this point, too) is much closer to critical
schools of thinking than the German one, which, after the Nazi era,
aimed for a ``value free,'' non-normative approach to social research,
isolating itself not least from the Frankfurt School
tradition,\footnote{Hanno Hardt, ``Am Vergessen scheitern: Essay zur
  historischen Identität der Publizistikwissenschaft,'' in \emph{Die
  Spirale des Schweigens: Zum Umgang mit der nationalsozialistischen
  Zeitungswissenschaft,} ed. Wolfgang Duchkowitsch, Fritz Hausjell, and
  Bernd Semrad (Münster: LIT, 2004); Andreas Scheu, \emph{Adornos Erben
  in der Kommunikationswissenschaft: Eine Verdrängungsgeschichte?}
  (Köln: von Halem, 2012).} but also from British Cultural Studies
(mentioned by Wagner as an influence on German media studies). British
cultural studies came to German communication research relatively late,
around the year 2000.\footnote{Katja Schwer, ``\,`Typisch deutsch'? Zur
  zögerlichen Rezeption der Cultural Studies in der deutschen
  Kommunikationswissenschaft,'' \emph{Münchner Beiträge zur
  Kommunikationswissenschaft} 2
  (2005)\href{https://epub.ub.uni-muenchen.de/521/(29.12.2022);}{;}
  Andreas Hepp, Friedrich Krotz, and Tanja Thomas, eds.,
  \emph{Schlüsselwerke der Cultural Studies} (Wiesbaden: Springer,
  2009).} A late adoption of Cultural Studies is also true for the
French tradition, but for other reasons: The French SIC's traditional
socio-semiotic approach included cultural views---and helped inspired
the British scholars.\footnote{See Olivesi, ``Sciences de l'information
  et da la communication''; Averbeck-Lietz, ``Sciences de l'information
  et de la Communication.''}

We can observe some main common traits among the articles published in
this special section:

\emph{First}, in their own way, the four approaches involve
epistemological, methodological, and historical vigilance, as well as a
clear knowledge of the ``risks'' (Wagner) taken when diving into
intersecting histories.

\emph{Second,} they show how such inquiries require a careful selection,
and then combination, of various scales, entries, and parameters, such
as the role of scientific disciplines, influence of places (be they
region, country, city, etc.), practices and dynamics of production,
circulation and/or reception, etc. In their own ways, all four
contributions intermingle the observation of research, teaching, and
professional fields, seized through the observation of national
disciplines and their international influences, and/or through various
travels of theories, authors, concepts, or topics. These processes
require a subtle and refined methodological approach. As noted for
instance by Hubé, who writes:

\newpage\begin{quote}
Far from confronting French and German approaches, this investigation
was only made possible by taking advantage of each national one.
\end{quote}

\emph{Third,} the contributions of this volume could not exist outside
the assumption that all knowledge is situated, but in ways that are
particularly difficult to untangle in such contexts. This implies that
the articles provide access to their relations with a given research,
but also with the researcher dealing with them (and the thickness of
their inquiries), and, finally, with the social role of science in
different (national) societies and historical periods.

In this sense, the four contributions (and the ones to come afterwards!)
furnish a better understanding of various (inter or trans-)\\\noindent national
contexts and traditions: theoretical, methodological, academic,
(inter)disciplinary, etc.

\hypertarget{outlook-why-and-how-to-strengthen-the-german-french-communication-milieu}{%
\subsection{Outlook: Why and How to Strengthen the German-French\\\noindent
Communication
Milieu?}\label{outlook-why-and-how-to-strengthen-the-german-french-communication-milieu}}

How can we develop theoretical and methodological foundations in order
to ``densify'' the concept of ``milieu'' and expand it beyond the
thematic fields? And why should we pursue our efforts towards a
German-French milieu (rather than any other random combination of
countries)?

More than a decade ago, and after a comparative analysis of the state of
communication studies in various countries, two researchers pointed out
a crucial problem about the ``internationalization'' of sciences:

\begin{quote}
While we will note some significant exceptions to this rule (e.g., to a
certain extent, \emph{Germany and France, insulated by stronger national
traditions}), the general tendency, particularly for smaller countries,
has not been towards internationalisation in a genuine sense. On the
contrary, it has been towards \emph{an increasing ``provincialisation,''
as a hegemonic centre progressively transforms and reshapes its
peripheries in its own image}. Given the reliance of communication and
media studies upon national traditions in other disciplines, fields and
areas, this development \emph{raises troubling questions about the
capacity of contemporary research projects to play an active role in
their contemporary societies, above and beyond standards imposed by an
artificial ``international'' benchmark.}\footnote{Juha Koivisto and
  Peter D. Thomas, \emph{Mapping Communication and Media Research:
  Conjunctures, Institutions, Challenges} (Tampere: Tampere University
  Press, 2010), 11 (our italics).}
\end{quote}

Our concerns echo this statement quite directly. For us, it is crucial
to maintain epistemic diversity, as well as an anchorage of academic
practices within their social and political contexts. This implies a
resistance to a dominant Anglo-American perspective, which imposes
itself in a standardized and non-satisfactory way (even for many
researchers working in the countries in question)---not necessarily
through a direct opposition, but mainly by putting our focus towards
what is of importance in (trans-)regional contexts.

France and Germany are, indeed, good places to resolutely sustain the
construction of a ``milieu''---a (trans-)regional cooperation between
countries that are geographically in contact but do not share the same
language. (In this way, what is at stake here differs from, for
instance, the case of Latin America as a world-regional field of
communication and media studies.) Both countries host a ``communication
science'' and/or ``media science'' discipline, developed in a very
original and situated way. Far from being peripheral in an economic or
political sense, these European countries have nonetheless lost their
centrality over the past decades, due to transversal movements that also
affect them internally, and which reflect within our disciplines as
partly also shown with this special section: the rise of populism, the
European unification process, the digital transformation, and an
increasing visibility of de-colonial and post-colonial voices, to name
just a few. The strength of a solid and reputable academic system,
combined with the relative weakness of our disciplines in the
``conversation of disciplines''\footnote{Robert T. Craig,
  ``Communication in the Conversation of Disciplines,'' \emph{Russian
  Journal of Communication} 1, no. 1 (2008).} as well as in the
international communication study field, could be turned into an
advantage as we build new paths towards transnational and transregional
research.

\section{Bibliography}\label{bibliography}

\begin{hangparas}{.25in}{1} 

\enlargethispage{\baselineskip}

Appadurai, Arjun. \emph{Modernity at Large: Cultural Dimensions of
Globalization}. Minneapolis: University of Minnesota Press, 1996.

Arnould, Paul. ``Milieu.'' In \emph{Dictionnaire des sciences humaines},
edited by Patrick Savidan and Sylvie Mesure, 763--65. Paris: Presses
Universitaires de France, 2006.

Averbeck-Lietz, Stefanie. ``Sciences de l'information et de la
Communication in Frankreich: Über eine fehlende Grenzüberschreitung
zwischen zwei Wissenschaftskulturen in Deutschland und Frankreich.''
\emph{Lendemains: Etudes Comparées sur la France}, no. 39 (2014):
12--39.

Averbeck-Lietz, Stefanie, and Maria Löblich.
``Kommunikationswissenschaft vergleichend und transnational: Eine
Einführung.'' In \emph{Kommunikationswissenschaft im internationalen
Vergleich: Transnationale Perspektiven,} edited by Stefanie
Averbeck-Lietz, 1--29. Wiesbaden: Springer VS, 2017.

Averbeck-Lietz, Stefanie, Jacques Bonnet, and Fabien Bonnet. ``Le
discours épistémologique des Sciences de l\textquotesingle information
et de la communication.'' \emph{Revue Française des sciences de
l\textquotesingle information et de la communication} 4 (2014).

Averbeck-Lietz, Stefanie, Fabien Bonnet, Sarah Cordonnier, and Carsten
Wilhelm. ``Communication Studies in France: Looking for a `Terre du
milieu'?'' \emph{Publizistik} 64 (2019): 1--18.
\url{https://doi.org/10.1007/s11616-019-00504-3}.

Averbeck-Lietz, Stefanie, and Sarah Cordonnier. ``French and German
Theories of Communication: Comparative Perspectives with Regard to the
Social and the Epistemological Body of Science.'' In \emph{The Handbook
of Global Interventions in Communication Theory,} edited by Yoshitaka
Miike and Jing Yin, 373--92. New York: Routledge, 2022.

Bosler, Sabine. ``Politiques publiques et légitimité des savoirs en
éducation aux médias: une approche comparative franco-allemande.''
\emph{Revue Française des Sciences de l\textquotesingle Information et
de la Communication} 22 (2022).
\url{http://dx.doi.org/10.4000/rfsic.11108}.

Cordonnier, Sarah, and Hedwig Wagner. ``Déployer l'interculturalité: Les
étudiants, un vecteur pour la réflexion académique sur l'interculturel:
Le cas des sciences consacrées à la communication et aux médias en
France et en Allemagne.'' In \emph{Interkulturelle Kompetenz in
deutsch-französischen Studiengängen}, edited by Gundula Hiller et al.,
221--34. Wiesbaden: Springer VS, 2017.
\url{https://doi.org/10.1007/978-3-658-14480-7_12}.

Cordonnier, Sarah, and Hedwig Wagner. ``L'interculturalité académique
entre cadrages et interstices: Une enquête franco-allemande sur les
sciences consacrées à la communication.'' In \emph{France-Allemagne:
incommunications et convergences}, edited by Gilles Rouet and Michael
Oustinoff, 169--82. Paris: CNRS Éditions, 2018.

Craig, Robert T. ``Communication in the Conversation of Disciplines.''
\emph{Russian Journal of Communication} 1, no. 1 (2008): 7--23.
\url{https://doi.org/10.1080/19409419.2008.10756694}.

Hardt, Hanno. ``Am Vergessen scheitern: Essay zur historischen Identität
der Publizistikwissenschaft.'' In \emph{Die Spirale des Schweigens: Zum
Umgang mit der nationalsozialistischen Zeitungswissenschaft,} edited by
Wolfgang Duchkowitsch, Fritz Hausjell, and Bernd Semrad, 153--61.
Münster: LIT, 2004.

Harkort, Viviane. ``Traduire~l'intraduisible: Les correspondants
allemands face à l'élection présidentielle~française.'' In
\emph{L'élection présidentielle de 2022: vers une réinvention~des
processus démocratiques?} {[}working title{]}, edited by Philippe Mark
and Nicolas Pélissier. Paris: L'Harmattan, forthcoming.

Hepp, Andreas, Andreas Breiter, and Uwe Hasebrink, eds.
\emph{Communicative Figurations: Transforming Communications in Times of
Deep Mediatizations}. London: Palgrave, 2017.

Hepp, Andreas, Friedrich Krotz, and Tanja Thomas, eds.
\emph{Schlüsselwerke der Cultural Studies}. Wiesbaden: Springer, 2009.

Koch, Ursula E. ``Zwischen Frankreich und Deutschland vermitteln.'' In
``\emph{Ich habe dieses Fach erfunden'': Wie die
Kommunikationswissenschaft an die deutschsprachigen Universitäten kam;
19 biographische Interviews}, edited by Michael Meyen and Maria Löblich,
136--50. Köln: von Halem, 2004.

Koivisto, Juha, and Peter D. Thomas. \emph{Mapping Communication and
Media Research: Conjunctures, Institutions, Challenges}. Tampere:
Tampere University Press, 2010.

Merkle, Susanne. \emph{Politischer Journalismus in Deutschland und
Frankreich: Ein Vergleich systemspezifischer Einflüsse und der Debatte
um TTIP in der Presse.} Wiesbaden: Springer, 2019.

Meyen, Michael. ``The Founding Parents of Communication: 57 Interviews
with ICA Fellows; An Introduction.'' \emph{International Journal of
Communication} 6 (2012), 1451--59.

Miike, Yoshitaka, and Jin Ying, eds. \emph{The Handbook of Global
Interventions in Communication Theory}. New York: Routledge, 2022.

Mœglin, Pierre, ed. \emph{Industrialiser l'éducation: Anthologie
commentée (1913--2012).} Saint Denis: Presses Universitaires de
Vincennes, 2016.

Olivesi, Stéphane. \emph{Sciences de l'information et de la
communication: Objets, savoirs, discipline}. Grenoble: Presses
Universitaires de Grenoble, 2006.

Scheu, Andreas. \emph{Adornos Erben in der Kommunikationswissenschaft:
Eine Verdrängungsgeschichte?} Köln: von Halem, 2012.

Schwer, Katja. ``\,`Typisch deutsch'? Zur zögerlichen Rezeption der
Cultural Studies in der deutschen Kommunikationswissenschaft.''
\emph{Münchner Beiträge zur Kommunikationswissenschaft} 2 (2005).
\url{https://epub.ub.uni-muenchen.de/521/}.

Volk, Sophia. \emph{Comparative Communication Research: A Study of the
Conceptual, Methodological and Social Challenges of International
Collaborative Studies in Communication Science}. Wiesbaden: Springer VS,
2022.

Wessler, Hartmut, and Stefanie Averbeck-Lietz. ``Grenzüberschreitende
Medienkommunikation: Konturen eines Forschungsfeldes im Prozess der
Konsolidierung.'' In ``Grenzüberschreitende Medienkommunikation.''
Special issue, \emph{Medien \& Kommunikationswissenschaft} 2 (2012):
5--18.

Wilhelm, Carsten, and Olivier Thévenin. ``The French Context of Internet
Studies: Sociability and Digital Practice.'' In
\emph{Kommunikationswissenschaft im internationalen Vergleich:
Transnationale Perspektiven}, edited by Stefanie Averbeck-Lietz,
161--83. Wiesbaden: Springer VS, 2017.

Wilhelm, Carsten. ``Comparer les imaginaires sociaux du numérique en
SIC: vers une théorie critique située des rationalisations numériques.``
\emph{Approches Théoriques en Information-Communication} 3, no. 2
(2021): 109--29.







\end{hangparas}


\end{document}