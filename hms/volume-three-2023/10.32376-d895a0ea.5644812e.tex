% see the original template for more detail about bibliography, tables, etc: https://www.overleaf.com/latex/templates/handout-design-inspired-by-edward-tufte/dtsbhhkvghzz

\documentclass{tufte-handout}

%\geometry{showframe}% for debugging purposes -- displays the margins

\usepackage{amsmath}

\usepackage{hyperref}

\usepackage{fancyhdr}

\usepackage{hanging}

\hypersetup{colorlinks=true,allcolors=[RGB]{97,15,11}}

\fancyfoot[L]{\emph{History of Media Studies}, vol. 3, 2023}


% Set up the images/graphics package
\usepackage{graphicx}
\setkeys{Gin}{width=\linewidth,totalheight=\textheight,keepaspectratio}
\graphicspath{{graphics/}}

\title[Media Studies in Germany]{Media Studies in Germany in the Context of Cultural Studies and Franco-German Cooperation} % longtitle shouldn't be necessary

% The following package makes prettier tables.  We're all about the bling!
\usepackage{booktabs}

% The units package provides nice, non-stacked fractions and better spacing
% for units.
\usepackage{units}

% The fancyvrb package lets us customize the formatting of verbatim
% environments.  We use a slightly smaller font.
\usepackage{fancyvrb}
\fvset{fontsize=\normalsize}

% Small sections of multiple columns
\usepackage{multicol}

% Provides paragraphs of dummy text
\usepackage{lipsum}

% These commands are used to pretty-print LaTeX commands
\newcommand{\doccmd}[1]{\texttt{\textbackslash#1}}% command name -- adds backslash automatically
\newcommand{\docopt}[1]{\ensuremath{\langle}\textrm{\textit{#1}}\ensuremath{\rangle}}% optional command argument
\newcommand{\docarg}[1]{\textrm{\textit{#1}}}% (required) command argument
\newenvironment{docspec}{\begin{quote}\noindent}{\end{quote}}% command specification environment
\newcommand{\docenv}[1]{\textsf{#1}}% environment name
\newcommand{\docpkg}[1]{\texttt{#1}}% package name
\newcommand{\doccls}[1]{\texttt{#1}}% document class name
\newcommand{\docclsopt}[1]{\texttt{#1}}% document class option name


\begin{document}

\begin{titlepage}

\begin{fullwidth}
\noindent\LARGE\emph{French-German Communication Research
} \hspace{25mm}\includegraphics[height=1cm]{logo3.png}\\
\noindent\hrulefill\\
\vspace*{1em}
\noindent{\Huge{Media Studies in Germany in the Context\\\noindent of Cultural Studies and Franco-German\\\noindent Cooperation\par}}

\vspace*{1.5em}

\noindent\LARGE{Hedwig Wagner}\par\marginnote{\emph{Hedwig Wagner, ``Media Studies in Germany in the Context of Cultural Studies and Franco-German Cooperation,'' \emph{History of Media Studies} 3 (2023), \href{https://doi.org/10.32376/d895a0ea.5644812e}{https://doi.org/ 10.32376/d895a0ea.5644812e}.} \vspace*{0.75em}}
\vspace*{0.5em}
\noindent{{\large\emph{Europa-Universität Flensburg, }, \href{mailto:hedwig.wagner@uni-flensburg.de}{hedwig.wagner@uni-flensburg.de}\par}} \marginnote{\href{https://creativecommons.org/licenses/by-nc/4.0/}{\includegraphics[height=0.5cm]{by-nc.png}}}

% \vspace*{0.75em} % second author

% \noindent{\LARGE{<<author 2 name>>}\par}
% \vspace*{0.5em}
% \noindent{{\large\emph{<<author 2 affiliation>>}, \href{mailto:<<author 2 email>>}{<<author 2 email>>}\par}}

% \vspace*{0.75em} % third author

% \noindent{\LARGE{<<author 3 name>>}\par}
% \vspace*{0.5em}
% \noindent{{\large\emph{<<author 3 affiliation>>}, \href{mailto:<<author 3 email>>}{<<author 3 email>>}\par}}

\end{fullwidth}

\vspace*{1em}


\hypertarget{abstract}{%
\section{\texorpdfstring{Abstract }{Abstract }}\label{abstract}}

In this article\footnote{Parts of this article will be published in
  German in Jonas Nesselhauf, ed., \emph{Handbuch
  Kulturwissenschaftliche Studien} (Berlin: de Gruyter, forthcoming).} I
describe the evolution of German media studies, and I do so with an
awareness of the selection, exclusion, and canonization at work in this
procedure. I explain the emergence of media studies in Germany from a
cultural studies research desideratum. I do this with a concern for the
establishment and institutionalization of media studies in Germany
within the mainly German-national scientific community as well as
German-language cultural studies. I focus primarily on the first two
decades of the institutionalization of media studies as an academic
discipline, with an attendant concern for the social and intellectual
forces at work during that epoch. I then link all of this to more recent
turns in media studies, with a concern for its central questions and
methodological approaches. I conclude with a note regarding the outlook
for German media studies. Though film studies has a certain place in
media studies, and many scholars see film studies as the core discipline
in media studies, I do not discuss film studies in this article. It
deserves an article on its own. Neither do I address some of the crucial
progenitors of media studies, including media theories of the 1920s or
the relevant writings since Greek antiquity that reflect on textual
media and cultural artifacts.

\enlargethispage{2\baselineskip}

\vspace*{2em}

\noindent{\emph{History of Media Studies}, vol. 3, 2023}


 \end{titlepage}

% \vspace*{2em} | to use if abstract spills over

\newthought{In this article} I outline a history of German media studies. The
structure of the encyclopedic knowledge presented here begins with a
definition of the research area of media studies, and then considers the
societal background of media studies. I survey the influence of two
prominent cultural turns and then gauge the proximity of media studies
to other academic formations. I evaluate how media studies differs from
other disciplines. I chart the institutionalization of German media
studies through attention to conferences, research centers, degree
programs, chairs, and publication of journals, handbooks, and
anthologies/readers. The second section is dedicated to examples and
exemplary questions. In the third section, I detail the embeddedness of
media studies in Germany within cultural studies research in the
German-speaking world. In the fourth section, I shed light on
Franco-German cooperation in media and communication studies. The fifth
section I devote to consideration of the potential of media studies,
criticism of media studies, and the outlook for media studies.

My own perspective here is informed by my engagement in the field of
European collaborations, with a strong focus on German-French research,
as well as by the research focus and institutional anchoring in the
field of European media studies.

\hypertarget{definition-of-the-research-field-of-media-studies}{%
\section{Definition of the Research Field of Media
Studies}\label{definition-of-the-research-field-of-media-studies}}

Media studies covers the subject area of the media (individual media and
media systems) in its own media-theoretical line of thought, which
considers the media in terms of the interdependence of media in the
sense of apparatuses and their transmitted content, technology, and
culture. The academic association \emph{Gesellschaft für
Medienwissenschaft} (GfM) formulates its scientific mission as follows:

\begin{quote}
Media studies is based on the assumption that an engagement with media,
with their technical and aesthetic, their symbolic and communicative
characteristics makes an essential contribution to the understanding of
history, culture and society. . . . Media studies aims at an analysis of
individual media and complex media systems with their respective
specific forms of representation, apparatuses, institutions; in doing
so, it employs a variety of methods that break down modes of production,
text structures, contents, cultural practices and contexts. At the same
time, it aims at a theory of media that attempts to determine the
characteristics of the media/mediality in general by drawing on research
in other disciplines.\footnote{Gesellschaft für Medienwissenschaft
  {[}German Media Studies Association{]}, ``Selbstverständnis,
  Forschungsfragen, Wissenschaftspolitik'' {[}Mission statement,
  research questions, science policy{]}, GfM (website).}
\end{quote}

The intrinsic interconnectedness of media or media systems with the
technical and the cultural forms one unbreakable triad:
technics/technologies, culture and humans (individually/collectively:
the public(s)/society), and the media placed in the center of this
threefold structure, built up by it and encompassing it. The other triad
is that of any cultural studies: history, theory, and aesthetics, which
make up the core areas.\footnote{Strategy Commission of the GfM, ``Core
  Areas of Media Studies: Resolution of the General Assembly of the
  GfM,'' Mitgliederversammlung der GfM {[}GfM General Assembly{]},
  Bochum, April 10, 2008.} According to the self-understanding of media
studies, the areas of ``society,'' ``technology,'' and ``culture'' are
subdivided into ``media theories and their methods,'' ``theories and
methods of media history,'' and, thirdly, the ``theories and methods of
media aesthetics.''\footnote{Strategy Commission of the GfM, ``Core
  Areas of Media Studies.''} In addition to hyphenated
sub-disciplines,\footnote{These include media economics, media
  psychology, media policy, and media law. Unnamed by GfM: media
  sociology and media geography.} which form intersections with a
multitude of cultural and social science disciplines, there are
transversal fields of knowledge, which are determined partly by their
thematic features, partly by their methodological
inclinations.\footnote{For example, actor network theory,
  deconstruction, discourse analysis, gender theories, communication
  theories, constructivism, critical theory, semiotics, and systems
  theory.} There are also interdisciplinary ``studies.'' The classical
subject areas of media studies, the individual media in their historical
succession, are only shown in the section dealing with media history,
due to the broad understanding of media that exists. Not only is the
subject area of media studies extremely inclusive, media studies is also
open to diverse methods and approaches. Limitations and thereby the
identity of the subject arise from thinking in terms of the media, which
provides information about ``the interactions of aesthetic, normative
and social dynamics in societies. . . . This interrelation marks the
epistemological location and the epistemological interest of general
media studies, in which the various fields of media studies are
integratively included.''\footnote{Strategy Commission of the GfM,
  ``Core Areas of Media Studies.''}

Media are technical in their basic conditions, always historically and
culturally shaped. They organize sensory perception (apperception) and
make perception possible in the first place. Media studies is based on a
broad concept of media. Alongside mass media, individual media, digital,
electronic, electric, and mechanical content-containing devices
(`media'), there are other kinds of media (concepts), such as
non-semantically decodable phenomena, e.g., the human voice
(non-technical, bodily function) or choice-making (selection processes
as cognitive operations with and without material/concrete
manifestations). The broad concept of media also includes cultural
techniques, such as the basic ones of reading, writing, calculating or,
respectively, alphabet, image, number, and cultural manifestations such
as dancing, for example. This conceptualization of media is neither
necessarily bound to targeted publics, nor to apparatus/device, nor to
content transmission. Media studies thinks of mediality fundamentally as
a characteristic and as the basic functionality of apparatuses,
artifacts and cultural techniques. Media are understood as entities
conditioning perception and thinking\footnote{Lorenz Engell and Bernhard
  Siegert, ``Medienphilosophie,'' \emph{Zeitschrift für Medien- und
  Kulturforschung} 2 (2010).} or as ``apparatuses, institutions and
functions, and the condition of processes of form
formation.''\footnote{Joachim Paech, ``Warum `Medien'?'' (lecture,
  University of Konstanz, Konstanz, Germany, 2006).}

These introductory remarks may already give the reader an idea of the
specificity of the German \emph{Medienwissenschaft} (media studies). In
the following, the split between the disciplines dedicated to media,
communication, and information will be described in detail, in order to
explain the partial non-comparability of media studies as an umbrella
term for \emph{Medienwissenschaft} and \emph{Kommunikationswissenschaft}
in an international context.\footnote{The same claim, being an umbrella
  discipline for media, communication, and information studies that does
  not exist elsewhere, is made by French information and communication
  studies (\emph{Sciences de l'information et de la communication}
  {[}SIC{]}). See Stefanie Averbeck-Lietz, Fabien Bonnet, Sarah
  Cordonnier, and Carsten Wilhelm, ``Communication Studies in France:
  Looking for a `Terre du milieu'?,'' \emph{Publizistik} 64 (2019).}

\hypertarget{differentiation-of-media-studies-from-communication-studies}{%
\section{\texorpdfstring{Differentiation of Media Studies from
Communication Studies
}{Differentiation of Media Studies from Communication Studies }}\label{differentiation-of-media-studies-from-communication-studies}}

Media studies (\emph{Medienwissenschaft}) is distinguished from social
science communication studies (\emph{Kommunikationswissenschaft})
primarily by their differing research methods, but also by
epistemological interest and thus theoretical foundations. The common
and divided history of these two strands has been of interest to
scholars of both disciplines\footnote{Gudrun Schäfer, ``\,`Sie stehen
  Rücken an Rücken und schauen in unterschiedliche Richtungen': Zum
  Verhältnis von Medienwissenschaft und Publizistik- und
  Kommunikationswissenschaft,'' in \emph{Über Bilder sprechen:
  Positionen und Perspektiven der Medienwissenschaft}, ed. Heinz-B.
  Heller (Marburg: Schüren, 2000); Werner Faulstich, ``Einführung: Zur
  Entwicklung der Medienwissenschaft,'' in \emph{Grundwissen Medien},
  ed. Werner Faulstich (München: Fink, 1994); Reinhold Viehoff, ``Von
  der Literaturwissenschaft zur Medienwissenschaft, Oder: vom Text über
  das Literatursystem zum Mediensystem,'' in \emph{Einführung in die
  Medienwissenschaft: Konzeptionen, Theorien, Methoden, Anwendungen},
  ed. Gebhard Rusch (Wiesbaden: Westdeutscher Verlag, 2002); Knut
  Hickethier, \emph{Einführung in die Medienwissenschaft} (Stuttgart:
  Metzler, 2003), 6--7.} in its particular and local history.\footnote{See
  Christian Filk, \emph{Episteme der Medienwissenschaft:
  Systemtheoretische Studien zur Wissenschaftsforschung eines
  transdisziplinären Feldes} (Bielefeld: Transcript, 2015), 23.}

Christian Filk points at the opposite evolution of international and
German-language differentiation:

\begin{quote}
While the \emph{disciplinary dissolution of boundaries}, which is
characteristic of the international development of media and
communication studies, is sedimented in distinctly
\textbf{intradisciplinary discourse formations}---here we should refer
in particular to the Anglo-American academic context---such and similar
transformations manifest themselves in other structures in this
country.\footnote{Filk, \emph{Episteme der Medienwissenschaft}, 23--24
  (emphasis in italics by Filk, my translation and emphasis in bold).}
\end{quote}

\noindent In the 1970s and 1980s, German-language scholars have seen (and actively
participated in) the erection of disciplinary boundaries between media
studies (\emph{Medienwissenschaft}) on the one hand and communication
studies (\emph{Kommunikationswissenschaft}) on the other hand, whereas
intradisciplinary discourse formations had previously been strong, at
least in media studies.

According to Knut Hickethier, media studies (\emph{Medienwissenschaft})
sees itself as a ``text and cultural science,'' whereas journalism
(\emph{Publizistik}) and communication studies
(\emph{Kommunikationswissenschaft}), often referred to in this
combination, sees itself ``primarily as a social science.'' Both were
created, not by ``splitting off a new subject from the parent science,''
but by ``internal differentiations within the disciplines,'' and in the
1980s and 1990s, ``media studies has taken on a life of its
own.''\footnote{Hickethier, \emph{Einführung in die Medienwissenschaft},
  6--7.} The

\begin{quote}
disregard for qualitative methods compared to quantitative methods {[}in
journalism/\emph{Publizistikwissenschaft}{]} favoured the
\textbf{emergence of media studies
(}\textbf{\emph{Medienwissenschaft}}\textbf{)}. . . . Media studies
(\emph{Medienwissenschaft}) is predominantly concerned with the media
film, television, radio and the internet, and here above all with the
entertaining and fictional forms. Journalism
(\emph{Publizistikwissenschaft}) particularly addresses journalistic
forms in the media press, television, radio and the internet. Media
studies (\emph{Medienwissenschaft}) works exemplarily and
analytically-interpretatively as well as theoretically and historically.
Journalism (\emph{Publizistikwissenschaft}) takes a more empirical
approach in the sense of statistical procedures.\footnote{Hickethier, 8
  (emphasis in original, my translation).}
\end{quote}

\noindent Communication studies, which focuses on the question of society in the
sense of the social reality of life and its expression through mass
media, is distinguished from media studies, which deals with the
interaction of media, technology, and culture and includes aesthetics
and history.

The wide range of media conceptualizations in \emph{Medienwissenschaft}
is a paramount difference to \emph{Kommunikationswissenschaft,} which
finds its reason in the epistemological grounding of the speculative
thinking of philosophy, which recognizes dialectics, in the inclusion of
non-evidence-based or non-sensually-perceptible (intelligible)
assumptions, in the recourse to, e.g., psychoanalysis, and in the
methods of hermeneutics and deconstruction. Further references to the
differences between media and communication studies often cite the
interpretative methods of text studies, the methods of aesthetic
analysis, whereas the quantitative and empirical---the
social-scientific---are named as the core competences of communication
studies.

The demarcation of German-language media studies from the independent
disciplines of German-language communication studies on the one hand,
and journalism or \emph{Publizistik} on the other, formed a scientific
configuration that is not found in other European/Western countries or
languages.\footnote{On the question of European comparability, see
  Hedwig Wagner, ed., \emph{Europäische Medienwissenschaft: Zur
  Programmatik eines Fachs} (Bielefeld: Transcript, 2020). On
  Canadian/North American comparability, see Norm Friesen, ed.,
  \emph{Media Transatlantic: Developments in Media and Communication
  Studies Between North American and German-Speaking Europe} (Cham:
  Springer, 2016).} In the following, only media studies in the sense of
\emph{Medienwissenschaft} is presented. Nevertheless, especially against
the background of a European research cooperation related to media,
which must necessarily go beyond the boundaries of scientific
disciplines, it should be conceded that a ``procedure fixated on
disciplinary nominalisms {[}can{]} hardly contribute anything to the
multi- and/or transdisciplinarity of the scientific complex `media
research' assumed here, if it does not---necessarily---end in aporia or
apologetics anyway.''\footnote{Filk, \emph{Episteme der
  Medienwissenschaft}, 25. The online book publication was preceded by
  the print book publication in 2009, and the preparation of this
  dissertation was completed in 2006. With ``media research'' now being
  established as an undertaking encompassing media studies and
  communication studies, the book's author not only sees the
  disciplinary divide as principally surmountable, but also states:
  ``Current evolutions in media research, however, point far beyond this
  specifically bi-disciplinary focus. If one applies the structural
  trend in research evolution identified in research on science,
  according to which the constitution of intended transdisciplinary
  research and development . . . has been a general feature in the field
  of science and tech-}

\hypertarget{societal-reasons-for-the-emergence-of-media-studies}{%
\section{\texorpdfstring{Societal Reasons for the Emergence of Media
Studies
}{Societal Reasons for the Emergence of Media Studies }}\label{societal-reasons-for-the-emergence-of-media-studies}}

Without a doubt, the social upheavals that led to the 1968 revolution
and that then triggered the '68ers stands in the background (also
biographically) of those who first turned to the analysis (and later
theorization) of media, primarily film and television. A conscious
turning away from the generation of fathers (who were suspected of
denying their involvement in National Socialism) led to a disruption of
continuity in society in general and at the universities in
particular---a place of intensified social analysis, discursive
comprehension of world conditions, and education in critical thinking.
In addition to these general, well-known reasons, however, there are
more specific, socio-demographic and\marginnote{nology for some time, a certain inevitability shows
  up that the epistemological core of (media) research, defined as an
  irreducible set of cognitive values and social practices, can neither
  be subsumed under general methodologies nor under scientific cultures
  of privileged provenance'' (25, my translation).} media-internal factors that
contributed to the breakaway movement of the first generation of media
scientists from their traditional academic disciplines.

The decade from 1961 to 1971 can be considered a period of university
expansion,\footnote{In 1960, there were 126 universities in Germany
  (West Germany only); in 1980, that number rose to 230 (West Germany
  only); and in 2000, it rose to 350 (West and East Germany). Ulrich
  Schreiterer, ``Hochschulen im Wettbewerb: mehr Markt, mehr Freiheit,
  mehr Unübersichtlichkeit'' {[}Universities in competition: more
  market, more freedom, more confusion{]}, Bundeszentrale für politische
  Bildung (blog), June 6, 2014, accessed February 25, 2023.} as the
percentage of an age cohort entering higher education increased from 4
percent in 1960 to 50 percent in 2014.\footnote{Schreiterer,
  ``Hochschulen im Wettbewerb.''} In the 1970s especially, the number of
students grew sharply, and their percentage went up exponentially in the
1980s, more so for female students than for males, reaching just below
1.75 million in 1989.\footnote{From under 500,000 in 1972 to 762,000 in
  1973/1974. Bundeszentrale für politische Bildung, ``Wachsender
  Studentenberg: Entwicklung der Studierendenzahlen in Deutschland;
  Studierende nach Geschlecht: WS 1947/48--WS 2013/14'' {[}Rising
  student enrollment: A growing number of students in Germany; Students
  by gender: Winter semester 1947/48--Winter semester 2013/14{]},
  accessed February 25, 2023,
  \href{https://www.bpb.de/themen/bildung/dossier-bildung/190350/wachsender-studentenberg-entwicklung-der-studierendenzahlen-in-deutschland/}{https://www.bpb.de/themen/bildung/} \href{https://www.bpb.de/themen/bildung/dossier-bildung/190350/wachsender-studentenberg-entwicklung-der-studierendenzahlen-in-deutschland/}{dossier-bildung/190350/wachsender-studentenberg-entwicklung-der-studierendenzahlen-in-deutschland/}.}

These socio-demographic changes led not only to an increase in the
number of universities and teachers, but also to a massive
differentiation of higher education into universities, universities of
applied sciences (from 1970 onwards), universities of education, as well
as art academies. Even within these institutions, a need for
differentiation of the traditional canon of subjects was present and
possible.

The development of the media itself contributed significantly to an
increasing mediatization of society, which gradually reached the general
consciousness: The filmmaking of the 1960s was marked by upheavals and
radical rejections, and so the highly left-wing (Marxist) students and
lecturers also adopted an attitude of rejection toward the traditional
aim for a coherent object of scientific debate to be analyzed by the
scientist.

Within the media system, there are further reasons internal to the
media: for instance, the emergence of television in the 1950s\footnote{The
  first public television station in West Germany, ARD, was founded in
  1950, with the first broadcast in 1952. The second German television
  channel, ZDF, began broadcasting in 1963.} and its gradual
establishment as the leading medium, replacing film and newsreels in the
cinema as well as radio in the 1960s. The sixties are the decade in
which television entered households and became the leading entertainment
medium. The question of media socialization (in childhood) shaped the
first media scholars' generation, which, in addition to television, also
witnessed the emergence of cinema festivals\footnote{E.g., Berlinale
  since 1951, Mannheimer Kultur- und Dokumentarfilmwoche since 1952 (now
  Internationales Filmfestival Mannheim-Heidelberg), Internationale
  Kurzfilmtage Oberhausen since 1954, and Filmfestival Leipzig since
  1955.} and art house cinemas\footnote{In 1970 the first German art
  house cinema, Abaton, was opened in Hamburg.} as well as the
increasing media practice in the private sphere, such as the emergence
of slides, 8mm film, and, later, video. The video apparatus, its
technical possibility of pausing, fast-forwarding, and rewinding moving
images was, according to the first media scholars,\footnote{See Paech,
  ``Warum `Medien'?''} the technical basis of an aesthetic analysis and
led to a systematic theory formation, the step that distinguishes
well-informed film criticism from scientific analysis. In the overall
structure of the media, the emergence of new media may also bring about
the ennoblement of the old. Thus, in addition to the development of
auteur film,\footnote{Important representatives of the ``New German
  Cinema'' are Volker Schlöndorff (\emph{Die Blechtrommel}, 1979),
  Alexander Kluge (\emph{Abschied von gestern}, 1966), Rainer Werner
  Fassbinder (\emph{Liebe ist kälter als der Tod}, 1969; \emph{Angst
  essen Seele auf}, 1973), and Werner Herzog (\emph{Aguirre, der Zorn
  Gottes}, 1972).} which distinguished itself from the conventional
entertainment fare, film became acceptable at universities, even beyond
literature's film adaptations and writers' radio plays, to be analyzed
in a Germanic studies way. The historicization of film and film history
also contributed to the need for trained expert knowledge that could no
longer be imparted through traditional study programs alone.

Media studies as an addressable field of knowledge and academic
discipline in Germany has only existed since the 1980s, very late
compared to other countries and younger than many other disciplines.
``Media'' in the sense of the means of communication, storage, and
information, and in the sense of ``\,`means of constructing
objects,'\,''\footnote{Günter Helmes and Werner Köster, eds.,
  \emph{Texte zur Medientheorie} (Stuttgart: Reclam, 2002).} as well as
``media theory'' itself, is a young academic field and an even younger
academic discipline. A ``specific complex of questions,''\footnote{Helmes
  and Köster\emph{,} 15.} or an aspectual approach, a knowledge
perspectivization, has developed in academia out of the blind spots of
other academic disciplines and ``at the same time anchored itself in the
consciousness of an interested public.''\footnote{Helmes and Köster, 15.}
If media are as old as human history and so fundamental that they
constitute the world, society, and human beings, the question arises as
to why it took so long for this realization to be gained.

Only with the process of a profound mediatization of society, i.e., with
the rise of the so-called mass media, has the universalization and
generalization of the problem (of the inescapable constitutional power
of media) become possible and necessary, which media theory wants to
contribute to illuminating.\footnote{Helmes and Köster, 15--16.}

The emergence of the subject can be explained by the blind spots and
desiderata of cultural studies, whose obliviousness to media and
technology was to be overcome, as emphasized by many media theorists,
such as Joachim Paech, who thus pays tribute to Friedrich Kittler, who
ended the ``media oblivion of text theory.''\footnote{Joachim Paech,
  ``Die Erfindung der Medienwissenschaft: Ein Erfahrungsbericht aus den
  1970er Jahren,'' in \emph{Was waren Medien?}, ed. Claus Pias (Zürich:
  Diaphanes, 2011), 51 (my translation).} This rhetorical trope of media
studies, its repeatedly cited self-justification, is expressed
particularly succinctly in a strategy paper by the GfM:

\begin{quote}
The long-lived media obliviousness of traditional cultural studies,
which only became apparent through media studies and has since been
eliminated, not only concerned the medial conditions of culture and its
appropriation, but also to a special degree the medial constitution of
educational processes.\footnote{GfM, "Medienkultur und Bildung,''
  Positionspapier der GfM, Strategiekommission und AG Medienkultur \&
  Bildung der GfM 2013, 3,
  \url{https://gfmedienwissenschaft.de/sites/default/files/pdf/2017-10/2013-GfM-Positionspapier.pdf}.}
\end{quote}

\noindent Media studies thereby entered a paradigmatic break with cultural studies
and formed its own epistemes.

When the media conditions of any production of cultural texts or
artifacts became the starting point of scientific analysis, and the
specific media properties---which were summarized in the concept of
mediality---became central, the epistemic foundation of media studies
were established. As Hickethier writes:

\begin{quote}
In contrast to the concept of the media, which is more strongly
connected to the objective form and the institutionalised structure, the
concept of mediality means, on the one hand, a property that is
determinative for all media in the same way. The mediality ``in itself''
is thus something that transcends media, something fundamental that
determines media communication as a whole. On the other hand, the term
refers to the set of characteristics assumed to be typical and regarded
as constitutive for respective media. This is understood to mean the
``film-specific'' or ``filmic'' in the case of film, the ``televisual''
in the case of television, the radio-phone in the case of
radio.\footnote{Hickethier, \emph{Einführung in die Medienwissenschaft},
  26 (my translation).}
\end{quote}

\noindent Areas of knowledge of media studies, though later institutionalized,
developed from the 1960s onwards in German studies, among others, which
brought expertise in radio and more especially film into the institutes
and teaching as Germanist media studies. These fields of knowledge found
their way into universities in the 1970s as institute/seminar names via
institutes dedicated to film studies, and later as film and television
studies (although the latter also has a completely different theoretical
canon).

The belated reception of the Frankfurt School (Adorno/Horkheimer's
writings on media and Kracauer's writings on film) in Germany after the
1960s was an initial spark that dominated media studies until the 1980s
and manifested itself in society as a whole as an object of media
criticism. This dominance was replaced by a paradigm shift in popular
culture and its reflection in cultural studies. The French
poststructuralism that followed this phase and other varieties of
postmodernism (rather, its academic disciplinary--configuring power of
reception) had an enormous influence on media studies---as in cultural
studies as a whole. In this phase, the study of popular culture (e.g.,
John Fiske, Stuart Hall) coexisted alongside Marshall McLuhan, the
Toronto School of communication, and the orientation toward Michel
Foucault, Jacques Derrida, and Gilles Deleuze/Felix Guattari as fields
of knowledge that formed focal points. Media studies established itself
institutionally on a broad basis in the 1980s and 1990s.

A break with the seamless connection to hermeneutically shaped cultural
studies occurred with the expulsion of the spirit from the
humanities,\footnote{Friedrich A. Kittler, \emph{Austreibung des Geistes
  aus den Geisteswissenschaften: Programme des Poststrukturalismus}
  (Paderborn: Schöningh, 1980).} i.e., the assertion of a media-based a
priori that granted technology primacy.

\hypertarget{the-turns}{%
\section{\texorpdfstring{The Turns }{The Turns }}\label{the-turns}}

Various important turns have had an influence on the formation and
development of media studies, including the spatial turn and the
performative turn. The \emph{performative turn}, which became dominant
in all cultural studies subjects in the 1990s and 2000s, has brought
together the media and the performative in media studies, especially in
media philosophy, as well as in gender media studies, which constitute
the interfaces of ``gender and media'' (as a fundamental renewal of
feminist film theory, which illuminated ``women and film''). Language
was conceived as a medium of action, of mediating the production of the
world, just as media became constitutive as performative actors, acting
in their performance, producing the world. Performativity and mediality
were assumed to be mutually constituting. Performativity as a concept in
media studies\footnote{See Andrea Seier, \emph{Remediatisierung: Die
  performative Konstitution von Gender und Medien} (Münster: LIT Verlag,
  2015).} as well as performativity and mediality in popular culture
media studies\footnote{Sybille Krämer, \emph{Technik als Kulturtechnik:
  Kleines Plädoyer für eine kulturanthropologische Erweiterung des
  Technikkonzeptes} (Ulm: Universität Ulm, 2004).} shaped theory
formation and media analyses.

The \emph{spatial turn,} so important for cultural studies,\footnote{See
  Doris Bachmann-Medick, \emph{Cultural Turns: Neuorientierung in den
  Kulturwissenschaften} (Reinbek bei Hamburg: Rowohlt, 2006), 284--329;
  Jörg Döring and Tristan Thielmann, eds., \emph{Spatial Turn: Das
  Raumparadigma in den Kultur- und Sozialwissenschaften} (Bielefeld:
  Transcript, 2008).} has produced, among other things, spatial
theory,\footnote{See Jörg Dünne and Stephan Günzel, eds.,
  \emph{Raumtheorie: Grundlagentexte aus Philosophie und
  Kulturwissenschaften} (Frankfurt: Suhrkamp, 2006).} spatial
studies,\footnote{See Stephan Günzel, \emph{Handbuch Raum} (Stuttgart:
  Metzler, 2009).} and works on topology\footnote{See Stephan Günzel,
  \emph{Topologie:} \emph{Zur Raumbeschreibung in den Kultur- und
  Medienwissenschaften} (Bielefeld: Transcript, 2007).} and
topography\footnote{See Sigrid Weigel, ``Zum `topographical turn':
  Kartographie, Topographie und Raumkonzepte in den
  Kulturwissenschaften,'' \emph{KulturPoetik} (2002).} in the
German-speaking world. The spatial turn reached its preliminary climax
in cultural studies with cartography and in media studies with
geo-media, media geosciences, and internet geographies. With the advent
of \emph{Media Geography: Theory---Analysis---Discussion,}\footnote{Jörg
  Döring and Tristan Thielmann, eds., \emph{Mediengeographie:
  Theorie---Analyse---Diskussionen} (Bielefeld: Transcript, 2009).} a
plea was made for geo-media science and geo-media theory. In the field
of visual geography, media studies emphasized its relevance to the
history of culture and knowledge,\footnote{Erhard Schüttpelz, ``Die
  medientechnische Überlegenheit des Westens: Zur Geographie und
  Geschichte von Bruno Latour's `Immutable Mobiles,'\,'' in
  \emph{Mediengeographie: Theory, Analysis, Discussion}, ed. Jörg Döring
  and Tristan Thielmann (Bielefeld: Transcript, 2009); Bruno Latour,
  ``Die Logistik der \emph{immutable mobiles,}'' in
  \emph{Mediengeographie: Theory, Analysis, Discussion}, ed. Jörg Döring
  and Tristan Thielmann (Bielefeld: Transcript, 2009).} and it
recognized the technicality of images as geo-medially
significant.\footnote{Jens Schröter, ``Das transplane Bild und der
  \emph{spatial turn,}`` in \emph{Mediengeographie: Theory, Analysis,
  Discussion}, ed. Jörg Döring and Tristan Thielmann (Bielefeld:
  Transcript, 2009).} Further on, it analyzed the geo-media impact in
individual media and apparatuses,\textsuperscript{44} explicated geo-media in terms of geo-browsing (in
the phenomenal areas of GIS, navigation, Google Earth, flight
simulation), and addressed the connection between media, globalization,
and the social.\textsuperscript{45} Mobile geo-media or locating media\textsuperscript{46} has
evolved as a new strand within geo-media.

\hypertarget{media-studies-in-conjunction-with-other-disciplines}{%
\section{\texorpdfstring{Media Studies in Conjunction with Other
Disciplines
}{Media Studies in Conjunction with Other Disciplines }}\label{media-studies-in-conjunction-with-other-disciplines}}

Media studies, a young academic discipline in Germany, embedded in the
canon of the humanities, is very close to cultural studies (in the sense
of ``\emph{die Kulturwissenschaft,}'' as a singular term, not
``\emph{die Kulturwissenschaften},'' the plural term),\textsuperscript{47} and it has, in some respects, developed parallel to
it.\textsuperscript{48} Media studies has emerged to a large extent from literary
studies and has close kinship relations to theater studies, with film
studies and, later, television studies as additional predecessor
disciplines. It has developed in further distinction from German studies
and is particularly close to the history of knowledge and cultural
studies (in the sense of the narrower canon of disciplines), because the
``development of the media techniques of generating, storage,
representation and transmission of perceptions, experiences and
knowledge''\textsuperscript{49} is one of the main aspects of media studies. The
subject of media studies has its core identity in the media's logic or
the entelechy of the media.

\newpage Media\marginnote{\textsuperscript{44} Wolfgang Hagen,
  ``Zellular---Parasozial---Ordal: Sketches for a Media Archaeology of
  the Mobile Phone,'' in \emph{Mediengeographie: Theorie, Analyse,
  Diskussionen}, ed. Jörg Döring and Tristan Thielmann (Bielefeld:
  Transcript, 2009).} as\marginnote{\textsuperscript{45} Pablo Abend, \emph{Geobrowsing: Google Earth
  und Co.; Nutzungspraktiken einer digitalen Erde} (Bielefeld:
  Transcript, 2013).} an\marginnote{\textsuperscript{46} Regine
  Buschauer and Katherine Willis, \emph{Locative Media:}
  \emph{Medialität und Räumlichkeit} (Bielefeld: Transcript, 2013).} object\marginnote{\textsuperscript{47} The
  plural form ``\emph{die Kulturwissenschaften}'' designates the
  humanities, whereas the singular form designates one specific academic
  discipline renewed in the eighties which seeks to investigate ``the
  social and technical institutions created by human beings, the forms
  of action and conflict formed between people, and their horizons of
  values and norms, especially insofar as these require special levels
  of symbolic and media-based mediation for their constitution,
  transmission, and development.'' Hartmut Böhme,
  ``Kulturwissenschaft,'' in \emph{Reallexikon der deutschen
  Literaturwissenschaft}, vol. 2 (Berlin: DeGruyter, 2000), 356.
  However, with respect to proximity to cultural studies, cf. Friedrich
  Kittler, \emph{Eine Kulturgeschichte der Kulturwissenschaft} (München:
  Fink, 2001).} of\marginnote{\textsuperscript{48} Independent developments took place in film and television
  studies. For investigation of parallels and differences, see
  "Kulturgeschichte als Mediengeschichte (oder vice versa)?,'' special
  issue, \emph{Archiv für Mediengeschichte} (Bauhaus University Weimar)
  6 (2006).} study\marginnote{\textsuperscript{49}\setcounter{footnote}{49} Strategy Commission of the GfM, ``Core Areas of
  Media Studies.''} forms possible sub-disciplinary
configurations far beyond cultural studies, as in the social sciences:
e.g., media sociology, media law, media economics, and media psychology.
And even in the natural sciences media are of disciplinary relevance,
e.g., media informatics.

Media as a designated research subject, represented using corresponding
expertise, is usually found in German studies; often in the form of
expertise in film history, theory, and aesthetics, and very often in
other philologies. Adaptations, genres, and formats, as well as
fundamental narratological questions and aesthetic-analytical procedures
and methods of interpretation, which are linked to general intellectual
history, are common to these disciplines. Media studies grew out of
pedagogy adjacent to German studies and established itself in the 1970s,
in its early days, as critical social studies in the form of media
studies. For this reason, in addition to German media studies, classical
media pedagogy and renewed media education are strong areas of knowledge
of a theoretical, practical, and praxeological nature, where scholars
from their respective disciplines meet.

In theater studies and art, visual/electronic media are often included
in seminars or institutes and represented in specializations of
professorships, too. Media studies was able to connect to the research
direction of cultural studies, which was dedicated to the materiality of
communication.\footnote{Hans-Ulrich Gumbrecht and Karl Ludwig Pfeiffer,
  \emph{Materialität der Kommunikation} (Frankfurt: Suhrkamp, 1988).}
Aesthetics is understood as a media science \emph{avant la lettre} and
is an integral part of the subject. Media studies does not make a clear
distinction between methods and theories, and names in particular
``philological-hermeneutic, art-scientific, philosophical, sociological
and psychological theories and methods''\footnote{Strategy Commission of
  the GfM, ``Core Areas of Media Studies.''} in its approach to
theorizing, although the transformative power and enrichment of the
repertoire of theories in cultural studies are emphasized.

Parts of media studies configured themselves as ``media theory'' and
were later perceived as German media theory in their
specific---internationally distinct---epistemic grounding. In this
current, German media theory, media philosophy, and cultural technology
research\footnote{Lorenz Engell and Bernhard Siegert,
  ``Medienphilosophie,'' \emph{Zeitschrift für Medien- und
  Kulturforschung} 2 (2010).} became important from the beginning of the
2000s.

\hypertarget{institutionalization}{%
\section{\texorpdfstring{Institutionalization
}{Institutionalization }}\label{institutionalization}}

The term \emph{Medienwissenschaft} showed up in the 1950s, first used by
Erich Feldmann.\footnote{Friedrich Knilli, ``Wie aus den Medien eine
  Wissenschaft wurde: Exposé für eine soziobiographische
  Fachgeschichte,'' \emph{Medienwissenschaft: Rezensionen/Reviews} 20,
  no. 1 (2003), 7.} In 1950, the Hans-Bredow-Institut in Hamburg was
founded, directed by Egmont, with its most prominent researcher, Gerhard
Maletzke. In 1961, followed the Institute for Language in the Technical
Age (\emph{Sprache im technischen Zeitalter}, SPRITZ) at the TU
(University of Technology) Berlin, founded by Walter Höllerer, in which
media studies as an autonomous discipline was established by Friedrich
Knilli. Knilli was appointed Professor of General Literary Studies, with
a focus on media studies, at the TU Berlin in 1972. In 1982, Knilli
founded the study program ``\emph{Medienberatung}'' (media advisory) at
the TU Berlin, the first genuine media studies program, for which he was
responsible, together with Siegfried Zielinski.\footnote{Siegfried
  Zielinski was a student of Knilli and was appointed professor of
  audio-visuals at the University of Salzburg in 1989.} Later on, three
of Knilli's students became appointed professors: Siegfried Zielinski
(Salzburg, Köln), Knut Hickethier (Hamburg), and Joachim Paech
(Konstanz).

As a sub-discipline of theater studies, there were professorships with a
partial media studies specialization in the Institute for Theater
Studies in Vienna from the beginning of the 1970s. Hans-Ulrich Gumbrecht
at the University of Siegen organized five interdisciplinary research
colloquia on the epistemological reorientation of the humanities between
1981 and 1989. The initial spark, formative for the subject, was the
media studies research association ``Aesthetics, Pragmatics and History
of Screen Media,'' which was a DFG Collaborative Research Centre at the
University of Siegen from 1985 to 2000, directed by Helmut Kreuzer, and
later by Helmut Schanze.\footnote{Werner Faulstich, Siegfried Schmidt,
  Irmela Schneider, Reinhold Viehoff, and Jutta Wermke participated and
  were later appointed professors.} Knilli described the social and
historical forces that united and divided early figures: ``What unites
the stubborn media scholars from the various West German universities in
the fifties, sixties and seventies is media criticism, for most of them
are children of two wars: the Second World War and the Cold War, which
at the same time splits them into several camps again.''\footnote{Knilli,
  ``Wie aus den Medien eine Wissenschaft wurde,'' 19.} In 1987,
Gumbrecht led the first DFG Research Training Group in the Humanities,
entitled ``Forms of Communication as Forms of Life.'' The Käthe
Hamburger Kolleg IKKM (2008--2020)\footnote{IKKM website, accessed
  January 28, 2022, https://www.ikkm-weimar.de.} and the International
College for Cultural Techniques Research and Media Philosophy at the
Bauhaus University Weimar have had a significant influence on media
studies. Furthermore, the DFG Research Training Group ``Locating Media''
at the University of Siegen, which existed from 2009 to 2021, was of
great influence.

An academic association for media studies was founded in 1985 as the
Society for Film and Television Studies (\emph{Gesellschaft für}
\emph{Film- und Fernsehwissenschaft,} GFF), later renamed the Society
for Media Studies (\emph{Gesellschaft für Medienwissenschaft,} GfM). The
GfM currently has twenty-nine working groups. Between 2001 and 2007, the
DFG research group ``\emph{Bild---Schrift---Zahl}''
(Image---Writing---Number) worked at the Humboldt University of
Berlin.\footnote{Image, writing, and number are basic media of modern
  communication and have been studied in their historical genesis up to
  their current effectiveness in digital encoding, storage, and
  processing. See Helmholtz-Zentrum für Kulturtechnik, ``Founding
  Project of the Helmholtz Centre for Cultural Technology: Theory and
  History of Cultural Techniques,'' Humboldt University of Berlin
  (website).}

In 2009, the first DFG symposium on media studies took place, marking
another symbolic stage in the establishment of the subject.\textsuperscript{59} The
conference series ``Hyperkult'' has had a strong impact on media
studies, especially media theory, since 1991.

\newpage The\marginnote{\textsuperscript{59}\setcounter{footnote}{59} See
  Ulrike Bergermann, \emph{Leere Fächer: Gründungsdiskurse in Kybernetik
  und Medienwissenschaft} (Münster: LIT Verlag, 2015), 15.} beginnings of institutionalized German-language media studies are
linked to research on literary adaptations (radio and film),\footnote{See,
  e.g., Friedrich Knilli, Knut Hickethier, and Wolf-Dieter Lützen, eds.,
  \emph{Literatur in den Massenmedien: Demontage von Dichtung?}
  (München: Hanser, 1976); Franz-Josef Albersmeier, \emph{Die
  Herausforderung des Films an die französische Literatur:}
  \emph{Entwurf einer "Literaturgeschichte des Films}'' (Heidelberg: C.
  Winter, 1985); Michael Schaudig, \emph{Literatur im Medienwechsel:
  Gerhart Hauptmanns Tragikomödie ``Die Ratten'' und ihre Adaptionen für
  Kino, Hörfunk, Fernsehen; Prolegomena zu einer Medienkomparatistik}
  (München: Verlegergemeinschaft Schaudig, 1992).} on the nexus of media
and the Holocaust,\footnote{See Friedrich Knilli and Siegfried
  Zielinski, \emph{Holocaust zur Unterhaltung} (Berlin: Elefanten Press,
  1982).} and on internationalization/Europeanization.\footnote{See,
  e.g., Siegfried Zielinski and Kurt Luger, eds., \emph{Europäische
  Audiovisionen: Film und Fernsehen im Umbruch}, Neue Aspekte in Kultur-
  und Kommunikationswissenschaft, vol. 7 (Wien: Österreichischer Kunst-
  und Kulturverlag, 1993); Siegfried Zielinski, \emph{Video:
  Apparat/Medium, Kunst, Kultur; Ein internationaler Reader} (Frankfurt:
  Peter Lang, 1992).} Before institutionalized media studies, ``media
studies,'' primarily linked with the name of Helmut Schanze,\footnote{Helmut
  Schanze, \emph{Medienkunde für Literaturwissenschaftler: Einführung
  und Bibliographie} (München: Fink, 1974).} was the term for a
configuration of knowledge that theorized mediality
``media-reflexively'' or in ``thinking the media.'' Since media studies
expertise developed within other disciplines, including theater studies
and German studies, and many media studies writings can be identified
avant la lettre in cultural history, a few publications will be
mentioned here by way of example, without making any claim to firstness
or exclusivity.

\hypertarget{first-publications}{%
\section{First Publications}\label{first-publications}}

The first monographs in the discipline of media studies were published
by Friedrich Knilli, who presented a work on radio plays in
1959.\footnote{Friedrich Knilli, \emph{Das Hörspiel in der Vorstellung
  der Hörer: Eine experimental-psychologische Untersuchung} (PhD diss.,
  University of Graz, 1959).} Helmut Schanze published
``\emph{Medienkunde für Literaturwissenschaftler}'' (Media Studies for
Literary Scholars) in 1974, followed by ``\emph{Literaturgeschichte als
Mediengeschichte}'' (Literary History as Media History) in 1976--1977.
In addition to German media studies, the first monographs and conference
papers with ``media analysis'' in the title can be traced back to 1972.
``Media analysis,'' one of the first main currents of media studies,
which also gave its name to the study program, was related to television
and audio-visual media and was supported by educational societies, among
others. ``Media knowledge'' (\emph{Medienkunde}) and ``media criticism''
are further pre-disciplinary or early configurations of knowledge,
followed by ``media work.'' In the second half of the 1980s, media
effects research became a field of knowledge, mainly in communication
studies, which attracted a lot of attention, as did media analysis as
cultural and social criticism. During this period, titles mentioning
``media theory,'' later a major paradigm of media studies, appeared for
the first time. At the end of the 1990s, the first compendia\footnote{See,
  e.g., Werner Faulstich, \emph{Medientheorien: Einführung und
  Überblick} (Göttingen: Vandenhoeck \& Ruprecht, 1991); Daniela Kloock
  and Angela Spahr, \emph{Medientheorien: Eine Einführung} (Paderborn:
  Fink, 1997).} and handbooks appeared in renowned academic publishing
houses. Moreover, in the second half of the 1990s, cultural studies (in
conjunction with media analysis) moved into the core knowledge area of
media studies.

\hypertarget{special-focus-areas}{%
\section{Special Focus Areas}\label{special-focus-areas}}

One important strand in media studies is the orientation toward the
history of knowledge, and with it towards the history of science, in a
constant effort to bring the two poles of tension, the technical and the
social, to a point of intersection, that of the media constituted for
reflection. It was precisely this basic disciplinary configuration that
led almost inevitably to the study of Actor--Network Theory (ANT).

Due to the theoretical reflection of the fundamental role of technology
for media in their social uses, German-language media studies is close
to English-language Science and Technology Studies (STS), in particular
to ANT, which emerged from it. ANT was strongly received\footnote{See,
  e.g., Gerd Kneer, Marcus Schroer, and Erhard Schüttpelz, eds.,
  \emph{Bruno Latours Kollektive: Kontroversen zur Entgrenzung des
  Sozialen} (Frankfurt: Suhrkamp, 2008); Tristan Thielmann and Erhard
  Schüttpelz, eds., \emph{Akteur---Medien---Theorie} (Bielefeld:
  Transcript, 2013).} in media studies from the mid-2000s onwards and
secured mainstream status until the mid-2010s, primarily through the
International College for Cultural Techniques Research and Media
Philosophy (IKKM, 2008--2020) at the Bauhaus University Weimar. ANT in
media studies elaborates the agency of human and non-human actors in
operative sign chains. Uncovering and conceptualizing the agency of
media, artifacts, and inscriptions---as well as the media-bound nature
of all technical and social processes\footnote{Thielmann and Schüttpelz,
  \emph{Akteur---Medien---Theorie}, 15.} from which technical,
scientific, and organizational operations emerge---has captured the
profound interconnectedness of media, technology, and the social.

\hypertarget{examples-and-exemplary-questions}{%
\section{Examples and Exemplary
Questions}\label{examples-and-exemplary-questions}}

Media studies was expected to provide an answer to the question: ``What
is a medium?''\footnote{Stephan Münker and Alexander Rösler, eds.,
  \emph{Was ist ein Medium?} (Frankfurt: Suhrkamp, 2008).} More
precisely, the question was aimed at the concept of media in media
studies, when the field---while aiming at the media (individual, mass
media, and media system) and claiming to have expertise in them and
their teaching---did not follow the vernacular concept of media.

Different definitions of media, each of them deriving from separate
theories and schools, coexist peacefully, and many distinctions between
media, such as those between storage and dissemination media, have
attracted interest in media studies. The subject identity of media
studies, which first distinguishes itself as a scientific discipline,
was acquired in the struggle for the concept of media. This has been
answered with the overarching, far more comprehensive category of
mediality. The ``autonomy of the media,'' the ``media specific,'' the
``thinking of the media property,'' and the ``media logic'' became the
core identity of the discipline. A medium is thought of as an
``actualization of the media potential,''\footnote{Lorenz Engell,
  ``Tasten, Wählen, Denken: Genese und Funktion einer philosophischen
  Apparatur,`` in \emph{Medienphilosophie: Beiträge zur Klärung eines
  Begriffs}, ed. Stefan Münker, Alexander Roesler, and Mike Sandbothe
  (Frankfurt: Fischer, 2003), 54\emph{--}55.} thus forcing an object
construction through philosophical reflection, diametrically opposed to
the everyday language concept of media, in which media are equated with
transmission apparatuses (and their typified contents). The concept of
``media'' in the substantial and historical sense is
rejected.\footnote{Lorenz Engell and Joseph Vogl, ``Zur Einführung,'' in
  \emph{Kursbuch Medienkultur: Die maßgeblichen Theorien von Brecht bis
  Baudrillard,} ed. Claus Pias et al. (Stuttgart: DVA, 1999), 10.} The
demarcation of ``media'' from forms of representation, techniques, and
symbolism as a sufficient justification or form of thinking and
negotiation of media is rejected, with the understanding that media
studies should instead find its legitimacy in the relationship of the
elements mentioned. Ulrike Bergermann has raised the question of whether
this is media studies ``as form without subject or object,'' as ``pure
media.''\footnote{Bergermann, \emph{Leere Fächer}, 20.} The foundational
discourse represents an attempt to legitimize media studies in the void.
A media science without an object or subject can exist, or can only
exist in this way, because media science finds its core in ``placing a
medium, a mediality, a translation, a blind spot, an empty centre at the
centre of a discipline, which, after all, does not receive its
legitimation as pure mediality research at all, but from the constantly
new elaboration of references between individual media/artifacts and
their mediality.''\footnote{Bergermann, 20.}

In the many introductions to media studies during the first decade of
the twenty-first century, the history of knowledge, mediality, and
epistemology were conceptually intertwined---explicitly or implicitly in
the context of the founding of a subject. To interrogate knowledge
production in terms of mediality---in terms of its media constituents
and legitimacy---has remained the mandate of media studies.

\begin{quote}
The path of media cultural studies leads into and out of the media and
their processes. It takes many branches via techniques and technologies,
materials and constitutions, the tangible and the invisible, and it
still often takes the (re)route via literature.\footnote{Oliver Ruf,
  Patrick Rupert-Kruse, and Lars C. Grabbe,
  \emph{Medienkulturwissenschaf} (Wiesbaden: Springer, 2022), 25.}
\end{quote}

\noindent The term ``media cultural studies'' is evidence of a foundation in
literary and cultural studies, but this is accomplished in a theory
formation that grasps the fundamental constitutional power of the media.
This constitutional force of media goes beyond individual media as a
dispositive that is at the same time (culturally) technically based,
symbolically configured, and operating via various types of signs
(image, number, etc.). A medium has an effect on the individual as a
subject-constituting power or on society as a collective. Of course, the
converse is also true, as media and technology are deposits of the
social.

\hypertarget{cultural-studies-research-in-the-german-speaking-world}{%
\section{Cultural Studies Research in the German-Speaking
World}\label{cultural-studies-research-in-the-german-speaking-world}}

Media studies, in its fundamental cultural-scientific
self-understanding, can grasp any mediation that is based on a symbolic
sign character as media-based and thus define writing, image, and
number\footnote{Sybille Krämer and Horst Bredekamp, eds., \emph{Bild,
  Schrift, Zahl} (Paderborn: Fink, 2003).} as media. It can also
identify the cultural techniques\footnote{For fundamental information on
  cultural techniques see Bernhard Siegert, \emph{Cultural Techniques:
  Grids, Filters, Doors, and Other Articulations of the Real} (New York:
  Fordham University Press, 2015).} of reading, writing, and arithmetic
as media configurations. The cultural techniques perform embodied,
culturally-configured mediations that are bound to symbolic signs. In
this way, media studies not only reaches far into the historical realm
(for example, the history of writing), but can also include any
thematization\footnote{See also the treatise by Günter Helmes on the Old
  Testament prohibition of images: Günter Helmes and Werner Köster,
  \emph{Texte zur Medientheorie} (Ditzingen: Reclam, 2002), 23.} of
image, writing, and number as a media-theoretical reflection in the
canon of media studies, whereby texts from Greek and Roman antiquity
also become parts of its foundation. The accompanying canonization, the
establishment of reference texts, are reconfigurations of classical
texts from history, aesthetics, literature, art, and philosophy under
the new interest-guiding perspective of mediation, of being media-based
as a condition sine qua non. Media also enable in the first place the
more fundamental processes of perception, thinking, and feeling, thus
tying media to the basic conditions of human existence. The
physiological condition of perception, the five human senses, become
mediated conditions by dint of their connection to perceiving and
knowing as cognition and knowledge. Conceived as media configurations of
knowledge, media and knowledge are short-circuited.

Media as an anthropological constant can then be thematized in
evolutionary terms as historically conditioned conditions of expression
that prove themselves socially, culturally, and politically in their
collective dimension as well as psychologically and pedagogically in
their individual dimension.

Under ``media,'' mediations located on different levels can be described
according to: a) the different sign natures (image, writing, number); b)
the arts (drama, painting, poetry, song and dance) and media as
intermedia aesthetic manifestations; and c) the specific configuration
levels of formats and genres.

In addition to Kittler's media studies, intermediality was another major
paradigm of media studies which was able to continuously differentiate
itself into sub-disciplines after its successful founding while still
maintaining an inner coherence in its identity core of media
theory.\footnote{Rainer Leschke, ``Medienwissenschaften und Geschichte:
  Morphologie einer Wissenschaft.''}

Rainer Leschke describes how

\begin{quote}
intermediality analysis not only ensured a systematic networking of
media studies within cultural studies, but as a science of the media
system and media cultures, it virtually offers itself as an integration
and anchor point of cultural studies. In this way, however, media
studies has established itself as an independent system of knowledge
with an object area that cannot be closed off in principle and an
integral networking within cultural studies.\footnote{Leschke, 14.}
\end{quote}

\noindent As detailed, media studies considers any form of communication involving
symbolic signs to be media-based, and ``sign'' here encompasses writing,
images, and numbers. It also examines cultural techniques such as
reading, writing, and arithmetic as forms of media configurations. The
establishment of reference texts from various disciplines, reinterpreted
under the perspective of mediation, contributes to the canonization of
media studies.

\newpage\hypertarget{franco-german-cooperation-in-media-and-communication-studies}{%
\section{Franco-German Cooperation in Media and Communication\\\noindent
Studies}\label{franco-german-cooperation-in-media-and-communication-studies}}

Due to the above-mentioned division of the sciences dedicated to media
and communication into cultural media studies and social communication
studies (in Germany), cross-border cooperation poses a particular
challenge. However, in many joint degree programs, and even more so in
the bi-national research alliances (DFG-ANR, German Research Committee
and French National Science Foundation), one finds examples of scholars
who successfully negotiate this unusual terrain. Curricula are
coordinated, if not developed jointly. Theses are jointly supervised.
Advisory services, i.e., academic orientation of Bachelor's and Master's
graduates, are coordinated. Even if this is not a transdisciplinary but
an interdisciplinary cooperation, there must not only be openness to
other methodological approaches, but these must also be appropriated in
an integrative manner. Even if the other canon of theories could be
handled additively, if linguistic and attention-economic obstacles were
overcome, and if a number of theorists/theories could be identified that
are read by both disciplines, this would turn out to be more difficult
or even incompatible when it comes to the research design and the
applied methodology or the form of argumentation.

Cooperation in German-French media studies requires cognizance of three
different levels of academic operation: a) institutional formations; b)
policies and politics of knowledge acquisition and of science; and c)
contours of intellectual practice, including research topics and
theoretical orientations. The institutional level concerns German-French
media organizations, about media-specific institutions, and thus
captures formalized relations of broadcasters, regulatory authorities,
and the like. German-French cooperation as it develops through its
infrastructure and through its technological standardizations is also at
the center of media studies. In terms of knowledge policy, questions
arise about continuities and discontinuities from analogue to digital,
especially in their different configurations in France and Germany. The
disciplinary configuration of those sciences dedicated to the media in
the individual German-speaking and Francophone countries is also an
important domain of inquiry, one that connects with issues pertaining to
the configuration of interdisciplinary research between German
studies/French studies, media studies, and socio-political fields of
knowledge. The most important scientific organizations (university and
research organizations, determined by laws, ministerial
responsibilities, and financial support for research) whose work
directly relates to the issue of German-French cooperation in media
studies are the German-French University (DFH-UFA), the bi-national
research alliances (DFG-ANR, the German Research Committee and French
National Science Foundation), the German Academic Exchange Service
(DAAD), as well as many EU-funded research institutions and projects.
Strategies of cooperation play an important role in the common
German-French study program \emph{Europäische Medienkultur}.

The connection between German-French cooperation and media studies
prompts questions about the form of institutionalization, the subject
area, the content, and the structure. In order to sound out perspectives
of German-French media studies, we must consider the practices of
teaching media studies in Germany and France. In terms of research, we
must identify overlaps, intersections, and crossing references in the
media of Germany and France. The conceptualization of Franco-German
media studies can start from the respective national academic
disciplines and pose questions about the cross-border dimension in
several respects. The main respects are:

\begin{enumerate}
\item
  the sociogenesis of the discipline (to be recorded via the founding of
  institutes and designation of chairs and courses of study),
\item
  the intellectual biographies of its bearers and their European
  intercultural experiences,
\item
  the disciplines from which they developed (history of knowledge or
  history of science),
\item
  the scientific organizations (university and research organizations,
  determined by laws, ministerial responsibilities, or financial support
  for research),
\item
  the intellectual currents (within the country, or more broadly, in
  Europe or around the world),
\item
  languages and language transfer (through translations, the phenomenon
  of reception, and waves), language barriers (independent parallelisms,
  opposites parallelisms, and national ``special paths''), and
\item
  the cooperation of science, including transnational joint study
  programs as well as the cooperation between media actors (in the
  cultural, in the media-economic, and in the field of media
  technology).
\end{enumerate}

The object of study takes its starting point in the national framework,
and we can ask whether or not this national framework can be transcended
to include the bi-/tri-/multi-national dimension. Firstly, the
cross-border dimension of the sociogenesis of the disciplines, which can
be grasped through the founding of institutes and the appointment of
professors and degree programs, shows that in France, as in Germany, the
founding of institutions and the appointment of professors has so far
taken place within a national framework. Anchoring in the national
scientific community is indispensable, even when international
networking is required.\footnote{Sarah Cordonnier and Hedwig Wagner,
  ``Déployer l'interculturalité: les étudiants, un vecteur pour la
  réflexion académique sur l'interculturel: Le cas des sciences
  consacrées à la communication en France et en Allemagne,'' in
  \emph{Interkulturelle Kompetenz in deutsch-französischen
  Studiengängen: Les compétences interculturelles dans les cursus
  franco-allemands} {[}Intercultural competency in German-French degree
  programs: Key competencies for higher education and employability{]},
  ed. Gundula Gwenn Hiller et al. (Wiesbaden: Springer VS, 2017), 232.}
The supposed internationality and universalizing tendency of media
theories---which produces theory that does not address the linguistic
and intellectual context of origin and its conditions and
limitations---becomes particular when addressed and more closely
specified culturally, or reveals itself as a geo-media-political
negotiating space. The consideration of geographical, institutional, and
national coordinates in the formation of media studies theory even leads
Norm Friesen and Richard Cavell to speak of the ``geography of media
studies.''\footnote{Norm Friesen and Richard Cavell, ``Introduction,''
  in \emph{Media Transatlantic: Developments in Media and Communication
  Studies between North American and German-Speaking Europe}, ed. Norm
  Friesen (Cham: Springer, 2016).} German-language pedagogy books on
media studies show an account that disregards national frames and
contexts of origin and traces lines of development purely from
theory.\footnote{Wagner, \emph{Europäische Medienwissenschaft}.}
Intellectual biographies and the history of reception have their place
in this, albeit very marginally, but these books rarely include the
institutional, geographical, and national dimensions of the origins of
media theory, unless these are addressed from the perspective of
regional studies, as for example in studies of Russian media, where
media theory is overlaid by a centuries-long cultural
imprint.\footnote{Ulrich Schmid, \emph{Russische Medientheorien} (Bern:
  Haupt Verlag, 2005).} The consideration of geographical,
institutional, and national coordinates in the formation of media theory
has so far been almost completely absent from the contributions of
European media studies,\footnote{An exception to this broad
  generalization can be found in my own book: Hedwig Wagner,
  \emph{Europäische Medienwissenschaft}.} including Franco-German media
studies.

The Franco-German degree programs, politically desired by the
Franco-German Council of Ministers and promoted by the Franco-German
University (DFH-UFA),\footnote{Université
  franco-allemande/Deutsch-Französische Hochschule (website), accessed
  February 5, 2022, https://www.dfh-ufa.org/.} are offered at
Bachelor\textquotesingle s and Master\textquotesingle s levels. There
are six such degree programs where media/communication form the main
focus of study. Being in charge of a German-French joint degree program
in media studies, the author's French colleague Sarah Cordonnier and the
author conducted an interview series on the intercultural literacy among
the academic staff. They thematized being in charge of a French-German
cooperation, in a study program as well as in a research project.
Created in 1997 with support from the Franco-German University (UFA),
the Franco-German joint degree program \emph{Europäische Medienkultur}
(EMK) is a long-term observation point of intercultural dynamics. The
particularities of its history and the disciplines associated with it
constitute relevant elements for the analysis or explanation of these
dynamics. The content of the curricula themselves also has some
interesting features. The courses offered by the EMK joint degree
program are linked to the \emph{Sciences de
l\textquotesingle information et de la communication} (SIC) in France
and the \emph{Medienwissenschaft} in Germany. In general, the subject
matter (media and communication) may seem to fit together harmoniously,
but there are important differences in the objects, methods, theories,
and modes of conceiving what needs to be investigated and taught.

SIC was constituted as a section (the seventy-first) by the National
Council of Universities in 1975. Some speak of it as an
``interdiscipline,'' and there are many tensions within it, as much
because of the initial juxtaposition between several fields (media
content studies, library sciences, part of science studies, and
museology---but not film studies) as well as the plurality of methods
and theoretical resources that researchers solicit and import from other
disciplines (sociology, language sciences, and semiotics).\footnote{Sarah
  Cordonnier, ``Looking Back Together to Become `Contemporaries in
  Discipline,'\,'' \emph{History of Media Studies} 1 (2021).}
\emph{Medienwissenschaft} has been developing since the end of the 1980s
and the 1990s, becoming independent of German studies and theatre
studies, and then of film studies, which nevertheless remains an
important subject; essentially based on speculative, hermeneutic
reasoning, \emph{Medienwissenschaft} is impregnated with French
poststructuralist references. This discipline is separate from
\emph{Kommunikationswissenschaft}, a social and empirical science, whose
objectives could, to a certain extent and in part, overlap with those of
SIC.

In the case of SIC, \emph{Medienwissenschaft}, and
\emph{Kommunikationswissenschaft}, the situation is even more singular.
Indeed, these disciplines are of recent vintage (or profoundly reshaped
recently, in the case of \emph{Kommunikationswissenschaft}\footnote{Averbeck-Lietz,
  ``Comparative History of Communication. Studies: France and Germany,''
  \emph{The Open Communication Journal} 2, no. 1 (2008).}), and they
have been shaped in contexts quite different from those which saw the
advent of sociology, geography, history, and demography at the end of
the nineteenth century. They do not even share a disciplinary title, and
have received little scientific attention, even in their own countries,
although student interest in the themes they cover is leading to
significant institutional and academic development.

The different disciplines in France and in Germany interact not as
entangled history (\emph{histoire croisée})---i.e., not as a joint
history of science---but rather as the history of knowledge in the sense
of intellectual currents, which can of course also mean, as described
above, the reception of individual theorists or schools, be it the
Frankfurt School or French post-structuralism.\footnote{For an account
  of the different histories of knowledge or history of science, see
  Averbeck-Lietz, "Comparative History''; for mediology, see Birgit
  Mersmann and Thomas Weber, eds., \emph{Mediologie als Methode}
  (Berlin: Avinus, 2008); and for the branching out of cybernetics in an
  international context, see Claus Pias, ed., \emph{Cybernetics: the
  Macy Conferences 1946--1953: The Complete Transactions} (Chicago:
  Chicago University Press, 2016).} Thus languages, language transfer
(through translations, the phenomenon of reception, and waves) and
language barriers (independent parallelisms, opposite parallelisms, and
national ``special paths'') are reflected and acted upon by the creation
of publishing houses\textsuperscript{88} and the translation policies of scientific
journals.\textsuperscript{89} Outside academia (but as a core domain of
employability of such transnational commuters) are the following media
institutions, mentioned by the German-French council for culture: ARTE,\marginnote{\textsuperscript{88} Thomas Weber, a media scholar, runs the
  Avinus-Verlag, where mediology and other French theory texts are
  translated and published.}
the\marginnote{\textsuperscript{89}\setcounter{footnote}{89} Take, for example, the \emph{Archiv für
  Mediengeschichte}, or the \emph{Zeitschrift für Kultur- und
  Mediengeschichte}.} German-French film academy Atelier Ludwigsburg-Paris, the Office of
Film and Media of the Institut Français in Germany, and
\emph{ParisBerlin}, a German-French news magazine.\footnote{Deutsch-Französischer
  Kulturrat (website), accessed February 11, 2022,
  http://www.dfkr.org/links/kultur-und-medien-allgemein/.}

The study of the academic disciplines \emph{Medienwissenschaft} in
Germany and \emph{Sciences de l\textquotesingle information et de la
communication} in France by Cordonnier and myself\footnote{Sarah
  Cordonnier and Hedwig Wagner, ``L'interculturalité académique entre
  cadrages et interstices,'' in \emph{France-Allemagne: Incommunications
  et convergences,} ed. Gilles Rouet and Michaël Oustinoff (Paris: CNRS
  Éditions, 2018); Cordonnier and Wagner, ``Déployer
  l'interculturalité''; Sarah Cordonnier and Hedwig Wagner, ``La
  discipline au prisme des activités internationales dans les
  trajectoires de chercheurs en France et en Allemagne (encadré),'' in
  \emph{Discipline, interdisciplinarité, indiscipline,} ed. Jean-Michel
  Besnier and Jacques Perriault (Paris: CNRS Éditions, 2013); Sarah
  Cordonnier and Hedwig Wagner, ``Academic Interculturality in
  Communication Studies'' (paper presented at the European Communication
  Research and Education Association's 5th European Communication
  Conference, Lisbon, November 12--15, 2014).} derives from the
intellectual biographies of the researchers and is informed by their
German-French intercultural experiences. We carried out the interviews
with German and French professors under the following headings: a) age
in intercultural academic engagement, b) the role of intercultural
academic institutions, and c) the strategic implementation of future
academic work. The results showed that academic interculturality is
anchored in: a) disciplinary contexts, which are national and
institutional; b) academic internationalization; c) established forms
and formats (in the sense of dispositives); and d) non-institutionalized
encounters.

Thus, only the institutional inscription of the intercultural experience
guarantees its durability, but it is then inscribed in multiple and
often paradoxical ways. Indeed, interculturality understood as an
encounter with researchers, students, theories, scientific practices,
and especially interculturality understood as the ``international
circulation of ideas'' with all the effects of inscription in national
contexts that it presupposes, resonates with the policies of academic
internationalization, where internationalization correlates with
prestige but often by going toward the lowest common scientific and
pedagogical denominator.\footnote{Cordonnier and Wagner, ``Déployer
  l'interculturalité,'' 232--33.} Nevertheless, the international is
needed in order to write a ``meaningful history'' of media studies, as
Cordonnier clarifies:

\begin{quote}
Meaningful history implies not only theoretical reading and
methodological compliance, but also clarification of irritating
differences that turn out to be more ``cultural'' than ``scientific'' .
. . the method of developing an argument, the writing style, the
respective weight of the theory and the empirical material, the interest
in remote domains or areas, and the acceptance that there is more than
one, in the words of Donsbach, ``right way to scientific knowledge.'' In
sum, meaningful history relies upon the resolute creation of an
international ``milieu,'' where the delicate balance between
methodological demand and political significance can be shaped afresh.
\footnote{Cordonnier, ``Looking Back Together,'' 2--3. This passage
  refers to Wolfgang Donsbach, ``The Identity of Communication
  Research,'' \emph{Journal of Communication} 56, no. 3 (2006): 437.}
\end{quote}

\hypertarget{potential-criticism-and-outlook}{%
\section{Potential, Criticism and
Outlook}\label{potential-criticism-and-outlook}}

The ``new materialist turn'' was a major development in the European
Research Area (ERA). It continued the established strand of the
materiality of communication in cultural studies and extended it to new
academic disciplines (feminism, philosophy, science studies, history,
and media studies), while expanding it from history (of technology) to
the field of Theory of Knowledge (TOK). The focus shifted from
exclusively text-based objects of study to the study of objects.
Apparatuses, artifacts, and all other objects of the means of
communication and information are understood as object-based traces
pertaining to the history of knowledge, the history of ideas, and the
circulation of ideas and knowledge. The focus on objects and ``boundary
objects'' has been particularly pronounced in Science and Technology
Studies (STS) and especially in Actor--Network Theory (ANT). The focus
on networks and networks within and beyond ANT created an
interdisciplinary field of research: network studies. The core of this
new approach is to explore the material level, the institutional level,
and the discursive level as such, with cognizance of their
interconnectedness and their manifold interdependencies. Their
interconnectedness brings new insights and counterbalances the
prevailing, purely institutional approach. This new strand of research
has been particularly fruitful for European studies, first and foremost
the \emph{Tensions of Europe}\footnote{Tensions of Europe (website),
  accessed October 30, 2021, https://www.tensionsofeurope.eu/.} research
network. The analysis of how superior information and communication
technologies (ICTs)---along with their foundation, the international
media-technical infrastructure---provide pan-European (cross-bloc) means
of communication is crucial and innovative in an East-West European
perspective. These new ICTs have had a strong impact on the importance
of communication.

German-language media studies is not only very broad due to the wide
range of individual media to which it devotes itself; the theoretical
orientations are also very wide-ranging. In addition to the expected
focus on individual media and media systems, media studies promotes a
broad definition of media---since its understanding of ``media''
deviates from the term's everyday usage. A sprawling concept of media
that can define objects and practices as media and at the same time
refuses to be delimited by sufficient, verifiable determinants of the
media is a concept of media that is ripe for criticism. However, the
strong historical orientation of media studies in combination with the
critical-analytical approach to history, culture, and society gives it a
fundamental and generalist claim to interpretation in order to provide a
foil for reflection on current and future media and social
developments.\footnote{``Positionen der GfM,'' Gesellschaft für
  Medienwissenschaft (website), accessed September 9, 2021,
  \url{https://gfmedienwissenschaft.de/positionen}.}

The media-scientific thinking advocated by Friedrich Kittler\footnote{Friedrich
  Kittler (1943--2011) was a German literary scholar and media theorist.
  His research focused, among other things, on writing systems and the
  theory and history of the cultural techniques of reading, writing, and
  arithmetic. Kittler\textquotesingle s media theory is characterized by
  a technological-materialist approach.} received great
veneration\footnote{In the 1980s, Kittler\textquotesingle s media theory
  gained increasing popularity among students, who were referred to as
  the ``\emph{Kittler-Jugend}'' {[}Kittler Youth{]} (e.g., Bernhard
  Siegert and Wolfgang Ernst).} and harsh criticism within and outside
the discipline. This school of thought, rejected by some scholars as
technological determinism, developed into cultural techniques studies.
Thinking about the technicality of the media was not, however,
restricted to the history of technology, but rather pursued as a
reconstruction of its conditions and effects in the history of the
humanities and social history, like the philosophy of technology, which
was also widely perceived and connected in media studies.

Rainer Leschke embarked on a critical self-questioning of media studies.
He sees the focus on the media essence, the persistent question of
mediality, as a perhaps-doomed attempt to lay an ontological foundation
for the discipline. Jens Schröter provides a critical reflection on the
emergence and evolution of media studies in Germany, as well as the
limited works of German media theory that have been published in
English. His chapter titled ``Disciplining Media Studies: An Expanding
Field and Its (Self-)Definition''\footnote{Jens Schröter, "Disciplining
  Media Studies: An Expanding Field and Its (Self-)Definition," in
  \emph{Media Transatlantic: Developments in Media and Communication
  Studies between North American and German-Speaking Europe,} ed. Norm
  Friesen (Cham: Springer, 2016).} is framed by Norm Friesen's
perspective in \emph{Media Transatlantic:} \emph{Developments in Media
and Communication Studies between North American and German-Speaking
Europe.}\footnote{Friesen, \emph{Media Transatlantic.}} Friesen insists
that media and communication studies must be re-mediated with regard to
geography, nation, and institutions. In the book's
introductory\footnote{``Finally, and most importantly, this book
  introduces readers to the new field of \emph{Medienwissenschaft} in
  German-speaking Europe---its debates, discourses and modes of
  self-legitimation.'' Friesen, \emph{Media Transatlantic}, 6.} and
comparative approach, Friesen writes entangled histories for media
studies, thus providing proof that Friedrich Kittler, often falsely
thought to be the founder of German media theory (not media studies in
Germany!), was influenced by Marshall McLuhan and Harold Adams Innis.

Jens Schröter is aware of the pitfalls of a desired disciplinary
unity\footnote{Sarah Cordonnier is also aware of these pitfalls, in the
  context of the French SIC. See Cordonnier, ``Looking Back Together,''
  4.} with respect to the dynamics of the field in research, the study
object, or the evolution of media. For the dynamics while creating and
evolving an academic discipline, Schröter refers to general findings
pointing out that disciplinary unity is a desired goal but remains
unattainable and has harmful homogenizing effects as well as
idealizations outside reality. Given the lack of disciplinary unity and
internal diversification, and the ``openness'' of media studies, he
states three crucial points for the disciplinary configuration of media
studies to which I add an easily overlooked fourth point:

\begin{quote}
\emph{Firstly}, there is the epistemic (theoretical, methodological) and
institutional \emph{development} of ``media studies'' as an independent
subject -- and the question of the feasibility or even desirability of
this.

\emph{Secondly}, there is the internal \emph{diversification} of media
studies, and whether this diversification will at some point dissolve
``media studies'' into new sub-disciplines.

\emph{Thirdly}, there is the \emph{changing relationship} between
``media studies'' and its neighbouring disciplines.\footnote{Schröter,
  "Disciplining Media Studies," 32.}
\end{quote}

I have to add:

\begin{quote}
\emph{Fourthly, Internationalisation}.\footnote{Although Schröter quotes
  two sources referencing the international aspect of media studies:
  Hickethier, \emph{Einführung in die Medienwissenschaft}; and Tarmo
  Malmberg, ``Nationalism and Internationalism in Media Studies: Europe
  and America since 1945'' (paper presented at the First European
  Communication Research Conference, Amsterdam, November 25--26, 2005).}
Recent evolutions have produced a changing relationship between media
studies in Germany and its European outlets throughout numerous European
collaborations in which a multifaceted relationship of national
determining factors and international or European necessities are
altering media studies. Furthermore, bi- or trinational study programs
come into play here, as well as translations (or not!) of main sources,
EU-funded research projects, and increasing Europeanisation accompanied
by international conferences and English language international
journals.
\end{quote}

Center-periphery dynamics within the academic discipline of media
studies is another aspect worth mentioning. Taking up Friesen's claim of
geographically located media studies, I note the geographical
differentiation within German-speaking media studies, whereby the
dynamics of the discipline is characterized by a center-periphery
exchange.

The academic association for media studies (\emph{Gesellschaft für
Medienwissenschaft}) has several hundred members,\footnote{After the
  2007 reform, GfM grew from about three hundred fifty members to about
  fifteen hundred members within five years, and it has maintained this
  level for the past ten years.~} but not many of them are based in the
small number of major media studies centers in Germany.\footnote{Major
  media studies centers that are not, it should be noted, based in the
  largest cities in Germany!} They have organized the collaborative
research centers (\emph{Sonderforschungsbereiche}) and research training
groups (\emph{Graduiertenkollegien}) of the German Research
Foundation,\footnote{Indeed, they are important for the development of
  the discipline, as pointed out by Schröter, "Disciplining Media
  Studies," 37--38.} programming concerning the history of media studies
in Germany, and contributed greatly to the discourses of the
self-understanding debate.\footnote{See Rainer Bohn, Eggo Müller, and
  Rainer Ruppert, eds., \emph{Ansichten einer künftigen
  Medienwissenschaft} (Berlin: Sigma, 1988); and Schröter, "Disciplining
  Media Studies."} They have written most of the manuals in this area,
without their taking into account the evolution of media studies outside
of these centers. As a result, much of the existing history of media
studies in Germany isn't represented.

The bulk of teaching (staff and module contents) within the more than
fifty study programs in media studies mentioned above is integrated into
a wide variety of institutional contexts. In some programs, the
disciplinary boundaries of \emph{Kommunikationswissenschaft} and
\emph{Medienwissenschaft} are imploding, whereas in others, media
studies is becoming part of a more encompassing cultural studies
perspective. Media scholars are often the only representatives of this
diverse discipline, which highlights their research areas as being at
the center of media studies.

\hypertarget{conclusion}{%
\section{Conclusion}\label{conclusion}}

According to the German Media Studies Association,\footnote{Gesellschaft
  für Medienwissenschaft, ``Selbstverständnis."} media studies is a
research field that focuses on the study of media, including individual
media and media systems. It examines the interdependence of media
through the lens of apparatuses, content transmission, technology, and
culture. The field aims to understand history, culture, and society by
engaging with the technical, aesthetic, symbolic, and communicative
characteristics of media.

Media studies recognizes that media are fundamentally technical and
culturally conditioned. They shape sensory perception and make
perception possible. The field considers media in a broad sense,
including mass media, individual media, digital and electronic devices,
as well as non-semantically decodable phenomena and cultural techniques.
Media are viewed as entities that condition perception and thinking,
encompassing apparatuses, institutions, functions, and processes of
cultural formation. Media studies is characterized by its focus on the
interactions between aesthetic, normative, and social dynamics in
societies. These factors define its epistemological location and
interest.

The specificity of German media studies, known as
\emph{Medienwissenschaft}, becomes apparent through its
conceptualization of media as encompassing various forms and functions
with mediality as core concept. The field differentiates itself from
other disciplines dedicated to media and communication.

Here I have asserted that media studies (\emph{Medienwissenschaft}) and
communication studies (\emph{Kommunikationswissenschaft}) differ
primarily in their research methods, epistemological interests, and
theoretical perspectives. While international media and communication
studies have seen a dissolution of disciplinary boundaries, particularly
in the Anglo-American context, German-language scholars have actively
established boundaries between media studies and communication studies.
Media studies is often viewed as a ``text and cultural
science''\footnote{Hickethier, \emph{Einführung in die
  Medienwissenschaft}, 6.} with a focus on film, television, radio, and
the internet, emphasizing entertainment and fictional forms. It employs
exemplary, analytical-interpretive, theoretical, and historical
approaches. On the other hand, communication studies primarily addresses
the social reality expressed through mass media and focuses on society.
It employs interpretative methods, aesthetic analysis, and quantitative
and empirical approaches.

The distinction between German-language media studies and communication
studies, as well as journalism (\emph{Publizistik}), is unique to the
German context. The wide range of media conceptualizations in media
studies (\emph{Medienwissenschaft}) is attributed to its epistemological
grounding in speculative thinking, including dialectics,
non-evidence-based assumptions, psychoanalysis, hermeneutics, and
deconstruction. In contrast, communication studies is often associated
with qualitative and quantitative empirical approaches. This
differentiation has resulted in a scientific configuration specific to
the German-language context, not commonly found in other European or
Western countries. However, for European research cooperation in media
studies, it is important to transcend disciplinary boundaries and
embrace multi- and/or transdisciplinarity.

Here I have also considered how far the emergence of media studies in
Germany can be understood to be the outcome of certain social forces.
Firstly, the social upheavals of the 1968 revolution and the subsequent
rejection of the previous generation led to a disruption of continuity,
particularly in universities. This created a space for critical
analysis, including the study of media as a means of understanding the
world and fostering critical thinking. Additionally, the expansion of
universities during the 1960s and 1970s, with a significant increase in
the number of students, including a higher percentage of female
students, led to a need for differentiation and new academic
disciplines.

The development of the media itself also played a role in the emergence
of media studies. The rise of television as the leading medium, along
with the availability of new media technologies such as video, shaped
the first generation of media scholars. These changes in media
consumption and practices influenced the formation of aesthetic analysis
and the need for expert knowledge in understanding film, television, and
other media forms. The belated reception of the Frankfurt
School\textquotesingle s writings on media and the influence of French
post-structuralism and postmodernism further contributed to the
establishment of media studies as an academic discipline in the 1980s
and 1990s.

Overall, the societal reasons for the emergence of media studies in
Germany include the social and generational disruptions of the 1960s,
the expansion of universities and the need for differentiation, the
transformation of media technologies, and the influence of critical
theories and cultural shifts. These factors led to a paradigmatic break
with traditional cultural studies and the formation of media studies as
a distinct field of knowledge and academic discipline.

I have also tried to provide a clear understanding of the connections
between media studies and other disciplines in Germany. Media studies in
Germany is closely related to cultural studies and has developed
parallel to it. It has emerged from literary studies and has connections
with theatre studies, film studies, and television studies. Media
studies is particularly aligned with the history of knowledge and
cultural studies, focusing on the development and implications of media
techniques in generating, storing, representing, and transmitting
perceptions, experiences, and knowledge.

Media studies extends beyond cultural studies and is relevant in other
disciplines such as media sociology, media law, media economics, media
psychology, and media informatics. In German studies, media are often
studied in the context of film history, theory, and aesthetics. Media
studies originated as critical social studies within pedagogy and German
studies, which led to the strong presence of media pedagogy and media
education as areas of knowledge. Visual/electronic media are also
studied within theater studies and art.

Aesthetics is considered an integral part of media studies, and the
discipline encompasses various methods and theories, including
philological-hermeneutic, art-scientific, philosophical, sociological,
and psychological approaches. Media studies in Germany has also
developed as ``media theory'' and gained recognition as German media
theory, contributing to media philosophy and cultural technology
research since the 2000s.

Overall, the institutionalization of media studies in Germany involved
the establishment of research institutes, the formation of dedicated
study programs, the establishment of new scholarly organizations, and
the interdisciplinary exploration of media\textquotesingle s cultural
and historical significance. Media studies in the German context has a
special focus on the history of knowledge and the intersection of
technology and society. This has led to a close connection with Science
and Technology Studies (STS), and with Actor--Network Theory (ANT). ANT
gained prominence in media studies from the mid-2000s to the mid-2010s,
and the concept of mediality became central to the discipline,
emphasizing the autonomy and specific logic of media. Media studies
grapples with the concept of a medium, rejecting the everyday language
understanding that equates it with transmission apparatuses, and instead
focusing on the broader notions of mediality and media potential. The
discipline explores the relationship between individual media/artifacts
and their mediality and seeks to understand knowledge production in
terms of media constituents and legitimacy. Media cultural studies,
rooted in literary and cultural studies, recognizes the constitutional
power of media as a dispositive that operates through various signs and
influences individuals and society.

In \emph{Cultural Studies Research in the German-Speaking World}, there
is a broad understanding of media as encompassing any form of
communication involving symbolic signs, including writing, images, and
numbers. It also recognizes cultural techniques such as reading,
writing, and arithmetic as media configurations. The discipline draws
from historical contexts, including the history of writing, and
incorporates texts from Greek and Roman antiquity. Media studies
considers media as fundamental to human existence, shaping perception,
cognition, and knowledge. It explores media as conditions of expression
with social, cultural, political, individual, psychological, and
pedagogical dimensions. The concept of media encompasses different sign
natures, artistic forms, intermedia manifestations, and specific
configuration levels of formats and genres. Intermediality is another
major paradigm in media studies, providing systematic networking within
cultural studies and serving as an integration point for various
disciplines. Media studies has established itself as an independent
field of knowledge with an open-ended object area and integral
connections to cultural studies.

The cooperation between Germany and France in media and communication
studies faces challenges due to the disciplinary division between
cultural media studies and social communication studies. However, joint
degree programs and research alliances have promoted collaboration. The
coordination of curricula, joint supervision of theses, and advisory
services demonstrate the effects of this cooperation. The three levels
that need to be considered in Franco-German cooperation concern: 1)
institutions, 2) knowledge acquisition and science policy, and 3)
content-related research topics and theories. The sociogenesis of the
disciplines, intellectual biographies, institutional and national
factors, and cooperation among scientific organizations play a role in
shaping media studies. The disciplines of \emph{Sciences de
l\textquotesingle information et de la communication} (SIC) and
\emph{Medienwissenschaft} have distinct characteristics and theoretical
resources. The history of knowledge, including language transfer and
barriers, influences the formation of media theory.
Institutionalization, subject area, content, and structure are important
considerations for Franco-German cooperation in media studies. The
international dimension is necessary for a meaningful history of media
studies, requiring understanding and integration of different approaches
and contexts.

Finally, in this article I described the ``new materialist turn'' in
academic disciplines, particularly in the social sciences and humanities
within the European Research Area. The materialist turn emphasizes the
study of objects, apparatuses, and artifacts as they trace the history
of knowledge and ideas. The focus on objects and boundary objects,
especially in Science \& Technology Studies and Actor--Network Theory,
led to the emergence of network studies and interdisciplinary research.
This approach gained importance in European studies, particularly in
analyzing the impact of information and communication technologies on
pan-European communication. Internationalization is a crucial aspect of
this project, with European collaborations, bi- or tri-national study
programs, and increasing Europeanization shaping the field.




\section{Bibliography}\label{bibliography}

\begin{hangparas}{.25in}{1} 



Abend, Pablo. \emph{Geobrowsing: Google Earth und Co.; Nutzungspraktiken
einer digitalen Erde}. Bielefeld: Transcript, 2013.

Albersmeier, Franz-Josef. \emph{Die Herausforderung des Films an die
französische Literatur:} \emph{Entwurf einer ``Literaturgeschichte des
Films.''} Heidelberg: C. Winter, 1985.

Averbeck-Lietz, Stefanie. ``Comparative History of Communication
Studies: France and Germany.'' \emph{The Open Communication Journal} 2,
no. 1 (2008): 1--13.

Averbeck-Lietz, Stefanie, Fabien Bonnet, Sarah Cordonnier, and Carsten
Wilhelm. ``Communication Studies in France: Looking for a `Terre du
milieu'?,'' \emph{Publizistik: Kommunikationswissenschaft International}
64 (2019): 363--80. \url{https://doi.org/10.1007/s11616-019-00504-3}.

Bachmann-Medick, Doris. \emph{Cultural Turns: New Orientations in the
Study of Culture.} Berlin: de Gruyter, 2016.
\url{https://doi.org/10.1515/9783110402988}.

Bergermann, Ulrike. \emph{Leere Fächer: Gründungsdiskurse in Kybernetik
und Medienwissenschaft.} Münster: LIT Verlag, 2015.

Bohn, Rainer, Eggo Müller, and Rainer Ruppert, eds. \emph{Ansichten
einer künftigen Medienwissenschaft}. Berlin: Sigma, 1988.

Buschauer, Regine, and Katherine Willis, eds. \emph{Locative Media:}
\emph{Medialität und Räumlichkeit.} Bielefeld: Transcript, 2013.

Cordonnier, Sarah. ``Looking Back Together to Become `Contemporaries in
Discipline.'\,'' \emph{History of Media Studies} 1 (2021).
\url{https://doi.org/10.32376/d895a0ea.b8153251}

Cordonnier, Sarah, and Hedwig Wagner. ``Déployer l'interculturalité: les
étudiants, un vecteur pour la réflexion académique sur l'interculturel;
Le cas des sciences consacrées à la communication en France et en
Allemagne.'' In \emph{Interkulturelle Kompetenz in deutsch-französischen
Studiengängen: Les compétences interculturelles dans les cursus
franco-allemands allemands} {[}Intercultural competency in German-French
degree programs: Key competencies for higher education and
employability{]}, edited by Gundula Gwenn Hiller, Hans-Jürgen Lüsebrink,
Patricia Oster-Stierle, and Christoph Vatter, 221--34. Wiesbaden:
Springer VS, 2017.

Cordonnier, Sarah, and Hedwig Wagner. ``La discipline au prisme des
activités internationales dans les trajectoires de chercheurs en France
et en Allemagne (encadré).'' In \emph{Hermès 67: Discipline,
interdisciplinarité, indiscipline}, edited by Jean-Michel Besnier and
Jacques Perriault, 137--40. Paris: CNRS Éditions, 2013.

Cordonnier, Sarah, and Hedwig Wagner. ``L'interculturalité académique
entre cadrages et interstices.'' In \emph{France-Allemagne:
Incommunications et convergences,} edited by Gilles Rouet and Michaël
Oustinoff, 169--82. Paris: CNRS Éditions, 2018.

Donsbach, Wolfgang. ``The Identity of Communication Research.''
\emph{Journal of Communication} 56, no. 3 (2006): 437.

Döring, Jörg. ``Zur Geschichte der Literaturkarte 1907\emph{--}2008.''
In \emph{Mediengeographie: Theorie---Analyse---Diskussion}, edited by
Jörg Döring and Tristan Thielmann, 247--90. Bielefeld: Transcript, 2009.

Döring, Jörg, and Tristan Thielmann, eds. \emph{Mediengeographie:
Theorie---Analyse---Diskussionen}. Bielefeld: Transcript, 2009.

Döring, Jörg, and Tristan Thielmann, eds. \emph{Spatial Turn: Das
Raumparadigma in den Kultur- und Sozialwissenschaften}. Bielefeld:
Transcript, 2008.

Dünne, Jörg, and Stephan Günzel, eds. \emph{Raumtheorie: Grundlagentexte
aus Philosophie und Kulturwissenschaften}. Frankfurt: Suhrkamp, 2006.

Engell, Lorenz. ``Tasten, Wählen, Denken: Genese und Funktion einer
philosophischen Apparatur.'' In \emph{Medienphilosophie: Beiträge zur
Klärung eines Begriffs,} edited by Stefan Münker, Alexander Roesler, and
Mike Sandbothe, 53\emph{--}77. Frankfurt: Fischer, 2003.

Engell, Lorenz, and Bernhard Siegert. ``Editorial: Medienphilosophie.''
\emph{Zeitschrift für Medien- und Kulturforschung} 2 (2010): 6--11.

Engell, Lorenz, and Joseph Vogl. ``Zur Einführung.'' In \emph{Kursbuch
Medienkultur: Die maßgeblichen Theorien von Brecht bis Baudrillard,}
edited by Claus Pias, Joseph Vogl, Lorenz Engell, Oliver Fahle, and
Britta Neitzel, 8\emph{--}11. Stuttgart: Deutsche Verlagsanstalt, 1999.

Faulstich, Werner. \emph{Medientheorien: Einführung und Überblick}.
Göttingen: Vandenhoeck \& Ruprecht, 1991.

Filk, Christian. \emph{Episteme der Medienwissenschaft:
Systemtheoretische Studien zur Wissenschaftsforschung eines
transdisziplinären Feldes}. Bielefeld: Transcript, 2015.

Friesen, Norm, ed. \emph{Media Transatlantic: Developments in Media and
Communication Studies Between North American and German-Speaking
Europe}. Cham: Springer, 2016.

Gesellschaft für Medienwissenschaft. ``Selbstverständnis,
Forschungsfragen, Wissenschaftspolitik'' {[}Mission statement, research
questions, science policy{]} GfM (website). Updated February 27, 2018.
\url{https://gfmedienwissenschaft.de/gesellschaft}.

Gumbrecht, Hans-Ulrich, and Karl Ludwig Pfeiffer. \emph{Materialität der
Kommunikation}. Frankfurt: Suhrkamp, 1988.

Günzel, Stephan. \emph{Handbuch Raum.} Stuttgart: Metzler, 2009.

Günzel, Stephan. \emph{Topologie:} \emph{Zur Raumbeschreibung in den
Kultur- und Medienwissenschaften.} Bielefeld: Transcript, 2007.

Hagen, Wolfgang. ``Zellular---Parasozial---Ordal: Skizzen zu einer
Medienarchäologie des Handys.'' In \emph{Mediengeographie:
Theorie---Analyse---Diskussionen}, edited by Jörg Döring and Tristan
Thielmann, 359--82. Bielefeld: Transcript, 2009.

Helmes, Günter, and Werner Köster, eds. \emph{Texte zur Medientheorie.}
Stuttgart: Reclam, 2002.

Helmholtz-Zentrum für Kulturtechnik. ``Founding Project of the Helmholtz
Centre for Cultural Technology: Theory and History of Cultural
Techniques.'' Humboldt University of Berlin (website). Accessed December
13, 2021.
\url{https://www.kulturtechnik.hu-berlin.de/bild-schrift-zahl/}.

Hickethier, Knut. \emph{Einführung in die Medienwissenschaft.}
Stuttgart: Metzler, 2003.

Kittler, Friedrich A. \emph{Austreibung des Geistes aus den
Geisteswissenschaften: Programme des Poststrukturalismus}. Paderborn:
Fink, 1980.

Kloock, Daniela, and Angela Spahr. \emph{Medientheorien: Eine
Einführung.} Paderborn: Fink, 1997.

Kneer, Gerd, Marcus Schroer, and Erhard Schüttpelz, eds. \emph{Bruno
Latours Kollektive: Kontroversen zur Entgrenzung des Sozialen}.
Frankfurt: Suhrkamp, 2008.

Knilli, Friedrich. \emph{Das Hörspiel in der Vorstellung der Hörer}:
\emph{Eine experimental-psychologische Untersuchung.} PhD diss.,
University of Graz, 1959.

Knilli, Friedrich. ``Wie aus den Medien eine Wissenschaft wurde: Exposé
für eine soziobiographische Fachgeschichte.'' \emph{Medienwissenschaft:
Rezensionen/Reviews} 20, no. 1 (2003).
\url{https://doi.org/10.17192/ep2003.1.2101}.

Knilli, Friedrich, Knut Hickethier, and Wolf-Dieter Lützen, eds.
\emph{Literatur in den Massenmedien: Demontage von Dichtung?} München:
Hanser, 1976.

Knilli, Friedrich, and Siegfried Zielinski. \emph{Holocaust zur
Unterhaltung}. Berlin: Elefanten Press, 1982.

Krämer, Sybille. \emph{Technik als Kulturtechnik: Kleines Plädoyer für
eine kulturanthropologische Erweiterung des Technikkonzeptes}. Ulm:
Universität Ulm, 2004.

Krämer, Sybille, and Horst Bredekamp, eds. \emph{Bild, Schrift, Zahl}.
Paderborn: Fink, 2003.

Latour, Bruno. ``Die Logistik der \emph{immutable mobiles}.'' In
\emph{Mediengeographie: Theorie---Analyse---Diskussionen}, edited by
Jörg Döring and Tristan Thielmann, 111\emph{--}44. Bielefeld:
Transcript, 2009.

Leschke, Rainer. ``Medienwissenschaften und Geschichte: Morphologie
einer Wissenschaft.'' Accessed June 30, 2023. \href{https://docplayer.org/storage/97/131650175/1688120570/R4jNja6ZxG-BoJ3esevqww/131650175.pdf}{https://docplayer} \href{https://docplayer.org/storage/97/131650175/1688120570/R4jNja6ZxG-BoJ3esevqww/131650175.pdf}{.org/storage/97/131650175/1688120570/R4jNja6ZxG-BoJ3esevqww/} \href{https://docplayer.org/storage/97/131650175/1688120570/R4jNja6ZxG-BoJ3esevqww/131650175.pdf}{131650175.pdf}
(link no longer accessible).

Malmberg, Tarmo. ``Nationalism and Internationalism in Media Studies:
Europe and America since 1945.'' Paper presented at the First European
Communication Research Conference, Amsterdam, November 25--26, 2005.
Accessed February 10, 2022.
\href{https://genderi.org/nationalism-and-internationalism-in-media-studies--europe-and.html}{https://genderi.org/nationalism-and-internationalism-in-media-} \href{https://genderi.org/nationalism-and-internationalism-in-media-studies--europe-and.html}{studies--europe-and.html}.

Mersmann\emph{,} Birgit, and Thomas Weber, eds. \emph{Mediologie als
Methode.} Berlin: Avinus, 2008.

Muhle, Maria. ``Medienwissenschaft als theoretisch-politisches Milieu.''
\emph{Zeitschrift für Medienwissenschaft} 10, no. 1 (2014):
137\emph{--}42.

Münker, Stephan, and Alexander Rösler, eds. \emph{Was ist ein Medium?}
Frankfurt: Fischer, 2008.

Paech, Joachim. ``Die Erfindung der Medienwissenschaft: Ein
Erfahrungsbericht aus den 1970er Jahren.'' In \emph{Was waren Medien?},
edited by Claus Pias, 31\emph{--}55. Zürich: Diaphanes, 2011.

Paech, Joachim. ``Warum `Medien'?'' Lecture, University of Konstanz,
Konstanz, Germany, 2006. Accessed February 15, 2022.
\url{http://medientheorie.com/aktuell/WasWarenMedien/03_Paech.mp3}.

Pias, Claus, ed. \emph{Cybernetics: The Macy Conferences 1946--1953; The
Complete Transactions}. Chicago: Macy Conferences, 2016.

Pias, Claus. ``Was waren Medien-Wissenschaften? Stichworte zu einer
Standortbestimmung.'' In \emph{Was waren Medien?}, edited by Claus Pias,
7\emph{--}30. Zürich: Diaphanes, 2011.

Ruf, Oliver, Patrick Rupert-Kruse, and Lars C. Grabbe.
\emph{Medienkulturwissenschaft}. Wiesbaden: Springer, 2022.

Schanze, Helmut. \emph{Medienkunde für Literaturwissenschaftler:
Einführung und Bibliographie}. München: Fink, 1974.

Schaudig, Michael. \emph{Literatur im Medienwechsel: Gerhart Hauptmanns
Tragikomödie ``Die Ratten'' und ihre Adaptionen für Kino, Hörfunk,
Fernsehen; Prolegomena zu einer Medienkomparatistik}. München:
Verlegergemeinschaft Schaudig, 1992.

Schmid, Ulrich. \emph{Russische Medientheorien}. Bern: Haupt Verlag,
2005.

Schröter, Jens. ``Das transplane Bild und der \emph{spatial turn.''} In
\emph{Mediengeographie: Theorie---Analyse---Diskussionen,} edited by
Jörg Döring and Tristan Thielmann, 167\emph{--}78. Bielefeld:
Transcript, 2009.

Schröter, Jens. "Disciplining Media Studies: An Expanding Field and Its
(Self-) Definition." In \emph{Media Transatlantic: Developments in Media
and Communication Studies between North American and German-Speaking
Europe,} edited by Norm Friesen, 29--48. Cham: Springer, 2016.

Schröter, Jens. ``Einleitung'' {[}Introduction{]}. In \emph{Handbuch
Medienwissenschaft}, edited by Jens Schröter, 1\emph{--}11. Stuttgart:
Metzler, 2014.

Schüttpelz, Erhard. ``Die medientechnische Überlegenheit des Westens:
Zur Geschichte und Geographie der \emph{immutable mobiles} Bruno
Latours.'' In \emph{Mediengeographie: Theorie---Analyse---Diskussionen,}
edited by Jörg Döring and Tristan Thielmann, 67\emph{--}110. Bielefeld:
Transcript, 2009.

Seier, Andrea. \emph{Remediatisierung: Die performative Konstitution von
Gender und Medien.} Münster: LIT Verlag, 2015.

Schreiterer, Ulrich. ``Hochschulen im Wettbewerb: mehr Markt, mehr
Freiheit, mehr Unübersichtlichkeit'' {[}Universities in competition:
more market, more freedom, more confusion{]}. Bundeszentrale für
politische Bildung (blog), June 6, 2014. Accessed February 25, 2023.
\href{https://www.bpb.de/themen/bildung/dossier-bildung/185865/hochschulen-im-wettbewerb-mehr-markt-mehr-freiheit-mehr-unuebersichtlichkeit/}{https://www.bpb.de/themen/bildung/dossier-bildung/} \href{https://www.bpb.de/themen/bildung/dossier-bildung/185865/hochschulen-im-wettbewerb-mehr-markt-mehr-freiheit-mehr-unuebersichtlichkeit/}{185865/hochschulen-im-wettbewerb-mehr-markt-mehr-freiheit-mehr-unuebersichtlichkeit/}.

Siegert, Bernhard. \emph{Cultural Techniques: Grids, Filters, Doors, and
Other Articulations of the Real}. New York: Fordham University Press,
2015.

Strategiekommission der GfM. ``Kernbereiche der Medienwissenschaft:
Beschluss der Mitgliederversammlung der GfM'' {[}Core areas of Media
Studies: Resolution of the General Assembly of the GfM{]}.
Mitgliederversammlung der GfM {[}GfM General Assembly{]}, Bochum, April
10, 2008. Accessed September 9, 2021.
\url{https://www.online.uni-marburg.de/blog-mediarep/blog/wp-content/uploads/2020/02/2008-GfM-Strategiepapier.pdf}.

Thielmann, Tristan, and Erhard Schüttpelz, eds.
\emph{Akteur---Medien---Theorie.} Bielefeld: Transcript, 2013.

Wagner, Hedwig, ed. \emph{Europäische Medienwissenschaft: Zur
Programmatik eines Fachs.} Bielefeld: Transcript, 2020.

Weigel, Sigrid. ``Zum `topographical turn': Kartographie, Topographie
und Raumkonzepte in den Kulturwissenschaften.'' \emph{KulturPoetik}
(2002): 151\emph{--}65.

Zielinski, Siegfried\emph{. Video: Apparat/Medium, Kunst, Kultur; Ein
internationaler Reader}. Frankfurt: Peter Lang, 1992.

Zielinski, Siegfried, and Kurt Luger, eds. \emph{Europäische
Audiovisionen: Film und Fernsehen im Umbruch}. Neue Aspekte in Kultur-
und Kommunikationswissenschaft, vol. 7\emph{.} Wien: Österreichischer
Kunst- und Kulturverlag, 1993.



\end{hangparas}


\end{document}
