% see the original template for more detail about bibliography, tables, etc: https://www.overleaf.com/latex/templates/handout-design-inspired-by-edward-tufte/dtsbhhkvghzz

\documentclass{tufte-handout}

%\geometry{showframe}% for debugging purposes -- displays the margins

\usepackage{amsmath}

\usepackage{hyperref}

\usepackage{fancyhdr}

\usepackage{hanging}

\hypersetup{colorlinks=true,allcolors=[RGB]{97,15,11}}

\fancyfoot[L]{\emph{History of Media Studies}, vol. 3, 2023}


% Set up the images/graphics package
\usepackage{graphicx}
\setkeys{Gin}{width=\linewidth,totalheight=\textheight,keepaspectratio}
\graphicspath{{graphics/}}

\title[Review of Brylowe \& Yeager]{\emph{Old Media and the Medieval Concept}, reviewed by Michele Kennerly} % longtitle shouldn't be necessary

% The following package makes prettier tables.  We're all about the bling!
\usepackage{booktabs}

% The units package provides nice, non-stacked fractions and better spacing
% for units.
\usepackage{units}

% The fancyvrb package lets us customize the formatting of verbatim
% environments.  We use a slightly smaller font.
\usepackage{fancyvrb}
\fvset{fontsize=\normalsize}

% Small sections of multiple columns
\usepackage{multicol}

% Provides paragraphs of dummy text
\usepackage{lipsum}

% These commands are used to pretty-print LaTeX commands
\newcommand{\doccmd}[1]{\texttt{\textbackslash#1}}% command name -- adds backslash automatically
\newcommand{\docopt}[1]{\ensuremath{\langle}\textrm{\textit{#1}}\ensuremath{\rangle}}% optional command argument
\newcommand{\docarg}[1]{\textrm{\textit{#1}}}% (required) command argument
\newenvironment{docspec}{\begin{quote}\noindent}{\end{quote}}% command specification environment
\newcommand{\docenv}[1]{\textsf{#1}}% environment name
\newcommand{\docpkg}[1]{\texttt{#1}}% package name
\newcommand{\doccls}[1]{\texttt{#1}}% document class name
\newcommand{\docclsopt}[1]{\texttt{#1}}% document class option name


\begin{document}

\begin{titlepage}

\begin{fullwidth}
\noindent\LARGE\emph{Book review
} \hspace{88mm}\includegraphics[height=1cm]{logo3.png}\\
\noindent\hrulefill\\
\vspace*{1em}
\noindent{\Huge{\emph{Old Media and the Medieval Concept}\newline\noindent reviewed by Michele Kennerly\par}}

\vspace*{1.5em}

\noindent\LARGE{Michele Kennerly} \href{https://orcid.org/0000-0001-8674-338X}{\includegraphics[height=0.5cm]{orcid.png}}\par\marginnote{\emph{Michele Kennerly, ``\emph{Old Media and the Medieval Concept}, reviewed by Michele Kennerly,'' \emph{History of Media Studies} 3 (2023), \href{https://doi.org/10.32376/d895a0ea.b703fec2}{https://doi.org/ 10.32376/d895a0ea.b703fec2}.} \vspace*{0.75em}}
\vspace*{0.5em}
\noindent{{\large\emph{Pennsylvania State University}, \href{mailto:mjk46@psu.edu}{mjk46@psu.edu}\par}} \marginnote{\href{https://creativecommons.org/licenses/by-nc/4.0/}{\includegraphics[height=0.5cm]{by-nc.png}}}



\end{fullwidth}

\vspace*{1em}


    
\noindent\small{Thora Brylowe and Stephen Yeager, eds. \emph{Old Media and the Medieval
Concept: Media Ecologies before Early Modernity}. 280 pp., figs., index.
Montréal: Concordia University Press, 2021.
\href{https://press.library.concordia.ca/projects/old-media-and-the-medieval-concept}{Open access download}.}

\vspace*{0.25em}

\newthought{\emph{Antiquus}. \emph{Medius}. \emph{Modernus}}. ``Before.'' ``Middle.''
``Just now.'' Those three Latin adjectives underwrite the stubborn
periodization schema by which the European past and studies of it tend
to be sorted. They're more familiar, of course, as ``the ancient,''
``the medieval,'' and ``the modern.''

Whereas \emph{antiquus} and \emph{modernus} were used as generational
descriptors in ancient Roman rhetorical theory, ``medieval,'' a
concatenation of \emph{media æva} (middle ages), comes to us courtesy of
Francis Bacon. In his 1620 \emph{Novum Organon} (both an homage and a
challenge to the old one of Aristotle), Bacon used \emph{media æva} to
mark the ostensibly unremarkable cultural and chronological space
between what, for him, were two self-evidently remarkable ones:
(Athenian and Roman) antiquity and his own. We might not all be
connected to Francis Bacon by six degrees socially,\footnote{"Francis
  Bacon Network {[}2, 1562-1626, 61-100\%{]}," \emph{Six Degrees of
  Francis Bacon}, accessed March 1, 2023,
  \url{http://www.sixdegreesoffrancisbacon.com/?ids=10000473\&min_confidence=60\&type=network}.}
but we are intellectually: His impatient attitude authorized an enduring
dismissiveness about the Middle Ages. It is periodization's flyover
country.

Evidence abounds for the Middle Ages being snubbed particularly
egregiously in major studies of media history. Following the cosmic
logic of Marshall McLuhan's \emph{The} \emph{Gutenberg
Galaxy},\footnote{Marshall McLuhan, \emph{The Gutenberg Galaxy: The
  Making of Typographic Man} (Toronto: University of Toronto Press,
  1962).} the Middle Ages were but gas and dust. Following the framework
of Friedrich Kittler's \emph{Discourse Networks 1800/1900},\footnote{Friedrich
  Kittler, \emph{Discourse Networks, 1800/1900} (Stanford, CA: Stanford
  University Press, 1990).} the Middle Ages are far from the century
across which the mediality of the literary was doing its most
transformative cultural work. Following the genealogy of John Guillory's
``The Genesis of the Media Concept,''\footnote{John Guillory, ``Genesis
  of the Media Concept,'' \emph{Critical Inquiry} 36, no. 2 (2010).} the
Middle Ages merit a single footnote. All the while, medievalists were
some of the earliest adopters of digital scholarly methods and
platforms, and some of those spotlighted medieval media.

Those two slights of the Middle Ages---one born in the seventeenth
century, the other in the twentieth---motivate \emph{Old Media and the Medieval Concept},

\enlargethispage{2\baselineskip}

\vspace*{2em}

\noindent{\emph{History of Media Studies}, vol. 3, 2023}


 \end{titlepage}


\noindent the inaugural volume in a series titled ``Media
Before 1800'' (notice the allusion to Kittler), in whose campaign I have
enlisted readers of this review already, using the rallying cries of the
co-editors, romanticist Thora Brylowe and medievalist Stephen M. Yeager.

Medievalists have been writing sumptuously about medieval media for some
time; what the medievalists in this volume do is explicitly engage the
terms and assumptions of twentieth- and twenty-first-century media
history and theory, and invite media theorists and historians, in turn,
not to pass over the Middle Ages. The frontispiece, which peeks through
a cut-out on the cover, is a medieval consanguinity diagram, and,
dodging gross bloodline-talk, one might call it emblematic of the
congruities between medieval studies and media studies that emerge when
one translates Bacon's \emph{media æva} as ``the mediating ages,'' as
the editors provocatively do.

The volume assumes a symmetrical structure of two parts of three
chapters each. The first part, ``Long Durations,'' contests the logic of
periodization, which might permit analogies between and among periods
but denies continuities. Brandon W. Hawk's chapter on ``The Genesis of
the Digital Concept'' leads. ``Digital'' exploded onto the cultural
scene about the same time as ``information'' (though neither appears
among Raymond Williams's keywords), two members of the vocabulary of
automated computation. Hawk deploys philological methods to track
\emph{digit}-based words across medieval Latin and English texts,
demonstrating its use in mathematical contexts that could be called
computational. Key to his chapter is medieval ``finger reckoning,''
giving numerical purpose to fingers and hands to make abstract
mathematical concepts more concrete. As he hits the end of his
genealogy, I think Hawk under-reaches when he limits the computational
use of digits (fingers) in our digital (not analogue) age to ``typing
and clicking with our fingers'' on our gadgets (35, 50). The
word-history he surfaces becomes even more compelling when one thinks of
the hands that make the gadgets themselves, from the start of the
digital age. For instance, Lisa Nakamura's work\footnote{Lisa Nakamura,
  ``Indigenous Circuits,'' \emph{Computer History Museum} (blog),
  January 2, 2014,
  \url{https://computerhistory.org/blog/indigenous-circuits/}.} highlights the
Indigenous women hired by Fairchild Semiconductor to make circuits due
to their ``nimble fingers.'' There are also all the fingers of Google
Books scanner-laborers that the company took pains to remove,\footnote{Asher
  Moses, ``Book Scans Reveal Google's Handiwork,'' \emph{The Sydney
  Morning Herald}, December 7, 2007,
  \href{https://www.smh.com.au/technology/book-scans-reveal-googles-handiwork-20071207-gdrrhl.html}{https://www.smh.com.au/technology/book-scans-reveal-googles-handiwork-20071207-gdrrhl.html}.}
thereby ensuring invisible laborers\footnote{Marlena Millikin, ``The
  Ghostly Disruption of Google Hands,'' \emph{The Future of the Book}
  (blog), March 11, 2016,
  \href{http://metadefect.com/2016/03/the-ghostly-disruption-of-google-hands/}{http://metadefect.com/2016/03/the-ghostly-disruption-of-google-hands/}.}
stayed that way.

Co-editor Stephen M. Yeager argues in his chapter, ``Protocol and
Regulation: Controlling Media Histories,'' that the control concepts of
protocol and regulation map on generatively to prevailing attitudes
toward the dominant media forms of pre-modern, modern, and post-modern
historical periods, which themselves shape larger views about the
character of those periods. Yeager's media examples are manuscripts,
printed books, television, and the internet. He pairs manuscripts and
the internet with the protocological (``distributed, non-hierarchical'')
and printed books and television with the regulatory (``centralized,
hierarchical''), thereby bracketing the tell-tale media of the modern
period (77). Yeager's groupings violate the rules of periodization,
which separate eras according to purported dissimilarities. I am excited
to see how this grand theoretical vision plays with Jeff Jarvis's
forthcoming book \emph{The Gutenberg Parenthesis}\footnote{Jeff Jarvis,
  \emph{The Gutenberg Parenthesis: The Age of Print and Its Lessons for
  the Age of the Internet} (New York: Bloomsbury, 2023).} (its snazzy
title has earlier origins\footnote{Greg Peverill-Conti and Brad Seawell,
  ``The Gutenberg Parenthesis: Oral Tradition and Digital
  Technologies,'' \emph{CommForum} (blog), April 1, 2010,
  \href{https://commforum.mit.edu/the-gutenberg-parenthesis-oral-tradition-and-digital-technologies-29e1a4fde271}{https://commforum.mit.edu/the-gutenberg-parenthesis-oral-tradition-and-digital-technologies-29e1a4fde271}.}).

``The Coconut Cup as Material and Medium: Extended Ecologies,'' by
Kathleen E. Kennedy, whose fascinating coconut work\footnote{Kathleen E.
  Kennedy, ``Coconuts in Medieval England Weren't as Rare as \emph{Monty
  Python and the Holy Grail} Made You Think,'' \emph{The Mary Sue}
  (website), October 14, 2015,
  \url{https://www.themarysue.com/monty-python-holy-grail-coconuts/}.} has
been rolling out for a few years, emerges as the most global and
ecological chapter in the volume. Though coconuts had long been used as
drinking vessels, their medieval importation into Europe from India
resulted in a new media form: coconut drinkware with silver stems (and
feet\footnote{Coconut cup and cover, Ashmolean Museum Oxford, accessed
  October 15, 2022, \url{https://collections.ashmolean.org/object/777202}.}).
She traces how the coconut cup competed with the maple-wood cup in
Europe and became an artifact of European colonization, whereby it then
competed with---and, crucially, failed to displace---the organic
drinkware of the Americas, especially that used for yerba and chocolate.
One of many noteworthy features of this chapter is Kennedy's choice,
contoured to this volume, of course, to treat the coconut cup as a
medium instead of a thing, actant, or object. For Kennedy, the coconut
cup ``transmit{[}ted{]} . . . cultural information'' (101). In that
light, James Carey's work on communication as culture would have rounded
out this chapter nicely (and been a better fit for what Kennedy is doing
than John Durham Peters's book \emph{Speaking into the Air}, which she
finds problematic). Her choice seems more deliberate than that, though:
To consider household items, which cultures often assign to be the
province of women, as media or technologies---words with much more
cachet outside of most theory circles than ``things'' and
``objects''---is to give them a status they don't often enjoy.

The three chapters in the second part of the volume, ``Affective
Affordances,'' exhibit the counterpart to \emph{longue durée}--based
theorizing: intensive interpretation of singular media forms,
respectively. The contributors select tags and glosses, letters, and
commentaries, all of which take their affective charge from their
institutional contexts (universities, cathedral schools, the church).

The first chapter, Fiona Somerset's ``Multimedia Verse,'' and the last
chapter, Alice Hutton Sharp's ``The \emph{Gloss} on Genesis and
Authority in the Cathedral Schools,'' highlight the communal and
material practices of what one might call accumulating interpretive
authority by accretion. None of the historical actors, all students who
have been anonymous for several hundred years, is trying to have either
the first or the last word. Instead, they are using pre-existing verses
and texts to enter into larger cultural conversations. Somerset centers
on deposits of verse-tag annotations, calling them a ``technology of
association'' (107) that arose from student-readers of an early
fourteenth-century copy of Gratian's \emph{Decretum} actively
interacting with the manuscript. For her part, Hutton Sharp treats
cathedral schools as interpretive communities whose students
collaborated to layer onto existing understandings of biblical passages
and thus reflect (and genuflect to) the learning of their teachers. The
chapter's media focus falls on how manuscript design shaped and was
shaped by commentary practices. That authorship cannot be attributed to
either verse tags or student glosses reminded me, at a slant, of Ann
Blair's work-in-progress\footnote{Ann Blair, ``Hidden Hands: Amanuenses
  and Authorship in Early Modern Europe,'' A. S. W. Rosenbach Lectures
  in Bibliography, Penn Libraries Repository, March 17, 18, 20, 2014,
  \url{https://repository.upenn.edu/rosenbach/8/}.} on the anonymous
amanuenses of big textual projects in the modern period. Their anonymity
made ``\emph{the} author'' possible.

In the penultimate chapter, ``Ex Illo Tempore: Time, Mediation, and the
\emph{Ars Dictaminis} in Letter 65 by Peter the Venerable,'' Jonathan M.
Newman treats the medieval letter as an object lesson in understanding
media. Because the letter is both a medium (an in-between material) and
a form (a shape emerging from a topical and structural conventions),
when letter writers reflect on absence, distance, permanence, speed,
attention, and affection, they're frequently issuing metacommentary on
the nature of media. Letter 65, sent from one twelfth-century abbot to
another, arose from institutional and interpersonal rivalries. \emph{Ars
dictaminis} (the medieval technique system for letter-writing) taught
its learners arts of address-over-distance that were exquisitely attuned
to social hierarchy. When writer and addressee are of equal status but
agonistically disposed, every dictaminal convention seems to bend toward
creative or even mischievous use. In his letter to Bernard of Clairvaux,
Peter uses affective commonplaces of Christian letter-writing to lead
Bernard to the ask: If you care for me, as one committed Christian
should for another, stop merely sending a quick hello through
intermediaries and instead write. Peter seems to want his relationship
with his rival on the record.

Somerset, Hutton Sharp, and Newman do not define or use the concept of
affordance in their pieces, nor do they push too hard when they compare
their units of medieval media analysis to digital ones (e.g., comment
threads and emails). My interpretation is that their chapters act as
action possibilities for readers, especially ones writing theories and
histories of purportedly digital age phenomena.

Altogether, \emph{Old Media and the Medieval Concept} boasts a
gregarious character. It seems eager to be heard and to chat. As this
review attests, while reading, I heard it talking with forthcoming
books; I'll add here Bruce Holsinger's \emph{On Parchment: Animals,
Archives, and the Making of Culture from Herodotus to the Digital
Age}\footnote{Bruce Holsinger, O\emph{n Parchment: Animals, Archives,
  and the Making of Culture from Herodotus to the Digital Age} (New
  Haven, CT: Yale University Press, 2023).} and Anna Shechtman's
book-in-progress on ``the media concept.''\footnote{Anna Shechtman,
  Personal website, accessed October 15, 2022,
  \url{https://www.annashechtman.com/}.} My ideas for reading groups and
graduate seminar units are scribbled throughout the margins of my copy
and on obliging scraps of paper around my house.

In their introduction, co-editors Brylowe and Yeager claim that ``a
thoroughgoing rethinking of media history will show the continuities,
reversals, and overlaps are more compelling than revolutions or epochs,
as they help us both to better understand what the artifacts of the past
have to teach us and to better respond to the contingencies of our own
historical moment'' (22). The six contributors took up the call. I hope
those of us writing and teaching histories of media, communication,
information, and rhetoric will, too.




\section{Bibliography}\label{bibliography}

\begin{hangparas}{.25in}{1} 



Blair, Ann. ``Hidden Hands: Amanuenses and Authorship in Early Modern
Europe.'' A. S. W. Rosenbach Lectures in Bibliography. Penn Libraries
Repository. March 17, 18, 20, 2014.
\url{https://repository.upenn.edu/rosenbach/8/}.

Guillory, John. ``Genesis of the Media Concept.'' \emph{Critical
Inquiry} 36, no. 2 (2010): 321--62.

Holsinger, Bruce. O\emph{n Parchment: Animals, Archives, and the Making
of Culture from Herodotus to the Digital Age}. New Haven, CT: Yale
University Press, 2023.

Jarvis, Jeff. \emph{The Gutenberg Parenthesis: The Age of Print and Its
Lessons for the Age of the Internet}. New York: Bloomsbury, 2023.

Kittler, Friedrich. \emph{Discourse Networks, 1800/1900}. Stanford, CA:
Stanford University Press, 1990.

McLuhan, Marshall. \emph{The Gutenberg Galaxy: The Making of Typographic
Man}. Toronto: University of Toronto Press, 1962.



\end{hangparas}


\end{document}