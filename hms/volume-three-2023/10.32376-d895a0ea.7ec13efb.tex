% see the original template for more detail about bibliography, tables, etc: https://www.overleaf.com/latex/templates/handout-design-inspired-by-edward-tufte/dtsbhhkvghzz

\documentclass{tufte-handout}

%\geometry{showframe}% for debugging purposes -- displays the margins

\usepackage{amsmath}

\usepackage{hyperref}

\usepackage{fancyhdr}

\usepackage{hanging}

\hypersetup{colorlinks=true,allcolors=[RGB]{97,15,11}}

\fancyfoot[L]{\emph{History of Media Studies}, vol. 3, 2023}


% Set up the images/graphics package
\usepackage{graphicx}
\setkeys{Gin}{width=\linewidth,totalheight=\textheight,keepaspectratio}
\graphicspath{{graphics/}}

\title[How German Communication Research Discovered Bourdieu]{How German Communication Research Discovered Bourdieu but Missed His Potential for the Study of (Populist) Political Communication} % longtitle shouldn't be necessary

% The following package makes prettier tables.  We're all about the bling!
\usepackage{booktabs}

% The units package provides nice, non-stacked fractions and better spacing
% for units.
\usepackage{units}

% The fancyvrb package lets us customize the formatting of verbatim
% environments.  We use a slightly smaller font.
\usepackage{fancyvrb}
\fvset{fontsize=\normalsize}

% Small sections of multiple columns
\usepackage{multicol}

% Provides paragraphs of dummy text
\usepackage{lipsum}

% These commands are used to pretty-print LaTeX commands
\newcommand{\doccmd}[1]{\texttt{\textbackslash#1}}% command name -- adds backslash automatically
\newcommand{\docopt}[1]{\ensuremath{\langle}\textrm{\textit{#1}}\ensuremath{\rangle}}% optional command argument
\newcommand{\docarg}[1]{\textrm{\textit{#1}}}% (required) command argument
\newenvironment{docspec}{\begin{quote}\noindent}{\end{quote}}% command specification environment
\newcommand{\docenv}[1]{\textsf{#1}}% environment name
\newcommand{\docpkg}[1]{\texttt{#1}}% package name
\newcommand{\doccls}[1]{\texttt{#1}}% document class name
\newcommand{\docclsopt}[1]{\texttt{#1}}% document class option name


\begin{document}

\begin{titlepage}

\begin{fullwidth}
\noindent\LARGE\emph{French-German Communication Research
} \hspace{27mm}\includegraphics[height=1cm]{logo3.png}\\
\noindent\hrulefill\\
\vspace*{1em}
\noindent{\Huge{How German Communication Research\\\noindent Discovered Bourdieu but Missed His\\\noindent Potential for the Study of (Populist)\\\noindent Political Communication\par}}

\vspace*{1.5em}

\noindent\LARGE{Benjamin Krämer} \href{https://orcid.org/0000-0002-1351-2414}{\includegraphics[height=0.5cm]{orcid.png}}\par\marginnote{\emph{Benjamin Krämer, ``How German Communication Research Discovered Bourdieu but Missed His Potential for the Study of (Populist) Political Communication,'' \emph{History of Media Studies} 3 (2023), \href{https://doi.org/10.32376/d895a0ea.7ec13efb}{https://doi.org/ 10.32376/d895a0ea.7ec13efb}.} \vspace*{0.75em}}
\vspace*{0.5em}
\noindent{{\large\emph{LMU Munich}, \href{mailto:kraemer@ifkw.lmu.de}{kraemer@ifkw.lmu.de}\par}} \marginnote{\href{https://creativecommons.org/licenses/by-nc/4.0/}{\includegraphics[height=0.5cm]{by-nc.png}}}

% \vspace*{0.75em} % second author

% \noindent{\LARGE{<<author 2 name>>}\par}
% \vspace*{0.5em}
% \noindent{{\large\emph{<<author 2 affiliation>>}, \href{mailto:<<author 2 email>>}{<<author 2 email>>}\par}}

% \vspace*{0.75em} % third author

% \noindent{\LARGE{<<author 3 name>>}\par}
% \vspace*{0.5em}
% \noindent{{\large\emph{<<author 3 affiliation>>}, \href{mailto:<<author 3 email>>}{<<author 3 email>>}\par}}

\end{fullwidth}

\vspace*{1em}


\hypertarget{abstract}{%
\section{Abstract}\label{abstract}}

Starting with an outline of Pierre Bourdieu's reception in
German-speaking communication research, this article identifies an
important omission: his writings on politics and the corresponding
symbolic struggles. In particular, the field of research on political
communication could profit from the approaches outlined in these
publications. This is exemplified by research on populist communication,
which has become one of the most important topics in the field of
political communication. This contribution argues that, together with
Bourdieu's theory of social class, his conception of politics could be a
fruitful way to understand current (right-wing) populism, with its
construction of ``the people,'' its particular claim to representation,
as well as its social-structural basis and appeal to certain groups.
However, there are several barriers immanent in the dominant approaches
to research on political communication that can be understood from the
history of the field and that have hindered the adoption of such a
perspective in German-speaking communication research.




\vspace*{2em}

\noindent{\emph{History of Media Studies}, vol. 3, 2023}


 \end{titlepage}

% \vspace*{2em} | to use if abstract spills over

\newthought{The reception of} a scholar's work, particularly from other disciplines,
is always selective. This is an almost tautological claim, because
scholarship always involves the selection of earlier research that new
theories and studies can build on. What shapes this selectivity, and how
different selections lead to different outcomes, are the more
interesting questions. This article describes a case of selective
appropriation across national, linguistic, and disciplinary borders with
a complex pattern of openness and closure: In international comparison,
German-speaking communication research\footnote{In this paper,
  ``German(-speaking) communication research'' refers to a subfield of
  communication research in which actors from Austria, Germany, and
  (German-speaking) Switzerland are active (thus also potentially
  competing for positions in departments where a certain fluency in
  German is usually required, a barrier for international candidates.)
  It thus seems justified to speak of a definable subfield. While most
  of these actors occasionally publish in German, many if not most of
  their publications are in English. I am only aware of a handful of
  German-speaking researchers working outside French-speaking areas who
  have ever published in French.} is somewhat open to social theory, but
less open to the approaches from Francophone \emph{sciences de
l'information et de la communication} (SIC) and to more semiological
approaches.\footnote{On the historical differences between the two
  fields, see Stefanie Averbeck, ``Comparative history of communication
  studies: France and Germany,'' \emph{The Open Communication Journal}
  2, no. 2 (2008). One of the rare exceptions of German-speaking
  researchers of populism connecting with French-speaking literature is
  the recent article by Peter Maurer \& Rajesh Sharma, ``L'usage de
  narratifs populistes dans les~tweets des candidats `contestataires'
  aux élections présidentielles en France (2017) et aux États-Unis
  (2016),'' \emph{Revue internationale de politique comparée} 29, no.
  2--3 (2022). However, apart from non-French-speaking scholars of
  populism, they mostly cite French-speaking political scientists and
  scholars from other disciplines rather than researchers in SIC.}
Political communication is a field within (German and international)
communication research that is somewhat less open to theories of society
and rather internationalized, but mostly oriented towards Anglophone
literature.

\enlargethispage{\baselineskip}

This contribution is a historical analysis of this status quo and an
attempt at counterfactual history: It tries to locate some of the
missing links arising from this disciplinary and cultural constellation,
to explore its historical roots, and to imagine what could have happened
if German or international research on political communication had
opened up to concepts from French social theory related to the political
use of language. It describes how German communication research
discovered the work of Pierre Bourdieu but missed his potential for the
study of political communication. It then illustrates the untapped
potential of his work for this field and the historical reasons why it
has not been used, focusing on the example of his writings on political
sociology and the phenomenon of populist communication as one of the
important concepts (or buzzwords) in recent research on political
communication.

If we want to make the most of our analysis of political trends and
maximize our contribution to current debates, we should perhaps be
prepared to cross several borders: from nomothetic toward interpretive
or semiological, constructivist, or performative perspectives; from
(social) psychology and mid-range sociological and politological
perspectives as the most important external sources from which concepts
are borrowed toward more general social theory; and from a cluster of
Anglophone (and Germanophone, Dutch, Scandinavian, etc.) literature and
scholars toward the Francophone sphere. Pierre Bourdieu's work
exemplifies some of these potential new directions, providing a theory
of society with its social classes and trajectories shaping their stance
toward political developments such as the rise of populism. And it
provides a performative theory of politics that sees language use as
something shaping social fields rather than merely describing or
evaluating social reality, such as populism trying to reconfigure the
political field and people's vision of society.

However, there are various barriers to the adoption of such approaches
stemming from the history of German-speaking communication, which will
be discussed below. Taken together, this contribution is an invitation
to reflect on past and future paths in German-speaking and international
research in political communication, both to the field itself and to
everyone interested in the history and sociology of the field of
communication research.

\hypertarget{a-short-history-of-pierre-bourdieus-reception-in-german-communication-research}{%
\section{A Short History of Pierre Bourdieu's Reception in German\\\noindent Communication
Research}\label{a-short-history-of-pierre-bourdieus-reception-in-german-communication-research}}

German-speaking communication research has been characterized
historically as a field that has turned from normative, individualist,
interpretive, and historical perspectives towards empirical social
science between 1945 and the 1970s.\footnote{See Maria Löblich,
  \emph{Die empirisch-sozialwissenschaftliche Wende in der Publizistik-
  und Zeitungswissenschaft} (Köln: Halem, 2010). On the earlier
  reception of sociological approaches, see Stefanie Averbeck,
  \emph{Kommunikation als Prozess: Soziologische Perspektiven in der
  Zeitungswissenschaft, 1927--1934} (Münster: Lit, 1999).} The most
prominent proponent of this social-scientific turn in German-speaking
communication research is probably Elisabeth Noelle-Neumann, the
founding figure of the ``Mainz School,'' which epitomized this
reorientation. The turn has been welcomed by its proponents as a sign of
renewal for a discipline that had been discredited by its political
conformity during National Socialism and as a path toward academic
legitimacy by means of scientification. On the other side, this
development has probably led to a reluctance toward anything that has
either a literary or art criticism, or ``grand theory,'' feel to it.

Those who were not willing to (completely) follow this turn or were
later dissatisfied with the emerging mainstream paradigm of effects- and
variable-oriented research had several possibilities:

\begin{enumerate}
\item
  To defend the \emph{older paradigm} despite its reputation as
  old-fashioned and descriptive---a reputation the proponents of the
  turn had successfully created over time.
\item
  To (re-)turn to \emph{historical approaches}. The older schools cannot
  be reduced to such perspectives, although some protagonists put a
  strong emphasis on the study of past media institutions and actors.
  Historical approaches did not completely vanish with the
  social-scientific turn. There is still a remaining or renewed niche
  for (often theoretically informed) historical scholarship that is
  accepted by the discipline as a legitimate, albeit minor field of
  research (as testified by a vibrant Section for Communication History
  in the German Communication Association and a number of professorships
  dedicated to the topic).
\item
  To turn to \emph{Marxism and critical theory.} Some scholars tried to
  establish a school of critical communication research as early as the
  1960s.\footnote{Andreas Scheu, \emph{Adornos Erben in der
    Kommunikationswissenschaft: Eine Verdrängungsgeschichte?} (Köln:
    Halem, 2012), 12.} However, in the rather centrist to conservative
  climate in German-speaking communication research during the
  post--World War II and Cold War eras, and the more general political
  climate with its suspicion toward left-wing civil servants
  (exemplified by the 1972 ``anti-radical decree'' essentially banning
  them from their professions), those approaches were marginalized in
  the following decades.\footnote{Scheu, \emph{Adornos Erben}.} One of
  the few scholars from the Frankfurt School still cited today in
  communication research is Jürgen Habermas.\footnote{As an example of
    his reception in communication research, see Hartmut Wessler,
    \emph{Habermas and the Media} (Cambridge, MA: Polity, 2018).} This
  may be explained by his reputation not as a radical critical theorist
  but one that has made him a kind of ``state philosopher'' of the
  Federal Republic of Germany. His theories of communicative action and
  deliberative ethics and democracy are used in German-speaking
  communication research less as a foundation of a fundamental critique
  of modern rationalization or of a radically new form of democracy, but
  more as a template for the empirical analysis of actual discourses
  along certain quantifiable norms of inclusivity and argumentativity.
\item
  To turn to \emph{constructivist social theory and epistemology}, most
  importantly the alternative epistemology of radical constructivism or
  Niklas Luhmann's social systems theory. In the 1990s, those looking
  beyond low- and middle-range theories and ad hoc models and hypotheses
  were often attracted to these schools. The ``radicalism'' of these
  approaches, whose reception was most importantly centered around the
  University of Münster, does not lie in their critique of the status
  quo but in their opposition to naive realism and commonplace
  conceptions of the social. However, the potential of these schools may
  have seemed exhausted in the late 2000s (although many ideas in
  Luhmann's work had not been adopted and others still somewhat
  reverberate in journalism studies or theories of the public sphere).
  Most of the German-speaking scientific community may have been tired,
  if not outright skeptical, of the abstract discussions of concepts
  such as autopoeisis (does only communication communicate or does
  social systems theory erase humans as the center of communication?),
  the ``code'' of journalism (is it published versus unpublished,
  topical versus non-topical, information versus no information?), and
  journalism's relationship with other systems (is it best described as
  interpenetration, intersection, irritation, de-differentiation,
  etc.?).
\item
  To ``import'' \emph{cultural studies approaches}.\footnote{For
    example, Andreas Hepp has published an early introduction to
    cultural studies for media and communication research: Andreas Hepp,
    \emph{Cultural Studies und Medienanalyse: Eine Einführung}
    (Wiesbaden: VS, 1999).} However, their reception came with its own
  difficulties:\footnote{On the following, see Katja Schwer, ``Typisch
    deutsch? Zur zögerlichen Rezeption der Cultural Studies in der
    deutschen Kommunikationswissenschaft,'' \emph{Münchener Beiträge zur
    Kommunikationswissenschaft}, no. 2 (2005).} Some scholars certainly
  sought a niche for constructivism as an alternative to a mainstream
  that claimed, based on its reading of critical rationalism, that
  objective knowledge is possible or can at least be approached by
  researchers and media professionals. However, this niche was already
  occupied by the cognitivist epistemology of radical constructivism
  instead of the social, culturalist, contextual, and self-reflexive
  perspective of the cultural studies approach. Furthermore, this
  approach had to compete with or had to endure the same suspicion as
  other critical approaches, most importantly in Germany in the form of
  critical theory. Meanwhile, many researchers who had introduced
  cultural studies approaches into German-speaking communication
  research have turned toward Continental social theorists or
  qualitative and critical approaches less strictly tied to cultural
  studies as foundation for their work.
\end{enumerate}

It may have been the search for alternative social theories that has led
those unsatisfied with both the dominant ``realist''
critical-rationalist paradigm and the established heterodoxy of social
systems theory to explore the work of Bourdieu, starting in the 2000s.
It addressed both dissatisfactions by proposing a perspective that is
non-individualist but still actor-centric, dynamic, and focused around
specific practices (as opposed to the seemingly static dichotomies of
Luhmann's codes)---an encompassing, abstract description of society that
is, at the same time, empirically fruitful.\footnote{The same could be
  said, for example, about the theories of Anthony Giddens or Jürgen
  Schimank that some German-speaking communication researchers have
  turned to. For example, Christoph Neuberger names the theories of
  Bourdieu, Giddens, Luhmann, and Schimank as foundations that have been
  used for the analysis of mediatization. See Christoph Neuberger,
  ``Journalismus und Medialisierung der Gesellschaft,'' in
  \emph{Journalismusforschung: Stand und Perspektiven}, ed. Klaus Meier
  and Christoph Neuberger (Baden-Baden: Nomos, 2023).}

Thus, Bourdieu seems to be a good fit not for the absolute mainstream
but for those looking beyond both the constructivist and functionalist
schools and the variable-centered approaches. Still, certain aspects of
his work were neglected, as will be shown below.

Quantitative social scientists may find testable predictions,
correspondence analyses, and other statistical data in Bourdieu's
publications; but also pages of jargon, paradoxical and seemingly
tautological phrases; long analyses of Flaubert's \emph{Sentimental
Education}, photography in rural France, or Heidegger's worldview; and
comparisons between political preferences and the taste for certain
types of cheese. This is in stark contrast with the literature usually
cited in a large part of the field---papers that review previous
research, introduce a number of concepts, and deduce a few hypotheses,
followed by a methodological section, the results, and a number of
rather straightforward conclusions. Bourdieu's prose, some of his
topics, and the complexity of his conceptual and theoretical framework
made his oeuvre (like almost all general social theory) seem less useful
to much of the dominant pole of the field but attractive to those
looking for more than constructs with established scales to combine into
hypotheses and models.

Although Bourdieu can be read as a ``pure'' social theorist instead of
an activist or critical scholar,\footnote{On the reception in Germany
  beyond communication research and long before the adoption of
  Bourdieu's theories in this field, see Michael Gemperle, ``The Double
  Character of the German Bourdieu: On the Twofold Use of Pierre
  Bourdieu's Work in the German-Speaking Social Sciences,''
  \emph{Sociologica} 2009, no. 1 (2009). He notes that the more recent
  reception has been increasingly apolitical.} and although his concepts
can be used for the most conventional empirical social-scientific
research, terms like bourgeoisie, domination, capital, etc., may still
have some connotations of class struggle or armchair leftism and are not
quite part of the usual vocabulary of German-speaking communication
research.

Where his work has been drawn upon, Bourdieu has therefore been read not
so much as the ``political'' intellectual or ``activist'' investigator
of \emph{misère} and suffering or of neoliberalism or as ``communication
theorist'' (see below). He has instead been considered in German
communication research (most notably in Munich) from three different
perspectives---very much in line with his reception in the French social
sciences and internationally:\footnote{See Thomas Wiedemann, ``Pierre
  Bourdieu: Ein internationaler Klassiker der Sozialwissenschaft mit
  Nutzen für die Kommunikationswissenschaft,'' \emph{Medien \&
  Kommunikationswissenschaft} 62, no. 1 (2014). Interestingly, Wiedemann
  (p. 97) emphasizes that Bourdieu's work ``has more to offer to the
  discipline {[}of communication research---obviously in the
  German-speaking context{]} than the publication `On television'
  \emph{and a number of works on language and symbolism}'' (my
  translation; emphasis added).}

\begin{enumerate}
\item
  as an analyst of cultural \emph{distinctions},\footnote{Pierre
    Bourdieu, \emph{La distinction: Critique sociale du jugement}
    (Paris: Minuit, 1979).} to study the homology between social class
  and media use;\footnote{E.g., Klaus Beck, Till Büser, and Christiane
    Schubert, ``Medialer Habitus, mediales Kapital, mediales
    Feld---oder: vom Nutzen Bourdieus für die Mediennutzungsforschung,''
    in \emph{Pierre Bourdieu und die Kommunikationswissenschaft:
    Internationale Perspektiven}, ed. Thomas Wiedemann and Michael Meyen
    (Köln: Halem, 2013); Benjamin Krämer, \emph{Mediensozialisation:
    Theorie und Empirie zum Erwerb medienbezogener Dispositionen im
    Lebensverlauf} (Wiesbaden: Springer VS, 2013); Michael Meyen,
    ``Medienwissen und Medienmenüs als kulturelles Kapital und als
    Distinktionsmerkmale: Eine Typologie der Mediennutzer in
    Deutschland,'' \emph{Medien und Kommunikationswissenschaft} 55, no.
    3 (2007); Michael Meyen and Senta Pfaff-Rüdiger, \emph{Internet im
    Alltag: Qualitative Studien zum praktischen Sinn von
    Onlineangeboten} (Münster: Lit, 2009); Helmut Scherer,
    ``Mediennutzung und soziale Distinktion,'' in \emph{Pierre Bourdieu
    und die Kommunikationswissenschaft. Internationale Perspektiven},
    ed. Thomas Wiedemann and Michael Meyen (Köln: Halem, 2013).}
\item
  as a field theorist and surveyor of scientific and other
  fields,\footnote{Pierre Bourdieu, \emph{Homo academicus} (Paris:
    Minuit, 1981); Pierre Bourdieu, \emph{Les règles de
    l\textquotesingle art: Genèse et structure du champ littéraire}
    (Paris: Seuil, 1992).} to describe the structures and trajectories
  in the field of communication research\textsuperscript{15} and to analyze
  the journalistic field;\textsuperscript{16} and
\item
  as a theorist of social practices and of the practical
  sense,\textsuperscript{17} to conceptualize
  journalistic or digital practices or practices of media
  use.\textsuperscript{18}
\end{enumerate}

Interestingly, Bourdieu has not been ``imported'' via French
\emph{sciences de l'information et de la communication} (SIC), but
directly from sociology as another social theorist just like Luhmann and
others. The reasons for this are twofold. First, there is little direct
interaction between French and German communication researchers, both
due to language barriers (with a lot of relevant research in France
still published in French, as opposed to German researchers publishing
some works, such as monographic dissertations, in German, but a large
part in English-language journals) and paradigmatic
differences.\textsuperscript{19} Second, Bourdieu is by no
means a central figure in SIC as the references (or lack thereof) to his
work in introductory textbooks demonstrate.\textsuperscript{20}

A topic and perspective in Bourdieu's work that has been largely
neglected can be found in his earlier writings on politics that analyze
how political concepts and discourses acquire their meaning in symbolic
struggles and the power of such discourses to constitute social
reality.\textsuperscript{21} These
writings, mostly published in the 1970s and early 1980s and often in
\emph{Actes de la recherche en sciences sociales}, may be grouped
together with other texts on the social role of language in general, on
``what speaking means''\textsuperscript{22}---not only in terms of denotations proper, but more
importantly, who is and feels entitled to speak, how different ways of
speaking practiced by different groups\newpage\noindent are valued,\marginnote{\textsuperscript{15} E.g., Nathalie Huber,
    \emph{Kommunikationswissenschaft als Beruf: Zum Selbstverständnis
    von Professoren des Faches im deutschsprachigen Raum} (Köln: Halem,
    2010); Maria Löblich and Andreas Scheu, ``Writing the History of
    Communication Studies: A Sociology of Science Approach,''
    \emph{Communication Theory} 21, no. 1 (2011); Michael Meyen, ``Der
    Machtpol des kommunikationswissenschaftlichen Feldes,''
    \emph{Studies in Communication and Media} 1, no. 3­­--4 (2012); Scheu,
    \emph{Adornos Erben}; Thomas Wiedemann, \emph{Walter Hagemann:
    Aufstieg und Fall eines politisch ambitionierten Journalisten und
    Publizistikwissenschaftlers} (Köln: Halem, 2015).} and\marginnote{\textsuperscript{16} E.g., Thomas Hanitzsch, ``Populist
    Disseminators, Detached Watchdogs, Critical Change Agents and
    Opportunist Facilitators: Professional Milieus, the Journalistic
    Field and Autonomy in 18 Countries,'' \emph{International
    Communication Gazette} 73, no. 6 (2011); Peter Maurer and Andreas
    Riedl, ``Why Bite the Hand That Feeds You? Politicians' and
    Journalists' Perceptions of Common Conflicts,'' \emph{Journalism}
    22, no. 11 (2011). For an analysis of the use of Bourdieu's concepts
    in international journalism research, see Phoebe Maares and Folker
    Hanusch, ``Interpretations of the Journalistic Field: A Systematic
    Analysis of How Journalism Scholarship Appropriates Bourdieusian
    Thought,'' \emph{Journalism} 23, no. 4 (2022).} how\marginnote{\textsuperscript{17} Pierre Bourdieu, \emph{Esquisse d\textquotesingle une
    théorie de la pratique, précédé de Trois études
    d\textquotesingle ethnologie kabyle} (Geneva: Droz, 1972); Pierre
    Bourdieu, \emph{Le sens pratique} (Paris: Minuit, 1980); Pierre
    Bourdieu, \emph{Raisons pratiques: Sur la théorie de
    l\textquotesingle action} (Paris: Seuil, 1994).} speaking\marginnote{\textsuperscript{18} E.g., Benjamin Krämer, \emph{Mediensozialisation:
    Theorie und Empirie zum Erwerb medienbezogener Dispositionen im
    Lebensverlauf} (Wiesbaden: Springer VS, 2013); Benjamin Krämer,
    ``Strategies of Media Use,'' \emph{Studies in Communication and
    Media} 2, no. 1 (2013); Christian Pentzold, ``Praxistheoretische
    Prinzipien, Traditionen und Perspektiven kulturalistischer
    Kommunikations- und Medienforschung,'' \emph{Medien und
    Kommunikationswissenschaft} 63 (2015); Christian Pentzold,
    \emph{Zusammenarbeiten im Netz: Praktiken und Institutionen
    internetbasierter Kooperation} (Wiesbaden: Springer VS, 2016);
    Johannes Raabe, \emph{Die Beobachtung journalistischer Akteure:
    Optionen einer empirisch-kritischen Jour-}}
brings about social reality---and on symbolic power as such.\textsuperscript{23}

This understanding of communication is rather at odds with the prevalent
conception in German-speaking communication research that locates
communicative ``power'' in the quantitative distribution of the
``manifest'' media content,\textsuperscript{24} i.e., the objectively measurable mentions of topics, actors,
and opinions, and the effects of this ``dose'' on a typical user or a
user with a set of preexisting attitudes.

Thus, although Bourdieu was attractive to a number of communication
researchers looking for social theory that can be read both critically
and as distanced analysis (but without the formalism of social systems
theory or the stigma of critical theory), there have been particular
barriers preventing the widespread adoption of Bourdieu's ideas in the
German-speaking areas and in research on political communication in
particular. To illustrate this last point and thus contribute to the
self-reflection of this field, I will not simply list these obstacles,
but use the potential of counterfactual history, taking the example of
the recently much discussed topic of populist communication.

To start with, it will be necessary to familiarize ourselves with the
history of research on populist communication that shapes the
intellectual landscape of this field today and mirrors many features of
the overall field of research on political communication.

\hypertarget{a-very-short-history-of-research-on-populist-communication}{%
\section{A Very Short History of Research on Populist
Communication}\label{a-very-short-history-of-research-on-populist-communication}}

To the best of my knowledge, the concept or phenomenon of populism had
not received considerable attention in the discipline of communication
research until the 2010s,\textsuperscript{25} with a small
number of publications in the 2000s (most notably by Gianpietro
Mazzoleni\textsuperscript{26}). Subsequently,
research quickly moved from a small number of publications investigating
the relationship between journalist and populist logics\textsuperscript{27} to a
strongly networked field (at least in Europe) that was able to take
stock of its common activities and accomplishments in a number of
collective volumes and special issues.\textsuperscript{28} In the context of European communication
research, my impression is that of a consolidation and convergence
toward a perspective that is very much in line with the most prevalent
approach in political communication research as a whole: Populism is
conceptualized rather formally as an ideology based on a combination of
dimensions that can then be operationalized in standardized content
analyses and questionnaires in order to explain the occurrence of
populist messages and attitudes or related constructs. By combining the
concept of populism with other constructs, usually from the repertoire
of political communication, journalism\marginnote{\emph{nalismusforschung}
    (Wiesbaden: VS, 2005); Ralph Weiß, ``\,`Praktischer Sinn,' soziale
    Identität und Fern-Sehen: Ein Konzept für die Analyse der Einbettung
    kulturellen Handelns in die Alltagswelt,'' \emph{Medien und
    Kommunikationswissenschaft} 48, no. 1 (2000); Johannes Raabe,
    ``Kommunikation und soziale Praxis: Chancen einer
    praxistheoretischen Perspektive für Kommunikationstheorie und
    -forschung,'' in \emph{Theorien der Kommunikations- und
    Medienwissenschaft: Grundlegende Diskussionen, Forschungsfelder und
    Theorieentwicklungen}, ed. Carsten Winter, Andreas Hepp, and
    Friedrich Krotz (Wiesbaden: VS, 2008). Previously cited authors
    Hanitzsch, Huber, Löblich, Meyen, Pfaff-Rüdiger, Scheu, Wiedemann,
    and Krämer have all worked or still work in Munich, which highlights
    the role of this single (but quite large) department for Bourdieu's
    reception in German-speaking communication research.} studies,\marginnote{\textsuperscript{19} Stefanie Averbeck, ``Comparative History of
  Communication Studies: France and Germany,'' \emph{The} \emph{Open
  Communication Journal} 2, no. 2 (2008).} and\marginnote{\textsuperscript{20} Some introductions
  to SIC do not cite Bourdieu at all (e.g., Daniel Bougnoux,
  \emph{Introduction aux sciences de la communication} {[}Paris: La
  Découverte, 2001{]}) or only with regard to rather peripheral aspects,
  as opposed to dedicating a certain amount of space to his cultural
  sociology, theory of practice, or works related to journalism (e.g.,
  Bruno Ollivier, \emph{Les sciences de la communication: Théories et
  acquis} {[}Paris: Armand Colin, 2007{]}).
  Of course, there are some scholars in French sociology studying
  communication and the media or working at the intersection between
  sociology and SIC who treat Bourdieu as canonical, such as Éric
  Maigret, \emph{Sociologie de la communication et des médias} (Paris:
  Armand Colin, 2015); Jean-Pierre Esquenazi, ``Les médias et leurs
  publics,`` in \emph{Sciences de l'information et de la communication},
  ed. Stéphane Olivesi (Paris: PUG, 2013).} media\marginnote{\textsuperscript{21} One exception is Benjamin Krämer, ``Eine Bourdieu'sche
  Kritik der politischen Urteilskraft,'' in \emph{Pierre Bourdieu und
  die Kommunikationswissenschaft: Internationale Perspektiven}, ed.
  Thomas Wiedemann and Michael Meyen (Köln: Halem, 2013).} or\marginnote{\textsuperscript{22}\setcounter{footnote}{22} Pierre Bourdieu, ``Ce que parler veut
  dire,'' in \emph{Questions de sociologie}, ed. Pierre Bourdieu (Paris:
  Minuit, 1980).} social
psychology, the degree or frequency of populist communication from
certain sources can be explained by differences among political actors
or media outlets. This approach further allows researchers to
experimentally investigate the effect of populist messages and to
identify the correlates of populist attitudes---such as low trust in the
media. This consolidation is certainly not without exceptions, but
German-speaking researchers are among the main actors in the field who
follow this paradigm.\textsuperscript{29} This leads to highly cumulative research efforts but also,
almost inevitably, to a number of omissions and rigidities, such as:

\begin{itemize}
\item
  that certain phenomena, such as ``the'' people, ``the'' elite, the
  in-group, the out-group, etc., and perhaps even ``populism'' itself,
  are more or less taken for granted (or even essentialized), rather
  than analyzed in detail with respect to their discursive constitution;
\item
  the relative neglect of overarching social contexts (classes or
  milieus), trajectories or biographies, and overall ideologies or
  worldviews, often in favor of a strictly experimental logic of
  stimulus and response or analyses including a small number of
  pre-existing attitudes and covariates;
\item
  the dominance of quantitative over qualitative, interpretive,
  biographical, historical, and iconographical studies that would focus
  on meaning, performance, narrative, style, and symbolism; and
\item
  a certain distance toward critical and more or less openly left-wing
  schools (including, for example, the Essex school of discourse
  analysis and populism research associated with Ernesto Laclau and
  Chantal Mouffe).
\end{itemize}

Somewhat distinct from the dominant paradigm, albeit not completely
disconnected from it, other researchers have emphasized the
performative, discursive, or stylistic aspects of populism. At the same
time, they have also established a more explicit connection with
communication, discursive practices, and the media (however, mostly
outside the German-speaking region\textsuperscript{30}). This sets
them apart from still other strands of populism research that remain
very vague with regard to these aspects or are mostly interested in
political theory, party strategies, electoral behavior, and the like.

Based on this characterization of current research on populist
communication by German-speaking researchers, we can now turn to the
counterfactual question of what could have happened if Bourdieu's
political sociology had been an established approach in German-speaking
research in political communication, and to the historical reasons why
this path has not been taken.

\hypertarget{vous-avez-dit-populiste-bourdieus-political-sociology-and-populist-communication}{%
\section{Vous\marginnote{\textsuperscript{23} Pierre
  Bourdieu, ``Sur le pouvoir symbolique,'' \emph{Annales: Économies,
  sociétés, civilisations} 32, no. 3 (1977).} Avez\marginnote{\textsuperscript{24} As in the famous definition of
  content analysis by Bernhard Berelson and Paul F. Lazarsfeld,
  \emph{The Analysis of Communication Content} (Chicago: University of
  Illinois, 1948), 6. Despite different challenging paradigms, such as
  the constructivist school, many scholars continue to adhere to the
  epistemological or pragmatic principle that media content is a
  reliably measurable object and as such an identifiable source of
  effects.} Dit `Populiste'? Bourdieu's Political Sociology\\\noindent
and Populist
Communication}\label{vous-avez-dit-populiste-bourdieus-political-sociology-and-populist-communication}}

\hypertarget{bourdieu-on-populism}{%
\subsection{Bourdieu on
Populism?}\label{bourdieu-on-populism}}

To\marginnote{\textsuperscript{25} There have been, however, a number of
  publications in political science with a certain focus on
  communication. Probably the most important in terms of later citations
  by communication researchers is Jan Jagers and Stefaan Walgrave,
  ``Populism as Political Communication Style: An Empirical Study of
  Political Parties\textquotesingle{} Discourse in Belgium,''
  \emph{European journal of political research} 46 (2007).} the\marginnote{\textsuperscript{26} Gianpietro Mazzoleni, ``Mediatization and Political
  Populism,'' in \emph{Mediatization of politics: Understanding the
  Transformation of Western Democracies}, ed. Frank Esser and Jesper
  Strömbäck (London: Palgrave Macmillan, 2014); Gianpietro Mazzoleni,
  ``Populism and the Media,'' in \emph{Twenty-First Century Populism:
  The Spectre of Western European Democracy}, ed. Daniele Albertazzi and
  Duncan McDonnell (London: Palgrave Macmillan, 2008); Gianpietro
  Mazzoleni, ``The Media and the Growth of Neo-Populism in Contemporary
  Democracies,'' in \emph{The Media and Neo-Populism: A Contemporary
  Comparative Analysis}, ed. Gianpietro Mazzoleni, Julianne Stewart, and
  Bruce Horsfield (Westport, CT: Praeger, 2003); Julianne Stewart,
  Gianpietro Mazzoleni, and Bruce Horsfield, ``Conclusion: Power to the
  Media Managers,'' in \emph{The Media and Neo-Populism: A Contemporary
  Comparative Analysis}, ed. Gianpietro Mazzoleni, Julianne Stewart, and
  Bruce Horsfield (Westport, CT: Praeger, 2003).} best\marginnote{\textsuperscript{27} E.g.,
  Stewart, Mazzoleni, and Horsfield, ``Conclusion''; Benjamin Krämer,
  ``Media Populism: A Conceptual Clarification and Some Theses on Its
  Effects,'' \emph{Communication Theory} 24, no. 1 (2014).} of\marginnote{\textsuperscript{28} E.g., Toril Aalberg et
  al., eds., \emph{Populist Political Communication in Europe} (New
  York: Routledge, 2017); Carsten Reinemann et al., eds.,
  \emph{Communicating Populism: Comparing Actor Perceptions, Media
  Coverage, and Effects on Citizens in Europe} (New York: Routledge,
  2019); Benjamin Krämer and Christina Holtz-} my knowledge, Bourdieu never mentioned the concept of
populism. So why read him as a theorist relevant to populism research
and speculate about the trajectories of German-speaking political
communication research had his concepts been adopted? Bourdieu came
close to populism in three very different ways.

First, without mentioning right-wing populism as a concept, Bourdieu
distinguished between a liberal and a reactionary type of conservatism
(the latter defined by its abhorrence of established politics and its
fixation on the hypocrisy of the ruling class---reminiscent of populist
anti-elitism).\textsuperscript{31} He
linked this difference to his conception of social classes, which can
thus be relevant to populism research (as discussed below).

Second, his writings on politics focus on the concept of representation
and of the idea and/or signifier of ``the popular'' (see below). We may
also refer to Bourdieu's field theory to analyze the changing
relationship between the journalistic and the political field due to
populism or as a factor enabling populism. However, this idea of a
changing relationship between politics and the media is more familiar to
scholars of political communication, even if it is rarely expressed in
terms of field theory, but more often in terms of concepts such as
mediatization.\textsuperscript{32} This is why we will focus on the former
aspect of representation and signification.

Third and finally, we may speculate that his endorsement of the comedian
Coluche's potential candidacy in the 1981 French presidential election
followed a kind of ``populist'' impetus. Bourdieu supported the idea
that ``just anybody can be a candidate'' and linked this to his
criticism of politicians' ``monopoly'' on politics and of the exclusion
of ``irresponsible'' outsiders from the arena of politics in the name of
technocratic and juridical ``competence'' (see below on this division of
labor and the resulting disenfranchisement of voters).\textsuperscript{33} However, his criticism of
technocracy and political exclusion is often counterbalanced by a
criticism of demagoguery. Still, he neither conceptualized the
counterpart to technocracy as ``populism,'' nor saw demagoguery
primarily in populist claims of representation. His critique instead
centered on the construction of public opinion by means of opinion
polls.\textsuperscript{34}

We also have to note that in Bourdieu's writings on politics and
symbolic struggles, the role of the media is hardly discussed.
Meanwhile, his most important publications on the media focus on the
heteronomy\marginnote{Bacha, eds.,
  \emph{Perspectives on Populism and the Media: Avenues for Research}
  (Baden-Baden: Nomos, 2020).} of\marginnote{\textsuperscript{29} E.g., Sven Engesser et al., ``Populism
  and Social Media: How Politicians Spread a Fragmented Ideology,''
  \emph{Information, Communication \& Society} 20, no. 8 (2017); Nayla
  Fawzi, ``Untrustworthy News and the Media as `Enemy of the People?'
  How a Populist Worldview Shapes Recipients' Attitudes toward the
  Media,'' \emph{The International Journal of Press/Politics} 24, no. 2
  (2019); Nayla Fawzi and Benjamin Krämer, ``The Media as Part of a
  Detached Elite? Exploring Antimedia Populism among Citizens and Its
  Relation to Political Populism,'' \emph{International Journal of
  Communication} 15 (2021); Jörg Matthes and Desirée Schmuck, ``The
  Effects of Anti-immigrant Right-Wing Populist Ads on Implicit and
  Explicit Attitudes: A Moderated Mediation Model,'' \emph{Communication
  Research} 44, no. 4 (2017); Martin Wettstein et al., ``What Drives
  Populist Styles? Analyzing Immigration and Labor Market News in 11
  Countries,'' \emph{Journalism \& Mass Communication Quarterly} 96, no.
  2 (2019); Dominique S. Wirz et al., ``The Effects of Right-Wing
  Populist Communication on Emotions and Cognitions toward Immigrants,''
  \emph{The International Journal of Press/Politics} 23, no. 4 (2018).
  With only one exception, all of these authors worked in Munich or
  Zurich at the time of publication or at least before or after. These
  two departments are among the largest, most well-resourced, and most
  productive in the German-speaking field and could be described, from
  the perspective of Bourdieusian field theory, as belonging to the
  orthodox dominant pole of the field. Together, they have produced a
  very large part of the literature on populist communication coming
  from German-speaking universities. Interestingly, the circles in
  Munich that are interested in Bourdieu and that publish on political
  communication are almost entirely distinct (with the exception of
  Krämer). There are of course German-speaking scholars outside this
  cluster of populism research who also use the approach of populism as
  multi-dimensional ideology---see, for example, with a rather detailed
  analysis of different types or host ideologies of populism that are
  combined with the usual dimensions of populism in the strict sense,
  Peter Maurer and Trevor Diehl, ``What Kind of Populism? Tone} the journalistic field and its power over other fields of
cultural production and the political field, and on the homogenization
and \emph{doxa} of the field. In other words, his focus here is not on
the media's role in symbolic struggles proper.\textsuperscript{35}

\hypertarget{bourdieu-on-representation}{%
\subsection{Bourdieu on
Representation}\label{bourdieu-on-representation}}

In the current and subsequent sections, we will start from a kind of
straw man, a stereotypically naive understanding of populism which is,
however, only a caricature of actual preconceptions about populism in
public and academic discourse. We will not attribute these conceptions
to individual authors but introduce them to provide a Bourdieusian
criticism and give them a Bourdieusian turn. This allows us to discuss
why German-speaking political communication research has not taken that
turn and what would have happened if it had.

One simple idea of politics would assume a static political space
constituted by voters' more or less fixed ideologies or policy
preferences, and by the strategic positioning of political parties that
maximize their votes. If certain voters no longer feel represented, for
example, by European conservative parties shifting toward the center or
social-democratic parties neglecting the interests of ``the working
class'' in favor of identity politics, new parties can position
themselves in these gaps---for example, right-wing populist parties
representing conservative workers (or so this explanation goes).

Whether it is ``the working class,'' ``the people of Padania,'' ``the
Hindus,'' or simply ``ordinary people,'' it would seem as if groups that
feel disadvantaged and unrepresented (at least relative to their actual
number and role in a country) are desperately looking for some
representatives (most often a single person more or less supported or
appointed by a party or movement). The group ``creates'' the
representative, i.e., chooses this figure to represent them.\textsuperscript{36}

But let us imagine another world, in which German-speaking communication
researchers of political communication had already read one of the most
prominent sociologists of their neighboring country when populism became
a thing in the German public sphere or in the Anglophone research
literature. These researchers could then point to Bourdieu and challenge
the restrictive idea of politics outlined above. As Bourdieu argues,
politics is only possible because actors have their own conceptions of
the social world, which can be confirmed or changed in political
discourse, and are often diffuse. There would be no struggles over the
description of the social world if everyone had a precise and infallible
idea about their social position and affiliation to social groups, and
if actors were not capable of identifying\marginnote{and
  Targets in the Twitter Discourse of French and American Presidential
  Candidates,'' \emph{European Journal of Communication} 35, no. 5
  (2020).} themselves\marginnote{\textsuperscript{30}\setcounter{footnote}{30} E.g., Niko Hatakka,
  \emph{Populism in the Hybrid Media System: Populist Radical Right
  Online Counterpublics Interacting with Journalism, Party Politics, and
  Citizen Activism} (Turku: University of Turku, 2019); Benjamin
  Moffitt, \emph{The Global Rise of Populism: Performance, Political
  Style, and Representation} (Stanford: Stanford University Press,
  2016); Lone Sorensen, \emph{Populist Communication: Ideology,
  Performance, Mediation} (Cham: Palgrave Macmillan, 2021).} with\marginnote{\textsuperscript{31} Bourdieu, \emph{La distinction,} 528--30.} different\marginnote{\textsuperscript{32} E.g., Gianpietro Mazzoleni, ``Mediatization and
  Political Populism,'' in \emph{Mediatization of Politics}, ed. Frank
  Esser and Jesper Strömbäck (London: Palgrave Macmillan, 2014).

  I would like to thank Peter Maurer who, in his open review of the
  present article, emphasized the use of Bourdieu's \emph{field theory}
  for the analysis of populism.}
descriptions.\marginnote{\textsuperscript{33} Pierre
  Bourdieu, \emph{Propos sur le champ politique} (Lyon: Presses
  universitaires de Lyon, 2000), 55--56.}  However,\marginnote{\textsuperscript{34} See Pierre Bourdieu, ``Störenfried Soziologie,'' in
  \emph{Wozu heute noch Soziologie?}, ed. Joachim Fritz-Vannahme
  (Wiesbaden: VS, 1996). German communication researchers could have
  also come across this short article on sociology and democracy by
  Bourdieu in the weekly newspaper \emph{Die Zeit}.} such\marginnote{\textsuperscript{35} Pierre Bourdieu,
  ``L'emprise du journalisme,'' \emph{Actes de la recherche en sciences
  sociales} 101--2 (1994); Pierre Bourdieu, \emph{Sur le télévision}
  (Paris: Liber, 1996); and Pierre Bourdieu, ``The Political Field, the
  Social Science Field, and the Journalistic Field,'' in \emph{Bourdieu
  and the Journalistic Field}, ed. Rodney Benson and Erik Neveu
  (Cambridge, MA: Polity Press, 2005). In the last of these, he
  discusses the commonalities between the political field, the field of
  social science, and the journalistic field (in particular, the
  struggles to impose a vision of society), and quickly returns to the
  subject of the heteronomy of journalism in general as a threat to all
  fields of cultural production.} struggles\marginnote{\textsuperscript{36}\setcounter{footnote}{36} Pierre
  Bourdieu, ``La délégation et le fétichisme politique,'' \emph{Actes de
  la recherche en sciences sociales} 52--53 (1984).} would also be pointless if,
independently of their social position, everyone could find any
description similarly plausible.\footnote{Pierre Bourdieu, ``Décrire et
  préscrire,'' \emph{Actes de la recherche en sciences sociales} 38
  (1981).}

Surely, Bourdieu postulates a homology between political space and the
space of social inequality (which he described in a particularly useful
way for the explanation of certain current populist movements, as I will
discuss below).\textsuperscript{38} A given group of
persons will find certain political offers of representation more
plausible and attractive. However, he describes how the representative
also creates the group in the full sense. Someone speaks for a group,
represents it, but it only fully exists and can be mobilized by virtue
of this representation. In the extreme, this relationship between
constitution and representation, between delegation and mobilization, is
circular.\textsuperscript{39}

The aspiring representative invests some symbolic work (in the form of
words, theories in the broadest sense, rituals, and other symbolic
means), and thus a way of seeing the world and living in it that has
often only been vaguely felt (often as discontent) now becomes manifest
and allows groups to recognize commonalities where a unifying principle
had not been seen before.\textsuperscript{40} A group as a mere aggregate recognizes the representation and
authorizes the representative, who then manifests and embodies the
group, and draws their power from the ability to mobilize it.\textsuperscript{41}

The German-speaking researchers who read their Bourdieu would maybe have
to be socialized somewhat differently. They would not only have to be
familiar with the academic prose that is typical in international
journals (which have supplanted the German-speaking journals and
monographs as the most legitimate outlets in German-speaking political
communication research), but they would also have to be somewhat
fascinated by this particular writer's complex, sometimes paradoxical
and cryptic but often, at the same time, vivid, metaphorical style.
Thus, it would not feel too strange when Bourdieu expresses the logic of
political representation (and of any kind of investiture) in religious
and magical metaphors: the mystery of ministry, political fetishism, or
social magic.\textsuperscript{42} Representatives sacrifice
themselves, ostentatiously giving up their person, and creating another,
a social one, such as ``the people.''\textsuperscript{43} For example, when police searched his
party's headquarters, Jean-Luc Mélenchon famously exclaimed: ``La
République, c'est moi!'' and ``I am more than Jean-Luc Mélenchon, I am
seven million people!''\textsuperscript{44}

Bourdieu's analysis of the constitution of groups by their
representation is not too different from discursive conceptions of
populism.\textsuperscript{45} However, the connection with his concept of habitus
allows for a quite substantial analysis of the schemata of perception
and evaluation that are specific to social classes and other groups and
make certain offers\marginnote{\textsuperscript{38} See Bourdieu, ``Décrire et préscrire.'' The
  political field is structured by two complementary principles: the
  homology with social groups and the historical and strategic relations
  of the parties. Positions and shifts can only be understood with
  regard to the internal logic and history of the field and in relation
  to the positions and strategies of other actors.} of\marginnote{\textsuperscript{39} Bourdieu, ``La délégation et le fétichisme
  politique.''} representation\marginnote{\textsuperscript{40} Pierre Bourdieu, ``La représentation
  politique,'' \emph{Actes de la recherche en sciences sociales} 36­--37
  (1981).} and\marginnote{\textsuperscript{41} Pierre
  Bourdieu, ``Le mystère du ministère: Des volontés particulières à la
  `volonté générale,'\,'' \emph{Actes de la recherche en sciences
  sociales} 140 (2001).} certainmarginnote{\textsuperscript{42} Bourdieu, ``La délégation et le fétichisme
  politique''; Bourdieu, ``Mystère du ministère''; Pierre Bourdieu,
  \emph{Sur l\textquotesingle État: Cours au Collège de France
  (1989--1992)} (Paris: Seuil, 2012), 400.} strategies\marginnote{\textsuperscript{43} Bourdieu, ``La délégation
  et le fétichisme politique.''} for\marginnote{\textsuperscript{44} For an analysis of these and other
  statements by Mélenchon in terms of a populist understanding of
  representation, see Pierre Rosanvallon, \emph{Le siècle du populisme:
  Histoire, théorie, critique} (Paris: Seuil, 2020), 55--56.}
representing\marginnote{\textsuperscript{45}\setcounter{footnote}{45} Ernesto Laclau, \emph{On Populist Reason} (London:
  Verso, 2005); Chantal Mouffe, \emph{For a Left Populism} (London:
  Verso, 2018).} social reality and representing social groups acceptable to
them.\footnote{Laclau and Mouffe represent a shift away from social
  class as the main basis for (progressive) political mobilization and
  democratic renewal, emphasizing that the constitution of ``the
  people'' depends on the historical circumstances. However, to address
  specific groups (which can be social classes as the focus on much of
  Bourdieu's work, but also other groups, e.g., based on gender),
  populist communication needs to fit their habitus. Therefore, a
  discursive approach can benefit from the analysis of the habitus of
  different groups. Such an analysis would likely not consider whole
  classes, but factions thereof, perhaps also in combination with
  gender, age, social trajectories, etc.).} Thus, even if our
hypothetical German-speaking communication researchers found the
discursive approach to populism research rather abstract and too far
from any empirical application, they would not simply turn it away, but
rather seek to specify it with regard to Bourdieusian conceptions of the
group-to-be-represented and the group-as-represented.

According to Bourdieu and his German disciples, if a strategy of
representation works out, then the result seems natural. Not only the
represented but also political opponents, commentators, and even
researchers will be convinced, for example, that ``ordinary people'' are
no longer taken seriously by established political actors, that they
feel their concerns are neglected, that the political category of
``ordinary people'' indeed exists in the first place (even if it is not
the whole population) and that its concerns are legitimate ones (even if
one does not share them).

The rather abstract discursive theories of how populists establish an
equivalence of different demands which they then articulate as the basis
for antagonism between the people and an elite could be thus refined and
extended in different ways based on Bourdieusian theories.\footnote{For
  a review and discussion, see Yannis Stavrakakis, ``Antinomies of
  Formalism: Laclau's Theory of Populism and the Lessons from Religious
  Populism in Greece,'' \emph{Journal of Political Ideologies} 9, no. 3
  (2004).}

However, these discursive theories of populism à la Laclau and Mouffe
have also been neglected in favor of ideational definitions (populism as
ideology)\footnote{E.g., Kirk Hawkins and Christóbal Rovira Kaltwasser,
  ``What the (Ideational) Study of Populism Can Teach Us, and What It
  Can't,'' \emph{Swiss Political Science Review} 23, no. 4 (2017); Cas
  Mudde, ``The Populist Zeitgeist,'' \emph{Government and Opposition}
  39, no. 4 (2004); Ben Stanley, ``The Thin Ideology of Populism,''
  \emph{Journal of Political Ideologies} 13, no. 1 (2008).} in research
on populist communication in German-speaking countries and in the larger
academic networks which this research is embedded in. This is probably
the case because such research is more open to empirical approaches than
to political theory and to research that positions itself as value-free
rather than explicitly political, e.g., post-Marxist. In this context,
the analysis of populist communication mostly starts with the centrality
of the people,\textsuperscript{49} not with its
construction, and explanations of effects start with the priming of
social identities,\textsuperscript{50}
not so much with how they are constructed and appropriated in the first
place. Only the exclusionary or anti-elitist part of the construction of
the people is sometimes posited as a central dimension of
populism.\textsuperscript{51}

When actually existing communication researchers attempt to explain the
success of populist leaders and the formation or confirmation of
populist attitudes, it may seem as if everything is a perfect fit: They
identify the pre-existing attitudes, the features of populist messages
or of the election programs, and the social-psychological mechanisms
that\marginnote{\textsuperscript{49} For a somewhat more nuanced analysis of
  ``people-centrism,'' see Martin Van Leuwen, ``Measuring
  People-Centrism in Populist Political Discourse: A Linguistic
  Approach,'' in \emph{Imagining the Peoples of Europe: Populist
  Discourses across the Political Spectrum}, ed. Jan Zienkowski and Ruth
  Breeze (Amsterdam: John Benjamins, 2019). Still, the analysis is about
  how politicians refer to the people in terms of syntactics and
  perspective as an indicator of how central ``the people'' is in their
  discourse, not of how it is performatively constituted.} have\marginnote{\textsuperscript{50} For a discussion of this social identity
  perspective with a relatively significant number of remarks on the
  construction of ``popular'' identities, see Michael Hameleers et al.,
  ``The Persuasiveness of Populist Communication: Conceptualizing the
  Effects and Political Consequences of Populist Communication from a
  Social Identity Perspective,'' in \emph{Communicating Populism:
  Comparing Actor Perceptions, Media Coverage, and Effects on Citizens
  in Europe}, ed. Carsten Reinemann et al. (New York: Routledge, 2019).} to\marginnote{\textsuperscript{51}\setcounter{footnote}{51} Communication researchers often refer to the
  conceptualization of Jan Jagers and Stefaan Walgrave, ``Populism as
  Political Communication Style: An Empirical Study of Political
  Parties' Discourse in Belgium,'' \emph{European Journal of Political
  Research} 46, no. 3 (2007). The authors consider exclusion as a
  potential but not necessary element of populism which, when added to
  ``thin'' populism that only ``refers'' to the people, creates
  ``excluding populism.'' In this case, exclusion is an aspect that
  politicians can freely combine with the others (at least logically
  freely, not necessarily strategically) and that researchers can
  operationalize independently from the other elements such as
  anti-elitism. This makes this conceptualization highly attractive for
  quantitative research, but it does not contribute that much to the
  analysis of how ``the people'' is constituted if it is supposed to be
  central to populism.} come together in order to convince voters that a populist
politician or party represents them.

However, when Bourdieu describes the dialectic of representation, he
also emphasizes the dilemma of empowerment and disempowerment. Bourdieu
starts from the diagnosis that modern political fields and symbolic
production follow a division of labor: One side is the specialized
production of cultural, political, and religious offers (among others),
the struggle for the symbolic means of representing society, and the
monopoly of symbolic power.\footnote{Which does not only take place
  based on descriptions in the narrow sense but also on the quite
  serious ``game'' of appropriation of nonverbal means and all kinds of
  metaphors and analogies. See Pierre Bourdieu, ``Un jeu chinois: Notes
  pour une critique sociale du jugement,'' \emph{Actes de la recherche
  en sciences sociales} 2--4 (1976).} On the other side, consumption is
largely limited to recognizing the types of available offers and
identifying with some of them.\textsuperscript{53} With regard to
political representation, this creates a dilemma, especially for the
most disadvantaged groups, as even their representatives have to make
concessions to conventional political language that is detached from the
groups' experience. To the represented, this language feels
``borrowed;'' it censors and euphemizes the description of social
reality and creates a distance between the representatives and the
represented.\textsuperscript{54} Representatives,
having the symbolic means and access to the means of communication that
the represented lack, can even disempower and betray those who, to gain
more power, have themselves represented. In this way, they may
appropriate the interests of the represented groups and usurp the
resulting power as a means to their own ends.\textsuperscript{55} The
resulting feeling (among their supporters) of powerlessness and
resignation is rarely addressed in research on populist communication,
whether this feeling is driven by negative coverage of populist actors
as self-interested or by more direct contact with their messaging, which
may feel inauthentic to some citizens or as a misrepresentation of their
situation, experience, and identity.

German-speaking research in political communication sees itself mainly
as basic research (at least in comparison to applied research that would
be directly applicable to, for example, strategic political
communication), but it is often interested in topics that are already
acknowledged as social problems in the political, journalistic, and
academic mainstream (such as the rise of populism). More or less
adopting the definition of such problems as they circulate in public
discourse, the research being conducted usually does not propose
immediate solutions but reveals relationships that become relevant in
the search for causes or answers to these established
problems.\textsuperscript{56} A theory on the dialectic
of populist representation would only very indirectly fit this schema as
it is not the most obvious topic in a field mostly oriented toward an
analysis of the immediate causes of a phenomenon that is socially
defined as relevant. In an alternative world, political communication
research could, however, analyze how populists\marginnote{\textsuperscript{53} Pierre Bourdieu, ``Questions de
  politique,'' \emph{Actes de la recherche en sciences sociales} 16
  (1977); Bourdieu, ``Sur le pouvoir symbolique.'' Bourdieu would
  probably deny that this has fundamentally changed in an online
  environment in which almost everyone could in principle express
  themselves politically. He would argue that there is still a
  difference between those who feel entitled and able to comment on
  political matters and those who do not, and between those who merely
  express their approval of existing conceptions and reproduce them
  without being able to successfully establish new ones, and those who
  have the symbolic and institutional power to do so.} can symbolically
dispossess and disempower their constituency. It might\marginnote{\textsuperscript{54} Bourdieu, \emph{La distinction}.} be\marginnote{\textsuperscript{55} Bourdieu,
  ``Décrire et préscrire''; Bourdieu, ``Mystère du ministère.''} relevant\marginnote{\textsuperscript{56}\setcounter{footnote}{56} It is my impression that German-speaking
  communication research has also tended to shift from research whose
  relevance was grounded in ideological controversies and in journalism
  education, such as research on the effects of journalists' political
  attitudes on media content or on the quality of private and public
  broadcasting, to research that is useful for the solution or
  prevention of social problems, such as research on health
  communication or prevention of extremism.} for a
criticism of authoritarian populism to demonstrate in detailed empirical
case studies how authoritarianism has disappointed or turned against its
supporters (and, not to forget, to document the symbolic and often
physical violence against the primary outgroups of each authoritarian
movement and government without overly euphemizing it in abstract
conceptions of populism). However, German-speaking researchers could
associate such a ``paradoxical'' form of theorizing (e.g., people being
both empowered and disempowered) with, in their view, ``empirically
unfruitful'' social systems theory, as it is an emblematic form in which
Luhmann formulated a large part of his framework.

\hypertarget{bourdieu-on-the-possible-class-basis-of-populism}{%
\subsection{Bourdieu on the Possible Class
Basis of
Populism}\label{bourdieu-on-the-possible-class-basis-of-populism}}

Sometimes, a rather unfruitful understanding of populism as pure
opportunism---a weathervane of public opinion,\footnote{Which, as
  Bourdieu explained, does not exist, at least not in the form that is
  presupposed by opinion polls: an aggregate of the answers to the same
  genuinely political questions (assuming that the questions are
  understood in the same way and that everyone, across all classes, is
  able to answer them based on a political logic proper). See Pierre
  Bourdieu, ``L'opinion publique n'existe pas,'' \emph{Les temps
  modernes} 318 (1973).} or the politics of ``simple solutions''
appealing to the uneducated masses, a politics that is or attempts to be
``popular''---still circulates in public discourse. Political
communication scholars informed by Bourdieu's sociology of education
could warn us not to assume that education, as it is actually practiced,
would simply endow young citizens with all necessary competences.
Sometimes, they are only registered rather than conveyed by the
education system and acquired earlier or outside school by a population
that is then selected by institutions of higher education.\footnote{Pierre
  Bourdieu and Jean-Claude Passeron, \emph{La reproduction: Éléments
  pour une théorie du système d\textquotesingle enseignement} (Paris:
  Minuit, 1970). Certainly, educational systems have been adapted since
  Bourdieu's and Passeron's analysis, not the least due to the reception
  of their and similar work. However, educational inequality is not a
  problem of the past, as ongoing research and policy discussions
  demonstrate.} It would also be classist to assume that a lack of
formal education automatically drives people into the arms of more
problematic varieties of populism;.\footnote{Without entering into the
  discussion over ``good'' and ``bad'' varieties of populism, it is safe
  to say that German-speaking communication research} It would be an oversimplification to assume
that all types of populism have to be prevented by education or that
education is necessarily sufficient to overcome any given form of
illiberal populism. However, that populism must have something to do
with certain ``popular'' classes seems to be conventional wisdom---maybe
based on etymology or the notion that populism, due to its presumed
``simplicity,'' can only speak to the relatively disadvantaged in
society.

In his essay \emph{``}Vous avez dit `populaire'?'' {[}Did you say
``popular''?{]},\textsuperscript{60} Bourdieu analyzed how the understanding of ``the
popular'' is prone to manipulation according to one's interests and
prejudices and condescending views. According to a Bourdieusian
interpretation, to praise the ``popular'' is also to affirm what is an
ambivalent product of social domination.\textsuperscript{61}

A simplistic historical narrative on the causes of right-wing populism
in particular is that of right-wing populist parties as the ``new
working class parties,'' since left-wing parties, with their turn toward
``identity politics,''\marginnote{has mostly
  justified its studies of populism based on its \emph{problematic}
  relationship with liberal democracy and, in the case of right-wing
  populism, standards of non-discrimination. More rarely, it is seen as
  a challenge to democracy potentially leading to greater responsiveness
  and the closure of gaps of representation. However, more elaborate
  positive visions or existing models of populism, such as those
  theorized by Laclau and Mouffe and practiced as inclusive, liberal
  movements, are usually not addressed. Whether they represent an ideal
  for democracy is nevertheless up for debate, but let us confine
  ourselves to the quick clarification that they may at least be
  considered less problematic than authoritarian, illiberal,
  exclusionary varieties.} have\marginnote{\textsuperscript{60} Pierre Bourdieu, ``Vous avez
  dit\emph{~}`populaire'?'' \emph{Actes de la recherche en sciences
  sociales} 46 (1983).} driven\marginnote{\textsuperscript{61}\setcounter{footnote}{61} Pierre Bourdieu and
  Jean-Claude Passeron, ``Sociologues des mythologies et mythologies de
  sociologues,'' \emph{Les Temps Modernes} 211 (1963); Bourdieu, ``Vous
  avez dit `populaire'?''; Pierre Bourdieu, ``Les usages du peuple,'' in
  \emph{Choses dites}, ed. Pierre Bourdieu (Paris: Minuit, 1987).} their former voters into the arms of
the extreme right.\footnote{For a recent criticism based on empirical
  data, see Tarik Abou-Chadi, Reto Mitteregger, and Cas Mudde,
  \emph{Left Behind by the Working Class? Social Democracy's Electoral
  Crisis and the Rise of the Radical Right} (Berlin:
  Friedrich-Ebert-Stiftung, 2021).\vspace{.1in}} In contrast, Bourdieu's conception
of social space can help us disentangle the various influences on
populist attitudes and the sometimes inconsistent findings on the
social-structural basis of populism (even if we focus on Central
European right-wing populism).

When the social-structural and cultural conditions of right-wing
populism are investigated, this usually takes one of two forms:

\begin{itemize}
\item
  an analysis along single variables, most often for descriptive or
  illustrative purposes (e.g., a comparison of the intention to vote for
  a right-wing populist party by gender that is presented in the context
  of a more encompassing argument) or as covariates or control variables
  in more complex models; or
\item
  an analysis along certain key theses or concepts, such as the ``losers
  of modernization'' or ``cultural backlash'' theses, the idea of a
  silent or noisy counter-revolution to the ``silent revolution'' of
  cultural change toward post-material values, or of a new divide
  between cosmopolitanism and communitarianism.\footnote{Pieter De Wilde
    et al., eds., \emph{The Struggle over Borders: Cosmopolitanism and
    Communitarianism} (Cambridge: Cambridge University Press, 2019);
    Piero Ignazi, ``The Silent Counter-Revolution: Hypotheses on the
    Emergence of Extreme Right-Wing Parties in Europe,'' \emph{European
    Journal of Political Research} 22, no. 1 (1992); Ronald Inglehart
    and Pippa Norris, ``Trump and the Populist Authoritarian Parties:
    The Silent Revolution in Reverse,'' \emph{Perspectives on Politics}
    15, no. 2 (2017); Lars Rensmann, ``The Noisy Counter-Revolution:
    Understanding the Cultural Conditions and Dynamics of Populist
    Politics in Europe in the Digital Age,'' \emph{Politics and
    Governance} 5, no. 4 (2017).\vspace{.08in}}
\end{itemize}

Even analyses based on the second type of explanation do not always draw
on an elaborate conception of social structure, often quickly proceeding
to the measurement of the variables implied in the hypotheses, and
publications in communication research touch upon such theses only very
briefly, if at all. The first type is more common in the field of
communication research---if any social-structural covariates or control
variables are included at all. Quantitative research in political
communication often focuses on other variables and relies on age,
education, and gender as proxies for more complex influences of social
status and socialization, mostly without specifying any conceptual
underpinnings or causal mechanisms. Research on the role of the media
for right-wing populism mostly remains detached from analyses of
social-structural causes, and German-speaking research on political
communication in general has not developed a substantial tradition of
social-structural analysis, whether Bourdieusian or otherwise. Due to
its historical roots, it often remains media-related public opinion
research, as in the beginnings of the Mainz School.\footnote{Elisabeth
  Noelle-Neumann also founded a polling institute and relied heavily on
  its results to substantiate her theory of public opinion.} It usually
starts with opinions on a given issue (more rarely with behavior, such
as voting), treating all individual opinions as equal in principle and
their aggregate as a relevant social phenomenon because it is the object
of both political influence and climate of opinion perceptions. Studies
then analytically move backwards to the media-related causes of these
opinions.\textsuperscript{65}

Bourdieu has contributed at least two important aspects to a fruitful
analysis of social inequality: multiple dimensions of social inequality\marginnote{\textsuperscript{65}\setcounter{footnote}{65} Alternatively, it has often been research on
  published opinion, diagnosing some biases in the representation of
  different camps and tracing them back to their causes---such as the
  opinions of journalists.}
and the role of trajectories in addition to positions.\footnote{Bourdieu,
  \emph{La distinction.}} First, he described social classes not only in
terms of their economic status or role in the relations of production,
but in terms of the distribution of different forms of capital (he found
economic and cultural capital to be most relevant in the context of his
analyses of habitus and lifestyle). The cultural dimension of inequality
also manifests itself in a homology between social positions and the
structure of schemata of perception and evaluation that then result in
differences in lifestyles and in different political and media choices.
Second, he drew attention to the role of social trajectories
(experienced or expected upward or downward mobility of individuals or
whole fractions of classes) in addition to current positions in shaping
people's schemata of perception and evaluation------in particular, their
views of society and their place in it, and consequently also of
politics. Bourdieu then explained the difference between liberal and
reactionary conservatism by the different expectations of traditional
and declining classes---whether or not they see a chance to preserve or
improve their position.\footnote{Bourdieu, \emph{La distinction},
  528--30.}

One of the few analyses of right-wing populism that explicitly and
strongly relies on Bourdieu's conception of social structure has been
published by German sociologist Cornelia Koppetsch.\footnote{Cornelia
  Koppetsch, \emph{Die Gesellschaft des Zorns: Rechtspopulismus im
  globalen Zeitalter} (Bielefeld: transcript, 2019). A number of
  criticisms have been raised against this book. For example, with
  respect to her rather undifferentiated and politically problematic
  diagnosis of a cosmopolitan (left-)liberal hegemony, see Floris
  Biskamp, ``Hegemonie? Welche Hegemonie? Teil IV einer Kritik an
  Cornelia Koppetschs Gesellschaft des Zorns,'' SozBlog: Blog der
  Deutschen Gesellschaft für Soziologie, last modified August 16, 2019.
  Furthermore, Koppetsch's university found that the book contained
  multiple instances of plagiarism and it was revealed in a criminal
  trial that the author's partner is a member of the German far-right
  populist party AfD. However, we are not concerned with her overall
  argument or potentially overly sympathetic attitudes toward right-wing
  populists, but only with the potential (which even Koppetsch may not
  fully use) of a Bourdieusian analysis of social structure for the
  analysis of right-wing populism.} In a somewhat transformed
Bourdieusian social space, she identifies three different groups whose
perceptions of decline, insecurity, and contestation of the established
social order lead them to support right-wing populism: the conservative
upper class, the traditional middle class, and the precarious lower
class. Without discussing this explanation and its merits in detail, it
seems promising to consider the common \emph{and} differential appeal of
right-wing populism to several classes and the alliances (and fault
lines) in a multidimensional social space, instead of a single class or
a small number of variables.\footnote{Or \emph{parts} of classes
  because, despite some exceptions, it would of course be mistaken to
  assume, for example, that non-white, non-heterosexual, etc. members of
  the working class are equally supportive of right-wing populism as
  their white etc. counterparts.} If such a perspective were combined
with the more or less established Bourdieusian analysis of practices of
media use, political communication researchers would be able to better
understand audiences of populist communication within the
social-structural and media-related opportunity structure of right-wing
populism.

As indicated above, German-speaking communication research often follows
a variables-based logic of analysis instead of a holistic one (thinking
in terms of the additive influence of individual properties instead of
complex situations, ways of living, or worldviews with interconnected
elements) and is more often interested in pre-existing attitudes instead
of social-structural positions or even trajectories as independent
variables or moderators. A Bourdieusian conception of social class,
based on the specific combination of different features of social
inequality and elements of lifestyles or evaluative schemata, is at odds
with this logic.\footnote{The frequent use of regression, ANOVA, or
  structural equation models instead of correspondence analysis, a tool
  preferred by Bourdieu for his analysis of social fields, also reflects
  these somewhat diverging logics.} In cases where a more holistic,
typological approach is chosen, the German-speaking tradition of milieu
analysis and consumer typologies is quite influential.\footnote{The
  perfect example of such a typology is the concept of
  ``Sinus-milieus,'' which has been developed by a private institute for
  social and market research but also aspires to social-scientific
  validity and recognition and has been adopted in academic research.}
Both in sociology and market research, the search for models of social
differentiation that reflect a presumed pluralization and
individualization of lifestyles has often led to the development of
typologies of milieus that while not outright ignoring Bourdieu's
original theories are increasingly disconnected from them. Communication
researchers have established a wide variety of typologies ranging from
classifications that are mostly data-driven and more or less exclusively
based on patterns of media use (sometimes, but not always, at the
intersection of academic and commercial audience research) to
theoretically informed conceptions of milieus that include social status
and political orientations.\footnote{For a typology focusing on media
  use in the context of other elements of lifestyles and value
  orientations and with an academic interest (but an emphasis on its
  applicability in applied audience research), see Peter H. Hartmann and
  Anna Schlomann, ``MNT 2015: Weiterentwicklung der
  MedienNutzerTypologie,'' \emph{Media Perspektiven} 2015, no. 11
  (2015). For a rather rare theoretically informed typology with a focus
  on politics and citing both German-speaking sociological theories of
  social milieus and Bourdieu's \emph{Distinction}, see Raphael Kösters
  and Olaf Jandura, ``A Stratified and Segmented Citizenry?
  Identification of Political Milieus and Conditions for Their
  Communicative Integration,'' \emph{Javnost---The Public} 26, no. 1
  (2019).}

There are thus three reasons for the hesitant adoption of Bourdieu's
theory of social class and trajectories in German-speaking communication
research, in particular, in political communication: the dominant
variable- and attitude-based logic; the frequent use of cross-sectional
or experimental designs and the neglect of biographies and trajectories;
and the fact that the rather small ``niche'' of typological and
lifestyle analysis is already occupied by other, often specifically
German conceptions.\footnote{In addition to the examples in the previous
  footnotes, German-speaking sociologists have theorized social status
  and lifestyles as more individualized but open to typologies. See,
  e.g., Ulrich Beck, ``Jenseits von Klasse und Stand? Soziale
  Ungleichheiten, gesellschaftliche Individualisierungsprozesse und die
  Entstehung neuer sozialer Formationen und Identitäten,'' in
  \emph{Soziale Ungleichheiten}, ed. Reinhard Kreckel (Göttingen:
  Schwartz, 1983); Stefan Hradil, \emph{Sozialstrukturanalyse in einer
  fortgeschrittenen Gesellschaft: Von Klassen und Schichten zu Lagen und
  Milieus} (Opladen: Leske and Budrich, 1987); Gerhard Schulze,
  \emph{Die Erlebnisgesellschaft: Kultursoziologie der Gegenwart}
  (Frankfurt: Campus, 1992).}

In a different world, German-speaking political communication
researchers would then analyze the biographies of voters and political
leaders. They would not only include ``control variables'' such as age,
education, and gender in their quantitative analysis, but more elaborate
measures of social status, lifestyles, and trajectories, both with
regard to their main effects and in interaction with mediated populist
or other political communication. Finally, researchers would also
conduct long-term studies of lifelong political socialization or the
transformation of political milieus. However, such approaches are rare
in a field of political communication research that mostly uses
experiments, cross-sectional surveys, and shorter panel studies. In
addition to the influence of its intellectual history, as one of the
major fields in German-speaking and international communication
research, political communication is also highly competitive and thus
incentivizes designs that allow for quick and efficient data collection.

\hypertarget{bourdieu-on-charisma-and-legitimacy}{%
\subsection{Bourdieu on Charisma and
Legitimacy}\label{bourdieu-on-charisma-and-legitimacy}}

One tendency in scholarship on populist communication is to attribute
the success of populism to the natural charisma of its leaders, their
talent for communication or even manipulation, and a style that is
naturally appealing to the masses. While this viewpoint is based on a
naïve preconception and cannot be attributed to specific scholars, it
nonetheless appears in many prominent publications on populism, where
authors explicitly argue or uncritically imply that populists must
somehow be particularly skilled communicators.

Prior to Bourdieu, Max Weber introduced the concept of charisma as an
extraordinary quality that is ascribed to a person who is therefore
recognized as a leader. Included in this definition was Weber's
observation that it is completely irrelevant how this property is
evaluated by a third-party observer: What matters is how it is judged by
the ``charismatically ruled.''\footnote{Max Weber, \emph{Wirtschaft und
  Gesellschaft} (Tübingen: Mohr, 1922), 140.} Bourdieu refers to Weber's
conception of charisma (and to Marx's analysis of fetishism) in his
analysis of political representation.\footnote{Bourdieu, ``La délégation
  et le fétichisme politique.''} Had they been familiar with Bourdieu's
take on charisma, German-speaking communication researchers could have
re-imported and updated Weber's concept in order to analyze how claims
of representation can be centered on the populist leader. In his
appropriation of the concept, Bourdieu emphasizes two aspects of
representation that ultimately lead to this ascription of unique
qualities that seem to exist on their own terms, to take on their own
life: the idea of forgotten or ignored work of representation and its
relationality.

The work of representation not only consists of communicative efforts
and institutional processes to establish a certain vision of society and
certain categorizations, but also to convey a sense of the legitimacy of
the representation. The power of the representative is then based on
certain pre-existing or more newly established beliefs that are produced
and reproduced in a given field, which in turn legitimize the
representative and their claims and support their recognition by the
(potentially) represented. The stronger the belief that the claim is not
arbitrary---and that words do not produce but only describe
reality---the stronger the relation between the represented and the
representative, and the stronger the belief in their claims.\footnote{Bourdieu,
  Pierre, ``Sur le pouvoir symbolique.''}

Without recognition, any claim to representation or power is nonsense:
``{[}Le roi,{]} c'est un fou qui se prend pour le roi avec l'approbation
des autres''---someone claiming to be the king is usually considered
``crazy,'' unless everyone agrees that he is the king (and it is still
strange or mysterious, as Bourdieu reminds us, how someone can transform
into a king just because people believe in him, his legitimacy or
qualities, and in the vision of social order that justifies his
power).\footnote{Bourdieu, \emph{Sur l\textquotesingle État}, 400.}

If the represented accept the claim of representation, it is because the
social categories it is based on have begun to appear self-evident. To
use them seems to describe social reality, not to manipulate it. And the
representative appears as the natural incarnation of the represented
group. All the work that was necessary to establish the claim to
representation can then be ignored or forgotten.

The political field does not have any clear rules or authorities that
would legitimate the ways of legitimating power. It thus constantly
fluctuates between legitimacy by science and by plebiscite, between
technocracy and democratic will, the force of conviction that something
is true and the force of recognition by a mobilizable social
group.\footnote{Bourdieu, ``Décrire et préscrire.''} Political claims
are not to be judged against a fixed reality, as their ``truth'' depends
on who utters them: If one is in the right position, a description or
prediction has the chance to become true (in the extreme case, it is a
self-fulfilling prophecy) and a promise can be kept by virtue of the
beliefs and size of the relevant groups and due to the right
institutional resources.\footnote{Bourdieu, ``Décrire et préscrire.''}
This is a leader's political capital, which can take several forms:
institutional (the symbolic, personal, and material resources
temporarily transferred by an organization to its officials in return
for loyalty to the organization or its positions) and personal (their
prestige and popularity).\footnote{Bourdieu, ``Décrire et préscrire.''}

While notables slowly accumulate their political capital over a whole
life, the capital of ``charismatic'' (or heroic, prophetic) leaders is
acquired in a situation of crisis, when the established institutions
cannot provide answers, and is legitimated retrospectively if the
rhetoric of crisis and the resulting mobilization are
successful.\footnote{Bourdieu, ``Décrire et préscrire.''} If populism is
always related to crises in one way or another, Bourdieu would most
probably agree with those who emphasize that a crisis is not simply
given as such but \emph{performed}.\footnote{Benjamin Moffitt, ``How to
  Perform Crisis: A model for Understanding the Key Role of Crisis in
  Contemporary Populism,'' \emph{Government \& Opposition} 50, no. 2
  (2015).}

This personal authority is, however, only possible through a group that
authorizes it---and this in turn authorizes the group in their identity
and unity: An actor speaks out what had been ignored or had only been
tacitly felt, the pre-linguistic and pre-reflective dispositions in a
population---sometimes theatrically in order not to let it pass over to
silence again---and this resonates with the group if this discourse
matches those dispositions.\footnote{Bourdieu, ``Décrire et préscrire.''}

German-speaking (and international) mainstream research on political
communication often sees populist leaders only as ``sources'' of
messages whose ``manifest content'' or effect is to be analyzed, not as
socially located, symbolic, admired (or hated) figures. Perhaps this
shows an understanding of ``modern'' social-scientific research (as
opposed to a past with supposedly merely ``individualist'' and not
actually ``analytical'' approaches) as necessarily abstract,
generalizable, and thus impersonal---or in other words, an attempt to
distance oneself from interpretive, ``subjective,'' not actually
``empirical'' research.

Certainly, research on political communication is often, but not always,
centered on the individual, in the sense that theories and studies focus
on individual attitudes and political action. The only exception are a
few researchers in political communication who strictly follow Luhmann's
``anti-humanist'' theory of social systems to analyze the relationship
between politics and the media. However, the interpretive analysis of
the symbolic character of individual populist leaders and of their
claims may be met with skepticism in German-speaking communication
research, because it may be seen as a case study that is not clearly
``empirical'' (but ``essayistic,'' ``impressionistic,'' merely
``descriptive,'' etc.) and of questionable generalizability.

We can imagine a set of German researchers, who instead of producing the
rather prosaic and formalist research on political communication we see
today would have somewhat rethought their style of research as a result
of intensive contact with Bourdieu. These readers of the French theorist
might have partly broken with their habitus in order to follow his
somewhat paradoxical approach to the analysis of the social magic of
representation: To understand this approach, it is necessary to look at
politics with a disenchanted regard in order to be again astonished that
it works and to grasp how it does so. Breaking the spell, we gain the
ability to see what is---despite its real social consequences---only
based on appearance and belief. Interpretative analyses should be
conducted that take into account how crises are performed, how leaders
try to authorize themselves, and how claims to power realize themselves
if, together with the right styles and pre-linguistic dispositions, they
produce charisma---or, on the other hand, how this all fails.

\hypertarget{bourdieu-on-reflexivity}{%
\subsection{Bourdieu on
Reflexivity}\label{bourdieu-on-reflexivity}}

Continuing with our counterfactual case, had German-speaking
communication researchers been familiar with Bourdieu's general emphasis
on reflexivity in the (social) sciences,\footnote{Pierre Bourdieu,
  \emph{Science de la science et reflexivité} (Paris: Raisons
  d\textquotesingle agir, 2001).} they might have reflected on the
forces that shape their field and their perspective on social reality,
and how their research itself is a social force that shapes social
reality. Over the course of the last several decades, these readers of
Bourdieu would have become aware of the reasons for the unquestioned
legitimacy of their field within the overall field of communication
researchers, of their unquestioned understanding of politics (the
\emph{doxa} of their field, as Bourdieu would put it), and of the
limitations of the prevalent realist epistemology.

Historically, at least since the social-scientific turn, research on
political communication can be considered the most legitimate field in
communication research, a field that does not have to legitimize itself
and that is almost equated with communication research as a
whole.\footnote{Other fields with similar legitimacy are media effects
  and journalism research. However, it may be said provocatively that
  they are in fact sub-fields of political communication research as
  journalism is almost reduced to political journalism and media effects
  have almost always been effects on political or politically relevant
  attitudes (although the rise of health and environmental communication
  and similar fields has somewhat changed this almost exclusive focus).}
Researchers engaging in self-reflection on the doxastic foundations of
their field may speculate that the price for this legitimacy is that
political communication research does not become overtly political
itself, keeping its demonstrative equidistance toward ``the extremes,''
and that it limits itself to a narrow understanding of the political as
institutionalized politics.\footnote{On the narrow definition of the
  political and the predominant functionalist and variable-based
  paradigm in (US and European) political communication research, see
  most recently, Sean Phelan and Pieter Maeseele, ``Where Is `The
  Political' in the Journal \emph{Political Communication}? On the
  Hegemonic Articulation of a Disciplinary Identity,'' \emph{Annals of
  the International Communication Association} (2023). Notably, these
  authors reference Laclau and Mouffe, but only cite Bourdieu once in
  passing, with some of his terminology tingeing the text.}

Even if German-speaking political communication research is usually not
critical or political in the sense of challenging the very idea of the
political, it is normative and in this sense ``critical'' by measuring
empirical phenomena against the norms of liberal democracy and
journalistic professionalism.\footnote{See the contributions in Matthias
  Karmasin, Matthias Rath, and Barbara Thomaß, \emph{Normativität in der
  Kommunikationswissenschaft} (Wiesbaden: Springer VS, 2013), on the
  prevalent forms of normativity in (German-speaking) communication
  research. In the same volume (p. 329--51), Christiane Eilders
  identifies the main pattern of normativity in political communication
  research as the expectation that the media inform the public and allow
  for the formation of opinions in the free exchange of positions, thus
  legitimizing democracy. She also discusses issues of integration and
  fragmentation in relation to online communication. Christiane Eilders,
  ``Öffentliche Meinungsbildung in Online-Umgebungen: Zur Zentralität
  der normativen Perspektive in der politischen
  Kommunikationsforschung,'' in \emph{Normativität in der
  Kommunikationswissenschaft} (Wiesbaden: Springer VS, 2013). Other
  contributions in the volume also deal with normativity in journalism
  and media effects research and public sphere theory. Among the
  critical schools, only feminist scholarship is represented in the
  volume with a separate article: Tanja Thomas, ``Feministische
  Kommunikations- und Medienwissenschaft,'' 397--420. While Thomas's
  article references Bourdieu, other chapters briefly address the
  Frankfurt School, Cultural Studies, and Marxism.} This implies a
functionalist understanding of the media as objective, i.e., informative
and balanced (with fears of bias and tabloidization, or more recently,
disinformation), and integrative (with fears of fragmentation or, more
recently, polarization).

Populism then becomes relevant as a threat to liberal democracy and to
quality journalism, undermining the trust in their existing institutions
and contributing to polarization by strengthening the extremes. Of
course, this understanding runs counter to a positive or critical
understanding of class antagonism or an antagonism based on unfulfilled
democratic potential and the closure of political space by elites, as
emphasized by some theorists of left-wing populism.

Based on Bourdieu's ``theory of political communication'' outlined
above, researchers could also reflect on the realist understanding of
the social world---and public opinion in particular---brought about by
the social-scientific turn and probably associated most strongly with
the Mainz School. In this paradigm, social phenomena are usually treated
as given entities with measurable properties, such as a social group
existing in society whose members have a range of sociodemographic
characteristics and opinions with their causes and consequences.

Furthermore, while typical studies in political communication may
acknowledge that many concepts are contested in political discourse and
that symbols or statements can be ambiguous, typical approaches and
methods often require researchers to treat meaning as more or less fixed
or the range of meanings as closed---for example, in standardized
questionnaires or content analyses (which is, of course, perfectly
legitimate in many contexts). A performative, almost circular
understanding of communication, such as in Bourdieu's conception of
symbolic power, assumes that communication can contribute to the
constitution of the phenomena it claims to describe; its ``truth value''
lies in the power to make itself true if the right social conditions are
met.\footnote{Bourdieu would of course be the last social theorist to
  deny that there is such a thing as true and false statements about the
  social world or the world in general. He would only insist that it is
  simplistic to assume that it is the sole purpose and/or effect of
  political, social-scientific, journalistic, and other communication to
  simply reflect reality or that all communication can be judged against
  this standard.} In contrast, a more straightforwardly realist paradigm
sees communication as conveying information and opinions. If it is
assumed that communication can always be judged against an external
standard of realism, then the main schemata of analysis are whether
communication contains information or disinformation, represents
opinions in an unbiased or unbiased way, or is objectively more or less
persuasive.

This realism concerning social entities and meaning leads to the risk of
contributing to what Bourdieu has called a ``theory effect''\footnote{Bourdieu,
  ``Décrire et préscrire.''} that (co-)creates and essentializes the
very social phenomena that a theory, ideology, or discourse describes,
if researchers do not break with unreflected scholarly or everyday
preconceptions of the social world.

Researchers informed by Bourdieu's reflexive social science would
therefore be careful about how they might reproduce the very categories
constructed in claims of representation or exclusion, and how their
research inevitably contributes to the ineluctably politicized
constitution of social reality: who is entitled to speak publicly, what
categories speakers take for granted or successfully establish, and
consequently, what the relevant objects and sides of ``public opinion''
are.

For example, in populism research, one may reproduce the idea of ``the
people'' as a seemingly natural, pre-existing basis of sovereignty and
exclusion.\footnote{On the risk of reifying ``the people,'' with a short
  reference to Bourdieu, and the lack of analyses of how ``the people''
  is constructed, see Benjamin De Cleen, ``The Populist Political Logic
  and the Analysis of the Discursive Construction of `The People' and
  `The Elite,'\,'' in \emph{Imagining the Peoples of Europe: Populist
  Discourses across the Political Spectrum}, ed. Jan Zienkowski and Ruth
  Breeze (Amsterdam: John Benjamins, 2019).} Or researchers might risk
adopting the categorization of outgroups that right-wing populists
perform, such as ``migrants'' as an almost eternal category of people
that will always remain alien and incompatible to the native population,
no matter their personal histories or legal status.

A performative perspective or one that is based on symbolic struggles
starts one step earlier, with the strategies and contexts in which those
meanings are elaborated. It does not only ask, for example, how central
``the people'' is in political discourse or in individual political
beliefs, but how it is constituted. So if German-speaking political
communication research had started to adopt Bourdieusian concepts and
reflexivity a while ago, it would be able to take different perspectives
on populism today.

\hypertarget{conclusion}{%
\section{Conclusion}\label{conclusion}}

German-speaking research on political communication is a rather
ahistorical field. There is not much interest in the field's history,
maybe because the presently prevailing form of empirical research (with
canonical methods, now facilitated by handy online tools, massively
increasing volumes of data and computing power, etc.) is seen as the end
of history or as the right path to progress by ever-accumulating
knowledge (the right path that was taken when speculative, subjective,
or ideological schools had been overcome). Theories like Bourdieu's may
be adopted in such a field as a repertoire for timeless concepts to be
cast into fixed scales, but would also offer the chance to study
historical changes in social structure and in the political or
journalistic field---for example, in order to identify the preconditions
and opportunity structures for the success of populism, the chance for a
theoretically informed social-historical and historical-reflective turn.

However, as part of the social-scientific turn, German-speaking
political communication research took a different intellectual and
institutional form and position and epitomizes that very turn, with
research on populist communication recently epitomizing political
communication research. The field is centered on communicators and
messages instead of social structure and ways of living, on constative
and persuasive instead of performative communication, and on directional
effects instead of the circular constitution of both sides of political
communication, the communicators (or representatives) and the audience
(or constituency). German-speaking political communication research is
usually variables-based and most often cross-sectional instead of based
on a holistic analysis of milieus and the trajectory of social groups,
and it mostly relies on a realist epistemology of the social. It sees
its relevance in the explanation of phenomena around institutionalized
politics, most importantly dynamics of opinions, not in the uncovering
or critique of ``social magic'' or dispossession. Bourdieu's reception
in German-speaking research on political and populist communication has
therefore not simply been hindered by the field's increasing orientation
toward Anglophone literature and publication outlets and the neglect of
Francophone contributions (his writings on politics have been translated
to German and English over time, so accessibility is not the main
barrier), but probably more importantly by the foundational, almost
doxastic ideas about what makes communication research
``(social-)scientific'' and relevant.

Maybe we could have reached the same conclusions without the
counterfactual encounter between Bourdieu and populism research by
German-speaking political communication researchers. However, it seems
that populism research in its dominant form and Bourdieu's concepts are
something of a perfect mismatch, given populism's relationship with
representation, charisma, social class, and the question of what makes
it relevant. Such encounters, in particular when compared to seemingly
similar ones (e.g., Bourdieu and media reception or journalism
research), have a unique potential for historical self-reflection and
for a renewal of fields of research with strongly institutionalized
perspectives and practices.




\newpage\section{Bibliography}\label{bibliography}

\begin{hangparas}{.25in}{1} 



Aalberg, Toril, Frank Esser, Carsten Reinemann, Jesper Strömbäck, and
Claes De Vreese, eds. \emph{Populist Political Communication in Europe}.
New York: Routledge, 2017.

Abou-Chadi, Tarik, Reto Mitteregger, and Cas Mudde. \emph{Left Behind by
the Working Class? Social Democracy's Electoral Crisis and the Rise of
the Radical Right}. Berlin: Friedrich-Ebert-Stiftung, 2021.

Averbeck, Stefanie. ``Comparative History of Communication Studies:
France and Germany.'' \emph{The} \emph{Open Communication Journal} 2,
no. 2 (2008): 1--13.
\url{http://dx.doi.org/10.2174/1874916X00802010001}.

Averbeck, Stefanie. \emph{Kommunikation als Prozess: Soziologische
Perspektiven in der Zeitungswissenschaft, 1927--1934}. Münster: Lit,
1999.

Beck, Klaus, Till Büser, and Christiane Schubert. ``Medialer Habitus,
mediales Kapital, mediales Feld--oder: vom Nutzen Bourdieus für die
Mediennutzungsforschung.'' In \emph{Pierre Bourdieu und die
Kommunikationswissenschaft: Internationale Perspektiven}, edited by
Thomas Wiedemann and Michael Meyen, 243--62. Köln: Halem, 2013.

Beck, Ulrich. ``Jenseits von Klasse und Stand? Soziale Ungleichheiten,
gesellschaftliche Individualisierungsprozesse und die Entstehung neuer
sozialer Formationen und Identitäten.'' In \emph{Soziale
Ungleichheiten}, edited by Reinhard Kreckel, 35--74. Göttingen:
Schwartz, 1983.

Biskamp, Floris. ``Hegemonie? Welche Hegemonie? Teil IV einer Kritik an
Cornelia Koppetschs Gesellschaft des Zorns.'' SozBlog: Blog der
Deutschen Gesellschaft für Soziologie. Last modified August 16, 2019.
\url{http://blog.soziologie.de/2019/08/hegemonie-welche-hegemonie-teil-iv-einer-kritik-an-cornelia-}\\\hspace{0.25in}\url{koppetschs-gesellschaft-des-zorns/}.

Bougnoux, Daniel. \emph{Introduction aux sciences de la communication.}
Paris: La Découverte, 2001.

Bourdieu, Pierre. ``Ce que parler veut dire.'' In \emph{Questions de
sociologie}, edited by Pierre Bourdieu, 95--112. Paris: Minuit, 1980.

Bourdieu, Pierre. ``Décrire et préscrire.'' \emph{Actes de la recherche
en sciences sociales} 38 (1981): 69--73.
\url{https://doi.org/10.3406/arss.1981.2120}.

Bourdieu, Pierre. \emph{Esquisse d'une théorie de la pratique, précédé
de Trois études d'ethnologie kabyle}. Geneva: Droz, 1972.

Bourdieu, Pierre. \emph{Homo academicus}. Paris: Minuit, 1981.

Bourdieu, Pierre. ``L'emprise du journalisme.'' \emph{Actes de la
recherche en sciences sociales} 101--102 (1994): 3--9.
\url{https://doi.org/10.3917/arss.p1994.101n1.0003}.

Bourdieu, Pierre. ``L'opinion publique n\textquotesingle existe pas.''
\emph{Les temps modernes} 29 (1973): 1292--309.

Bourdieu, Pierre. ``La délégation et le fétichisme politique.''
\emph{Actes de la recherche en sciences sociales} 52--53 (1984): 49--55.
\url{https://doi.org/10.3406/arss.1984.3331}.

Bourdieu, Pierre. \emph{La distinction: Critique sociale du jugement}.
Paris: Minuit, 1979.

Bourdieu, Pierre. ``La représentation politique.'' \emph{Actes de la
recherche en sciences sociales} 36--37 (1981): 3--24.
\url{https://doi.org/10.3406/arss.1981.2105}.

Bourdieu, Pierre. ``Le mystère du ministère: Des volontés particulières
à la `volonté générale.'\,'' \emph{Actes de la recherche en sciences
sociales} 140 (2001): 7--11.
\url{https://doi.org/10.3406/arss.2001.2831}.

Bourdieu, Pierre. \emph{Le sens pratique}. Paris: Minuit, 1980.

Bourdieu, Pierre. \emph{Les règles de l\textquotesingle art: Genèse et
structure du champ littéraire}. Paris: Seuil, 1992.

Bourdieu, Pierre. ``Les usages du peuple.'' In \emph{Choses dites},
edited by Pierre Bourdieu, 178--84. Paris: Minuit, 1987.

Bourdieu, Pierre. \emph{Propos sur le champ politique}. Lyon: Presses
universitaires de Lyon, 2000.

Bourdieu, Pierre. ``Questions de politique.'' \emph{Actes de la
recherche en sciences sociales} 16 (1977): 55--89.
\url{https://doi.org/10.3406/arss.1977.2568}.

Bourdieu, Pierre. \emph{Raisons pratiques: Sur la théorie de l'action}.
Paris: Seuil, 1994.

Bourdieu, Pierre. \emph{Science de la science et reflexivité}. Paris:
Raisons d'agir, 2001.

Bourdieu, Pierre. ``Störenfried Soziologie.'' In \emph{Wozu heute noch
Soziologie?}, edited by Joachim Fritz-Vannahme, 65--70. Wiesbaden: VS,
1996.

Bourdieu, Pierre. \emph{Sur l'État: Cours au Collège de France
(1989--1992).} Paris: Seuil, 2012.

Bourdieu, Pierre. ``Sur le pouvoir symbolique.'' \emph{Annales:
Économies, sociétés, civilisations} 32 (1977): 405--11.
\url{https://doi.org/10.3406/ahess.1977.293828}.

Bourdieu, Pierre. \emph{Sur le télévision}. Paris: Liber, 1996.

Bourdieu, Pierre. ``The Political Field, the Social Science Field, and
the Journalistic Field.'' In \emph{Bourdieu and the Journalistic Field},
edited by Rodney Benson and Erik Neveu, 29--47. Cambridge, MA: Polity
Press, 2005.

Bourdieu, Pierre. ``Un jeu chinois: Notes pour une critique sociale du
jugement.'' \emph{Actes de la recherche en sciences sociales} 2--4
(1976): 91--101. \url{https://doi.org/10.3406/arss.1976.3467}.

Bourdieu, Pierre. ``Vous avez dit `populaire'?'' \emph{Actes de la
recherche en sciences sociales} 46 (1983): 98--105.
\url{https://doi.org/10.3406/arss.1983.2179}.

Bourdieu, Pierre, and Jean-Claude Passeron. \emph{La reproduction:
Éléments pour une théorie du système d'enseignement}. Paris: Minuit,
1970.

Bourdieu, Pierre, and Jean-Claude Passeron. ``Sociologues des
mythologies et mythologies de sociologues.'' \emph{Les Temps Modernes}
211 (1963): 998--1021.

De Cleen, Benjamin. ``The Populist Political Logic and the Analysis of
the Discursive Construction of `The People' and `The Elite.'\,'' In
\emph{Imagining the Peoples of Europe: Populist Discourses across the
Political Spectrum}, edited by Jan Zienkowski and Ruth Breeze, 19--42.
Amsterdam: John Benjamins, 2019.

De Wilde, Pieter, Ruud Koopmans, Wolfgang Merkel, Oliver Strijbis, and
Michael Zürn, eds. \emph{The Struggle over Borders: Cosmopolitanism and
Communitarianism}. Cambridge: Cambridge University Press, 2019.

Eilders, Christiane. ``Öffentliche Meinungsbildung in Online-Umgebungen:
Zur Zentralität der normativen Perspektive in der politischen
Kommunikationsforschung.`` In \emph{Normativität in der
Kommunikationswissenschaft}, edited by Matthias Karmasin, Matthias Rath,
and Barbara Thomaß, 329--51. Wiesbaden: Springer VS, 2013.

Engesser, Sven, Nicole Ernst, Frank Esser, and Florin Büchel. ``Populism
and Social Media: How Politicians Spread a Fragmented Ideology\emph{.''
Information, Communication \& Society} 20, no. 8 (2017): 1109--26.
\url{https://doi.org/10.1080/1369118X.2016.1207697}.

Esquenazi, Jean-Pierre. ``Les médias et leurs publics.`` In
\emph{Sciences de l'information et de la communication}, edited by
Stéphane Olivesi, 9--24. Paris: PUG, 2013.

Fawzi, Nayla. ``Untrustworthy News and the Media as `Enemy of the
People?' How a Populist Worldview Shapes Recipients' Attitudes toward
the Media.'' \emph{The International Journal of Press/Politics} 24, no.
2 (2019): 146--64. \url{https://doi.org/10.1177/1940161218811981}.

Fawzi, Nayla, and Benjamin Krämer. ``The Media as Part of a Detached
Elite? Exploring Antimedia Populism Among Citizens and Its Relation to
Political Populism.'' \emph{International Journal of Communication} 15
(2021): 3292--314.
\url{https://ijoc.org/index.php/ijoc/article/view/14795}.

Gemperle, Michael. ``The Double Character of the German Bourdieu: On the
Twofold Use of Pierre Bourdieu's Work in the German-Speaking Social
Sciences.'' \emph{Sociologica} 2009, no. 1 (2009): 1--33.
\url{https://doi.org/10.2383/29573}.

Hameleers, Michael, Carsten Reinemann, Desiree Schmuck, and Nayla Fawzi.
``The Persuasiveness of Populist Communication: Conceptualizing the
Effects and Political Consequences of Populist Communication From a
Social Identity Perspective.'' In \emph{Communicating Populism:
Comparing Actor Perceptions, Media Coverage, and Effects on Citizens in
Europe}, edited by Carsten Reinemann, James Stanyer, Toril Aalberg,
Frank Esser, and Claes De Vreese, 143--67. New York: Routledge, 2019.

Hanitzsch, Thomas. ``Populist Disseminators, Detached Watchdogs,
Critical Change Agents and Opportunist Facilitators: Professional
Milieus, the Journalistic Field and Autonomy in 18 Countries.''
\emph{International Communication Gazette} 73, no. 6 (2011): 477--94.
\url{https://doi.org/10.1177/1748048511412279}.

Hartmann, Peter H., and Anna Schlomann. ``MNT 2015: Weiterentwicklung
der MedienNutzerTypologie.'' \emph{Media Perspektiven} 2015, no. 11
(2015): 497--504.

Hatakka, Niko. \emph{Populism in the Hybrid Media System: Populist
Radical Right Online Counterpublics Interacting with Journalism, Party
Politics, and Citizen Activism}. Turku: University of Turku, 2019.

Hawkins, Kirk, and Christóbal Rovira Kaltwasser. ``What the (Ideational)
Study of Populism Can Teach Us, and What It Can't.'' \emph{Swiss
Political Science Review} 23, no. 4 (2017): 526--42.
\url{https://doi.org/10.1111/spsr.12281}.

Hepp, Andreas. \emph{Cultural Studies und Medienanalyse: Eine
Einführung.} Wiesbaden: VS, 1999.

Hradil, Stefan. \emph{Sozialstrukturanalyse in einer fortgeschrittenen
Gesellschaft: Von Klassen und Schichten zu Lagen und Milieus}. Opladen:
Leske and Budrich, 1987.

Huber, Nathalie. \emph{Kommunikationswissenschaft als Beruf: Zum
Selbstverständnis von Professoren des Faches im deutschsprachigen Raum}.
Köln: Halem, 2010.

Ignazi, Piero. ``The Silent Counter-Revolution: Hypotheses on the
Emergence of Extreme Right-Wing Parties in Europe.'' \emph{European
Journal of Political Research} 22, no. 1 (1992): 3--34.
\url{https://doi.org/10.1111/j.1475-6765.1992.tb00303.x}.

Inglehart, Ronald, and Pippa Norris. ``Trump and the Populist
Authoritarian Parties: The Silent Revolution in Reverse.''
\emph{Perspectives on Politics} 15, no. 2 (2017): 443--54.
\url{https://doi.org/10.1017/S1537592717000111}.

Jagers, Jan, and Stefaan Walgrave. ``Populism as Political Communication
Style: An Empirical Study of Political Parties' Discourse in Belgium.''
\emph{European Journal of Political Research} 46, no. 3 (2007): 319--45.
\url{https://doi.org/10.1111/j.1475-6765.2006.00690.x}.

Karmasin, Matthias, Matthias Rath, and Barbara Thomaß.
\emph{Normativität in der Kommunikationswissenschaft}. Wiesbaden:
Springer VS, 2013.

Koppetsch, Cornelia. \emph{Die Gesellschaft des Zorns: Rechtspopulismus
im globalen Zeitalter.} Bielefeld: transcript, 2019.

Kösters, Raphael, and Olaf Jandura. ``A Stratified and Segmented
Citizenry? Identification of Political Milieus and Conditions for Their
Communicative Integration.'' \emph{Javnost---The Public} 26, no. 1
(2019): 33--53. \url{https://doi.org/10.1080/13183222.2018.1554845}.

Krämer, Benjamin. ``Eine Bourdieu'sche Kritik der politischen
Urteilskraft.'' In \emph{Pierre Bourdieu und die
Kommunikationswissenschaft: Internationale Perspektiven}, edited by
Thomas Wiedemann and Michael Meyen, 263--86. Köln: Halem, 2013.

Krämer, Benjamin. ``Media Populism: A Conceptual Clarification and Some
Theses on Its Effects.'' \emph{Communication Theory} 24, no. 1 (2014):
42--60. \url{https://doi.org/10.1111/comt.12029}.

Krämer, Benjamin. \emph{Mediensozialisation: Theorie und Empirie zum
Erwerb medienbezogener Dispositionen im Lebensverlauf}. Wiesbaden:
Springer VS, 2013.

Krämer, Benjamin. ``Strategies of Media Use.'' \emph{Studies in
Communication and Media} 2, no. 1 (2013): 199--222.
\url{http://dx.doi.org/10.5771/2192-4007-2013-2-199}.

Krämer, Benjamin, and Christina Holtz-Bacha, eds. \emph{Perspectives on
Populism and the Media: Avenues for Research}. Baden-Baden: Nomos, 2020.

Laclau, Ernesto. \emph{On Populist Reason}. London: Verso, 2005.

Löblich, Maria. \emph{Die empirisch-sozialwissenschaftliche Wende in der
Publizistik- und Zeitungswissenschaft}. Köln: Halem, 2010.

Löblich, Maria, and Andreas Scheu. ``Writing the History of
Communication Studies: A Sociology of Science Approach.''
\emph{Communication Theory} 21, no. 1 (2011): 1--22.
\url{https://doi.org/10.1111/j.1468-2885.2010.01373.x}.

Maares, Phoebe, and Folker Hanusch. ``Interpretations of the
Journalistic Field: A Systematic Analysis of How Journalism Scholarship
Appropriates Bourdieusian Thought.'' \emph{Journalism} 23, no. 4 (2022):
736--54. \url{https://doi.org/10.1177/1464884920959552}.

Maigret, Éric. \emph{Sociologie de la communication et des médias}.
Paris: Armand Colin, 2015.

Matthes, Jörg, and Desirée Schmuck. ``The Effects of Anti-Immigrant
Right-Wing Populist Ads on Implicit and Explicit Attitudes: A Moderated
Mediation Model.'' \emph{Communication Research} 44, no. 4 (2017):
556--81. \url{https://doi.org/10.1177/0093650215577859}.

Maurer, Peter, and Trevor Diehl. ``What Kind of Populism? Tone and
Targets in the Twitter Discourse of French and American Presidential
Candidates.'' \emph{European Journal of Communication} 35\emph{,} no. 5
(2022): 453--68. \url{https://doi.org/10.1177/0267323120909288}.

Maurer, Peter, and Andreas Riedl. ``Why Bite the Hand that Feeds You?
Politicians' and Journalists' Perceptions of Common Conflicts.''
\emph{Journalism} 22, no. 11 (2020): 2855--72.
\url{https://doi.org/10.1177/1464884919899304}.

Maurer, Peter, and Rajesh Sharma. ``L'usage de narratifs populistes dans
les~tweets des candidats `contestataires' aux élections présidentielles
en France (2017) et aux États-Unis (2016).'' \emph{Revue internationale
de politique comparée} 29, no. 2--3 (2022): 83--106.
\url{https://doi.org/10.3917/ripc.292.0083}.

Mazzoleni, Gianpietro. ``Mediatization and Political Populism.'' In
\emph{Mediatization of Politics: Understanding the Transformation of
Western Democracies}, edited by Frank Esser and Jesper Strömbäck,
42--56. London: Palgrave Macmillan, 2014.

Mazzoleni, Gianpietro. ``Populism and the Media.'' In \emph{Twenty-First
Century Populism: The Spectre of Western European Democracy}, edited by
Daniele Albertazzi and Duncan McDonnell, 49--64. London: Palgrave
Macmillan, 2008.

Mazzoleni, Gianpietro. ``The Media and the Growth of Neo-Populism in
Contemporary Democracies.'' In \emph{The Media and Neo-Populism: A
Contemporary Comparative Analysis}, edited by Gianpetro Mazzoleni,
Julianne Stewart, and Bruce Horsfield, 1--20. Westport, CT: Praeger,
2003.

Meyen, Michael. ``Der Machtpol des kommunikationswissenschaftlichen
Feldes.'' \emph{Studies in Communication and Media} 1, no. 3--4 (2012):
299--321. \url{https://doi.org/10.5771/2192-4007-2012-3-299}.

Meyen, Michael. ``Medienwissen und Medienmenüs als kulturelles Kapital
und als Distinktionsmerkmale: Eine Typologie der Mediennutzer in
Deutschland.'' \emph{Medien und Kommunikationswissenschaft} 55, no. 3
(2007): 333--54. \url{https://doi.org/10.5771/1615-634x-2007-3-333}.

Meyen, Michael, and Senta Pfaff-Rüdiger. \emph{Internet im Alltag:
Qualitative Studien zum praktischen Sinn von Onlineangeboten}. Münster:
Lit, 2009.

Moffitt, Benjamin. ``How to Perform Crisis: A Model for Understanding
the Key Role of Crisis in Contemporary Populism.'' \emph{Government \&
Opposition} 50, no. 2 (2015): 189--217.
\url{https://doi.org/10.1017/gov.2014.13}.

Moffitt, Benjamin. \emph{The Global Rise of Populism: Performance,
Political Style, and Representation}. Stanford: Stanford University
Press, 2016.

Mouffe, Chantal. \emph{For a Left Populism}. London: Verso, 2018.

Mudde, Cas. ``The Populist Zeitgeist.'' \emph{Government and Opposition}
39, no. 4 (2004): 541--63.
\url{https://doi.org/10.1111/j.1477-7053.2004.00135.x}.

Neuberger, Christoph. ``Journalismus und Medialisierung der
Gesellschaft.'' In \emph{Journalismusforschung: Stand und Perspektiven},
edited by Klaus Meier and Christoph Neuberger, 337--72. Baden-Baden:
Nomos, 2023.

Ollivier, Bruno. \emph{Les sciences de la communication: Théories et
acquis}. Paris: Armand Colin, 2007.

Pentzold, Christian. ``Praxistheoretische Prinzipien, Traditionen und
Perspektiven kulturalistischer Kommunikations- und Medienforschung.''
\emph{Medien und Kommunikationswissenschaft} 63 (2015): 229--45.
\url{https://doi.org/10.5771/1615-634x-2015-2-229}.

Pentzold, Christian. \emph{Zusammenarbeiten im Netz: Praktiken und
Institutionen internetbasierter Kooperation}. Wiesbaden: Springer VS,
2016.

Phelan, Sean, and Pieter Maeseele. ``Where Is `The Political' in the
\emph{Journal Political Communication}? On the Hegemonic Articulation of
a Disciplinary Identity,'' \emph{Annals of the International
Communication Association} 47, no. 2 (2023): 202--21.
\url{https://dx.doi.org/10.1080/23808985.2023.2169951}.

Raabe, Johannes. \emph{Die Beobachtung journalistischer Akteure:
Optionen einer empirisch-kritischen Journalismusforschung}. Wiesbaden:
VS, 2005.

Raabe, Johannes. ``Kommunikation und soziale Praxis: Chancen einer
praxistheoretischen Perspektive für Kommunikationstheorie und
-forschung.'' In \emph{Theorien der Kommunikations- und
Medienwissenschaft: Grundlegende Diskussionen, Forschungsfelder und
Theorieentwicklungen}, edited by Carsten Winter, Andreas Hepp, and
Friedrich Krotz, 363--81. Wiesbaden: VS, 2008.

Reinemann, Carsten, James Stanyer, Toril Aalberg, Frank Esser, and Claes
de Vreese, eds. \emph{Communicating Populism: Comparing Actor
Perceptions, Media Coverage, and Effects on Citizens in Europe}. New
York: Routledge, 2019.

Rensmann, Lars. ``The Noisy Counter-Revolution: Understanding the
Cultural Conditions and Dynamics of Populist Politics in Europe in the
Digital Age.'' \emph{Politics and Governance} 5, no. 4 (2017): 123--35.
\url{https://doi.org/10.17645/pag.v5i4.1123}.

Rosanvallon, Pierre. \emph{Le siècle du populisme: Histoire, théorie,
critique}. Paris: Seuil, 2020.

Scherer, Helmut. ``Mediennutzung und soziale Distinktion.'' In
\emph{Pierre Bourdieu und die Kommunikationswissenschaft: Internationale
Perspektiven}, edited by Thomas Wiedemann and Michael Meyen, 100--22.
Köln, Halem, 2013.

Scheu, Andreas. \emph{Adornos Erben in der Kommunikationswissenschaft:
Eine Verdrängungsgeschichte?} Köln: Halem, 2012.

Schulze, Gerhard. \emph{Die Erlebnisgesellschaft: Kultursoziologie der
Gegenwart.} Frankfurt: Campus, 1992.

Sorensen, Lone. \emph{Populist Communication: Ideology, Performance,
Mediation}. Cham, Switzerland: Palgrave Macmillan, 2021.

Stanley, Ben. ``The Thin Ideology of Populism.'' \emph{Journal of
Political Ideologies} 13, no. 1 (2008): 95--110.
\url{https://doi.org/10.1080/13569310701822289}.

Stavrakakis, Yannis. ``Antinomies of Formalism: Laclau's Theory of
Populism and the Lessons from Religious Populism in Greece.''
\emph{Journal of Political Ideologies} 9, no. 3 (2004): 253--67.
\url{https://doi.org/10.1080/1356931042000263519}.

Stewart, Julianne, Gianpietro Mazzoleni, and Bruce Horsfield.
``Conclusion: Power to the Media Managers.'' In \emph{The Media and
Neo-Populism: A Contemporary Comparative Analysis}, edited by Gianpetro
Mazzoleni, Julianne Stewart, and Bruce Horsfield, 217--36. Westport, CT:
Praeger, 2003.

Thomas, Tanja. ``Feministische Kommunikationsund Medienwissenschaft.``
In \emph{Normativität in der Kommunikationswissenschaft}, edited by
Matthias Karmasin, Matthias Rath, and Barbara Thomaß, 397--420.
Wiesbaden: Springer VS, 2013.

Van Leuwen, Martin. ``Measuring People-Centrism in Populist Political
Discourse: A Linguistic Approach.'' In \emph{Imagining the Peoples of
Europe: Populist Discourses Across the Political Spectrum}, edited by
Jan Zienkowski and Ruth Breeze, 315--40. Amsterdam: John Benjamins,
2019.

Weber, Max. \emph{Wirtschaft und Gesellschaft}. Tübingen: Mohr, 1922.

Weiß, Ralph. ``\,`Praktischer Sinn,' soziale Identität und Fern-Sehen:
Ein Konzept für die Analyse der Einbettung kulturellen Handelns in die
Alltagswelt.'' \emph{Medien und Kommunikationswissenschaft} 48, no. 1
(2000): 42--62. \url{https://doi.org/10.5771/1615-634x-2000-1-42}.

Wendelin, Manuel. ``Systemtheorie als Innovation in der
Kommunikationswissenschaft: Inhaltliche Hemmnisse und institutionelle
Erfolgsfaktoren im Diffusionsprozess.'' \emph{Communicatio Socialis} 41,
no. 4 (2008): 341--59.
\url{https://ejournal.communicatio-socialis.de/index.php/cc/article/view/355}.

Wessler, Hartmut. \emph{Habermas and the Media}. Cambridge, MA: Polity,
2018.

Wettstein, Martin, Frank Esser, Florin Büchel, Christian Schemer,
Dominique S. Wirz, Anne Schulz, Nicole Ernst, Sven Engesser, Philipp
Müller, and Werner Wirth. ``What Drives Populist Styles? Analyzing
Immigration and Labor Market News in 11 Countries.'' \emph{Journalism \&
Mass Communication Quarterly} 96, no. 2 (2019): 516--36.
\url{https://doi.org/10.1177/1077699018805408}.

Wiedemann, Thomas. ``Pierre Bourdieu: Ein internationaler Klassiker der
Sozialwissenschaft mit Nutzen für die Kommunikationswissenschaft.''
\emph{Medien \& Kommunikationswissenschaft} 62, no. 1 (2014): 83--101.
\url{https://doi.org/10.5771/1615-634x-2014-1-83}.

Wiedemann, Thomas. \emph{Walter Hagemann: Aufstieg und Fall eines
politisch ambitionierten Journalisten und Publizistikwissenschaftlers}.
Köln: Halem, 2015.

Wirz, Dominique S., Martin Wettstein, Anne Schulz, Philipp Müller,
Christian Schemer, Nicole Ernst, Frank Esser, and Werner Wirth. ``The
Effects of Right-Wing Populist Communication on Emotions and Cognitions
toward Immigrants.'' \emph{The International Journal of Press/Politics}
23, no. 4 (2018): 496--16.
\url{https://doi.org/10.1177/1940161218788956}.



\end{hangparas}


\end{document}